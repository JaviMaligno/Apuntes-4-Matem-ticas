\documentclass[twoside]{report}
\usepackage[utf8x]{inputenc}
\usepackage[spanish]{babel}
\usepackage{amssymb}
\usepackage{amsmath}
\usepackage{amsthm}
\usepackage{color} 
\usepackage{hyperref}
\hypersetup{colorlinks=true,citecolor=red, linkcolor=red}
\usepackage{subfiles}
%\usepackage[T1]{fontenc}

\SetUnicodeOption{mathletters}
\SetUnicodeOption{autogenerated}

\usepackage{graphicx}
\usepackage{pgf,tikz}
\usetikzlibrary{arrows}
\usetikzlibrary{babel}
\newcommand*\circled[1]{\tikz[baseline=(char.base)]{
            \node[shape=circle,draw,inner sep=2pt] (char) {#1};}}

\usepackage{makeidx}
\makeindex

\newcommand{\nuevotermino}[1]{\index{#1}\textbf{#1}}

\renewcommand{\baselinestretch}{1,4}
\usepackage[papersize={210mm,297mm},
            twoside,
            includehead,
            top=1in,
            bottom=1in,
            inner=0.75in,
            outer=1.0in,
			bindingoffset=0.35in]{geometry}


\theoremstyle{definition}
\newtheorem{theorem}{Teorema}[section]
\newtheorem{propi}[theorem]{Propiedades}
\newtheorem{condition}{Condition}
\newtheorem{coro}[theorem]{Corolario}
\newtheorem{obser}[theorem]{Observación}
\newtheorem{defi}[theorem]{Definición}
\newtheorem{example}[theorem]{Ejemplo}
\newtheorem{lema}[theorem]{Lema}
\newtheorem{nota}[theorem]{Nota}
\newtheorem{prop}[theorem]{Proposición}
\newtheorem*{dem}{Demostración}

\numberwithin{equation}{section}

\newcommand{\R}{\mathbb{R}}
\newcommand{\Q}{\mathbb{Q}}
\newcommand{\Z}{\mathbb{Z}}
\newcommand*{\QED}{\hfill\ensuremath{\blacksquare}}
\begin{document}
\begin{titlepage}
	\centering
	{\huge\bfseries Apuntes de Programación Matemática\par}
	\vspace{2cm}
	{\Large Javier Aguilar\par}
	{\Large Rafael González\par}
	{\Large Diego Pedraza\par}
	\vfill
	Esta obra está licenciada bajo la Licencia Creative Commons Atribución 3.0 España. Para ver una copia de esta licencia, visite \url{http://creativecommons.org/licenses/by/3.0/es/} o envíe una carta a Creative Commons, PO Box 1866, Mountain View, CA 94042, USA.

	{\large \today\par}
\end{titlepage}
\tableofcontents
\addtocounter{chapter}{1}
\subfile{PM2}
\subfile{PM3}
\subfile{PM4}
\subfile{PM5}
\subfile{PM6}
\subfile{PM7}
\printindex
\end{document}
