\documentclass[twoside]{article}
\usepackage{../estilo-ejercicios}
\newcommand{\x}{{\mathbf{x}}}
\newcommand{\y}{{\mathbf{y}}}
%--------------------------------------------------------
\begin{document}

\title{Geometría Aplicada}
\author{Rafael González López, Javier Aguilar Martín, Diego Pedraza López}
\maketitle

\begin{ejercicio}{3.1}
Hallar los coeficientes del polinomio $p(t)= 1-t+2t^2+t^3$ en las bases de Bernstein $\{B_i^3(t)\}_{i=0}^3$ y $\{B_i^4(t)\}_{i=0}^4$.
\end{ejercicio}
\begin{solucion}
Tomamos coordenadas de los $B_i^k(t)$ respecto de la base $\beta_1 =\{1,t,t^2,t^3\}$ y la base $\beta_2 = \{1,t,t^2,t^3,t^4\}$.
\begin{align*}
B_0^3(t) &= 1 - 3t +3t^2 -t^3 & B_0^4(t) &= 1 -4t+6t^2-4t^3+t^4\\
B_1^3(t) & = 3t-6t^2+3t^3 & B_1^4(t) &= 4t-12t^2+12t^3-4t^4\\
B_2^3(t) &= 3t^2 -3t^3 & B_2^4(t) &= 6t^2-12t^3+6t^4\\
B_3^3(t) &= t^3 & B_3^4(t) &=4t^3-4t^4\\
 & & B_4^4(t) &= t^4
\end{align*}
Lo que nos deja las siguientes matrices de cambio de base con respecto a las bases canónicas
$$
M_1 = 
\begin{pmatrix}
1 & 0 & 0 & 0\\
-3 & 3 & 0 &0\\
3 & -6 & 3 & 0\\
-1 & 3 & -3 & 1
\end{pmatrix} \qquad \qquad  M_2 = 
\begin{pmatrix}
1  & 0   & 0   & 0  & 0\\
-4 & 4   & 0   & 0  & 0\\
6  & -12 & 6   & 0  & 0\\
-4 & 12  & -12 & 4  & 0\\
1  & -4  & 6   & -4 & 1
\end{pmatrix}
$$ 
Pasamos a calcular directamente las coeficientes
\begin{gather*}
M_1^{-1}\begin{pmatrix}
1\\
-1\\
2\\
1
\end{pmatrix}
= 
\begin{pmatrix}
1\\
2/3\\
1\\
3
\end{pmatrix} \qquad \qquad M_2^{-1}  \begin{pmatrix}
1\\
-1\\
2\\
1\\
0
\end{pmatrix} = 
 \begin{pmatrix}
1\\
3/4\\
5/6\\
3/2\\
3
\end{pmatrix} 
\end{gather*}
\end{solucion}
\end{document}