\documentclass[TAN.tex]{subfiles}
\begin{document}

\chapter{Funciones aritméticas}
\section{Divisibilidad}

\section{Las funciones $d(n)$ y $σ(n)$}

\begin{prop} Si $f$ y $g$ son multiplicativas la función $f * g$ definida por
\[ f * g (n) = \sum_{d|n} f(d)g(n/d) \]
es también multiplicativa

\begin{dem}
\[ f*g(nm) = \sum_{c|nm}f(c)g\left(\frac{nm}{c}\right) = \sum_{a|n,b|m}f(ab)g\left(\frac{nm}{ab}\right) \]
Usando que $f$ y $g$ son multiplicativas:
\begin{align*}
	f*g(nm) & = \sum_{a|n,b|m} f(a)f(b)g(n/a)g(m/b)  = \sum_{a|n}f(a)g(n/a) \sum_{b|m}f(b)g(m/b) \\
	& = (f*g)(m) \cdot (f*g)(n)
\end{align*}
\end{dem}
\end{prop}

\section{Las funciones $φ(n)$ de Euler y $μ(n)$ de Möbius}

\section{Series de Dirichlet}

\section{Convergencia de series de Dirichlet}

\section{Crecimiento de funciones multiplicativas}

\section{Sumación parcial}

\section{El orden medio de $d(n)$}

\section{Método de exclusión-inclusión}

\section{Algunas otras funciones aritméticas}
\end{document}
