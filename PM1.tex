\documentclass[PM.tex]{subfiles}
\begin{document}
\chapter{Análisis convexo}
\section{Conjuntos convexos}

\begin{defi}
Un conjunto $S\subseteq\R^n$ es \textbf{convexo} si $\forall x^1,x^2\in S$ y $\forall\ \lambda\in[0,1]$ se cumple que $\lambda x^1 + (1-\lambda)x^2\in S$.
\end{defi}

\begin{defi}
Un \textbf{semiespacio} es un conjunto de la forma $H^-=\{x\in\R^n: a'x\leq b, a\in\R^n, b\in\R\}$ o de la forma $H^+=\{x\in\R^n: a'x\geq b, a\in\R^n\,b\in\R\}$.
\end{defi}

\begin{prop}
$H^-$ es convexo.
\end{prop}

\begin{dem}
Sea $x(\lambda)\equiv \lambda x^1 + (1-\lambda)x^2, \lambda\in[0,1]$. Se tiene lo siguiente
\[
a'x(\lambda)=a'(\lambda x^1 + (1-\lambda)x^2)=\lambda a'x^1 +(1-\lambda)a'x^2\leq \lambda b+(1-\lambda)b=b
\]
Como queríamos demostrar. $\QED$
\end{dem}

\begin{prop}
Si $S_1$ y $S_2$ son convexos entonces $S_1\cap S_2$ es convexo
\end{prop}
\begin{dem}
Definimos $x(\lambda)$ igual que en la demostración anterior. Dados $x^1,x^2\in S_1$ se tiene por convexidad de $S_1$ que $x(\lambda)\in S_1$. Análogamente, $x(\lambda)\in S_2$. Por lo tanto, $x(\lambda)\in S_1\cap S_2$. $\QED$
\end{dem}

\begin{defi}
Un \textbf{poliedro} es el conjunto de puntos definido por la intersección de un número finito de semiespacios. Visto matricialmente, dados $A\in\R^{m\times n}, b\in R^n$ se define el poliedro $P=\{x\in R^n: Ax\leq b\}$. Si denotameos por $a_i$ a la i-ésima fila de $A$ vemos que este conjunto se reescribe como $\{x\in\R^n: a_ix\leq b_i\ \forall i=1,\dots, n\}$, que es equivalente a la primera definición. Un poliedro acotado se llama \textbf{politopo}. 
\end{defi}

\begin{coro}
Los poliedros son convexos.
\end{coro}

\begin{defi}
Se define el \textbf{rayo} por un punto $x\in\R^n$ como el conjunto $\{y\in\R^n: y=\lambda x, \lambda\geq 0\}$.
\end{defi}

\begin{defi}
Un punto $x\in S$ con $S$ convexo se dice \textbf{punto extremo} si $x=\lambda x^1 +(1-\lambda)x^2$ con $x^1,x^2\in S$ y $\lambda\in (0,1)$ implica que $x^1=x^2=x$.
\end{defi}
\begin{nota}  En el caso de los poliedros, los puntos extremos coinciden con la idea de vértice. En una bola euclídea cerrada su frontera es de puntos extremos.
\end{nota}

\begin{defi}
Diremos que $d\in\R^n$ es una \textbf{dirección} de $S\subseteq\R^n$ si $\forall x\in S$ se tiene que $x+\alpha d\in S\ \forall\alpha\geq 0$.
\end{defi}

\begin{defi} Una dirección $d$ es \textbf{extrema} si $d=\alpha^1d^1+\alpha^2d^2$, siendo $d^i$ direcciones de $S$ y $\alpha^1,\alpha^2>0$ implica que $d^1$ es proporcional a $d^2$.
\end{defi}

\begin{defi}
Dados $r$ puntos $x^1,\dots, x^r\in\R^n$ llamamos \textbf{combinación convexa} de estos puntos a $\sum_{i=1}^r\lambda_i x^i$ si $\lambda_i\geq 0\ \forall i=1,\dots, r$ y $\sum_{i=1}^r\lambda_i=1$.
\end{defi}

\section{Teorema de Carathéodory}
\begin{defi} Sean $a^1,\dots, a^d\in\R^n$ afinmente independientes (esto es, que fijado un punto, los vectores diferencia de este con los demás son linealmente independientes). Se denomina \textbf{símplex $(d-1)$-dimensional} a 
\[
S(d-1)= \{x\in\R^n:x=\sum_{i=1}^d\lambda_i a^i,\lambda_i\geq 0, \sum_{i=1}^d\lambda_i =1\}
\]
\end{defi}
\begin{example}
$S(1)$ es un segmento, $S(2)$ es un triángulo relleno y $S(3)$ es un tetraedro relleno.
\end{example}

\begin{defi} Dado un conjunto $S\subset\R^n$, su \textbf{envolvente} (envoltura) \textbf{convexa} $CO(S)$ es
\[
CO(S)=\{x\in\R^n: x=\sum_{i=1}^d\lambda_i x^i, r<+\infty, x^i\in S^1, i=1,\dots, r, \lambda_i\geq 0, \sum_{i=1}^d\lambda_i =1\}
\]
$CO(S)$ es el menor convexo que contiene a $S$.
\end{defi}

\begin{defi} Diremos que $C\subset\R^n$ es un cono si contiene a todos sus rayos, es decir, $\forall x\in C$ $\lambda x\in C, \lambda\geq 0$.
\end{defi}
\begin{example}
Una semirrecta que pasa por el origen es un cono. No todos los conos son conjuntos convexos.

\definecolor{zzttqq}{rgb}{0.6,0.2,0.}
\begin{tikzpicture}[line cap=round,line join=round,>=triangle 45,x=1.0cm,y=1.0cm]
\clip(-3,-2.5) rectangle (3,2.5);
\fill[color=zzttqq,fill=zzttqq,fill opacity=0.10000000149011612] (0.,0.) -- (2.,3.) -- (4.,3.) -- cycle;
\fill[color=zzttqq,fill=zzttqq,fill opacity=0.10000000149011612] (0.,0.) -- (-4.5,-3.) -- (-2.5,-3.) -- cycle;
\fill[color=zzttqq,fill=zzttqq,fill opacity=0.10000000149011612] (0.,0.) -- (4.5,-3.) -- (2.5,-3.) -- cycle;
\draw [color=zzttqq] (0.,0.)-- (2.,3.);
\draw [color=zzttqq] (2.,3.)-- (4.,3.);
\draw [color=zzttqq] (4.,3.)-- (0.,0.);
\draw [color=zzttqq] (0.,0.)-- (-4.5,-3.);
\draw [color=zzttqq] (-4.5,-3.)-- (-2.5,-3.);
\draw [color=zzttqq] (-2.5,-3.)-- (0.,0.);
\draw [color=zzttqq] (0.,0.)-- (4.5,-3.);
\draw [color=zzttqq] (4.5,-3.)-- (2.5,-3.);
\draw [color=zzttqq] (2.5,-3.)-- (0.,0.);
\draw (2.,2.)-- (2.,-2.);
\draw (0.,-4.126140413226097) -- (0.,4.020852846215659);
\draw [domain=-6.367650924229384:10.452891711961001] plot(\x,{(-0.-0.*\x)/2.8572768088505205});
\begin{scriptsize}
\draw [fill=black] (2.,2.) circle (2.5pt);
\draw [fill=black] (2.,-2.) circle (2.5pt);
\end{scriptsize}
\end{tikzpicture}
\end{example}\

\begin{theorem}[Teorema de Carathéodory]
Sea $S\subseteq\R^n$. Si $x\in CO(S)$ entonces:
\[ x = \sum_{i=1}^{n+1} λ_i x^i \]
con $x^i\in S, λ_i ≥ 0$ y $\sum_{i=1}^{n+1}λ_i=1$. Es decir, cualquier punto de la envoltura se puede expresar como combinación lineal convexa de a lo sumo $n+1$ puntos.
\end{theorem}
\begin{dem}
Sea $r \in CO(S)$ y supongamos que:
\[ x = \sum_{i=1}^r λ_i x^i \]
Si $r ≤ n+1$, el teorema está probado. Si $r > n+1$, consideramos los $r-1$ puntos linealmente independientes $x²-x¹, x³-x¹, \dots, x^r-x¹$. Entonces existen $μ_2,\dots,μ_r$ no todos nulos tales que:
\[ 0 = \sum_{i=2}^r μ_i (x^i-x¹) = \sum_{i=2}^r μ_i x^i - x^1\sum_{i=2}^r μ_i\]
Tomamos $μ_1 := -\sum_{i=2}^r μ_i$ de manera que se tiene que:
\[ \sum_{i=1}^r μ_i x^i = 0\]
\[ \sum_{i=1}^r μ_i = 0 \]
Sea $α \in \R$. Consideramos:
\[ x = x - α0 = \sum_{i=1}^r λ_i x^i - α \sum_{i=1}^r μ_i x^i = \sum_{i=1}^r (λ_i - α μ_i)x^i \]
Definimos $γ_i = λ_i - α μ_i$ y buscamos $α$ de manera que $γ_i x^i$ sea una combinación convexa. Tomando:
\[ α = \min_{1≤i≤r} \left\{ \frac{λ_i}{μ_i} : μ_i > 0\right\} = \frac{λ_{i*}}{μ_{i*}} \]
Entonces se tiene que $γ_i \geq 0$ y:
\[ \sum_{i=1}^r γ_i = \sum_{i=1}^r (λ_i -α μ_i) = \sum_{i=1}^r λ_i - α \sum_{i=1}^r μ_i = 1 - α \cdot 0 = 1\]
Sin embargo, $γ_{i*} = λ_{i*} - \frac{λ_{i*}}{μ_{i*}}μ_{i*} = 0$, luego $x$ es una combinación convexa de a lo sumo $r-1$ puntos de $S$. Podemos repetir esta construcción hasta llegar a $r = n+1$.
$\QED$
\end{dem}
\end{document}