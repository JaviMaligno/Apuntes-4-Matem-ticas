\documentclass[twoside]{report}
\usepackage[utf8x]{inputenc}
\usepackage[spanish]{babel}
\usepackage{amssymb}
\usepackage{amsmath}
\usepackage{amsthm}
\usepackage{hyperref}
\usepackage{subfiles}

\SetUnicodeOption{mathletters}
\SetUnicodeOption{autogenerated}

\renewcommand{\baselinestretch}{1,4}
\usepackage[papersize={210mm,297mm},
            twoside,
            includehead,
            top=1in,
            bottom=1in,
            inner=0.75in,
            outer=1.0in,
			bindingoffset=0.35in]{geometry}

\theoremstyle{definition}
\newtheorem{theorem}{Teorema}[section]
\newtheorem{propi}[theorem]{Propiedades}
\newtheorem{condition}{Condition}
\newtheorem{consecuencia}[theorem]{Consecuencia}
\newtheorem{observacion}[theorem]{Observación}
\newtheorem{coro}[theorem]{Corolario}
\newtheorem{defi}[theorem]{Definición}
\newtheorem{example}[theorem]{Ejemplo}
\newtheorem{lemma}[theorem]{Lema}
\newtheorem{nota}[theorem]{Nota}
\newtheorem{prop}[theorem]{Proposición}
\newtheorem*{dem}{Demostración}
\newcommand*{\QED}{\hfill\ensuremath{\blacksquare}}

\newtheorem{summary}{Summary}
\numberwithin{equation}{section}

\newcommand{\R}{\mathbb{R}}
\newcommand{\X}{\mathbb{X}}
\newcommand{\resi}{\varepsilon_L}
\providecommand{\conv}[1]{\overset{#1}{\longrightarrow}}
\providecommand{\convcs}{\xrightarrow{CS}}
\providecommand{\supf}{\mathbb{X}}
\providecommand{\lrg}{\longrightarrow}
%--------------------------------------------------------
\begin{document}
\chapter{Geometría}
\section{Paralelismo en superficie}
Sean $\X:U\to\R^3$ s.s., $\alpha=\X(u(t),v(t))$ c.p.r. en $\X$.

\begin{defi}
Diremos que $X(t)$ es un campo vectorial tangente a lo largo de $\alpha$ si $X(t) \in T_{\alpha(t)}(\X)$. Por ejemplo, $\dot{\alpha}(t)$ o $S(t)$.
\end{defi}

\begin{defi}
Sea X(t) un c.v.t. a lo largo de $\alpha$. Se dirá que $X(t)$ es paralelo a lo largo de $\alpha$ si $X'(t) \parallel N(t)$. El parámetro t no es esencial para la definición de campo paralelo.
\end{defi}

\begin{example}
$\X(u)$ va a ser el plano euclídeo $OXY$ y $\alpha$ una curva contenida en él. Consideramos el campo vectorial $X(t)=(a(t),b(t),0)$ que es paralelo a lo largo de $\alpha$ si $X'(t)\parallel N=(0,0,1)$. Esto nos conduce a que $(a'(t),b'(t),0)\parallel (0,0,1)\Leftrightarrow a'(t)=b'(t)=0\Leftrightarrow X(t)=(a,b,0)$ constante. Vemos pues que el campo es paralelo en el sentido euclídeo.
\end{example}

\begin{example}
Consideramos la esfera $S^2$ y tomamos como curva $\alpha$ el ecuador de la misma. El campo vectorial $X(t)=(0,0,1)$ es paralelo a lo largo de $\alpha$ puesto que $X'(t)=(0,0,0)\parallel N(t)$.
\end{example}

\begin{example}
Volvamos a considerar la esfera, esta vez parametrizada como
\[ \X(\theta,\varphi)=(\cos\theta\cos\varphi,\sen\theta\cos\varphi,\sen\varphi)\quad \theta\in(0,2\pi),\varphi\in\left(-\frac{\pi}{2},\frac{\pi}{2}\right)\]
Definimos entonces la curva $\alpha(\theta)=\X(\theta,\frac{\pi}{4})=(\frac{\sqrt{2}}{2}\cos\theta,\frac{\sqrt{2}}{2}\sen\theta,\frac{\sqrt{2}}{2})$, que es un meridiano. De este modo, el vector tangente es $X(t)=\X_2(\theta,\frac{\pi}{4})$. Así pues, $X'(\theta)=(\frac{\sqrt{2}}{2}\sen\theta,-\frac{\sqrt{2}}{2}\cos\theta,0)$. Basta comprobar que $X'(t)$ no es perpendicular a $\X_2$ para comprobar que no es un campo paralelo a lo largo de $\alpha$.
\end{example}

\begin{theorem}
El c.v.t. $X(t)=\sum_{k=1}^n X^k \X_k$ a lo largo de α es campo paralelo a lo largo de α si y sólo si:
\[ \frac{dX^k(t)}{dt} + \sum_{i,j=1}^2 Γ_{ij}^k X^i(t) \frac{du^j}{dt} = 0 \quad \forall k=1,2 \]
\end{theorem}
\begin{dem}
Derivando $X(t)$:
\begin{align*} X'(t) & = \sum_{k=1}^2 \left(\frac{dX^k(t)}{dt}\X_k + \sum_{j=1}^2 X^k(t)\X_{kj}\frac{du^j}{dt}\right) \\
 & \overset{\text{Ec. Gauss}}{=} \sum_{k=1}^2 \left(\frac{dX^k(t)}{dt}\X_k + \sum_{j=1}^2 X^k(t)\left[\sum_{i=1}^2 Γ_{kj}^i \X_i + L_{kj}N\right]\frac{du^j}{dt}\right) \\
 & = \sum_{k=1}^2 \left[ \frac{dX^k}{dt}(t) + \sum_{i,j=1}^2 Γ_{ij}^k X^i(t) \frac{du^j}{dt}\right]\X_k + \left[\sum_{i,j=1}^2 L_{ij} X^i(t) \frac{du^j}{dt}\right]N(t)
\end{align*}
Llamamos al primer sumando \textbf{parte  tangente} y al segundo sumando, \textbf{parte normal}. De aquí:
\[ X'(t) \parallel N(t) \Leftrightarrow \frac{dX^k}{dt} + \sum Γ_{ij}^k X^i(t) \frac{du^j}{dt} = 0\quad \forall k=1,2\]
$\QED$
\end{dem}

\begin{defi}
Llamamos \textbf{derivada covariante} de $X(t) = \sum_{k=1}^2 X^k \X_k$ a:
\[ \frac{DX(t)}{dt} = \sum_{k=1}^2 \left[ \frac{dX^k(t)}{dt} + \sum_{i,j=1}^2 Γ_{ij}^k X^i(t) \frac{du^j}{dt}\right]\X_k\]
\end{defi}

\begin{consecuencia}
$X(t)$ c.v.t. es paralelo a lo largo de $α$ si y sólo si $\frac{DX(t)}{dt} = 0$.
\end{consecuencia}

\begin{consecuencia}
El concepto de \emph{paralelismo} (campo paralelo) es intrínseco de la superficie. Recuérdese que lo símbolos de Christoffel son intrínsecos a la superficie, ya que:
\[ Γ_{ij}^k = \frac{1}{2} \sum_{h=1}^2 g^{hk} (g_{ih,j}+g_{jh,i}-g_{ij,h}) \]
donde $g_{xy,z} = \dfrac{\partial g_{xy}}{\partial u^z}$.
\end{consecuencia}

\begin{theorem}[Existencia y unicidad de campos paralelos]
Sea $α(t) = \X(u¹(t), u²(t))$ c.p.r. en $\X : U \to \R³$. Entonces existe un único c.v.t. $X(t)$ paralelo a lo largo de $α$ con $X(t_0)=v_0$.
\end{theorem}

\begin{dem}
Si existe $X(t) = \sum_{k=1}^2 X^k \X_k$ c.v.t. paralelo a lo largo de $α$, $X^k(t)$ cumple que:
\[ \frac{dX^k(t)}{dt} + \sum_{i,j=1}^2 Γ_{ij}^k(t) X^i(t) \frac{du^j}{dt} = 0 \quad k=1,2 \]
\[ X^1(t_0) = v_0^1  \qquad X^2(t_0) = v_0^2 \]
Estamos ante 2 problemas de Cauchy. Por el teorema de Picard, existe solución y es única.
$\QED$
\end{dem}

\begin{defi}
El campo $X(t)$ obtenido integrando el sistema anterior y que cumple $X(t_0) = v_0$, se llama el \textbf{transporte paralelo} de $v_0$ a lo largo de $α$.
\end{defi}

%\begin{observacion}
%Dada $α(t)$ y $v_0$, el campo $X(t)$ paralelo es único, pero si el camino $α$ va de $p$ a $q$ de $X^k$ de $X(t)$ satisface las ecuaciones diferenciales de $(1.13)$
%\end{observacion}

\begin{observacion}
El transporte paralelo depende del camino.
\end{observacion}

\begin{example}
En $S^2(1)$, consideramos la curva $α(θ) = (\cos θ \cos φ_0, \sen θ \cos φ_0, \sen φ_0)$ con ${θ \in [0,2\pi]}$. Se tiene:
\[ \X_1 = (- \sen θ \cos φ, \cos u \cos φ, 0) \]
\[ \X_2 = (- \cos θ \sen φ, - \sen θ \sen φ, \cos φ) \]
\[ g_{11} = \cos²φ \qquad g_{12} = 0 \qquad g_{22} = 1 \]
\[ Γ_{12}^1 = - \tan φ \qquad Γ_{11}^2 = \cos φ \sen φ \]
\[ u¹ = θ, u² = φ \]

Para que $X(θ) = X^1(θ) \X_1(θ,φ_0) + X^2 \X_2(θ, φ_0)$ sea c.v.t. paralelo, se debe cumplir que:
\[ \frac{dX^1}{dθ} - \tan φ_0 X^2 = 0 \]
\[ \frac{dX^2}{dθ} + \cos φ_0 \sen φ_0 X^1 = 0 \]
De ahí se obtiene integrando que:
\[ X^1(θ) = \frac{1}{\cos φ_0} \sen ( θ \sen φ0) \]
\[ X^2(θ) = \cos(θ \sen φ_0) \]
Luego:
\[ X(θ) = \frac{1}{\cos φ_0} \sen (θ \sen φ_0) \X_1 - \cos (θ \sen φ_0) \X_2 \]
Sin embargo, $X(0) = \X_2(0,φ_0) \neq X(2\pi) = \frac{1}{\cos φ}\sen (2\pi \sen φ_0) \X_1 - \cos (2 \pi \sen φ_0) \X_2$. El transporte paralelo del vector $X(0)$ no va a mantener siempre constante el vector al pasar la curva por el mismo punto.
\end{example}
\begin{prop}
Sean $X(t),Y(t)$ c.v.t. paralelos a lo largo de una curva $α(t) = \X(u¹(t), u²(t))$. Entonces el transporte paralelo de vectores a lo largo de $\alpha$ conserva módulos y ángulos.
\end{prop}
\begin{dem}
Consideramos el producto escalar $X(t)\cdot Y(t)$. Derivando respecto de t y teniendo en cuenta que $X'(t) \parallel Y'(t) \parallel N(t)$:
\[
(X(t)\cdot Y(t))' = X'(t)\cdot Y(t) + X(t) \cdot Y'(t) = 0
\]
Como el producto escalar es constante, el módulo $\sqrt{X(t)\cdot X(t)}$ también lo ha de ser. Además, para el ángulo se tiene:
\[
\cos{\varphi(t)}=\frac{X(t)\cdot Y(t)}{|X(t)||Y(t)|} = cte
\]
\end{dem}
\begin{consecuencia}
Si tenemos una superficie $α(t) = \X(u¹(t), u²(t))$, $P\in \alpha$ y dos $v_1,v_2$ que son base de $T_P(\X)$. Entonces, si trasladamos dichos vectores mediante un transporte paralelo a Q, estos mantendrán el ángulo y sus respectivos módulos. Si tomamos $v_1$ y $v_2$ ortonormales, su transporte paralelo forma también una base ortonormal en Q. Esto se denomina \textbf{conexión de la geometría de la superficie M en P y Q}.
\end{consecuencia}
\begin{defi} Sea $α(t) = \X(u¹(t), u²(t))$ c.p.r. en $\X : U \to \R³$. Se llama \textbf{autoparalela} si el c.v.t. $\alpha'(t)$ es un campo paralelo a lo largo de $\alpha$.
\end{defi}
\begin{prop}
Una curva $\alpha(t)$ es autoparalela si y solo sí en $\alpha(t)$ t es función lineal del parámetro natural y es geodésica.
\end{prop}
\begin{dem}
Supongamos que $\alpha$ es autorparalela. Entonces $|\alpha'(t)| = cte = a'$, por tanto $\dfrac{ds}{dt}=a'$, luego $s = at + b$. Además:
\[
\alpha' = \dot{\alpha} \frac{ds}{dt} =\dot{\alpha} \frac{1}{a} \qquad
 \alpha''  = \ddot{\alpha}\frac{1}{a^2}
 \]
Como $\alpha'' \parallel N$, entonces $\ddot{\alpha}=k(s)n(s)\parallel N$. Por tanto, $\alpha$ es geodésica.
\end{dem}

\begin{coro}
Sea $X(s)$ c.v.t. paralelo a lo largo de $\alpha$. Sea $\varphi(s)=\widehat{(X(s),\dot{\alpha}(s))}$ entonces:
\begin{align*}
\cos{\varphi(s)}&=X(s)\dot{\alpha}(s)\\
 -\dot{\varphi}(s)\sin\varphi(s)& =  X'(s)\dot{\alpha}(s)+ X(s)\ddot{\alpha}(s)= X(s)(K_g(s)S(s)+K_n(s)N(s))=K_g(s)X(s)S(s)\\
  &= K_g(s)|X(s)||S(s)|cos{(\varphi+\frac{\pi}{2})}=-K_g\sin{\varphi}
\end{align*}
Luego $K_g = \dot{\varphi}$, similar a la curvatura plana.
\end{coro}
\begin{example}
Consideramos $S^2(1)$ y, sobre ella, $\alpha,\beta$ círculos máximos de polo norte (N) a polo sur (S). Por ser círculos máximos, $\alpha,\beta$ son geodésicas. Sea $X\in T_N(S^1)$ tangente a $\alpha$. Si realizamos un transporte paralelo de X hasta S por $\alpha$ usando $\dot{\alpha}$, obtenemos un vector $X_\alpha$ tangente a $\alpha$ en S. Sea $Y\in T_N(S^1)$ tangente a $\beta$ obteniendo, análogamente, $Y_\beta$. Sea $\theta=\widehat{\alpha,\beta}$. En el polo norte $\theta=\widehat{X,Y}$. Como $\dot{\beta}$ es un c.v.t. paralelo a $\beta$, podemos transportar $X$ a lo largo de $\beta$, obteniendo un vector $X_\beta$. Como $X$ y $\dot{\beta}$ mantienen sus ángulos, se tiene que $\widehat{X_\alpha,X_\beta}=2\theta$. Curiosiamente: $2\theta=A(R)$, donde R es la región esférica que delimitan $\alpha$ y $\beta$.
\end{example}
\begin{propi}
Sean $X(t),Y(t)$ c.v.t. a lo largo de $\alpha(t)$. Se tiene:
\begin{itemize}
\item La derivada covariante es un operador lineal.
\item $\dfrac{D(f(t)X(t))}{dt} = X(t)f'(t)+\dfrac{DX(t)}{dt}f(t)$.
\item $(X(t)\cdot Y(t))' = X'(t)Y(t)+X(t)Y'(t) = \dfrac{DX(t)}{dt}Y(t)+X(t)\dfrac{DY(t)}{dt}$.
\end{itemize}
\end{propi}


\section{La curvatura geodésica de una curva en una superficie}

\begin{prop}
Dada $\X : U \to \R^3$ s.s., existe un cambio de parámetro regular de forma que las líneas paramétricas de los nuevos parámetros son ortogonales (es decir, $\overline{g}_{12}=0$).
%Dibujo: Cambio de variable
\end{prop}

\begin{dem}
Empezamos buscando una curva $β(σ)=\X(u¹(σ),u²(σ))$ tal que sea ortogonal a las $u²$-paramétricas. Se tiene que el vector tangente a $β$ es:
\[ β'(σ) = \X_1 \frac{du¹}{dσ}+\X_2 \frac{du²}{dσ} \]
Para que $β'(σ) \cdot \X_2 = 0$:
\[ β'(σ) \cdot \X_2 = g_{12} \frac{du¹}{dσ} + g_{22} \frac{du²}{dσ} = 0 \]
El problema se reduce a integrar la 1-forma:
\[ g_{12} du¹ + g_{22} du² = 0 \]
Existe un factor integrante $μ(u¹,u²)\neq 0$ tal que $μ g_{12} du¹+μ g_{22} du² = dh(u¹,u²)=h_1 du¹+h_2 du²$ para algún $h$. Necesariamente $μ g_{12} = h_1$ y $μ g_{22} = h_2$. Sea el cambio de parámetros:
\[ f^{-1} = \begin{cases}
	\overline{u}¹ = u¹\\
	\overline{u}² = h(u¹,u²)
\end{cases}\]
El cambio de parámetros está bien definidio ya que $f^{-1} \in \mathcal{C}^\infty$ y $|J(f^{-1})| = \begin{vmatrix}1 & 0\\h1 & h_2\end{vmatrix} = h_2 = μ g_{22} \neq 0$. Luego:
\[ \overline{g}_{ij} = \sum g_{hk} \frac{du^h}{d\overline{u}^i} \frac{du^k}{d\overline{u}^j}\]
En particular:
\[ \overline{g}_{12} = \sum g_{hk} \frac{du^h}{d\overline{u}^1} \frac{du^k}{d\overline{u}^2} \]
Como $J(f^{-1}) = \begin{pmatrix}1 & 0\\h_1 & h_2\end{pmatrix} = \begin{pmatrix}1 & 0\\μ g_{12} & μ g_{22}\end{pmatrix}$, $J(f) = (J(f^{-1})^t)^{-1} = \begin{pmatrix}1 & 0 \\ -\frac{g_{12}}{g_{22}} & \frac{1}{μg_{22}}\end{pmatrix} = \begin{pmatrix}\frac{du¹}{d\overline{u}^1} & \frac{du^1}{d\overline{u}^2}\\\frac{du^2}{d\overline{u}^1} & \frac{du^2}{d\overline{u}^2}\end{pmatrix}$
\begin{align*}
	\overline{g}_{12} & = g_{11} \frac{du¹}{d\overline{u}¹} \frac{du¹}{d\overline{u}²} + g_{12} \frac{du¹}{d\overline{u}¹} \frac{du²}{d\overline{u}} + g_{21} \frac{du²}{d\overline{u}¹} \frac{du¹}{d\overline{u}²} + g_{22} \frac{du²}{d\overline{u}¹} \frac{du²}{d\overline{u}²} \\
	& = g_{11} \cdot 1 \cdot 0 + g_{12} \cdot 1 \cdot \left(\frac{1}{µ g_{22}}\right) + g_{21} \cdot \left( \frac{-g_{12}}{g_{22}}\right) \cdot 0  + g_{22} \left(\frac{-g_{12}}{g_{22}}\right) \frac{1}{μ g_{12}} \\
	& = \frac{1}{μ} - \frac{1}{μ} = 0
\end{align*}

Buscamos una fórmula para $K_g(s)$. Podemos suponer ya que las líneas paramétricas son ortogonales ($g_{12} = 0$). Sea $e_1(s) = \frac{\X_1}{|\X_1|} = \frac{\X_1}{\sqrt{g_{11}}}$ y $e_2(s) = \frac{\X_2}{|\X_2|} = \frac{\X_2}{\sqrt{g_{22}}}$ de manera que $\{e_1(s), e_2(s)\}$ es una base ortonormal sobre la curva $α(s)$. Sea $θ(s) = (\widehat{e_1(s), \dot{α}(s))}$. Luego:
\[ \dot{α} = \cos θ e_1 + \sen θ s_2 \]
\[ S = - \sen θ e_1 + \cos θ e_2\]

Obtenemos la derivada covariante de $\dot{α}$:
\begin{align*}
	\frac{D\dot{α}}{ds} & = tg(\frac{d\dot{α}}{ds}) = tg(-\sen{θ} \dot{θ} e_1 + \cos{θ} \dot e_1 + \cos θ \dot{θ} e_2 + \sen θ \dot e_2) \\
	& = - \sen θ \dot{θ} e_1 + \cos θ \frac{D e_1}{ds} + \cos θ \dot{θ} e_2 + \sen θ \frac{De_2}{ds}\\
	\frac{de_1}{ds} & = \dot{e}_1 = \frac{d}{ds}\left(\frac{\X_1}{\sqrt{g_{11}}}\right) = = \frac{d}{ds} \frac{1}{\sqrt{g_{11}}} \X_1 + \frac{1}{\sqrt{g_{11}}} \left(\X_{11} \frac{du¹}{ds} + \X_{12} \frac{du²}{ds}\right)
\end{align*}
Como $e_1(s) \cdot e_2(s) = 0$, luego $\dot{e}_1 \cdot e_2 = -e_1 \cdot \dot{e}_2$. Sea $w(s) = \dot{e_1}\cdot e_2$
\[ \frac{d}{ds} (e_i \cdot e_i) = 0 = 2 \dot{e}_i \cdot e_i \Rightarrow \dot{e}_i \perp e_i \]
\begin{equation} \frac{de_i}{ds} \cdot e_i = 0 \Rightarrow \frac{D e_i}{ds} \cdot e_i = 0 \Rightarrow \frac{De_i}{ds} \parallel e_j \text{ con } j \neq i \end{equation}
Por lo tanto:
\[ \frac{De_1}{ds} \cdot e_2 = \frac{de_1}{ds} \cdot e_2 = \dot{e}_1 \cdot e_2 = w(s)\]
Análogamente, $\frac{De_2}{ds} \cdot e_1 = -w(s)$. Usando además que (1.2.1):
\[ \frac{De_1}{ds} = w e_2 \]
\[ \frac{De_2}{ds} = -w e_1\]
Finalmente:
\begin{align*}
	\frac{D\dot{α}}{ds} & = -\sen θ \dot{θ} e_1 + \cos θ w e_2 + \cos θ \dot{θ} e_2 + \sen θ (-w) e_1 \\
	& = (-\dot{θ} -w) \sen θ e_1 + (\dot{θ} + w) \cos θ e_2 = (-\sen θ e_1 + \cos θ e_2) (\dot{θ}  w) = (\dot{θ} + w)\cdot S
\end{align*}
\end{dem}

\begin{theorem}
En las condiciones anteriores se tiene:
\[ K_g(s) = \dot{θ}(s) + \frac{1}{2 \sqrt{g_{11}g_{22}}} \left((g_{22})_1 \dot{u}²(s) - (g_{11})_2 \dot{u}¹(s) \right)\]
\end{theorem}

\begin{dem}
\[ \ddot{α}(s) = κ n = K_g \cdot S + K_n \cdot N \]
Multipicando por $S$:
\[ \ddot{α}(s) \cdot S = K_g \Rightarrow K_g = tg\left(\frac{d\dot{α}}{ds}\right) \cdot S = \frac{D\dot{α}}{ds} \cdot S = \dot{θ} + w \]
\end{dem}

Calculamos $w=\dot{e}_1 \cdot e_2$:
\[ \dot{e}_1 = \left(\frac{\X_1}{\sqrt{g_{11}}}\right)' = \frac{d}{ds} (\frac{1}{\sqrt{g_{11}}}) \X_1 + \frac{1}{\sqrt{g_{11}}} (\X_{11} \cdot \dot{u}^1 + \X_{12} \dot{u}²) \]
Por otro lado:
\[ 0 = g_{12} = \X_1 \cdot \X_2 \Rightarrow 0 = \X_{11} \X_2 + \X_1 \X_{12} \Rightarrow \X_{11} \cdot \X_2 = - \X_1 \cdot \X_{12} \]
\[ g_{11} = \X_1 \cdot \X_1 \Rightarrow 2 \X_1 \cdot \X_{12} = (g_{11})_2 \]
Usando las dos igualdades: $\X_{11}\cdot \X_2 = \dfrac{-(g_{11})_2}{2}$. Análogamente:
\[ g_{22} = \X_2 \cdot \X_2 \Rightarrow 2 \X_{12} \cdot \X_2 = (g_{22})_1 \]
Luego
\begin{align*} 
ω(s) & = \dot{e}_1 \cdot e_2 = \left[\frac{d}{ds} (\frac{1}{\sqrt{g_{11}}}) \X_1 + \frac{1}{\sqrt{g_{11}}} (\X_{11} \cdot \dot{u}^1 + \X_{12} \dot{u}²)\right] \cdot e_2 \\
	& = \left(\frac{1}{\sqrt{g_{11}}}\right)' \X_1 \cdot e_2 + \frac{1}{\sqrt{g_{22}}\sqrt{g_{11}}} (\X_{11} \dot{u}¹+ \X_{12} \dot{u}²) \cdot \X_2 = \frac{1}{\sqrt{g_{11} g_{22}}}
	\left[(\X_{11} \cdot \X_2)  \dot{u}^1 + (\X_{12} \cdot \X_2) \dot{u}²\right] \\
	& = \frac{1}{2 \sqrt{g_{11}g_{22}}} \left((g_{22})_1 \dot{u}²(s) - (g_{11})_2 \dot{u}¹(s) \right)
\end{align*}
\end{document}