\documentclass[twoside]{article}
\usepackage{../estilo-ejercicios}


\usepackage{enumerate}
%--------------------------------------------------------
\begin{document}

\title{Variable Compleja - Relación 4}
\author{Rafael González López}
\maketitle


\begin{ejercicio}{1}
Determinar el orden de las funciones:
$$
e^{az^4},\quad z^2e^{2z}-e^{3z},\quad e^z \cos z,\quad \cos\sqrt{z}, \quad e^5z-3e^{2z^2}, \quad e^{\cos z},\quad \int_0^1 e^{tz^2}dt
$$
\end{ejercicio}
\begin{solucion}
\begin{itemize}
\item[]
\item Si $a=0$, entonces $\rho=0$. En otro caso
$$
|e^{az^4}|\leq e^{|a||z|^4} \quad \forall z \in \C
$$
Luego $\rho\leq 4$. Supongamos que $\rho <4$, entonces $\exists r,A,B>0$ con $\rho<r<4$ tales que
$$
|e^{az^4}|\leq Ae^{B|z|^r} \quad \forall z \in \C
$$
Consideremos $z\in \R_{>0}$. En ese caso $|e^{az^4}|=e^{az^4}$ y $ Ae^{B|z|^r} = Ae^{Bz^r}$. Es claro que cuando $z\to \infty$ la desigualdad deja de ser cierta.
\item Naturalmente si $|z|>1$ entonces $|z^2e^{2z}|\geq |e^{2z}|$, por lo que $z^2e^{2z}$ tiene orden $\geq 1$. Además usando que $z^2$ tiene orden $\leq 1$
$$
|z^2e^{2z}|\leq |z|^2e^{2|z|}\leq Ae^{3|z|}
$$
Por tanto $z^2e^{2z}$ tiene orden $1$. Como $e^{3z}$ también y la diferencia es no nula, tenemos que es de orden $1$.
\item Sabemos que $\cos z = (e^{iz}+e^{-iz})/2$, luego
$$
|e^z \cos z| \leq e^{|z|}e^{|z|} = e^{2|z|}
$$
Por tanto, tiene orden $\leq 1$. El orden debe ser necesariamente 1, pues si no lo fuese razonaríamos como en el segundo apartado.
\item Sabemos que $\cos \sqrt{z} = (e^{i\sqrt{z}}+e^{-i\sqrt{z}})/2$,
\end{itemize}
\end{solucion}

\newpage

\begin{ejercicio}{2}
Sea
$$
f(z)=\frac{\sqrt{2}e^{-\pi i/8}\cos\left(\frac{\pi z}{2}\right)-e^{-\frac{\pi i}{2}z^2}}{\cos(\pi z)}
$$
\begin{enumerate}[(a)]
\item Demostrar que $f$ es una función entera, es decir sus singularidades aisladas son evitables.
\item ¿Cuál es el orden de $f(z)$?
\end{enumerate}
\end{ejercicio}
\begin{solucion}
\begin{enumerate}[(a)]
\item[]
\item Sabemos que el $\cos \pi z$ se anula si y solo si $z=(2k+1)/2$ con $k\in \Z$
\item 
\end{enumerate}
\end{solucion}

\newpage

\begin{ejercicio}{3}
Sea $\rho>1$.
\begin{enumerate}[(a)]
\item Demostrar que el producto
$$
\prod_{n=0}^\infty \left(1-\frac{z}{\rho^n}\right) 
$$
define una función entera $f(z)$.
\item ¿Cuál es el orden de la función $f$?
\item La función $f(z)$ y $f(\rho z)$ están conectadas por una ecuación funcional. Encontrar y demostrar esta ecuación.
\item Demostrar que $f(z)$ tiene orden $0$.
\end{enumerate}
\end{ejercicio}
\begin{solucion}
\begin{enumerate}[(a)]
\item[]
\item Es claro que para cada $r>0$, $|z|< r$, $F_n(z)=1-z/\rho^n$ verifica 
$$
|F_n(z)-1| = |z|/\rho^n < r/\rho^n \qquad \sum_{n=0}^\infty r/\rho^n = r \frac{\rho}{\rho-1}
$$
Por tanto, el producto define una función holomorfa en $|z|<r$. Como $r$ es arbitrario, tenemos que define una función $f(z)$ que es entera.
\item Sabemos que tenemos la desigualdad
$$
\abs{\prod_{n=0}^\infty \left(1-\frac{z}{\rho^n}\right)}  = \prod_{n=0}^\infty \abs{1-\frac{z}{\rho^n}} = \exp\left(\sum_{n=0}^\infty \log \abs{1-\frac{z}{\rho^n}}\right) \leq  \exp\left(\sum_{n=0}^\infty \log \left({1+\frac{|z|}{\rho^n}}\right)\right)
$$
Supongamos que el orden de $f(z)$ es $\rho>0$ y consideremos $0<r<\rho$. Podemos considerar además $r<1$. No es difícil probar que $\forall \varepsilon < 1$ $\exists C >0$ de manera que $\log(x+1)\leq C x^\varepsilon$ $\forall x \geq 0$. Por tanto:
$$
\sum_{n=0}^\infty \log \left({1+\frac{|z|}{\rho^n}}\right) \leq C |z|^{r}\sum_{n=1}^\infty \frac{1}{(\rho^{r})^n}
$$
Si $\rho<1$ entonces $\rho^{r}<1$ y la suma converge a un cierto número $B$. Si denotamos por $A = B\cdot C$ entonces es claro que se tiene de manera contradictoria
$$
\abs{\prod_{n=0}^\infty \left(1-\frac{z}{\rho^n}\right)} \leq e^{A|z|^{r}}
$$
\item Es claro que 
$$
f(z\rho) = \prod_{n=0}^\infty  \left(1-\frac{z}{\rho^{n-1}}\right)  = (1-z\rho)\prod_{n=1}^\infty  \left(1-\frac{z}{\rho^{n-1}}\right)  = (1-z\rho)f(z)
$$
\item Demostrado en (b).
\end{enumerate}
\end{solucion}

\newpage

\begin{ejercicio}{4}
Probar que si $f$ es una función entera de orden finito que omite dos valores, entonces $f$ es constante. Esto es cierto también para cualquier función entera y se conoce como el teorema pequeño de Picard.
\end{ejercicio}
\begin{solucion}
Supongamos que $f$ omite dos valores $a$ y $b$. Entonces $f(z)-a$ (resp. con $b$) es una función holomorfa que no se anula en $\C$. Como $f$ es de orden finito $\rho$, como aplicación directa del teorema de Hadamard tenemos que existe un polinomio $g(z)$ de grado menor o igual $k=\suelo{\rho}$ tal que $f(z)-a=e^{g(z)}$. Análogamente para $b$ tenemos el resultado para un polinomio $h(z)$. Por tanto, $f(z)=e^{g(z)}+a=e^{h(z)}+b$. Como estamos en $\C$ sabemos que si $h$ o $g$ son no constantes, existe algún $z_0\in \C$ tal que $h(z_0)=g(z_0)$, por lo que 
$$
f(z_0) = e^{g(z_0)}+a=e^{h(z_0)}+b \Rightarrow a = b
$$
Lo cuál no es posible por hipótesis, luego $h$ y $g$ son polinomios constantes y, por tanto, $f(z)=e^{g}+a$ que es constante.
\end{solucion}

\newpage


\begin{ejercicio}{5}
Supongamos que $f$ sea entera y no se anule y que ninguna de sus derivadas se anule. Probar que si $f$ es además de orden finito, entonces $f(z)=e^{az+b}$ para algunas constantes $a$ y $b$.
\end{ejercicio}
\begin{solucion}
Aplicamos el Teorema de Hadamard. Como $f$ no se anula y tiene orden finito $r$, sabemos que existe un polinomio $p(z)$ de grado $\leq \suelo{r}$ tal que
$$
f(z)=e^{p(z)}
$$
Sabemos que $f'(z)=p'(z)e^{p(z)}$ se anula si y solo si $p'(z)$ se anula. Como por hipótesis no se anula, deducimos que $p'(z)$ es una constante no nula y $p(z)$ es de la forma $az+b$.
\end{solucion}
\newpage

\begin{ejercicio}{6}
Proba que la ecuación $e^{z}-z=0$ tiene infinitas soluciones en $\C$.
\end{ejercicio}
\begin{solucion}
Por reducción al absurdo, supongamos que no es así, es decir, que existe una cantidad finita $s$ (en principio puede ser nula). Claramente $f(z)=e^z-z$ es una función entera de orden $1$. Como aplicación de teorema de Hadamard tenemos que
$$
f(z)=e^{az+b}\prod_{n=1}^s (1-z/a_n)e^z = e^{(a+s)z+b}\prod_{n=1}^s(1-z/a_n) = e^{(a+s)z+b}p(z)
$$
Igualando a $0$, tenemos que podemos tomar $b=0$ (o cualquier otro múltiplo entero de $2\pi i$). Podemos suponer que $a+s\neq 0$ pues otro caso es trivial. Claramente $f'(z)=e^z$ no se anula en $\C$. Sin embargo
$$
(e^{(a+s)z}p(z))' = ((a+s)p(z)+p'(z))e^{(a+s)z}
$$
se anula necesariamente en algún punto a menos que $p(z)$ sea un polinomio de grado $1$ ($s=1$) o una constante ($s=0$). Es fácil comprobar en ambos casos la igualdad sería absurda.
\end{solucion}
\newpage
\begin{ejercicio}{7}
Deducir del teorema de Hadamard que si $F$ es entera de orden de crecimiento $\rho$ no entero, entonces $F$ tiene infinitos ceros.
\end{ejercicio}
\begin{solucion}
El resultado se derivará inmediatamente a partir de la siguiente proposición. 
\begin{prop}
Sean $f,g$ dos funciones enteras con orden $\leq k$ entonces el producto $fg$ tiene orden $\leq k$.
\end{prop}
\begin{dem}
Sabemos que $\exists A,B,C,D>0$ tales que $\forall z\in \C$
$$
|f(z)|\leq Ae^{B|z|^k} \qquad |g(z)|\leq Ce^{D|z|^k}
$$
Por tanto, $\forall z \in \C$
$$
|f(z)g(z)| \leq ACe^{B|z|^k+D|z|^k} = ACe^{(B+D)|z|^k}
$$
\end{dem}
Una consecuencia inmediata es que podemos sustituir el producto de dos funciones por cualquier producto finito de funciones. Pasemos a resolver el problema. Supongamos que $f$ tiene una cantidad finita de ceros $a_1,\dotsc,a_s$. Entonces, por el Teorema de Hadamard tenemos que
$$
f(z)=e^{P(z)}z^m\prod_{n=1}^s E_k(z/a_n)
$$
donde $k$ es el entero tal que $k<\rho<k+1$. Analicemos esta descomposición. 
\begin{itemize}
\item Por un lado tenemos que cualquier polinomio tiene orden $0$ pues para todo $\rho>0$
$$
\lim_{|z|\to\infty} \frac{|P(z)|}{e^{|z|^\rho}} = 0
$$
Luego, en particular, tiene orden $\leq k$.
\item Por otro lado, si $P(z)$ es un polinomio de grado $k$ entonces para $|z|>1$ suficientemente grande
$$
|e^{P(z)}|\leq e^{|P(z)|}\leq e^{k|z|^k}
$$
Luego tiene orden $\leq k$. 
\item Tenemos $E_k(z/a_n)$ es el producto de un polinomio y una exponencial elevada a un polinomio de grado $k$. Usando la proposición y las consideraciones anteriores tenemos que $E_k(z/a_n)$ tiene orden $\leq k$. 
\end{itemize}
Por tanto, podemos escribir $f(z)$ como un producto finito de funciones de orden $\leq k$, de lo que colegimos que tiene orden $\leq k$, lo cuál es claramente una contradicción.
\end{solucion}
\newpage
\begin{ejercicio}{8}
Encontrar la región en que se trasnforma $\{z\in\C\mid |z|<1,\,\Im(z)>0\}$.
\end{ejercicio}
\begin{solucion}
Pensemos primero en qué se transforma la frontera de la región. Sabemos que $w(0)=-i/2$, $w(-1) = (2+i)/(i-2)$ y $w(1)= (2-i)/(2+i)$, que forman parte de una circunferencia. 
\end{solucion}
\newpage

\begin{ejercicio}{9}
\end{ejercicio}
\begin{solucion}
\end{solucion}
\newpage

\begin{ejercicio}{10}
Encontrar los transformados de los discos $|z|=r$ y las semirrectas $\arg(z)=\alpha$ mediante la transformación $T(z)=\frac{1}{2}(z+1/z)$.
\end{ejercicio}
\begin{solucion}
\begin{enumerate}[(a)]
\item[]
\item 
\item 
\end{enumerate}
\end{solucion}
\newpage

\begin{ejercicio}{11}
\end{ejercicio}
\begin{solucion}
\end{solucion}
\newpage
\end{document}