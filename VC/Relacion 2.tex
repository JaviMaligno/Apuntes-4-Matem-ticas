\documentclass[twoside]{article}
\usepackage{../estilo-ejercicios}

\usepackage{enumerate}
%--------------------------------------------------------
\begin{document}

\title{Variable Compleja - Relación 1}
\author{Rafael González López}
\maketitle
\begin{ejercicio}{1}
Demostrar que serie 
$$
\frac{1}{z}+\frac{1}{z(z+1)}+\frac{1}{z(z+1)(z+2)}+\cdots + \frac{1}{z(z+1)(z+2)\cdots  (z+n)}+\cdots$$
converge uniformemente en compactos de $\C\setminus\{0,-1,-2,-3,\dotsc\}$. Demostrar que la serie define una función meromorfa en el plano complejo con polos en los enteros negativos y el $0$.

¿Son los polos simples? ¿Cuál es el residuo en $z=-n$?
\begin{solucion}
Probemos primeramente que la serie converge uniformemente en compactos. Sea $\Omega = \C\setminus\{0,-1,-2,\dotsc\}$ y $K\subset B(0,r)$ un compacto de $\Omega$. Podemos suponer que $r>1$. Para probarlo vamos a utilizar el criterio M de Weierstrass. Tomamos $g_n = \prod_{j=0}^n \left(z+j\right)^{-1}$. Podemos suponer que $n\geq 2r$.  Tenemos entonces la siguiente acotación sobre $|g_n|$.

\begin{align*}
\abs{\prod_{j=0}^n \left(z+j\right)^{-1}} &= \prod_{j=0}^n \left|z+j\right|^{-1} =  \prod_{j=0}^{2r-1}\left|z+j\right|^{-1}  \prod_{j=2r}^n \left|z+j\right|^{-1}\\
&\leq  \prod_{j=0}^{2r-1}\left|z+j\right|^{-1}  \prod_{j=2r}^n (j-|z|)^{-1} \leq \prod_{j=0}^{2r-1}|z+j|\frac{1}{r^{n-2r}}\\
& \leq \frac{M}{r^{n-2r}}
\end{align*}
Hemos utilizado que el índice en el segundo producto $j\geq 2r$ y $|z|<r$
$$
|z|<r \Leftrightarrow -|z|>-r \qquad j-|z|> 2r-r = r \Leftrightarrow \frac{1}{j-|z|}< \frac{1}{r}
$$
\newpage
La última desigualdad se tiene dado que $h(z)=\prod_{j=0}^{2r-1}|z+j|^{-1}$ es una función continua en el compacto $K$. Como $r>1$, tomando índices adecuadamente, es claro que podemos aplicar el criterio M de Weierstrass. Por tanto, hemos probado que la serie converge uniformemente en cada compacto.\\

Vamos a ver que la serie define una función meromorfa en el plano complejo. Primeramente, tomando 
$$
 f_n(z) = \sum_{i=0}^n g_i(z) = \sum_{i=0}^n \prod_{j=0}^i (z+j)^{-1}
$$
Utilizando el apartado anterior, es claro que estamos en las condiciones del Teorema 2.5.2 de los apuntes, que establece:

\begin{theorem}
Si $\{f_n\}_{n=1}^\infty$ es una sucesión de funciones holomorfas que converge uniformemente a una función $f$ en cada subconjunto compacto de $\Omega$, entonces $f$ es una función holomorfa en $\Omega$.
\end{theorem}

Por tanto, nuestra serie define una función holomorfa $f(z)$ en $\Omega$. Para probar que la función es meromorfa en el plano complejo, tenemos que ver que todas las discontinuidades son polos. Sea $a\in \N\cup\{0\}$. Vamos a demostrar que todas las discontinuidades son polos son simples a la vez que calculamos sus residuos, pues esto es equivalente a que 
$$
\lim_{z\to -a}(z+a)f(z) \in\C\setminus\{0\}
$$
Primeramente, es claro que utilizando un argumento análogo al anterior podemos probar que $h_n = (z+a)f_n$ define una función holomorfa en $\Omega \cup\{-a\}$, por lo que 
$$
\lim_{z\to -a}(z+a)f(z) \in\C
$$
y además la serie

$$
p_n(z) = \sum_{i=1}^n g_i(z) (z-a) = \sum_{i=0}^n \prod_{j=0}^i (z+j)^{-1}(z+a)
$$

converge uniformemente en cada compacto de $\Omega\cup\{-a\}$.  Tenemos que si $i\geq a$
\begin{align*}
\lim_{z\to -a}(z+a)\prod_{j=0}^i \left(z+j\right)^{-1}& =  \lim_{z\rightarrow-a}\left(\prod_{j=0}^{a-1}(z+j)^{-1}\right)\frac{(z+a)}{(z+a)}\left(\prod_{j=a+1}^{n}(z+j)^{-1}\right)\\
&=\lim_{z\rightarrow-a}\left(\prod_{j=1}^{a}(z+a-i)^{-1}\right)\frac{(z+a)}{(z+a)}\left(\prod_{j=1}^{n-a}(z+a+j)^{-1}\right) \\
&=\left(\prod_{j=1}^{a}(-1)(i)^{-1}\right)(1)\left(\prod_{j=1}^{n-a}(i)^{-1}\right)=\frac{(-1)^a}{a!(n-a)!}
\end{align*}
Por otra parte, si $i<a$ entonces
$$
\lim_{z\to -a}(z+a)\prod_{j=0}^i \left(z+j\right)^{-1} = 0
$$
Utilizando la convergencia uniforme en compactos puede intercambiarse límite y sumatorio, quedando finalmente
$$
\lim_{z\to -a}(z+a)f(z) = \sum_{n=a}^\infty \frac{(-1)^a}{a!(n-a)!} = \frac{(-1)^a}{a!}\sum_{n=0}^\infty \frac{1}{n!} = \frac{(-1)^a e}{a!} \in \C\setminus\{0\}
$$
\end{solucion}
\end{ejercicio}
\newpage

\begin{ejercicio}{2}
Demostrar que la función definida por la integral 
$$
F(z)=\int_0^\infty\frac{tdt}{(t^2+z^2)(e^{2\pi t}-1)}
$$
es analítica en $\Re(z) >0$.
\end{ejercicio}
\begin{solucion}
Queremos ver que $F(s)$ es analítica utilizando el teorema de convergencia. Las dos primeras condiciones son claras. Consideramos $\Re(s)>\sigma >0$. Tengamos en cuenta que si $z=x+yi$, realizamos el siguiente razonamiento.
\begin{enumerate}
\item Si $|y|\leq x/2$, entonces es claro que $x^2-y^2 \geq x^2-x^2/4 \geq 3\sigma^2/4\geq 0$, $t\geq 0$ y usamos que $|z|\geq |\Re(z)|$. Por tanto,
$$
\abs{t^2+x^2-y^2+2xyi} \geq |t^2+x^2-y^2| = t^2 + x^2-y^2 \geq x^2-y^2 \geq \frac{3\sigma^2}{4}
$$
\item Si $|y|>x/2$ entonces $|2xyi|\geq 2x|y| \geq x^2$. Además, $|z|\geq |\Im(z)|$, luego
$$
\abs{t^2+x^2-y^2+2xyi} \geq |2xyi|\geq 2x|y| \geq x^2 \geq \sigma^2 \geq \frac{3\sigma^2}{4}
$$
\end{enumerate}
Deducimos, por tanto, que en el dominio considerado
$$
\abs{t^2+z^2} = \abs{t^2+x^2-y^2+2xyi} \geq \frac{3\sigma^2}{4} 
$$
Por tanto, podemos acotar la función del integrando de la siguiente manera
\begin{align*}
\abs{\frac{1}{t^2+z^2}\frac{t}{e^{2\pi t}-1}} \leq \frac{4}{3\sigma^2}\frac{t}{e^{2\pi t}-1}= g(t)
\end{align*}
Claramente $g(t)$ es una función continua en $[0,+\infty)$ y estrictamente positiva. Podemos probar que la integral converge, por ejemplo, por paso al limite. Claramente $e^{-\pi t}$ es integrable en $[0,+\infty)$, luego
$$
\lim_{t\to \infty}\frac{\frac{t}{e^{2\pi t}-1}}{e^{-\pi t}} = \lim_{t\to \infty}\frac{t}{e^{\pi t}-e^{-\pi t}} = 0
$$
Por lo que $g(t)$ también es integrable en $[0,+\infty)$. Por tanto, tenemos el resultado.
\end{solucion}
\newpage

\begin{ejercicio}{3}
Calcular la integral
$$
\int_0^\infty \frac{x^a}{(1+x^2)^2}dx \qquad -1<a<3
$$
\end{ejercicio}
\begin{solucion}
Consideramos $-1<a<3$ y tomamos un $\log x$ con $0<arg(x)<2\pi$.  Tomamos un keycontour $\Gamma_{\varepsilon,R}$ centrado en $z=0$ de formar que el corredor sea simétrico respecto del eje  real positivo. Calculamos los residuos de los polos en el interior del contorno, que son claramente dos polos de segundo orden.
\begin{align*}
\lim_{x\to i} \left(\frac{x^a}{(x+i)^2}\right)' &= \lim_{x\to i}\frac{e^{\log{x}(a-1)}(a(x+i)-2x)}{(x+i)^3} = \frac{e^{\pi (a-1)i/2}2i(a-1)}{8i^3}\\ 
&=-\frac{(a-1)}{4}e^{\pi (a-1)i/2} =\frac{(a-1)}{4}ie^{\pi ai/2} \\
\lim_{x\to -i} \left(\frac{x^a}{(x-i)^2}\right)' &= \lim_{x\to -i}\frac{e^{\log{x}(a-1)}(a(x-i)-2x)}{(x-i)^3} =\frac{e^{3\pi (a-1)i/2}2i(-a+1)}{-8i^3}\\ 
&=-\frac{(a-1)}{4}e^{3\pi (a-1)i/2} = -\frac{(a-1)}{4}e^{3\pi ai/2}
\end{align*}

De manera que
$$
\int_{\Gamma_{\varepsilon,R}} \frac{e^{a \log x}}{(1+x^2)^2}dx = 2\pi i\left( \frac{a-1}{4}ie^{\pi a/2}-\frac{a-1}{4}ie^{3\pi i a /2}\right)
$$
Veamos qué ocurre en las cuatro integrales que conforman la integral de nuestro contorno. Primeramente, en $C_\varepsilon$ (la circunferencia interior)
$$
\abs{\int_{C_\varepsilon}\frac{e^{a \log x}}{(1+x^2)^2} }=\int_0^{2\pi} \abs{\frac{\varepsilon^a e^{ait} e^{it} i \varepsilon}{(1+\varepsilon^2e^{it})^2}}dt\leq \int_0^{2\pi}\frac{\varepsilon^a\varepsilon}{(1-\varepsilon^2)^2}dt = \frac{\varepsilon^a}{(1-\varepsilon^2)^2}2\pi \varepsilon 
$$ 
Luego se anula cuando $\varepsilon \to 0$, ya que $a>-1$. Análogamente
$$
\abs{\int_{C_R} \frac{e^{a \log x}}{(1+x^2)^2} dx } \leq  \frac{R^a}{(R^2-1)^2}2\pi R
$$
Que se anula cuando $R\to<\infty$, pues $a<3$. Por la elección del $arg(x)$ tenemos que en los laterales del corredor, cuando hacemos tender su separación a $0$, tendremos que tomar en un caso $arg(x)=0$ y en otro $arg(x)=2\pi$, de donde deducimos que tienden a la integral que deseamos calcular $I$ y a $-e^{2\pi a}I$. Luego
\begin{align*}
I(1-e^{2\pi i a}) = 2\pi i\left( \frac{a-1}{4}ie^{\pi a/2}-\frac{a-1}{4}ie^{3\pi i a /2}\right)
\end{align*}
\end{solucion}
\newpage


\begin{ejercicio}{4}
Calcular la integral
$$
\int_0^\infty \frac{\sin ax}{\sinh x}dx \qquad a \in \R
$$
\end{ejercicio}
\begin{solucion}
Tengamos en cuenta ambas función son impares, luego el cociente es par. Por tanto,
$$
\int_0^\infty \frac{\sin ax}{\sinh x}dx 
 = \frac{1}{2} \int_{-\infty}^\infty \frac{\sin ax}{\sinh x}dx = \frac{1}{2}\Im\left( \int_{-\infty}^\infty \frac{e^{iax}}{\sinh x}dx\right)
$$
Consideramos ahora $\gamma_R$ el rectángulo apoyado en el segmento $[-R,R]$. Como tenemos una discontinuidad en el origen y en $2\pi i$, cambiamos este contorno por uno que hay $2$ semicircunferencias de radio $\varepsilon$ (la parte de abajo. De esta forma en $\gamma_R$ no hay ninguna singularidad y en su contorno solo queda la del $0$. Calculamos el residuo que es un polo simple
$$
\lim_{x\to0}x\frac{e^{iax}}{\sinh x} = \lim_{z\to0}\frac{e^{iax}+iaxe^{iax}}{\cosh x} = 1
$$
Dividimos nuestra curva de manera natural en $8$, los $6$ segmentos y las $2$ semicircunferencias. Tengamos en cuenta que en los dos segmentos inferiores cuando $R\to \infty$ la integral converge a la que deseamos. En los dos segmentos inferiores también lo hace pero con signo opuesto y multiplicado por $e^{-2\pi a}$. Por tanto, veamos qué ocurre en los segmentos verticales y las semicircunferencias.
\begin{itemize}
\item En la semicircunferencia inferior $C^1_\varepsilon$ tenemos la parametrización $\varepsilon e^{ti}$, $t=[\pi,2\pi]$.
\begin{align*}
\abs{\int_{C^1_\varepsilon} \frac{e^{iax}}{\sinh x}dx} &= \abs{\int_{\pi}^{2\pi} \frac{e^{iae^{it}\varepsilon}\varepsilon e^{it}}{\sinh(\varepsilon e^{it})}dt}\\
& \leq 2 \varepsilon e^{|a|\varepsilon}\int_\pi^{2\pi}\frac{dt}{\abs{e^{\varepsilon e^{it}}-e^{-\varepsilon e^{it}}}}\\
&\leq 2 \varepsilon e^{|a|\varepsilon}\int_\pi^{2\pi}\frac{dt}{}
\end{align*}
$$
\abs{e^{\varepsilon e^{it}}-e^{-\varepsilon e^{it}}} \geq e^\varepsilon - |e^{-\varepsilon^{e^{it}}}| \geq e^\varepsilon - |e^{-\varepsilon(\cos(t)+i\sin(t))}| = e^\varepsilon - |e^{-\varepsilon\cos(t)}| = e^\varepsilon - \frac{1}{|e^{\varepsilon \cos(t)}|}
$$
\end{itemize}
\end{solucion}
\newpage


\begin{ejercicio}{5}
Sea $f\func{\Omega}{\C}$ una función holomorfa en una región $\Omega$ que contiene el eje real $\R$. Demostrar que si existe una sucesión convergente de números reales $\{x_n\}$ tales que $f(x_n)\in \R$ entonces $f$ toma valores reales para todo $x\in \R$.
\end{ejercicio}
\begin{solucion}
Tengamos en cuenta que $\{x_n\}$ no es una sucesión constante, pues en ese caso no sería cierto. Por ejemplo, para la sucesión constantemente el origen, la función $f(z)=zi$ verifica que $f(x_n)=0\in \R$ pero no toma más valores reales a lo largo de la recta real. Por tanto, supongamos que la sucesión es no constante. Sea $h(z)=\overline{z}$, definimos $\Omega^* = h(\Omega)$. Como $h$ es un isomorfismo continuo, luego es una aplicación abierta. Se desprende, por tanto, que $\Omega^*$ es una región que contiene al eje real. Definimos además
$$
G':=\Omega \cap \Omega^*
$$
Naturalmente $G'$ es abierto por ser intersección de abierto. Para ver que es región consideremos el siguiente razonamiento.  Aunque $G'$ podría ser no conexo, lo que es claro es que $\R$ debe estar en alguna completamente contenido en alguna componente conexa que denominamos $G$. Definimos ahora \begin{equation*}\begin{split}
	g : & \  G \rightarrow \C \\
	& z \mapsto g(z) = f(z) - \overline{f(\overline{z})},
	\end{split}
	\end{equation*}
	que está bien definida por nuestra elección de $G$. Además, como $f(z)$ y $ \overline{f(\overline{z})}$ son funciones holomorfas en $G$, se sigue que $g$ también lo es. \\
	Además, tenemos que \begin{equation*}
	g(x_n) = f(x_n) - \overline{f(\overline{x_n})} = f(x_n) - f(x_n) = 0 \  \forall n \in \N.
	\end{equation*}
	Así, como $(x_n)$ es una sucesión convergente en $G$ y no constante, se sigue que $g \equiv 0$ en $G$, lo que implica que \begin{equation*}
	f(z) = \overline{f(\overline{z})} \quad \text{para todo} \ z \in G.
	\end{equation*}
	Como $G$ contiene a la recta real, se tiene que \begin{equation*}
	f(x) = \overline{f(x)} \quad \text{para todo} \ x \in \R,
	\end{equation*}
	lo que prueba que $f$ toma valores reales en toda la recta real. 
\end{solucion}
\newpage


\begin{ejercicio}{6}
Sea $f\func{\Omega}{\C}$ una función holomorfa en una región $\Omega$ tal que para cierto $\varepsilon>0$ el disco de centro $0$ y radio $\varepsilon$ esté contenido en $\Omega$. Demostrar que $f$ toma valores reales en el segmento $(-\varepsilon,\varepsilon)$ y en el segmento $(-i\varepsilon,i\varepsilon)$ si y solo si el desarrollo de $f$ en el entorno de $0$ es de la forma
$$
f(z)=\sum_{n=0}^\infty a_nz^{2n}
$$
con los coeficientes $a_n \in \R$.
\end{ejercicio}
\begin{solucion}
Una de las implicaciones es clara. Si $f$ tiene dicho desarrollo es el origen, que sabemos es válido al menos en $|z|<\varepsilon$, si $z\in (-\varepsilon,\varepsilon)$ y $a_n\in \R$ entonces $a_nz^{2n} \in \R$. Por tanto, la suma que supones convergente también es real. Análogamente, para $z\in (-\varepsilon i,\varepsilon i)$, $z^{2n} \in (-\varepsilon^{2n},\varepsilon^{2n})$, luego $a_n z^{2n}\in \R$.

Recíprocamente, supongamos que $f$ toma valores reales en dichos segmentos. Podemos suponer que no son idénticamente cero al ser holomorfa. 
\begin{itemize}
\item Consideramos el desarrollo en serie de la función
$$\sum_{n=0}^\infty a_n z^n
$$
Para $z\in(-\varepsilon,\varepsilon)$, $z=\overline{z}$. Como toma valores reales $f(z)=f(\overline{z})=\overline{f(\overline{z})}$, luego
$$
\sum_{n=0}^\infty a_n z^n = \sum_{n=0}^\infty \overline{a_n} \overline{z} = \sum_{n=0}^\infty \overline{a_n}z^n
$$
De donde deducimos que $a_n=\overline{a_n}$, por lo que son reales.
\item Además, también para $z\in (-\varepsilon,\varepsilon)$ entonces tenemos que $f(iz) = \overline{f(zi)}$ y toma valores reales y teniendo en cuenta que $a_n \in \R$
$$
\sum_{n=0}^\infty a_n (iz)^n = \sum_{n=0}^\infty {\overline{a_n ({iz})^n}} = \sum_{n=0}^\infty {a_n ({-iz})^n}  
$$
Por lo que 
$$
a_{3+4n} (iz)^{3+4n} = a_{3+4n}(-i)z^{3+4n} = a_{3+4n}(-iz)^{3+4n} = -a_{3+4n}(-i)z^{3+4n}
$$
de donde $-a_{3+4n}=a_{3+4n}$, de donde se deduce que es nulo.
$$
a_{1+4n} (iz)^{1+4n} = a_{1+4n}iz^{1+4n} = a_{1+4n}(-iz)^{1+4n} -a_{1+4n}iz^{1+4n}
$$
de donde $a_{1+4n}=-a_{1+4n}$, de donde se deduce que es nulo.  Es decir, es nula para todos los impares, lo que demuestra el resultado.
\end{itemize}
\end{solucion}
\newpage


\begin{ejercicio}{7}
Considerar el compacto $K$ de la figura: Consistente en un disco de radio $2$ menos la unión de dos discos de radio $1$.
$$
K=\{z\in \C\mid |z|\geq 2, \; |z-1|\geq 1,\; |z+1|\geq 1\}
$$
¿Podemos aproximar una función holomorfa en un entorno de $K$, uniformemente en $K$ por funciones racionales con polos en $-1$ y $1$?
\end{ejercicio}
\begin{solucion}

\end{solucion}
\newpage


\begin{ejercicio}{8}
Sea $G$ el abierto interior del compacto $K$ definido en el problema anterior. ¿Es posible aproximar cualquier función holomorfa en $G$ uniformemente en cada compacto $L\subset G$ mediante polinomios? Demostrar que la respuesta es correcta.
\end{ejercicio}
\begin{solucion}
\end{solucion}
\newpage


\begin{ejercicio}{9}
Supongamos que $f$ es una función continua no nula en $\overline{\mathbb{D}}$ que es holomorfa en $\mathbb{D}$. Probar que si 
$$
|f(z)|=1 \text{ siempre que } |z|=1
$$
entonces $f$ es constante.
\end{ejercicio}
\begin{solucion}
Primeramente, como $f$ es analítica y no se anula en el disco unidad, entonces tiene sentido considerar en $|z|>1$ la función $f(1/z)$, que será analítica en $|z|>1$. Sabemos que si $g(z)$ es analítica entonces $\overline{g(\overline{z})}$ es analítica.  Por tanto, podemos considerar la función de $f$ dada por 
$$
g(z)=
\begin{cases}
f(z) & |z|\leq 1\\
\overline{f\left({1}/{\overline{z}}\right)} & |z|>1
\end{cases}
$$
Es claro que $g$ es analítica en $|z|\neq 1$. Tengamos en cuenta lo siguiente. Sabemos que $g$ es continua en $\overline{\mathbb{D}}$, que es compacto, luego $g$ está acotada por una cierta constante $M>0$. Ahora tengamos en cuenta que si $|z|>1$ entonces $\abs{1/\overline{z}}<1$, luego 
$$|g(z)| = |\overline{f\left({1}/{\overline{z}}\right)}| = \abs{f\left({1}/{\overline{z}}\right)}<M
$$
Además, sabemos que $|f(z)|=1$ en $|z|=1$. Por tanto, si probamos que $g$ es entera, entonces por el Teorema de Liouville tendremos que $g$, y por tanto $f$, es constante.
\end{solucion}
\newpage
\end{document}