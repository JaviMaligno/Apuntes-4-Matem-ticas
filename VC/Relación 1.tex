\documentclass[twoside]{article}
\usepackage{../estilo-ejercicios}

\usepackage{enumerate}
%--------------------------------------------------------
\begin{document}

\title{Variable Compleja - Relación 1}
\author{Rafael González López}
\maketitle

\begin{ejercicio}{1}
Interpretar geométricamente las relaciones siguientes:
  \begin{AutoMultiColItemize}
  \item[a)] $|z-1|+|z+i|=4$
  \item[b)] $|z-2|+|z+2|>3$
  \item[c)] $|z-a|=|z-b|$
  \item[d)] $|z^2-z|\leq 1$
  \item[e)] $\Re(z)+\Im(z)<1$
  \item[f)] $0<\Re(iz)<1$
  \item[g)] $\alpha < \arg(z) < \beta$
  \end{AutoMultiColItemize}
suponiendo en $g)$ que $-\pi<\alpha<\beta<\pi$. 
\end{ejercicio}
\begin{solucion}
Sea $z=x+yi$. 
\begin{enumerate}[a)]
\item Tomando cuadrados apropiadamente
\begin{gather*}
\sqrt{(x-1)^2+y^2}+\sqrt{x^2+(y+1)^2}=4\\
(x-1)^2+y^2 + x^2+(y+1)^2 + 2\sqrt{(x-1)^2+y^2}\sqrt{x^2+(y+1)^2} = 16\\
2\sqrt{(x-1)^2+y^2}\sqrt{x^2+(y+1)^2} = 16 - (2 x^2 - 2 x + 2 y^2 + 2 y + 2)\\
4((x-1)^2+y^2)(x^2+(y+1)^2) = (16 - (2 x^2 - 2 x + 2 y^2 + 2 y + 2))^2\\
15 x^2 + x (-2 y - 16) + 15 y^2 + 16 y = 48	
\end{gather*}
Luego estamos ante una elipse que tiene el origen en su interior.
\item Análogamente al primer apartado, tenemos el exterior de una elipse que no contine al origen.
\item Si $a=b$, entonces todo $z$ es solución. Si son distintos, entonces
$$
(x-a)^2+y^2 = (x-b)^2 +y^2  \Leftrightarrow x^2+a^2-2ax = x^2+b^2 -2bx 
$$
Luego tenemos la recta vertical $x = \dfrac{b^2-a^2}{2(b-a)}=\dfrac{b+a}{2}$.
\item 
\item Dado que $\Re(z)=x$ y $\Re(z)=y$, tenemos que $x+y<1$, es decir, el semiplano abierto limitado por la recta $y=1-x$ que contiene al origen.
\item $iz= ix-y$, luego $\Re(iz)= -y$. Por lo que $-1<y<0$ o es la banda horizontal abierta comprendida entre $y=-1$ e $y=0$.
\item 
\end{enumerate}
\end{solucion}
\newpage
\begin{ejercicio}{2}
\end{ejercicio}
\begin{solucion}
\end{solucion}
\newpage


\begin{ejercicio}{3}
Sea $g$ una holomorfa en un abierto $\Omega$ conexo. Probar que en cualquiera de los siguientes casos
\begin{enumerate}[(i)]
\item Si $\Re(g)$ es constante.
\item Si $\Im(g)$ es constante.
\item Si $|g|$ es constante.
\end{enumerate}
podemos concluir que $g$ es constante.
\end{ejercicio}
\begin{solucion}
Los tres resultados se desprenden inmediatamente a partir de las ecauciones de Cauchy y el hecho de que el abierto $\Omega$ sea conexo, pues si no lo fuese, podría haber componentes conexas donde la constante variase. Denotemos $\Re(g)=u$, $\Im(g)=v$.
\begin{enumerate}[(i)]
\item Si $u$ es constante, entonces las ECR nos dicen que
$$
0 = \frac{\partial u}{\partial x} = \frac{\partial v}{\partial y} \qquad 0 = \frac{\partial u}{\partial x} = - \frac{\partial u}{\partial y}$$
Por lo que 
$$
0 = \frac{\partial v}{\partial x}  =\frac{\partial v}{\partial y} 
$$ 
Por lo que $v$ es constante.
\item Este apartado es completamente análogo al anterior.
\item Una posibilidad sería la siguiente. Si $|g|$ es constante, entonces la imagen es una cierta circunferencia de $\C$. Utilizando el Teorema de la aplicación abierta (Tema 3), tenemos que la imagen de una región por una aplicación holomorfa es cerrada si y solo si es constante. Sin embargo, vamos a probarlo de manera análoga a lo que hemos estado haciendo. Si $|g|$ es constante, entonces $|g|^2 =u^2+v^2 = c$. Podemos suponer que $c>0$, pues en otro caso es trivial. Por tanto, derivando
$$
u\frac{\partial u}{\partial x} + v\frac{\partial v}{\partial x} = 0 \qquad u\frac{\partial u}{\partial y} + v\frac{\partial v}{\partial y} = 0
$$
Sustituyendo las ECR
$$
u\frac{\partial u}{\partial x} - v\frac{\partial u}{\partial y} = 0 \qquad u\frac{\partial u}{\partial y} + v\frac{\partial u}{\partial x} = 0
$$
Multiplicamos la primera ecuación por $u$, la segunda por $v$ y sumamos, obteniendo
$$(u^2+v^2)\frac{\partial u}{x} = 0 \Rightarrow \frac{\partial u}{x} = 0$$
Si multiplicamos la primera por $-v$ y la segunda por $u$, al sumar obtenemos
$$(u^2+v^2)\frac{\partial u}{y} = 0 \Rightarrow \frac{\partial u}{y} = 0 
$$
Por tanto, $u$ es constante y por el primer apartado tenemos que $v$ también lo es.
\end{enumerate}
\end{solucion}
\newpage


\begin{ejercicio}{4}
Para $a$ y $b$ complejo y $a\neq 0$ se define $a^b = e^{b\log a}$. Como $\log a$ toma infinitos valores, en general $a^b$ toma infinitos valores.
\begin{enumerate}[(a)]
\item ¿Qué relación hay entre dos valores $v$ y $w$ de $a^b$?
\item Sea $b=n/m$ un número racional, cociente de dos números naturales $n$ y $m$, ¿cuántos valores distintos toma en este caso la potencia $a^b$? Si $v$ es uno de los valores de $a^b$, ¿cuánto vale $v^m$?
\item ¿Cuáles son los valores de $i^i$?
\end{enumerate}
\end{ejercicio}
\begin{solucion}
\begin{enumerate}[(a)]
\item[]
\item Entre dos ramas del $\log z$ distan un mútliplo entero de $2\pi i$, luego podemos suponer que $v=e^{b\log a}$ y $w = e^{b(\log a+2k\pi i)}$ para algún $k\in \Z$. Por tanto
$$
\frac{w}{v} = e^{b2k\pi i}
$$
\item A partir de las consideraciones anteriores, tenemos
$$
a^b=a^{n/m}=e^{\frac{n}{m}(\log a +2k\pi i)} = e^{\frac{n}{m}\log a}e^{2\pi i\frac{kn}{m}}
$$
El primer término del producto es fijo, ¿pero cuánto valores puede tomar el segundo? Exactamente $m$ valores. En cualquier caso,
$$
v^m= (a^b)^m = \left(e^{\frac{n}{m}\log a}e^{2\pi i\frac{kn}{m}}\right)^m = e^{n\log a}e^{2\pi ikn} = e^{n\log a} = a^n
$$
\item 
$$
i^i = e^{i \log i} = e^{i(\log|i|+i\arg i +2k\pi i)}=e^{i(\pi/2 i + 2k\pi i)}= e^{-\pi/2+2\pi k}
$$
\end{enumerate}
\end{solucion}
\newpage


\begin{ejercicio}{5}
\end{ejercicio}
\begin{solucion}
\end{solucion}
\newpage

\begin{ejercicio}{6}
Demostrar que la ecuación diferencial
$$
zu''(z)=u(z)$$
admite una única solución analítica en el entorno de $z=0$ tal que $u(0)=0$, $u'(0)=1$.
¿Cuál es la serie de potencias de $u(z)$ en el entorno del origen? ¿Para qué valores de $z$ se satisface la ecuación diferencial?
Demostrar que para todo $z\in \C$ 
$$
e^z = 1 +  \int_0^\infty u(zt)e^{-t}\frac{dt}{t}
$$
\begin{solucion}
\begin{itemize}
\item[]
\item Si existe tal función analítica sabemos por teoría que su serie de potencias habrá de verificar la ecuación diferencial, es decir,
$$
z\sum_{n=2}^\infty n(n-1) a_n z^{n-2}= \sum_{n=2}^\infty n(n-1 )a_{n} z^{n-1} =  \sum_{n=1}^\infty (n+1)na_{n+1} z^{n} = \sum_{n=0}^\infty a_n z^n
$$
Por la igualdad de los coeficientes, $a_{n+1}=\dfrac{a_n}{n(n+1)}$ para $n\geq 1$ con $a_0 = 0$ y $a_1=1$. Por tanto, podemos escribir para todo $n\geq 0$ $a_n = \dfrac{1}{\Gamma(n)\Gamma(n+1)}$. Calculemos el radio de convergencia. Podemos suponer que $n>>0$ y aplicar la fórmula de Stirling
\begin{gather*}
\limsup_n |a_n|^{1/n} = \lim_n \sqrt[n]{\dfrac{1}{\Gamma(n)\Gamma(n+1)}} = \lim_n {\dfrac{1}{\sqrt[n]{(n-1)!n!}}} = \lim_n \frac{e^2}{n^2} = 0
\end{gather*}
Esto es claro, pues $\sqrt[n]{n!}\sim n/e$. Luego $R=\infty$, es decir, la serie converge en todo $\C$. 

Notemos que hemos probado que existe una función analítica que verifica la ecuación diferencial (la serie definida por los coeficientes anteriormente calculados), sabemos que es única donde converja por el Teorema de Unicidad de Funciones Analíticas y de hecho converge en todo $\C$ (luego la ecuación diferencial se satisface para todo $z\in \C$). 
\item Pasemos a probar la última igualdad integral. Para ello intercambiaremos suma por integral, por lo que vamos a probar antes que estamos en las hipóteis del teorema de convergencia dominada. Sea $X=(0,\infty)$.
\begin{gather*}
\int_0^\infty u(zt)e^{-t}\frac{dt}{t} = \int_0^\infty \sum_{k=1}^\infty a_k (zt)^k e^{-t}\frac{dt}{t}\\
f(t)=\sum_{k=1}^\infty a_kz^kt^{k-1}e^{-t}  \qquad f_n(t) = \sum_{k=1}^n a_kz^kt^{k-1}e^{-t} \quad \forall z\in \C
\end{gather*}
Tenemos que $f_n \to f$, $f_n$ es medible por ser suma finita de funciones continuas en $X$. Finalmente escogemos la siguiente $g$. Ante la falta de información sobre nuestra función $u$ no puedo encontrar una $g$ que valga para toda $z$. Supongamos que $|z|<M$. Entonces $\exists n_0$ tal que $\forall k>n_0$ $2^{k-1} M^k<k!$, de donde
\begin{align*}
\sum_{k=1}^\infty \abs{a_kz^kt^{k-1}e^{-t}} &= \sum_{k=1}^\infty  \frac{|z|^kt^{k}e^{-t}}{k!(k-1)!t} < e^{-t}\sum_{k=1}^{n_0} \frac{(Mt)^k}{k!(k-1)!t} + e^{-t}\sum_{n=n_0+1}^\infty \frac{(Mt)^k}{k!(k-1)!t}\\
&\leq e^{-t}P(t) + e^{-t}\sum_{n=1}^\infty \frac{t^{k-1}}{(k-1)!2^{k-1}} = e^{-t}P(t)+e^{-t/2} =g(t)
\end{align*}
Claramente $\int_X g(t)dt < \infty$. Por tanto
\begin{align*}
\int_0^\infty u(zt)e^{-t}\frac{dt}{t} & = \int_0^\infty \sum_{k=1}^\infty a_k (zt)^k e^{-t}\frac{dt}{t}= \sum_{k=1}^\infty a_k z^k  \int_0^\infty t^{k-1}e^{-t}dt\\
&= \sum_{k=1}^\infty \frac{1}{\Gamma(k)\Gamma(k+1)} \Gamma(k) z^k =  \sum_{k=1}^\infty \frac{z^k}{k!} = e^z -1
\end{align*}
\end{itemize}
\end{solucion}
\end{ejercicio}
\end{document}