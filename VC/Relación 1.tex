\documentclass[twoside]{article}
\usepackage{../estilo-ejercicios}

\newcommand{\sumno}{\sum_{n=0}^{\infty}}
\newcommand{\sumn}{\sum_{n=1}^{\infty}}
\usepackage{enumerate}
%--------------------------------------------------------
\begin{document}

\title{Variable Compleja - Relación 1}
\author{Rafael González López\\
Pendientes: 1}
\maketitle

\begin{ejercicio}{1}
Interpretar geométricamente las relaciones siguientes:
  \begin{AutoMultiColItemize}
  \item[a)] $|z-1|+|z+i|=4$
  \item[b)] $|z-2|+|z+2|>3$
  \item[c)] $|z-a|=|z-b|$
  \item[d)] $|z^2-z|\leq 1$
  \item[e)] $\Re(z)+\Im(z)<1$
  \item[f)] $0<\Re(iz)<1$
  \item[g)] $\alpha < \arg(z) < \beta$
  \end{AutoMultiColItemize}
suponiendo en $g)$ que $-\pi<\alpha<\beta<\pi$. 
\end{ejercicio}
\begin{solucion}
Sea $z=x+yi$. 
\begin{enumerate}[a)]
\item Tomando cuadrados apropiadamente
\begin{gather*}
\sqrt{(x-1)^2+y^2}+\sqrt{x^2+(y+1)^2}=4\\
(x-1)^2+y^2 + x^2+(y+1)^2 + 2\sqrt{(x-1)^2+y^2}\sqrt{x^2+(y+1)^2} = 16\\
2\sqrt{(x-1)^2+y^2}\sqrt{x^2+(y+1)^2} = 16 - (2 x^2 - 2 x + 2 y^2 + 2 y + 2)\\
4((x-1)^2+y^2)(x^2+(y+1)^2) = (16 - (2 x^2 - 2 x + 2 y^2 + 2 y + 2))^2\\
15 x^2 + x (-2 y - 16) + 15 y^2 + 16 y = 48	
\end{gather*}
Luego estamos ante una elipse que tiene el origen en su interior.
\item Análogamente al primer apartado, tenemos el exterior de una elipse que no contine al origen.
\item Si $a=b$, entonces todo $z$ es solución. Si son distintos, entonces
$$
(x-a)^2+y^2 = (x-b)^2 +y^2  \Leftrightarrow x^2+a^2-2ax = x^2+b^2 -2bx 
$$
Luego tenemos la recta vertical $x = \dfrac{b^2-a^2}{2(b-a)}=\dfrac{b+a}{2}$.
\item Tenemos la ecuación $|z^2-z|\leq 1$, entonces
$$
|z||z-1|\leq 1 \Leftrightarrow |z||z-1| \leq 1 \Leftrightarrow (x^2+y^2)((x-1)^2+y^2) \leq 1
$$
\item Dado que $\Re(z)=x$ y $\Re(z)=y$, tenemos que $x+y<1$, es decir, el semiplano abierto limitado por la recta $y=1-x$ que contiene al origen.
\item $iz= ix-y$, luego $\Re(iz)= -y$. Por lo que $-1<y<0$ es la banda horizontal abierta comprendida entre $y=-1$ e $y=0$.
\item 
\end{enumerate}
\end{solucion}
\newpage
\begin{ejercicio}{2}
Sea $f$ una serie de potencias centrada en el origen. Probar que $f$ tiene un desarrollo en serie de potencias alrededor de cualquier punto en su disco de convergencia.
\end{ejercicio}
\begin{solucion}
Consideremos $f(z)=\sum_{n=0}^\infty a_n z^n$. Entonces
\begin{align*}
f(z)&=\sum_{n=0}^\infty a_n z^n \\
&= \sum_{n=0}^\infty a_n (z_0 + (z-z_0))^n\\
&= \sum_{n=0}^\infty a_n \sum_{k=0}^n\binom{n}{k}z_0^{n-k}(z-z_0)^k
\end{align*}
Es decir queremos que
$$\sum_{n=0}^\infty \sum_{k=0}^n | a_n \binom{n}{k} z_0^{n-k} (z-z_0)^k | =
\sum_{n=0}^\infty \sum_{k=0}^n |a_n| \binom{n}{k} |z_0|^{n-k} |z-z_0|^k < +\infty$$
La última suma es de términos positivos así que la podemos reordenar como queramos y si nos da algo finito ya está.  Bien proseguimos viendo que
$$\sum_{n=0}^\infty \sum_{k=0}^n |a_n| \binom{n}{k} |z_0|^{n-k} |z-z_0|^k |=
\sum_{n=0}^\infty  |a_n| (|z_0|+|z-z_0|)^n$$
Esto es finito si $|z_0|+|z-z_0| < R$ siendo $R$ el radio de covergencia de la serie original
Es decir, siempre que $|z-z_0|<R-|z_0|$ podremos reagrupar los términos de la serie como queramos.
\begin{align*}
f(z)&=\sum_{n=0}^\infty a_n \sum_{k=0}^n\binom{n}{k}z_0^{n-k}(z-z_0)^k\\
&=\sum_{n=0}^\infty \left(\sum_{k=n}^\infty a_k \binom{k}{n}z_0^{k-n}\right) (z-z_0)^n
\end{align*}
Por tanto, si $z_0$ está en el disco de convergencia podemos encontrar un desarrollo en serie de potencial centrado en $z_0$.
\end{solucion}
\newpage


\begin{ejercicio}{3}
Sea $g$ una holomorfa en un abierto $\Omega$ conexo. Probar que en cualquiera de los siguientes casos
\begin{enumerate}[(i)]
\item Si $\Re(g)$ es constante.
\item Si $\Im(g)$ es constante.
\item Si $|g|$ es constante.
\end{enumerate}
podemos concluir que $g$ es constante.
\end{ejercicio}
\begin{solucion}
Los tres resultados se desprenden inmediatamente a partir de las ecauciones de Cauchy y el hecho de que el abierto $\Omega$ sea conexo, pues si no lo fuese, podría haber componentes conexas donde la constante variase. Denotemos $\Re(g)=u$, $\Im(g)=v$.
\begin{enumerate}[(i)]
\item Si $u$ es constante, entonces las ECR nos dicen que
$$
0 = \frac{\partial u}{\partial x} = \frac{\partial v}{\partial y} \qquad 0 = \frac{\partial u}{\partial x} = - \frac{\partial u}{\partial y}$$
Por lo que 
$$
0 = \frac{\partial v}{\partial x}  =\frac{\partial v}{\partial y} 
$$ 
Por lo que $v$ es constante.
\item Este apartado es completamente análogo al anterior.
\item Una posibilidad sería la siguiente. Si $|g|$ es constante, entonces la imagen es una cierta circunferencia de $\C$. Utilizando el Teorema de la aplicación abierta (Tema 3), tenemos que la imagen de una región por una aplicación holomorfa es cerrada si y solo si es constante. Sin embargo, vamos a probarlo de manera análoga a lo que hemos estado haciendo. Si $|g|$ es constante, entonces $|g|^2 =u^2+v^2 = c$. Podemos suponer que $c>0$, pues en otro caso es trivial. Por tanto, derivando
$$
u\frac{\partial u}{\partial x} + v\frac{\partial v}{\partial x} = 0 \qquad u\frac{\partial u}{\partial y} + v\frac{\partial v}{\partial y} = 0
$$
Sustituyendo las ECR
$$
u\frac{\partial u}{\partial x} - v\frac{\partial u}{\partial y} = 0 \qquad u\frac{\partial u}{\partial y} + v\frac{\partial u}{\partial x} = 0
$$
Multiplicamos la primera ecuación por $u$, la segunda por $v$ y sumamos, obteniendo
$$(u^2+v^2)\frac{\partial u}{x} = 0 \Rightarrow \frac{\partial u}{x} = 0$$
Si multiplicamos la primera por $-v$ y la segunda por $u$, al sumar obtenemos
$$(u^2+v^2)\frac{\partial u}{y} = 0 \Rightarrow \frac{\partial u}{y} = 0 
$$
Por tanto, $u$ es constante y por el primer apartado tenemos que $v$ también lo es.
\end{enumerate}
\end{solucion}
\newpage


\begin{ejercicio}{4}
Para $a$ y $b$ complejo y $a\neq 0$ se define $a^b = e^{b\log a}$. Como $\log a$ toma infinitos valores, en general $a^b$ toma infinitos valores.
\begin{enumerate}[(a)]
\item ¿Qué relación hay entre dos valores $v$ y $w$ de $a^b$?
\item Sea $b=n/m$ un número racional, cociente de dos números naturales $n$ y $m$, ¿cuántos valores distintos toma en este caso la potencia $a^b$? Si $v$ es uno de los valores de $a^b$, ¿cuánto vale $v^m$?
\item ¿Cuáles son los valores de $i^i$?
\end{enumerate}
\end{ejercicio}
\begin{solucion}
\begin{enumerate}[(a)]
\item[]
\item Entre dos ramas del $\log z$ distan un mútliplo entero de $2\pi i$, luego podemos suponer que $v=e^{b\log a}$ y $w = e^{b(\log a+2k\pi i)}$ para algún $k\in \Z$. Por tanto
$$
\frac{w}{v} = e^{b2k\pi i}
$$
\item A partir de las consideraciones anteriores, tenemos
$$
a^b=a^{n/m}=e^{\frac{n}{m}(\log a +2k\pi i)} = e^{\frac{n}{m}\log a}e^{2\pi i\frac{kn}{m}}
$$
El primer término del producto es fijo, ¿pero cuánto valores puede tomar el segundo? Exactamente $m$ valores. En cualquier caso,
$$
v^m= (a^b)^m = \left(e^{\frac{n}{m}\log a}e^{2\pi i\frac{kn}{m}}\right)^m = e^{n\log a}e^{2\pi ikn} = e^{n\log a} = a^n
$$
\item 
$$
i^i = e^{i \log i} = e^{i(\log|i|+i\arg i +2k\pi i)}=e^{i(\pi/2 i + 2k\pi i)}= e^{-\pi/2+2\pi k}
$$
\end{enumerate}
\end{solucion}
\newpage


\begin{ejercicio}{5}
Hallar todas las soluciones de las ecuaciones:
\begin{align*}
&a)\; \sin z+  \cos z = 2 & &b)\;\sin z -\cos z = 3 & &c)\;\sin z -\cos z = i\\
&d)\;\cosh z -\sinh z = 1 & &e)\;\sinh z -\cosh z = 2i & &f)\;2\cosh z +\sinh z = i
\end{align*}
\end{ejercicio}
\begin{solucion}
\begin{enumerate}[a)]
\item[]
\item Utilizando la definción de $\sen$ y $\cos$ complejos
\begin{gather*}
\frac{e^{iz}-e^{-iz}}{2i}+\frac{e^{iz}-+e^{-iz}}{2} = 2 
\end{gather*}
Haciendo el cambio $w=e^{iz}\neq 0$ llegamos a
\begin{gather*}
\frac{1-i}{2}w^2 -2w +\frac{1+i}{2} = 0
\end{gather*}
Resolviendo, obtenemos $w_1= (\sqrt{2}+1)e^{\pi i/4}$ y $w_2 = (\sqrt{2}-1)e^{\pi i/4}$. Desaciendo el cambio, tenemos $w= e^{zi}$ y tomando logaritmo
$$
z = \begin{cases}
\pi/4 +2\pi k + \log(\sqrt{2}+1)i\\
\pi/4 +2\pi k + \log(\sqrt{2}-1)i
\end{cases}
$$
\item[c)] Primeramente, escribimos la definición
$$
\frac{e^{iz}-e^{-iz}}{2i}-\frac{e^{iz}-+e^{-iz}}{2} = i
$$
Realizamos el cambio $w=e^{iz}$.
$$
-\frac{1+i}{2}w^2-iw -\frac{1-i}{2}= 0
$$
Obtenemos las soluciones $w_1 = -(1+i)(\sqrt{3}+1)/2$ y $w_2=(1+i)(\sqrt{3}-1)/2$. Por tanto, deshaciendo el cambio y tomando logaritmo
$$
zi = \begin{cases}
\log(\sqrt{3}+1)-\log(2)/2-3\pi i /4 + 2k\pi i\\
\log(\sqrt{3}-1)-\log(2)/2 + \pi/4 + 2k\pi i
\end{cases}
$$
\item[f)] Primeramente, escribimos la definición
$$
2\frac{e^z+e^{-z}}{2}+\frac{e^z-e^{-z}}{2} = i
$$
Realizando el cambio $w=e^z$
$$
3w^2-2iw+1 = 0
$$
Luego $w_1=i$, $w_2-i/3$. Por tanto, deshacemos el cambio y obtenemos
$$
z = \begin{cases}
i \pi/2 + 2k\pi i\\
-\log 3 - i\pi/2 + 2k\pi i
\end{cases}
$$
\end{enumerate}
\end{solucion}
\newpage

\begin{ejercicio}{6}
Demostrar que la ecuación diferencial
$$
zu''(z)=u(z)$$
admite una única solución analítica en el entorno de $z=0$ tal que $u(0)=0$, $u'(0)=1$.
¿Cuál es la serie de potencias de $u(z)$ en el entorno del origen? ¿Para qué valores de $z$ se satisface la ecuación diferencial?
Demostrar que para todo $z\in \C$ 
$$
e^z = 1 +  \int_0^\infty u(zt)e^{-t}\frac{dt}{t}
$$
\begin{solucion}
\begin{itemize}
\item[]
\item Si existe tal función analítica sabemos por teoría que su serie de potencias habrá de verificar la ecuación diferencial, es decir,
$$
z\sum_{n=2}^\infty n(n-1) a_n z^{n-2}= \sum_{n=2}^\infty n(n-1 )a_{n} z^{n-1} =  \sum_{n=1}^\infty (n+1)na_{n+1} z^{n} = \sum_{n=0}^\infty a_n z^n
$$
Por la igualdad de los coeficientes, $a_{n+1}=\dfrac{a_n}{n(n+1)}$ para $n\geq 1$ con $a_0 = 0$ y $a_1=1$. Por tanto, podemos escribir para todo $n\geq 0$ $a_n = \dfrac{1}{\Gamma(n)\Gamma(n+1)}$. Calculemos el radio de convergencia. Podemos suponer que $n>>0$ y aplicar la fórmula de Stirling
\begin{gather*}
\limsup_n |a_n|^{1/n} = \lim_n \sqrt[n]{\dfrac{1}{\Gamma(n)\Gamma(n+1)}} = \lim_n {\dfrac{1}{\sqrt[n]{(n-1)!n!}}} = \lim_n \frac{e^2}{n^2} = 0
\end{gather*}
Esto es claro, pues $\sqrt[n]{n!}\sim n/e$. Luego $R=\infty$, es decir, la serie converge en todo $\C$. 

Notemos que hemos probado que existe una función analítica que verifica la ecuación diferencial (la serie definida por los coeficientes anteriormente calculados), sabemos que es única donde converja por el Teorema de Unicidad de Funciones Analíticas y de hecho converge en todo $\C$ (luego la ecuación diferencial se satisface para todo $z\in \C$). 
\item Pasemos a probar la última igualdad integral. Para ello intercambiaremos suma por integral, por lo que vamos a probar antes que estamos en las hipótesis del teorema de convergencia dominada. Sea $X=(0,\infty)$.
\begin{gather*}
\int_0^\infty u(zt)e^{-t}\frac{dt}{t} = \int_0^\infty \sum_{k=1}^\infty a_k (zt)^k e^{-t}\frac{dt}{t}\\
f(t)=\sum_{k=1}^\infty a_kz^kt^{k-1}e^{-t}  \qquad f_n(t) = \sum_{k=1}^n a_kz^kt^{k-1}e^{-t} \quad \forall z\in \C
\end{gather*}
Tenemos que $f_n \to f$, $f_n$ es medible por ser suma finita de funciones continuas en $X$. Finalmente escogemos la siguiente $g$. Ante la falta de información sobre nuestra función $u$ no puedo encontrar una $g$ que valga para toda $z$. Supongamos que $|z|<M$. Entonces $\exists n_0$ tal que $\forall k>n_0$ $2^{k-1} M^k<k!$, de donde
\begin{align*}
\sum_{k=1}^\infty \abs{a_kz^kt^{k-1}e^{-t}} &= \sum_{k=1}^\infty  \frac{|z|^kt^{k}e^{-t}}{k!(k-1)!t} < e^{-t}\sum_{k=1}^{n_0} \frac{(Mt)^k}{k!(k-1)!t} + e^{-t}\sum_{n=n_0+1}^\infty \frac{(Mt)^k}{k!(k-1)!t}\\
&\leq e^{-t}P(t) + e^{-t}\sum_{n=1}^\infty \frac{t^{k-1}}{(k-1)!2^{k-1}} = e^{-t}P(t)+e^{-t/2} =g(t)
\end{align*}
Claramente $\int_X g(t)dt < \infty$. Por tanto
\begin{align*}
\int_0^\infty u(zt)e^{-t}\frac{dt}{t} & = \int_0^\infty \sum_{k=1}^\infty a_k (zt)^k e^{-t}\frac{dt}{t}= \sum_{k=1}^\infty a_k z^k  \int_0^\infty t^{k-1}e^{-t}dt\\
&= \sum_{k=1}^\infty \frac{1}{\Gamma(k)\Gamma(k+1)} \Gamma(k) z^k =  \sum_{k=1}^\infty \frac{z^k}{k!} = e^z -1
\end{align*}
\end{itemize}
\begin{nota}
Una solución más elegante para el último apartado es la siguiente. Para probar que podemos intercambiar suma e integral, utilizando el teorema de convergencia dominada, utilizamos la siguiente función.
\begin{align*}
\abs{\int_0^\infty u(zt)e^{-t}\frac{dt}{t}} &= \abs{\int_0^\infty \sum_{k=1}^\infty a_k (zt)^k e^{-t}\frac{dt}{t}} \leq \int_0^\infty \abs{ \sum_{k=1}^\infty a_k (zt)^k} e^{-t}\frac{dt}{t} \\
&\leq \int_0^\infty  \sum_{k=1}^\infty \abs{a_k (zt)^k }e^{-t}\frac{dt}{t} \overset{a_k>0}{=} \int_0^\infty  \sum_{k=1}^\infty a_k \abs{zt}^k e^{-t}\frac{dt}{t} \\ 
&=\int_0^\infty  \sum_{k=1}^\infty a_k \abs{z}^k{t}^k e^{-t}\frac{dt}{t} \overset{\text{Beppo Levi}}{=} \sum_{k=1}^\infty a_k \abs{z}^k  \int_0^\infty t^{k-1}e^{-t}dt\\
&= \sum_{k=1}^\infty \frac{1}{\Gamma(k)\Gamma(k+1)} \Gamma(k) \abs{z}^k =  \sum_{k=1}^\infty \frac{z^k}{k!} = e^{\abs{z}} -1
\end{align*}
Por tanto, es claro que podemos tomar $g(t)=\abs{u(zt)}e^{-t}/t$.
\end{nota}
\end{solucion}
\end{ejercicio}

\newpage 
\begin{ejercicio}{7}
\begin{enumerate}[(a)]
\item[]
\item Dado $x\in \C$ encontrar el mayor abierto en el que la función
$$
z\mapsto \frac{ze^{xz}}{e^z-1}
$$
es analítica.
\item Se define $B_n(x)$ como los coeficientes que aparecen en el desarrollo en serie de potencias
$$
\frac{ze^{xz}}{e^z-1}= \sum_{n=0}^\infty B_n(x)\frac{z^n}{n!}
$$
Demostrar que $B_n(x)$ es un polinomio de grado $n$. Calcular los tres primeros.
\item Demostrar que 
$$
B_{k+1}(x+1)-B_{k+1}(x)=(k+1)x^k\qquad k\geq 0
$$
\item Deducir la fórmula de Bernoulli
$$
\sum_{n=0}^{N-1} n^{k-1} = \frac{B_k(N)-B_k(0)}{k}
$$
\end{enumerate}
\end{ejercicio}
\begin{solucion}
\begin{enumerate}[(a)]
\item[]
\item Es claro que allá donde no se anule el denominador, para cada $x$ tenemos una función analítica que intentaremos prolongar allí donde se anule si es posible. Primeramente, es claro que el denominador se anula en $2k\pi i$ para $k\in \Z$. Es claro que el Teorema de Riemann asegura que si podemos prolongar continuamente la función en algún punto, la función será holomorfa en ese punto. Veamos que la función se puede prolongar solo en $z=0$.
\begin{itemize}
\item Si $x=0$ entonces, para $z=0$
$$
\lim_{z\to 0}\frac{z}{e^z-1} = \lim_{z\to 0}\frac{1}{e^z} = 1
$$
Pero si $k\neq 0$ entonces
$$
\lim_{z\to 2\pi k i}\frac{z}{e^z-1} = \infty
$$
\item Si $x\neq 0$ entonces, para $z=0$
$$
\lim_{z\to 0}\frac{ze^{zx}}{e^z-1} = \lim_{z\to 0}\frac{e^{zx}+zxe^{zx}}{e^z} = 1
$$
Pero si $k\neq 0$ entonces
$$
\lim_{z\to 2k\pi i}\frac{ze^{zx}}{e^z-1} = \infty
$$
Pues en este caso $|(2k\pi i)e^{(2k\pi i)x}| = 2k\pi e^{-2k\pi \Im(x)}$ es finito y no nulo y el denominador tiende a $0$.
\end{itemize}
Por tanto el mayor abierto al que se puede extender nuestro conjunto de funciones de manera analítica es
$$
\Omega = \C\setminus\{2k\pi i \mid k\in \Z,\; k\neq 0\}
$$
\item Denotemos, para cada $x$ a nuestra función por $f_x(z)$. Entonces, como es analítica en $0$, tiene una serie de potencias asociada en $|z|<2\pi$. 
\[ \sum_{n=0}^\infty \frac{x^n}{n!}z^{n+1}= z e^{zx} =(e^z-1) \left(\sum_{n=0}^\infty B_n(x) \frac{z^n}{n!}\right)=\left( \sum_{n=1}^\infty \frac{z^n}{n!}\right)\left(\sum_{n=0}^\infty B_n(x) \frac{z^n}{n!}\right)\]
Si multiplicamos las dos series
\[ \sum_{n=0}^\infty \frac{x^n}{n!}z^{n+1}= \left( \sum_{n=1}^\infty \frac{z^n}{n!}\right)\left(\sum_{n=0}^\infty B_n(x) \frac{z^n}{n!}\right) = \sum_{n=1}^\infty\sum_{k=0}^{n}\frac{B_k(x)}{(n-k)!k!} z^n \]
Por tanto, si igualamos coeficientes obtenemos que $\forall n\geq 0$
$$
\frac{x^{n}}{n!} = \sum_{k=0}^{n} \frac{B_k(x)}{(n+1-k)!k!}
$$
Veamos por inducción que esto es un polinomio. Para $n=0$ tenemos $B_0(x) = 1$, que es claramente un polinomio de grado $0$. Supongamos que es cierto para $n-1$ y probémoslo para $n$. Tenemos que
$$
B_n(x) = x^n - n!\sum_{k=0}^{n-1} \frac{B_k(x)}{(n+1-k)!k!}
$$
Por hipótesis de inducción, $B_k(x)$ es un polinomio de grado $k$ para $k=0,\dotsc,n-1$, por lo que $B_n(x)$ es un polinomio de grado $n$.
\begin{itemize}
\item El caso $n=0$ lo hemos visto, $B_0(x)=1$.
\item Si $n=1$,
$$
B_1(x)=x-1!\frac{B_0}{2!0!}=x-\frac{1}{2}
$$
\item Si $n=2$,
\begin{align*}
B_2(x)&=x^2 - 2!\left(\frac{B_0(x)}{3!0!}+\frac{B_1(x)}{2!1!}\right)\\
&= x^2 -B_1(x) - \frac{B_0(x)}{3}\\
&=x^2 -x+\frac{1}{2}-\frac{1}{3}\\
&= x^2-x + \frac{1}{6}
\end{align*}
\end{itemize}
%(z^1/1! + z^2/2! + z^3/3! + z^4/4! + ...)(B0 + B1z^1 + B2z^2/2 + B3z^3/6 + ...)
%
%z1B0/1!0! + z2B1/1!1! + z3B2/1!2! + z4B3/1!3! + ...
%            z2B0/2!0! + z3B1/2!1! + z4B2/2!^2
%                      + z3B0/3!0!   
\item Tengamos en cuenta que como las series de potencias se pueden sumar término a término. Sabemos que este desarrollo es válido en un entorno de $0$
$$
\sum_{k=0}^\infty(B_k(x+1)-B_k(x))\frac{z^k}{k!}=\frac{ze^{(x+1)z}}{e^z-1}- \frac{ze^{xz}}{e^z-1} = \frac{ze^{xz}(e^z-1)}{e^z-1} = \sum_{k=0}^\infty \frac{x^{k}}{k!}z^{k+1}
$$
Por tanto deducimos
$$
\frac{B_{k+1}(x+1)-B_{k+1}(x)}{(k+1)!} = \frac{x^k}{k!}
$$
De donde se deduce el resultado.
\item Es claro que la fórmula tiene sentido para $k\geq 1$. Vamos a probar la fórmula por inducción. Para el caso $n=1$ tenemos
$$\sum_{n=0}^{1-1} n^{k-1} =0= 0^k(k+1)  = B_k(1)-B_k(0)
$$
Supongamos que es cierto para $n=N-1$ y veámoslo para $n=N$
\begin{align*}
\sum_{n=0}^{N-1} n^{k-1} &= (N-1)^{k-1} + \sum_{n=0}^{N-2} n^{k-1} \\
&=\frac{k(N-1)^{k-1}}{k}+ \frac{B_k(N-1)-B_k(0)}{k} \\
&=\frac{B_k(N)-B_k(N-1)}{k}  + \frac{B_k(N-1)-B_k(0)}{k}\\
&= \frac{B_k(N)-B_k(0)}{k}
\end{align*}
\end{enumerate}
\end{solucion}

\newpage 
\begin{ejercicio}{8}
Se definen los números $g_n$ por la recurrencia \begin{equation*}
g_0 = 0, \quad g_1 = 1, \quad g_n = -2ng_{n-1} + \sum_{k=0}^n \binom{n}{k}g_k g_{n-k}, \qquad n \geq 1.
\end{equation*}
Consideramos la serie de potencias \begin{equation*}
G(z) = \sumno \frac{g_n}{n!}z^n.
\end{equation*}
Traducir la recurrencia en una ecuación para la serie $G(z)$. Encontrar una fórmula explícita para $g_n$.
\end{ejercicio}
\begin{solucion}
Observemos primero que $G(z) = z + \sum_{n=2}^{\infty} \frac{g_n}{n!}z^n$. Aplicando ahora la fórmula de recurrencia para los coeficientes $g_n$ con $n \geq 2$ obtenemos \begin{equation*}
\begin{split}
G(z) & =  z+ \sum_{n=2}^\infty \frac{-2ng_{n-1}}{n!} z^n + \sum_{n=2}^\infty \sum_{k=0}^n \binom{n}{k}g_kg_{n-k} \frac{z^n}{n!} \\ & = z -2zG(z) + \sum_{n=2}^\infty \sum_{k=0}^n \frac{g_n g_{n-k}}{n!(n-k)!}z^n \\ & = z - 2zG(z) + \left(\sumn \frac{g_n}{n!}z^n \right) \left( \sum_{k=1}^\infty \frac{g_k}{k!}z^n \right) \\ & = z - 2zG(z) + G(z)^2.
\end{split}
\end{equation*}
Obtenemos por tanto despejando $G$ que \begin{equation*}
G(z)^2 - (1+2z)G(z) + z = 0.
\end{equation*}
Resolvemos la ecuación: \begin{equation*}
G(z) = \frac{1+2z \pm \sqrt{(1+2z)^2 - 4z}}{2} = \frac{1}{2} \left( 1 + 2z - \sqrt{4z^2+1} \right).
\end{equation*}
Hemos elegido la raíz negativa para que se cumpla $G(0) = g_0 = 0$. \\
Una vez que tenemos la expresión explícita de $G$, vamos a desarrollarla en serie de potencias centrada en 0 para obtener la expresión de $g_n$.

Tenemos que para cualquier $\mu \in \C$ \begin{equation*}
(1+z)^\mu = \sumno \binom{\mu}{n} z^n,
\end{equation*}
donde se define \begin{equation*}
\binom{\mu}{n} = \frac{\Gamma(\mu +1)}{n! \Gamma (\mu - n +1)}.
\end{equation*}
Así, en nuestro caso tenemos que \begin{equation*}
\sqrt{1 + 4z^2} = \sumno \binom{1/2}{n}2^{2n} z^{2n}
\end{equation*}
y por tanto \begin{equation*}
G(z) = z + \sumno \binom{1/2}{n} 2^{2n-1}z^{2n},
\end{equation*}
de donde deducimos por tanto que $g_{2n+1} = 0$ para todo $n \geq 1$ y \begin{equation*}
g_{2n} = n! \binom{1/2}{n} 2^{2n-1}, \quad \text{para todo} \  n \geq 1. 
\end{equation*}
\end{solucion}

\newpage 
\begin{ejercicio}{9}
Probar las siguientes desigualdades:
\begin{enumerate}[(a)]
\item Para todo $z\in \C$ se tiene
$$\abs{e^z-1}\leq e^{\abs{z}}-1 \leq \abs{z}e^{\abs{z}}
$$
\item Para todo $z\in \C$ con $\abs{z}\leq 1$ se tiene
$$
\abs{e^z-1}\leq 2|z|
$$
\end{enumerate}
\end{ejercicio}
\begin{solucion}
\begin{enumerate}[(a)]
\item[]
\item Probemos la primera desigualdad
\begin{align*}
\abs{e^z-1} &= \abs{\sum_{n=0}^\infty \frac{z^n}{n!}-1} = \abs{\sum_{n=1}^\infty \frac{z^n}{n!}} \leq \sum_{n=1}^\infty \frac{\abs{z}^n}{n!} =  \sum_{n=0}^\infty \frac{\abs{z}^n}{n!} -1 = e^{\abs{z}}-1
\end{align*}
Pasemos a la segunda
\begin{align*}
e^{\abs{z}}-1 &=  \sum_{n=0}^\infty \frac{\abs{z}^n}{n!} -1 =  \sum_{n=1}^\infty \frac{\abs{z}^n}{n!} \leq |z|\sum_{n=1}^\infty \frac{\abs{z}^{n-1}}{(n-1)!} = |z|e^{|z|}
\end{align*}
\item Utilizamos el desarrollo anterior
\begin{align*}
e^{\abs{z}}-1 &=  \sum_{n=0}^\infty \frac{\abs{z}^n}{n!} -1 =  \sum_{n=1}^\infty \frac{\abs{z}^n}{n!} = |z|\sum_{n=1}^\infty \frac{\abs{z}^{n-1}}{n!} \leq |z|\sum_{n=1}^\infty\frac{1}{n!}= |z|(e-1)\leq 2|z|
\end{align*}
\end{enumerate}
\end{solucion}
\end{document}