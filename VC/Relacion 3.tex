\documentclass[twoside]{article}
\usepackage{../estilo-ejercicios}

\usepackage{enumerate}
%--------------------------------------------------------
\begin{document}

\title{Variable Compleja - Relación 1}
\author{Rafael González López\\
1f,3,4}
\maketitle
\begin{ejercicio}{1}
Encontrar los polos y partes singulares de las siguientes funciones
\begin{gather*}
(a)\; \frac{1}{z^2+5z+6} \qquad (b)\; \frac{1}{(z^2-1)^2} \qquad (c)\; \frac{1}{\sin z} \qquad (d)\; \cot z\\
(e)\; \frac{1}{\sin^2 z}\qquad (f)\; \frac{1}{z^m(1-z)^n} \qquad\text{($m$ y $n$ enteros postivos)}
\end{gather*}
\end{ejercicio}
\begin{solucion}
\begin{enumerate}[(a)]
\item[]
\item Tenemos que $z^2+5z+6=(z+2)(z+3)$. Luego los polos son $-2,-3$, ambos simples. Si $w=$ entonces
$$
\frac{1}{z+2}=\frac{1}{1-(-(z+3))} = - \sum_{n=0}^\infty (-1)^n(z-3)^n
$$
Desarrollos válido en $|z-3|<1$ Por tanto
$$
\frac{1}{z^2+5z+6} = \frac{1}{z-3} + \sum_{n=0}^\infty (-1)^{n+1}(z-3)^n
$$
Para el otro polo, consideramos
$$
\frac{1}{z+3} = \frac{1}{1-(-(z+2))} = \sum_{n=0}^\infty (-1)^n(z-2)^n
$$
Desarrollo válido en $|z-3|<1$. Por tanto
$$
\frac{1}{z^2+5z+6} = \frac{1}{z-2} + \sum_{n=0}^\infty (-1)^{n+1}(z-2)^n
$$
\newpage
\item Tenemos que $(z^2-1)^2 = (z+1)^2(z-1)^2$. Consideramos el desarollo
$$
\sum_{k=0}^\infty (k+1)n^k = \frac{1}{(n-1)^2}
$$
Por tanto, si $z-1 = z+1 -2 = {2((z+1)/2-1)}$
$$\frac{1}{(z^2-1)^2}=  \frac{1}{(z+1)^2} \sum_{n=0}^\infty\frac{(n+1)}{2^{n+2}}(z+1)^n = \frac{1/4}{(z+1)^{-2}}+\frac{2/8}{(z+1)} + \sum_{n=0}^\infty \frac{(n+3)}{2^{n+4}}(z+1)^{n} 
$$
Para la el otro polo procedemos análogamente, usando que $z+1=2+(z-1) = 2(1-(-(z-1)/2))$. Por tanto
\begin{align*}
\frac{1}{(z^2-1)^2}&=  \frac{1}{(z-1)^2} \sum_{n=0}^\infty\frac{(-1)^n(n+1)}{2^{n+2}}(z-1)^n\\
&= \frac{1/4}{(z+1)^{-2}}-\frac{2/8}{(z+1)} + \sum_{n=0}^\infty \frac{(-1)^{n+2}(n+3)}{2^{n+4}}(z+1)^{n}
\end{align*}
\item Sabemos que nuestra función tiene polos simples en $k\pi$ para $k\in \Z$ y además son los únicos.
Tengamos en cuenta que
$$\frac{\sin z}{z-k\pi} = \frac{1}{z-k\pi}\sum_{k=0}^\infty \frac{(z-k\pi)^{2n+1}(-1)^{2n}}{(2n+1)!} = \sum_{k=0}^\infty \frac{(z-k\pi)^{2n+1}(-1)^{2n}}{(2n+1)!} $$
Si denotamos por $a_k$ los coeficientes de la inversa función inversa (que tiene algún desarollo en $\pi k$ por no anularse nuestra función) y $b_k$ los propios, tenemos la relación
$$
c_n = \sum_{k=0}^{n}a_k b_{n-k}
$$
Calcular el término general de la serie sería latoso, pero como los polos son simples, basta calcular el primero.  $1 = c_0 = a_0 \cdot 1$. Por tanto las partes singulares son 
$$
\frac{1}{z-k\pi}
$$
\item Tenemos polos simples en $k\pi$. Desarrollando la serie del $\cos(z-k\pi)$ y usando la fórmula del apartado anterior tenemos que
$$
\cos z = \sum_{k=0}^\infty \frac{z^{2n}}{(2n)!}
$$
Por tanto el primer coeficiente de $(z-k\pi) \cos z / \sin z$ es $1 \cdot 1 = 1$. Por tanto la parte singulares simplemente
$$\frac{1}{z-k\pi}
$$
\item Análogamente a (c), tenemos polos de segundo orden en $k\pi$. Por la relación obtenida en ese apartado, podemos calcular los dos primeros términos utilizando la fórmula del término del producto. Denotemos por $r_n$ los coeficientes de 
$$\frac{\sin^2 z}{(z-k\pi)^2}$$
Tenemos que $0 = a_1 + b_1 = a_1 $, luego $b_1= 0$. Por tanto $r_0 = a_0^2 = 1$ y $r_1= 2 a_0a_1 = 0$. Es decir, la parte singular es
$$
\frac{1}{(z-k\pi)^2}
$$
\item Es claro que tenemos un polo de orden $m$ en $z=0$ y un polo de orden $n$ en $z=1$. Como $n,m$ son enteros positivos 
\end{enumerate}
\end{solucion}
\newpage

\begin{ejercicio}{2}
Dos cuestiones no relacionadas
\begin{itemize}
\item ¿Cuántos ceros tiene el polinomio $z^6-2z^5 +7z^4+z^3-z+1$ en el disco unidad?
\item ¿Cuántos ceros tiene el polinomio $z^4-6z+3$ en la corona $1<|z|<2$.
\end{itemize}
\end{ejercicio}
\begin{solucion}
\begin{itemize}
\item[]
\item Aplicamos el Teorema de Rouche. Tomamos $f(z)=7z^4$ y $g(z)=z^6-2z^5+z^3-z+1$. Entonces en $|z|=1$
$$
|f(z)| = 7 > 6 = |z^6| + |-2z^5| + |z^3| + |-z| +1 \geq |g(z)|
$$
Por tanto, hay 4 raíces en el disco unidad.
\item  Tengamos en cuenta que si tomamos $f(z)=-6z$ y $g(z)=z^4+3$, entonces en $|z|=1$ por el Teorema de Rouché
$$
|f(z)1 = 6 > 1+3 = |z|^4+ 3 \geq |z^4+3|
$$
Por tanto, nuestro polinomio tiene un cero en $|z|<1$. Si tomamos ahora $f(z)=z^4$ y $g(z)=-6z+3$ entonces en $|z|=2$
$$
|f(z)| = 2^4 = 16 \geq 15 = 12 + 3 = |-6z| + 3 \geq |-6z+3|
$$
Por tanto, en $|z|=2$ y $|z|=1$ no hay raíces. En $|z|<2$ hay 4 y en $|z|<1$ hay 1. Es decir, en $1<|z|<2$ hay 3 raíces.
\end{itemize}
\end{solucion}
\newpage

\begin{ejercicio}{3}
Probar que una singularidad aislada de $f(z)$ no puede ser un polo de $e^{f(z)}$ 
\end{ejercicio}
\begin{solucion}
Sea $a\in \C$ es una singularidad aislada de $f(z)$. Distingamos casos
\begin{itemize}
\item Si $a$ es una singularidad evitable, entonces en un entorno de $a$, $|f(z)|<M$ para algún $M>0$. Por tanto $|e^{f(z)}| \leq |e^M|$ en un entorno de $a$. Por el Teorema de Riemann, esta singularidad también es evitable.

\end{itemize} entonces 
\end{solucion}
\newpage

\begin{ejercicio}{4}
Probar que si una función analítica $f(z)$ tiene un cero de orden $N$ en $z=a$, entonces $f(z)=g(z)^N$ para alguna función $g(z)$ analítica en un entorno de $a$ y que satisface $g'(a)\neq 0$.
\end{ejercicio}
\begin{solucion}

\end{solucion}
\newpage

\begin{ejercicio}{5}
Demostrar que el polinomio $p(z)=2x^6-6z^5+4z^4-z^2+z-1$ tiene sus ceros en el disco cerrado de radio $3$.
\begin{solucion}
Vamos a aplicar el Teorema de Rouché tal y como se enuncia en el libro Complex Analysis (página 91), que establece lo siguiente.
\begin{theorem}
Sean $f$ y $g$ dos funciones holomorfas en un abierto que contiene a la circunferencia $C$ y su interior. Si 
$$|g(z)|<|f(z)| \qquad \forall z \in C
$$
Entonces $f$ y $f+g$ tienen el mismo número de ceros en el  círculo $C$ y su interior. 
\end{theorem}
Consideremos $f(z)=2z^6-6z^5+4z^4$, $g(z)=p(z)-f(z)$ y $C$ la circunferencia de centro el origen y radio $R=3$. Dado que $f(z)=2z^4(z-1)(z-2)$, $f$ tiene sus 6 raíces en el interior de $C$. Tenemos además que $\forall z \in C$ 
\begin{align*}
|g(z)|&=|-z^2+z-1|\leq |z^2|+|z|+1 = 3^2+3+1=13\\
|f(z)|&=|2z^6-6z^5+4z^4|=|2z^4||z^2-3z+2|=162|z-1||z-2|\\
&\geq 162||z|-1|||z|-2| = 324 > 13 \geq |g(z)|
\end{align*}
Basta aplicar el teorema para tener el resultado.
\end{solucion}
\end{ejercicio}

\newpage
\begin{ejercicio}{6}
Calcular las integrales
$$
\int_{|z|=2}\frac{dz}{z^3(z^{10}-2)} \qquad \int_{|z|=2}\frac{dz}{(z-3)(z^5-1)}
$$
\end{ejercicio}
\begin{solucion}
En ambos casos aplicaremos el Teorema de los residuos.
\begin{enumerate}
\item Si $w = 1/z$ entonces $1/w = z$, luego $-dw/w^2 = dz$. Por tanto 
$$
\int_{|z|=2}\frac{dz}{z^3(z^{10}-2)}= \int_{|w|=1/2} \frac{dw}{w^2 (1/w)^3((1/w)^{10}-2)} = \int_{|w|=1/2}\frac{w^{11}}{1+2w^{10}}
$$
Observemos que las raíces verifica que $2w^{10} = -1$, luego $|w| = \frac{1}{\sqrt[10]{2}} > \frac{1}{2}$. Por tanto, todas las discontinuidades están en el exterior de $|w|=1/2$, por lo que la integral es $0$.
\item Si $w = 1/z$ entonces $1/w = z$, luego $-dw/w^2 = dz$. Por tanto 
$$
\int_{|z|=2}\frac{dz}{(z-3)(z^5-1)}= \int_{|w|=1/2} \frac{dw}{w^2 (1/w-3)((1/w)^{5}-1)} = \int_{|w|=1/2}\frac{w^{4}}{(1-3w)(1-w^5)}
$$
Por tanto, solo tenemos un polo simple en $w=1/3$. Calculamos el residuo
$$
\lim_{w\to 1/3} (w-1/3)\frac{w^4}{(1-3w)(1-w^5)} = -\frac{1}{242}
$$
Por tanto, la integral vale
$$-2\pi i \frac{1}{242} = -\frac{\pi i}{121}$$
\end{enumerate}
\end{solucion}
\newpage

\begin{ejercicio}{7}
\begin{enumerate}[(a)]
\item Demostrar que todos los ceros de $z^4-8z+10$ están en el anillo $1<|z|<3$
\item Calcular la integral $$\int_C \frac{z^3dz}{z^4-8z+10}$$
siendo $C$ la circunferencia de centro $0$ y radio $3$.
\end{enumerate}
\end{ejercicio}
\begin{solucion}
\begin{enumerate}[(a)]
\item Procedemos análogamente al segundo apartado del Ejercicio 2. Utilizamos el Teorema de Rouché. Tomamos $f(z)=z^4$, $g(z)=-8z+10$. Entonces, en $|z|=3$,
$$
|f(z)|=|z^4| = 81 > 34 = 24 + 10 = |-8z|+10 \geq |-8z+10| = |g(z)|
$$
Por tanto, nuestro polinomio tiene cuatro raíces en $|z|<3$. Si tomamos $f(z)=10$ y $g(z)=z^4-8z$ entonces, en $|z|=1$,
$$
|f(z)| = 10 > 9 = 1+8 = |z^4|+|-8z| \geq |z^4-8z| =|g(z)|
$$
Como $f(z)=10$ no se anula en el disco unidad, deducimos que nuestro polinomios no tiene raíces en $|z|\leq 1$. Por tanto, las cuatro raíces están en el anillo. 
\item Con el cambio $z=1/w$
$$
\int_C \frac{z^3dz}{z^4-8z+10} = \int_{|w|=1/3} \frac{dw}{w(10w^4-8w^3+1)}
$$
Observemos que $dz = -dw/w^2$. ¿Dónde ha ido ese signo menos? Pues pensemos en la parametrización de $C$ $z=3e^{it}$, $t=[0,2\pi]$. Entonces $w=e^{-it}/3$. Por tanto, se recorrería en sentido contrario. Tengamos en cuenta que 
$$
|1-8w^3| \geq 1-|8w^3| = \frac{19}{27} > \frac{10}{81} = |10w^4|
$$
Dado que si $1-8w^3=0$ entonces $|w|=1/2$, $h(w)=10w^4-8w^3+1$ no tiene raíces en $|z|\leq 1/3$. En nuestra nueva integral solo tenemos un polo simple en $w=0$. Calculamos su residuo
$$
\lim_{w\to 0}wh(w) = 1
$$
Por tanto, la integral suma $2\pi i$.
\end{enumerate}
\end{solucion}
\newpage

\begin{ejercicio}{8}

\end{ejercicio}
\begin{solucion}

\end{solucion}
\newpage


\begin{ejercicio}{9}
Probar que si una función analítica $f(z)$ tiene un cero de orden $N$ en $z=a$, entonces $f(z)=g(z)^N$ para alguna función $g(z)$ analítica en un entorno de $a$ y que satisface $g'(a)\neq 0$.
\end{ejercicio}
\begin{solucion}
Debe haber habido algún extraño problema por el cuál este enunciado se corresponde con el del Ejercicio 4.
\end{solucion}
\newpage


\begin{ejercicio}{10}

\end{ejercicio}
\begin{solucion}

\end{solucion}
\newpage

\end{document}