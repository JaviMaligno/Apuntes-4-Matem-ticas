\documentclass[twoside]{article}
\usepackage{../estilo-ejercicios}


\usepackage{enumerate}
%--------------------------------------------------------
\begin{document}

\title{Variable Compleja - Relación 6}
\author{Rafael González López}
\maketitle


\begin{ejercicio}{1}
Contesta breve y razonadamente a las siguientes cuestiones:
\begin{enumerate}[a)]
\item Sea $f\colon\Omega\to \C$ holomorfa. Quiero encontrar una primitiva $F$ de $f$, ¿qué condición en $\Omega$ garantiza la existencia de $F$?
\item Pon un ejemplo de función entera  de orden $\rho$ no entero. 
\item Si desarrollamos $\frac{1}{\cos z}$ entorno al punto $z=2+i$, ¿cuál es el radio de convergencia?
\item ¿Puedes dar algún ejemplo de una función holomorfa $f\colon\Omega\to\C$ y un compacto 
$K\subset\Omega$ de manera que $f$ no se pueda aproximar uniformemente en $K$ por polinomios?
\item ¿Qué aplicación tiene el teorema de Riemann para el problema de Dirichlet?
\end{enumerate}
\end{ejercicio}
\begin{solucion}
\begin{enumerate}[a)]
\item[]
\item La condición más general que hemos estudiado es que $\Omega$ sea simplemente conexo.
\item La función $\cos \sqrt{z}$.
\item Basta encontrar la distancia entre $2+i$ y la discontinuidad más próxima de la función a dicho punto. Un análisis heurístico revela que esta es $z=\pi/2$, luego $R=\sqrt{1 + (2 - \pi/2)^2}$.
\item Tomamos $f(z)=1/z$ definida en $\Omega = \C\setminus\{0\}$ y el compacto $K=\{z\mid 1\leq |z|\leq 2\}$.
\item 
\end{enumerate}
\end{solucion}

\newpage
\begin{ejercicio}{2}
Se define la sucesión $(F_n)$ como los coeficientes del desarrollo de potencias 
\[(1-z-z^2)^{-1}=\sum_{n=0}^\infty F_n z^n.\]
\begin{enumerate}[a)]
\item Demostrar que existen y son únicos los números $F_n$ definidos de ese modo.
\item Calcula $F_n$ para $0\le n\le 6$. 
\item ¿Puedes encontrar una recurrencia para $F_n$?
\end{enumerate}
\end{ejercicio}
\begin{solucion}
\begin{enumerate}[a)]
\item[]
\item Notemos que las raíces del polinomio son $A= -(1+\sqrt{5})/2$ y $B=(\sqrt{5}-1)/2$. Sabemos que
\begin{align*}
\frac{1}{1-z-z^2} &= \frac{1/\sqrt{5}}{z-A}-\frac{1/\sqrt{5}}{z-B}\\
&= \frac{1/(\sqrt{5}B)}{1-z/B} - \frac{1/(\sqrt{5}A)}{1-z/A}\\
& = 
\frac{1}{\sqrt{5}}\sum_{n=0}^\infty z^n\left(\frac{1}{B^{n+1}}-\frac{1}{A^{n+1}}\right)
\end{align*}
La existencia y unicidad es consecuencia de que $f$ sea holomorfa en un entorno del origen.
\item $F_0 = 0$. $F_1 = 1$, $F_2 = 1$, $F_3 = 2$, $F_4=3$, $F_5=5$ y $F_6 = 8$.
\item Claramente sigue la sucesión de Fibonacci, descrita por $a_0 =0$, $a_1=1$ y $a_n = a_{n-1}+a_{n-2}$ para $n\geq 2$.
\end{enumerate}
\end{solucion}

\newpage
\begin{ejercicio}{3}
Consideremos la integral 
$\displaystyle{F(z)=\int_0^{+\infty}\frac{\cos t}{(z+t)^2}\,dt}$.
\begin{enumerate}[a)]
\item Sea $\varepsilon>0$ un número real fijado. Demostrar que $F$  está bien definida y es analítica en la región $\Re z>\varepsilon$.
\item Sea $\varepsilon>0$ y $R>0$ dos números reales fijados. Probar que $F(z)$ está bien definida y es analítica en la región $\Omega^+_{R,\varepsilon}=\{z\in\C: |\Re z|\le R, \Im z>\varepsilon\}.$
\item Demostrar también que $F$ está bien definida y es holomorfa en la región\newline 
$\Omega^-_{R,\varepsilon}=\{z\in\C: |\Re z|\le R, \Im z<-\varepsilon\}.$
\item ¿Dónde podemos decir que $F$ es holomorfa?
\end{enumerate}
\end{ejercicio}
\begin{solucion}
\begin{enumerate}[a)]
\item[]
\end{enumerate}
\end{solucion}

\newpage
\begin{ejercicio}{4}
Consideremos la función $f(z)=e^{2\pi z^2}-1$.
\begin{enumerate}[a)]
\item Determinar los ceros de $f$ y la multiplicidad de cada uno de ellos. 
\item ¿Cuál es el orden de $f$?
\item Escribir el producto de Hadamard para $f(z)$.  Simplificar lo mas posible el resultado. 
\end{enumerate}
\end{ejercicio}
\begin{solucion}
\begin{enumerate}[a)]
\item[]
\item Tomando logaritmo en $e^{2\pi z^2} = 1$ tenemos directamente que
$$
2\pi z^2 = 2\pi i k \qquad z^2 = ik \qquad z = \pm \frac{1+i}{2}\sqrt{k} 
$$
Denotemos por $a_k$ los ceros de $f$. La multiplicidad de cualquiera de ellos es $1$, pues
$$
\lim_{z\to a_k} \frac{e^{2\pi z^2}-1}{z-a_k} = \lim_{z\to a_k}4\pi z {e^{2\pi z^2}} \neq 0
$$
\item Trivialmente 2.
\item Lo de simplificar igual está complicado, pero vamos a ver qué sale
\begin{align*}
f(z) & = e^{2\pi z^2}-1\\
&= e^{Az^2+Bz+C}z\prod_{n=1}^\infty E_2(z/a_n)E_2(z/a_{-n})\\
&= e^{Az^2+Bz+C}z\prod_{n=1}^\infty (1-z/a_n)(1+z/a_n)e^{z^2/a_n^2}\\
&=e^{Az^2+Bz+C}z\prod_{n=1}^\infty (1-z^2/a_n^2)e^{z^2/a_n^2}\\
&=e^{Az^2+Bz+C}z\prod_{n=1}^\infty (1+z^2i/n)e^{z^2i/n}\\
\end{align*}
\end{enumerate}
\end{solucion}

\newpage


\end{document}