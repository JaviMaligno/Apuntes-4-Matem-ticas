\documentclass[HS.tex]{subfiles}
\begin{document}
\chapter{Homología simplicial}
\section{Álgebra homológica}
Nuestros objetos serán $\R$-espacio vectoriales de dimensión finita (o $R$-módulos con $R$ dominio de ideales principales).

\begin{defi}
Dado un diagrama de $\R$-e.v. y homomorfismos:
\[ M_{i-1} \xrightarrow{f_{i-1}} M_i \xrightarrow{f_i} M_{i+1} \]
se dice que la sucesión es \emph{exacta} en $M_i$ si $\ker f_i = \Ima f_{i-1}$. Si
\[ \cdots \rightarrow M_{i-2} \xrightarrow{f_{i-2}} \rightarrow M_{i-1} \xrightarrow{f_{i-1}} M_i \xrightarrow{f_i} M_{i+1} \xrightarrow{f_{i+1}} M_{i+2} \rightarrow \cdots \]
es exacto para todo $i \in \Z$, entonces diremos que es una \emph{sucesión exacta larga}.
\end{defi}

\begin{lemma}
Se verifica
\begin{enumerate}[a)]
\item $0 \rightarrow M_1 \xrightarrow{f_1} M_2$ es exacta sii $f_1$ es inyectiva.
\item $M_2 \xrightarrow{f_2} M_3 \rightarrow 0$ es exacto sii $f_2$ es sobreyectivo.
\item $0 \rightarrow M_1 \xrightarrow{f_1} M_2 \xrightarrow{f_2} M_3 \rightarrow 0$ es exacto sii $f_1$ es inyectiva, $f_2$ es sobreyectiva y $f_2$ induce un isomorfismo entre $\coker f_1$ y $M_3$. Este tipo de sucesiones es conocido como \emph{sucesión exacta corta}.
\end{enumerate}
\end{lemma}
\begin{dem}
\begin{enumerate}[a)]
\item $0 \rightarrow M_1 \xrightarrow{f_1} M_2$ es exacta sii $\ker f_1 = \{0\}$ sii $f_1$ es inyectiva.
\item $M_2 \xrightarrow{f_2} M_3 \rightarrow 0$ es exacto sii $\Ima f_2 = M_3$ sii $f_2$ es sobreyectivo.
\item $0 \rightarrow M_1 \xrightarrow{f_1} M_2 \xrightarrow{f_2} M_3 \rightarrow 0$ es exacto sii $f_1$ es inyectiva (apartado (a)), $f_2$ es sobreyectiva (apartado (b)) y $\ker f_2 = \Ima f_1$, es decir $\coker f_1 = M_2/\Ima f_1 = M_2/\ker f_2 \cong \Ima f_2 = M_3$ (por el primer teorema de isomorfía).
\end{enumerate}
\end{dem}

\begin{prop}
Si $0 \rightarrow V_1 \xrightarrow{f} V_2 \xrightarrow{g} V_3 \rightarrow 0$ es una sucesión exacta corta de $\R$-e.v. de dimensión finita, entonces $V_2 \cong V_1 \oplus V_3$.
\end{prop}
\begin{dem}
Sea $\{b_1,\dots,b_r\}$ una base de $V_1$ ($\dim V_1 = r$). Entonces $\{f(b_1), \dots, f(b_r)\}$ es base del subespacio vectorial $f(V_1) \subseteq V_2$. Extendiendo a una base $\{f(b_1), \dots, f(b_r),w_{r+1},\dots,w_n\}$ de $V_2$ ($\dim V_2 = n$). Comprobemos que $\{g(w_{r+1}),\dots,g(w_n)\}=:\mathcal{B}$ es una base de $V_3$.
\begin{itemize}
\item $\mathcal{B}$ es sistema generador: Sea $v \in V_3$, entonces existe $a \in V_2$ con $g(a)=v$ pues $g$ es sobreyectiva. Podemos expresar $a=\sum a_i f(b_i) + \sum \overline{a_j} w_j$. Entonces $v=g(a)= \sum \overline{a_j}g(w_j)$.
\item $\mathcal{B}$ es linealemnte independiente: Supongamos que existe $\sum λ_j g(w_j) = 0$. Entonces $\sum λ_j w_j \in \ker g = \Ima f$, pero $w_j \notin \Ima f$ $\forall j$, luego $λ_j = 0$ $\forall j$.
\end{itemize}
\end{dem}

Véase que hemos probado con toda sucesión exacta corta de espacios vectoriales \emph{escinde}. Usaremos este lema (demuéstrese como ejercicio):
\begin{lemma}[Splitting Lemma]
Se dice que $0 \rightarrow V_1 \xrightarrow{f} V_2 \xrightarrow{g} V_3 \rightarrow 0$ \emph{escinde} si se cumple uno de las siguientes condiciones equivalentes:
\begin{enumerate}
\item $V_2 \cong V_1 \oplus V_3$.
\item $\exists r : V_3 \to V_2$ tal que $g \circ r = id_{V_3}$.
\item $\exists t : V_2 \to V_1$ tal que $t \circ f = id_{V_1}$. 
\end{enumerate}
\end{lemma}

\begin{lemma}[Lema de los cinco]
Sea el diagrama conmutativo:
\[\begin{tikzcd}
	M_1 \arrow[r,"θ_1"] \arrow[d,two heads,"ψ_1"] & M_2 \arrow[r,"θ_2"] \arrow[d,"ψ_2","\cong"'] & M_3 \arrow[r,"θ_3"] \arrow[d,"ψ_3"] & M_4 \arrow[r,"θ_4"] \arrow[d,"ψ_4","\cong"'] & M_5 \arrow[d,hook,"ψ_5"]\\
	N_1 \arrow[r,"φ_1"] & N_2 \arrow[r,"φ_2"] & N_3 \arrow[r,"φ_3"] & N_4 \arrow[r,"φ_4"] & N_5
\end{tikzcd}\]
donde las filas son exactas, $ψ_2$ y $ψ_4$ son isomorfismos, $ψ_1$ es epimorfismos y $ψ_5$ es monomorfismo. Entonces $ψ_3$ es isomorfismo.
\end{lemma}
\begin{dem}
Lo demostramos por diagram-chasing\footnote{Es más fácil hacerlo que explicarlo}.

Sea $m_3 \in M_3$ tal que $ψ_3(m_3) = 0$, luego $0=φ_3ψ_3(m)=ψ_4θ_3(m_3)$.
Como $ψ_4$ es isomorfismo, $θ_3(m_3)=0$.
Por exactitud, existe $m_2 \in M_2$ tal que $θ_2(m_2)=m_3$. Sea $n_2 = ψ_2(m_2)$.
Como $0=ψ_3θ_2(n)=φ_2ψ_2(n)$, por exactitud existe $n_1 \in N_1$ tal que $φ_1(n_1) = n_2$.
Como $ψ_1$ es sobreyectivo, existe $m_1 \in M_1$ tal que $ψ_1(m_1)=n_1$. 
Ahora bien, $ψ_2(θ_1(m_1))=φ_1(ψ_1(m_1))=ψ_2(m_2)$.
Como $ψ_2$ es inyectiva, esto quiere decir que $m_2 = θ_1(m_1)$
Luego $m_3 = θ_2(m_2)=θ_2(θ_1(m_1))=0$ por exactitud y $ψ_3$ es inyectiva.
\end{dem}

\begin{defi}
Un complejo de cadenas $\mathcal{C}$ es un diagrama del tipo siguiente:
\[ \cdots \rightarrow C_{n+1} \xrightarrow{δ_{n+1}} C_n \xrightarrow{δ_n} C_{n-1} \xrightarrow{δ_{n-1}} \cdots \]
donde $C_i$ son $\R$-e.v. llamados \emph{$n$-cadenas}, $δ_i$ son homomorfismos de $\R$-e.v. llamados \emph{operadores borde}, tal que $δ_n \circ δ_{n+1} = 0$ para todo $n \in \Z$. Habitualmente, denotamos $\mathcal{C}_* = \{(C_n,δ_n)\}_{n \in \Z}$.

Para todo $n \in \Z_+$, denotamos $Z_n := \ker δ_n \subseteq C_n$. A los elementos de $Z_n$ los llamamos \emph{$n$-ciclos}.
Definimos $B_n := \Ima δ_{n+1} \subseteq C_n$, cuyos elementos se llaman \emph{$n$-bordes}.
Como $δ_n \circ δ_{n+1} = 0$, tenemos que $B_n \subseteq Z_n$.
De esta manera, definimos $H_n(\mathcal{C}) = Z_n / B_n$ como el \emph{$n$-ésimo $\R$-e.v. de homología} del complejo de cadenas $\mathcal{C}$.
\end{defi}

\begin{defi}
Sean $\mathcal{C}=\{(C_n,δ_n)\}$ y $\mathcal{C}'=\{(C_n',δ_n')\}$ dos complejos de cadenas, se define $\mathcal{C}\oplus\mathcal{C}'$ como el el siguiente complejo: para cada $n \in \Z$, $(\mathcal{C} \oplus \mathcal{C}')_n = C_n \oplus C_n'$ y $d : C_n \oplus C_n' \to C_{n-1} \oplus C_{n-1}'$ definido como $d(c,c')=(δ_n(c),δ_n'(c'))$.
\end{defi}

\begin{defi}
Sean $\mathcal{C}_1=\{(C_n^1,δ_n^1)\}$ y $\mathcal{C}_2=\{(C_n^2,δ_n^2)\}$ dos complejos de cadenas. Un homomorfismo de complejos de cadenas $f : \mathcal{C}_1 \to \mathcal{C}_2$ consiste en una familia $f = \{f_n : C_n^1 \to C_n^2\}_{n \in \Z}$ con $f_n$ homomorfismo de $\R$-e.v. tal que para todo $n \in \Z$, $δ_n^2 \circ f_n = f_{n-1} \circ δ_{n-1}^1$, es decir, el siguiente diagrama conmute:
\[\begin{tikzcd}
	C_n^1 \arrow[r,"f_n"] \arrow[d,"δ_n^1"] & C_n^2 \arrow[d,"δ_n^2"]\\
	C_{n-1}^1 \arrow[r,"f_{n-1}"] & C_{n-1}^2
\end{tikzcd}\]
\end{defi}

\begin{prop}
Si $f : \mathcal{C}^1 \to \mathcal{C}$ un homomorfismo de complejo de cadenas, entonces $f$ induce para cada $n \in \Z$ un homomorfismo:
\[ f_* : H_n(\mathcal{C}_1) \to H_n(\mathcal{C}_2) \]
\[ f_* ([z]) = [f_n(z)] \]
\end{prop}
\end{document}