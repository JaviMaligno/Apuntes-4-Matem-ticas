\documentclass[HS.tex]{subfiles}
\begin{document}

\chapter{Homología simplicial}

\section{Complejos de cadenas simpliciales}

\begin{defi}
Sea $\sigma=(v_0,\dots, v_n)$, se llama \emph{orientación} de $\sigma$ a cada una de las dos clases de equivalencia que se obtienen al permutar vértices dependiendo de su paridad. A cada una la denotaremos $\sigma$ y $\overline{\sigma}$ (también $-\sigma$).
\end{defi}

\begin{defi}
$C_n(K;\F)$ es el espacio vectorial de las $n$-cadenas  de $K$, es decir, todas las combinaciones lineales finitas en coeficientes en $\F$ de $n$-símplices de $K$. Formalmente
\[
C_n(K,\F)=\frac{\F[\sigma,\overline{\sigma}]}{\gene{\sigma+\overline{\sigma}}}\cong \F[\sigma]
\]
es decir, consideramos iguales los símplices con distinta orientación.
\end{defi}

\end{document}