\documentclass[HS.tex]{subfiles}

\begin{document}


%\hyphenation{equi-va-len-cia}\hyphenation{pro-pie-dad}\hyphenation{res-pec-ti-va-men-te}\hyphenation{sub-es-pa-cio}

\chapter{Complejos simpliciales}

\section{Definiciones básicas}

\begin{defi}
Dados los puntos $a_0,\dots, a_n\in\R^n$ afínmente independientes, se define el \emph{$n$-símplice} generado por ellos 
$$\sigma=(a_0,\dots, a_n):=\{x\in\R^n\mid x=\sum_{i=0}^n\lambda_ia_i, \sum_{i=0}^n\lambda_i=1,\lambda_i\geq 0\}.$$
Al número $n$ se le llama \emph{dimensión} del símplice.
\end{defi}

\begin{ej}\
\begin{enumerate}
\item $n=0$: punto, 0-símplice o vértice $(a_0)$.
\item $n=1$: segmento, 1-símplice o arista $(a_0a_1)$.
\item $n=2$: triángulo o 2-símplice   $(a_0a_1a_2)$.
\item $n=3$: tetraedro o 3-símplice $(a_0a_1a_2a_3)$.
\end{enumerate}
\end{ej}

\begin{nota}
Todo $n$-símplice es convexo, pues es la envolvente convexa de los puntos que lo generan. Además es compacto de $\R^n$.
\end{nota}

\begin{defi}
Dado un $n$-símplice $\sigma=(a_0,\dots,a_n)$, se llama \emph{cara} de $\sigma$, y se denota $\tau\leq \sigma$, a cualquier $k$-símplice ($k\leq n$) generado por $k+1$ vértices de $\sigma$. En el caso de que $k<n$ se dirá que $\tau$ es \emph{cara propia} de $\sigma$ y se denotará por $\tau<\sigma$.
\end{defi}

\begin{defi}
Dado el $n$-símplice $\sigma=(a_0,\dots, a_n)$, se define el interior de $\sigma$, denotado por $\mathring{\sigma}$, como
$$\mathring{\sigma}=\{x\in\sigma\mid \lambda_i>0\}.$$
\end{defi}

\begin{ej}\
\begin{enumerate}
\item Para $n=0$, $\sigma=(a_0)$, $\mathring{\sigma}=\sigma$.
\item Para $n=1$, $\sigma=(a_0a_1)$, $\mathring{\sigma}=\sigma-\{a_0,a_1\}$. 
\item Para $n=2$, $\sigma=(a_0a_1a_2)$, $\mathring{\sigma}=\sigma-\{(a_0a_1),(a_1,a_2),(a_0a_2)\}$.
\item Para $n=3$, $\sigma=(a_0a_1a_2a_3)$, $\mathring{\sigma}=\sigma-\{(a_0a_1a_2),(a_0a_1a_3),(a_0a_2a_3),(a_1,a_2a_3)\}$.
\end{enumerate}
\end{ej}

\begin{defi}
Dado un $n$-símplice generado por $\sigma=(a_0,\dots, a_n)$, se define el \emph{borde} de $\sigma$, denotado por $\partial\sigma$, como 
$$\partial\sigma=\bigcup_{\tau<\sigma}\tau.$$
\end{defi}

\begin{prop}
Dado un $n$-símplice $\sigma=(a_0,\dots, a_n)$, se verifica que:
\begin{enumerate}
\item $\sigma=\mathring{\sigma}\sqcup\partial\sigma$.
\item $\mathring{\sigma}=\sigma-\partial\sigma$.
\item $\partial\sigma=\sigma-\mathring{\sigma}$.
\end{enumerate}
\end{prop}

\begin{defi}
Dado $\sigma$ un $n$-símplice, se llama \emph{cara abierta} de $\sigma$ al interior de cualquiera de sus caras.
\end{defi}
\begin{prop}
Todo símplice es unión disjunta de sus caras abiertas.
\end{prop}
\begin{dem}
Sea $x\in\sigma=(a_0,\dots,a_n)\Rightarrow x=\sum_{i=0}^n\lambda_ia_i$ con $\lambda_i\geq 0$ y $\sum_{i=0}^n\lambda_i=1$. Consideramos la cara de $\sigma$ generada por aquellos vértices con $\lambda_i>0$, a cuyo conjunto de índices denotaremos $I=\{i_0,\dots, i_k\}$. Entonce $x=\sum_{i\in I}\lambda_ia_i$, luego $(a_{i_0},\dots,a_{i_k})\leq\sigma$ y contiene a $x$ en su interior. 

Dado que la expresión de $x$ respecto a un sistema de coordenadas es única, esto implica que $(a_{i_0},\dots,a_{i_k})$ es la única cara cuyo interior contiene a $x$.
$\QED$
\end{dem}

\begin{defi}
Un complejo simplicial $K$ en $\R^n$ es una familia de símplices, tales que:
\begin{enumerate}
\item Si $\sigma\in K$ y $\tau\leq\sigma$, $\tau\in K$.
\item Si $\sigma,\sigma'\in K$ tales que $\sigma\cap\sigma'\neq\emptyset$, entonces $\sigma\cap\sigma'\in K$.
\end{enumerate}
\end{defi}

\begin{ej}
\begin{enumerate}
\item Todo $n$-símplice tiene de modo natural estructura de complejo simplicial.
\item Cualquier grafo es un complejo simplicial formado por vértices y aristas.
\item $K=\{(a_0),(a_1),(a_2),(a_0a_1),(a_0a_2),(a_2a_1)\}$.
\item Las triangulaciones de superficies son complejos simpliciales. 
\item Dos aristas que se corten en el interior no lo serían.
\item Un triángulo en el que no estamos incluyendo un vértice dentro de $K$, no lo sería.
\end{enumerate}
\end{ej}\

\begin{defi}
Dado un complejo simpilicial $K$, se define la \emph{realización geométrica} o \emph{poliedro subyacente} de $K$ como el espacio topológico $(|K|,\Tau_e|_{|K|})$ donde $|K|=\underset{\sigma\in K}{\bigcup}\sigma$.
\end{defi}

\begin{defi}
Dado un complejo simplicial $K$ y dado $L\subset K$, si $L$ tiene estructura de complejo simplicial, entonces se dice que $L$ es subcomplejo de $K$.
\end{defi}
\begin{ej}\
\begin{enumerate}
\item Dado un símplice $\sigma$, al considerar todas sus caras propias tenemos un subcomplejo cuya realización es $\partial\sigma$.
\item \underline{El \emph{$p$-esqueleto} de un complejo simplicial $K$}. Se define $\dim(K)=\max\{\dim(\sigma)\mid\sigma\in K\}$. Si $\dim(K)=n$, el $p$-esqueleto de $K$ es el subcomplejo de $K$ dado por
$$K^{(p)}=\{\sigma\in K\mid \dim(\sigma)\leq p\}$$
para $0\leq p\leq n$. Los $p$-esqueletos forman una \emph{filtración}\footnote{\url{https://en.wikipedia.org/wiki/Filtration_(mathematics)}} de $K$. Se deja como ejercicio probar que el $p$-esqueleto es un subcomplejo para todo $0\leq p\leq n$.
\end{enumerate}
\end{ej}

\begin{prop}
Sea $K$ un complejo simplicial finito y $|K|$ su realización geométrica. Entonces $|K|$ es un compacto.
\end{prop}
\begin{dem}
Como todo símplice es compacto y $|K|$ es unión finita de todos sus símplices, entonces $|K|$ es compacto. \QED
\end{dem}

\begin{lemma}
Sea $K$ un complejo simplicial finito, se verifica que $|K|$ es unión disjunta de los interiores de todos sus símplices, es decir, $|K|=\underset{\sigma\in K}\bigsqcup\mathring{\sigma}$.
\end{lemma}
\begin{dem}
$\boxed{\subseteq}$ Dado $x\in |K|$, es claro que existe $\sigma\in K$ tal que $x\in\sigma=(v_0,\dots, v_n)$, pues $|K|=\bigcup_{\sigma\in K}|\sigma|$.  Entonces $x=\sum\lambda_i v_i$. Nos quedamos con los $v_i$ tales que $\lambda_i>0$. De este modo, el símplice $\tau$ generado por dichos vértices es el único de $K$ que contiene a $x$ en su interior. A este símplice lo llamamos \emph{símplice soporte} de $x$. Entonces $x\in\mathring{\tau}\leq \sigma\in K$. Por lo que $x\in \underset{\sigma\in K}\bigsqcup\mathring{\sigma}$.

$\boxed{\supseteq}$ Sea $x\in \underset{\sigma\in K}\bigsqcup\mathring{\sigma}$, entonces existe un único $\sigma\in K$ tal que $x\in\mathring{\sigma}$, luego $x\in |K|$.
\end{dem} 

\begin{defi} Sea $K$ un complejo simplicial y sea $x\in |K|$. Se define la \emph{estrella} de $x$ en $K$ como 
\[
st(x;K)=\{\sigma\in K\mid \exists \tau (\sigma\leq \tau\land x\in\tau)\}.
\]
Esto es, la estrella de $x$ en $K$ la forman todos los símplices de $K$ que contienen a $x$ y todas sus caras.
\end{defi}

\begin{nota}
$st(x;K)$ es un subcomplejo de $K$. Efectivamente, si $\sigma\in st(x;K)$ entonces $\exists \tau$ tal que $\sigma\leq\tau$ y $x\in\tau$. Si $\eta\leq\sigma$, entonces $\eta\leq\tau$, por lo que el mismo $\tau$ para que $\eta\in st(x;K)$. Por otra parte, si $\sigma_1,\sigma_2\in st(x;K)$ y $\sigma_1\cap\sigma_2\neq\emptyset$, entonces $\sigma_1\cap\sigma_2\in K$. Por otra parte, existen $\tau_1,\tau_2\in K$ tales que $x\in\tau_1\cap\tau_2$, y como $\tau_1\cap\tau_2\in K$ y $\sigma_1\cap\sigma_2\leq \tau_1\cap\tau_2$, se tiene que $\sigma_1\cap\sigma_2\in st(x;K)$.
\end{nota}

\begin{defi}
Dado $K$ complejo simplicial y $x\in |K|$, se define el \emph{link} de $x$ en $K$ como
\[
lk(x;K)=\{\sigma\in st(x;K)\mid x\notin\sigma\}.
\]
\end{defi}

\begin{nota}
Análogamente, se tiene que $lk(x;K)$ es subcomplejo tanto de $K$ como de $st(x;K)$. En efecto, si $\sigma\in lk(x;K)$ y $\tau\leq\sigma$ entonces $\tau\in st(x;K)$ y $x\notin \tau$. Además, si $\sigma,\tau\in lk(x;K)$ con $\sigma\cap\tau\neq\emptyset$, entonces $\sigma\cap\tau\in st(x;K)$ y $x\notin\sigma\cap\tau$. 
\end{nota}

\begin{defi}
Sea $K$ un complejo simplicial y sea $\sigma\in K$, se define la \emph{estrella} de $\sigma$ en $K$
\[
st(\sigma;K)=\{\tau\in K\mid \exists\eta\in K (\tau\leq\eta\land \sigma\leq\eta)\}.
\]

Esto es, $st(\sigma;K)$ la forman los símplices que contienen a $\sigma$ como cara y además las caras de dichos símplices.
\end{defi}

\begin{nota}
Se comprueba que cuando $\sigma$ es un vértice, la definición coincide con la estrella del vértice. Además, $st(\sigma;K)$ es subcomplejo de $K$. 
\end{nota}

\begin{defi}
Dado $K$ complejo simplicial y $\sigma\in K$, se define el \emph{link} de $\sigma$ en $K$ como
\[
lk(\sigma;K)=\{\tau\in st(\sigma;k)\mid \sigma\not\leq \tau\}.
\]
También se puede definir equivalentemente como 
\[
lk(\sigma;K)=\{\tau\in st(\sigma;k)\mid \sigma\cap \tau=\emptyset\}.
\]
Se tiene que $lk(\sigma;K)$ es subcomplejo de $K$ y de $st(\sigma;K)$.
\end{defi}

\begin{defi}
Sea $K$ un complejo simplicial y sea $x\in|K|$, se define la \emph{estrella abierta} de $x$ en $|K|$ como
\[
\mathring{st}(x;K)=\bigcup_{x\in\sigma}\mathring{\sigma}
\]
\end{defi}

\begin{prop}
Sea $K$ complejo simplicial y $x\in|K|$, se tiene que
\[
|\mathring{st}(x;K)|=|st(x;K)|-|lk(x;K)|.
\]
\end{prop}
\begin{dem}
$\boxed{\subseteq}$ Sea $y\in |\mathring{st}(x;K)|$. Entonces existe $\sigma\in |K|$ tal que $y\in|\mathring{\sigma}|$, en particular $y\in\sigma$. Como $\sigma\leq\sigma$ y $y\in\sigma$, $y\in |st(x;K)|$. Además, por definición $|lk(x;K)|$ no tiene ningún punto de $|\mathring{st}(x;K)|$, luego $y\notin|\mathring{st}(x;K)|$.

$\boxed{\supseteq}$ Sea $y\in |st(x;K)|-|lk(x;K)|$. Por definición de la estrella, $\exists\eta\in |K|$ tal que $x\in\eta$ y $y\in\eta$. Además, como $y$ no está en el link, $\exists \tau\in |K|$ tal que $y\in\tau$ con $\tau\in |st(x;K)|$ y $x\in\tau$. Consideramos entonces $y\in\tau\cap\eta\in |st(x;K)|$ Como $\tau\cap\eta$ es unión disjunta de sus caras abiertas, entonces, o bien $\tau\cap\eta$ contiene a $y$ en su interior o bien lo contiene alguna de sus caras propias en su interior. En cualquier caso, $y\in |\mathring{st}(x;K)|$.  \QED
\end{dem}

\end{document}