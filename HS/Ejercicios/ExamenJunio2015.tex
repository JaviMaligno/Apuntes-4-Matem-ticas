\documentclass[twoside]{article}
\usepackage{../../estilo-ejercicios}
\newcommand{\colapso}{{\searrow\!\!\!\!\searrow}}
%--------------------------------------------------------
\begin{document}

\title{Examen de Homología Simplicial (Junio de 2016)}
\author{Javier Aguilar Martín}
\maketitle

\begin{ejercicio}{1}
Dado un complejo simplicial $K$ y $v\in K^0$, se define el borrado de $v$ en $K$ como :
\[
delt(v;K)=\{\sigma\in K\mid v\notin\sigma\}.
\]
Se pide:
\begin{enumerate}
\item Probar que $del(v;K)$ es un subcomplejo de $K$ tal que $lk(v;K)\subseteq del(v;K)$.
\item Probar que $K=st(v;K)\cup del(v;K)$ y $lk(v;K)=st(v;K)\cap del(v;K)$.
\item Probar que si existe $v\in K^0$ tal que $lk(v;K)$ es colapsable, entonces $K$ colapsa a $del(v;K)$.
\end{enumerate}
\end{ejercicio}
\begin{solucion}\
\begin{enumerate}
\item Si $v\notin\sigma$ se tiene también $x\notin\tau$ para cualquier $\tau\leq\sigma$. Si $v\notin\sigma$ y $v\notin\mu$ claramente $v\notin \sigma\cap\tau$. Con esto tenemos probado que $del(v;K)$ es un subcomplejo de $K$. Que $lk(v;K)\subseteq del(v;K)$ se tiene trivialmente de la definición del link, ya que son símplices que no contienen a $v$.

\item La primera igualdad es evidente pues si $v\in \sigma$ entonces $\sigma\in st(v;K)$ y si $v\notin\sigma$ entonces $\sigma\in del(v;K)$. La otra igualdad es también sencilla pues si $\sigma\in st(v;K)\cap del(v;K)$, entonces $v\notin \sigma$ pero $\sigma\in st(v;K)$, y este tipo de símplices son los que forman $lk(v;K)$. 

\item Sketch of the proof:
\begin{itemize}
\item Probar que para que el link sea colapsable el vértice debe estar en el borde
\item Colapsar la estrella al link
\item Hecho porque el link está en el del
\end{itemize}
\end{enumerate}
\end{solucion}

\newpage

\begin{ejercicio}{2}
Sean $A=\{(x,y,z)\in\R^3\mid x^2+y^2+z^2=1\}$, $B=\{(x,y,z)\in\R^3\mid x^2+y^2+z^2\leq 4, z=0\}$ y $C=\{(x,y,z)\in\R^3\mid x=0,y=0, 0\leq z\leq 1\}$. Dar triangulaciones de $A$, $B$ y $C$ y calcular los $\F$-espacios vectoriales de homología reducida de $A\cup B$, $A\cup C$ y $A\cup B\cup C$ usando dichas triangulaciones e indicando explícitamente sus generadores.
\end{ejercicio}
\begin{solucion}
Como hay que hacer el cálculo de homologías con las triangulaciones que demos, no serán tan naturales como si simplemente quisiéramos triangular los espacios por separado. Para la esfera $A$ usamos la triangulación como octaedro.  Para $C$ conservamos la triangulación usual de un segmento como complejo generado por un 1-símplice. En cuando a $B$, hacemos la siguiente triangulación:

(La dibujaré)

Pasamos a calcular las homologías:
\begin{itemize}
\item $A\cup B$ es colapsable a la esfera con una membrana en su interior, así que dividimos el espacio en la membrana $K_1$ y la esfera $K_2$, cuyas homologías son conocidas, y cuya intersección es una circunferencia, que también tiene homología conocida. La circunferencia es el borde de la membrana y el borde de cualquiera de las semiesferas, por lo que tenemos el mapeo nulo $H_1(K_1\cap K_2)\to H_1(K_1)\oplus H_1(K_2)=0$. Tenemos directamente $H_1(K_1)=0$ por estar entre ceros en Mayer-Vietoris y $\widetilde{H}_0(K)=0$ por ser conexo. En cuanto a $H_2(K)$, tenemos en Mayer-Vietoris
\[
0\to\F\xrightarrow{j_{*2}}H_2(K)\xrightarrow{\Delta_{*2}}\F\to 0
\]
Por escindibilidad, $H_2(K)=\F\oplus\F$, donde un generador es la memebrana con el hemisferio norte y el otro con el hemisferio sur.

\item $A\cup C$ colapsa sobre $A$, cuya homología reducida es $\widetilde{H}_p(A)=0$ si $p\neq 2$, $\widetilde{H}_2(A)=\F$ generado por la suma de todos sus 2-símplices.

\item Colapsamos $A\cup B\cup C$ de la misma forma que en $A\cup B$. Descomponemos este complejo $K$ en $K_1=A\cup C$ y $K_2=B$, donde estamos abusando de notación llamando $B$ a lo que queda tras colapsar. La intersección  sería justamente $D=S^1\sqcup\{v\}$, donde el punto es un extremo de $C$. Sabemos que $\widetilde{H}_0(D)=\F\gene{v-v_i}$ para cualquier $v_i$ fijo de la triangulación de $S^1$ y $H_p(D)=H_p(S^1)$ para todo $p\neq 0$. Es claro que $K_1$ colapsa sobre $A$, así que ya podemos sustituir en Mayer-Vietoris. Tenemos entonces
\[
0\to \F\to H_2(K)\to \F\to 0
\]
con lo que $H_2(K)=\F\oplus\F$ con los mismos generadores que tenía $A\cup B$. 
%Para obtener $\widetilde{H}_0(K)$ usaremos la homología no reducida. Tenemos
\[
0\to H_1(K)\to\F\to\F\oplus\F\to 0
\]
Como la última no trivial es sobreyectiva, deducimos que $H_1(K)=\F$. Podemos tomar como generador un 1-ciclo que contenga a $C$. 
\end{itemize}
\end{solucion}




\end{document}
