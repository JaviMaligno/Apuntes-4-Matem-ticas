\documentclass[HS.tex]{subfiles}
\begin{document}

\chapter{Invariancia homotópica de la homología simplicial}

\section{Generalidades sobre Álgebra Homológica}
Sea $X$ un espacio topológico y supongamos que existe un complejo simplicial $K$ y un homeomorfismo $h \colon X \to |K|$.
Entonces podríamos definir $H_n(X) := H_n(K)$.
El par $(K,h)$ se llama \emph{trianguación} de $X$.
Esto estará bien definido si $H_n$ no depende de la triangulación.
Esta cuestión será abordada por la \emph{invariancia homotópica de la homología}.

\section{Nociones básicas de teoría de homotopía}
\begin{defi}\label{homotopia}
Dos aplicaciones continuas $f,g \colon X \to Y$ entre espacios topológicos se dicen homotópicas si existe una aplicación $H \colon X \times I \to Y$ continua (que llamamos \emph{homotopía}) cumpliendo:
\[ H(x,0) = f(x) \quad \forall x \in X, \]
\[ H(x,1) = g(x) \quad \forall x \in X. \]
Dos aplicaciones $f,g$ \emph{homotópicas} por $H$ se denotan $f \simeq_H g$. 
\end{defi}

\begin{defi}
Si en la definición anterior $f(x) = g(x)$ para todo $x \in A \subseteq X$ y además
\[ H(x,t) = f(x) = g(x),\quad \forall x \in A \forall t \in I, \]
entonces diremos que $f \simeq_H g$ \emph{relativa} a $A$.
Esto es que $A$ es un conjunto de puntos que la homotopía ``no mueve''.
\end{defi}

\begin{prop}
La relación ``ser homotópica (relativa) a'' entre aplicacioens continuas es una relación de equivalencia.
\end{prop}
\begin{dem}
En primer lugar, $f\simeq_H f$ con $H(x,t)=f(x)$ para todo $x$. Si $f\simeq_G g$, entonces $g\simeq_G f$ con $G(x,t)=H(x,1-t)$. Si $f\simeq_F g$ y $g\simeq_G h$, entonces $f\simeq_H h$, donde $H(x,t)=F(x,2t)$ para  $x\in [0,1/2]$ y $H(x,t)=G(x,2t-1)$ para $x\in [1/2,1]$.\QED
\end{dem}

\begin{defi}
Dos espacios topológicos $X$ e $Y$ son del mismo \emph{tipo de homotopía} (u \emph{homotópicamente equivalente}) si existe $f \colon X \to Y$, existe $g \colon Y \to X$ continua cumpliendo:
\[ f \circ g \simeq Id_Y \quad g \circ f \simeq Id_X. \]
En este caso diremos que tanto $f$ como $g$ son \emph{equivalencias de homotopía} e \emph{inversas homotópicas} una de la otra.
\end{defi}

\begin{defi}
Si un espacio topológico $X$ es del mismo tipo de homotopía que un punto, entonces diremos que $X$ es \index{contráctil}\emph{contráctil}.
\end{defi}
\begin{ej}
Todo subespacio convexo $C \subseteq \R^n$ es contráctil.
\end{ej}
\begin{prop}
$X$ es contráctil si y sólo si $Id_X \simeq cte$.
\end{prop}
\begin{dem}
Si $X$ es contráctil, entonces existen $f:X\to \{x_0\}$ y $g:\{x_0\}\to X$ tales que $g\circ f\simeq Id_X$. Necesitariamente $g$ es una aplicación constante, por lo que $g\circ f$ es constante, de donde se deduce el resultado.

Si $Id_X \simeq cte$, entonces sea $x_0$ la constante a la que es homotópica. Sea entonces la inclusión $i:\{x_0\}\hookrightarrow X$. Se tiene claramente que la aplicación constante $x_0$ es inversa homotópica de la inclusión, pues $x_0\circ i=Id_{\{x_0\}}$ y $i\circ x_0\simeq Id_X$ por hipótesis. \QED
\end{dem}

\begin{defi}
Dados $A \subseteq X$ e.t., una \emph{retracción} de $X$ en $A$ es una aplicación continua:
\[ r \colon X \to A \]
cumpliendo que $r(a) = a$ $\forall a \in A$.
Si $i \colon A \to X$ denota la inclusión, entonces $r \circ i = Id_A$.
Si además, se cumple que $i \circ r \simeq id_X$, diremos que $A$ es un \emph{retracto de deformación} de $X$.
Si además, $i \circ r \simeq_H id_X$ relativa a $A$, entonces diremos que $A$ es un \emph{retracto de deformación fuerte} de $X$.
\end{defi}

\begin{nota}
Si $A \subseteq X$ es retrato de deformación de $X$, entonces $A$ y $X$ son del mismo tipo de homotopía, pues tanto $r$ como $i$ son equivalentes de homotopía, y una es la inversa homotópica de la otra.
\end{nota}
\begin{ej}
Damos un ejemplo de retracto que no es de deformación.
Consideremos $X = S^1 \times I$ el cilindro. Consideramos $A = \{z\} \times I$, con $z \in S^1$.
La retracción $r \colon X \to A$ definida por $r(s,t) = (z,t) \in A$ no es retracto de deformación. Si lo fuera, entonces el cilindro tendría el mismo tipo de homotopía que $A$, pero es fácil comprobar que $A$ es contráctil y que el cilindro retrae con deformacións sobre $S^1$, que es bien sabido que no es contráctil por tener grupo fundamental no trivial.
\end{ej}

\section{Complejo de cadenas simpliciales}
\begin{defi}
Sea $K$ un  complejo simplicial, y $\sigma \in K$ un $n$-símplice (sin pérdida de generalidad) $\sigma = (v_0,v_1,\dots,v_n)$.
Definimos la relación de equivalencia $(v_0,v_1,\dots,v_n) \sim (v_{\pi(0)},v_{\pi(1)},\dots,v_{\pi(n)})$ si $\pi$ es una permutación par en $\{0,1,\dots,n\}$.
\end{defi}
Es decir, la relación de equivalencia inducida por la acción del grupo de las permutaciones pares sobre los $n$-símplices.
Es sencillo ver que esto realmente es una relación de equivalencia.

Mediante esta relación de equivalencia, aparecen dos clases de equivalencia (si $n > 0$, ya que para $n=0$ solo hay una), que llamamos \index{orientación}\emph{orientaciones} sobre $\sigma$, que denotaremos por $\sigma$ y $\overline{\sigma}$.

Cuidado, aquí usamos el mismo símbolo para $\sigma$ el $n$-símplice y $\sigma$ al $n$-símplice orientado.
Escribiremos $(v_0,v_1,\dots,v_n)$ para referirnos a un $n$-símplice y $[v_0,v_1,\dots,v_n]$ para referirnos a un $n$-símplice orientado.

Dado un complejo simplicial $K$, procedemos a definir un complejo de cadenas $\{(C_n(K;\mathbb{F}), \partial_n)\}_{n\geq 0}$. 
\begin{defi}
Definimos $C_n(K;\mathbb{F})$ como el $\mathbb{F}$-espacio vectorial generado por todos los $n$-símplices orientados de $K$ cocientado por el subespacio vectorial $\gene{\sigma + \overline{\sigma} \mid \sigma\ n\text{-símplice de }K}$.
\end{defi}
Por ejemplo, $C_0(K;\mathbb{F})$ es el $\mathbb{F}$-espacio vectorial generado por los vértices de $K$.

\begin{defi}
Definimos los \index{borde}\emph{operador borde} $\partial_n \colon C_n(K; \mathbb{F}) \to C_{n-1}(K; \mathbb{F})$ como un operador lineal generado por:
\[ \partial_n[v_0,v_1,\dots,v_n] := \sum_{i=1}^n (-1)^i [v_0,v_1,\dots,\widehat{v_i},\dots,v_n]\]
\end{defi}

\begin{ej}
\[ \partial_2[v_0,v_1,v_2] = [v_1,v_2] - [v_0,v_2] + [v_0,v_1] = [v_0,v_1] + [v_1,v_2] + [v_2,v_0] \]
\end{ej}

\begin{nota}
Obsérvese que $\partial_0 \colon C_0(K; \mathbb{F}) \to 0$.

En el caso $n=1$, quitaremos los corchetes, es decir $\partial_1[v_0,v_1] = [v_1]-[v_0] = v_1 - v_0$.
\end{nota}

\begin{lemma}
El operador borde $\partial_n \colon C_n(K; \mathbb{F}) \to C_{n-1}(K; \mathbb{F})$ está bien definido.
\end{lemma}
\begin{proof}
Tenemos que ver que si tomamos otro representante de $[v_0,\dots,v_n]$, tenemos la misma imagen.
Basta probar realmente que $\partial([v_0,v_1,\dots,v_n]) = -\partial_n([v_1,v_0,v_2\dots,v_n])$.
\[ \partial([v_0,v_1,\dots,v_n]) = [v_1,v_2,\dots,v_n]-[v_0,v_2,\dots,v_n] + \sum_{i\neq0,1}(-1)^i[v_0,v_1,\dots,\widehat{v_i},\dots,v_n]\]
\[ \partial([v_1,v_0,\dots,v_n]) = [v_0,v_2,\dots,v_n]-[v_1,v_2,\dots,v_n] + \sum_{i\neq0,1}(-1)^i[v_1,v_0,\dots,\widehat{v_i},\dots,v_n]\]
Los dos primeros términos de la primera ecuación son opuestos a los de la segunda. Además, cada término del sumatorio aparece en la otra ecuación con una trasposición en los vértices.
Luego $\partial([v_0,v_1,\dots,v_n]) = -\partial([v_1,v_0,\dots,v_n])$.
\end{proof}

\begin{lemma}
Se tiene que $\partial_n \circ \partial_{n+1} = 0$ para todo $n$.
\end{lemma}
\begin{proof}
\begin{align*}
\partial_n(\partial_{n+1}([v_0,\dots,v_{n+1}])) & = \partial_n \left(\sum_{i=0}^{n+1} (-1)^i [v_0,\dots,\widehat{v_i},v_{n+1}]\right)\\
& = \sum_{i=0}^{n+1} (-1)^i \partial_n([v_0,\dots,\widehat{v_i},v_{n+1}])\\
& = \sum_{i=0}^{n+1} (-1)^i \left(\sum_{j<i} (-1)^j [v_0,\dots,\widehat{v_j},\dots,\widehat{v_i},\dots,v_{n+1}]\right. \\
& \qquad\qquad \left.+ \sum_{j>i} (-1)^{j-1} [v_0,\dots,\widehat{v_i},\dots,\widehat{v_j},\dots,v_{n+1}]\right)
\end{align*}
Si $k < l$, el símplice orientado $[v_0,\dots,\widehat{v_k},\dots,\widehat{v_l},\dots,v_{n+1}]$ aparece dos veces en la combinación lineal anterior: cuando $(i,j)=(l,k)$ y cuando $(i,j)=(k,l)$.
En la primera aparición, tiene signo $(-1)^{k+l}$, en la otra aparición tiene signo $(-1)^{k+l-1}$, luego su suma es $0$.
Como todos los símplices se anulan, tenemos el lema.
\end{proof}

Entonces tenemos el complejo de cadenas simpliciales orientadas en $K$:
\[ \dots \to C_{n+1}(K; \mathbb{F}) \xrightarrow{\partial_{n+1}} C_n(K; \mathbb{F}) \xrightarrow{\partial_n} C_{n-1}(K; \mathbb{F}) \to \dots \to C_0(K; \mathbb{F}) \to 0 \]
De aquí podemos definir el $n$-ésimo espacio vectorial de homología simplicial de $K$ (todos sus símplices con el mismo signo de orientación):
\[ H_n(K; \mathbb{F}) := \frac{\ker \partial_n}{\Ima \partial_{n+1}}\]


\begin{prop}
El complejo de cadenas $\{(C_n(K;\F),\partial_n)\}_{n\geq 0}$ admite un aumento.
\end{prop}

\begin{dem}
Hay que buscar $\varepsilon:C_0(K;\F)\to\F$ homomorfismo sobreyectivo tal que $\varepsilon\circ\partial_1=0$. Basta tomar la aplicación $v\mapsto \varepsilon(v)=1$ para todo vértice $v\in K$. Claramente es un homomorfismo sobreyectivo. También es fácil ver que $\varepsilon\circ\partial_1=0$, pues
\[
\varepsilon(\partial_1[v_i,v_j])=\varepsilon(v_i-v_j)=\varepsilon(v_i)-\varepsilon(v_j)=1-1=0
\]
\QED
\end{dem}
El complejo
\[
\cdots\to C_{n+1}(K;\F)\overset{\overline{\partial}_{n+1}}{\to}C_n(K;\F)\overset{\overline{\partial}_n}{\to}\cdots\to C_0(K;\F)\overset{\varepsilon}{\to}\F\to 0
\]
da lugar a la homología simplicial reducida dada por 
\[
\widetilde{H}_n(K;\F)=\frac{\ker\overline{\partial}_n}{\Ima\overline{\partial}_{n+1}}
\]
donde $\overline{\partial}_0=\varepsilon$. Recordemos que $\widetilde{H}_n(K;\F)=H_n(K;\F)$ cuando $n\neq 0$ y $H_0(K;\F)=\widetilde{H}_0(K;\F)\oplus\F$.

\begin{prop}
Sean $K$ y $L$ complejos simpliciales y sea $\varphi:K\to L$ una aplicación simplicial. Entonces $\varphi$ induce homomorfismos $\varphi_*:C_n(K;\F)\to C_n(L;\F)$ para todo $n\geq 0$. Es más, se tienen diagramas conmutativos
\[
\begin{tikzcd}
C_n(K;\F)\arrow[r, "\varphi_*"]\arrow[d, "\partial_n"] & C_n(L;\F)\arrow[d, "\partial_n"]\\
C_{n-1}(K;\F)\arrow[r, "\varphi_*"] & C_{n-1}(L;\F)
\end{tikzcd}
\]
\end{prop}
\begin{dem}
Dado $n>0$, definimos $\varphi_*:C_n(K;\F)\to C_n(L;\F)$ de la siguiente manera
\[
\varphi_*([v_0,\dots, v_n])=\begin{cases}
[\varphi(v_0), \dots, \varphi(v_n)] & \varphi(v_i)\neq\varphi(v_j)\forall i\neq j\\
0& c.c.
\end{cases}
\]
Si $(v_0,\dots, v_n)$ es un $n$-símplice de $K$ entonces su imagen es un símplice de $L$ por ser $\varphi$ simplicial. Esta aplicación está bien definida además por la definición de las clases de equivalencia. Veamos la conmutatividad. Supongamos que tenemos un símplice con todos sus vértices distintos entre sí:
\[
\partial_n(\varphi_*([v_0,\dots, v_n]))=\partial_n([\varphi(v_0), \dots, \varphi(v_n)])=
\]
\[
\sum_{i=0}^n(-1)^i[\varphi(v_0), \dots,\widehat{\varphi(v_i)},\dots,  \varphi(v_n)]=\sum_{i=0}^n(-1)^i\varphi_*([v_0,\dots, \hat{v}_i,\dots, v_n])=
\]
\[
\varphi_*\left(\sum_{i=0}^n(-1)^i[v_0,\dots, \hat{v}_i,\dots, v_n]\right)=\varphi_*(\partial_n[v_0,\dots, v_n])
\]

Por otro lado, si algún vértice se repite, es decir, $\varphi(v_i)=\varphi(v_j)$ para algún $i\neq j$ (supondremos por sencillez que solo en un caso), entonces $\partial_n(\varphi_*([v_0,\dots, v_n]))=0$. Mientras que
\[
\varphi_*(\partial_n[v_0,\dots, v_n])=\varphi_*\left(\sum_{k=0}^n(-1)^k[v_0, \dots, \hat{v}_k,\dots, v_n]\right)=\sum_{k=0}^n(-1)^k\varphi_*([v_0,\dots, \hat{v}_k,\dots, v_n])
\]
Para todo $k\neq i,j$ se tiene que $\varphi_*[v_0,\dots, \hat{v}_k,\dots, v_n]=0$, así que la suma anterior nos da
\[
(-1)^i\varphi_*[v_0,\dots, \hat{v}_i,\dots, v_n]+(-1)^j\varphi_*([v_0,\dots, \hat{v}_j,\dots, v_n])=
\]
\[
(-1)^i[\varphi(v_0),\dots, \widehat{\varphi(v_i)},\dots, \varphi(v_n)]+(-1)^j[\varphi(v_0),\dots, \widehat{\varphi(v_j)},\dots, \varphi(v_n)]
\]
Pongamos que $i<j$ sin pérdida de generalidad. Entonces
\[
[\varphi(v_0),\dots, \widehat{\varphi(v_i)},\dots, \varphi(v_n)]=(-1)^{j-i-1}[\varphi(v_0),\dots, \widehat{\varphi(v_j)},\dots, \varphi(v_n)]
\]
y al sustituir obtenemos 0, como deseábamos.
\QED
\end{dem}

\begin{nota}\
\begin{enumerate}
\item \underline{Homología de un vértice (0-símplice)}: si $K=\{v\}$, su complejo de cadenas asociado es trivial en todos los niveles salvo en $C_0(K)=\gene{v}$. Por tanto, $H_q(K)=0$ para todo $q\geq 1$ y $H_0(K)=\frac{\ker\partial_0}{\Ima\partial_1}=\ker\partial_0=\gene{v}\cong\F$. Se tiene entonces que $\dim H_0(K)=1$ y en el resto de casos $\dim H_q(K)=0$. A estos números se los llama \emph{números de Betti}. En el caso de la homología reducida, $\widetilde{H}_0=\frac{\ker\varepsilon}{\Ima\partial_1}=\ker\varepsilon=0$. 

\item \underline{Homología de un cono}: Sea $K=vL$ un cono simplicial. Tenemeos la inclusión natural $i:\{v\}\to K$, que induce una aplicación $i_*: \{(C_n,\partial_n)\}\to \{(C_n(K),\partial_n)\}$. La idea es probar que $i_*$ induce una equivalencia de homotopía entre los complejos de cadenas. Sea $g:\{(C_n(K),\partial_n)\}\to \{C_n(v),\partial_n)\}$ como $g_n(\sigma)=0$ para $n\geq 1$ y $g_p(v_i)=v$ para todo $v_i\in K^0$. Es claro que $g\circ i_*=Id_{\CC(v)}$. Vamos a ver que $i_*\circ g\simeq Id_{\CC(K)}$. Definimos una homotopía $h: \CC(K)\to\CC(K)[+1]$ como 
\[
h_n ([v_0,\dots,v_n]) = \begin{cases}
	[v, v_0,\dots,v_n] & \text{ si }v_i \neq v\text{ para todo }i\\
	0 & \text{ si existe }i_0\text{ con }v_{i_0} = v
\end{cases}
\]

\[
\begin{tikzcd}
& C_n(K)\arrow[r, "\partial_n"]\arrow[ld, "h_n" above]\arrow[d, shift left, "Id_n" left=5pt]\arrow[d, shift right, "i_{*n} \circ g_n" right=5pt] & C_{n-1}(K)\arrow[ld, "h_{n-1}"]\\
C_{n+1}(K)\arrow[r, "\partial_{n+1}"] & C_n(K)
\end{tikzcd}
\]

Sea $σ$ un $n$-símplice de $K$. Si $σ$ no contiene $v$, entonces:
\[ i_{*n} \circ g_n ([v_0,\dots,v_n]) - Id_n([v_0,\dots,v_n]) = 0 - Id_n(v_0,\dots,v_n) = -[v_0,\dots,v_n] \]
\begin{align*}
(\partial_{n+1} \circ h_n & + h_{n-1} \circ \partial_n)(v_0,\dots,v_n) = \partial_{n+1}([v,v_0,\dots,v_n]) + h_{n-1}\left(\sum_{i=0}^n (-1) ([v_0,\dots,\widehat{v_i},\dots,v_n])\right)\\
& = \sum_{i=-1}^n (-1)^i (v,v_0,\dots,\widehat{v_i},\dots,v_n) + \sum_{i=0}^n(-1)^i h_{n-1}(v_0,\dots,\widehat{v_i},\dots,v_n)\\
& = -(v_0,\dots,v_n) + \sum_{i=0}^n (-1)^i ([v,v_0,\dots,\widehat{v_i},\dots,v_n]) + \sum_{i=0}^n (-1)^{i+1} ([v,v_0,\dots,\widehat{v_i},\dots,v_n])\\
& = -[v_0,\dots,v_n]
\end{align*}
Si por el otro caso, $σ$ contiene un vértices $v_{i_0}=v$ tenemos que:
\[ i_{*n} \circ g_n ([v_0,\dots,v_n]) - Id_n([v_0,\dots,v_n]) = 0 - Id_n([v_0,\dots,v_n]) = -[v_0,\dots,v_n] \]
Como $h_n([v_0,\dots,v_n])=0$ ahora:
\begin{align*}
(\partial_{n+1} \circ h_n & + h_{n-1} \circ \partial_n)[v_0,\dots,v_n] =  h_{n-1}(\partial_n[v_0,\dots,v_n])\\
& = h_{n-1}\left(\sum_{i=0}^n (-1)^i (v_0,\dots,\widehat{v_i},\dots,v_n)\right)\\
& = \sum_{i=0}^n (-1)^i h_{n-1}[v_0,\dots,\widehat{v_i},\dots,v_n]\\
& = (-1)^{i_0}[v,v_0,\dots,\widehat{v_{i_0}},\dots,v_n] \\
& = -[v_0,\dots,v_n]
\end{align*}

La demostración se para aquí por el profesor.

Esto demuestra que:
\[ H_n(vL) = \begin{cases}
0 & \text{ si }n ≥ 1\\
\gene{v} & \text{ si }n=0
\end{cases}\]
\[ \widetilde{H_n}(vL) = 0 \]
\end{enumerate}
\end{nota}

\end{document}