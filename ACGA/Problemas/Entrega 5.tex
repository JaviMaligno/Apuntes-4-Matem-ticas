\documentclass[twoside]{article}
\usepackage{../../estilo-ejercicios}
\usepackage{comment}
%--------------------------------------------------------
\begin{document}

\title{Algebra Conmutativa y Geometría Algebraica}
\author{Javier Aguilar Martín, Rafael González López}
\maketitle

\begin{ejercicio}{1}\
Sea $I \subseteq k[x_0, x_1,\dots, x_n]$ un ideal.
\begin{enumerate}
\item Probar que $I$ es homogéneo si y solo si para todo $f \in I$, las
componentes
homogéneas de $f$ también están en $I$.
\item  Si $I$ es homogéneo, probar que $I$ es radical si y solo si para todo $f \in k[x_0, x_1,\dots, x_n]$
\textbf{homogéneo} tal que $f^n \in I$ para algún $n \geq 1$ se tiene que $f \in I$.
\item Si $I$ es homogéneo, probar que $I$ es primo si y solo si para todos $f, g \in
k[x_0, x_1,\dots, x_n]$ \textbf{homogéneos} tales que $fg \in I$ se tiene que $f \in I$ o $g \in I$.
\end{enumerate}
\end{ejercicio}
\begin{solucion}\
\begin{enumerate}
\item $\boxed{\Rightarrow}$ Supongamos que $I=\langle f_1,\dots, f_m\rangle$ donde los $f_i$ son homogéneos. Entonces, dado $f\in I$, podemos escribir $f=\sum_{i=1}^mf_ig_i$ para ciertos polinomios $g_i\in k[x_0,x_1,\dots,x_n]$. Sea $g_i=\sum_j g_{ij}$ la descomposición en componentes homogéneas de cada $g_i$. Entonces obtenemos 
$$f=\sum_{i=1}^m\sum_j f_ig_{ij}.$$
Como el producto de polinomios homogéneos es homogéneo, hemos obtenido una descomposición de $f$ en sus componentes homogéneas, las cuales están en $I$ porque son de la forma $f_ig_{ij}$. 

$\boxed{\Leftarrow}$ Si para todo polinomio $f\in I$ sus componentes homogéneas están en $I$, basta tomar el conjunto de todas las componentes homogéneas de polinomios de $I$, que claramente generan el ideal. Como $k[x_0,x_1,\dots,x_n]$ es noetheriano sabemos que podremos seleccionar una cantidad finita de componentes homogéneas que generen $I$, por lo que $I$ es homogéneo.

\newpage

\item $\boxed{\Rightarrow}$ Si $I$ es radical, la condición se cumple para todo polinomio, en particular para los homogéneos.

$\boxed{\Leftarrow}$ Sea $f\in k[x_0,x_1,\dots,x_n]$ un polinomio tal que $\exists n\geq 2$ de forma que $f^n\in I$. Tenemos que probar que $f\in I$. Sea $f=\sum_i h_i$ la descomposición de $f$ en sus componentes homogéneas. Tenemos que $\left(\sum_i h_i\right)^n\in I$. Al desarrollarlo, obtenemos una suma de productos de las componentes homogéneas de $f$, y como $f^n\in I$, cada uno de estos productos está en $I$, en particular, $h_i^n\in I\ \forall i$. Por hipótesis, al ser $h_i$ homogéneos, esto implica que $h_i\in I\ \forall i$. Por tanto, $f\in I$, como queríamos demostrar.

\item $\boxed{\Rightarrow}$ Si $I$ es primo, la condición se cumple para todo polinomio, en particular para los homogéneos.

$\boxed{\Leftarrow}$ Sean $f,g\in k[x_0,x_1,\dots,x_n]$ tales que $fg\in I$. Vamos a probar por reducción al absurdo que o bien $f\in I$ o bien $g\in I$. Supongamos que $f,g\notin I$ y escribamos las descomposiciones en componentes homogéneas $f=\sum_{i=1}^m \alpha_i$, $g=\sum_{j=1}^l\beta_j$. Como $f,g\notin I$, hay alguna componente homogénea en cada polinomio que tampoco está en $I$. Supongamos sin pérdida de generalidad que $\alpha_1,\beta_1\notin I$. 

Como $fg\in I$, con $I$ homogéneo, por el primer apartado tenemos que todas las componentes homogéneas de $fg$ están en $I$. La primera componente de $fg$ es $\alpha_1\beta_1\in I$, que por hipótesis esto implica que o bien $\alpha_1\in I$ o bien $\beta_1\in I$. Sin embargo, esto entra en contradicción con la elección de $\alpha_1$ y $\beta_1$, por lo que hemos llegado a un absurdo.


\end{enumerate}

\end{solucion}


\newpage
\begin{ejercicio}{1}\
Sea $I \subseteq k[x_0, x_1,\dots, x_n]$ un ideal.
\begin{enumerate}
\item Probar que $I$ es homogéneo si y solo si para todo $f \in I$, las
componentes
homogéneas de $f$ también están en $I$.
\item  Si $I$ es homogéneo, probar que $I$ es radical si y solo si para todo $f \in k[x_0, x_1,\dots, x_n]$
\textbf{homogéneo} tal que $f^n \in I$ para algún $n \geq 1$ se tiene que $f \in I$.
\item Si $I$ es homogéneo, probar que $I$ es primo si y solo si para todos $f, g \in
k[x_0, x_1,\dots, x_n]$ \textbf{homogéneos} tales que $fg \in I$ se tiene que $f \in I$ o $g \in I$.
\end{enumerate}
\end{ejercicio}
\begin{solucion}
\begin{enumerate}
\item[]
\item Supongamos que $I$ es homogéneo, es decir, (dado que estamos en un anillo noetheriano), $I=\langle f_1,\dots, f_m\rangle$ con $f_i$ son homogéneos. Sea $f\in I$, sabemos que $f=\sum_{i=1}^mf_ig_i$ con $g_i\in k[x_0,x_1,\dots,x_n]$. Sea $g_{ij}$ la componente homogénea de grado $j$ de $g_i$, donde $d=\max_{i} \deg{g_i}$. Entonces
$$f=\sum_{i=1}^m f_i g_i= \sum_{i=1}^m\sum_{j=0}^d f_ig_{ij}.$$
Obviamente, $f_i g_{ij}$ es un polinomio homogéneo, por ser producto de homogéneos. Agrupando, tenemos que cada componente homogénea de $f$ es la forma $ \sum_{h,k} f_hg_{hk} = \sum_{h}f_h\sum_{k} g_{hk}= \sum_{h}f_h q_h$, luego pertenecen a $I$.

Recíprocamente, supongamos que las componentes homogéneas de $f$ están en $I$ $\forall f \in I$. Como $k[x_0,x_1,\dotsc,x_n]$ es noetheriano, sabemos que $I=\langle f_1,\dots, f_m\rangle$. Sea $f_{ij}$ la componente homogénea de grado $j$ de $f_i$, tomando índice hasta $d=\max_{i} \deg{f_i}$. Si $S=\{f_{ij} \mid i=1,\dotsc m,\,j=0,\dotsc d\}$ entonces, como $S\subset I$ por hipótesis, $\gene{S} \subset I$. La inclusión recíproca es trivial, pues con los elementos de $S$ podemos escribir los $f_i$. Por tanto $I=\gene{S}$, es decir, está generado por polinomios homogéneos.

\item Supongamos que $I$ es radical, entonces en $I$ se cumple la condición para todo $f\in k[x_0,x_1,\dots,x_n]$. En particular, se cumple para los polinomios homogéneos.

Recíprocamente, supongamos que se verifica que para todo $f \in k[x_0, x_1,\dots, x_n]$ \textbf{homogéneo} tal que $f^n \in I$ para algún $n \geq 1$ se tiene que $f \in I$. Para ver que es radical, sea $f \in k[x_0, x_1,\dots, x_n]$ tal que $f^n \in I$. Por reducción al absurdo, supongamos que $\exists f$ tal que $f^n \in I$ para algún $n\geq 1$ pero $f\notin I$. Podemos escribir $f = \sum_{i=1}^{\deg{f}} f_i$ donde $f_i$ es la componente homogénea de grado $i$ de $f$. 

Como $f \notin I$, en particular alguna de sus componentes homogéneas debe no pertenecer a $I$. Sea $e=\max{\{i\mid f_i \notin I\}}$. Consideremos ahora la componente homogénea de $f$ de grado $n\cdot e$. Esto es una suma en $n$ índices
$$
(f^n)_{n\cdot e} = \underset{i_1+\dotsc+i_n=ne}{\sum_{i_1,\dotsc,i_n =1}^d} f_{i_1}\cdots f_{i_n}
$$
Por hipótesis, si $i>e$ entonces $f_i \in I$. Por el Principio del Palomar, en cada sumando salvo $f_e^n$, debe existir un $i_j >e$. Por tanto, todos los sumandos pertenecen a $I$ salvo, a priori, $f_e^n$. Como $f^n \in I$, por el primer apartado sus componentes homogéneas también, luego $(f^n)_{n\cdot e}\in I$ y, por tanto, $f_e^n \in I$. Pero por hipótesis, esto implica que $f_e\in I$, lo cuál es una contradicción.
\item Supongamos que $I$ es primo, entonces en $I$ se cumple la condición para todo $f\in k[x_0,x_1,\dots,x_n]$. En particular, se cumple para los polinomios homogéneos.

Recíprocamente, supongamos que para todos $f, g \in
k[x_0, x_1,\dots, x_n]$ \textbf{homogéneos} tales que $fg \in I$ se tiene que $f \in I$ o $g \in I$. Por reducción al absurdo, supongamos que $\exists f,g \notin I$ tales que $fg\in I$. Razonando como en los apartados anteriores, descomponemos $f$ y $g$ en sus componentes homogéneas
$$
f =\sum_{i=0}^{\deg f} f_i \qquad g =\sum_{j=0}^{\deg g}g_j
$$
Como $f,g\notin I$, en particular alguna de sus componentes homogéneas debe no pertenecer a $I$. Sean $e=\max{\{i\mid f_i\notin I\}}$ y $s=\max{\{j\mid g_i\notin I\}}$, la componente homogénea de $fg$ de grado $s+e$ es 
$$
(fg)_{s+e} = f_0g_{s+e}+f_1g_{s+e-1}+\dotsc + f_{s+e-1}g_1 + f_{s+e}g_0 
$$
Por hipótesis sobre $e$ y $s$, $\forall i >e$ y $\forall j>s$, $f_i,g_j\in I$. En particular, podemos asegurar que todos los sumandos de la expresión anterior están en $I$ salvo a priori $f_eg_s$. Utilizando el primera apartado, tenemos que $(fg)_{s+e}\in I$ (puesto que $fg\in I$), por lo que usando lo anterior, tenemos que $f_eg_s\in I$. Como $f_e,g_s$ son homogéneos, por hipótesis, tenemos que $f_e,g_s\in I$, lo cuál es una contradicción. 
\end{enumerate}
\end{solucion}
\end{document}