\documentclass{article}
\usepackage{../estilo-ejercicios}
\usepackage[a4paper, total={6in, 9in}]{geometry}
\newcommand{\A}{{\mathbb{A}}}
\newcommand{\m}{{\mathfrak{m}}}
\DeclareMathOperator{\GL}{GL}
\DeclareMathOperator{\SL}{SL}

\begin{document}
\begin{center}
\large Ejercicio de Álgebra Conmutativa y Geometría Algebraica

\normalsize Diego Pedraza López
\end{center}

\begin{ejercicio}{8}
Sea $S^{-1}A$ el anillo total de fracciones de $A$, donde $S \subseteq A$ es el conjunto de elementos que no son divisores de cero, y sea $φ : A \to S^{-1}A$ el homomorfismo canónico. Probar que $φ$ es inyectivo, y que $S$ es el mayor subconjunto multiplicativo de $A$ para el que $φ$ es inyectivo. Probar que $φ$ es un isomorfismo si y sólo si todo elemento de $A$ que no sea unidad es un divisor de cero.
\end{ejercicio}

\begin{sol}
Calculemos $\ker φ$. Para un $a \in A$, si $φ(a)=a/1=0/1$, entonces existe $s \in S$ tal que:
\[ a \cdot 1 \cdot s = 0 \cdot 1 \cdot s  \Leftrightarrow a \cdot s = 0 \]
Como $S$ no contiene divisores de cero, entonces $a \cdot s = 0 \Rightarrow a = 0$, luego $\ker φ = \{0\}$. Luego $φ$ es homomorfismo inyectivo.

Sea $T \supsetneq S$ un subconjunto multiplicativo de $A$, veamos que $ψ : A \to T^{-1}A$ no es inyectivo. Si $T$ contiene el $0$, entonces $T^{-1}A=\{0/1\}$, luego si $A$ no es un anillo trivial, $ψ$ no puede ser inyectivo. Si $T$ contiene un divisor de cero $t$, entonces existe un elemento $a \in A$ no nulo tal que $a\cdot t = 0$. Por lo tanto:
\[ a \in \ker ψ \Rightarrow \ker ψ \neq \{0\} \]
Luego $ψ$ no es inyectivo. Entonces $S$ es el mayor subconjunto multiplicativo para el que $φ$ es inyectivo.

Demostremos que:
\[ φ \text{ es isomorfismo} \Leftrightarrow \text{Si }a \in A \text{ no es unidad, entonces }a\text{ es divisor de cero} \]
Podemos transformar la segunda sentencia en otra equivalente más fácil de probar:
\[ φ \text{ es isomorfismo} \Leftrightarrow \text{Si }s \in S \text{ entonces }s\text{ es unidad} \]
\begin{itemize}
	\item[($\Rightarrow$)] Supongamos que $φ$ es isomorfismo y sea $s \in S$. Como $S^{-1}A$ debe contener $1/s$ y $φ$ es sobreyectiva, entonces existe $a \in A$ tal que $φ(a)=1/s$. Como $1=φ(a)φ(s)=φ(as)$ y $φ$ es homomorfismo inyectivo, entonces $as=1$ y $s$ es unidad.
	
	\item[($\Leftarrow$)] Supongamos que todo elemento de $S$ es unidad. Sea $a/s \in S^{-1}A$. Entonces:
	\[ \frac{a}{s} \sim \frac{as^{-1}}{1} \]
	pues $a=as^{-1}s$. Pero $\dfrac{as^{-1}}{1}=φ(as^{-1}) \in \text{Im}(φ)$. Luego $φ$ es sobreyectiva y, en consecuencia, isomorfismo.
\end{itemize}
\end{sol}
\end{document}