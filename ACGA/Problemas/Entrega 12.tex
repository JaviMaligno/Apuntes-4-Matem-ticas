\documentclass[twoside]{article}
\usepackage{../../estilo-ejercicios}
%\DeclareMathOperator{\Ima}{Im}
\DeclareMathOperator{\Char}{char}
%--------------------------------------------------------
\begin{document}

\title{Algebra Conmutativa y Geometría Algebraica}
\author{Rafael González López, Javier Aguilar Martín, Diego Pedraza López}
\maketitle

\begin{ejercicio}{1}
Sea $A = \bigoplus_{n≥0} A_n$ un anillo graduado.
\begin{enumerate}
	\item Probar que $A^{+k} := \bigoplus_{n≥k} A_n$ es un ideal de $A$ para todo $k ≥ 1$.
	\item Probar que $A$ es noetheriano si y sólo si $A_0$ es noetheriano y $A$ es una $A_0$-álgebra finitamente generada.
\end{enumerate}
\end{ejercicio}
\begin{solucion}\mbox{}
\begin{enumerate}
	\item Sean $a,b \in A^{+k}$. Existen $a_i \in A_i$ con $i=k,\dots,m$ tal que $a = \sum_{i=k}^{m} a_i$. También existen $b_j \in A_j$ con $j=k,\dots,n$ tal que $b = \sum_{j=k}^n b_j$. Sin pérdida de generalidad, podemos suponer que $m≥n$. Para todo $j$ tal que $n≥j>m$, podemos tomar $b_j = 0 \in A_j$. Entonces podemos escribir $b = \sum_{i=k}^m b_i$. Finalmente, como $A_i$ es subgrupo aditivo, $a_i+b_i \in A_i$ para todo $i$, luego:
	\[ a + b = \sum_{i=1}^m (a_i + b_i) \in A^{+k}\]
	Sea $a \in A^{+k}$ y $b \in A$. Existen $a_i \in A_i$ con $i=k,\dots,m$ tal que $a = \sum_{i=k}^{m} a_i$. Además existen $b_j \in A_j$ con $j=0,\dots,n$ tal que $b = \sum_{j=0}^{n} b_j$. Como $A_i \cdot A_j \subseteq A_{i+j}$, tenemos que $a_i \cdot b_j \in A_{i+j}$. Tenemos que $i+j ≥ k$ para todo $i≥k$ y $j≥0$, luego $a_i \cdot b_j \in A^{+k}$. Entonces:
	\[ ab = \sum_{i=k}^{m} \sum_{j=0}^{n} a_i b_j \in A^{+k} \]

	Luego $A^{+k}$ es ideal de $A$ para todo $k≥0$.

	\item Supongamos primero que $A$ es noetheriano. Tenemos entonces que $A_0 \cong A/A^{+1}$. Como $A$ es noetheriano y $A^{+1}$ es ideal, $A/A^{+1}$ es noetheriano. Cogemos un sistema generador finito de elementos homogéneos de $A^{+1}=\langle a_1,\dots,a_n \rangle$ de grados $d_1,\dots,d_n$ respectivamente. Sea $\tilde{A}=A_0[a_1,\dots,a_n]$ un $A_0$-álgebra. Veamos que $\tilde{A}=A$. Claramente $\tilde{A} \subseteq A$.

	Demostraremos por inducción que $A_n \subseteq \tilde{A}$ para todo $n ≥ 0$, lo que implica que $A = \bigoplus_{i≥0} A_n \subseteq \tilde{A}$. 
	Para $n=0$ está clara la inclusión. Para $n≥1$, supuesta la hipótesis como cierta para $0,\dots,n-1$, tomamos un $x \in A_n$. Como $x \in A^{+1}$, existen $x_i \in A$ tal que $x = \sum x_i a_i$.
	Como $x$ es elemento homogéneo de grado $n$, tenemos que todas las componentes homogéneas de grado distinto de $n$ se anulan, luego podemos quedarnos con la componente homogénea $\tilde{x}_i$ de $x_i$ de grado $\max(n-d_i,0)$. Entonces $\tilde{x}_i \in M_{\max(n-d_i,0)}$. Como $d_i > 0$ podemos usar la hipótesis de inducción de manera que $x_i \in \tilde{A}$, luego $x=\sum x_ia_i \in \tilde{A}$.

	Luego $A=\tilde{A}$ y, en consecuencia, $A$ es un $A_0$-álgebra finitamente generada.

	Supongamos ahora que $A_0$ es noetheriano y $A$ es una $A_0$-álgebra finitamente generada. Entonces $A=A_0[a_1,\dots,a_n]$ para $a_1,\dots,a_n \in A$. Como $A_0$ es noetheriano, por el teorema de la base de Hilbert, $A$ es noetheriano.
\end{enumerate}
\end{solucion}

\newpage

\begin{ejercicio}{2}
 
Sean $X, Y ⊆ \mathbb{P}^n_k$
dos conjuntos algebraicos de dimensión $d$ tales
que $\dim(X ∩Y ) < d$. Probar que el grado de $X ∪Y$ es la suma de los grados de
$X$ e $Y$ . Deducir que el grado de un conjunto finito formado por $d$ puntos es $d$.
\end{ejercicio}

\begin{solucion}\
\begin{itemize}
\item Vamos a tratar primero el caso en el que $X$ e $Y$ sean irreducibles. Sea $I_1=\I(X),  I_2=\I(Y), I=\I(X\cup Y)$. Entonces, como $I_1\cap I_2=I$ y la dimensión de la intersección es menor que la de los conjuntos por separado, tenemos las siguiente sucesión exacta
\[
0\to R/I\to R/I_1\oplus R/I_2\to R/(I_1+I_2)\to 0
\]

Por el primer apartado del ejercicio 4 de esta relación tenemos que $\varphi_{R/I}+\varphi_{R/(I_1+I_2)}=\varphi_{R/I_1\oplus R/I_2}$ y por el segundo apartado de dicho ejercicio $\varphi_{R/I_1\otimes R/I_2}=\varphi_{R/I_1} + \varphi_{R/I_2}$, o lo que es lo mismo,
\[
\varphi_{X\cup Y}+\varphi_{X\cap Y}=\varphi_{X} + \varphi_{Y}
\]
Como $\dim(X)=\dim(Y)$, los dos polinomios del lado derecho tienen el mismo grado, por lo que el término líder de la suma es $\deg(X)/d!+\deg(Y)/d!$. Por otro lado, como la intersección tiene dimensión menor, entonces el coeficiente líder del polinomio de la izquierda es del de $\varphi_{X\cup Y}$, que es simplemente $\deg(X\cup Y)/d!$, por lo que tenemos la igualdad $\deg(X\cup Y)=\deg(X)+\deg(Y)$.
\end{itemize}
\end{solucion}
\newpage

\begin{ejercicio}{4}
 Sea $A=k[x_0,\dotsc,x_n]$, visto como anillo graduado.
\begin{enumerate}
\item Sea $0\to M\to N\to P \to 0$ una sucesión exacta de $A$-módulos graduados finitamente generados. Probar que $\phi_N(t)=\phi_M(t)+\phi_P(t)$.
\item Si $N=M_1\oplus\cdots\oplus M_r$, probar que $\phi_N(t)=\sum_{i=1}^r\phi_{M_i}(t)$.
\end{enumerate}
\end{ejercicio}

\begin{solucion}Vamos a utilizar el siguiente resultado demostrado en Teoría.
\begin{lemma}\label{lemma:1}
Sea $A = k[x_0,\dots,x_n]$ y $M$ un $A$-módulo graduado finitamente generado con $M = \bigoplus_{r≥0} M_r$. Entonces para todo $r$, $M_r$ es un $k$-espacio vectorial de dimensión finita.
\end{lemma}
Pasamos a resolver el ejercicio.
\begin{enumerate}
\item  Como todos son $A$-módulos graduados, podemos considerar sus graduaciones
$$
M=\bigoplus_{t\geq 0} M_t \qquad N=\bigoplus_{t\geq 0} N_t \qquad M=\bigoplus_{t\geq 0} P_t $$
Además, existe un monomorfismo $f\func{M}{N}$ y un epimorfismo $g\func{N}{P}$ de $A$-módulos graduados tales que
\begin{equation}\label{eq:1}
f(M_t) \subset N_t \qquad g(N_t) \subset P_t
\end{equation}
\begin{lemma}\label{lemma:2}
Para todo $t\geq 0$ se tiene que 
$$g(N_t)=P_t$$
\end{lemma}
\begin{proof}
Como $g$ es sobreyectiva $\forall p\in P_t$ $\exists s\in M$ tal que $g(s)= p \in P_t$. Sea $p\in P_t$, $\exists x_{i_1},\dotsc,x_{i_m}$, distintos entre sí, con $x_{i_k}\in M_{i_k}$ tales que $s =\sum_{k=1}^m x_{i_k}$ y 
$$
g(s)=g\left(\sum_{k=1}^m x_{i_k}\right) = \sum_{k=1}^m g(x_{i_k}) = p \in P_t
$$ Por hipótesis, $g(x_{i_k})\in P_{i_k}$, luego la única manera de que esa suma resulte un término de $P_t$ es que todos sean $0$ salvo, a lo sumo, un término tal que $i_q = t$ (auqnue también podría ser que $x_{i_q}=0$). Por tanto $x_{i_q}$ es una preimagen de $s$. 
\end{proof}
\begin{lemma}\label{lemma:3} La sucesión
$$0\to M_t\overset{f|_{M_t}}{\to} N_t\overset{g|_{N_t}}{\to} P_t \to 0$$
es una sucesión exacta de $k$-e.v. 
\end{lemma}

\begin{proof} Veamos que se verifican las propiedades.
\begin{itemize} 
\item $P_t,N_t$ y $M_t$ son $k$-e.v. de dimensión finita por el Lema \ref{lemma:1}.
\item Las restricciones de $f$ y $g$ a $M_t$ y $N_t$ respectivamente están bien definidas por la hipótesis de que son homomorfismos de $A$-módulos graduados (Ecuación \ref{eq:1}).
\item $f,g$ son homomorfismos de $k$-e.v. por ser homomorfismos de $A$-módulos, luego las restricciones también lo son. Además, $f$ es inyectiva por hipótesis, por lo que $f|_{M_t}$ también lo es. $g|_{N_t}$ es sobreyectiva por el Lema \ref{lemma:2}.
\item Sabemos que $\Ima(f)=\ker g$. Es claro que $\Ima(f|_{M_t})= \Ima(f)\cap N_t$. Una contención es trivial, sea $x\in \Ima(f)\cap N_t$, si no estuviese en $M_t$ podríamos razonar como en el Lema 2. Por definición, $\ker{g|_{N_t}}=\ker g \cap N_t$, luego
$$
\Ima(f|_{M_t})= \Ima(f)\cap N_t =\ker g \cap N_t = \ker{g|_{N_t}}
$$
\end{itemize}
\end{proof}
Una consecuencia inmediata del Lema \ref{lemma:3} es que $P_t \cong N_t/f(M_t)$, luego 
 $$
\dim_k(P_t) = \dim_k(N_t)-\dim_k(f(M_t)) \overset{\text{f inyectiva}}{=} \dim_k(N_t)-\dim_k(M_t)
$$
$$\phi_P(t) = \phi_N(t)-\phi_M(t) \Rightarrow \phi_N(t)=\phi_M(t)+\phi_P(t)$$
\item Podemos probarlo por inducción sobre $r$. 
\begin{itemize}
\item Para $r=1$ es trivial, no hay nada que probar. 
\item Para $r=2$, veamos que se deduce del apartado anterior si demostramos que 
$$
0\to M_1 \to M_1 \oplus M_2 \to M_2 \to 0$$
es una secuencia exacta de $A$-módulos graduados finitamente generados. Suponemos que la graduación de $M_i$ es la inducida por $N$. Sabemos que podemos identificar $N = \{(u,v)\mid u\in M_1,\;v\in M_2\}$ luego consideramos $M=\bigoplus_{t\geq 0} S_t$, $M_i = \bigoplus_{t\geq 0} M_i \cap S_t$.
\begin{itemize}
\item Tomamos $f\func{M_1}{N}$, $f(u,0)=(u,0)$, claramente un monomorfismo de $A$-módulos graduados. 
\item Ahora consideramos $g\func{N}{M_2}$, $g(u,v)=(0,v)$. Claramente es un epimorfismo de $A$-módulos graduados. 
\item Es claro que $\Ima(f)=M_1 = \ker g$.
\end{itemize}
\item Si suponemos cierto para $r=k$, veámoslo para $r=k+1$. Sabemos que
$$N=M_1 \oplus \cdots \oplus M_k \oplus M_{k+1} = (M_1 \oplus \cdots \oplus M_k) \oplus M_{k+1} = M' \oplus M_{k+1}$$
Como hemos demostrado el caso $k=2$, sabemos que $\phi_N(t)=\phi_{M'}(t)+\phi_{M_{k+1}}(t)$. Aplicando la hipótesis de inducción a $M'$ tenemos $\phi_{M'}(t)=\sum_{i=1}^k \phi_{M_i}(t)$. Por tanto $\phi_N(t)=\sum_{i=1}^{k+1}\phi_{M_i}(t)$, como queríamos probar.
\end{itemize}
\end{enumerate}
\end{solucion}

\end{document}