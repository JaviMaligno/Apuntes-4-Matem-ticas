\documentclass{article}
\usepackage{../estilo-ejercicios}
\newcommand{\A}{{\mathbb{A}}}
\newcommand{\m}{{\mathfrak{m}}}
\DeclareMathOperator{\GL}{GL}
\DeclareMathOperator{\SL}{SL}

\begin{document}
\begin{center}
\large Ejercicio de Álgebra Conmutativa y Geometría Algebraica

\normalsize Diego Pedraza López
\end{center}

\begin{ejercicio}{8}
Sea $A$ un DFU, e $I = \langle p \rangle \subseteq A$ un ideal principal primo. Probar que no existe ningún ideal primo no nulo estrictamente contenido en $I$. Deducir que, si $X \subset \mathbb{A}_k^n$ es una hipersuperficie (es decir, un conjunto algebraico definido por una sola ecuación), entonces no existe ningún otro conjunto algebraico irreducible $Y$ distinto del total que contenga a $X$. ¿Es cierto si no suponemos que $Y$ sea irreducible?
\end{ejercicio}

\begin{sol}
Supongamos que existe un ideal $J$ primo no nulo contenido $\langle p \rangle$. Sea $a \in J$, como $A$ es un DFU, entonces hay elementos primos $q_1,\dots, q_r$ en $A$ con $a=q_1\cdot \cdots \cdot q_r$. Como $a \in J$ e $J$ es primo, al menos un factor $q \in \{q_1,\dots,q_r\}$ debe estar en $J$. Como $q \in J \subseteq \langle p\rangle$, entonces $q=pr$ para algún $r \in A$.
\begin{itemize}
	\item Si $r$ es unidad, entonces $q\cdot r^{-1} = p$, luego $\langle p \rangle \subseteq J$ y, en consecuencia, $J = \langle p \rangle$.
	\item Si $r$ no fuera unidad, entonces $q$ sería reducible. Sin embargo, $A$ es un dominio de integridad, luego sus elementos primos son irreducibles. Por lo tanto, este caso no se puede dar.
\end{itemize}
Entonces si $J$ es un ideal primo contenido en $I$, necesariamente $I = J$.

Sea $X \subset \mathcal{A}_k^n$ una hipersuperficie. Como $k$ es un cuerpo, en particular es dominio de factorización única, luego $k[x_1,\dots,x_n]$ es un dominio de factorización única. Como $X$ es una hipersuperficie, $\I(X)$ está generado por un polinomio irreducible (proposición 1.26). Como estamos en un DFU, $\I(X)$ es un ideal principal primo. Sea $Y$ un conjunto algebraico irreducible $Y$ distinto del total y supongamos que $Y \supseteq X$. Entonces $\I(Y) \subseteq \I(X)$. Como $Y$ es irreducible, $\I(Y)$ es primo. Por lo demostrado anteriormente, entonces $\I(Y) = \I(X)$. Por la correspondencía biunívoca entre ideales primos y conjuntos algebraicos irreducibles (corolario 1.18), entonces $X=Y$.

Si $Y$ no es irreducible, basta tomar $Y = X \cup \{P\}$, donde $P \in \mathbb{A}_k^n$ es un punto que no está en $X$. Aquí estamos suponiendo que $X$ es distinto del total.
\end{sol}
\end{document}