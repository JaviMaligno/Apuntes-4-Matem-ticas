\documentclass[twoside]{article}
\usepackage{../../estilo-ejercicios}
\DeclareMathOperator{\GL}{GL}
\DeclareMathOperator{\SL}{SL}
%--------------------------------------------------------
\begin{document}

\title{Algebra Conmutativa y Geometría Aplicada}
\author{Javier Aguilar Martín, Rafael González López}
\maketitle

\begin{ejercicio}{1}
Sea $A$ un anillo y $S\subset A$ un conjunto multiplicativa. Si $f\func{A}{B}$ es un homomorfismo de anillos tal que $f(S)\subset B^\times$, probar que existe un único homomorfismo de anillos $\tilde{f}\func{S^{-1}A}{B}$ que existe a $f$, en el sentido de que $\tilde{f}(a/1) = f(a)$.
\newline
Deducir que si $A$ es un dominio de integridad y $f\func{A}{L}$ un homomorfismo de anillos inyectivo, donde $L$ es un cuerpo, entonces $f$ se extiende de manera única a $K(A)$, el cuerpo de fracciones de $A$.
\begin{solucion}
Vamos a tratar de dar una definición de $\tilde{f}$. Sea $p/q$, sabemosq que $\tilde{f}(p/1)=f(p)$. Definimos $\tilde{f}(p/q)=k$ donde $k\in B$ es tal que $kf(q)=f(p)$. Tenemos que comprobar que está bien definido y que es homomorfismo.
\begin{itemize}
\item Supongamos que $\exists k,k'\in B$ tales que $kf(q)=k'f(q)=f(p)$. Como $f(q)\in B^\times$, podemos multiplicar por su inversa, obteniendo que $k=k'$. Por tanto, la aplicación está bien definida. De hecho, $\tilde{f}(p/q)=f(p)f(q)^{-1}$.
\item Para ver que es homomorfismo de anillos con unidad, comprobamos las tres propiedades de la definición. Notaremos indistamente $1$ al elemento neutro de la segunda operación tanto en $A$ como en $B$.
\begin{itemize}
\item Sea $1/1$ el neutro del producto en $S^{-1}A$, 
$$\tilde{f}(1/1)=f(1)f(1)^{-1}=1$$
\item Sean $p/q,p'/q' \in S^{-1}A$, entonces
\begin{align*}
\tilde{f}(p/q+p'/q') &= 	\tilde{f}\left(\frac{pq'+p'q}{qq'}\right) = f(pq'+p'q)f(qq')^{-1} = \\
 &= f(pq'+p'q)f(q)^{-1}f(q')^{-1} =
(f(pq')+f(p'q))f(q)^{-1}f(q')^{-1} = \\
&= (f(p)f(q')+f(p')f(q))f(q)^{-1}f(q')^{-1}  =\\
 &= f(p)f(q)^{-1}+f(p')f(q')^{-1} = \tilde{f}(p/q)+\tilde{f}(p'/q')
\end{align*}

\item Sean $p/q,p'/q' \in S^{-1}A$, entonces
\begin{align*}
\tilde{f}(p/q\cdot p'/q') &= 	\tilde{f}\left(\frac{pp'}{qq'}\right) = f(pp')f(qq')^{-1} = \\
 &= f(p)f(p')f(q)^{-1}f(q')^{-1} = \\
 &= f(p)f(q)^{-1}f(p')f(q')^{-1} =\\
 &= \tilde{f}(p/q)\tilde{f}(p'/q')
\end{align*}
\end{itemize}
Por tanto, hemos probado que nuestra definición de $\tilde{f}$ es un homomorfismo que extiende de anillo que extiende a $f$.
\end{itemize}
Vista la existencia, resta ver la unicidad. Supongamos que existe otro homomrfismo $\tilde{g}$ que extiende a $f$ además de $\tilde{f}$. En tal caso, como $\tilde{f}$ y $\tilde{g}$ extienden $f$, $\tilde{f}(a/1)=\tilde{g}(a/1)$. Por propiedades básicas de homomorfismo de anillos, basta ver que son iguales para los elementos de la forma $1/q$, con $q\in S$.
$$
f(q)f(q)^{-1} = \tilde{f}(q/q)= \tilde{f}(1/1) = 1 =\tilde{g}(1/1) =\tilde{g}(q/q) =\tilde{g}(q/1)\tilde{g}(1/q)=f(q)\tilde{g}(1/q)
$$
Usando que $f(q)\in B^\times$, obtenemos que $\tilde{g}(1/q)=f(q)^{-1}=\tilde{f}(1/q)$. Por tanto, si ${p/q\in S^{-1}A}$, entonces 
$$
\tilde{g}(p/q) = \tilde{g}(p/1\cdot 1/q) = \tilde{g}(p/1)\tilde{g}(1/q) = f(p)f(q)^{-1} = \tilde{f}(p/q)
$$
Para ver la segunda parte del ejercicio no tenemos que más que considerar que, al ser $f$ inyectiva, necesariamente $f(A^*)\subset L^*$, pues $L$ es cuerpo y su único elemento no unidad es el $0$. La hipótesis de que $A$ sea dominio es necesaria para definir el cuerpo de fracciones. No tenemos más que aplicar la primera parte a $S=A^{*}$, pero precisamente ${A^*}^{-1}A = K(A)$. 
\end{solucion}
\end{ejercicio}

\newpage

\begin{ejercicio}{6}\
Si $A$ es un anillo reducido (es decir, sin elementos nilpotentes no
nulos), probar que $S^{-1}A$ es reducido para todo conjunto multiplicativo $S \subseteq A$.
Recíprocamente, si $A_\mathfrak{m}$ es reducido para todo ideal maximal $\mathfrak{m} \subseteq A$, probar que
$A$ es reducido.
\end{ejercicio}
\begin{solucion}\
Probamos la primera cuestión por reducción al absurdo. Supongamos que $S^{-1}A$ no es reducido. Entonces existe $\frac{a}{s}\in S^{-1}A$, $\frac{a}{s}\not\sim 0$ tal que para algún $n\in\N$ $\left(\frac{a}{s}\right)^n=\frac{a^n}{s^n}\sim 0\Leftrightarrow \exists u\in S\mid a^nu=0$. Vamos a suponer que $u\neq 0$, puesto que de lo contrario todos los elementos estarían relacionados con el cero y trivialmente no habría elementos nilpotentes.  Nótese que $a^n\neq 0$ por hipótesis. Por tanto deducimos que o bien $au\in A$ es nilpotente, pues $(au)^n=a^nu^n=a^n u u^{n-1}=0$, contradiciendo que $A$ sea reducido; o bien $au=0$. Pero si $au=0$, entonces $\frac{a}{s}\sim 0$, que también es una contradicción con lo que estábamos suponiendo.\\

Pasemos ahora a la segunda cuestión. Sea $a\in A$ y supongamos que $\exists n\in\N$ tal que $a^n=0$. Entonces $\left(\frac{a}{1}\right)^n\sim 0\in A_\mathfrak{m}$ para todo ideal maximal $\mathfrak{m}$. Dado que $A_\mathfrak{m}$ es reducido, esto significa que $\frac{a}{1}\sim 0$. Entonces, para todo $\mathfrak{m}$, existe $t_\mathfrak{m}\in A\setminus\mathfrak{m}$ tal que $at_\mathfrak{m}=0$. Consideremos el conjunto $I=\{t\in A\mid at=0\}$. Claramente $I$ es un ideal, pues si $s,t\in I$ entonces $a(s+t)=as+at=0$, y si $t\in I,s\in A\setminus I$, entonces $ats=0$. Sin embargo, $I\not\subseteq \mathfrak{m}$ para todo ideal maximal $\mathfrak{m}$, ya que como hemos visto, para cada uno de estos ideales existe $t_\mathfrak{m}\in A\setminus\mathfrak{m}$ tal que $at_\mathfrak{m}=0$. Esto quiere decir que $I=A$, de donde se deduce que $a=0$, con lo que no es nilpotente. 

%0=1a=a
\end{solucion}

\newpage

\begin{ejercicio}{8}
Sea $S^{-1}A$ el anillo total de fracciones de $A$, donde $S \subseteq A$ es el conjunto de elementos que no son divisores de cero, y sea $φ : A \to S^{-1}A$ el homomorfismo canónico. Probar que $φ$ es inyectivo, y que $S$ es el mayor subconjunto multiplicativo de $A$ para el que $φ$ es inyectivo. Probar que $φ$ es un isomorfismo si y sólo si todo elemento de $A$ que no sea unidad es un divisor de cero.
\end{ejercicio}

\begin{solucion}
Calculemos $\ker φ$. Para un $a \in A$, si $φ(a)=a/1=0/1$, entonces existe $s \in S$ tal que:
\[ a \cdot 1 \cdot s = 0 \cdot 1 \cdot s  \Leftrightarrow a \cdot s = 0 \]
Como $S$ no contiene divisores de cero, entonces $a \cdot s = 0 \Rightarrow a = 0$, luego $\ker φ = \{0\}$. Luego $φ$ es homomorfismo inyectivo.

Sea $T \supsetneq S$ un subconjunto multiplicativo de $A$, veamos que $ψ : A \to T^{-1}A$ no es inyectivo. Si $T$ contiene el $0$, entonces $T^{-1}A=\{0/1\}$, luego si $A$ no es un anillo trivial, $ψ$ no puede ser inyectivo. Si $T$ contiene un divisor de cero $t$, entonces existe un elemento $a \in A$ no nulo tal que $a\cdot t = 0$. Por lo tanto:
\[ a \in \ker ψ \Rightarrow \ker ψ \neq \{0\} \]
Luego $ψ$ no es inyectivo. Entonces $S$ es el mayor subconjunto multiplicativo para el que $φ$ es inyectivo.

Demostremos que:
\[ φ \text{ es isomorfismo} \Leftrightarrow \text{Si }a \in A \text{ no es unidad, entonces }a\text{ es divisor de cero} \]
Podemos transformar la segunda sentencia en otra equivalente más fácil de probar:
\[ φ \text{ es isomorfismo} \Leftrightarrow \text{Si }s \in S \text{ entonces }s\text{ es unidad} \]
\begin{itemize}
	\item[($\Rightarrow$)] Supongamos que $φ$ es isomorfismo y sea $s \in S$. Como $S^{-1}A$ debe contener $1/s$ y $φ$ es sobreyectiva, entonces existe $a \in A$ tal que $φ(a)=1/s$. Como $1=φ(a)φ(s)=φ(as)$ y $φ$ es homomorfismo inyectivo, entonces $as=1$ y $s$ es unidad.
	
	\item[($\Leftarrow$)] Supongamos que todo elemento de $S$ es unidad. Sea $a/s \in S^{-1}A$. Entonces:
	\[ \frac{a}{s} \sim \frac{as^{-1}}{1} \]
	pues $a=as^{-1}s$. Pero $\dfrac{as^{-1}}{1}=φ(as^{-1}) \in \text{Im}(φ)$. Luego $φ$ es sobreyectiva y, en consecuencia, isomorfismo.
\end{itemize}
\end{solucion}

\end{document}