\documentclass[twoside]{article}
\usepackage{../../estilo-ejercicios}
\DeclareMathOperator{\Ima}{Im}

%--------------------------------------------------------
\begin{document}

\title{Algebra Conmutativa y Geometría Aplicada}
\author{Javier Aguilar Martín, Rafael González López}
\maketitle

\begin{ejercicio}{3}\
Sea $A \subseteq B$ una extensión entera de dominios de integridad.
 
\begin{enumerate}
\item Si
$$\mathfrak{p}_0 \subsetneq \mathfrak{p}_1 \subsetneq \dots \subsetneq \mathfrak{p}_r$$
es una cadena creciente de ideales primos en $A$, probar que existe una
cadena creciente
$$\mathfrak{q}_0 \subsetneq \mathfrak{q}_1 \subsetneq \dots \subsetneq \mathfrak{q}_r$$
de ideales primos en $B$ tales que $\mathfrak{p}_i = A \cap \mathfrak{q}_i$.
\item Si $\mathfrak{p} \subseteq A$ es un ideal primo y $\mathfrak{q}_1 \subseteq \mathfrak{q}_2 \subseteq B$ dos ideales primos de $B$ tales
que $\mathfrak{q}_1 \cap A = \mathfrak{q}_2 \cap A = \mathfrak{p}$, probar que $\mathfrak{q}_1 = \mathfrak{q}_2$.
\item Deducir de los apartados anteriores que si $f : X \to Y$ es un morfismo
finito y dominante entre variedades afines, entonces $\dim(Y ) = \dim(X)$.

\end{enumerate}
\end{ejercicio}
\begin{solucion}\

\begin{enumerate}
\item
\item
\item
\end{enumerate}
\end{solucion}

\end{document}