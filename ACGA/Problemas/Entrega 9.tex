\documentclass[twoside]{article}
\usepackage{../../estilo-ejercicios}
\DeclareMathOperator{\Ima}{Im}

%--------------------------------------------------------
\begin{document}

\title{Algebra Conmutativa y Geometría Aplicada}
\author{Javier Aguilar Martín, Rafael González López}
\maketitle

\begin{ejercicio}{2}
Sean $N,P$ submódulos de un $A$-módulo $M$, y $$(N:P) = \{a\in A\mid am\in N \;\forall m \in P\}$$
\begin{enumerate}
\item Probar que $(N:P)$ es un ideal de $A$.
\item Si $N=\{0\}$, $(N:M)$ se llama el anulador de M y se denota por $an(M)$. Probar que $M$ tiene una estructura natural de $(A/an(M))$-módulo.
\item Si $M,N$ son dos $A$-módulos, probar que $an(M+N)=an(M)\cap an(N)$.
\item Probar que $(N:P)=an((N+P)/N)$.
\end{enumerate}
\end{ejercicio}
\begin{solucion}\
\begin{enumerate}
\item[]
\item Se sigue de estas dos consideraciones:
\begin{itemize}
\item Veamos que es subgrupo con la suma. Sean $p,s\in (N:P)$ entonces $pm,sm\in N$ $\forall m \in P$. Como $N$ es submódulo, en particular es grupo abeliano, luego es cerrado para su operación. Se tiene entonces que $(p-s)m=pm-sm \in N$ $\forall m\in P$. 
\item Veamos que es cerrado para el producto con $A$. Sea $p\in (N:P)$, $s\in A$, entonces $pm\in N$ $\forall m\in P$. Por definición de módulo, si $s\in A$ y $p\in N$ entonces $sp\in N$, por lo que $(sp)m=s(pm)\in N$ $\forall m \in P$.
\end{itemize}
\newpage
\item Sabemos por el apartado anterior que $an(M)$ es un ideal de $A$, luego tiene sentido considerar el anillo cociente $A/an(M)$. Recordemos que:
$$
an(M) = \{a \in A\mid am = 0 \; \forall m\in M\}
$$
Si $M$ como $A$-módulo era el par $(M,\mu)$, entonces el $A/(an(M))$-módulo como el $(M,\sigma)$ donde $\sigma([a],x)=\mu(a,x)$. Vamos a ver que $\mu$ está bien definido y satisface los axiomas:
\begin{itemize}
\item Sean $a,b\in A$ tales que $[a]=[b]$ entonces $[a-b]=[0]$, luego $(a-b
)m=0$ $\forall m \in M$, de donde $am=bm$ $\forall m\in M$. Por tanto, $\sigma$ está bien definida.
\item Como $\mu$ satisface los axiomas de la definición de módulo, $\sigma$ los satisface trivialmente.
\end{itemize}
\item Directamente de la definición se tiene:
\begin{align*}
an(M+N) &= \{a \in A\mid as = 0 \; \forall s\in M+N\} \\
&= \{a \in A\mid a(n+m) = 0 \; \forall n\in N,\; \forall m\in M\}\\
&= \{a \in A\mid an= 0 \; \forall n\in N\}\cap \{a \in A\mid am = 0 \; \forall m\in M\}\\
&=an(M)\cap an(N)
\end{align*}
Si hubiera alguna duda en la penúltima igualdad, tengamos en cuenta que la contención hacia atrás es trivial, mientras que si $a(m+n)$, tomando $m=0$ con $n$ cualquiera y $n=0$ con $m$ cualquiera tenemos la conteción contraria.
\item Recordemos las definiciones:
\begin{align*}
(N:P) &= \{a\in A\mid am\in N \;\forall m \in P\}\\
an((N+P)/N)&= \{a \in A\mid a[m] = 0 \; \forall [m]\in (N+P)/N\}
\end{align*}

\end{enumerate}

\end{solucion}
\newpage
\begin{ejercicio}{3}\
Sea $0 \to N \to M \to P \to 0$ una sucesión exacta de $A-$módulos.
\begin{enumerate}
\item Si $N$ y $P$ son finitamente generados, probar que $M$ también lo es.
\item Si $M$ es finitamente generado, ¿lo es también $N$? ¿Y $P$?
\item Si $N$ y $P$ son noetherianos, probar que $M$ también lo es.
\item Si $M$ es noetheriano, probar que $N$ y $P$ también lo son.
\end{enumerate}
\end{ejercicio}
\begin{solucion}\
Vamos a utilizar la siguiente notación para los morfismos de la sucesión exacta $$0\to N  \overset{\psi}{\to} M  \overset{\phi}{\to} P \to 0$$ De la exactitud de la sucesión obtenemos de modo inmediato que $\psi$ es inyectiva y que $\phi$ es sobreyectiva. 
\begin{enumerate}
\item Supongamos que $N$ tiene generadores $n_1,\dots, n_r$ y $P$ tiene generadores $p_1,\dots, p_l$. Nótese que $\Ima\psi$ está de forma trivial finitamente generado, pues basta tomar las imágenes de sus generadores. Entonces podemos tomar $p_i=\phi(m_i), i=1,\dots, l$ para ciertos elementos $m_i\in M$. Sea $m\in M$, entonces existen $b_1,\dots, b_m\in A$ tales que $\phi(m)=\sum_{i=1}^lb_ip_i=\phi\left(\sum_{i=1}^lb_im_i\right)$, luego $m-\sum_{i=1}^lb_im_i\in\ker\phi=\Ima\psi$. Por lo que $m-\sum_{i=1}^lb_im_i=\sum_{j=1}^r a_j\psi(n_j)$ para ciertos $a_1,\dots,a_r\in A$. Por tanto basta despejar para obtejer que $m=\sum_{j=1}^r a_r\psi(n_r)+\sum_{i=1}^lb_im_i$, por lo que $M$ está finitamente generado por $\psi(n_1),\dots, \psi(n_r)$ y $m_1,\dots, m_l$.

\item Sea $A$ un anillo. Entonces $A$ es un $A$-módulo finitamente generado por su elemento unidad. En este caso, los submódulos coinciden con los ideales por definición. Luego basta dar un ejemplo de anillo con ideales no finitamente generados, pues si $A$ es un anillo y $B$ es un ideal, con la inclusión y la aplicación cociente tenemos de forma inmediata la sucesión exacta
$$0\to B \overset{i}{\hookrightarrow} A\overset{\pi}{\to} A/B\to 0$$
Tomemos entonces $A=\Z[x_1,x_2,\dots]$. Sea entonces $B=\langle x_1,x_2,\dots  \rangle$ el ideal generado por todas la variables. Evidentemente no puede ser generado por una cantidad finita de variables, pues cualquiera que excluyamos no podrá ser generada por el resto. 

Si $M$ es finitamente generado, entonces $P$ también lo es, pues podemos tomar un conjunto finito de generadores $\langle m_1,\dots, m_r\rangle = M$, de modo que $\langle \phi(m_1),\dots, \phi(m_r)\rangle =\phi(M)=P$, por lo que tenemos un conjunto finito de generadores de $P$.

\item Sea $M_0\subseteq M_1\subseteq\cdots$ una cadena ascendente de submódulos de $M$. Esta cadena induce sendas cadenas ascendentes $\psi^{-1}(M_0)\subseteq\psi^{-1}(M_1)\subseteq\cdots$ en $N$ y $\phi(M_0)\subseteq\phi(M_1)\subseteq\cdots$ en $P$. Por hipótesis ambas cadenas son estacionarias. 

Dados $n_1,n_2\in\N$ tales que $\psi^{-1}(M_{n_1})=\psi^{-1}(M_{n_1+1})=\cdots$ y $\phi(M_{n_2})=\phi(M_{n_2+1})=\cdots$, tomamos $n=\max\{n_1,n_2\}$. Consideramos entonces el siguiente diagrama conmutativo
\[
\begin{tikzcd}
0\arrow[r] & \psi^{-1}(M_{n})\arrow[r,"\psi"]\arrow[d, "Id"] & M_n\arrow[r,"\phi"]\arrow[d, hookrightarrow, "i"] & \phi(M_{n})\arrow[d,"Id"]\arrow[r] & 0\\
0\arrow[r] & \psi^{-1}(M_{n+1})\arrow[r,"\psi"]& M_{n+1}\arrow[r,"\phi"] & \phi(M_{n+1})\arrow[r] & 0
\end{tikzcd}
\]
Se tiene claramente que las filas son sucesiones exactas. 

Vamos a probar que $i:M_n\hookrightarrow M_{n+1}$ es sobreyectiva, lo cual implicará que en realidad es la identidad y por tanto la cadena será estacionaria (para un $n_0>n$ se podría hacer el mismo razonamiento y encontraríamos que efectivamente todos los módulos a partir de $n$ coinciden).

Sea $b'\in M_{n+1}$, entonces $\exists c=\phi(b')\in\phi(M_{n+1})=\phi(M_n)$. A su vez, $\exists b\in M_n$ tal que $\phi(b)=c=\phi(b')$, por lo que $b-b'\in\ker\phi=\Ima\psi$ por exactitud. Por tanto, $\exists a\in\psi^{-1}(M_{n+1})=\psi^{-1}(M_n)$ tal que $\psi(a)=b-b'$, luego $b'=b-\psi(a)=i(b-\psi(a))$, como queríamos demostrar.
 
  
\item Sea $N_0\subseteq N_1\subseteq\cdots$ una cadena ascendente de submódulos de $N$. Entonces, mediante $\psi$ da lugar a una cadena ascendente en $M$, $\psi(N_0)\subseteq \psi(N_1)\subseteq\cdots$, que debe ser estacionaria, es decir, $\exists n\in\N$ tal que $\psi(N_n)=\psi(N_{n+1})=\cdots$. Por la inyectividad de $\psi$ esto implica que $N_n=N_{n+1}=\cdots$, por lo que es estacionaria. 

Sea $P_0\subseteq P_1\subseteq\cdots$ una cadena ascendente de submódulos de $P$. Entonces, por la sobreyectividad de $\phi$ tenemos que esa cadena es igual a $\phi(M_0)\subseteq\phi(M_1)\cdots$ para alguna cadena creciente de submódulos en $M$. Por tanto, $\exists n\in\N$ tal que $M_n=M_{n+1}=\cdots$, con lo que $\phi(M_n)=\phi(M_{n+1})=\cdots$, lo cual significa que la cadena original era estacionaria. 
\end{enumerate}
\end{solucion}

\newpage
\end{document}