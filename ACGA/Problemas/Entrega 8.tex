\documentclass[twoside]{article}
\usepackage{../../estilo-ejercicios}
\DeclareMathOperator{\Ima}{Im}

%--------------------------------------------------------
\begin{document}

\title{Algebra Conmutativa y Geometría Aplicada}
\author{Javier Aguilar Martín, Rafael González López}
\maketitle

\begin{ejercicio}{5}\
\emph{La curva normal racional}. La imagen $X_d$ de la inmersión de
Veronese $\PP^1_k \to \PP^d_k$ para $n = 1$ se llama la \textbf{curva normal racional} en $\PP^d_k$.
\begin{enumerate}
\item Probar que $X_d$ es el conjunto de puntos $(x_0 : x_1 : \dots : x_d) \in \PP^d_k$ tales que
la matriz 
\[
\begin{pmatrix}
x_0 & x_1 &\cdots& x_{d-1}\\
x_1 & x_2 &\cdots& x_d
\end{pmatrix}
\]
tiene rango 1.
\item Probar que $X_d$ es la clausura proyectiva de la imagen del morfismo $\phi :\A^1_k \to \A^d_k$ dado por $\phi(t) = (t, t^2, \dots , t^d)$.
\item Probar que tres puntos distintos de $X_d$ nunca están en la misma recta.
\end{enumerate}
\end{ejercicio}
\begin{solucion}\

\end{solucion}
\newpage 

\end{document}