\documentclass[twoside]{article}
\usepackage{../estilo-ejercicios}

%--------------------------------------------------------
\begin{document}

\title{Algebra Conmutativa y Geometría Aplicada}
\author{Rafael González López}
\maketitle

\begin{ejercicio}{6}
Sea $Y\subset X \subset \mathbb{A}_k^n$ dos variedades. Definimos $\mathcal{K}(X,Y)$ como el conjunto de funciones racionales $f \in \mathcal{K}(X)$ que están definidas en al menos un punto de $Y$. Probar que $\mathcal{K}(X,Y)$ es un anillo local con cuerpo residual $\mathcal{K}(Y)$.
\begin{solucion}
Como $\mathcal{K}(X,Y)\subset \mathcal{K}(X)$, simplemente vamos a probar que es subanillo usando la caracterización de los mismos. Sean $(U,f),(V,g)\in \mathcal{K}(X,Y)$ entonces $\exists U'\subset U\cap Y$, $V'\subset V\cap Y$ abiertos de Zariski no vacíos y $p,p',q,q'\in\mathcal{A}(Y)$ con $q(x)\neq 0$ $\forall x \in U'$ y $q'(x)\neq 0$ $\forall x \in V'$ tales $f(x)=\dfrac{p(x)}{q(x)}$ $\forall x\in U'$ y $g(x)=\dfrac{p'(x)}{q'(x)}$ $\forall x\in V'$. 

Sabemos que $(U,f)\ast (V,g) = (U\cap V, f_{U\cap V}\ast g_{U\cap V})$, donde $\ast$ representa tanto $(-)$ como $(\cdot)$. Tenemos que probar que $(U,f)\ast (V,g) \in \mathcal{K}(X,Y)$. Sabemos que $U'\cap V' \neq \emptyset$, pues los abiertos de Zariski son densos, y que $U'\cap V' \subset U\cap V$. Entonces, $\forall x \in U'\cap V'$ $q(x)q'(x)\neq 0$, por lo que en todo punto de dicho abierto, la función $\dfrac{p(x)}{q(x)}\ast\dfrac{p'(x)}{q'(x)}$ está bien definida. En particular, $(U,f)\ast (V,g)$ está bien definida en algún punto de $Y$, lo que basta para ver que $\mathcal{K}(X,Y)$ es subanillo.

Sea el ideal definido como
$$
I = \{f \in \mathcal{K}(X,Y) \mid \text{$f(x)=0$ $\forall x \in Y$ donde esté definido}\}
$$
Claramente $I$ es un ideal, pues es trivialmente grupo con la suma y el producto de un elemento de $\mathcal{K}(X,Y)$ y un elemento de $f$ sigue perteneciendo a $Y$. 

Sea $f\in \mathcal{K}(X,Y)-I$, entonces $\exists x \in Y$ tal que $f$ está definido en $Y$ y $f(x)\neq 0$. Por tanto, el abierto definido $f(x)\neq 0$ tiene intersección no vacía con $Y$, y por ende, su inversa estará bien definida en algún punto de $Y$, por lo que $f^{-1}\in \mathcal{K}(X,Y)$. Esto implica que $I$ es un ideal maximal, pues si no lo fuera, $\exists m$ ideal maximal con $I \subsetneq m$, y  $f\in \mathcal{K}(X,Y)-I$ tal que $f\in m$, pero como $f$ es unidad, $m = \mathcal{K}(X,Y)$. Además, $\mathcal{K}(X,Y)$ debe ser local, pues si existiese otro ideal maximal, debería contener algún elemento de $\mathcal{K}(X,Y)-I$.

Finalmente, veamos que $\mathcal{K}(X,Y)/I \cong \mathcal{K}(Y)$. Consideremos el siguiente homeomorfismo de anillos
\begin{gather*}
\phi\func{\mathcal{K}(X,Y)}{\mathcal{K}(Y)}\\
(U,f)\mapsto (U\cap Y,f)
\end{gather*}
Obviamente $f(1)=1$ y 
\begin{gather*}
\phi((U,f)\ast (V,g)) = \phi((U\cap V, f\ast g)) = (U\cap V \cap Y, f\ast g) = ((U\cap Y)\cap (V\cap Y), f \ast g) =\\
(U\cap Y, f) \ast (V\cap Y, g) = \phi((U,f))\ast \phi((V,f)) 
\end{gather*}
Además, $(U,f)\in \ker(\phi)$ si y solo si $f(x)=0$ $\forall x \in U \cap Y$, lo cual es exactamente decir que $\ker(\phi)=I$.r
\end{solucion}
\end{ejercicio}
\end{document}