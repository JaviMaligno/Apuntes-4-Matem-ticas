\documentclass[twoside]{article}
\usepackage{../../estilo-ejercicios}
\DeclareMathOperator{\Ima}{Im}

%--------------------------------------------------------
\begin{document}

\title{Algebra Conmutativa y Geometría Aplicada}
\author{Javier Aguilar Martín, Rafael González López}
\maketitle

\begin{ejercicio}{1}\
Sean $X, Y,Z$ variedades cuasi-proyectivas, y $\phi : X \dashrightarrow Y$ y $\psi : Y \dashrightarrow Z$ aplicaciones racionales.
\begin{enumerate}
\item Probar que si $(U, \tilde{\phi})$ y $(V, \tilde{\phi}')$ (donde $U, V \subseteq X$ son abiertos densos) son
dos representaciones de $\phi$, entonces $\tilde{\phi}$ es dominante si y solo si $\tilde{\phi}'$ lo es, y
que por tanto la definición de aplicación racional dominante tiene sentido.
\item Probar que si $\phi$ y $\psi$ son dominantes, la composición  $\psi\circ\phi$ está bien definida
(en el sentido de que existen representaciones de $\phi$ y $\psi$ que se pueden
componer, y la composición no depende de las representaciones elegidas)
y que la composición es también dominante.
\end{enumerate}
\end{ejercicio}
\begin{solucion}\
\begin{enumerate}
\item Para probar este apartado se usará el siguiente resultado de topología general.
\begin{lemma} \label{lema1}
Dada una aplicación continua $f:X\to Y$, para cualquier subconjunto $A\subseteq X$, $\overline{f(\overline{A})}=\overline{f(A)}$. 
\end{lemma}
\begin{proof}
Por monotonía de la clausura se tiene $\overline{f(A)}\subseteq\overline{f(\overline{A})}$. Para la otra inclusión, recordemos que dado $A\subseteq X$, $f$ es continua si y solo si $f(\overline{A})\subseteq\overline{f(A)}$, por lo que basta volver a tomar clausura para obtener $\overline{f(\overline{A})}\subseteq\overline{f(A)}$.
\end{proof}
 Por la definición de función racional tenemos que $\tilde{\phi}|_{U\cap V}=\tilde{\phi}'|_{U\cap V}$. Supongamos que $\tilde{\phi}$ es dominante. Entonces $\overline{\tilde{\phi}'(U\cap V)}=\overline{\tilde{\phi}(U\cap V)}=\overline{\tilde{\phi}(\overline{U\cap V})}$, ya que los morfismos son continuos en la topología de Zariski y por tanto podemos aplicar el lema \ref{lema1}. Usando ahora que los abiertos no vacíos de Zariski son densos y la intersección de dos cualesquiera de ellos también es denso, obtenemos $\overline{\tilde{\phi}(\overline{U\cap V})}=\overline{\tilde{\phi}(X)}=Y,$ por ser $\tilde{\phi}$ dominante. En definitiva, $\overline{\tilde{\phi}'(U\cap V)}=Y$, por lo que claramente $\tilde{\phi}'$ es dominante, ya que $\overline{\tilde{\phi}'(V)}\supseteq\overline{\tilde{\phi}'(U\cap V)}=Y$. El recíproco es análogo.
 \newpage
\item Sean $(U,\tilde{\phi}),(V,\tilde{\psi})$ representaciones de $\phi$ y $\psi$ respectivamente. Claramente la composición $\tilde{\psi}\circ\tilde{\phi}$ está bien definida en el abierto denso $W=U\cap\tilde{\phi}^{-1}(V)$ (lo es por ser intersección de abiertos densos, gracias a la continuidad de $\tilde{\phi}$), luego basta tomar las representaciones $(W,\tilde{\phi}|_W),(V,\tilde{\psi})$. Nótese que la composición está bien definida pues $\tilde{\phi}(W)\subseteq \tilde{\phi}(\tilde{\phi}^{-1}( V))\subseteq V$.

Veamos que la composición no depende de las representaciones elegidas. Sean $(U,\tilde{\psi}\circ\tilde{\phi}),(V,\hat{\psi}\circ\hat{\phi})$ dos representaciones de la composición. Tenemos, por ser $\phi$ racional, que $\tilde{\phi}|_{U\cap V}=\hat{\phi}|_{U\cap V}$. En particular esto implica que $\tilde{\phi}(U\cap V)= \hat{\phi}(U\cap V)$. Sean $\tilde{W}$ y $\hat{W}$ dominios abiertos de $\tilde{\psi}$ y $\hat{\psi}$ respectivamente tales que $\tilde{\phi}(U\cap V)\subseteq\tilde{W},\hat{\phi}(U\cap V)\subseteq\hat{W}$. Dichos abiertos existen puesto que ambos morfismos deben estar definidos en abiertos no vacíos, los cuales deben contener a $\tilde{\phi}(U\cap V)=\hat{\phi}(U\cap V)$ puesto que la composición está bien definida.  Entonces, por ser $\psi$ racional, $\tilde{\psi}|_{\tilde{W}\cap\hat{W}}=\hat{\psi}|_{\tilde{W}\cap\hat{W}}$, luego en particular $\tilde{\psi}|_{\tilde{\phi}(U\cap V)}=\hat{\psi}|_{\hat{\phi}(U\cap V)}$, es decir, $$\tilde{\psi}\circ\tilde{\phi}|_{U\cap V}=\hat{\psi}\circ\hat{\phi}|_{U\cap V}$$ como queríamos demostrar.

Para ver que la composición es dominante, sea $(W,\tilde{\psi}\circ\tilde{\phi})$ un representación de $\psi\circ\phi$. Veamos primero el siguiente resultado.

\begin{lemma}\label{lema2}
Un morfismo $g:X\dashrightarrow Y$ es dominante si y solo si $g^{-1}(U)\neq\emptyset\ \forall U\subseteq Y$ abierto denso.
\end{lemma}
\begin{proof}
Supongamos que $g$ es dominante y sea $U$ un abierto denso. Por reducción al absurdo, supongamos que $g^{-1}(U)=\emptyset$. Entonces $\Im{g}\subseteq Y\setminus U$, que es cerrado por ser complementario de un abierto, luego es igual a su clausura, que no es el total.

Recíprocamente, supongamos que $g^{-1}(U)\neq\emptyset\ \forall U\subseteq Y$. Si, $g$ no fuera dominante, entonces existiría un abierto no vacío $V\subseteq Y$ tal que $\Im{g}\subseteq Y\setminus V$, pero entonces $g^{-1}(V)=\emptyset$, lo cual contradice la hipótesis.

\end{proof}

 Sea $U\subseteq Z$ un abierto denso, entonces aplicando el lema \ref{lema2}, $V=\tilde{\psi}^{-1}(U)\neq\emptyset$ por ser $\tilde{\psi}$ dominante y abierto por ser continua.  Por el mismo motivo, $\tilde{\phi}^{-1}(V)$ es un abierto no vacío. En definitiva, $(\tilde{\psi}\circ\tilde{\phi})^{-1}(U)\neq\emptyset$, por lo que la aplicación es dominante utilizando el lema \ref{lema2}.

\end{enumerate}
\end{solucion}

\newpage

\begin{ejercicio}{2} Sea $X\subset \A^3_\C$ la cuádrica definida por la ecuación $x^2 + yz =1$. Construir una aplicación birracional entre $X$ y $\A^2_\C$, probando que $X$ es racional. Determinar dos abiertos $U\subset X$ y $V\subset \A^2_\C$ tales que la aplicación birracional definida anteriormente induzca un isomorfismo entre ellos.
\end{ejercicio}
\begin{solucion}
\end{solucion}
\newpage 

\end{document}