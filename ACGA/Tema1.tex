\documentclass[ACGA.tex]{subfiles}
\begin{document}

\chapter{Conjuntos algebraicos afines. Anillos de coordenadas. Topología de Zariski. Anillos locales.}
Vamos a trabajar en el espacio afín de dimensión n $\A^n$ sobre un cuerpo $\K$ fijo. 
\begin{defi}
Un subcojunto $X\subset \A^n$ es un \textbf{conjunto algebraico} si existe un conjunto $S\subset \K[x_1,\dotsc,x_n]$ tal que $X$ es el conjunto de los ceros comunes de los polinomios de S. Es decir:
\[
X=\{(x_1,\dotsc,x_n)\in\A^n\mid f(x_1,\dotsc,x_n)=0\;\forall f\in S\}=V(S)
\]
\end{defi}
\begin{prop}Veamos algunas propiedades de los conjuntos algebraicos:
\begin{itemize}
\item En $\A^1$, los conjuntos algebraicos son finitos, salvo $X=\A^1$ ($S=\{0\}$). El recíproco también es cierto, es decir, cualquier conjunto finito es algebraico.
\item En general $V(\{0\})=\A^n$ y $V(\{1\})=\emptyset$. Además, los conjuntos X unitarios siempre son algebraicos, pues tomamos $S=\{x_i-a_i\mid i=1,\dotsc,n\}$.
\item En $\A^2$, por ejemplo, $X=\{(a,b)\} = V(\{x-a,y-b\})$ y en general no tiene por qué existir un conjunto S unitario tal que $V(S)=X$. Si tenemos $X=\{(a,b),(c,d)\}$ entonces $S=\{(x-a)(x-c),(y-b)(y-d), (x-a)(y-d), (y-b)(x-c)\}$. En general, podemos generalizarlo a n puntos considerando todas las combinaciones 2 a 2 obteniendo $2^n$ polinomios.
\end{itemize}
\end{prop}
\begin{prop}
Si X e Y son algebraicos tales que $X=V(S)$ e $Y=V(T)$ entonces $X\cap Y = V(T\cup S)$. Además si $S\cdot T = \{f\cdot g \mid f \in S,\; g \in T\}$ entonces $X\cup Y = V(S\cdot T)$. 
\end{prop}
\begin{dem}
Todas las inclusiones son sencillas. Veamos una. Sea $x\notin X\cup Y$, entonces $x\notin X$ y $x\notin Y$. Entonces existen $f\in S$ y $g\in T$ tal que $f(x)\neq 0$ y $g(x)\neq 0$. Se sigue $f(x)g(x)\neq 0$, luego $x\notin V(S\cdot T)$. 
\end{dem}

\begin{defi}
Los conjuntos algebraicos de $\A^n$ son los cerrados de una topología llamada \textbf{Topología de Zariski}.
\end{defi}
\begin{prop}
En general $V(S)=V(\langle S\rangle)$ $\forall S \subset  \K[x_1,\dotsc,x_n]$.
\end{prop}
\begin{dem}
Sabemos que en general si $S\subset T$ entonces $V(T)\subset V(S)$, por lo que tomando $T=\langle S \rangle$, $V(\langle S\rangle)\subset V(S)$. Además, si $x\in V(S)$ entonces $f(x)$ se anula $\forall f \in S$, por lo que cualquier combinación de la forma
\[
g_1 f_1(x)+ \dotsc + g_rf_r(x) = 0 \quad \forall f_i \in S
\]
Por lo que $x\in V(\langle S\rangle)$.
\end{dem}
\begin{defi}
Un anillo se dice \textbf{noetheriano} si todo ideal es finitamente generado.
\end{defi}
\begin{prop}
Son anillos noetherianos:
\begin{itemize}
\item Los anillos finitos
\item Los dominios de ideales principales (y por tanto los cuerpos).
\end{itemize}
\end{prop}
\begin{nota}
Son anillos no noetherianos:
\begin{itemize}
	\item $k[x_1,x_2,\dots,] = \bigcup_{n \geq 1} k[x_1,\dots,x_n]$.
	\item El anillo de funciones continuas $\{f \func{\R}{\R}\}$.
\end{itemize}
\end{nota}
\begin{prop}\mbox{}
\begin{enumerate}
	\item $A$ es noetheriano.
	\item Toda cadena creciente $I_1 \subseteq I_2 \subseteq I_3 \subseteq \dots$ de ideales es estacionara, es decir, existe $r_0$ tal que $I_r = I_{r_0} \forall r \geq r_0$.
\end{enumerate}
\end{prop}

\begin{dem}\mbox{}
\begin{itemize}
	\item[($1\Rightarrow 2$)]
	Consideramos la cadena creciente de ideales:
	\[ I_1 \subseteq I_2 \subseteq I_3 \subseteq \dots \subseteq I_r \subseteq \dots \]
	Tomamos $I = \bigcup_{n\geq 1} I_n$. Tenemos que $I$ es un ideal, por lo tanto está finitamente generado como $I = \langle a_1, \dots, a_r \rangle$. Cada $a_i \in I_{n_i}$ para algún $n_i$. Entonces $a_i \in I_n$  $\forall i$, si $n > \max\{n_1,\dots,n_r\}$. Llegamos entonces a que $I \subseteq I_n \subseteq I$, es decir $I_n = I$. Por lo tanto $I_{n+k} = I$ $\forall k \geq 0$. 

	\item[($2\Rightarrow 1$)]
	Sea $I \subseteq A$ y $x_1 \in I$. Entonces $\langle x_1 \rangle \subseteq I \subseteq A$. Si $I = \langle x_1 \rangle$, hemos acabado. Si no $x_2 \in I \setminus \langle x_1 \rangle$, tomamos $\langle x_1, x_2 \rangle \subseteq I \subseteq A$. Si $I = \langle x_1, x_2 \rangle$, hemos acabado. Si no, repetimos el argumento tomando un $x_3 \in I \setminus \langle x_1,x_2 \rangle$. Como la cadena creciente debe ser estacionaria, llegamos a que en algún momento, la cadena de ideales $\langle x_1,\dots,x_n\rangle$ debe estabilizarse y deducimos que $I = \langle x_1,\dots,x_n\rangle$. Por lo tanto $A$ es noetheriano.

\end{itemize}
\end{dem}

\begin{teorema}[de la base de Hilbert]
Si $A$ es un anillo noetheriano, entonces $A[x]$ también lo es.
\end{teorema}

Por lo tanto, si $A$ es un anillo noetheriano, entonces $A[x_1,\dots,x_n]$ también lo es. Llegamos además a que todo conjunto algebraico se puede definir como el conjunto de ceros de un número finito de polinomios:

\[ V(S) = V(\langle S \rangle) = V(\langle f_1,\dots,f_r\rangle) =  V(\{f_1,\dots,f_r\}) \]

\begin{defi}
Decimos que un cuerpo $\K$ es \textbf{algebraicamente cerrado} si todo polinomio no constante en $\K[x]$ tiene una raíz en $k$.
\end{defi}

\begin{defi}
Definimos el \textbf{radical de un ideal} $I$ como:
\[ \sqrt{I} = \text{rad}(I) = \{f \in k[x_1,\dots,x_n] \mid f^n \in I \text{ para algún }n \in \mathbb{N}\} \]
\end{defi}

\begin{propi}\mbox{}
\begin{itemize}
	\item $\sqrt{I}$ es un ideal (ejercicio).
	\item $I \subset \sqrt{I}$.
	\item $\sqrt{\sqrt{I}} = \sqrt{I}$.
	\item $V(I) = V(\sqrt{I})$.
\end{itemize}
\end{propi}

\begin{nota}
En cuerpos algebraicamente cerrado, hay una correspondencia entre los conjuntos algebraicos (en el contexto de la geometría) y los ideales radicales (en el contexto del álgebra). Nótese que $(x-a)$ y $(x-a)^n$ tienen las mismas raíces para $a\in\mathbb{K}$ y $n\in\mathbb{N}$. 
\end{nota}

\begin{ej}
Por ejemplo, en $\mathbb{K}[x]$ y $\mathbb{A}^1$, tenemos como conjuntos algebraicos los conjuntos finitos más $\mathbb{A}$ (idea geométrica). Esto se corresponde en el álegebra con un ideal radical $I$, que se corresponde con un polinomio mónico $g$. Es decir, $I=\langle g\rangle$ e $I=\sqrt{I}$. Esto último ocurre si y solo si dado $f\in I$ cumple que  si $g\mid f^n$ para algún $n\geq 1$ entonces $g\mid f$. Esto se cumple si $g$ es irreducible (o sea, el ideal es primo). En general, si $I$ es primo entonces $I$ es radical. Más generalmente, la divisibilidad se cumple si y solo si $g$ es producto de irreducibles distintos entre sí, es decir, el ideal $I$ es primo si y solo si $g$ es producto de polinomios irreducibles distintos.\\
Si tenemos el conjunto $A=\{\alpha_1,\dots, \alpha_r\}$, le hacemos corresponder el ideal generado por $h=(x-\alpha_1)\cdots(x-\alpha_r)$. 
\end{ej}

Nos falta ver que si $I,J$ son dos ideales radicales distintos, entonces $\mathcal{V}(I)\neq \mathcal{V}(J)$, con lo que tendríamos una biyección entre ideales radicales y conjuntos algebraicos. Dado un subconjunto $X\subseteq \mathbb{A}^n$,  definimos $\mathcal{I}(X)=\{f\in\mathbb{K}[x]\mid f(x)=0\ \forall x\in X\}$. Es trivial probar que $\mathcal{I}(X)$ es ideal. Tenemos que probar que es radical. Suponemos que $f^n\in\mathcal{I}(X)$, por lo que $f^n(x)=0\ \forall x\in X$. Como estamos en un cuerpo esto implica que $f(x)=0\ \forall x\in X$, luego $f\in\mathcal{I}(X)$. 

Veamos que $\mathcal{I}$ y $\mathcal{V}$ inducen una biyección entre el cojunto de subconjuntos algebraicos de $\mathbb{A}^n$ y el conjunto de ideales radicales de $\mathbb{K}[x_1,\dots,x_n]$. Sea $X$ un conjunto algebraico y $S$ el conjunto de polinomios correspondiente, vamos a comprobar que $\mathcal{V}(\mathcal{I}(X))=X$ y que $\mathcal{I}(\mathcal{V}(S))=S$. Claramente por definición $\mathcal{V}(\mathcal{I}(X))\supseteq X$. Por otro lado, si tomamos un conjunto de polinomios $S$, por definición $S\subseteq \mathcal{I}(\mathcal{V}(S))$. Para la otra contención, si $X$ es algebraico, entonces $X=\mathcal{V}(R)$ para algún conjunto de polinomios $R$. Utilizando que $\mathcal{I}(\mathcal{V}(R))\supseteq R$, se deduce aplicando $\mathcal{V}$ a ambos lados que $\mathcal{V}(\mathcal{I}(X))\subseteq \mathcal{V}(R)=X$. Nos queda por probar una contención. Spongamos que $\mathcal{V}(I)=\emptyset=\mathcal{V}(\langle 1\rangle)$. El ideal radical correspondiente sería $I=\sqrt{I}=\langle1 \rangle$. Equivalentemente, si $I\neq\langle 1\rangle$, entonces $\mathcal{V}(I)\neq\emptyset$. A continuación se va a usar la siguiente proposición:
\begin{prop}
Si $k\subset K$ es una extensión de cuerpos y $K$ es finitamente generado como $k-$álgebra, entonces $K$ es una extensión finita de $k$. 
\end{prop}

\begin{ejer}
Probar que $\Q$ no es $\Z-$álgebra finitamente generada. 
\end{ejer}

Continuando con lo anterior, vamos a suponer $I\neq\langle 1\rangle$. Sabemos que $I\subset M$ ideal maximal. Así que construimos el cuerpo $K=\mathbb{K}[x_1,\dots,x_n]/M$ que es una extensión del cuerpo $\mathbb{K}$, finitamente generada por $\overline{x}_1,\dots, \overline{x}_n$, así que es extensión finita. Como $\mathbb{K}$ es algebraicamente cerrado, su extensión algebraica es él mismo, así que $\mathbb{K}\cong K$ mediante el homomorfismo inducido por la inclusión. Recordemos que $\mathbb{A}^n=\mathbb{K}^n$. Si tomamos $f\in I$ y $(\overline{x}_1,\dots, \overline{x}_n)\in K$, etonces $f(\overline{x}_1,\dots, \overline{x}_n)=\overline{f(x_1,\dots,x_n)}=\overline{0}\in K$.
 


\end{document}
