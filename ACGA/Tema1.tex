\documentclass[ACGA.tex]{subfiles}
\begin{document}

\chapter{Conjuntos algebraicos afines. Anillos de coordenadas. Topología de Zariski. Anillos locales.}
Vamos a trabajar en el espacio afín de dimensión n $\A^n$ sobre un cuerpo $\K$ fijo. 
\section{Conjuntos Algebraicos.}
\begin{defi}
Un subcojunto $X\subset \A^n$ es un \textbf{conjunto algebraico} si existe un conjunto $S\subset \K[x_1,\dotsc,x_n]$ tal que $X$ es el conjunto de los ceros comunes de los polinomios de S. Es decir:
\[
X=\{(x_1,\dotsc,x_n)\in\A^n\mid f(x_1,\dotsc,x_n)=0\;\forall f\in S\}=V(S)
\]
\end{defi}
\begin{prop}Veamos algunas propiedades de los conjuntos algebraicos:
\begin{itemize}
\item En $\A^1$, los conjuntos algebraicos son finitos, salvo $X=\A^1$ ($S=\{0\}$). El recíproco también es cierto, es decir, cualquier conjunto finito es algebraico.
\item En general $V(\{0\})=\A^n$ y $V(\{1\})=\emptyset$. Además, los conjuntos X unitarios siempre son algebraicos, pues tomamos $S=\{x_i-a_i\mid i=1,\dotsc,n\}$.
\item En $\A^2$, por ejemplo, $X=\{(a,b)\} = V(\{x-a,y-b\})$ y en general no tiene por qué existir un conjunto S unitario tal que $V(S)=X$. Si tenemos $X=\{(a,b),(c,d)\}$ entonces $S=\{(x-a)(x-c),(y-b)(y-d), (x-a)(y-d), (y-b)(x-c)\}$. En general, podemos generalizarlo a n puntos considerando todas las combinaciones 2 a 2 obteniendo $2^n$ polinomios.
\end{itemize}
\end{prop}
\begin{prop}
Si X e Y son algebraicos tales que $X=V(S)$ e $Y=V(T)$ entonces $X\cap Y = V(T\cup S)$. Además si $S\cdot T = \{f\cdot g \mid f \in S,\; g \in T\}$ entonces $X\cup Y = V(S\cdot T)$. 
\end{prop}
\begin{dem}
Todas las inclusiones son sencillas. Veamos una. Sea $x\notin X\cup Y$, entonces $x\notin X$ y $x\notin Y$. Entonces existen $f\in S$ y $g\in T$ tal que $f(x)\neq 0$ y $g(x)\neq 0$. Se sigue $f(x)g(x)\neq 0$, luego $x\notin V(S\cdot T)$. \QED
\end{dem}

\begin{defi}
Los conjuntos algebraicos de $\A^n$ son los cerrados de una topología llamada \textbf{Topología de Zariski}.
\end{defi}
\begin{prop}
En general $V(S)=V(\langle S\rangle)$ $\forall S \subset  \K[x_1,\dotsc,x_n]$.
\end{prop}
\begin{dem}
Sabemos que en general si $S\subset T$ entonces $V(T)\subset V(S)$, por lo que tomando $T=\langle S \rangle$, $V(\langle S\rangle)\subset V(S)$. Además, si $x\in V(S)$ entonces $f(x)$ se anula $\forall f \in S$, por lo que cualquier combinación de la forma
\[
g_1 f_1(x)+ \dotsc + g_rf_r(x) = 0 \quad \forall f_i \in S
\]
Por lo que $x\in V(\langle S\rangle)$.\QED
\end{dem}
\begin{defi}
Un anillo se dice \textbf{noetheriano} si todo ideal es finitamente generado.
\end{defi}
\begin{prop}
Son anillos noetherianos:
\begin{itemize}
\item Los anillos finitos
\item Los dominios de ideales principales (y por tanto los cuerpos).
\end{itemize}
\end{prop}
\begin{nota}
Son anillos no noetherianos:
\begin{itemize}
	\item $k[x_1,x_2,\dots,] = \bigcup_{n \geq 1} k[x_1,\dots,x_n]$.
	\item El anillo de funciones continuas $\{f \func{\R}{\R}\}$.
\end{itemize}
\end{nota}
\begin{prop}\mbox{}
\begin{enumerate}
	\item $A$ es noetheriano.
	\item Toda cadena creciente $I_1 \subseteq I_2 \subseteq I_3 \subseteq \dots$ de ideales es estacionara, es decir, existe $r_0$ tal que $I_r = I_{r_0} \forall r \geq r_0$.
\end{enumerate}
\end{prop}

\begin{dem}\mbox{}
\begin{itemize}
	\item[($1\Rightarrow 2$)]
	Consideramos la cadena creciente de ideales:
	\[ I_1 \subseteq I_2 \subseteq I_3 \subseteq \dots \subseteq I_r \subseteq \dots \]
	Tomamos $I = \bigcup_{n\geq 1} I_n$. Tenemos que $I$ es un ideal, por lo tanto está finitamente generado como $I = \langle a_1, \dots, a_r \rangle$. Cada $a_i \in I_{n_i}$ para algún $n_i$. Entonces $a_i \in I_n$  $\forall i$, si $n > \max\{n_1,\dots,n_r\}$. Llegamos entonces a que $I \subseteq I_n \subseteq I$, es decir $I_n = I$. Por lo tanto $I_{n+k} = I$ $\forall k \geq 0$. 

	\item[($2\Rightarrow 1$)]
	Sea $I \subseteq A$ y $x_1 \in I$. Entonces $\langle x_1 \rangle \subseteq I \subseteq A$. Si $I = \langle x_1 \rangle$, hemos acabado. Si no $x_2 \in I \setminus \langle x_1 \rangle$, tomamos $\langle x_1, x_2 \rangle \subseteq I \subseteq A$. Si $I = \langle x_1, x_2 \rangle$, hemos acabado. Si no, repetimos el argumento tomando un $x_3 \in I \setminus \langle x_1,x_2 \rangle$. Como la cadena creciente debe ser estacionaria, llegamos a que en algún momento, la cadena de ideales $\langle x_1,\dots,x_n\rangle$ debe estabilizarse y deducimos que $I = \langle x_1,\dots,x_n\rangle$. Por lo tanto $A$ es noetheriano. \QED

\end{itemize}
\end{dem}

\begin{teorema}[de la base de Hilbert]
Si $A$ es un anillo noetheriano, entonces $A[x]$ también lo es.
\end{teorema}

Por lo tanto, si $A$ es un anillo noetheriano, entonces $A[x_1,\dots,x_n]$ también lo es. Llegamos además a que todo conjunto algebraico se puede definir como el conjunto de ceros de un número finito de polinomios:

\[ V(S) = V(\langle S \rangle) = V(\langle f_1,\dots,f_r\rangle) =  V(\{f_1,\dots,f_r\}) \]

\begin{defi}
Decimos que un cuerpo $\K$ es \textbf{algebraicamente cerrado} si todo polinomio no constante en $\K[x]$ tiene una raíz en $k$.
\end{defi}

\begin{defi}
Definimos el \textbf{radical de un ideal} $I$ como:
\[ \sqrt{I} = \text{rad}(I) = \{f \in k[x_1,\dots,x_n] \mid f^n \in I \text{ para algún }n \in \mathbb{N}\} \]
\end{defi}

\begin{propi}\mbox{}
\begin{itemize}
	\item $\sqrt{I}$ es un ideal (ejercicio).
	\item $I \subset \sqrt{I}$.
	\item $\sqrt{\sqrt{I}} = \sqrt{I}$.
	\item $V(I) = V(\sqrt{I})$.
\end{itemize}
\end{propi}

\begin{nota}
En cuerpos algebraicamente cerrado, hay una correspondencia entre los conjuntos algebraicos (en el contexto de la geometría) y los ideales radicales (en el contexto del álgebra). Nótese que $(x-a)$ y $(x-a)^n$ tienen las mismas raíces para $a\in\mathbb{K}$ y $n\in\mathbb{N}$. 
\end{nota}

\begin{ej}
Por ejemplo, en $\mathbb{K}[x]$ y $\mathbb{A}^1$, tenemos como conjuntos algebraicos los conjuntos finitos más $\mathbb{A}$ (idea geométrica). Esto se corresponde en el álegebra con un ideal radical $I$, que se corresponde con un polinomio mónico $g$. Es decir, $I=\langle g\rangle$ e $I=\sqrt{I}$. Esto último ocurre si y solo si dado $f\in I$ cumple que  si $g\mid f^n$ para algún $n\geq 1$ entonces $g\mid f$. Esto se cumple si $g$ es irreducible (o sea, el ideal es primo). En general, si $I$ es primo entonces $I$ es radical. Más generalmente, la divisibilidad se cumple si y solo si $g$ es producto de irreducibles distintos entre sí, es decir, el ideal $I$ es primo si y solo si $g$ es producto de polinomios irreducibles distintos.\\
Si tenemos el conjunto $A=\{\alpha_1,\dots, \alpha_r\}$, le hacemos corresponder el ideal generado por $h=(x-\alpha_1)\cdots(x-\alpha_r)$. 
\end{ej}

Nos falta ver que si $I,J$ son dos ideales radicales distintos, entonces $\mathcal{V}(I)\neq \mathcal{V}(J)$, con lo que tendríamos una biyección entre ideales radicales y conjuntos algebraicos. Dado un subconjunto $X\subseteq \mathbb{A}^n$,  definimos $\mathcal{I}(X)=\{f\in\mathbb{K}[x]\mid f(x)=0\ \forall x\in X\}$. Es trivial probar que $\mathcal{I}(X)$ es ideal. Tenemos que probar que es radical. Suponemos que $f^n\in\mathcal{I}(X)$, por lo que $f^n(x)=0\ \forall x\in X$. Como estamos en un cuerpo esto implica que $f(x)=0\ \forall x\in X$, luego $f\in\mathcal{I}(X)$. 

Veamos que $\mathcal{I}$ y $\mathcal{V}$ inducen una biyección entre el conjunto de subconjuntos algebraicos de $\mathbb{A}^n$ y el conjunto de ideales radicales de $\mathbb{K}[x_1,\dots,x_n]$. Sea $X$ un conjunto algebraico y $S$ el conjunto de polinomios correspondiente, vamos a comprobar que $\mathcal{V}(\mathcal{I}(X))=X$ y que $\mathcal{I}(\mathcal{V}(S))=S$. Claramente por definición $\mathcal{V}(\mathcal{I}(X))\supseteq X$. Por otro lado, si tomamos un conjunto de polinomios $S$, por definición $S\subseteq \mathcal{I}(\mathcal{V}(S))$. Para la otra contención, si $X$ es algebraico, entonces $X=\mathcal{V}(R)$ para algún conjunto de polinomios $R$. Utilizando que $\mathcal{I}(\mathcal{V}(R))\supseteq R$, se deduce aplicando $\mathcal{V}$ a ambos lados que $\mathcal{V}(\mathcal{I}(X))\subseteq \mathcal{V}(R)=X$. Nos queda por probar una contención. Spongamos que $\mathcal{V}(I)=\emptyset=\mathcal{V}(\langle 1\rangle)$. El ideal radical correspondiente sería $I=\sqrt{I}=\langle1 \rangle$. Equivalentemente, si $I\neq\langle 1\rangle$, entonces $\mathcal{V}(I)\neq\emptyset$. A continuación se va a usar la siguiente proposición:
\begin{prop}
Si $k\subset K$ es una extensión de cuerpos y $K$ es finitamente generado como $k-$álgebra, entonces $K$ es una extensión finita de $k$. 
\end{prop}

\begin{ejer}
Probar que $\Q$ no es $\Z-$álgebra finitamente generada. 
\end{ejer}

Continuando con lo anterior, vamos a suponer $I\neq\langle 1\rangle$. Sabemos que $I\subset M$ ideal maximal. Así que construimos el cuerpo $K=\mathbb{K}[x_1,\dots,x_n]/M$ que es una extensión del cuerpo $\mathbb{K}$, finitamente generada por $\overline{x}_1,\dots, \overline{x}_n$, así que es extensión finita. Como $\mathbb{K}$ es algebraicamente cerrado, su extensión algebraica es él mismo, así que $\mathbb{K}\cong K$ mediante el homomorfismo inducido por la inclusión. Recordemos que $\mathbb{A}^n=\mathbb{K}^n$. Si tomamos $f\in I$ y $(\overline{x}_1,\dots, \overline{x}_n)\in K$, entonces $f(\overline{x}_1,\dots, \overline{x}_n)=\overline{f(x_1,\dots,x_n)}=\overline{0}\in K$. 

Se tiene en general que $\mathcal{I}(\mathcal{V}(I))\supseteq\sqrt{I}$ para cualquier ideal. Vamos a probar la inclusión contraria. Sea $I\subset \mathbb{K}[x_1,\dotsc,x_n]$, $I=\langle f_1,\dotsc,f_r\rangle$, $f\in \mathcal{I}(\mathcal{V}(I))$. Añadimos una variables $\mathbb{K}[x_1,\dotsc,x_n,y]$ y consideramos $J=\langle f_1,\dotsc,f_r,1-y\dot f\rangle = I^e + \langle 1-yf\rangle$. Se tiene que $\mathcal{V}(J) = \mathcal{V}(I^e) \cap \mathcal{V}(1-yf)=\emptyset$ pues al ser $f\in \mathcal{I}(\mathcal{V}(I))$, que los $f_i(x_1,\dots, x_n)=0$ $(i=1,\dots,r)$ implica que $f(x_1,\dotsc,x_n)=0$. Entonces $1-yf(x_1,\dotsc,x_n) = 1\neq 0$, por lo que $1-yf(x_1,\dotsc,x_n)\notin\mathcal{V}(J)$. 

Por tanto, $J=\mathbb{K}[x_1,\dotsc,x_n,y]=\langle 1 \rangle$. Tenemos que $1\in J$, así pues, 
$$
1=h_1(x_1,\dotsc,x_n,y)f_1(x)+\dotsc h_r(x_1,\dotsc,x_n,y) f_r(x)+ h(1-yf)\in \mathbb{K}[x_1,\dotsc,x_n,y].$$
Realizando el cambio de variable $y=\frac{1}{f(x_1,\dots,x_n)}=\frac{1}{f(x)}$ obtenemos
\begin{equation}
1 = h_1(x_1,\dotsc,x_n,\frac{1}{f(x)})f_1(x)+\dotsc+h_r(x_1,\dotsc,x_n,\frac{1}{f(x)})f_r(x)\in  \mathbb{K}(x_1,\dotsc,x_n).\label{1}
\end{equation}
Por último, despejando,
$$f(x_1,\dotsc,x_n)^s = g(x_1,\cdots,x_n) = g_1 f_1 + \cdots g_n f_n \in I$$

Donde $\frac{g(x_1,\dotsc,x_n)}{f(x_1,\dotsc,x_n)^s}$ es el resultado de sumar todas las fracciones de \ref{1}, puesto que todas en su denominador tienen solamente alguna potencia de $f(x_1,\dotsc,x_n)$.

Esto significa que $f^s\in I$ para algún $s\geq 1$, por lo que $\mathcal{I}(\mathcal{V}(I))\subseteq\sqrt{I}$, con lo que hemos probado la igualdad. En particular, si $I$ es radical, $\mathcal{I}(\mathcal{V}(I)) = I$. Este resultado que hemos obtenido se conoce como el siguiente teorema.

\begin{teorema}[\textbf{De los ceros de Hilbert (Nullstellensatz)}]
Sea $\mathbb{K}$ un cuerpo algebraicamente cerrado y sea $I\subset\mathbb{K}[x_1,\dots,x_n]$ un ideal. Si $f\in\mathbb{K}[x_1,\dots,x_n]$ se anula en $\mathcal{V}(I)$, entonces existe un número natural $s\geq 1$ tal que $f^s\in I$. 

\end{teorema}

\begin{nota}
Podemos asegurar que el cambio $y=\frac{1}{f(x)}$ está bien definido porque surge del homomorfismo $\mathbb{K}[x_1,\dots,x_n,y]\rightarrow\mathbb{K}(x_1,\dots,x_n)$ tal que $h(x_1,\dots,x_n)\mapsto h(x_1,\dots,x_n,\frac{1}{f(x)})$. Está bien definidido porque $\frac{1}{f(x)}$ es una función racional bien definida y se corre el peligro de que se anulen denominadores puesto que partimos de un anillo de polinomios. Sin embargo, este cambio no sería válido en como homomorfismo  $\mathbb{K}(x_1,\dots,x_n,y)\rightarrow\mathbb{K}(x_1,\dots,x_n)$, puesto que la imagen de $\frac{1}{yf(x)-1}$ no estaría bien definida.
\end{nota}

\section{Funciones regulares. Anillos de coordenadas.}
Consideramos ahora el ideal $I = \langle x²z,z²y\rangle \subset \K[x,y,z]$. Se tiene que:
\[ \mathcal{V}(I) = \begin{cases}
	x²z = 0\\
	z²y = 0
\end{cases} \equiv \{x=y=0\}\cup\{ z=0\} \}\]
Sea $f(x,y,z) = \sum_{i,j,k}a_{i,j,k}x^iy^jz^k \in \mathcal{I}(\mathcal{V}(I))$. Debe cumplirse que:
\[ f(x,y,0) = \sum_{i,j}a_{i,j,0}x^iy^j = 0 \]
Como estamos en cuerpo algebraicamente cerrado, $\K$ no puede ser finito, por lo que se deduce que $a_{ij0}=0$ $\forall i,j$.
Similarmente:
\[ f(0,0,z) = 0 = \sum_{k}a_{0,0,k}z^k = 0 \Rightarrow a_{0,0,z} = 0 \]
Luego para un polinomio no trivial, $k ≥ 1$ y $i ≥ 1$ ó $j ≥ 1$. Esto se traduce a que podemos sacar factor común $z$ y $x$ ó $y$:
\[ f(x,y,z) = z \sum_{i,j,k ≥ 1} a_{i,j,k} x^iy^jz^{k-1} = z(x\cdot g(x,y,z)+y\cdot h(x,y,z)) \in \langle zx,zy \rangle \]
Entonces $\langle xz,zy\rangle = \sqrt{\langle x²z,z²y\rangle}$.


\begin{defi} Sea $X \subset A^n$ un conjunto algebraico.
Una función $f : X \to \K$ es \textbf{regular} si existe un polinomio 
$P \in \K[x_1,\dots,x_n]$ tal que $f(x) = P(x)$ $\forall x \in X$.
Denotamos $\mathcal{A}(X)$ al conjunto de funciones regulares en $X$. Es un anillo, llamado \textbf{anillo de coordenadas} de $X$.
\end{defi}

Sea $φ : k[x_1,\dots,_n] \to \mathcal{A}(X)$. Por definición, $\Im(φ) = \mathcal{A}(X)$ y $\ker(φ) = \mathcal{I}(X)$. Por el teorema de isomorfía, $\mathcal{A}(X) \cong \K[x_1,\dots,x_n]/\mathcal{I}(X)$.

\begin{defi}\label{definicion}
Un anillo es \textbf{reducido} si no tiene elementos nilpotentes distintos de $0$.
\end{defi}
Supongamos ahora que $\mathcal{I}(X)$ es un ideal radical. Entonces, si $a^r\in\mathcal{I}(X)$ tenemos que $a\in\mathcal{I}(X)$. Traducido al cociente, esto quiere decir que si $\overline{a}^r = \overline{0}$, $\overline{a}=0$. En otras palabras, $\mathcal{I}(X)$ es radical si y sólo si $\K[x_1,\dots,x_n]/\mathcal{I}(X)$ es reducido.

\begin{lemma}
Un ideal $m \subset \K[x_1,\dots,x_n]$ es maximal si y sólo si no existe un ideal radical $I$ con $m \subsetneq I \subsetneq \K[x_1,\dots,x_n]$.
\end{lemma}

\begin{dem}
Es consecuencia directa de que $I \subseteq \sqrt{I}$ y de la definición de ideal maximal, y de que si $\sqrt{I}$ es el total, entonces $I$ es el total.\QED
\end{dem}

\begin{teorema}
$\mathcal{I}(X)$ es maximal si y sólo si no existe un conjunto algebraico $Y$ tal que $\emptyset \subsetneq Y \subsetneq X$. Por lo tanto, hay una correspondencia entre los conjuntos algebraicos con un solo punto y los ideales maximales
\end{teorema}

\begin{ejer}
$m_p = \langle x_1-a_1, \dots,x_n-a_n \rangle$ es un ideal maximal de $\K[x_1,\dots,x_n]$.
\end{ejer}

\begin{nota}
Recordemos de \emph{Estructuras Algebraicas} la correspondencia entre los ideales de $A$ que contienen a un ideal $I$ y los ideales del anillo cociente $A/I$. Esto es, si $I\subseteq J\subseteq A$, el ideal correspondiente es simplemente $J/I$. Eso nos lleva al siguiente resultado. Para un conjunto algebraico $X \subset \A^n$, hay una correspondencia entre los puntos de $X$ y los ideales maximales de $\mathcal{A}(X)$.
\end{nota}

\begin{defi}
Un conjunto algebraico es \textbf{reducible} si existen conjuntos algebraicos $Y$ y $Z$ distintos de $X$ tales que $X = Y \cup Z$. En caso contrario, decimos que $X$ es \textbf{irreducible}.
\end{defi}

\begin{ej}\mbox{}
\begin{enumerate}
	\item Si $P \in \A¹$, entonces $\{P\}$ es irreducible. Si $P_1,\dots,P_r \in \A^1$ son distintos, $\{P_1,\dots,P_r\}$ es reducible.
	\item $X=\A¹$ es irreducible, puesto que los únicos conjuntos algebraicos en este espacio distintos del total son los conjuntos finitos.
	\item $X=\mathcal{V}(y-x²) \subset \A²$ es irreducible. Se podrá probar con la siguiente proposición gracias a que $\mathcal{I}(X)$ es primo por ser $y-x^2$ irreducible.
\end{enumerate}
\end{ej}

\begin{prop}
Sea $\emptyset \neq X \subseteq \A^n$ algebraico, son equivalentes:
\begin{enumerate}
	\item $X$ es irreducible.
	\item $\mathcal{I}(X)$ es primo.
	\item $\mathcal{A}(X)$ es un dominio de integridad.
\end{enumerate}
	Por lo tanto hay una correspondencia entre conjuntos algebraicos irreducibles y los ideales primos.
\end{prop}
\begin{dem}\mbox{}
\begin{itemize}
	\item[$(1) \Rightarrow (2)$] Supongamos que $\mathcal{I}(X)$ no es primo. Entonces existen $f$ y $g$ en $\K[x_1,\dots,x_n]$ tal que $f\cdot g \in \mathcal{I}(X)$ pero $f$ y $g$ no están en $\mathcal{I}(X)$. Como $X$ es algebraico, existe $I$ con $\mathcal{V}(I)=X$. Veamos que:
	\[ X = \mathcal{V}(I + \langle f\rangle) \cup \mathcal{V}(I+\langle g\rangle) = Y \cup Z \]
	Está claro que $Y \cup Z \subseteq X$. Sea $x \in X$, como $f(x)g(x) = 0$. Entonces o bien $f(x)=0$ o bien $g(x)=0$, es decir, $x \in Y$ ó $x \in Z$. Luego $X = Y \cup Z$.
	
	\item[$(2) \Rightarrow (1)$] Supongamos que $X = Y \cup Z$ con $Y \neq X \neq Z$. De $Y \subsetneq X$, se deduce que $\mathcal{I}(X) \subsetneq \mathcal{I}(Y)$. Luego existe $f \in \mathcal{I}(Y) \setminus \mathcal{I}(X)$ y $g \in \mathcal{I}(Z) \setminus \mathcal{I}(X)$. Como $f \cdot g = 0$ en $Y$ y en $Z$, $f \cdot g = 0$ en $X$.
	
	\item[$(3) \Leftrightarrow (2)$] Usando que todo ideal primo es radical y el isomorfismo $\mathcal{A}(X)\cong\mathbb{K}[x_1,\dots,x_n]/\mathcal{I}(X)$ la prueba está realizada previamente, tras la definición \ref{definicion}.\QED
\end{itemize}
\end{dem}

\begin{defi}
Una \textbf{variedad algebraica} es un conjunto algebraico irreducible.
\end{defi}

\begin{prop}
Todo conjunto algebraico puede expresarse como una unión finita de conjuntos irreducibles. Además, la descomposición es única en caso de ser minimal.
\end{prop}
\begin{dem}
De no ser así, podríamos crear una sucesión decreciente infinita de conjuntos algebraicos irreducibles, que se corresponde con una sucesión creciente infinita de ideales primos. Como $\K[\xn{n}]$ es noetheriano, no existe tal sucesión.

Supongamos ahora que $X=X_1\cup\cdots, \cup X_r=Y_1\cup\cdots\cup Y_s$ descomposiciones minimales, es decir, para todo $i$, $X_i$ no está contenido en la unión de los demás. 
\begin{lemma}
Sea $Y$ irreducible, $Y\subseteq Z_1\cup Z_2$ con $Z_1,Z_2$ algebraicos. Entonces $Y\subseteq Z_1$ o $Y\subseteq Z_2$. 
\end{lemma}
\begin{proof}
$Y=Y\cap( Z_1\cup Z_2)=(Y\cap Z_1)\cup(Y\cap Z_2)$, luego por ser $Y$ irreducible, o bien $Y\cap Z_1=Y$ o bien $Y=Y\cap Z_2$, de donde se deduce el resultado. Por inducción se prueba para cualquier unión finita.
\end{proof}
Usando el lema anterior tenemos que, $X_1\subseteq X_1\cup X_2\cdots X_r=Y_1\cup Y_2\cdots\cup Y_s$ implica que $X_1\subseteq Y_j$ para algún $j$. Análogamente $Y_j\subseteq X_i$ para algún $i$. Como $X_1\not\subseteq X_i$ para $i\neq j$, se llega a que $X_1=Y_j$, con lo que finaliza la prueba.\QED


\end{dem}

\begin{ej}
$\mathcal{V}(\langle x^2z,z^2y\rangle)=\mathcal{V}(\langle xz,yz\rangle)=\mathcal{V}(z)\cup\mathcal{V}(x,y)$. El ideal generado por $z$ es primo porque está generado por un polinomio irreducible. El generado por $x,y$ también lo es pues al hacer el producto entre dos polinomios que tengan términos solo en $z$, también resultará un polinomio con términos solo en $z$, por lo que no estará en el ideal.
\end{ej}

\begin{prop}
Si existe un homomorfismo $f:A\to B$ con $B$ dominio de integridad tal que $\ker{f}=I$, entonces $I$ es primo.
\end{prop}

\begin{dem}
Sean $a,b\in A$ tales que $ab\in\ker{f}$, entonces $f(a)f(b)=f(ab)=0$. Como $B$ es dominio de integridad, o bien $f(a)=0$ o $f(b)=0$, por lo que o bien $a\in\ker{f}$ o bien $b\in\ker{f}$, como queríamos demostrar.\QED
\end{dem}

\begin{ej}\label{ej}
En el ejemplo anterior, podemos ver que $\langle x,y\rangle$ es primo probando usando la proposición.  Basta tomar el homomorfismo $\K[x,y,z]\to\K[z]$ tal que $f(x,y,z)\mapsto f(0,0,z)$. En este caso el $\ker$ es justamente $\langle x,y\rangle$.
\end{ej}

\begin{defi}
Las \textbf{componentes irreducibles} de $X$ son los conjuntos algebraicos que aparecen en la descomposición de $X$ como unión de irreducibles. Dicho de otra forma, las componentes irreducibles de $X$ son los conjuntos algebraicos irreducibles maximales contenidos en $X$.
\end{defi}
Esta definición la podemos traducir al álgebra. Si tenemos una componente irreducible maximal $Y\subset X$, podemos asociar un ideal primo minimal $\mathcal{I}(Y)\supset\mathcal{I}(X)$.
\begin{defi}
Sea $A$ un anillo, un ideal primo \textbf{minimal} de $A$ es un ideal primo $P\subset A$ tal que no existe $Q\subsetneq P$ primo.
\end{defi}
\begin{defi}
Sea $I\subset A$ un ideal. Los primos minimales de $I$ son los ideales $I\subseteq P\subseteq A$ primos tales que no existe otro primo $I\subseteq Q\subsetneq P\subseteq A$.
\end{defi}

\begin{ej}
\begin{enumerate}
\item En cualquier dominio de integridad, el único primo minimal es el $\{0\}$. 
\item En $\Z/\Z_{12}$ serían primos minimales $\langle 2\rangle$ y $\langle 3\rangle$. El trivial ya no es primo por no ser dominio de integridad, el total nunca lo es, y $\langle 4\rangle,\langle 6\rangle$ están contenidos en los anteriores respectivamente. En general, en $\Z/\Z_{n}$, los ideales primos minimales son los generados por un divisor primo de $n$. 
\item $\K[x,y]/\langle xy\rangle$. En este caso los ideales primos minimales de este anillo se corresponden con los ideales primos minimales de $\K[x,y]$ que contengan a $\langle xy\rangle$. Tenemos que $\mathcal{V}(xy)=\mathcal{V}(x)\cup\mathcal{V}(y)$. Por lo que los ideales que buscábamos son $\langle x\rangle$ y $\langle y\rangle$.
\item $\mathcal{V}(\langle xy,x^2z-xz\rangle)\subseteq\mathbb{A}^3$. Primero resolvemos $X\equiv\begin{cases}
xy=0\\
x^2-xz=0.
\end{cases}$ Concluimos que $X=\mathcal{V}(x)\cup\mathcal{V}(y,z)\cup\mathcal{V}(y,x-1)$. El primero es claramente irreducible porque el ideal es primo al ser $\langle x\rangle$ está generado por un polinomio irreducible. De forma análoga al ejemplo \ref{ej} se prueba para los otros dos conjuntos. 
\end{enumerate}
\end{ej}

\begin{prop}
La dimensión de $\mathbb{A}^2$ es $2$
\end{prop}
\begin{dem}
Es claro que existe una cadena ${(0,0)}\subsetneq\V(x-y)\subsetneq\mathbb{A}^2$, por lo que $dim(\mathbb{A}^2)\geq 2$. Supongamos que existe una cadena $Z\subsetneq Y\subsetneq X\subsetneq\mathbb{A}^2$ de variedades algebraicas. Entonces, $X=\V(f_1,\dots,f_n)\subseteq\V(f_1)=\V(g_1)\cup\cdots\cup\V(g_s)$, donde $f_1=g_1\cdots g_s$. Por tanto, $X\subseteq\V(g_i)$ para algún $i$. Por otro lado, $Y=(g, h_1,\dots,h_t)$. Si $g\mid h_i\forall i\Rightarrow X\subseteq Y\Rightarrow X=Y$, con lo que llegaríamos a una contradicción. Por tanto, existe algún $h_i$ que no es múltiplo de $g$, es decir, $Y\subseteq\V(g,h_i)$. Si tenemos $\V(g,h)$ donde $g$ es irreducible si $g\not\mid h$, entonces $\V(g,h)$ es finito. Probémoslo por reducción al absurdo.

Supongamos que $\V(g,h)$ es infinito. Las primeras o las segundas coordenadas de esos puntos forman un conjunto infinito. Por fijar ideas, supongamos que hay infinitas coordenadas $x$. Podemos considerar $g,h\in k(x)[y]$. Usando $\gcd(g,h)=1$ (porque $g$ es irreducible y no divide a $h$), podemos utilizar Bézout (puesto que estamos en el anillo de polinomios en $y$ sobre el cuerpo de fracciones $k(x)$) para escribir $a(x,y)g(x,y)+b(x,y)h(x,y)=1$. Tanto $a$ como $b$ tienen denominadores solo en $x$.  Así que quitando denominadores obtenemos 
$$\tilde{a}(x,y)g(x,y)+\tilde{b}(x,y)h(x,y)=P(x).$$
Entonces, $P(x)$ se anula en todos los ceros comunes de $g$ y $f$, que son infinitos, por lo que $P(x)\equiv 0$, lo cual es contradicción con haberlo obtenido eliminado denominadores (no podía ser nulo). 

Por tanto, $\V(g,h)$ es finito e $Y$ sería el último término de la cadena. 
\end{dem}

\begin{ej}\
\begin{enumerate}
\item Si $A$ es cuerpo, $dim_{Krull}(A)=0$.
\item $dim_{Krull}(\Z)=1$, pues $\{0\}\subsetneq\langle p\rangle\subsetneq\Z$. En general si a $A$ es DIP la dimensión de Krull es $1$.
\item $dim_{Krull}(k[\xn{n}])=n$
\end{enumerate}
\end{ej}

\end{document}
