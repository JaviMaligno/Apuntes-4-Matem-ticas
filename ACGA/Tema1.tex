\documentclass[ACGA.tex]{subfiles}
%\usepackage{amsmath,amssymb}
%\usepackage[utf8]{inputenc}
%\usepackage[spanish]{babel}
%\usepackage[]{graphicx}
%\usepackage{enumerate}
%\usepackage{amsthm}
%\usepackage{tikz-cd}
%\usetikzlibrary{babel}
%\usepackage{pgf,tikz}
%\usepackage{mathrsfs}
%\usetikzlibrary{arrows}
%\usetikzlibrary{cd}
%\usepackage[spanish]{babel}
%\usepackage{fancyhdr}
%\usepackage{titlesec}
%\usepackage{floatrow}
%\usepackage{makeidx}
%\usepackage[tocflat]{tocstyle}
%\usetocstyle{standard}
%\usepackage{color}
%\usepackage{hyperref}
%\hypersetup{colorlinks=true,citecolor=red, linkcolor=blue}
%%\usepackage{ntheorem}
%
%
%\renewcommand{\baselinestretch}{1,4}
%\setlength{\oddsidemargin}{0.25in}
%\setlength{\evensidemargin}{0.25in}
%\setlength{\textwidth}{6in}
%\setlength{\topmargin}{0.1in}
%\setlength{\headheight}{0.1in}
%\setlength{\headsep}{0.1in}
%\setlength{\textheight}{8in}
%\setlength{\footskip}{0.75in}
%
%\newtheorem{teorema}{Teorema}[section]
%\newtheorem{defi}[teorema]{Definición}
%\newtheorem{coro}[teorema]{Corolario}
%\newtheorem{lemma}[teorema]{Lema}
%\newtheorem{ej}[teorema]{Ejemplo}
%\newtheorem{ejs}[teorema]{Ejemplos}
%\newtheorem{observacion}[teorema]{Observación}
%\newtheorem{observaciones}[teorema]{Observaciones}
%\newtheorem{prop}[teorema]{Proposición}
%\newtheorem{propi}[teorema]{Propiedades}
%\newtheorem{nota}[teorema]{Nota}
%\newtheorem{notas}[teorema]{Notas}
%\newtheorem*{dem}{Demostración}
%\newtheorem{ejer}[teorema]{Ejercicio}
%\newtheorem{consec}[teorema]{Consecuencia}
%\newtheorem{consecs}[teorema]{Consecuencias}
%
%\providecommand{\abs}[1]{\lvert#1\rvert}
%\providecommand{\sen}[1]{sen #1}
%\providecommand{\norm}[1]{\lVert#1\rVert}
%\providecommand{\ninf}[1]{\norm{#1}_\infty}
%\providecommand{\numn}[1]{\norm{#1}_1}
%\providecommand{\gabs}[1]{\left|{#1}\right|}
%\newcommand{\bor}[1]{\mathcal{B}(#1)}
%\newcommand{\R}{\mathbb{R}}
%\newcommand{\N}{\mathbb{N}}
%\newcommand{\Q}{\mathbb{Q}}
%\newcommand{\C}{\mathbb{C}}
%\newcommand{\Pro}{\mathbb{P}}
%\newcommand{\Tau}{\mathcal{T}}
%\newcommand{\verteq}{\rotatebox{90}{$\,=$}}
%\newcommand{\vertequiv}{\rotatebox{110}{$\,\equiv$}}
%\providecommand{\lrg}{\longrightarrow}
%\providecommand{\func}[2]{\colon{#1}\longrightarrow{#2}}
%\newcommand*{\QED}{\hfill\ensuremath{\blacksquare}}
%\newcommand*\circled[1]{\tikz[baseline=(char.base)]{
%            \node[shape=circle,draw,inner sep=1.5pt] (char) {#1};}}
%\newcommand*{\longhookarrow}{\ensuremath{\lhook\joinrel\relbar\joinrel\rightarrow}}
%
%\newenvironment{solucion}{\begin{trivlist}
%\item[\hskip \labelsep {\textit{Solución}.}\hskip \labelsep]}{\end{trivlist}}
%
%
%\def\quot#1#2{%
%    \raise1ex\hbox{$#1$}\Big/\lower1ex\hbox{$#2$}%
%}
%\def\quott#1#2{%
%    \hbox{$#1$}\Big/\lower1ex\hbox{$#2$}%
%}
%
%\makeatletter
%\renewcommand\tableofcontents{%
%  \null\hfill\textbf{\Large\contentsname}\hfill\null\par
%  \@mkboth{\MakeUppercase\contentsname}{\MakeUppercase\contentsname}%
%  \@starttoc{toc}%
%}
%
%\pagestyle{fancy}
%\fancyhf{}
%\rhead{Topología de Superficies (Grado en Matemáticas)}
%\lhead{Curso 2016/2017}
%\cfoot{\thepage}

\begin{document}
%\title{Topología de Superficies}
%\author{Antonio Rafael Quintero Toscano\\ Javier Aguilar Martín}
%\date{Curso 2016/2017}
%\maketitle

\renewcommand\chaptername{\Huge Tema}

\titleformat{\chapter}[display]
    {\normalfont\huge\bfseries}{\chaptertitlename\ \thechapter}{10pt}{\Huge}
\titlespacing*{\chapter}{0pt}{-1cm}{10pt}



\chapter{Conjuntos algebraicos afines. Anillos de coordenadas. Topología de Zariski. Anillos locales.}
Vamos a trabajar en el espacio afín de dimensión n $\mathbb{A}^n$ sobre un cuerpo $\mathbb{K}$ fijo. 
\begin{defi}
Un subcojunto $X\subset \mathbb{A}^n$ es un \textbf{conjunto algebraico} si existe un conjunto $S\subset \mathbb{K}[x_1,\dotsc,x_n]$ tal que $X$ es el conjunto de los ceros comunes de los polinomios de S. Es decir:
\[
X=\{(x_1,\dotsc,x_n)\in\mathbb{A}^n\mid f(x_1,\dotsc,x_n)=0\;\forall f\in S\}=V(S)
\]
\end{defi}
\begin{prop}Veamos algunas propiedades de los conjuntos algebraicos:
\begin{itemize}
\item En $\mathbb{A}^1$, los conjuntos algebraicos son finitos, salvo $X=\mathbb{A}^1$ ($S=\{0\}$). El recíproco también es cierto, es decir, cualquier conjunto finito es algebraico.
\item En general $V(\{0\})=\mathbb{A}^n$ y $V(\{1\})=\emptyset$. Además, los conjuntos X unitarios siempre son algebraicos, pues tomamos $S=\{x_i-a_i\mid i=1,\dotsc,n\}$.
\item En $\mathbb{A}^2$, por ejemplo, $X=\{(a,b)\} = V(\{x-a,y-b\})$ y en general no tiene por qué existir un conjunto S unitario tal que $V(S)=X$. Si tenemos $X=\{(a,b),(c,d)\}$ entonces $S=\{(x-a)(x-c),(y-b)(y-d), (x-a)(y-d), (y-b)(x-c)\}$. En general, podemos generalizarlo a n puntos considerando todas las combinaciones 2 a 2 obteniendo $2^n$ polinomios.
\end{itemize}
\end{prop}
\begin{prop}
Si X e Y son algebraicos tales que $X=V(S)$ e $Y=V(T)$ entonces $X\cap Y = V(T\cup S)$. Además si $S\cdot T = \{f\cdot g \mid f \in S,\; g \in T\}$ entonces $X\cup Y = V(S\cdot T)$. 
\end{prop}
\begin{dem}
Todas las inclusiones son sencillas. Veamos una. Sea $x\notin X\cup Y$, entonces $x\notin X$ y $x\notin Y$. Entonces existen $f\in S$ y $g\in T$ tal que $f(x)\neq 0$ y $g(x)\neq 0$. Se sigue $f(x)g(x)\neq 0$, luego $x\notin V(S\cdot T)$. 
\end{dem}

\begin{defi}
Los conjuntos algebraicos de $\mathbb{A}^n$ son los cerrados de una topología llamada \textbf{Topología de Zariski}.
\end{defi}
\begin{prop}
En general $V(S)=V(\langle S\rangle)$ $\forall S \subset  \mathbb{K}[x_1,\dotsc,x_n]$.
\end{prop}
\begin{dem}
Sabemos que en general si $S\subset T$ entonces $V(T)\subset V(S)$, por lo que tomando $T=\langle S \rangle$, $V(\langle S\rangle)\subset V(S)$. Además, si $x\in V(S)$ entonces $f(x)$ se anula $\forall f \in S$, por lo que cualquier combinación de la forma
\[
g_1 f_1(x)+ \dotsc + g_rf_r(x) = 0 \quad \forall f_i \in S
\]
Por lo que $x\in V(\langle S\rangle)$.
\end{dem}
\begin{defi}
Un anillo se dice \textbf{noetheriano} si todo ideal es finitamente generado.
\end{defi}
\begin{prop}
Son anillos noetherianos:
\begin{itemize}
\item Los anillos finitos
\item Los dominios de ideales principales (y por tanto los cuerpos).
\end{itemize}
\end{prop}
\end{document} 