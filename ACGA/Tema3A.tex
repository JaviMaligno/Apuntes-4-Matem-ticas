\documentclass[ACGA.tex]{subfiles}
%\usepackage[utf8x]{inputenc}
%\usepackage[spanish]{babel}
%\usepackage{amsmath, amssymb, amsthm, epsf, graphicx, amscd, amsfonts}
%\usepackage[colorlinks]{hyperref}
%\usepackage{xmpincl}
%\usepackage{fancyhdr}
%
%
%\pagestyle{fancy}
%\lhead[\thepage]{\rightmark}
%\rhead[\leftmark]{\thepage}
%\cfoot[]{}
%
%\addto\captionsspanish{ \renewcommand{\chaptername}{Tema} }
%
%\newtheorem{thm}{Teorema}[chapter]
%\newtheorem{cor}[thm]{Corolario}
%\newtheorem{lem}[thm]{Lema}
%\newtheorem{prop}[thm]{Proposición}
%\newtheorem{defn}[thm]{Definición}
%\newtheorem{rem}[thm]{Observaciones}
%\newtheorem{eje}[thm]{Ejemplos}
%
%\newtheorem{ejercicio}{Ejercicio}[chapter]
%
%\newcommand{\RR}{\mathbb R}
%\newcommand{\CC}{\mathbb C}
%\newcommand{\AAA}{\mathbb A}
%\newcommand{\PP}{\mathbb P}
%\newcommand{\Ank}{\AAA^n_k}
%\newcommand{\Pnk}{\PP^n_k}
%\newcommand{\Pmk}{\PP^m_k}
%\newcommand{\Amk}{\AAA^m_k}
%\newcommand{\calA}{{\mathcal A}}
%\newcommand{\II}{{\mathcal I}}
%\newcommand{\VV}{{\mathcal V}}
%\newcommand{\KK}{{\mathcal K}}
%\newcommand{\OO}{{\mathcal O}}
%\newcommand{\mm}{{\mathfrak m}}
%
%\title{Notas y ejercicios de Geometría Algebraica}
%\author{Departamento de Álgebra \\ Universidad de Sevilla}
%\date{Septiembre de 2017}




\begin{document}

%\setcounter{chapter}{2}
%\maketitle
%
%\vspace*{\fill}
%
%\copyright{2011-17 Antonio Rojas León}
%
%\bigskip
%
%Este trabajo está publicado bajo licencia Creative Commons 3.0 España (Reconocimiento - No Comercial - Compartir bajo la misma licencia)
%
%\url{http://creativecommons.org/licenses/by-nc-sa/3.0/es/}
%
%\bigskip
%
%Usted es libre de:
%\begin{itemize}
% \item copiar, distribuir y comunicar públicamente la obra
%\item hacer obras derivadas
%\end{itemize}
%
%Bajo las condiciones siguientes:
%\begin{itemize}
% \item {\bf Reconocimiento:} Debe reconocer los créditos de la obra maestra especificada por el autor o el licenciador (pero no de una manera que sugiera que tiene su apoyo o apoyan el uso que hace de su obra).
% \item {\bf No comercial:} No puede utilizar esta obra para fines comerciales.
%\item {\bf Compartir bajo la misma licencia:} Si altera o transforma esta obra, o genera una obra derivada, sólo puede distribuir la obra generada bajo una licencia idéntica a ésta.
%\end{itemize}
%
%
%
%
%\newpage

\chapter{Morfismos entre variedades afines}

\section{Morfismos}

Como es habitual en Matemáticas, una vez descritos los objetos con los que vamos a trabajar, el siguiente paso es definir cuáles son las aplicaciones ``interesantes'' entre dos objetos fijados. 

\begin{defi}
 Sean $Y\subseteq\Ank$ y $Z\subseteq\Amk$ dos conjuntos algebraicos. Un {\bf morfismo} entre $Y$ y $Z$ es una aplicación $\phi:Y\to Z$ tal que existen $m$ funciones regulares $f_1,\ldots,f_m\in \calA(Y)$ con $\phi(x)=(f_1(x),\ldots,f_m(x))$ para todo $x\in Y$.
\end{defi}

Los morfismos pueden definirse de manera más intrínseca:

\begin{prop}
 Una aplicación $\phi:Y\to Z$ es un morfismo si y sólo si para toda función regular $f:Z\to k$ la composición $f\circ\phi:Y\to k$ es una función regular.
\end{prop}

 \begin{proof}
  Sea $\phi:Y\to Z$ un morfismo, dado por $\phi(x)=(f_1(x),\ldots,f_m(x))$. Toda función regular $f:Z\to k$ viene dada por un polinomio $f\in k[x_1,\ldots,x_m]$, y entonces $f\circ\phi$ viene dada por el polinomio $g(x):=f(f_1(x),\ldots,f_m(x))\in k[x_1,\ldots,x_n]$.

Recíprocamente, supongamos que $f\circ\phi$ es regular para toda función regular $f$. En particular, si $f(x)=x_i$ para un $i=1,\ldots,m$ fijo, $f\circ\phi$ (que es la $i$-ésima coordenada de $\phi$) es regular, y por tanto viene dada por un polinomio $f_i\in k[x_1,\ldots,x_n]$. Entoces $\phi(x)=(f_1(x),\ldots,f_m(x))$.
 \end{proof}

De esta última caracterización se deduce que la composición de dos morfismos es un morfismo. Todo morfismo $\phi:Y\to Z$ define una aplicación $\phi^\star:\calA(Z)\to\calA(Y)$ dada por $f\mapsto f\circ\phi$. Esta aplicación respeta la suma y el producto y preserva las constantes, y por tanto es un homomorfismo de $k$-álgebras. Además, si $\phi':X\to Y$ es otro morfismo de conjuntos algebraicos, $(\phi\circ\phi')^\star=\phi'^\star\circ\phi^\star$. Veamos que todo homomorfismo de $k$-álgebras $\calA(Z)\to\calA(Y)$ se obtiene de esta forma:

\begin{prop}\label{funtor}
 Sean $Y\subseteq\Ank$ y $Z\subseteq\Amk$ dos conjuntos algebraicos, y $\psi:\calA(Z)\to\calA(Y)$ un homomorfismo de $k$-álgebras. Existe un único morfismo $\phi:Y\to Z$ tal que $\psi=\phi^\star$.
\end{prop}

\begin{proof}
 Para cada $i=1,\ldots,m$, sea $f_i(x)=\psi(x_i)\in\calA(Y)$. Si $\phi:Y\to Z$ es un morfismo tal que $\phi^\star=\psi$, entonces la $i$-ésima coordenada de $\phi$ debe ser $x_i\circ\phi=\phi^\star(x_i)=\psi(x_i)=f_i$, y por tanto $\phi(x)=(f_1(x),\ldots,f_m(x))$. Veamos que este $\phi$ cumple la condición buscada.

 Por construcción tenemos que $\phi^\star(x_i)=\psi(x_i)$ para todo $i$, y por tanto $\phi^\star=\psi$, ya que los $x_i$ generan la $k$-álgebra $\calA(Z)$. Falta comprobar que la imagen de $\phi$ está en $Z$. Como $Z=\V(\I(Z))$, basta comprobar que $g(\phi(x))=0$ para todo $g\in\I(Z)$ y todo $x\in X$. Pero $g(\phi(x))=(\phi^\star g)(x)=\psi(g)(x)=\psi(0)(x)=0$, ya que $g=0$ en $\calA(Z)$. 
\end{proof}

La proposición \ref{funtor}, junto con el ejercicio \ref{full}, muestran que:

\begin{coro}\label{equivafin}
 Las aplicaciones $X\mapsto \calA(X)$, $\phi\mapsto \phi^\star$ inducen una equivalencia de categorías (mediante un functor contravariante) entre la categoría de conjuntos algebraicos afines y morfismos de conjuntos algebraicos y la categoría opuesta a la de $k$-álgebras finitamente generadas reducidas y homomorfismos de $k$-álgebras. Si $A$ es una $k$-álgebra reducida finitamente generada, entonces $k[x_1,\dots,x_n]\to A$ es sobreyectivo, por lo que $A\cong k[x_1,\dots,x_n]/I$, donde $I$ es lel $\ker$ de esta aplicación, que por ser $A$ reducida es un ideal radical, así que se corresponde con algún $\I(X)$, de modo que el cociente anterior es isomorfo a $\calA(X)$.
\end{coro}

\begin{ej}
Sea $\varphi:\A^2\to\V(xy-z^2)=Y\subset\A^3: (s,t)\mapsto(s^2,t^2,st)$. Definimos la función regular $f:Y\to k, f(x,y,z)=x^2y+yz^3+2xyz$. Entonces $\varphi^*(f):X\to k$ es la aplicación $(s,t)\mapsto s^4t^2+s^3t^5+2s^3t^3$, es decir que
$\varphi^*:\calA(X)\to\calA(Y)$ se define como 
\begin{align*}
&x\mapsto s^2\\
&y\mapsto t^2\\
&z\mapsto st
\end{align*}
\end{ej}

\section{Variedades cuasi-afines}

El concepto de variedad algebraica afín es demasiado restringido para poder trabajar con flexibilidad. Por eso vamos a aumentar la categoría de variedades afines añadiéndole los subconjuntos abiertos de éstas.

\begin{defi}
 Una subconjunto $X\subseteq\Ank$ se denomina {\bf variedad cuasi-afín} en $\Ank$ si es un abierto de Zariski de una variedad afín $Z$. En otras palabras, si existen una variedad afín $Z$ y un conjunto algebraico $Y\subseteq Z$ tales que $X=Z\backslash Y$. 
\end{defi}

Por tanto, para toda variedad cuasi-afín $X$ existen polinomios $f_1,\ldots,f_r$ y $g_1,\ldots,g_s$ en $k[x_1,\ldots,x_n]$ tales que $X$ es el conjunto de puntos $x\in\Ank$ con $f_i(x)=0$ para \emph{todo} $i$ y $g_j(x)\neq 0$ para \emph{algún} $j$.
Como todo abierto no vacío de $Z$ es denso, $X$ determina $Y$ y $Z$: $Z$ es la clausura $\overline X$ e $Y$ es el complementario de $X$ en $Z$.

\begin{defi}
 Sea $X\subseteq\Ank$ una variedad cuasi-afín. Una aplicación $f:X\to k$ se dice {\bf función regular} si existe $\bar f\in\KK(\overline X)$ definida en $X$ tal que $f=\bar f_{|X}$. Se denota
$$\OO(U)=\{f\in K(X)\mid f\text{ definida en todo punto de } U\}$$
\end{defi}

Como las funciones de $\KK(\overline X)$ definidas en $\overline X$ son las funciones regulares, este definición es compatible con la anterior en el caso de que $X$ sea afín. Al igual que en el caso afín, denotaremos por $\calA(X)$ el anillo de funciones regulares en $X$.

\begin{defi}
 Sean $Y\subseteq\Ank$ y $Z\subseteq\Amk$ variedades cuasi-afines. Un {\bf morfismo} entre $Y$ y $Z$ es una aplicación $\phi:Y\to Z$ tal que existen $m$ funciones regulares $f_1,\ldots,f_m:Y\to k$ con $\phi(x)=(f_1(x),\ldots,f_m(x))$ para todo $x\in Y$.
\end{defi}

Al igual que en el caso de morfismos entre variedades afines, tenemos:
\begin{ejer}
 Probar que una aplicación $\phi:Y\to Z$ entre variedades cuasi-afines es un morfismo si y sólo si para toda función regular $f:Z\to k$ la composición $f\circ\phi:Y\to k$ es regular.
\end{ejer}

Un morfismo $\phi:Y\to Z$ de variedades cuasi-afines induce por tanto un homomorfismo de $k$-álgebras $\phi^\star:\calA(Z)\to\calA(Y)$ entre sus anillos de funciones regulares. Sin embargo, el recíproco en este caso \emph{no} es cierto, como vemos en el ejemplo siguiente:

\begin{ejs}\
\begin{enumerate}
\item $\varphi:\A-\{0\}\to\V(xy-1)\subset\A^2: t\mapsto (t,\frac{1}{t})$. Si tomamos por ejemplo, $f(x,y)=x^2y-2xy+3x^2$, entonces $\varphi^*(f)(t)=t-2+\frac{3}{t}$. Si $\varphi:X\to Y$ es un morfismo, entonces $\varphi^*:\OO(Y)\to\OO(X):f\mapsto f\circ\varphi$ es un homomorfismo de $k$-álgebras.
\item No existe un isomorfismo $\varphi:\A^2-\{(0,0)\to\A^2: (x,y)\to (x,y)$. De existir, $\varphi^*:k[x,y]=\calA(\A^2)\to k[x,y]=\OO(\A^2-\{(0,0)\}$ quedaría definido como la aplicación $x\mapsto x, y\mapsto y$ por la defnición de $\varphi^*$. Pero esta aplicación es la identidad, que como el functor es fully faithfull significa que su preimagen era la identidad entre los conjuntos iniciales, la cual no es isomorfismo.

\end{enumerate}
\end{ejs}

\begin{ejer}
 Sean $Y=\A^2_\C$ y $Z=Y-\{(0,0)\}$. Por el ejercicio \ref{a2menos0} $\OO_{Z}(Y)=\OO_{Z}(Z)=\C[x,y]$. Probar que no existe ningún morfismo $\phi:Y\to Z$ tal que $\phi^\star:\C[x,y]\to\C[x,y]$ sea la identidad.
\end{ejer}


\begin{ejs}\label{ejemorf}
\emph{
 \begin{enumerate}
  \item La aplicación $\phi:x\mapsto (x,1/x)$ es un morfismo entre $\A^1_k-\{0\}$ y $\V(\{xy-1\})\subseteq\A^2_\C$.
\item Dada una variedad $Z\subseteq\Ank$, la aplicación dada por $(x_1,\ldots,x_{n-1},x_n)\mapsto (x_1,\ldots,x_{n-1},0)$ es un morfismo entre $Z$ y el hiperplano $H$ definido por $x_n=0$. Su imagen es la \emph{proyección de $Z$ sobre $H$}. Si $Z$ es afín, esta proyección no es necesariamente una variedad afín.
 \end{enumerate}
}
\end{ejs}

Sea $\phi:Y\to Z$ un morfismo de variedades cuasi-afines, $y\in Y$ y $z=\phi(y)\in Z$. Dada una función racional definida en $z$  $f\in\OO_{Z,z}:=\OO_{\overline Z,z}$, se obtiene una función racional $\phi^\star f=f\circ\phi$ en $Y$ definida en $y$. En efecto, supongamos que $f=g/h$ en un entorno $U$ de $z$, con $g,h\in k[x_1,\ldots,x_m]$. Entonces $\phi^\star f=(g\circ \phi)/(h\circ\phi)$ está definida en un entorno $\phi^{-1}(U)$ de $y$. Por tanto, $\phi$ induce un homomorfismo de $k$-álgebras locales $\phi^\star:\OO_{Z,z}\to\OO_{Y,y}$.

Hay que tener en cuenta sin embargo que, en general, $\phi$ \emph{no} induce un homomorfismo $\phi^\star:\KK(Z)\to\KK(Y)$. Esto se debe a que dada $f=g/h:Z\dashrightarrow k$ es posible que $h\circ\phi$ sea idénticamente nula. Por ejemplo, si $\phi:\A^1_k\to\A^2_k$ viene dada por $\phi(t)=(t,0)$ no tiene sentido hablar de $\phi^\star f$, donde $f(x,y)=1/y$. El homomorfismo $\phi^\star:\calA(Z)\to\calA(Y)$ se podrá extender a los cuerpos de fracciones si y sólo si es inyectivo.

\begin{defi}
 Un morfismo de variedades cuasi-afines $\phi:Y\to Z$ se dice {\bf dominante} si su imagen $\phi(Y)$ es densa en $Z$. 
\end{defi}

Por ejemplo, el morfismo $\V(\{xy-1\})\subseteq\A^2_k\to\A^1_k$ dado por $\phi(x,y)=x$ es dominante, puesto que su imagen es el abierto $\A^1_k-\{0\}$.

\begin{nota}
Todo morfismo sobreyectivo es dominante.
\end{nota}

\begin{prop}\label{dominanteinyectivo}
 Sea $\phi:Y\to Z$ un morfismo de variedades cuasi-afines. $\phi$ es dominante si y sólo si el homomorfismo correspondiente $\phi^\star:\calA(Z)\to\calA(Y)$ es inyectivo.
\end{prop}

\begin{proof}
 Supongamos que $\phi$ es dominante, y sea $f\in\ker\phi^\star$. Entonces $f\circ\phi=0$, y por tanto $f$ se anula en todo punto de la imagen de $\phi$. Como el conjunto de puntos donde $f$ se anula es un cerrado y la imagen de $\phi$ es densa, concluimos que $f$ se anula en todo punto de $Z$. Por tanto $f=0$ y $\phi^\star$ es inyectivo.

Recíprocamente, supongamos que $\phi^\star$ es inyectivo. Sea $f\in\calA(Z)$ una función regular que se anule en todo punto de la imagen de $\phi$, es decir, tal que $f\circ\phi=\phi^\star f=0$. Por ser $\phi^\star$ inyectiva, $f$ debe ser nula en todo $Z$. Es decir, $\I(\phi(Y))\subseteq\I(Z)$. Aplicando $\V$ y usando el ejercicio \ref{clausura} concluimos que $Z\subseteq\overline Z\subseteq\overline{\phi(Y)}$, es decir, $\phi(Y)$ es denso en $Z$. 
\end{proof}

Por tanto, un morfismo dominante $\phi:Y\to Z$ de variedades cuasi-afines induce un homomorfismo (necesariamente inyectivo) $\phi^\star:\KK(Z)\hookrightarrow\KK(Y)$ de cuerpos.


\section{Isomorfismos}

\begin{defi} Un morfismo $\phi:Y\to Z$ entre variedades cuasi-afines se dice un {\bf isomorfismo} si existe un morfismo $\psi:Z\to Y$ tal que $\phi\circ \psi:Z\to Z$ y $\psi\circ \phi:Y\to Y$ son el morfismo identidad. En ese caso, $\psi$ se denomina el {\bf inverso} de $\phi$, y se denota por $\phi^{-1}$. Dos variedades cuasi-afines se dicen {\bf isomorfas} si existe un isomorfismo entre ellas.
 \end{defi}

 Por el corolario \ref{equivafin}, $\phi$ es un isomorfismo de variedades \emph{afines} si y sólo si $\phi^\star$ es un isomorfismo de $k$-álgebras, y dos variedades \emph{afines} son isomorfas si y sólo si sus anillos de coordenadas lo son. Esto no es cierto en general si las variedades son sólo cuasi-afines, como muestra el primer ejemplo siguiente:

\begin{ejs}
 \emph{
 \begin{enumerate}
\item Si $\phi:\A^2_\C-\{(0,0)\}\to\A^2_\C$ es el morfismo dado por la inclusión, $\phi^\star:\C[x,y]\to\C[x,y]$ es un isomorfismo de $\C$-álgebras (la identidad) por el ejercicio \ref{a2menos0}.
  \item El morfismo $\phi$ del ejemplo \ref{ejemorf} (1) es un isomorfismo: su inverso viene dado por $(x,y)\mapsto x$. 
\item Si $Z\subseteq\Ank$ es una hipersuperficie, la variedad cuasi-afín $\Ank-Z$ es isomorfa a una variedad afín: en efecto, si $Z$ viene definida por la ecuación $f=0$, el morfismo $(x_1,\ldots,x_n)\mapsto(x_1,\ldots,x_n,1/f(x_1,\ldots,x_n))$ es un isomorfismo entre $\Ank-Z$ y la hipersuperficie afín de $\A^{n+1}_k$ definida por la ecuación $x_{n+1}f(x_1,\ldots,x_n)-1=0$.
 \end{enumerate}
}
\end{ejs}

Un isomorfismo es claramente una biyección pero, al contrario de lo que ocurre con los homomorfismos entre objetos algebraicos (grupos, anillos, módulos), un morfismo biyectivo no es necesariamente un isomorfismo:

\begin{ejer}
 Sea $\phi:\A^1_\C\to\V(\{y^2-x^3\})\subseteq\A^2_\C$ el morfismo dado por $t\mapsto (t^2,t^3)$. Probar que $\phi$ es biyectivo, incluso un homeomorfismo, pero no es un isomorfismo.
\end{ejer}

\section{Aplicaciones racionales}

Las aplicaciones racionales se definen a partir de los morfismos exactamente igual que las funciones racionales se definieron a partir de las funciones regulares:

\begin{defi}
 Sean $Y\subseteq\Ank$ y $Z\subseteq\Amk$ dos variedades cuasi-afines. Una {\bf aplicación racional} $\psi:Y\dashrightarrow Z$ es un par $(U,\phi)$ donde $U\subseteq Y$ es un abierto no vacío y $\phi:U\to Z$ un morfismo, módulo la relación de equivalencia siguiente: $(U,\phi)\sim(U',\phi')$ si $\phi_{|U\cap U'}=\phi'_{|U\cap U'}$.
\end{defi}

Como es habitual, si no se especifica $U$ se sobreentiende que es el mayor abierto donde la expresión dada para $\psi$ tenga sentido.

\begin{defi}
 Se dice que una aplicación racional $\psi:Y\dashrightarrow Z$ {\bf está definida en $y\in Y$} si existe un abierto $U\subseteq Y$ con $y\in U$ y un morfismo $\phi:U\to Z$ tales que $\psi=(U,\phi)$. El conjunto de puntos de $Y$ donde $\psi$ está definida es un abierto llamado el {\bf abierto de definición} de $\psi$.
\end{defi}

\begin{defi}
 Una aplicación racional $\psi=(U,\phi):Y\dashrightarrow Z$ se dice {\bf dominante} si $\phi$ lo es.
\end{defi}

\begin{ejer}
 Probar que la definición anterior no depende del par $(U,\phi)$ elegido para representar $\psi$. Es decir, probar que si $(U,\phi)\sim(U',\phi')$, entonces $\phi$ es dominante si y sólo si $\phi'$ lo es.
\end{ejer}

Toda aplicación racional dominante $\psi:Y\dashrightarrow Z$ induce un homomorfismo de $k$-álgebras $\psi^\star:\KK(Z)\to\KK(Y)$ (es decir, un homomorfismo de cuerpos que actúa trivialmente en $k$), que es simplemente el inducido por $\phi$ si $\psi=(U,\phi)$. Veamos que el recíproco es cierto:

\begin{prop}\label{morfismoaracional}
 Dado un homomorfismo de $k$-álgebras $\xi:\KK(Z)\to\KK(Y)$, existe una única aplicación racional dominante $\psi:Y\dashrightarrow Z$ tal que $\psi^\star=\xi$.
\end{prop}

\begin{proof}
 Toda aplicación racional dominante $\psi:Y\dashrightarrow Z$ es también una aplicación racional dominante $\psi:\overline Y\dashrightarrow\overline Z$ y viceversa (por restricción a $Y\cap\psi^{-1}(Z)$), así que podemos suponer que $Y$ y $Z$ son afines. Para cada $i=1,\ldots,m$, sea $f_i=\xi(x_i)\in\KK(Y)$, y sea $U_i\subseteq Y$ un abierto donde $f_i$ esté definida. Veamos que $\psi=(U,\phi)$ es la aplicación racional buscada, donde $U=U_1\cap\cdots\cap U_m$ y $\phi=(f_1,\ldots,f_m): U\to\Ank$. Por construcción, $\psi^\star(x_i)=f_i=\xi(x_i)$, y por tanto $\psi^\star=\xi$, ya que los $x_i$ generan $\KK(Z)$ como cuerpo sobre $k$.

 Veamos que la imagen de $\phi$ está contenida y es densa en $Z$. Como $\overline{\phi(U)}=\V(\I(\phi(U)))$, basta probar que $\I(\phi(U))=\I(Z)$. Sea $g\in\I(Z)$, entonces $g=0$ en $\KK(Z)$, y por tanto $\xi(g)=0$, es decir $g\circ\psi=0$ como función racional en $Y$. En particular $g\circ\phi=0$ en un abierto no vacío de $U$, y por tanto en todo $U$ (ejercicio \ref{extensionfuncion}). Recíprocamente, si $g\in\I(\phi(U))$, $g\circ\phi(y)=0$ para todo $y\in U$, y por tanto $\xi(g)=g\circ\psi=0$ como función racional. Como $\xi$ es inyectiva, concluimos que $g=0$ en $\KK(Z)$, y por tanto en $\calA(Z)$ (de nuevo por el ejercicio \ref{extensionfuncion}).
\end{proof}

En general, la composición de dos aplicaciones racionales $\psi:Y\dashrightarrow Z$ y $\psi':X\dashrightarrow Y$ no está definida, ya que $\psi$ puede no estar definida en ningún punto de la imagen de $\psi'$: por ejemplo, si $\psi:\A^2_\C\dashrightarrow\A^2_\C$ viene dada por $\psi(x,y)=(x/y,0)$ la composición $\psi\circ \psi$  no está definida. Sin embargo, si $\psi'$ es dominante sí es posible definir la composición (ejercicio \ref{composicion_racional}). Además, con las notaciones anteriores, $(\psi\circ\psi')^\star=\psi'^\star\circ\psi^\star$. La proposición anterior nos dice que:

\begin{prop}
  Las aplicaciones $Z\mapsto \KK(Z)$, $\psi\mapsto \psi^\star$ inducen una equivalencia de categorías entre la categoría de variedades afines y aplicaciones racionales dominantes y la categoría opuesta a la de cuerpos finitamente generados sobre $k$ y $k$-homomorfismos de cuerpos.
\end{prop}

\begin{proof}
 Lo único que falta probar es que todo cuerpo $K$ finitamente generado sobre $k$ es el cuerpo de funciones de alguna variedad afín. Sean $x_1,\ldots,x_n$ generadores de $K$ sobre $k$. La $k$-álgebra $k[x_1,\ldots,x_n]\subseteq K$ es el anillo de coordenadas de algún conjunto algebraico (ejercicio \ref{full}) que necesariamente ha de ser una variedad $X$, porque todo subanillo de un cuerpo es un dominio de integridad. Por tanto, $K=K(\calA(X))=\KK(X)$.
\end{proof}


\section{Variedades birracionalmente equivalentes}


\begin{defi}
 Sean $Y$ y $Z$ variedades cuasi-afines. Una aplicación racional $\psi:Y\dashrightarrow Z$ se dice {\bf birracional} si existe una aplicación racional $\psi':Z\dashrightarrow Y$ tal que $\psi'\circ\psi:Y\dashrightarrow Y$ y $\psi\circ\psi':Z\dashrightarrow Z$ son la identidad (como aplicaciones racionales). Las variedades $Y$ y $Z$ se dicen {\bf birracionalmente equivalentes} si existe una aplicación birracional $\psi:Y\dashrightarrow Z$. Una variedad $Y$ se dice {\bf racional} si es birracionalmente equivalente a $\Ank$ para algún $n$.
\end{defi}

 \begin{ejs}\emph{
  Sea $Z\subseteq\A^3_\C$ la cuádrica definida por la ecuación $z^2=xy$. Las aplicaciones racionales $\A^2_\C\to Z$ dada por $(a,b)\mapsto(\frac{b^2}{b^2-a},\frac{a^2}{b^2-a},\frac{ab}{b^2-a})$ y $Z\to\A^2_\C$ dada por $(x,y,z)\mapsto (\frac{y}{x-1},\frac{z}{x-1})$ son inversas la una de la otra, y por tanto son birracionales. En particular, $Z$ es racional.}
 \end{ejs}

\begin{prop}\label{birracional}
 Sean $Y$ y $Z$ variedades cuasi-afines. Las siguientes condiciones son equivalentes:

a) $Y$ y $Z$ son birracionalmente equivalentes.

b) Existen abiertos $U\subseteq Y$ y $V\subseteq Z$ isomorfos.

c) $\KK(Y)$ y $\KK(Z)$ son isomorfos. 

\end{prop}

\begin{proof}
 Veamos que $(a)\Rightarrow(b)$. Sea $\phi:Y\dashrightarrow Z$ birracional, y $\psi:Z\dashrightarrow Y$ su inversa. Sea $U'\subseteq Y$ (respectivamente $V'\subseteq Z$) un abierto donde $\phi$ (resp. $\psi$) esté definida. Entonces $\psi\circ\phi$ está definida (y es igual a la identidad) en $U'\cap \phi^{-1}(V')$. Análogamente, $\phi\circ\psi$ está definida (y es la identidad) en $V'\cap\psi^{-1}(U')$. Los abiertos buscados son entonces $U=\phi^{-1}(V'\cap\psi^{-1}(U'))$ y $V=\psi^{-1}(U'\cap \phi^{-1}(V'))$.

Si $U\subseteq Y$ y $V\subseteq Z$ son abiertos isomorfos, sus anillos de funciones regulares también son isomorfos. En particular, también lo son sus cuerpos de fracciones, que son justamente $\KK(Y)$ y $\KK(Z)$. 

Finalmente, $(c)\Rightarrow(a)$ es una consecuencia directa de la proposición \ref{morfismoaracional}.
\end{proof}

\begin{coro}
 Una variedad cuasi-afín $Z$ es racional si y sólo si su cuerpo de funciones es isomorfo a $k(x_1,\ldots,x_n)$ para algún $n$.
\end{coro}

\begin{coro}\label{behipersuperficie}
 Toda variedad afín $Z$ de dimensión $d$ es birracionalmente equivalente a una hipersuperficie en $\A^{d+1}_k$
\end{coro}

\begin{proof} Por el teorema de estructura de los cuerpos finitamente generados, $\KK(Z)$ se pude expresar como una extensión finita (y por tanto algebraica) de $k(x_1,\ldots,x_d)$. Es decir, existe un polinomio $f\in k(x_1,\ldots,x_d)[y]$ tal que $\KK(Z)=k(x_1,\ldots,x_d)[y]/\langle f\rangle$. Quitando denominadores, podemos escribir $f=g/h$, con $g,h$ sin factores comunes y $g\in k[x_1,\ldots,x_d,y]$. La hipersuperficie definida por $g$ en $\A^{d+1}_k$ es entonces birracionalmente equivalente a $Z$, ya que su cuerpo de funciones es el cuerpo de fracciones de $k[x_1,\ldots,x_d,y]/\langle g\rangle$, que es isomorfo a $\KK(Z)$.
\end{proof}





 




\section{Morfismos entre variedades cuasi-proyectivas}

Sea $X\subseteq\Pnk$ una variedad cuasi-proyectiva. La definición de morfismo de $X$ en el espacio afín es similar a la del caso de una variedad cuasi-afín:

\begin{defi}
 Una aplicación $\phi:X\to \Amk$ se dice un {\bf morfismo} si existen $m$ funciones regulares $f_1,\ldots,f_m\in\OO(X)$ tales que $\phi(x)=(f_1(x),\ldots,f_m(x))$ para todo $x\in X$. Si $Y\subseteq\Amk$ es una variedad cuasi-afín, un morfismo $\phi:X\to Y$ es un morfismo $\phi:X\to\Amk$ tal que $\phi(X)\subseteq Y$.
\end{defi}

\begin{ejer}
 Si $X\subseteq\Ank$ es una variedad cuasi-afín, probar que $\phi:\theta_0(X)\to\Amk$ es un morfismo según esta definición si y sólo si $\phi\circ\theta_0:X\to\Amk$ es un morfismo según la definición del tema anterior. Por tanto las dos definiciones de morfismo son compatibles para variedades cuasi-afines.
\end{ejer}

La definición de los morfismos hacia el espacio proyectivo es un poco más delicada. En las definiciones de morfismo que hemos visto hasta ahora la condición que tiene que cumplir una aplicación $\phi:X\to Y$ para serlo es local. Es decir, es una condición (el que cada coordenada de $\phi$ sea regular) que tiene que cumplirse en un entorno de cada punto de $X$. En particular, si $X=\bigcup_{i\in I} U_i$ es un recubrimiento abierto de $X$, la aplicación $\phi$ es un morfismo si y sólo si cada restricción $\phi_{|U_i}$ lo es. Si queremos que los morfismos hacia el espacio proyectivo sigan verificando esta propiedad (y sean compatibles con las definiciones anteriores) sólo hay una forma posible de definirlos:

\begin{defi}
 Una aplicación $\phi:X\to\Pmk$ se dice un ${\bf morfismo}$ si es continua y, para todo $0\leq i\leq m$, la restricción $\phi:\phi^{-1}(U_i)\to U_i$ es un morfismo entre $\phi^{-1}(U_i)$ y el espacio afín $U_i$ (usando las notaciones de la sección \ref{recubrimientoafin}). Si $Y\subseteq\Pmk$ es una variedad cuasi-proyectiva, un morfismo $\phi:X\to Y$ es un morfismo $\phi:X\to\Pmk$ tal que $\phi(X)\subseteq Y$.
\end{defi}
 
En la práctica, para construir morfismos entre variedades cuasi-proyectivas usaremos el siguiente resultado más explícito:

\begin{prop}
 Sean $X\subseteq\Pnk$ e $Y\subseteq\Pmk$ variedades cuasi-proyectivas. Una aplicación $\phi:X\to Y$ es un morfismo si y sólo si para todo $z\in X$ existen un abierto $U\subseteq X$ conteniendo a $z$ y $m+1$ polinomios homogéneos $f_0,f_1,\ldots,f_m\in k[x_0,x_1,\ldots,x_n]$ \emph{del mismo grado} que no se anulan simultáneamente en ningún punto de $U$ tales que $\phi(x_0:x_1:\ldots:x_n)=(f_0(x_0,\ldots,x_n):f_1(x_0,\ldots,x_n):\ldots:f_m(x_0,\ldots,x_n))$ para todo $x=(x_0:x_1:\ldots:x_n)\in U$.
\end{prop}

Es importante notar que, aunque el valor de $f_i(x_0,\ldots,x_n)$ no está determinado por el punto $x=(x_0:x_1:\ldots:x_n)$ (ya que si elegimos otras coordenadas para $x$ cambiará el valor de $f_i$), el \emph{punto proyectivo} $(f_0(x_0,\ldots,x_n):f_1(x_0,\ldots,x_n):\ldots:f_m(x_0,\ldots,x_n))$ sí lo está: si elegimos otras coordenadas $(y_0:y_1:\ldots:y_n)$ para $x$, existirá un $\lambda\neq 0$ tal que $y_i=\lambda x_i$ para todo $i$. Como los $f_i$ son homogéneos del mismo grado $d$, se tendrá entonces que $f_i(y_0,\ldots,y_n)=\lambda^d f_i(x_0,\ldots,x_n)$ para todo $i$, y por tanto $(f_0(x_0,\ldots,x_n):f_1(x_0,\ldots,x_n):\ldots:f_m(x_0,\ldots,x_n))=(f_0(y_0,\ldots,y_n):f_1(y_0,\ldots,y_n):\ldots:f_m(y_0,\ldots,y_n))$.

\begin{proof}
 Sea $\phi:X\to Y$ un morfismo. Los abiertos $V_i:=\phi^{-1}(U_i)$ recubren $X$, construiremos las funciones $f_0,f_1,\ldots,f_m$ localmente en cada $V_i$. Sea por ejemplo $i=0$ (la construcción es similar para todo $i$). Como $\phi:V_0\to U_0$ es un morfismo, para todo $z\in V_0$ existe un entorno $z\in V\subseteq V_0$ tal que para todo $x\in V$ $\phi(x)$ (visto como elemento de $U_0\subseteq\Pmk$) viene dado por $(1:g_1(x)/h_1(x):\ldots:g_m(x)/h_m(x))$, donde para cada $j=1,\ldots,m$ $g_j$ y $h_j$ son polinomios homogéneos del mismo grado y $h_j$ no se anula en $V$. Multiplicando todas las coordenadas por el producto de los denominadores (que no se anula en $V$) obtenemos que $\phi(x)=(f_0(x):f_1(x):\ldots:f_m(x))$, donde $f_0=h_1\cdots h_m$ y $f_j=g_jh_1\cdots h_m/h_j$ para $j=1,\ldots,m$ son polinomios homogéneos del mismo grado y $f_0$ no se anula en $V$.

Recíprocamente, supongamos que se cumple la condición del enunciado. En primer lugar, $\phi$ debe ser continua, ya que la continuidad es una propiedad local y $\phi$ es continua en cada abierto $U$ en el que pueda escribirse como $(f_0:f_1:\ldots:f_m)$ (ejercicio). Veamos que $\phi:V_0\to U_0$ es un morfismo. Sea $z\in V_0$, por hipótesis existe un entorno $V\subseteq V_0$ de $z$ y $m+1$ polinomios homogéneos $f_0,f_1,\ldots,f_m\in k[x_0,x_1,\ldots,x_n]$ del mismo grado tales que $\phi(x)=(f_0(x):f_1(x):\ldots:f_m(x))$ para todo $x\in V$. Como $\phi(x)\in U_0$ que, por definición, es el abierto en el que $x_0\neq 0$, se tiene que $f_0$ no se anula en $V$. Por tanto podemos escribir $\phi(x)=(1:f_1(x)/f_0(x):\ldots:f_m(x)/f_0(x))$. Visto como punto afín, $\phi(x)$ viene entonces dado por $(f_1(x)/f_0(x),\ldots,f_m(x)/f_0(x))$. Cada coordenada viene dada por el cociente de dos polinomios homogéneos del mismo grado, y por tanto $\phi$ es un morfismo en $V$.

De la misma forma se prueba que $\phi:V_j\to U_j$ es un morfismo para todo $j=1,\ldots,m$, y por tanto $\phi:X\to Y$ es un morfismo según la definición anterior.
\end{proof}

Al definir un morfismo $\phi:X\to Y$ de esta forma, hay que tener cuidado de que la definición sea compatible: si $\phi=(f_0:f_1:\ldots:f_m)$ en un abierto $U$ y $\phi=(g_0:g_1:\ldots:g_m)$ en otro abierto $V$, se tiene que cumplir que $(f_0:f_1:\ldots:f_m)=(g_0:g_1:\ldots:g_m)$ en $U\cap V$, es decir, que $f_ig_j=f_jg_i$ para todos $0\leq i<j\leq m$ en todo punto de $U\cap V$. 

\begin{ejs}
 \emph{Sea $X\subseteq\PP^3_k$ la variedad definida por $x_0x_1-x_2x_3=0$. Las aplicaciones $(x_0:x_1:x_2:x_3)\mapsto (x_0:x_3)$ y $(x_0:x_1:x_2:x_3)\mapsto (x_2:x_1)$ definen morfismos $X\cap (U_0\cup U_3)\to\PP^1_k$ y $X\cap (U_1\cup U_2)\to\PP^1_k$, que se ``pegan'' para dar como resultado un morfismo $\phi:X\to\PP^1_k$, ya que en la intersección de los dominos se tiene que $x_0x_1=x_2x_3$, y por tanto $(x_0:x_3)=(x_1:x_2)$.
}
\end{ejs}

\begin{ejer}
 Probar que la composición de morfismos entre variedades cuasi-proyectivas es un morfismo.
\end{ejer}


\begin{defi}
 Un morfismo $\phi:X\to Y$ entre variedades cuasi-proyectivas se dice un {\bf isomorfismo} si existe un morfismo $\psi:Y\to X$ tal que $\psi\circ\phi$ y $\phi\circ\psi$ sean la identidad. En tal caso se dice que las variedades $X$ e $Y$ son {\bf isomorfas}.
\end{defi}

\begin{ejs}
 \emph{Sea $X\subseteq\PP^2_k$ la variedad definida por $x_1^2+x_2^2-x_0^2=0$ (la clausura proyectiva de la cónica $x^2+y^2-1=0$). Definimos $\phi(x_0:x_1)=(x_1^2+x_0^2:x_1^2-x_0^2:2x_0x_1)$. Como $x_1^2+x_0^2$ y $x_1^2-x_0^2$ no se anulan simultáneamente, $\phi$ define un morfismo $\PP^1_k\to X$. El morfismo $\psi:X\to\PP^1_k$ dado por $\psi(x_0:x_1:x_2)=(x_0-x_1:x_2)$ en $X-\{(1:1:0)\}$ y por $\psi(x_0:x_1:x_2)=(x_2:x_0+x_1)$ en $X-\{(1:-1:0)\}$ es inverso de $\phi$, y por tanto $X$ es isomorfa a $\PP^1_k$.}
\end{ejs}

\section{Aplicaciones racionales entre variedades cuasi-proyectivas}

La definición de aplicación racional para variedades cuasi-proyectivas es la misma que para variedades cuasi-afines:

\begin{defi}
 Sean $X\subseteq\Pnk$ e $Y\subseteq\Pmk$ dos variedades cuasi-proyectivas. Una {\bf aplicación racional} $\psi:X\dashrightarrow Y$ es un par $(U,\phi)$ donde $U\subseteq X$ es un abierto no vacío y $\phi:U\to Y$ un morfismo, módulo la relación de equivalencia siguiente: $(U,\phi)\sim(U',\phi')$ si $\phi_{|U\cap U'}=\phi'_{|U\cap U'}$.
\end{defi}

Como $(U,\phi)\sim (U',\phi_{|U'})$ para todo abierto no vacío $U'\subseteq U$, podemos suponer que $\phi(U)\subseteq Z_i$ para algún $i$ (donde $Z_i=Z\cap U_i$ según la notación de la sección \ref{recubrimientoafin}). Por tanto, una función racional $\psi$ viene dada por un abierto $U\subseteq X$ y $m$ funciones regulares $f_1/g_1,\ldots,f_m/g_m$ en $U$ (donde para cada $j$, $f_j$ y $g_j$ son polinomios homogéneos del mismo grado) mediante la fórmula $\psi(x)=(1:f_1(x)/g_1(x):\ldots:f_m(x)/g_m(x))$ (en el caso $i=0$, en general el $1$ iría en la posición $i$-ésima). Multiplicando por el producto de los $g_j$, obtenemos $\psi(x)=(g(x):g(x)f_1(x)/g_1(x):\ldots:g(x)f_m(x)/g_m(x))$, donde $g$ y cada $gf_j/g_j$ son polinomios homogéneos del mismo grado. En resumen:

\begin{prop}
 Una aplicación racional $\psi:X\dashrightarrow Y$ viene dada por $m+1$ polinomios $f_0,f_1,\ldots,f_m\in k[x_0,x_1,\ldots,x_n]$ homogéneos del mismo grado tales que $\langle f_0,f_1,\ldots,f_m\rangle\not\subseteq\I(X)$, mediante la fórmula $\psi(x_0:\ldots:x_n)=(f_0(x_0,\ldots,x_n):\ldots:f_m(x_0,\ldots,x_n))$. La aplicación $\psi$ está definida en el conjunto $X\backslash\V(\{f_0,\ldots,f_m\})$ de puntos $x\in X$ en los que algún $f_j$ no se anula. Otro conjunto de polinomios $g_0,g_1,\ldots,g_m$ homogéneos del mismo grado define la misma función racional si y sólo si $f_ig_j-f_jg_i=0$ en $X$ para todos $0\leq i<j\leq m$. 
\end{prop}

\begin{proof}
 Sólo falta por probar la última afirmación. Supongamos que $(f_0:\ldots:f_m)$ y $(g_0:\ldots:g_m)$ definen la misma aplicación racional, es decir, las dos funciones coinciden en el abierto $U$ en el que ambas están definidas. Entonces para todo $x=(x_0:\ldots:x_n)\in U$ se tiene que $(f_0(x_0,\ldots,x_n):\ldots:f_m(x_0,\ldots,x_n))=(g_0(x_0,\ldots,x_n):\ldots:g_m(x_0,\ldots,x_n))$, es decir, existe $\lambda\neq 0$ tal que $g_j(x_0,\ldots,x_n)=\lambda f_j(x_0,\ldots,x_n)$ para todo $j$. En particular, $f_i(x_0,\ldots,x_n)g_j(x_0,\ldots,x_n)-f_j(x_0,\ldots,x_n)g_i(x_0,\ldots,x_n)=\lambda(f_i(x_0,\ldots,x_n)f_j(x_0,\ldots,x_n)-f_j(x_0,\ldots,x_n)f_i(x_0,\ldots,x_n))=0$. Como $f_ig_j-f_jg_i=0$ en $U$, que es un abierto denso de $X$, se tiene que $f_ig_j-f_jg_i=0$ en todo $X$.
 
Recíprocamente, supongamos que $f_ig_j-f_jg_i=0$ en $X$ para todos $0\leq i,j\leq m$, y sea $x=(x_0:\ldots:x_n)\in X$ un punto en el que $(f_0:\ldots:f_m)$ y $(g_0:\ldots:g_m)$ estén definidas, es decir, que al menos un $f_{i_0}$ y al menos un $g_{j_0}$ no se anulen en $x$. Entonces para todo $j$ se tiene que $f_{i_0}g_j-f_jg_{i_0}=0$, es decir, $g_j(x_0,\ldots,x_n)=[g_{i_0}(x_0,\ldots,x_n)/f_{i_0}(x_0,\ldots,x_n)]f_j(x_0,\ldots,x_n)$. En primer lugar, haciendo $j=j_0$ vemos que $\lambda:=g_{i_0}(x_0,\ldots,x_n)/f_{i_0}(x_0,\ldots,x_n)$ no se anula, y entonces $g_j(x_0,\ldots,x_n)=\lambda f_j(x_0,\ldots,x_n)$ para todo $j$. Por tanto, $(f_0(x_0,\ldots,x_n):\ldots:f_m(x_0,\ldots,x_n))=(g_0(x_0,\ldots,x_n):\ldots:g_m(x_0,\ldots,x_n))$.
\end{proof}

Terminamos repitiendo las siguientes definiciones y resultados del tema anterior, enunciados para el caso proyectivo:

\begin{defi}
 Sean $X$ e $Y$ variedades cuasi-proyectivas. Una aplicación racional $\psi:X\dashrightarrow Y$ se dice {\bf birracional} si existe una aplicación racional $\psi':Y\dashrightarrow X$ tal que $\psi'\circ\psi:X\dashrightarrow X$ y $\psi\circ\psi':Y\dashrightarrow Y$ son la identidad (como aplicaciones racionales). En tal caso, las variedades $X$ e $Y$ se dicen {\bf birracionalmente equivalentes}. Una variedad cuasi-proyectiva $X$ se dice {\bf racional} si es birracionalmente equivalente a $\Ank$ (o lo que es lo mismo, a $\Pnk$) para algún $n$.
\end{defi}

\begin{prop}\label{birracproy}
 Sean $X$ e $Y$ variedades cuasi-proyectivas. Las siguientes condiciones son equivalentes:

a) $X$ e $Y$ son birracionalmente equivalentes.

b) Existen abiertos $U\subseteq X$ y $V\subseteq Y$ isomorfos.

c) $\KK(X)$ y $\KK(Y)$ son isomorfos. 

\end{prop}

La prueba es la misma que en el caso afín (proposición \ref{birracional}).


\end{document}
