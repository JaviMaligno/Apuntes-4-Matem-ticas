\documentclass[twoside]{article}
\usepackage{../estilo-ejercicios}

%--------------------------------------------------------
\begin{document}

\title{Algebra Conmutativa y Geometría Aplicada}
\author{Rafael González López}
\maketitle

\begin{ejercicio}{1}
Describir todos los subconjuntos algebraicos de $\mathbb{A}_k^1$.
\begin{solucion}
Quitando el caso $X=\mathbb{A}_k^1$, veamos que el resto de conjuntos algebraicos son precisamente los conjuntos finitos. Sabemos que los conjuntos finitos son siempre algebraicos. Si existiese un conjunto algebraico infinito $X$, entonces habría de existir $f \in k[x]$ tal que $V(f)=X$. Además $f$ habría de anularse en cada punto de $X$, pero como $X$ es infinito, esto implica que $f\equiv 0$ y $X=\mathbb{A}_k^1$.
\end{solucion}
\end{ejercicio}

\newpage 
\begin{ejercicio}{2}
Probar que en $\mathbb{A}^n_k$ con la topología de Zariski, dos abiertos no vacíos tienen intersección no vacía (todo abierto es denso).
\begin{solucion}
Por reducción al absurdo, supongamos que $\exists A,B$ abiertos no vacíos de $\mathbb{A}_k^n$ con la topología de Zariski, tales que $A\cap B =\emptyset$. Esto es equivalente a que $A^c \cup B^c = \mathbb{A}_k^n$. Podemos suponer que $A\neq B$, pues el otro caso es trivial. Sabemos además que existen $f,g\in k[\xn{n}]$, $f\neq g$, polinomios tales que $\V(f) = A^c$ y $\V(g)=B^c$. En tal caso $\V(f)\cup \V(g)=\V(fg) = \mathbb{A}^n_k$, pero esto solo puede ser si $fg = 0$ $\forall x\in \mathbb{A}^n_k$, pero al estar en un cuerpo algebraicamente cerrado (y por tanto infinito), $fg\equiv 0$. Dado que $k[\xn{n}]$ es dominio, se tiene que, o bien $f\equiv 0$, o bien $g \equiv 0$. Sin pérdida de generalidad, supongamos que es $f$. Entones $A^c = \V(f) = \mathbb{A}_k^n$, luego $A=\emptyset$, lo cual contradice las hipótesis.
\end{solucion}
\end{ejercicio}

\newpage 
\begin{ejercicio}{3}
Probar que $\I(X)$ es un ideal radical de $k[\xn{n}]$.
\begin{solucion}
Que $\I(X)$ es una ideal es claro, pues sean $f,g \in \I(X)$ entonces $f(x)+g(x)=0$ $\forall x\in X$. Además, si $f\in \I(X)$ y $g \in k[\xn{x}]$, entonces $f(x)g(x)=0$ $\forall x\in X$. El hecho de ser radical se desprende de que si $f^n\in \I(X)$ entonces $f^n(x) =0$ $\forall x \in X$, pero como estamos en un dominio, se tiene que $f(x)=0$ $\forall x\in X$, luego $f\in \I(X)$.
\end{solucion}
\end{ejercicio}	


\newpage 
\begin{ejercicio}{4}
Sea $S\subset k[\xn{n}]$. Probar que $\sqrt{\langle{S}\rangle}\subset \I(\V(S))$.
\begin{solucion}
Sea $f\in \sqrt{\langle{S}\rangle}$. Entonces $\exists n \in \N$ tal que $f^n \in \gene{S}$. En particular, $\exists f_1,\dotsc,f_k\in S$, $g_1,\dotsc,g_k \in k[\xn{n}]$ tales que
$$
f^n = f_1 g_1 + \dotsc + f_k g_k 
$$
Dado que $f_i\in S$, $f_i(x)=0$ $\forall x \in \V(S)$. Por tanto, $f^n(x) = 0$ $\forall x \in \V(s)$. Por definición
$$
\I(\V(S)) =  \{ h \in k[\xn{n}] \mid h(x)=0 \; \forall x\in \V(S)\}
$$
Por tanto, $f^n \in \I(\V(S))$. Pero por el Ejercicio 3, sabemos que $\I(\V(S))$ es un ideal radical, luego $f\in \I(\V(S))$, como queríamos probar.
\end{solucion}
\end{ejercicio}	
\end{document}