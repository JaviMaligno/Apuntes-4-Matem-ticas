\documentclass[ACGA.tex]{subfiles}








\begin{document}

\chapter{Proyectivo}

\section{No sé qué cojones poner aquí\\ Un saludo a FC}
El proyectivo posee diferencias sutanciales con respecto al espacio afín. En el proyectivo se verifica que
$$
\dim(L_1 + L_2) + \dim(L_1 \cap L_2) = \dim(L_1) + \dim(L_2),
$$
a diferencia de en los espacios afines, donde hay que distinguir si se cortan o no se cortan las variedades. 

Dos rectas siempre se cortan en el plano proyectivo en al menos un punto, así como una recta y una cónica se cortan en dos puntos. En general, si tenemos una curva definida por un polinomio de grado $d$ y una recta cualquiera siempre se cortan en $d$ puntos (contando multiplicidad).

Algunas cosas que conviene recordar. $\PP^n = \PP^{n-1}\cup \A^n$

Las raíces de un polinomio no están definidas en el proyectivo, pues $(1:1:1)$ es raíz de $x_0 + x_1^2 - 2x_0x_1$, pero no $(2:2:2)$. Consideramos entonces los polinomios homogéneos. 

\begin{defi}
Diremos que un conjunto $X\subset \Pnk$ es algebraico si existe $S\subset k[x_0,\xn{n}]$ un conjunto de polinomios homogéneos tal que $X=\{x\in\Pnk \mid f(x)=0 \; \forall f\in S\}$.
\end{defi}
Como siempre $\emptyset = \V(1)$ y $\Pnk = \V(0)$. $\V(\{a_0,\dotsc,a_n\}) = \{a_ix_j - a_j x_i \mid 0\leq i \leq j \leq n\}$. Cualquier variedad proyectiva lineal es un conjunto algebraico. Las cónicas, cuádricas e hipercuádricas son conjuntos algebraicos.

Si $S\subset k[x_0,\xn{n}]$ es un subcojunto arbitrario de polinomios entonces definimos $\V(S)$ como $\V(S\cap k[x_0,\xn{n}]^h)$. 

\begin{prop}
Si $S\subset k[x_0,\xn{n}]^h$, entonces $\V(S)=\V(\gene{S})$. 
\end{prop}
\begin{dem}
Una inclusión es trivial. Para la otra tenemos que ver que si $f\in \gene{S}$ homogéneo, entonces $f=\sum g_i f_i$ con $f_i \in S$. Si $h_i$ es la componente homogénea de $g_i$ de grado $d-d_i$ entonces $\sum g_i f_i = \sum h_i f_i$. Si $x\in \V(S)$ entonces $f_i(x)=0$ $\forall i$, por lo que $\sum h_i (x) f_i (x) =0$. 
\end{dem}
\begin{defi}
Un ideal $I\subset k[x_0,\xn{n}]$ es homogéneo si se puede generar por un conjunto de polinomios homogéneos. 
\end{defi}
\begin{prop}
Todo conjunto algebraico de $\Pnk$ se puede definir mediante un conjunto finito de polinomios homogéneos. Basta tomar todas las componentes homogéneas de un conjunto de generadores.
\end{prop}
\begin{prop} La unión finita y la intersección arbitraria de conjuntos arbitrarios es algebraica, dando lugar a la topología de Zariski.
\end{prop}
¿Se sigue cumplliendo que existe una correspondencia biunívoca entre los conjuntos algebraicos de $\Pnk$ y los ideales radicales homogéneos de $k[x_0,\xn{n}]$?
Sea $x$ es algebraico entonces 
$$\I(X)=\gene{\{f\in k[x_0,\xn{n}]^h \mid f(x)=0\}}$$.
\begin{prop}
Si $\V(X)=0$, entonces $\sqrt{I}=\gene{1}$ o bien $\sqrt{I}=\gene{x_0,\dotsc,x_n}$.
\end{prop}
\begin{dem}
Pensemos primero en el caso afín. Si $\Va(I)\subset \A^{n+1}$ entonces $\Va(I)\subset\{(0,\dotsc,0)\}$, por lo que $\gene{x_0,\dotsc,x_n}\subset \sqrt{I}$.
\end{dem}
Sea $I$ homogéneo tal que $X=\Vp(I)\neq \emptyset$. Entonces, si $\pi$ es la aplicación que pasa de $\A^{n+1}$ a $\PP^n$, si tenemos un conjunto algebraico $X$ en $\PP^n$, $\{0\}\cup\pi^{-1}(X)$ es el cono afín de $X$. Tenemos que el cono afín de $X$ es precisamente $\Va(I)$.

Sea $f\in \I_p(\Vp(I))$ homogéneo, entonces $f$ se anula en todo punto de $\Va(I)$ (por ser homogéneo y no constante). Por tanto, sabemos que $f\in \I_a(\Va(I))$ y, por el Teorema de los ceros en el caso afín, deducimos que $f\in \sqrt{I}$. Por tanto, $\I_p(\Vp(I))=\sqrt{I}$.

Tenemos definiciones análogas de conjuntos algebraicos irreducibles y equivalencia entre irreducible e $\I(X)$ primo, descomposición de conjuntos algebraicos en irreducibles de manera única, dimensión de un conjunto algebraico, etc.

\begin{ejer}
Intentemos descomponer $\V(x_0^2-x_1x_2,x_0x_1-x_2x_3)$. Claramente $\V(x_0,x_2)$ está contenido. Quitando ese caso, podemos suponer que $x_2\neq 0$, por lo que podemos trabajar en el afín con $x_2=1$, quedando el sistema: $x_0^2-x_1=0$ y $x_0x_1-x_3=0$. Esto nos da el sistema $x_0 = t$, $x_1=t^2$, $x_3=t^3$. Por tanto, en el espacio afín esto sería $\V(x_0^2-x_1,x_0^3-x_3)$. Homogeneizando, $\V(x_0^2-x_1x_2,x_0^3-x_3x_2^2)$. 

Dos polinomios de grado d y e en el plano proyectivo se cortan en de puntos.

Si tienes un ideal generado por r polinomio es en P^n entonces cualquier componente irreducible tiene como mínimo n-r. 
\end{ejer}
\end{document}
