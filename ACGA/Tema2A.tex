\documentclass[ACGA.tex]{subfiles}
%\usepackage[utf8x]{inputenc}
%\usepackage[spanish]{babel}
%\usepackage{amsmath, amssymb, amsthm, epsf, graphicx, amscd, amsfonts}
%\usepackage[colorlinks]{hyperref}
%\usepackage{xmpincl}
%\usepackage{fancyhdr}
%
%
%\pagestyle{fancy}
%\lhead[\thepage]{\rightmark}
%\rhead[\leftmark]{\thepage}
%\cfoot[]{}
%
%\addto\captionsspanish{ \renewcommand{\chaptername}{Tema} }
%
%\newtheorem{thm}{Teorema}[chapter]
%\newtheorem{cor}[thm]{Corolario}
%\newtheorem{lem}[thm]{Lema}
%\newtheorem{prop}[thm]{Proposición}
%\newtheorem{defn}[thm]{Definición}
%\newtheorem{rem}[thm]{Observaciones}
%\newtheorem{eje}[thm]{Ejemplos}
%
%\newtheorem{ejercicio}{Ejercicio}[chapter]
%
%\newcommand{\RR}{\mathbb R}
%\newcommand{\CC}{\mathbb C}
%\newcommand{\AAA}{\mathbb A}
%\newcommand{\PP}{\mathbb P}
%\newcommand{\Ank}{\AAA^n_k}
%\newcommand{\Pnk}{\PP^n_k}
%\newcommand{\Pmk}{\PP^m_k}
%\newcommand{\Amk}{\AAA^m_k}
%\newcommand{\calA}{{\mathcal A}}
%\newcommand{\II}{{\mathcal I}}
%\newcommand{\VV}{{\mathcal V}}
%\newcommand{\KK}{{\mathcal K}}
%\newcommand{\OO}{{\mathcal O}}
%\newcommand{\mm}{{\mathfrak m}}
%
%\title{Notas y ejercicios de Geometría Algebraica}
%\author{Departamento de Álgebra \\ Universidad de Sevilla}
%\date{Septiembre de 2017}




\begin{document}

%\setcounter{chapter}{1}
%\maketitle
%
%\vspace*{\fill}
%
%\copyright{2011-17 Antonio Rojas León}
%
%\bigskip
%
%Este trabajo está publicado bajo licencia Creative Commons 3.0 España (Reconocimiento - No Comercial - Compartir bajo la misma licencia)
%
%\url{http://creativecommons.org/licenses/by-nc-sa/3.0/es/}
%
%\bigskip
%
%Usted es libre de:
%\begin{itemize}
% \item copiar, distribuir y comunicar públicamente la obra
%\item hacer obras derivadas
%\end{itemize}
%
%Bajo las condiciones siguientes:
%\begin{itemize}
% \item {\bf Reconocimiento:} Debe reconocer los créditos de la obra maestra especificada por el autor o el licenciador (pero no de una manera que sugiera que tiene su apoyo o apoyan el uso que hace de su obra).
% \item {\bf No comercial:} No puede utilizar esta obra para fines comerciales.
%\item {\bf Compartir bajo la misma licencia:} Si altera o transforma esta obra, o genera una obra derivada, sólo puede distribuir la obra generada bajo una licencia idéntica a ésta.
%\end{itemize}
%
%
%
%
%\newpage

\chapter{Variedades en el espacio proyectivo}

\section{Conjuntos algebraicos proyectivos}

Recordemos la definición de espacio proyectivo.

\begin{defi}
 Sea $k$ un cuerpo. El {\bf espacio proyectivo de dimensión $n$ sobre $k$}, denotado $\Pnk$, es el conjunto de $(n+1)$-uplas no nulas $\A^{n+1}_k-\{(0,\ldots,0)\}$ módulo la relación de equivalencia siguiente: $(x_0,x_1,\ldots,x_n)\sim(y_0,y_1,\ldots,y_n)$ si existe un $\lambda\in k^\star$ tal que $y_i=\lambda x_i$ para todo $i=1,\ldots,n$. La clase de equivalencia de $(x_0,x_1,	\ldots,x_n)$ se denota $(x_0:x_1:\ldots:x_n)$.
\end{defi}

Dado un polinomio $f\in k[x_0,x_1,\ldots,x_n]$, no está definido el valor de $f$ en un punto $x\in\Pnk$, ya que este valor será distinto dependiendo del representante que elijamos en $\A^{n+1}_k$ para $x$. Sin embargo, si $f$ es \emph{homogéneo}, la expresión $f(x)=0$ sí tiene sentido: si $x=(x_0:x_1:\ldots:x_n)=(y_0:y_1:\ldots:y_n)$, existe un $\lambda\neq 0$ tal que $y_i=\lambda x_i$, y por tanto $f(y_0,y_1,\ldots,y_n)=\lambda^df(x_0,x_1,\ldots,x_n)$, donde $d$ es el grado de $f$. En particular, $f(y_0,y_1,\ldots,y_n)=0$ si y sólo si $f(x_0,x_1,\ldots,x_n)=0$. En tal caso, decimos que $f(x)=0$.

Para todo subconjunto $S\subseteq k[x_0,x_1,\ldots,x_n]$ formado por polinomios homogéneos, definimos $\V(S)\subseteq \Pnk$ como el conjunto de los $x\in\Pnk$ tales que $f(x)=0$ para todo $f\in S$. En general, para $S\subseteq k[x_0,x_1,\ldots,x_n]$ arbitrario, definimos $\V(S)\subseteq \Pnk$ como el conjunto de los $x\in\Pnk$ tales que $f(x)=0$ para todo polinomio homogéneo $f\in S$. 

\begin{defi}
 Un subconjunto $X\subseteq\Pnk$ se dice {\bf algebraico} si existe $S\subseteq k[x_1,\ldots,x_n]$ formado por polinomios homogéneos tal que $X=\V(S)$.
\end{defi}

\begin{prop}
Si $S\subset k[x_0,\xn{n}]^h$, entonces $\V(S)=\V(\gene{S})$. 
\end{prop}
\begin{dem}
Una inclusión es trivial. Para la otra tenemos que ver que si $f\in \gene{S}$ homogéneo, entonces $f=\sum g_i f_i$ con $f_i \in S$. Si $h_i$ es la componente homogénea de $g_i$ de grado $d-d_i$ entonces $\sum g_i f_i = \sum h_i f_i$. Si $x\in \V(S)$ entonces $f_i(x)=0$ $\forall i$, por lo que $\sum h_i (x) f_i (x) =0$. 
\end{dem}

 \begin{ejs}\emph{\begin{enumerate}
                   \item El conjunto vacío $\emptyset=\V(\{1\})$ y el espacio total $\Pnk=\V(\{0\})$ son conjuntos algebraicos.
\item El conjunto formado por un solo punto $a=(a_0:a_1:\ldots:a_n)$ es algebraico: $\{a\}=\V(\{a_ix_j-a_jx_i|0\leq i<j\leq n\})$.
\item Toda subvariedad lineal de $\Pnk$ es un conjunto algebraico, puesto que está definida por una cantidad finita de ecuaciones lineales homogéneas.
 \end{enumerate}
}
 \end{ejs}

Exactamente igual que en el caso afín, se prueba:

\begin{prop}
  La intersección arbitraria y la unión finita de subconjuntos algebraicos de $\Pnk$ es un conjunto algebraico.
\end{prop}

Y por tanto podemos definir:

\begin{defi}
 Se llama \emph{topología de Zariski} sobre $\Pnk$ a la topología cuyos cerrados son los conjuntos algebraicos.
\end{defi}

\section{Conjuntos algebraicos proyectivos e ideales homogéneos}

Como la suma de polinomios homogéneos no es homogéneo en general, en el caso proyectivo no tiene sentido hablar del ``ideal de polinomios que se anulan en un conjunto $X\subseteq\Pnk$''. Pero sí podemos considerar el ideal generado por ellos: dado $X\subseteq\Pnk$, definimos $\I(X)$ como el \emph{ideal} generado por todos los polinomios homogéneos $f\in k[x_0,x_1,\ldots,x_n]$ tales que $f(x)=0$ para todo $x\in X$. 

\begin{defi}
 Un ideal $I\subseteq k[x_0,x_1,\ldots,x_n]$ se dice \emph{homogéneo} si está generado por polinomios homogéneos.
\end{defi}

\begin{ejer}
 Probar que un ideal $I\subseteq k[x_0,x_1,\ldots,x_n]$ es homogéneo si y sólo si las componentes homogéneas de todo polinomio $f\in I$ están también en $I$.
\end{ejer}

\begin{ejer}
 Para que un ideal homogéneo $I\subseteq k[x_0,x_1,\ldots,x_n]$ sea radical es suficiente que para todo $f\in k[x_0,x_1,\ldots,x_n]$ \emph{homogéneo} tal que $f^m\in I$ para algún $m\geq 1$ se tenga que $f\in I$. 
\end{ejer}

Del ejercicio anterior se deduce en particular que $\I(X)$ es un ideal radical de $k[x_0,x_1,\ldots,x_n]$.

 La correspondencia entre ideales radicales y conjuntos algebraicos es un poco más complicada en el caso proyectivo. Para empezar, veamos qué ideales $I$ verifican $\V(I)=\emptyset$.

\begin{prop}\label{idealirrelevante} Sea $I\subseteq k[x_0,x_1,\ldots,x_n]$ un ideal homogéneo. Las condiciones siguientes son equivalentes:
 \begin{enumerate}
  \item $\V(I)=\emptyset$
\item Para cada $0\leq i\leq n$, existe un $m_i\geq 1$ tal que $x_i^{m_i}\in I$
\item $\sqrt{I}=\langle 1\rangle$ ó $\langle x_0,x_1,\ldots,x_n\rangle$.
 \end{enumerate}

\end{prop}

\begin{proof}
$1\Rightarrow 2)$ Sea $Z\subseteq\A^{n+1}_k$ el conjunto algebraico definido por $I$ visto como ideal corriente. Todo punto de $Z$ distinto de $(0,0,\ldots,0)$ da lugar a un punto proyectivo, que estaría en $\V(I)$. Como $\V(I)=\emptyset$, concluimos que $Z\subseteq\{(0,0,\ldots,0)\}$. Aplicando el Nullstellensatz, se tiene que $\langle x_0,x_1,\ldots,x_n\rangle\subseteq\I(Z)=\sqrt{I}$. En particular, $x_i\in\sqrt{I}$ para todo $i$, y por tanto existe un $m_i\geq 1$ tal que $x_i^{m_i}\in I$.

$2\Rightarrow 3)$ Para todo $0\leq i\leq n$, $x_i\in \sqrt{I}$, así que $\langle x_0,x_1,\ldots,x_n\rangle\subseteq\sqrt{I}$. Como $\langle x_0,x_1,\ldots,x_n\rangle$ es maximal, debe ser $\sqrt{I}=\langle x_0,x_1,\ldots,x_n\rangle$ ó $\sqrt{I}=\langle 1\rangle$.

$3\Rightarrow 1)$ Si $\sqrt{I}=\langle 1\rangle$, entonces $1^m=1\in I$, y por tanto $\V(I)=\emptyset$. Si $\sqrt{I}=\langle x_0,x_1,\ldots,x_n\rangle$, para cada $0\leq i\leq n$ existe un $m_i\geq 1$ tal que $x_i^{m_i}\in I$. Por tanto para todo $(a_0:a_1:\ldots:a_n)\in\V(I)$ se tiene que $a_i^{m_i}=0$, por lo que $a_0=a_1=\ldots=a_n=0$, lo cual es imposible para un punto de $\Pnk$.
\end{proof}

Para el resto de ideales, la versión proyectiva del Nullstellensatz es la siguiente:


\begin{teorema}
 Sea $I\subseteq k[x_0,x_1,\ldots,x_n]$ un ideal homogéneo tal que $\V(I)\neq\emptyset$. Entonces $\I(\V(I))=\sqrt{I}$.
\end{teorema}

\begin{proof}
 La contención $\sqrt{I}\subseteq\I(\V(I))$ es trivial, ya que $\I(\V(I))$ es un ideal radical que contiene a $I$. Veamos la contención opuesta. Como $\I(\V(I))$ está generado por polinomios homogéneos, basta probar que todo polinomio homogéneo de $\I(\V(I))$ está en $\sqrt{I}$. Sea $f\in \I(\V(I))$ homogéneo no nulo. Como $\V(I)\neq \emptyset$, $f$ no es una constante. Sea $Z\subseteq\A^{n+1}_k$ el conjunto algebraico \emph{afín} definido por $I$ visto como ideal corriente. Entonces $f$ se anula en todo punto de $Z$ distinto de $(0,0,\ldots,0)$ (ya que todo punto distinto del origen corresponde a un punto del espacio proyectivo en el que $f$ se anula). Pero también se anula en $(0,0,\ldots,0)$ por ser homogéneo y no constante. Por tanto $f\in\I(Z)=\sqrt{I}$ por el Nullstellensatz afín.
\end{proof}

A partir de los dos últimos resultados obtenemos:

\begin{coro}
 Las aplicaciones $I\mapsto\V(I)$ y $X\mapsto\I(X)$ definen una correspondencia biunívoca entre el conjunto de ideales radicales homogéneos de $k[x_0,x_1,\ldots,x_n]$ distintos de $\langle x_0,x_1,\ldots,x_n\rangle$ y el conjunto de subconjuntos algebraicos de $\Pnk$, que invierte las contenciones.
\end{coro}

En particular, como $k[x_0,x_1,\ldots,x_n]$ es noetheriano, todo subconjunto algebraico de $\Pnk$ es el conjunto de ceros de una cantidad finita de polinomios homogéneos. 

\section{Variedades proyectivas. Dimensión}

En esta sección trasladaremos al caso proyectivo algunos conceptos que ya hemos estudiado en el caso afín. 

\begin{defi}
 Un conjunto algebraico $X\subseteq\Pnk$ se dice {\bf reducible} si existen dos subconjuntos algebraicos propios $Y\subsetneq X$ y $Z\subsetneq X$ tales que $X=Y\cup Z$. De lo contrario, $X$ se dice {\bf irreducible}. Una {\bf variedad proyectiva} es un subconjunto algebraico de $\Pnk$ irreducible.
\end{defi}

\begin{ejer}\label{primohomogeneo}
 Sea $I\subsetneq k[x_0,x_1,\ldots,x_n]$ un ideal propio homogéneo. Probar que $I$ es primo si y sólo si para todos $f,g\in k[x_0,x_1,\ldots,x_n]$ \emph{homogéneos} tales que $fg\in I$, o bien $f\in I$ o bien $g\in I$.
\end{ejer}

 \begin{prop}
  Un conjunto algebraico $X\subseteq\Pnk$ no vacío es irreducible si y sólo si $\I(X)$ es un ideal primo.
 \end{prop}

\begin{proof}
 Sea $X$ irreducible y no vacío, y supongamos que $\I(X)$ no es primo. Por el ejercicio \ref{primohomogeneo}, existen $f,g\in k[x_0,x_1,\ldots,x_n]$ homogéneos tales que $fg\in\I(X)$, pero ni $f$ ni $g$ están en $\I(X)$. Definiendo $Y=\V(f)\cap X$ y $Z=\V(g)\cap Y$ se tiene entonces que $Y\subsetneq X$, $Z\subsetneq X$, e $Y\cup Z =\V(fg)\cap X=X$, contradiciendo la irreducibilidad de $X$.

Recíprocamente, supongamos que $X=Y\cup Z$ con $Y\subsetneq X$ y $Z\subsetneq X$. Sean $f\in \I(Y)\backslash \I(X)$ y $g\in \I(Z)\backslash\I(X)$ homogéneos. Entonces $fg$ es homogéneo y se anula en todo punto de $Y\cup Z=X$, y por tanto $fg\in\I(X)$. De modo que $\I(X)$ no puede ser primo.
\end{proof}

\begin{coro}
 Las aplicaciones $I\mapsto\V(I)$ y $Z\mapsto\I(Z)$ definen una correspondencia biunívoca entre el conjunto de ideales primos homogéneos de $k[x_1,\ldots,x_n]$ distintos de $\langle x_0,x_1,\ldots,x_n\rangle$ y el conjunto de variedades proyectivas no vacías en $\Pnk$, que invierte las contenciones.
\end{coro}

El siguiente resultado se traslada palabra por palabra desde el caso afín, con idéntica demostración:

\begin{prop}
 Todo conjunto algebraico $X\subseteq \Pnk$ puede descomponerse como una unión finita $X=Z_1\cup\cdots\cup Z_r$ de variedades proyectivas. Si la descomposición es minimal (es decir, si ningún $Z_i$ puede eliminarse de ella sin que la unión deje de ser $X$), las variedades $Z_i$ están unívocamente determinadas, y se denominan {\bf componentes irreducibles} de $X$.
\end{prop}

\begin{defi} La {\bf dimensión} de un conjunto algebraico $X\subseteq\Pnk$ es el mayor entero $n$ tal que existe una cadena estrictamente creciente $Z_0\subsetneq Z_1\subsetneq\cdots\subsetneq Z_n\subsetneq X$ de variedades proyectivas. Una variedad proyectiva de dimensión $1$ (respectivamente $2$, $n-1$) en $\Pnk$ se denomina una {\bf curva} (resp. {\bf superficie}, {\bf hipersuperficie}) proyectiva.
\end{defi}

El resultado siguiente, que no demostraremos, también es paralelo al correspondiente resultado afín:

\begin{prop}\label{principalproy}
 Una variedad proyectiva $Z\subseteq\Pnk$ es una hipersuperficie si y sólo si $\I(Z)$ está generado por un polinomio homogéneo irreducible. 
\end{prop}

\begin{nota}\
\begin{itemize}


\item Dos polinomios de grado $d$ y $e$ en el plano proyectivo se cortan en $de$ puntos.

\item Si tienes un ideal generado por $r$ polinomios en $\mathbb{P}^n$ entonces cualquier componente irreducible tiene como mínimo $n-r$.
\end{itemize}
\end{nota}

\begin{ejer}
Intentemos descomponer $\V(x_0^2-x_1x_2,x_0x_1-x_2x_3)$. Claramente $\V(x_0,x_2)$ está contenido. Quitando ese caso, podemos suponer que $x_2\neq 0$, por lo que podemos trabajar en el afín con $x_2=1$, quedando el sistema: $x_0^2-x_1=0$ y $x_0x_1-x_3=0$. Esto nos da el sistema $x_0 = t$, $x_1=t^2$, $x_3=t^3$. Por tanto, en el espacio afín esto sería $\V(x_0^2-x_1,x_0^3-x_3)$. Homogeneizando, $\V(x_0^2-x_1x_2,x_0^3-x_3x_2^2)$. 

Como se puede observar, es bastante complicado resolver este problema en general, para abordarlo usaremos los resultados que veremos a continuación.

 
\end{ejer}

Vamos a tener en cuenta que la definición de irreducibilidad es válida para cualquier espacio topológico, y que tomando complementarios se peude exprespar como que $X$ es irreducible si y solo si para cualquier par de abiertos $U,V\subseteq X$ no vacíos, entonces $U\cap V\neq\emptyset$.

\begin{defi}
La \textbf{clausura proyectiva} $\overline{X}$ de un conjunto algebraico afín $X\subseteq\A^n$ es el menor conjunto algebraico de $\mathbb{P}^n$ tal que $X\subseteq\overline{X}$.
\end{defi}

\begin{prop}
Supongamos que $X=\bigcup U_i$ con $U_i$ abierto irreducible, y $U_i\cap U_j\neq\emptyset$, entonces $X$ es irreducible.
\end{prop}
\begin{dem}
Sean $U,V\subseteq X$ abiertos no vacíos. Entonces $\exists i\mid U\cap U_i\neq\emptyset\ \exists j\mid V\cap U_j\neq\emptyset$. Por hipótesis $V\cap U_j\neq\emptyset$ y $U_i\cap U_j\neq\emptyset$ siendo las dos interesecciones abiertas. Por tanto $U_i\cap U_j\cap V\neq\emptyset$. Por otro lado, $U_i\cap U_j\cap U\neq\emptyset$. Como $U_i$ es irreucible, $U_i\cap U_j\cap U\cap V\neq\emptyset$, por lo que en particular $U\cap V\neq\emptyset$.  \QED
\end{dem}
En $\Pnk$, $U_i:=\Pnk\setminus\{x_i=0\}$ es un abierto (de Zariski) en $\Pnk$ tal que $\Pnk=\bigcup_{i=0}^n U_i$. Tenemos que $U_i\cong\A^n$. Vamos a ver que la restricción de la topología de Zariski del proyectivo a $\A^n$ es la topología de Zariski en $\A^n$. 
\begin{proof}
Toamos $i=0$ y hacemos la identificación $U_0\cong A^n\hookrightarrow\mathbb{P}^n$ tal que $(a_1,\dots,a_n)\mapsto(1:a_1:\dots :a_n)$. Entonces $\mathbb{P}^n\setminus U_0=H_0$ hiperplano del infinito. 

Los cerrados de la topología inducida en $\A^n$ por la topología de Zariski de $\mathbb{P}^n$ son los conjuntos de la forma $Z\cap\A^n$ con $Z\subseteq\mathbb{P}^n$ algebraico. Veamos que esta topología es la de Zariski en $\A^n$.

Sea $X=Z\cap\A^n$, con $Z\subseteq\mathbb{P}^n$ algebraico. $Z=\V(J)$ con $J\subseteq k[x_0,\xn{n}]$ ideal homogéneo. $Z\cap\A^n=\{(a_1,\dots,a_n)\mid (1:a_1:\dots:a_n)\in Z\}$, pero $(1:a_1:\dots:a_n)\in Z\Leftrightarrow f(1,a_1,\dots,a_n)=0\ \forall f\in J$. 

Sea $X\subseteq\A^n$ algebraico, vamos a probar que $X=\A^n\cap\overline{X}$. Es evidente $X\subseteq\A^n\cap\overline{X}$ por definición. Como $X$ es cerrado en $\A^n$, es igual a su clausura afín. La clausura en la restricción es la intersección de la clausura en el proyectivo con el subespacio de la restricción, que es justamente lo que tenemos. 


\end{proof}

\begin{prop}
Dado $X=\V^{af}(I)\subseteq\A^n$ (se puede suponer $I=\I(X)$), entonces $\overline{X}=\V^{pr}(\{f^h\mid f\in I\}=I^h)$. 
\end{prop}
\begin{dem}

Probamos la incluisión hacia la derecha. Sea $(a_1,\dots,a_n)\in X$. Visto como punto proyectivo es $(1:a_1:\dots:a_n)$. Sea $f\in I$, $f^h(1,a_1,\dots,a_n)$, $f=(f^h)^{dh}(a_1,\dots,a_n)=0$. 


Para la otra inclusión, como $\overline{X}=\V(J)\supset\V^{pr}(I^h)$ con $J\subseteq k[x_0,\xn{n}]$ ideal homogéneo, basta probar que $J\subseteq I^h$. Sea $g\in J$ homogéneo, $g$ se anula en todo punto de $\overline{X}$. Si $(a_1,\dots,a_n)\in X$, entonces $(1:a_1:\dots:a_n)\in\overline{X}$, luego $g(1:a_1:\dots:a_n)=0$, así que $g(1,x_1,\dots,x_n)=g^{dh}$ se anula en todo punto de $X$, por lo que $g^{dh}\in\I^{af}(\V^{af}(I))=\sqrt{I}=I$, luego $(g^{dh})^h\in I^h$. Se tiene que $g=x_0^r(g^{dh})^h\in I^h$. 
\end{dem}

Por tanto, $\overline{\V^{af}(I)}=\V^{pr}(I^h)$ y $\A^n\cap\V^{pr}(J)=\V^{af}(J^{dh})$. Esta correspondencia no es una biyección. Es cierto, que si tomamos $X$, hacemos $\overline{X}$, luego $\A^n\cap\overline{X}=X$. Por tanto la correspondencia es inyectiva. Pero en sentido inverso no da lo mismo, pues si tenemos un conjunto algebraico contenido en el hiperplano del infinito, entonces no podemos expresarlo como intresección con el espacio afín. 

Sin embargo, si $Z\subseteq\mathbb{P}^n$ es irreducible y $Z\cap\A^n\neq\emptyset$ entonces $Z=\overline{Z\cap\A^n}$.  La inclusión a la izquierda es evidente porque $Z$ es un cerrado que contiene al conjunto al que le hacemos la clausura. Y si la inclusión fuera estricta, entonces $Z=\overline{Z\cap\A^n}\cap (Z\cap H_\infty)$, pero como $Z$ es irreducible o bien se tiene el resultado o bien $Z\subseteq H_\infty$, lo cual sería contradicción.

Por tanto, tenemos una biyección entre el conjunto de variedades algebraicas afines en $\A^n$ y el conjunto de variedades proyectivas en $\mathbb{P}^n$ no contenidas en $H_\infty$. 

\begin{ej}
$X=\{(t,t^2,t^2)\}=\V(y-x^2,z-x^2)$. Homogeneizando obtenemos $\V(yt-x^2,zt-x^2)$. Los puntos afines se obtienen con $t=1$. En el resto, $t=0$, luego $x=0$. Así, obtenemos la descomposibción $\V(yt-x^2,zt-x^2)=\V(t,x)\cup\V(zy-x^2,y-z)=Z$. Para el segundo conjunto, al que llamaremos $Y$, necesitamos polinomios homogéneos, por eso no hemos cogido los originales. Tenemos $X\subseteq Y\subsetneq Z$. 

Tenemos que:
\[ \V(x_0^2-x_1x_2, x_0x_1-x_2x_3) = \V(x_0,x_2) \cup Y \]
donde $Z=\V_{af}(x_0^2-x_1,x_0x_1-x_3) = \{(t,t^2,t^3)\}$ e $Y=\overline{Z}=\V(x_0^2-x_1x_2,x_0x_1-x_3x_2,x_1^2-x_0x_3)$. Veamos que $Y$ irreducible. Como:
\[ Y = (Y \cap U_0) \cup (Y \cap U_1) \cup (Y \cap U_2) \cup (Y \cap U_3) \]
donde $U_i = \{x_i \neq 0\}$.
\[ Y \cap U_0 = \V(1-x_1x_2,x_1-x_2x_3,x_1^2-x_3) = \{(t,1/t,t^2)\} \]
Como $Y \cap U_0$ está parametrizado, es irreducible. Análogamente:
\[ Y \cap U_1 = \V(x_0^2-x_2,x_0-x_2x_3,1-x_0x_3) = \{(t,t^2,1/t)\} \]
\[ Y \cap U_2 = \{(t,t^2,t^3)\} \]
\[ Y \cap U_3 = \V(x_0^2-x_1x_2, x_0x_1-x_2, x_1^2-x_0) = \{(t^2,t,t^3)\} \]
son irreducibles. Como $Y$ es unión de irreducibles disjuntos 2 a 2, $Y$ es irreducible.
\end{ej}

\begin{nota}
En general, dado $f\in k[x_0,\xn{n}]$, podemos obtener el polinomio \textbf{deshomogeneizado} $f^{dh}=f(1,x_1,\dots,x_n)\in k[\xn{n}]$. La operación ``inversa'' (realmente no es inversa porque la composición no es la identidad en ambos sentidos) sería, dado $g\in k[\xn{n}]$, entonces $g^h=x_0^{\deg{g}}g(\frac{x_1}{x_0},\dots,\frac{x_n}{x_0})\in k[x_0,\xn{n}]$. Tomando una potencia mayor de $x_0$ también se obtendría un polinomio homogéneo. 
\end{nota}

\begin{ejer}
$\exists r\geq 0$ tal que $x_0^r(f^{dh})^h=f$.
\end{ejer}

\section{Funciones regulares y racionales. Variedades cuasi-proyectivas}

En el caso de los conjuntos algebraicos proyectivos, no podemos definir las funciones regulares $f:X\to k$ como aquéllas dadas por un polinomio, ni siquiera homogéneo, ya que el valor del polinomio en un punto dado $x=(x_0:x_1:\ldots:x_n)$ dependerá de las coordenadas elegidas para representar al punto. Si $f$ es un polinomio homogéneo de grado $d$ y multiplicamos las coordenadas de $x$ por $\lambda\neq 0$, el valor de $f(x_0,x_1,\ldots,x_n)$ quedará multiplicado por $\lambda^ d$. Sin embargo, si $g$ es otro polinomio homogéneo del mismo grado tal que $g(x)\neq 0$, el valor del cociente $f(x)/g(x)$ no depende de las coordenadas elegidas para $x$. Esto justifica la siguiente definición:

\begin{defi}
 Sea $Z\subseteq\Pnk$ una variedad proyectiva. Una {\bf función racional} $f:Z\dashrightarrow k$ es un par $(U,f)$ donde $U\subseteq Z$ es un abierto no vacío y $f:U\to k$ es una aplicación tal que existen $g,h\in k[x_0,x_1,\ldots,x_n]$ homogéneos del mismo grado con $h(x)\neq 0$ y $f(x)=g(x)/h(x)$ para todo $x\in U$, módulo la siguiente relación de equivalencia: $(U,f)\sim (U',f')$ si $f_{|U\cap U'}=f'_{|U\cap U'}$.
\end{defi}

\begin{nota}
Si en $U$, $f=P/Q$ y en $V$, $g=R/S$. $(U,f)=(V,g)$ si  $PS=QR$ en $U \cap V$, es decir, si $PS-QR\in I(X)$.
\end{nota}

\begin{ejer}
 Definir la suma y el producto de funciones racionales en $Z$, y probar que con esas operaciones forman un cuerpo, que llamaremos el {\bf cuerpo de funciones} de $Z$ y denotaremos ${\mathcal K}(Z)$. Para un conjunto algebraico $X$ cualquiera solo podemos garantizar que $\KK(X)$ sea anillo.
\end{ejer}

\begin{ej}
En $X=\mathbb{P}^1$:
\[ K(X) = \left\{\frac{P(x_0,x_1)}{Q(x_0,x_1)} \mid P,Q \in k[x_0,x_1] \text{ homógeneo de mismo grado y }Q(x_0,x_1)\neq 0\right\} / \sim \]

Veamos que $K(X)$ es cuerpo. Sea $φ : K(X) \to k(t)$ definida como:
\[ φ\left(\frac{P(x_0,x_1)}{Q(x_0,x_1)}\right)  = \frac{P(1,t)}{Q(1,t)} \]
Se puede ver con facilidad que $φ$ está bien definida y es inyectiva. Para ver que es sobreyectiva basta observar que:
\[ \frac{R(t)}{S(t)} = φ \left(\frac{R^h}{S^h} \cdot x_0^{\text{gr}(S)-\text{gr}(R)} \right)\]
Luego $φ$ es isomorfismo y $K(X)$ es cuerpo.
\end{ej}

Si $Z\subseteq\Pnk$ es una variedad y $f:Z\dashrightarrow k$ una función racional, se dice que $f$ {\bf está definida en} el punto $x\in Z$ si existe un abierto $U\subseteq Z$ con $x\in U$ y polinomios $g,h\in k[x_0,x_1,\ldots,x_n]$ homogéneos del mismo grado con $h(y)\neq 0$ y $f(y)=g(y)/h(y)$ para todo $y\in U$. El conjunto de puntos donde $f$ está definida es un abierto, llamado el {\bf abierto de definición} de $f$. El conjunto de funciones racionales definidas en un punto $x\in Z$ es un subanillo local del cuerpo de funciones, que llamaremos el {\bf anillo local de $Z$ en $x$}, denotado $\OO_{Z,x}$.  Así mismo:
\[ \OO_Z(U) = \bigcap_{x \in U} \OO_{Z,x} \]

\begin{prop}
Si $X$ es irreducible, $O_{X,a}$ y $O_X(U)$ son subanillos de $K(X)$. $O_{X,a}$ es un anillo local de ideal maximal $\{f \in O_{X,a} \mid f(a)=0\}$.
\end{prop}

\begin{defi}
 Un subconjunto $X\subseteq\Pnk$ se denomina {\bf variedad cuasi-proyectiva} si es un abierto de Zariski de una variedad proyectiva $Z$.
\end{defi}

En ese caso, existen polinomios homogéneos $f_1,\ldots,f_r$ y $g_1,\ldots,g_s$ en $k[x_0,x_1,\ldots,x_n]$ tales que $Z$ es el conjunto de puntos $x\in\Pnk$ en los que $f_1(x),\ldots,f_r(x)$ se anulan y al menos uno de $g_1(x),\ldots,g_s(x)$ no se anula. 

\begin{defi}
 Sea $Z\subseteq\Pnk$ una variedad cuasi-proyectiva. Una función racional $f:Z\dashrightarrow k$ definida en todo punto de $Z$ se denomina {\bf función regular}. El conjunto de funciones regulares en $Z$ es un anillo, denotado $\OO(Z)$.
\end{defi}

\begin{nota}
Recordemos que en el caso afín, toda función racional es regular.
\end{nota}

\begin{ej}
En $\mathbb{P}^1$, veamos qué necesitamos para que $P(x_0,x_1)/Q(x_0,x_1)$ esté definida en todo punto de $\mathbb{P}^1$. Para todo $(a_0:a_1)$, existe $R,S \in k[x_0,x_1]$ homogéneos de mismo grado tal que $S(a_0,a_1)\neq 0$ y $P/Q=R/S$. Supongamos que $Q(a_0,a_1)=0$, entonces $(a_0x_1-a_1x_0) | Q(x_0,x_1)$. Por otro lado: $P(a)S(a)=R(a)Q(a)$, luego $P(a)=0$ luego $(a_0x_1-a_1x_0) | P(x_0,x_1)$. Entonces podemos eliminar dicho factor de $P$ y $Q$ para obtener $P'$ y $Q'$ que no se anula en $(a_0:a_1)$ y de un grado menor. Como todo polinomio no constante se anula, sólo acabaremos cuando $P/Q=A/B$ donde $A,B \in k$. 
\end{ej}

\section{El recubrimiento afín de una variedad proyectiva}\label{recubrimientoafin}

En esta sección veremos cómo podemos reducir el estudio de las variedades proyectivas (al menos localmente) al de variedades afines, para las cuales tenemos métodos algebraicos potentes.

Para cada $0\leq i\leq n$, sea $U_i\subseteq\Pnk$ el abierto definido por $x_i\neq 0$. Como todo punto del espacio proyectivo tiene al menos una coordenada no nula, los abiertos $U_i$ recubren $\Pnk$. Sea $\theta_i:\Ank\to U_i$ la aplicación dada por $\theta_i(x_1,\ldots,x_n)=(x_1:\ldots:x_{i-1}:1:x_{i+1}:\ldots:x_n)$.

\begin{ejer}
 Probar que $\theta_i$ es continua para las topologías de Zariski.
\end{ejer}


\begin{prop}
 La aplicación $\theta_i$ es un homeomorfismo, y para toda $f\in\OO(U_i)$, $f\circ\theta_i$ es una función regular en $\Ank$. La aplicación $\theta_i^\star:\OO(U_i)\to\calA(\Ank)=k[x_1,\ldots,x_n]$ dada por $\theta_i^\star(f)=f\circ\theta_i$ es un isomorfismo de $k$-álgebras.
\end{prop}

\begin{proof}
 Para ver que $\theta_i$ es un homeomorfismo, basta comprobar que $\psi_i:(x_0:x_1:\ldots:x_n)\mapsto (x_0/x_i,\ldots,x_{i-1}/x_i,x_{i+1}/x_i,\ldots,x_n/x_i)$ es una aplicación inversa y que es continua. Sea $Z=\V(\{h_1,\ldots,h_r\})\subseteq\Ank$ un cerrado, su imagen inversa por esta aplicación sería el conjunto de puntos $x\in U_i$ tales que $h_j(x_0/x_i,\ldots,x_{i-1}/x_i,x_{i+1}/x_i,\ldots,x_n/x_i)=0$ para todo $j$. Como $x_i\neq 0$, multiplicando por una potencia de $x_i$ obtenemos $r$ polinomios homogéneos $p_j=x_i^{d_j}h_j(x_0/x_i,\ldots,x_{i-1}/x_i,x_{i+1}/x_i,\ldots,x_n/x_i)$ tales que $\psi_i^{-1}(Z)=U_i\cap \V(\{p_1,\ldots,p_r\})$, y por tanto $\psi_i^{-1}(Z)$ es un cerrado en $U_i$.

 Sea $f\in\OO(U_i)$, veamos que $f\circ\theta_i$ es una función regular. Como es claramente una función racional, por el corolario \ref{definidatodopunto} basta ver que está definida en todo punto $a\in\Ank$. Como $f$ está definida en $\theta_i(a)\in U_i$, existen un abierto $V\subseteq U_i$ con $\theta_i(a)\in V$ y $g,h\in k[x_0,x_1,\ldots,x_n]$ homogéneos del mismo grado tales que $h(y)\neq 0$ y $f(y)=g(y)/h(y)$ para todo $y\in V$. Entonces $g\circ\theta_i$ y $h\circ\theta_i$ son polinomios en $k[x_1,\ldots,x_n]$ tales que $h\circ\theta_i(z)\neq 0$ y $f\circ\theta_i(z)=g\circ\theta_i(z)/h\circ\theta_i(z)$ para todo $z\in\theta_i^{-1}(V)$ (que es abierto por el ejercicio anterior). Por tanto $f\circ\theta_i$ está definida en $a$.

La aplicación $\theta_i^\star$ preserva sumas, productos y constantes, y por tanto es un homomorfismo de $k$-álgebras. Para ver que es un isomorfismo, construyamos el homomorfismo inverso. Sea $p\in k[x_1,\ldots,x_n]$, y sea $\psi(p)(x_0,x_1,\ldots,x_n)=p(x_0/x_i,\ldots,x_{i-1}/x_i,x_{i+1}/x_i,\ldots,x_n/x_i)$. Reduciendo a denominador común, podemos escribir $\psi(p)(x_0,x_1,\ldots,x_n)=q(x_0,x_1,\ldots,x_n)/x^d_i$, donde $q$ es un polinomio homogéneo de grado $d$. Como $x^d_i$ no se anula en $U_i$, $\psi(p)$ define una función regular en $U_i$. Es un ejercicio fácil comprobar que $\theta_i^\star$ y $\psi$ son inversas la una de la otra.
\end{proof}

Supongamos ahora que $Z\subseteq\Pnk$ es una variedad proyectiva (respectivamente cuasi-proyectiva). Entonces $Z$ está recubierta por los subconjuntos abiertos $Z_i:=Z\cap U_i$, y podemos identificar cada $Z_i$ con la variedad afín (resp. cuasi-afín) $\theta_i^{-1}(Z)\subseteq\Ank$. 

\begin{prop}
 Una función $f:Z\to k$ es regular si y sólo si $f\circ\theta_i:\theta_i^{-1}(Z_i)\to k$ es regular para todo $i$. 
\end{prop}

\begin{proof}
 $f$ es regular si y sólo si su restricción a cada abierto $Z_i$ lo es (ya que la regularidad depende sólo de un entorno de cada punto). Por otra parte, la aplicación biyectiva $\theta_i:\Ank\to U_i$ se restringe a una aplicación biyectiva $\theta_i:\theta_i^{-1}(Z_i)\to Z_i$, de tal manera que $f:Z_i\to k$ es regular si y sólo si $f\circ\theta_i:\theta_i^{-1}(Z_i)\to k$ lo es (repitiendo la prueba de la proposición anterior).
\end{proof}

\begin{ejer}
 Sea $Z=\V(f_1,\ldots,f_r)\subseteq\Pnk$ una variedad proyectiva, con $f_1,\ldots,f_r\in k[x_0,x_1,\ldots,x_n]$ homogéneos. Probar que $\theta^{-1}_i(Z_i)=\V(\hat f_1,\ldots,\hat f_r)\subseteq\Ank$, donde $\hat f_i(x_1,\ldots,x_n)=f_i(x_1,\ldots,x_{i-1},1,x_{i+1},\ldots,x_n)$.
\end{ejer}

\section{La inmersión del espacio afín en el espacio proyectivo}

Para $i=0$, la aplicación $\theta_0:\Ank\to\Pnk$ definida en la sección anterior nos da una inmersión (topológica) de $\Ank$ como el subconjunto abierto de $\Pnk$ definido por $x_0\neq 0$. Si no se especifica lo contrario, siempre que consideremos al espacio afín como subconjunto del espacio proyectivo lo haremos mediante esta inmersión. El complementario $H_\infty$ de $\Ank$ en $\Pnk$ (es decir, el hiperplano definido por $x_0=0$) se denominará {\bf hiperplano del infinito}.

Sea ahora $X\subseteq \Ank$ un conjunto algebraico. Su imagen por $\theta_0$ es un subconjunto (no necesariamente algebraico) de $\Pnk$. 

\begin{defi}
 La {\bf clausura proyectiva} $\overline X$ de $X$ es la clausura de $X$ en $\Pnk$ para la topología de Zariski, es decir, el menor subconjunto algebraico de $\Pnk$ que contiene a $X$.
\end{defi}

\begin{ejs}
 \emph{\begin{enumerate}
       \item  Si $X$ es la recta $x_1=0$ en $\A^2_k$, su clausura proyectiva es la recta $x_1=0$ en $\PP^2_k$, que contiene a $X$ y al ``punto en el infinito'' $(0:0:1)$.
\item Si $X$ es la curva $x_1x_2=1$ en $\A^2_k$, su clausura proyectiva es la curva $x_1x_2=x_0^2$ en $\PP^2_k$, que contiene a $X$ y a los dos ``puntos en el infinito'' $(0:1:0)$ y $(0:0:1)$.
       \end{enumerate}
}
\end{ejs}

 
Como ${\overline X}^{\Ank}={\overline X}^{\Pnk}\cap \Ank$ y $X$ es cerrado en $\Ank$, se tiene que ${\overline X}\cap\Ank=X$. En particular, la aplicación $X\mapsto\overline X$ entre conjuntos algebraicos afines y conjuntos algebraicos proyectivos es inyectiva.

\begin{lemma} Si $X\subseteq\Ank$ es irreducible, también lo es su clausura proyectiva.
 \end{lemma}

\begin{proof}
 Supongamos que $\overline X=Y\cup Z$, donde $Y,Z\subseteq\overline X$ son subconjuntos cerrados. Entonces $X={\overline X}\cap\Ank=(Y\cap\Ank)\cup(Z\cap\Ank)$. Como $Y\cap\Ank$ y $Z\cap\Ank$ son cerrados en $X$, que es irreducible, o bien $X=Y\cap\Ank$ o bien $X=Z\cap\Ank$. Supongamos que $X=Y\cap\Ank$, entonces $X\subseteq Y$. Pero $Y$ es un cerrado y $\overline X$ es el menor cerrado de $\Pnk$ que contiene a $X$, así que $Y=\overline X$.
\end{proof}


\begin{prop}
 Sea $Y\subseteq\Pnk$ una variedad proyectiva tal que $Y\cap \Ank\neq\emptyset$. Entonces $Y=\overline{Y\cap\Ank}$.
\end{prop}

\begin{proof}
 Como $\overline{Y\cap\Ank}$ es el menor cerrado de $\Pnk$ que contiene a $Y\cap\Ank$, se tiene que $\overline{Y\cap\Ank}\subseteq Y$. Supongamos que la contención fuera estricta, entonces existiría un $f\in k[x_0,x_1,\ldots,x_n]$ homogéneo que se anula en $\overline{Y\cap\Ank}$ pero no en todo $Y$. En particular, $f$ se anula en $Y\cap\Ank$. Sean $Z=\V(f)\cap Y$ e $Y_\infty=Y\cap H_\infty$. Como $Y\cap\Ank\subseteq\V(f)\cap Y$, se tiene que $Y\supseteq Z\cup Y_\infty\supseteq (Y\cap\Ank)\cup(Y\cap H_\infty)=Y$, por lo que $Y=Z\cup Y_\infty$. Por otra parte, $Z\neq Y$ (ya que $f$ se anula en todo $Z$ pero no en todo $Y$) e $Y_\infty\neq Y$ (ya que $Y\cap\Ank\neq\emptyset$ por hipótesis). Esto contradice la irreducibilidad de $Y$. 
\end{proof}

\begin{coro}
 Las aplicaciones $Z\mapsto\overline Z$ e $Y\mapsto Y\cap\Ank$ definen una correspondencia biunívoca entre el conjunto de variedades afines no vacías en $\Ank$ y el conjunto de variedades proyectivas en $\Pnk$ no contenidas en $H_\infty$.
\end{coro}

Veamos ahora cómo describir esta correspondencia explícitamente:

\begin{prop}
 Sea $Y=\V(\{f_1,\ldots,f_r\})\subseteq\Pnk$ una variedad proyectiva, donde los $f_i\in k[x_0,x_1,\ldots,x_n]$ son polinomios homogéneos. Sea $f'_i(x_1,\ldots,x_n):=f_i(1,x_1,\ldots,x_n)\in k[x_1,\ldots,x_n]$ (el ``deshomogeneizado'' de $f_i$ con respecto a $x_0$) para todo $i=1,\ldots,r$. Entonces $Y\cap\Ank=\V(\{f'_1,\ldots,f'_r\})\subseteq\Ank$.
\end{prop}

\begin{proof}
 Un punto $(x_1,\ldots,x_n)\in\Ank$ está en $Y$ si y sólo si el punto proyectivo correspondiente $(1:x_1:\ldots:x_n)$ está en $Y$, es decir, si y sólo si $f_i(1,x_1,\ldots,x_n)=0$ para todo $i$. Pero por la definición de $f'_i$, esto es lo mismo que decir que $f'_i(x_1,\ldots,x_n)=0$ para todo $i$.
\end{proof}

Para cada $f\in k[x_1,\ldots,x_n]$, definimos el {\bf homogeneizado de $f$ con respecto a $x_0$} como el polinomio homogéneo $f^h(x_0,x_1,\ldots,x_n):=x_0^df(x_1/x_0,\ldots,x_n/x_0)\in k[x_0,x_1,\ldots,x_n]$, donde $d$ es el grado de $f$ (es decir, multiplicamos cada monomio de $f$ por la potencia de $x_0$ necesaria para hacerlo homogéneo de grado $d$). El polinomio original $f$ se puede recuperar a partir de $f^h$, ya que $f(x_1,\ldots,x_n)=f^h(1,x_1,\ldots,x_n)$.

\begin{ejer}\label{homodeshomo}
 Si $g\in k[x_1,\ldots,x_n]$ es un polinomio cualquiera y $f\in k[x_0,x_1,\ldots,x_n]$ un polinomio homogéneo tal que $f(1,x_1,\ldots,x_n)=g(x_1,\ldots,x_n)$, probar que $f=x_0^eg^h$ para algún $e\geq 0$. 
\end{ejer}


\begin{prop}
 Sea $Z=\V(I)\subseteq\Ank$ una variedad afín, donde $I\subseteq k[x_1,\ldots,x_n]$ es un ideal primo. Entonces $\overline Z=\V(I^h)$, donde $I^h\subseteq k[x_0,x_1,\ldots,x_n]$ es el ideal generado por el conjunto de los homogeneizados de los elementos de $I$ con respecto a $x_0$.
\end{prop}

\begin{proof}
 Si $f\in I$, entonces $f(x_1,\ldots,x_n)=0$ para todo $(x_1,\ldots,x_n)\in Z$. Dicho de otra forma, $f^h(1,x_1,\ldots,x_n)=0$, y por tanto $Z\subseteq\V(I^h)$. Como $\overline Z$ es la clausura de $Z$ en $\Pnk$ y $\V(I^h)$ es cerrado concluimos que $\overline Z\subseteq\V(I^h)$. 

Recíprocamente, sea $\overline Z=\V(J)$, donde $J$ es un ideal homogéneo. Para todo $g\in J$ homogéneo y para todo $(x_1,\ldots,x_n)\in Z$ se tiene entonces que $g(1,x_1,\ldots,x_n)=0$. Por tanto, $g'(x_1,\ldots,x_n):=g(1,x_1,\ldots,x_n)\in\I(Z)=I$. Por el ejercicio \ref{homodeshomo} se tiene entonces que $g=x_0^eg'^h$ para algún $e\geq 1$, y por tanto $g\in I^h$. Así que $J\subseteq I^h$, y tomando $\V$ concluimos que $\V(I^h)\subseteq\V(J)=\overline Z$.
\end{proof}

En el resultado anterior no es posible reemplazar $I$ por un conjunto finito de generadores, como muestra el ejemplo siguiente:

\begin{ejs}
 \emph{Sea $Z\subseteq\A^3_\C$ la variedad $\V(\{x_1^2-x_2,x_1^2-x_3\})$. Su clausura proyectiva $\overline Z$ {\it no es} el conjunto algebraico definido por los polinomios $x_1^2-x_2x_0$ y $x_1^2-x_3x_0$. Este conjunto algebraico tiene dos componentes irreducibles: la definida por $x_1^2-x_2x_0=x_2-x_3=0$ (la verdadera clausura proyectiva de $Z$) y la componente ``intrusa'' definida por $x_0=x_1=0$.}
\end{ejs}

En general, si $Z=\V(\{f_1,\ldots,f_r\})$ es una variedad afín, el conjunto algebraico proyectivo $\V(\{f_1^h,\ldots,f_r^h\})$ contiene a la clausura proyectiva de $Z$ como componente irreducible, posiblemente con otras componentes irreducibles contenidas en $H_\infty$.













\begin{defi} Una variedad cuasi proyectiva es un subconjunto abierto de una variedad proyectiva. Una variedad cuasi afín es un conjunto abierto de un variedad afín.
\end{defi}

\begin{ej}
En $\mathbb{P}^1$, veamos qué necesitamos para que $P(x_0,x_1)/Q(x_0,x_1)$ esté definida en todo punto de $\mathbb{P}^1$. Para todo $(a_0:a_1)$, existe $R,S \in k[x_0,x_1]$ homogéneos de mismo grado tal que $S(a_0,a_1)\neq 0$ y $P/Q=R/S$. Supongamos que $Q(a_0,a_1)=0$, entonces $(a_0x_1-a_1x_0) | Q(x_0,x_1)$. Por otro lado: $P(a)S(a)=R(a)Q(a)$, luego $P(a)=0$ luego $(a_0x_1-a_1x_0) | P(x_0,x_1)$. Entonces podemos eliminar dicho factor de $P$ y $Q$ para obtener $P'$ y $Q'$ que no se anula en $(a_0:a_1)$ y de un grado menor. Como todo polinomio no constante se anula, sólo acabaremos cuando $P/Q=A/B$ donde $A,B \in k$. 
\end{ej}

\begin{prop}Vamos a ver que las definiciones que hemos visto de funciones regulares no son conceptos distintos, sino que una es un caso particular de la otra.
\end{prop}
\begin{dem}
\begin{lemma}
 Las siguientes aplicaciones son homeomorfismos
\begin{gather*} 
\theta_i \func{\A^n}{U_i\subset \PP^n}\\
(a_1,\dotsc,a_n) \rightarrow (a_1,a_2,\dotsc,a_{i-1},1,a_{i+1},\dotsc,a_n)
\end{gather*}
Para verlo, consideramos por simplificar el caso $i=0$. Sea $Z\subset U_0$ cerrado. Entonces $Z=U_0\cap \V(f_1,\dotsc,f_r)$ con $f_i \in k[x_0,\dotsc,x_n]^h$. Entonces
$$
\theta_0^{-1}(Z) = \{(a_1,\dotsc,a_n)\in\A^n \mid (1:a_1,\dotsc,a_n)\in Z\} = \Va(f_1^{dh},\dotsc,f_r^{dh})
$$
Sea $Y\subset \A^n$ cerrado. Entonces $Y=\Va(g_1,\dotsc,g_s)$. Entonces
$$
{\left(\theta_0^{-1}\right)}^{-1}(Y) = \{(x_0:\dotsc:x_n)\in U_0 \mid x_0^{gr(g_i)}\cdot g_i(\frac{x_1}{x_0},\dotsc,\frac{x_n}{x_0}=0\;\forall i\} = U_0 \cap \Vp(g_1^h,\dotsc,g_s^h)
$$
\end{lemma}
\begin{lemma}Si $f\func{U_0}{k}$ es regular entonces $f \circ \theta_0\func{\A^n}{k}$ es regular. 

Basta ver que es racional y definida en todos los puntos. $\exists P,Q\in k[x_0,\dotsc,x_n]^h$ del mismo grado tales que $f(x)=\dfrac{P(x)}{Q(x)}$ en el abierto $\{Q(x)\neq 0\}$. Entonces $f\circ \theta_0 (a_1,\dotsc,a_n)= \dfrac{P(1,a_1,\dotsc,a_n)}{Q(1,a_1,\dotsc,a_n)} = \dfrac{P^{dh}(q)}{Q^{dh}(q)}$ definida en todos los puntos. 
\end{lemma}
La aplicación $\Phi \func{ O_{\PP^n}(U_0)}{\mathcal{A}(\A^n)}$ tal que $\Phi(f) = f\circ \theta_0$ es un homomorfimos de $k$-álgebras que deja fijas las constantes. Si $g\in\mathcal{A}(\A^n)$ entonces $g(x)=R(x)$ con $R\in k[x_1,\dotsc,x_n]$. Si denotamos $\psi = \Phi^{-1}$ entonces $\psi(g) = \dfrac{R^h}{x_0^{gr(R)}}=R(\frac{x_1}{x_0},\dotsc,\frac{x_n}{x_0})$.

Además $Z\subset \PP^n$ proyectivo, $Y=Z\cap U_0 \subset \A^n$ afín, entonces $\mathcal{A}(Y)\cong O_Z(Y)$.
\end{dem}


\end{document}
