\documentclass[twoside]{article}
\usepackage{../estilo-ejercicios}

%--------------------------------------------------------
\begin{document}

\title{Ciencias de la Computación}
\large{Ejercicios}

\author{Javier Aguilar Martín}
\maketitle

\begin{ejercicio}{6}
Sean $f$ y $g$ computables (de aridad $1$). Probar que la función $h(x) = f(g(x))$ es
computable.
\end{ejercicio}
\begin{solucion}
Sean $f,g:\N-\to\N$ GOTO-computables. Por ser $f$ y $g$ GOTO-computables existen expansiones de las macros $W\leftarrow f(V)$ y $W\leftarrow g(V)$. Sea $P$ el programa
\[
\begin{cases}
Z\leftarrow g(X)\\
Y\leftarrow f(Z)
\end{cases}
\]  
Entonces $[[P]]^{(1)}=f\circ g$. 
\end{solucion}

\begin{ejercicio}{10}
Sea $g : \N \to \N$ biyectiva y GOTO-computable. Demostrar que $g^{-1}$ es GOTO-computable.
\end{ejercicio}
\begin{solucion}
La idea será, para cada $X$, buscar por fuerza bruta el $Z$ tal que $X=g(Z)$. Como la función es total, el programa siempre termina.
\[
P=\begin{cases}
[B] &Z_2\leftarrow g(Z_1)\\
&IF\ X=Z_1 \ GOTO\ A\\
&Z_1\leftarrow Z_1+1\\
&GOTO\ B\\
[A] & Y\leftarrow Z_1
\end{cases}
\]
También podríamos haber ido calculando directamente $g(Y)$. En cualquier caso, $[[P]]^{(1)}=g^{-1}$. 
\end{solucion}

HACER 15 Y 16.

\end{document}