\documentclass[twoside]{article}
\usepackage{../estilo-ejercicios}

%--------------------------------------------------------
\begin{document}

\title{Ciencias de la Computación}

\author{Javier Aguilar Martín}
\maketitle

\begin{ejercicio}{6}
Sean $f$ y $g$ computables (de aridad $1$). Probar que la función $h(x) = f(g(x))$ es
computable.
\end{ejercicio}
\begin{solucion}
Sean $f,g:\N-\to\N$ GOTO-computables. Por ser $f$ y $g$ GOTO-computables existen expansiones de las macros $W\leftarrow f(V)$ y $W\leftarrow g(V)$. Sea $P$ el programa
\[
\begin{cases}
Z\leftarrow g(X)\\
Y\leftarrow f(Z)
\end{cases}
\]  
Entonces $[[P]]^{(1)}=f\circ g$. 
\end{solucion}

\begin{ejercicio}{10}
Sea $g : \N \to \N$ biyectiva y GOTO-computable. Demostrar que $g^{-1}$ es GOTO-computable.
\end{ejercicio}
\begin{solucion}
La idea será, para cada $X$, buscar por fuerza bruta el $Z$ tal que $X=g(Z)$. Como la función es total, el programa siempre termina.
\[
P=\begin{cases}
[B] &Z_2\leftarrow g(Z_1)\\
&IF\ X=Z_1 \ GOTO\ A\\
&Z_1\leftarrow Z_1+1\\
&GOTO\ B\\
[A] & Y\leftarrow Z_1
\end{cases}
\]
También podríamos haber ido calculando directamente $g(Y)$. En cualquier caso, $[[P]]^{(1)}=g^{-1}$. 
\end{solucion}


\begin{ejercicio}{15}
En este ejercicio analizaremos los programas lineales. Los programas lineales son
los diseñados utilizando el lenguaje $\textsc{GOTO}_l$, obtenido a partir de $\textsc{GOTO}$ eliminando la instrucción
condicional.
\begin{enumerate}
\item Sea $k\in \N$. Demostrar que la función constante $C_k(x) = k$ es $\textsc{GOTO}_l$-computable.
\item Demostrar que si $p$ es un $\textsc{GOTO}_l$-programa de longitud $k$, entonces $[[p]]^{(1)}(x) \leq k$.
\item Probar que si $p$ es un programa que calcula $C_k$, entonces la longitud de $p$ es mayor o igual
que $k$.
\item Probar que la función $f(x) = x + 1$ no es $\text{GOTO}_l$-computable (luego $\textsc{GOTO}$ y $\textsc{GOTO}_l$ no son
equivalentes).
\end{enumerate}
\end{ejercicio}
\begin{solucion}\
\begin{enumerate}
\item Basta repetir $k$ veces la instrucción $Y\leftarrow Y+1$.
\item La única forma de alcanzar un valor superior a $k$ es haciendo más de $k$ incrementos, pero entonces tendríamos una longitud de programa mayor que $k$. 
\item El algoritmo del primer apartado es el más corto posible para calcular $C_k$ ya que solo involucra la variable de salida y el incremento. Por tanto, cualquier otro programa que la calcule debe ser al menos tan largo como este, que tiene longitud $k$. 
\item Supongamos que sí lo es. Entonces habría un programa de longitud $k$ que calcula $f(x)=x+1\ \forall x\in\N$. En particular, calcula $f(k)=k+1$, pero eso contradice el apartado 2.
\end{enumerate}
\end{solucion}

\begin{ejercicio}{16}
Sea $\textsc{GOTO}_f$ el lenguaje que se obtiene a partir de $\textsc{GOTO}$ restringiendo el uso de la
instrucción

$I\equiv IF\ V\neq 0\ GOTO\ L$

a etiquetas L para instrucciones que aparecen en el programa después de I (o son de salida).

\begin{enumerate}
\item Probar que si $f : \N- \to \N$ es $\textsc{GOTO}_f$-computable, entonces existe una partición finita
$\{A_1,\dots,A_l\}$ de $\N$ tal que en cada $A_i$, la función es constante.
\item Probar que $\textsc{GOTO}_f$ y $\textsc{GOTO}_l$ no son equivalentes.
\end{enumerate}
\end{ejercicio}
\begin{solucion}\
\begin{enumerate}
\item
\item 
\end{enumerate}
\end{solucion}

\begin{ejercicio}{}
Dar un programa GOTO que calcule las siguientes funciones.
\begin{enumerate}
\item $f(x)=2^x$. Pensar que $f_1(0)=1, f_!(x+1)=2f_1(x)$
\end{enumerate}
\end{ejercicio}

\begin{solucion}
\begin{enumerate}

\item La primera línea será el caso base y cada vez que se actualice el valor de $Y$ tendremos la $h$. 
\begin{align*}
&Y\leftarrow Y+1\\
[A]&IF\ X=Z(\text{macro})\ GOTO\ E\\
&Y\leftarrow Z\cdot Y (\text{macro})\\
&Z\leftarrow Z+1\\
&GOTO\ A
\end{align*}
Alternativamente, podríamos prescindir de $Z$ e ir multiplicando por 2 usando otra macro, mientras hacemos decrecer $X$. Pero en general es preferible el anterior para mantener el valor de $X$. 

\item $f_2(x)=x!$
\end{enumerate}
\end{solucion}

Intentar el 14 de la Rel 1

\end{document}