\documentclass[twoside]{article}
\usepackage{../estilo-ejercicios}
\usepackage{wasysym}
\usetikzlibrary{automata,positioning}
\usepackage{mathdots}
\newcommand{\sii}{{\Leftrightarrow}}
%--------------------------------------------------------
\begin{document}

\title{Ciencias de la Computación}

\author{Javier Aguilar Martín, Diego Pedraza López}
\maketitle

\begin{ejercicio}{1}
Sea $f : \N \to \N$ la función definida por
$$f(0) = 1,\quad f(n) = f(\lfloor\frac{n}{2}\rfloor) + n,\ \text{si }n \geq 1.$$
\begin{enumerate}
\item Diseñar un programa \texttt{GOTO} que calcule $f$.

(Solo podrán utilizarse las macros: \texttt{GOTO L}, $\texttt{W}\leftarrow \texttt{V}$ y $\texttt{W}\leftarrow   \lfloor\frac{\texttt{V}}{2}\rfloor$, siendo $\texttt{V}$ y $\texttt{W}$ variables
distintas. El programa tendrá 12 líneas como máximo).
\item 
\end{enumerate}
\end{ejercicio}
\begin{solucion}
\begin{enumerate}
\item Observemos que $f(\lfloor n/2\rfloor)=f(\lfloor\lfloor n/2\rfloor/2\rfloor)+\lfloor n/2\rfloor+n$. Con esto en mente podemos hacer el programa que se pide.
\begin{align*}
& Y\leftarrow Y+1\\
&Z_1\leftarrow X\\
& IF\ Z_1\neq 0\ GOTO\ A\\
& GOTO\ E\\
[C]&Z_2\leftarrow Z_1\\
[A]& Y\leftarrow Y+1\\
&Z_2\leftarrow Z_2-1\\
&IF\ Z_2\neq 0\ GOTO\ A\\
&GOTO\ B\\
[B]&Z_1\leftarrow \left\lfloor\frac{Z_1}{2}\right\rfloor\\
& IF\ Z_1\neq 0\ GOTO\ C\\
\end{align*}


\item
\end{enumerate}
\end{solucion}

\newpage

\begin{ejercicio}{5}
Consideremos el alfabeto $\Sigma = \{a, b\}$. Describir una máquina de
Turing que calcule la función $f : \Sigma^* \times \Sigma^* \to \Sigma^*$ definida por $f(x, y) = x$
\end{ejercicio}
\begin{solucion}
\url{http://turingmachinesimulator.com/shared/xsuwvlyyse}
\end{solucion}
\end{document}