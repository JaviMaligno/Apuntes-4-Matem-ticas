\documentclass[twoside]{article}
\usepackage{../estilo-ejercicios}
\usepackage{wasysym}
\usetikzlibrary{automata,positioning}
\usepackage{mathdots}
\newcommand{\sii}{{\Leftrightarrow}}
%--------------------------------------------------------
\begin{document}

\title{Ciencias de la Computación}

\author{Javier Aguilar Martín, Diego Pedraza López}
\maketitle

\begin{ejercicio}{1}
Sea $f : \N \to \N$ la función definida por
$$f(0) = 1,\quad f(n) = f(\lfloor\frac{n}{2}\rfloor) + n,\ \text{si }n \geq 1.$$
\begin{enumerate}
\item Diseñar un programa \texttt{GOTO} que calcule $f$.

(Solo podrán utilizarse las macros: \texttt{GOTO L}, $\texttt{W}\leftarrow \texttt{V}$ y $\texttt{W}\leftarrow   \lfloor\frac{\texttt{V}}{2}\rfloor$, siendo $\texttt{V}$ y $\texttt{W}$ variables
distintas. El programa tendrá 12 líneas como máximo).
\item Pruébese (utilizando la función recorrido de $f$) que $f$ es primitiva recursiva.
\end{enumerate}
\end{ejercicio}
\begin{solucion}
\begin{enumerate}
\item Observemos que $f(\lfloor n/2\rfloor)=f(\lfloor\lfloor n/2\rfloor/2\rfloor)+\lfloor n/2\rfloor$. Con esto en mente podemos hacer el programa que se pide.

\begin{align*}
& Y\leftarrow Y+1\\
&Z_1\leftarrow X\\
& IF\ Z_1\neq 0\ GOTO\ A\\
& GOTO\ E\\
[C]&Z_2\leftarrow Z_1\\
[A]& Y\leftarrow Y+1\\
&Z_2\leftarrow Z_2-1\\
&IF\ Z_2\neq 0\ GOTO\ A\\
&GOTO\ B\\
[B]&Z_1\leftarrow \left\lfloor\frac{Z_1}{2}\right\rfloor\\
& IF\ Z_1\neq 0\ GOTO\ C\\
\end{align*}


\item
\end{enumerate}
\end{solucion}

\newpage

\begin{ejercicio}{2}
Se pide:
\begin{enumerate}
\item Probar que la siguiente función $F : \N^2\dashrightarrow \N$ es \texttt{GOTO}-computable, siendo
\[
F(x,e)=\begin{cases}
1 & \text{si la computación del programa de código }e\text{ sobre }x\text{ tiene longitud impar}.\\
2 & \text{si la computación del programa de código }e\text{ sobre }x\text{ tiene longitud par}.\\
\uparrow & en otro caso.
\end{cases}
\]
(\textbf{Indicación}: Utilícese el predicado \texttt{STEP}).
\item Probar que existe una función primitiva recursiva $g : \N^2 \to \N$ tal que para todo
$e_1, e_2 \in \N$ y todo $x \in \N$ se tiene:
$$\varphi_{g(e_1,e_2)}^{(1)}=\varphi_{e_1}^{(1)}+\varphi_{e_2}^{(1)}.$$
\end{enumerate}
\end{ejercicio}
\begin{solucion}
\end{solucion}

\newpage

\begin{ejercicio}{3}
\end{ejercicio}\
\begin{enumerate}
\item Sea $\texttt{GOTO}^*$ el conjunto de los programas \texttt{GOTO} que solo contienen instrucciones condicionales
del tipo \texttt{IF V} $\neq 0$ \texttt{GOTO L} (etiquetadas o no) cuando \texttt{V} no es una variable de
entrada. Pruébese que el conjunto $S = \{\#(\texttt{P}) : \texttt{P} \in \texttt{GOTO}^*\}$ es primitivo recursivo.

\item Sea $\texttt{GOTO}_A$ el modelo de computación definido como \texttt{GOTO}, pero sin instrucciones de
decremento y añadiendo instrucciones de asignación del tipo $\texttt{V}\leftarrow \texttt{W}$ (con o sin etiquetas)
y condicionales del tipo \texttt{IF V} $=$ \texttt{W GOTO L}, para cada par de variables distintas \texttt{V} y \texttt{W}.
Probar que \texttt{GOTO} y $\texttt{GOTO}_A$ son equivalentes.
\end{enumerate}
\begin{solucion}
\end{solucion}

\newpage

\begin{ejercicio}{4}
Sea $g : \N^2 \to \N$ definida por $g(n, x) = p^x_n - 1$. Se pide:
\begin{enumerate}
\item Calcular $\mathcal{U}_1(2, g(1000, 2))$.
\item Dado $n \in \N$. ¿se verifica $\texttt{STEP}^{(1)}(2, g(1000, n), 1000)$? Razónese la respuesta.
\end{enumerate}
\end{ejercicio}
\begin{solucion}
\end{solucion}

\newpage

\begin{ejercicio}{5}
Consideremos el alfabeto $\Sigma = \{a, b\}$. Describir una máquina de
Turing que calcule la función $f : \Sigma^* \times \Sigma^* \to \Sigma^*$ definida por $f(x, y) = x$
\end{ejercicio}
\begin{solucion}
\url{http://turingmachinesimulator.com/shared/xsuwvlyyse}
\end{solucion}
\end{document}