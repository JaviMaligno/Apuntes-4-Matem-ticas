\documentclass[twoside]{article}
\usepackage{../estilo-ejercicios}
\usepackage{wasysym}
\usetikzlibrary{automata,positioning}
\usepackage{mathdots}
\usepackage{listings}

%--------------------------------------------------------
\begin{document}

\title{Ciencias de la Computación}

\author{Javier Aguilar Martín}
\maketitle

\begin{ejercicio}{1}
Calcúlese $\#(p)$ donde $p$ el siguiente programa:
\begin{lstlisting}[mathescape=true]
    IF X $\neq$ 0 GOTO A
[B] X $\leftarrow$ X + 1
    IF X $\neq$ 0 GOTO B
[A] Y $\leftarrow$ Y + 1
\end{lstlisting}
\end{ejercicio}
\begin{solucion}
	\[ \#(I_1) = \langle 0, \langle 3, 1\rangle \rangle = 2^0(2\langle 3,1\rangle+1)-1 = 2(2^3(2\cdot 1+1)-1) = 46 \]
	\[ \#(I_2) = \langle 2, \langle 1, 1\rangle \rangle = 2^2(2\langle1,1\rangle+1)-1 = 4(2(2^1(2\cdot1+1)-1)+1)-1 = 43 \]
	\[ \#(I_3) = \langle 0, \langle 4, 1\rangle \rangle = 94 \]
	\[ \#(I_4) = \langle 1, \langle 1, 0\rangle \rangle = 5 \]
	\[ \#(p) = [\#(I_1),\#(I_2),\#(I_3),\#(I_4)]-1 = 2^{46}3^{43}5^{94}7^5-1 \]
\end{solucion}

\newpage

\begin{ejercicio}{2}
Sea $n ≥ 1$. Probar que para cada $f \in \mathcal{P}^{(n)}$ existen infinitos índices $e \in \mathbb{N}$ tales que $φ_e^{(n)} = f$.
\end{ejercicio}
\begin{solucion}
Por teoría sabemos que si existe al menos un $e \in \mathbb{N}$ tal que $f = φ_e^{(n)}$. Supongamos ahora que para un $e$ cualquiera tal que $f = φ_e^{(n)}$, veamos que $e'=(e+1) \cdot \langle 0,\langle 0,1\rangle \rangle \dot{-}1$ también cumple que $f = φ_e^{(n)}$. Este código se corresponde con un programa que hace lo mismo que el programa con número $e$ con un SKIP trivial a la primera variable de entrada. No afecta al valor de $Y$ final, luego $φ_{e'}^{(n)} = f$. Además $e'\neq e$. Entonces, por inducción, tenemos que hay infinitos $e \in \mathbb{N}$ tales que $φ_e^{(n)} = f$. 
\end{solucion}

\newpage

\begin{ejercicio}{3}
Sean $\texttt{GOTO}_l$ y $\texttt{GOTO}_f$ los modelos introducidos en la primera relación de problemas.
Probar que los siguientes conjuntos son recursivos:
\begin{itemize}
\item $A = \{\#(\texttt{p}) : \texttt{p}$ es un $\texttt{GOTO}_l-$programa$\}$.
\item $B = \{\#(\texttt{p}) : \texttt{p}$ es un $\texttt{GOTO}_f-$programa$\}$.
\end{itemize}
\end{ejercicio}
\begin{solucion}\
\begin{itemize}
\item $x\in A\Leftrightarrow (\forall i)_{i\leq long(x+1)} (l(r((x+1)_i))\leq 2)$.
\item $x\in B\Leftrightarrow (\forall i)_{i\leq long(x+1)} (l(r((x+1)_i)\leq 2\lor l(r((x+1)_i)-2\geq l((x+1)_i))$.
\end{itemize}
\end{solucion}

\newpage

\begin{ejercicio}{4}
¿Es ciero que $\mathcal{U}_2(1,2,3) = \mathcal{U}_2(3,2,1)$? Justifíquese la respuesta.
\end{ejercicio}
\begin{solucion}
Tenemos que $\mathcal{U}_2(1,2,3) = [\![\texttt{p}]\!]^{(2)}(1,2)$ con $\#(p)= 3$,  un programa de una instrucción $I$ con $\#(I_1) = 2$. Esta instrucción se corresponde con $X \leftarrow X+1$. Luego $\mathcal{U}_2(1,2,3) = 0$. Por otro lado, $\mathcal{U}_2(3,2,1) = [\![\texttt{q}]\!]^{(2)}(3,2)$ con $\#(q)= 1$, luego $[\![\texttt{q}]\!]^{(2)}$ se corresponde con la función nula y $\mathcal{U}_2(3,2,1) = 0$ también pues es el mismo programa pero etiquetando la instrucción.
\end{solucion}

\newpage

\begin{ejercicio}{5}
Calcúlese:
\begin{enumerate}
\item $\mathcal{U}_1(2,575)$.
\item $\mathcal{U}_1(x,q^2-1)$, donde $q$ es un número primo.
\item $\mathcal{U}_2(1,1,3^m-1)$ donde $m$ tal que $m+1$ es potencia de 2.
\item $\mathcal{U}_2(1,1,\#(U_1))$ 
\end{enumerate}
\end{ejercicio}
\begin{solucion}\
\begin{enumerate}
\item  $\mathcal{U}_1(2,575)=[\![U_1]\!]^{(2)}(2,575)=[\![\texttt{p}]\!]^{(1)}(2)$, siendo $\#(\texttt{p})=575$. $\#(\texttt{p})+1=576=2^6 3^2=[6,2]$ (el programa tiene 2 instrucciones). Sabemos que $\#(I_1)=6,\#(I_2)=2$.$\#(I_1)=\langle a,\langle b,c\rangle\rangle=\langle a,d\rangle=2^a(2d+1)-1$, de donde deducimos que $a=0,d=3=\langle b,c\rangle=2^b(2c+1)-1$, lo que significa que $b=2,c=2$. Análogamente calculamos $\#(I_2)=\langle 0, \langle 1,0\rangle\rangle$. Por tanto $I_1=Z\leftarrow Z-1, I_2=Y\leftarrow Y+1$. Así pues, $[\![\texttt{p}]\!]^{(1)}(2)=1$. 
\item  Buscamos $\#(\texttt{q})=q^2-1$. Sea $k$ tal que $q=p_k$. $\#(\texttt{q})+1=q^2=p_k^2=[0,\dots,0,2]$, de longitud $k$. Las $k-1$ primeras instrucciones son $Y\leftarrow Y$, y la última es $X\leftarrow X+1$. 
\item  Buscamos $[\![\texttt{p}]\!]^{(2)}(1,1)$ siendo $\#(\texttt{p})=3^m-1$. Tenemos que $\#(\texttt{p})+1=3^m=[0,m]$, luego la primera instrucción es $Y\leftarrow Y$, y la segunda cumple $\#(I)=m=\langle a,\langle b,c\rangle\rangle$, de donde deducimos que $m+1=2^a(2d+1)$. Como $m+1$ es potencia de 2, $d=0\Rightarrow b=c=0$. Por lo que la instrucción es $\langle a,\langle 0,0\rangle\rangle$, que es de la forma $[L] Y\leftarrow Y$, con $\#(L)=a$. El valor del programa es siempre 0.
\item $\mathcal{U}_2(1,1,\#(U_1))=[\![U_1]\!]^{(2)}(1,1)=[\![\texttt{p}]\!]^{(1)}(1)$, siendo $\#(\texttt{p})=1$, con lo que $\#(\texttt{p})+1=2=[1]$. Es decir, que el programa consiste en la instrucción $[A_1] Y\leftarrow Y$, que vale 0.
\end{enumerate}
\end{solucion}

\newpage
\begin{ejercicio}{6}
Probar que son recursivos:
\begin{enumerate}
\item $A = \{\#(\texttt{p}) :$ en \texttt{p} no aparece ninguna variable auxiliar$\}$.
\item $B = \{x :$ el programa de código $x$ posee, a lo sumo, 7 instrucciones y alguna es un DECREMENTO $\}$.
\item $C = \{\#(\texttt{p}) : |\texttt{p}| > 5$ y la última instrucción es un condicional $\}$.
\item $D = \{x :$ el programa de código $x^2$ para sobre $x$ en exactamente $x$ pasos$\}$.
\item $E = \{\#(\texttt{p}) :$ el valor de la variable \texttt{Y}, en el paso 100 de la computación de \texttt{p} sobre 0, es 32$\}$.
\end{enumerate}
\end{ejercicio}
\begin{solucion}\
\begin{enumerate}
\item Podemos escribir este conjunto como $\{e\in\N: e$ es el número de Gödel de un programa sin variables auxiliares$\}$. Sea $long(e+1)$ la longitud del programa que tiene ese código y $(e+1)_i$ la $i$-ésima instrucción de dicho programa. $e\in A\Leftrightarrow(\forall j)_{\leq long(e+1)}[j\geq 1\rightarrow (l(r((e+1)_j))=1\lor 2r(r((e+1)_j)))]$ que es primitivo recursivo.
\item $e\in B\Leftrightarrow long(e+1)\leq 7 \land (\exists i)_{\leq long(e+1)}(l(r((e+1)_i))=2$. Primitivo recursivo. Si el existencial no estuviera acotado seguiría sirviendo porque sería recursivo.
\item $e\in C\Leftrightarrow long(e+1)>5\land r(r((e+1)_{long(e+1)})\geq 2$.
\item $x\in D\Leftrightarrow \texttt{STEP}(x,x^2,x)=1\land \texttt{STEP}(x,x^2,x-1)=0$. 
\item $x\in E\Leftrightarrow (r(\texttt{di}(0,x,100)))_1=32$
\end{enumerate}
\end{solucion}
\newpage

\begin{ejercicio}{7}
Sea $h : \N \to \N$ recursiva, total y estríctamente creciente. Probar que
\[ l((μ z) \mathcal{T}_n(\vec{x},e,z)) = l((μ z) \mathcal{T}_n (\vec{x},e, \langle l(z),h(r(z))\rangle ))\]
\end{ejercicio}
\begin{solucion}
\end{solucion}

\newpage
\begin{ejercicio}{8}
Encontrar $f : \N \to \N$ primitiva recursiva e inyectiva tal que para todo $n$, $φ_{f(n)} = \mathcal{O}$.
\end{ejercicio}
\begin{solucion}
La estrategia que usaremos será hacer corresponde $f(n)$ con el programa que tiene $n$ órdenes de la forma $Y \leftarrow Y-1$. Como $\#(Y \leftarrow Y-1) = \langle 0, \langle 2,0\rangle \rangle = 4$, podemos definir
\[\begin{cases}
	f(0) = 2^4-1 \\
	f(n+1) = (f(n)+1)\cdot p_{n+2}^4-1
\end{cases}\]
\end{solucion}
que es claramente primitiva recursiva, inyectiva y $φ_{f(n)} = \mathcal{O}$.

\newpage
\begin{ejercicio}{9}
Probar que la función total $f : \N \to \N$ definida por
\[ f(x) = \begin{cases}
	φ_x(x)+1 &\text{ si }φ_x(x) \downarrow\\
	0 & \text{ en otro caso}
\end{cases}\]
no es recursiva (Indiciación: Si fuese recursiva, tomar $e$ tal que $φ_e = f$).
\end{ejercicio}
\begin{solucion}
Usaremos reducción al absurdo. Supongamos que $f \in \mathcal{P}$. En este caso, existiría $e \in \N$ tal que $f = φ_e$. Como $f$ es total, está definida en $f(e)$, luego $φ_e(e)\downarrow$. Pero entonces $φ_e(e) = f(e) = φ_e(e)+1$. Esto es absurdo.
\end{solucion}

\newpage

\begin{ejercicio}{10}
Sea $f : \N^n \to \N$ una función total \texttt{GOTO}-computable y $p$ un \texttt{GOTO}-programa tal que $[\![p]\!]^{(n)} = f$. Supongamos que existe $g : \N^n \to \N$ primitiva recursiva tal que para todo $(x_1,\dots,x_n) \in \N^n$ se verifica
\[ \texttt{STEP}^{(n)}(x_1,\dots,x_n,\#(\texttt{p}),g(x_1,\dots,x_n)) \]
Probar que la función $f$ es primitiva recursiva.
\end{ejercicio}
\begin{solucion}
\end{solucion}
\newpage
\begin{ejercicio}{11}
Sea $\texttt{p} \in \texttt{GOTO}_p$. Probar que es recursiva la función.
\[ g(x) = \begin{cases}
	\text{núm. de Gödel de la sucesión de valores que toma }Y\text{ en la ejecución de }p\text{ sobre }x &\text{ si }[\![p]\!](x)\downarrow\\
	\uparrow & \text{c.c}
\end{cases} \]
\end{ejercicio}
\begin{solucion}
Sea $f(x,t)=\left(r(di^{(1)}(x,\#(p),t)\right)_1$. Tenemos que $f : \N^2 \to \N$ es primitiva recursiva. Entonces la función recorrido $\hat{f}(x,t) = [f(x,0),\dots,f(x,t)]$ es primitiva recursiva. Sea $h(x) = (μt) (\texttt{STEP}^{(1)} (x,\#(p),t))$, que es recursiva (en particular es $μ$-recursión). Como $g(x) = \hat{f}(x,h(x))$, $g$ es recursiva.
\end{solucion}

\newpage
\begin{ejercicio}{12}
Probar que la siguiente función es recursiva: $h(x,y) = \max\{φ_0(y),\dots,φ_x(y)\}$.
\end{ejercicio}
\begin{solucion}
Se tiene que:
\[ \begin{cases}
	h(0,y) = U_1(y,0)\\
	h(x+1,y) = \max(h(x,y),U_1(y,x+1))
\end{cases}\]
Es decir, $h = R(g',h')$ donde $g'(y) = U_1(y,0)$ y $h'(x,y,z)=\max(z,U_1(y,x+1))$. Como $g'$ y $h'$ son recursiva, $h$ es recursiva.
\end{solucion}

\newpage

\begin{ejercicio}{18}
Sea \texttt{p} un programa \texttt{GOTO} para el que existe $t \in \N$ tal que:
$\texttt{STEP}^{(1)}(1, \#(\texttt{p}), t) \neq \texttt{STEP}^{(1)}(1, \#(\texttt{p}), t + 1)$.
\begin{enumerate}
\item ¿Es cierto que $[\![p]\!]^{(1)}(1) \downarrow$? Justifica la respuesta.
\item Sea \texttt{p} tal que $\#(\texttt{p}) = 898$.
\begin{itemize}
\item[(a)] Calcula el valor de $t$ que verifica la desigualdad anterior.
\item[(b)] Calcula el valor de $(r(\texttt{di}^{(1)}(1, 898, t - 1)))_1$.
\item[(c)] Calcula $\varphi_{898}(0)$.
\end{itemize}
\end{enumerate}
\end{ejercicio}
\begin{solucion}\
\begin{enumerate}
\item Sí, de hecho para en $t+1$ pasos exactamente, pues por definición $\texttt{STEP}^{(1)}(1, \#(\texttt{p}), t) \leq \texttt{STEP}^{(1)}(1, \#(\texttt{p}), t + 1)$, luego para que sean distintos $\texttt{STEP}^{(1)}(1, \#(p), t)=0$ y $\texttt{STEP}^{(1)}(1, \#(\texttt{p}), t + 1)=1$, por lo que efectivamente para exactamente en $t+1$ pasos. 
\item Empezando descifrando qué programa tiene ese código. $\#(\texttt{p}+1)=899=29\cdot 31$, luego el programa tiene 11 instrucciones. Las 9 primeras son todas de la forma $Y\leftarrow Y$. Calculamos las otras 2. $29=2^a(2y+1)-1\Rightarrow 30=2^a(2y+1)\Rightarrow a=1, y=7$, y a su vez $y=7=2^b(2c+1)-1\Rightarrow 8=2^b(2c+1)\Rightarrow b=3, c=0$. Lo que significa que $I_{10}: [A_1]\ IF\ Y\neq 0\ GOTO\ A_1$.

Realizamos el mismo proceso para la última instrucción. $31=2^a(2y+1)-1\Rightarrow 32=2^a(2y+1)\Rightarrow a=5, y=0\Rightarrow b=c=0$. Por lo que $I_{11}: [E_1]: Y\leftarrow Y$. 
\begin{itemize}
\item[(a)] Como el valor de \texttt{Y} permanece inalterado, el programa para en 11 pasos, por lo que $t=10$.
\item[(b)] El valor de \texttt{Y} y el de \texttt{X} es 0 todo el tiempo, lo que $(r(\texttt{di}^{(1)}(1, 899, t - 1)))_1=0$.
\item[(c)] Como $\varphi_{898}(0)=\varphi_{\#(\texttt{p})}(0)\Rightarrow\varphi_{899}(0)=[\![\texttt{p}]\!](0)=0$.
\end{itemize}
\end{enumerate}
\end{solucion}


\end{document}