\documentclass[twoside]{article}
\usepackage{../estilo-ejercicios}
\usepackage{wasysym}
\usetikzlibrary{automata,positioning}
\usepackage{mathdots}
\newcommand{\PR}{\mathcal{PR}}
%--------------------------------------------------------
\begin{document}

\title{Ciencias de la Computación}

\author{Javier Aguilar Martín}
\maketitle

\begin{ejercicio}{4}
Dada $f : \N \to \N$ definimos $it_f : \N^2 \to \N$ como
$$\forall (n, x) \in \N^2, it_f (n, x) = f^n(x) = f(f(\overset{n}{\cdots}f(x)))$$
Probar que si $f \in \mathcal{PR}$ entonces $it_f \in \mathcal{PR}$.

\end{ejercicio}
\begin{solucion}
Adoptamos por convenio que $f^0(x)=x$. Es claro que $f^{n+1}(x)=ff^n(x)$.
Entonces
\[
\begin{cases}
it_f(x,0)=x\\
it_f(n+1,x)=f(it_f(n,x))
\end{cases}
\]
Entonces $it_f=\mathcal{R}(g,h)$, donde $g(x)=x$, que es $\mathcal{PR}$ y $h(n,x,z)=f(z)\Rightarrow h=\mathcal{C}(f,\Pi^3_3)$ que también es $\mathcal{PR}$
\end{solucion}

\newpage

\begin{ejercicio}{6}
Probar que la función $f:\N\to\N$ dada por $$f(x)=\text{número de dígitos de }x\text{ en base }10$$ es primitiva recursiva.
\end{ejercicio}
\begin{solucion}
Nótese, que al sumar 1, el cambio en el número de cifras se produce si el resultado de la suma es una potencia de 10. Por tanto, 
\begin{align*}
&f(0)=1\\
&f(x+1)=\begin{cases}
f(x)+1 & x+1\text{ es potencia de } 10\\
f(x) & c.c.
\end{cases}
\end{align*}
Definimos $h(x,f(x))=\begin{cases}
f(x)+1 & x+1\text{ es potencia de } 10\\
f(x) & c.c.
\end{cases}$, por lo que $h(x,z)=\begin{cases}
z+1 & x+1\text{ es potencia de } 10\\
z & c.c.
\end{cases}$

Si las funciones de cada caso y las funciones características de las condiciones son primitivas recursivas entonces la función definida por casos es primitiva recursiva, por lo que $h$ lo sería y por tanto $f$ también. Vamos a ver que el predicado de la condición es primiticvo recursivo. $u$ es potencia de $10$ si y solo si existe $u=10^w$. El predicado $u=10^w$ es primitivo recursivo por ser ecuación entre funciones primitivas recursivas. Llamamos $P(u,w)\equiv u=10^w$. Bastaría comprobar si $P(u,j)=1$ para algún $j=0,\dots, u$. Entonces $u$ es una potencia de 10 si y solo si $P(u,w)=sgn(\sum_{j=0}^u P(u,j))$. Atención porque esta suma no es fija, por lo que no podemos argumentar que la suma es primitiva recursiva. Pero el índice de la suma es primitiva recursiva, por lo que la suma lo es.


También podemos definir $f(x)=(\mu t)_{\leq x+1}(t>0\land 10^t> x)$ 
\end{solucion}

\newpage

\begin{ejercicio}{7}
Sea $f(0)=0$, $f(1)=1$, $f(2)=2^2$, $f(3)=3^{3^3}$, $f(n)= n^{n^{\iddots^n}}$. Probar que $f\in\mathcal{PR}$
\end{ejercicio}
\begin{solucion}
La recurrencia será de la forma $F(x,k)=x^{x^{\iddots^x}}$ ($k$ veces) y $F(x,k+1)=x^{F(x,k)}$. Buscamos ver que $f(x)=F(x,x)$, probando previamente que $F$ es primitiva recursiva. Necesitamos que $F(x,0)=g(x)$ y $F(x,k+1)=h(x,k,F(x,k))$. 
\begin{align*}
&f(0)=F(0,0)=g(0)=0\\
&f(1)=F(1,1)=1^{F(0,0)}=1^{g(1)}=1\\
&f(2)=F(2,2)=2^{F(2,1)}=2^{2^{F(2,0)}}=2^{2^{g(2)}}
\end{align*}
Bata tomar $g(x)=sgn(x)=\begin{cases}
0 & x=0\\
1 & x\neq 0
\end{cases}$. Tomamos como $h(x,y,z)=x^z=exp(x,z)=exp(\Pi^3_1(x,y,z),\Pi^3_3(x,y,z))$, por lo que $h=\mathcal{C}(exp; \Pi^3_1,\Pi^3_3)$. Faltaría probar por inducción que $\forall n\in\N\ f(n)=F(n,n)$. Ya hemos visto que para $n=0$ es cierto. Entonces $F(n,n)=n^{F(n,n-1)}=\cdots = f(n)$. 
\end{solucion}

\newpage

\begin{ejercicio}{8}
Probar que las siguientes funciones de $\N$ en $\N$ son primitivas recursivas:
\begin{enumerate}
\item $\begin{cases}
f(0)=0\\
f(x) =\text{la suma de los divisores de } x & x\neq 0
\end{cases}$
\item $g(x)=$ el número de primos menores o iguales que $x$.
\item $h(x)=$ el único $n$ tal que $n\leq\sqrt{2}x<n+1$.
\end{enumerate}
\end{ejercicio}
\begin{solucion}
\begin{enumerate}
\item Es una definición por casos con predicados $\mathcal{PR}$. La función 0 es primitiva reursiva. Solo tenemos que convertir la suma en una suma acotada. 
$$\sum_{d|x}d=\sum_{j=0}^xP(j,x)j$$
donde $P(x,j)=j|x$. Entonces $g(x,z)=\sum_{j\leq z}P(x,j)j\in\mathcal{PR}$. Entonces $h(x)=g(x,x)\in\mathcal{PR}$. Así que definimos finalmente
$$f(x)=\begin{cases}
0(x) & x=0\\
h(x) & x\neq 0
\end{cases}$$
\item $g(x)=\sum_{j\leq x}Primo(j)$
\item Sabiendo que es único, el menor será el único. Así que solo hay que verificar que la condición se cumple. Como $n\leq\sqrt{2}x$, también $n\leq 2x^2$ (valdría cualquier cota superior a la que tenemos).
$$h(x)=(\mu n)_{\leq 2x^2}(n^2\leq 2x^2\land 2x^2\leq (n+1)^2)$$
\end{enumerate}
\end{solucion}

\newpage

\begin{ejercicio}{9}
Sea $f : \N \to \N$ la función cuyos valores sucesivos son
$$1, 1, 2, 1, 2, 3, 1, 2, 3, 4, 1, 2, 3, 4, 5, \dots$$
Probar que $f \in \mathcal{PR}$.
\end{ejercicio}
\begin{solucion}
Consideramos $\frac{k(k+1)}{2}\leq x < \frac{(k+1)(k+2)}{2}$, esto significaría que $x$ estaría en el $k-$ésimo bloque. Es decir, buscamos el primer número triángular tal que el siguiente se pase de $x$. Por ejemplo, $x=17$ está entre el quinto bloque y el sexto. La posición en la que acaba el bloque anterior (empezando por 0) es 14, por lo que basta contar hasta llegar, o lo que es lo mismo $f(x)=17-14=3$.
$$f(x)=x-\frac{r(r+1)}{2}$$
donde $r=(\mu t)_{\leq x}\left(\frac{(t+1)(t+2)}{2}>x\right)$. 
\end{solucion}

\newpage

\begin{ejercicio}{12}
Sea $k \in \N$ un número fijo y $f, g : \N \to \N$ tales que $\forall x \in N, f(x + 1) < x + 1$.
Probar que si $f, g \in \mathcal{PR}$ entonces la función $h : \N \to \N$ definida por
\begin{align*}
&h(0) = k
&h(x + 1) = g(h(f(x + 1)))
\end{align*}
es primitiva recursiva (utilícese la \textbf{función recorrido}).
\end{ejercicio}
\begin{solucion}
Por indución en $n$ podemos ver que $h(n)$ está definida para todo $n$, pues si está definida para todo número menor que $n+1$, entonces también está definida para $n+1$ puesto que $f(n+1)<n+1$. 

La función recorrido guarda los valores de $h$ en una lista codificada: $\hat{h}(0)=[h(0)],\hat{h}(x)=[h(0),\dots,h(x)]$. Si $\hat{h}\in\mathcal{PR}$, entonces $h\in\mathcal{PR}$, pues $h(x)=(\hat{h}(x))_{x+1}$ (la componente $x+1$-ésima). Probemos entonces que $\hat{h}\in\mathcal{PR}$. $\hat{h}(0)=[k]=2^k$, $\hat{h}(n+1)=[h(0),\dots,h(n),h(n+1)]=\underbrace{p_1^{h(0)}\cdots p_{n+1}^{h(n)}}_{\hat{h}(n)}p_{n+2}^{h(n+1)}=\hat{h}(n)p_{n+1}^{h(n+1)}$. Tenemos que eliminar $h$ de la definición y para eso utilizamos la ecuación de recurrencia. $\hat{h}(n+1)=\hat{h}(n)p_{n+1}^{g(h(f(n+1)))}$. Como $f(n+1)<n+1$, $h(f(n+1))$ está condificado en la lista, luego $h(f(n+1))=(\hat{h}(n))_{f(n+1)+1}$. Finalmente
$$\hat{h}(n+1)=\hat{h}(n)p_{n+1}^{g(\hat{h}(n))_{f(n+1)+1})}$$
Así pues $\hat{h}=\mathcal{R}(2^k, H)$, donde $H(n,z)=zp_{n+2}^{g(z)_{f(n+1)+1}}$, que es PR. 
\end{solucion}

\newpage

\begin{ejercicio}{14}
Consideremos las funciones totales $f, g : \N \to \N$ caracterizadas por las relaciones
siguientes:
\[
\begin{cases}
f(0) = 0\\
f(n + 1) = 1 + g(n)
\end{cases}\begin{cases}
g(0) = 0\\
g(n + 1) = 2 + f(n)
\end{cases}
\]
Probar que las funciones $f$ y $g$ son primitivas recursivas. (Indicación: Considérese la función
$h : \N \to \N$ definida por $h(n) = [f(n), g(n)]$).
\end{ejercicio}
\begin{solucion}
El corchete es la codificación que se ha visto en clase. $h(0)=\langle 0,0\rangle =2^0(2\cdot 0+1)-1=1-1=0$. $h(n+1)=\langle f(n+1),g(n+1)\rangle=\langle 1+g(n),2+f(n)\rangle=\langle 1+r(h(n)),2+l(h(n))\rangle$. Entonces $h=\mathcal{R}(0,H)$, donde $h(0)=0, h(n+1)=H(n,h(n))$, siendo $H(n,z)=\langle 1+r(z),2+l(z)\rangle$. Como la función par (la del corchete) es $\PR$, y cada lado es una proyección, $H$ es $\PR$, con lo que $h\in\mathcal{PR}$.
\end{solucion}

\newpage

\begin{ejercicio}{15}
Dada una función parcial $f : \N \dashrightarrow \N$ definimos la función parcial $v_f : \N\dashrightarrow \N$
como sigue:
\[ v_f (x) = |\{y \in \N : y < x \land f(y) = 0\}|\]
Probar que si $f$ es primitiva recursiva, entonces la función $v_f$ también lo es.

\end{ejercicio}
\begin{solucion}
$v_f(0)=0, v_f(x+1)=v_f(x)+\overline{sg}(f(x))$. Otra posibilidad es observar $v_f(x)=\sum_{t< x}\overline{sg}(f(x))$. 
\end{solucion}

\newpage

\begin{ejercicio}{16}
Dado un predicado primitivo recursivo $R(x, y)$, definamos
\[g(x, z) = \max_{y\leq z}R(x, y)\]
como el mayor $y \leq z$ tal que $R(x, y)$, si existe, y 0 si no existe ningún elemento $y \leq z$ tal que
$R(x, y)$. Probar que $g \in \mathcal{PR}$.
\end{ejercicio}
\begin{solucion}
$(x,z)$ es el elemento $y\leq z\land R(x,y)\land \forall u\leq z(u>y\rightarrow \neg R(x,u))$, que es un predicado $\mathcal{PR}$ al que llamaremos $P(x,y,z)$. Entonces $g(x,z)=(\mu y)_{\leq z} P(x,y,z)$. Hemos definido esta función en los apuntes de modo que si no existe tal mínimo en predicados entonces devuelve 0, pero si no fuera así, podría definirse por casos.
\end{solucion}

\newpage

\begin{ejercicio}{19}
Probar que las siguientes funciones son primitivas recursivas
\begin{enumerate}
\item $\begin{cases}
h(0) = 3\\
h(x + 1) =\sum_{t=0}^x h(t)
\end{cases}$

\item $\begin{cases}
f(0) = 1, f(1) = 4, f(2) = 6\\
f(x + 3) = f(x) + f(x + 1)^2 + f(x + 2)^3
\end{cases}$
\end{enumerate}
\end{ejercicio}
\begin{solucion}\
\begin{enumerate}
\item Usamos la función recorrido. $\hat{h}(0)=[3]=2^3=8$, $\hat{h}(x+1)=\hat{h}(x)p_{x+2}^{\sum_{t=0}^xh(t)}=\hat{h}(x)p_{x+2}^{\hat{h}(x)_{t+1}}$. Podemos escribir $\hat{h}(x+1)=H(x,\hat{h}(x))$, donde $H(x,z)=zp_{x+2}^{\sum_{t\leq x} (z)_{t+1}}=zp_{x+2}^{s(x,z)}$. La suma acotada es PR, así que todo lo demás es PR.
\item Podemos usar $F(0)=[1,4,6]=2\cdot 3^4\cdot 5^6$, $F(x)=[f(x),f(x+1),f(x+2)]$. Entonces $F(x+1)=[f(x+1),f(x+2),f(x+3)]=[(F(x))_2,(F(x))_3,(F(x))_1+(F(x))_2^2+(F(x))_3^3]$. 
\end{enumerate}
\end{solucion}

\newpage

\begin{ejercicio}{22}
Sea $f : \N^2 \to \N$ definida por: $f(u, 0) = 1$ y $f(u, x) = f(u, x\dot{-}
(u+1))+5$ si $x > 0$.
Se pide:
\begin{enumerate}
\item Probar que para todos $u, x$ se verifica que $f(u, x)\downarrow$. ¿Es inyectiva?
\item Diseñar un programa \texttt{GOTO, p}, tal que $[\![\texttt{p}]\!] = f$.
\item Probar que $f$ es primitiva recursiva.
\end{enumerate}
\end{ejercicio}
\begin{solucion}\
\begin{enumerate}
\item Tenemos que $f(u,0)\downarrow$ para todo $u$. Si, dado $x\in\N$, $f(u,z)\downarrow\ \forall z< x$ entonces por la definición, $f(u,x)\downarrow$. Claramente no es inyectiva porque $f(u,0)=1\ \forall u\in\N$. 
\item 
\begin{align*}
& Y\leftarrow Y+1\\
& Z\leftarrow X_2\\
& IF\ Z\neq 0\ GOTO\ A\\
& GOTO\ E\\
[A]& Z\leftarrow Z-X_1-1\\
& Y\leftarrow Y+5\\
& IF\ Z\neq 0\ GOTO\ A\\
\end{align*}
\item Fijamos el parámetro $u$ y usamos la función recorrido $\hat{f}_u(x)=[f(u,0),\dots,f(u,x)]=\prod_{i=1}^{x+1}p_i^{f(u,i-1)}=\hat{f}_u(x-1)p_{x+1}^{f(u,x)}=\hat{f}_u(x-1)p_{x+1}^{f(u,x\dot{-}(u+1))}=\hat{f}_u(x-1)p_{x+1}^{(\hat{f}_u(x))_{x\dot{-}(u+1)+1}}$. Por lo que $\hat{f}_u\in\PR$ para cada $u$. Ahora, si $x\dot{-}(u+1)+1=0$, entonces $f(u,x)=f(v,x)\ \forall v\geq u$, luego $\hat{f}(r,x)=\hat{f}_r(x)$, donde $$r=(\mu u)_{\leq x}((\hat{f}_u(x))_{x\dot{-}(u+1)+1}=(\hat{f}_{u+1}(x))_{x\dot{-}(u+2)+1}).$$ Por lo que $\hat{f}(u,x)\in\PR$, y esto implica que $f\in\PR$. 
\end{enumerate}
\end{solucion}
\newpage

\begin{ejercicio}{}
La función de Fibonacci no es $\mathcal{PR}$.
\end{ejercicio}
\begin{solucion}
Supongamos que tenemos una función $F(n)\in\mathcal{PR}$ que guarda los valores de $f(n)$ y $f(n+1)$ (lo expresaremos como un par) y que tenemos funciones $\mathcal{PR}$ $L(F(n))=f(n)$ y $R(F(n))=f(n+1)$. Entonces podríamos calcular $F(n+1)=(f(n+1),f(n+1))=(R(F(n)), L(F(n))+R(F(n)))$ y sería $\mathcal{PR}$. Definimos la función de Cantor $\langle x,y\rangle =2^x(2y+1)-1$. Si $\langle x,y\rangle =n\Rightarrow 2^x(2y+1)=n+1$, luego $x$ será el exponente de 2 en la descomposición en primos de $n+1$ y con eso podemos obtener también $y$. Así pues, $x=L(n), y=R(n)$. 
\end{solucion}

\newpage
\begin{ejercicio}{24}
Sea $f(x,y) = (μz)[\text{Primo}(z) \land x = z \cdot y]$.
\begin{itemize}
	\item Determínense el dominio y el rango de $f$. ¿Es $f$ inyectiva?
	\item ¿Es $f$ recursiva?
\end{itemize}
\end{ejercicio}
\begin{solucion}
Claramente $(x,y) \in \text{dom}(f)$ si y sólo si $(\exists z)(\text{Primo}(z) \land x = z\cdot y)$. Esto es equivalente a que $\text{Primo}(qt(x,y))$.

Se tiene $z \in \text{rango}(f)$ sii $(\exists x)(\exists y) (f(x,y)=z)$ sii $(\exists x)(\exists y) (x=z\cdot y \land \text{Primo}(z))$ sii $\text{Primo}(z)$.

Todas las funciones involucradas son recursivas, luego $f$ es recursiva.
\end{solucion}

\newpage

\begin{ejercicio}{26}
Sea $f : \N \to \N$, recursiva y estríctamente creciente (es decir, para todo $x \in \N,
f(x) < f(x + 1))$. Probar que $rang(f)$ es un conjunto recursivo (Indicación: Probar que para todo
$n$, $f(n) \geq n)$.
\end{ejercicio}
\begin{solucion}
Tal como se nos indica, empezamos probando que para todo $n$, $f(n) \geq n$. Necesariamente $f(0)\geq 0$. Si $f(n-1)\geq n-1$, entonces por ser estrictamente creciente, $f(n)>f(n-1)$, luego $f(n)\geq n$. 

Por definición, $y\in rang(f)\Leftrightarrow (\exists x)(f(x)=y)$. Por lo anterior, $y=f(x)\geq x$, luego podemos acotar el existencial como $(\exists x)_{\leq y}(f(x)=y)$. Por tanto $rang(f)$ es recursivo.
\end{solucion}

\newpage

\begin{ejercicio}{27}
Sea $\theta(x)$ un predicado recursivo de aridad 1. Pruébese que la siguiente función es
recursiva:
\[
f(n) =\begin{cases}
0 & \text{si } n=0\\
n\text{-ésimo número }x \text{ tal que }\theta(x) & \text{si }n > 0
\end{cases}
\]
\end{ejercicio}
\begin{solucion}
Basta con comprobar que la función del segundo caso es recursiva, pues la primera lo es claramente y los predicados de cada caso lo son (de hecho todo es primitivo recursivo). Entonces, $x$ es el enésimo número que verifica $\theta$ si y solo si $\sum_{y\leq x}\theta(y)=n$. Esta función es recursiva, por lo que se tiene el resultado.
\end{solucion}

\newpage

\begin{ejercicio}{28}
Sea $f : \N \to \N$ recursiva y biyectiva. Probar que $f^{-1}$ es recursiva.
\end{ejercicio}
\begin{solucion}
Ya vimos un programa \texttt{GOTO} para calcular la inversa cuando $f$ es primitiva recursiva. Al ser $f$ total el mismo programa es válido, por lo que $f^{-1}$ es recursiva. 
\end{solucion}

\newpage

\begin{ejercicio}{29}
Sea $\theta$ un predicado primitivo recursivo de aridad 1. Denimos $f : \N^2 \to \N$ como:
$f(x, y) = z$, donde $z$ es el número de veces que $\theta$ se satisface en el intervalo $[0, x + y]$.
\begin{enumerate}
\item Probar que $f$ es recursiva diseñando un programa \texttt{GOTO} que la calcule.
\item Probar que $f$ es primitiva recursiva.
\end{enumerate}
\end{ejercicio}
\begin{solucion}\
\begin{enumerate}
\item 
\begin{align*}
& Z\leftarrow X_1+X_2\\
[C]& IF\ Z\neq 0\ GOTO\ A\\
&IF\ \theta(Z)\ GOTO\ D\\
& GOTO\ E\\
[D] & Y\leftarrow Y+1\\
& GOTO\ E\\
[A] & IF\ \theta(Z)\ GOTO\ B\\
& Z\leftarrow Z-1\\
& GOTO\ C\\
[B]& Y\leftarrow Y+1\\
& Z\leftarrow Z-1\\
& GOTO\ C
\end{align*}
\item $z$ es el número de veces que $\theta$ se satisface en el intervalo $[0, x + y]$ si y solo si $\sum_{n\leq x+y}\theta(n)=z$, por lo que es es primitivo recursivo. 
\end{enumerate}
\end{solucion}

\newpage

\begin{ejercicio}{30}
Dadas $H : \N^{n+1} \to \N$, $g : \N^n \to \N$ y $h : \N^{n+2} \to \N$, diremos que $f$ se define por \textbf{recursión limitada} mediante $g$, $h$ y $H$, (se notará por $f = R_l(g,h;H)$) si
\[ f = R(g,h) \text{ y } \forall \vec{x} \forall y [f(x,y) ≤ H(x,y)] \]

Se define $\mathcal{E}^0$ como el menor conjunto de funciones tal que contiene a la función constante e igual a $0$, la función siguiente y las proyecciones, y es cerrado bajo composición y recursión limitada.

Probar por inducción en la clase $\mathcal{E}^0$ que para toda función $f : \N^n \to \N$, de dicha clase, existen $k$ y $j ≤ n$ tales que para toda $n$-upla $\vec{x}$:
\[ f(\vec{x}) ≤ x_j + k \]
\end{ejercicio}
\begin{sol}
Lo demostramos por inducción en la definición de $f \in \mathcal{E}^0$. En el caso base:
\begin{itemize}
	\item Si $f = 0$, $f(x) ≤ x_1+0$.
	\item Si $f = S$, $f(x) ≤ x+1$.
	\item Si $f = Π_i^n$, $f(\vec{x}) ≤ x_i+0$.
\end{itemize}
Aplicamos el paso de inducción. Supongamos que $f = \C(g;h_1,\dots,h_n)$ siendo $g : \N^n \to N$ y $h_1,\dots,h_n : \N^m \to \N$ tales que por hipótesis de inducción, existen $k_0,k_1,\dots,k_n \in \N$, $j_0 : 1 ≤ j_0≤n$, y para cada $1≤i≤m$, $1≤j_i≤m$ verificando:
\[ g(x_1,\dots,x_n) ≤ x_{j_0}+k_0 \]
\[ h_i(x_1,\dots,x_m) ≤ x_{j_i} + k_i \quad \forall 1≤i≤n\]
Entonces:
\[ f(x_1,\dots,x_m) = g(h_1(x_1,\dots,x_m),\dots,h_n(x_1,\dots,x_m)) ≤ h_{j_0}(x_1,\dots,x_m) + k_0 ≤ x_{j_{j_0}} + (k_{j_0} + k_0) \]

Supongamos ahora que $f = R_l(g,h;H)$, $f : \N^{n+1} \to \N$, $g : \N^n \to \N$, $h : \N^{n+2} \to \N$, $H : \N^{n+1}\to\N$ y verifican que:
\begin{enumerate}
	\item Para ciertos $1≤i≤n$, $k \in \N$, $g(\vec{x})≤x_i+k$.
	\item Para ciertos $1≤j≤n+2$, $k' \in \N$, $h(x_1,\dots,x_{n+2})≤x_j+k'$.
	\item  Para cierto $1≤l≤n+1$, existe $k'' \in \N$ con $H(x_1,\dots,x_{n+1}) ≤ x_l + k''$.
\end{enumerate}
Entonces, puesto que $f = R_l(g,h;H)$, $\forall (\vec{x},y) \in \N^{n+1}$
\[ f(\vec{x},x_{n+1}) ≤ H(\vec{x},x_{n+1}) ≤ x_l + k'' \]
\end{sol}

\end{document}