\documentclass[11pt]{article}
\usepackage{../estilo-ejercicios}
\usepackage{wasysym}
\usetikzlibrary{automata,positioning}
\usepackage{mathdots}
\usepackage{listings}

\setlength{\parindent}{0em}
\setlength{\parskip}{1em}
\newcommand\tq{{\ \ /\ \ }}
\let \sii \Leftrightarrow
\let \luego \Rightarrow

\begin{document}
\begin{ejercicio}{1}
\end{ejercicio}
\begin{solucion}
\begin{enumerate}
	\item 
	\item \[ A = \{ e \in \N : \exists x : F(x,e,t) = x \} \]
entonces:
\[ e \in A \sii \exists x \exists t : F(x,e,t) = x \]
Tomando $R(e,x) \sii F(x,e,t) = x$. Como $F$ es recursivo, $R$ es recursivo. Por el ejercicio 5.1 (consecuencia del teorema de proyección), $A$ es r.e.
\end{enumerate}
\end{solucion}

\begin{ejercicio}{2}
\end{ejercicio}
\begin{solucion}
\begin{enumerate}
	\item Sea
	\[  Γ = \{g \in \mathcal{P}^{(1)} : rang(g) \subseteq Prim\} \]
	Vemos que $Γ \neq \emptyset$ con el programa constante $2$ y vemos que $Γ \neq \mathcal{P}^{(1)}$ con el programa constante $1$. Entonces $A_1 = I_Γ$ es no recursivo por el teorema de Rice.
	
	\item % ¿Se puede hacer por teorema de Rice o hay que hacerlo por argumento diagonal?
\end{enumerate}
\end{solucion}
\end{document}
