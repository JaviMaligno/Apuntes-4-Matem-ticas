\documentclass[PM.tex]{subfiles}
\begin{document}

\chapter{Dualidad en programación lineal}
\section{Introducción}
A cada problema de programación lineal

\begin{center}
\begin{tabular}{cccccc}
(P) & $\min$ & $c_1'x_1$ & $+c_2'x_2$   & $+c_3'x_3$\\
(u') & sa: & $A_{11}x_1$ & $+A_{12}x_2$ & $+ A_{13}x_3$ & $≥ b_1$\\
(v') &     & $A_{21}x_1$ & $+A_{22}x_2$ & $+ A_{23}x_3$ & $≤ b_2$\\
(w') &     & $A_{31}x_1$ & $+A_{32}x_2$ & $+ A_{33}x_3$ & $= b_3$\\
     &     &       $x_1$ &              &               & $≥ 0$\\
     &     &             &      $x_2$   &               & $≤ 0$\\
     &     &             &              &         $x_3$ & libre
\end{tabular}
\end{center}
al que llamaremos primal (P) y donde $x_1, x_2$ y $x_3$ son vectores de variables de decisión, le vamos a asociar otro problema de programación lineal al que llamaremos dual (D) dado por las siguiente transformación:

\begin{center}
\begin{tabular}{cccccc}
(D) & $\max$ & $u'b_1$ & $+v'x_2$ & $+w'x_3$\\
    & sa: & $u'A_{11}$ & $+v'A_{12}$ & $+w'A_{13}$ & $≤ c_1'$\\
    &     & $u'A_{21}$ & $+v'A_{22}$ & $+w'A_{23}$ & $≥ c_2'$\\
    &     & $u'A_{31}$ & $+v'A_{32}$ & $+w'A_{33}$ & $= c_3'$\\
    &     &       $u$  &             &             & $≥ 0$\\
    &     &            &      $v$    &             & $≤ 0$\\
    &     &            &             &      $w$    & libre
\end{tabular}
\end{center}

La transformación que lleva (P) en (D) crea por cada restricción de (P) una variable de decisión en (D) y por cada variable de decisión en (P) una restricción en (D) de a forma que se indica en la siguiente tabla:

\begin{center}
\begin{tabular}{|c|c|}
\hline
	$\min$ & $\max$\\
\hline
	restricción ≥ & variable ≥ 0\\
\hline
	restricción ≤ & variable ≤ 0\\
\hline
	restricción = & variable libre\\
\hline
	variable ≥ 0 & restricción ≤\\
\hline
	variable ≤ 0 & restricción ≥\\
\hline
	variable libre & restricción = \\
\hline
\end{tabular}
\end{center}

\begin{example}
Consideramos el problema
\begin{align*}(P)
	\min c'x\\
	\text{sa:} Ax ≥ b\\
	x ≥ 0
\end{align*}
podemos transformarlo en el problema:
\begin{align*}(P')
	\min c'x + 0x_a\\
	\text{sa:} Ax-Ix_a = b\\
	x, x_a ≥ 0 
\end{align*}
y finalmente pasarlo al problema dual
\begin{align*}(D)
	\max u'b\\
	\text{sa:} u'A ≤ c'\\
	u ≥ 0
\end{align*}
\end{example}
\begin{nota} Denotaremos $X = \{x \in \R^n : Ax ≥ b, x ≥ 0\}$ y $U = \{ u \in \R^m \mid u'A ≤ c', u ≥ 0 \}$.
\end{nota}
\begin{theorem}[Teorema de dualidad débil]
Para todo $x \in X$ y $u \in U$ se verifica que $c'x ≥ u'b$.
\end{theorem}

\begin{dem}
$\forall u \in U, \forall x \in X$, se tiene que $u'A ≤ c', x ≥ 0$, luego $u'A x ≤ c'x$. Como $u≥0$ y $Ax≥b$, llegamos a que $u'b ≤ u'Ax ≤ c'x$, lo que demuestra el teorema.
\end{dem}

\begin{obser}
En particular, también se verifica que:
\[ \max\limits_{u \in U} u'b ≤ \min\limits_{x\in X} c'x \]
\end{obser}

\begin{coro}
Si $x^* \in X$ y $u^* \in U$ y $c'x^* = (u^*)'b$, entonces $x^*$ y $u^*$ son soluciones óptimas de (P) y (D) respectivamente.
\end{coro}

\begin{dem}
Por la observación anterior, sigue inmediatamente.
\end{dem}

\begin{theorem}[Teorema de dualidad fuerte]
Si $X \neq \emptyset$ y $U \neq \emptyset$, entonces existen $x^*$ y $u^*$ soluciones óptimas de (P) y (D) respectivamente. Además, $c'x^* = {u^*}'b$.
\end{theorem}

\begin{dem}
Como $U \neq \emptyset$, existe algún punto $\overline{u} \in U$. Entonces $\forall x \in X$, $x'c ≥ \overline{u}'b$. Por tanto (P) no puede decrecer a $-\infty$ y esto indica que debe haber una solución en un punto extremo $x^* \in X$, determinado por una base $B$. Así que:
\[ x^* = \begin{pmatrix}B^{-1} b \\ 0\end{pmatrix}, \qquad B^{-1}b ≥ 0 \]

Veamos ahora que $(u^*)' = c_B'B^{-1}$ es la solución óptima de (D). Es decir, hay que ver que (1) $(u^*)' \in U$ y (2) $(u^*)'b = c'x^*$. Vemos que:
\[ (u^*)'A ≤ c' \equiv (u^*)' (B \ N) ≤ (c_B' \ c_N') \]
Luego:
\[ (u^*)'B ≤ c_B' \equiv c_B'B^{-1}B ≤ c_B' \equiv c_B' ≤ c_B'\]
\[ (u^*)'N ≤ c_N' \equiv c_B'B^{-1}N ≤ c_N' \equiv c_N' - c_B'B^{-1}N ≥ 0\]
Como las ecuaciones del miembro derecho se cumplen, llegamos a que $(u^*)'A ≤ c'$. Por otro lado, dado que $x^*$ es solución óptima de (P) entonces los costes reducidos de las variables asociadas a $x_a$ en $Ax - I x_a = b$ deben ser no negativas. Para cualquier columna $a \in A$ o $-e_j \in -I$, se tiene que:
\[ c_a - c_B' B^{-1} a ≥ 0 \]
\[ c_{e_j} - c_B' B^{-1}(-e_j) ≥ 0 \equiv  -c_B'B^{-1}(-e_j) ≥ 0 \equiv c_B'B^{-1}e_j = (u^*)'e_j ≥ 0 \]
para todo $j$, luego $u^* ≥ 0$, lo que demuestra (1). Por último:
\[ (u^*)'b = c_B'B^{-1}b = \begin{pmatrix}c_B' & c_N'\end{pmatrix}\begin{pmatrix}B^{-1}b\\0\end{pmatrix} = c'x^* \]
\end{dem}

\begin{lema}
El sistema
\begin{equation*}\label{sistema1}\begin{cases}
	Ax - tb ≥ 0\\
	-u'A + tc' ≥ 0\\
	u'b - c'x ≥ 0
\end{cases}\end{equation*}
admite una solución $(x_0,u_0,t_0) ≥ 0$ y verificando
\begin{equation*}\label{sistema2}\begin{cases}
	Ax_0 - t_0b +u_0     > 0\\
	-u'_0A + t_0c' + x_0  > 0\\
	u'_0b - c'x_0 + t_0 > 0
\end{cases}\end{equation*}
\end{lema}

\begin{dem}
Basta ver que la matriz del sistema es antisimétrica:
\[ \begin{pmatrix}
	0  &  A  & -b\\
	-A' &  0  & c\\
	b'& -c'  & 0 
\end{pmatrix}\begin{pmatrix}
	u\\x\\t
\end{pmatrix} ≥ 0 \]
\end{dem}

\begin{theorem}[Teorema de dualidad más fuerte de Gale]
Dados (P) y (D) entonces uno y sólo uno de los siguientes casos son ciertos:
\begin{enumerate}
	\item Ambos problemas tienen soluciones óptimas y sus valores coinciden.
	\item Un problema es infactible y el otro tiene solución ilimitada.
	\item Los dos problemas son infactibles.
\end{enumerate}
\end{theorem}

\begin{dem} Consideramos la solución $(x_0,u_0,t_0)\geq 0$ del sistema del lema anterior. Distinguimos 2 casos:
\begin{enumerate}
	\item $t_0 > 0$. Consideramos $x^* = \frac{x_0}{t_0}$, $u^* = \frac{u_0}{t_0}$ y $t^* = 1$. $(x^*,u^*,t^*)$ sigue siendo solucion del sistema homogéneo. Es decir:
\[\begin{cases}
	Ax^* - b ≥ 0\\
	-(u^*)'A + c' ≥ 0\\
	(u^*)'b - c'x^* ≥ 0
\end{cases}\]
Por lo tanto:
\[\begin{cases}
	Ax^* ≥ b & \Rightarrow x^* \in X\\
	(u^*)'A ≤ c' & \Rightarrow u^* \in U\\
	(u^*)'b ≥ c'x^* & \Rightarrow c'x^* = (u^*)'b
\end{cases}\]
Es decir, $(x^*,u^*)$ es solución óptima de (P) y (D) respectivamente.
	\item $t_0=0$. Se tiene:
\[\begin{cases}
	Ax_0 ≥ 0\\
	-(u_0)'A  ≥ 0\\
	(u_0)'b - c'x^* > 0
\end{cases}\]
	Podemos poner mayor estricto en la última ecuación porque $(u_0)'b-c'x^*+t_0 > 0$.
	\begin{enumerate}
	\item Supongamos que existe $\overline{x} \in X$ y $\overline{u} \in U$, de manera que:
	\[\begin{cases}
		A\overline{x} ≥ b & \overline{x} ≥ 0\\
		\overline{u}'A ≤ c' & \overline{u} ≥ 0
	\end{cases}\]
	\begin{itemize}
	\item Usando que $Ax_0 ≥ 0$ y $\overline{u} ≥ 0$, obtenemos que $\overline{u}'Ax_0≥0$. Si a esto lo sumamos que $\overline{u}'A ≤ c'$, tenemos que $0\leq\overline{u}'Ax_0≤c'x_0 $. En particular $c'x_0 ≥ 0$.
	\item  Usando que $A\overline{x} ≥ b$ y que $u_0 ≥ 0$ deducimos que $u_0'A\overline{x}≥u_0'b $. Por otro lado tenemos que $u'A ≤ 0$ y que $\overline{x} ≥ 0$, luego $0\geq u'A\overline{x}$. Deducimos que $0 \geq u_0'b$.
\end{itemize}
 De estos dos puntos deducimos que $c'x_0  \geq u_0'b$, pero por hipótesis $u_0'b >  c'x^*$ con lo que llegamos a una contradicción.
	\item Supongamos que $\overline{x}$ es solución de $P$ y $U = \emptyset$. Tenemos lo siguiente
	\[ \begin{cases}Ax_0 \geq 0 \\ -u_0'A ≥ 0 \\ u_0'b-c'x_0 > 0 \\  A\overline{x} ≥ b,\overline{x}≥ 0\end{cases} \]
	Vemos que:
	\[ A(\overline{x} + λx_0) = A\overline{x} + λ A x_0 ≥ b \quad \forall λ\geq 0\]
	Como $\overline{x} + λ x_0 ≥ 0$, $\overline{x} + λ x_0 \in X$ para todo $λ > 0$. Además $c'(\overline{x} + λ x_0) = c'\overline{x} + λ c' x_0$ pero veremos que $c'x_0 < 0$.
	\begin{itemize}
	\item Por un lado tenemos que  $u_0'A ≤ 0$ y que $\overline{x} ≥ 0$. Se tiene por tanto que $u_0 A\overline{x} \leq 0$.
	\item Por otro lado, $A \overline{x}\geq b$ y $u_0 \geq 0$, por lo que $u_0'A\overline{x} \geq u_0'b$.
	\end{itemize}
	De estos dos puntos se deduce que $u_0'b\leq 0$, pero de las hipótesis sabemos que $c'x_0 < u_0'b$, luego $c'x_0 <0$. Pero entonces $c'(\overline{x} + λ x_0) \to -\infty$ cuando $λ \to \infty$. Esto es que no puede darse solucion únicamente de un problema y, por descarte, el último caso es que ninguno tenga solución.
	\end{enumerate}
\end{enumerate}
\end{dem}

\begin{theorem}[Teorema de holgura complementaria fuerte]
Si $X \neq \emptyset$ y $U \neq \emptyset$, existen $(\overline{x},\overline{u})$ óptimos de (P) y (D) tales que:
\[ A\overline{x} -b + \overline{u} > 0\]
\[ -\overline{u}A + c' +\overline{x}' > 0\]
\end{theorem}

\begin{dem}
Estamos en el caso 1 del teorema anterior y, por tanto, $t_0 > 0$. Consideramos $\overline{x} = \frac{x_0}{t_0}$, $\overline{u}=\frac{u_0}{t_0}$ y $\overline{t} = 1$. $(\overline{x},\overline{u},\overline{t})$ es solución del sistema \eqref{sistema1} y cumple (evaluando en \eqref{sistema2}):
\[ A\overline{x} -b + \overline{u} > 0\]
\[ -\overline{u}A + c' +\overline{x}' > 0\]
\end{dem}

\begin{theorem}[Otro teorema de holgura complementaria] Sean $x^*$, $u^*$ soluciones factibles de P y D. Entonces $x^*$ y $u^*$ son soluciones óptimas si y sólo si $u^*(Ax^*-b)=0$, $(u'^{*} A-c')x^*=0$.
\end{theorem}
\begin{dem} Si se verifican las dos igualdades entonces:
\[
\begin{cases}
u'^* A x^* = u'^*b \\
u'^*A x^* = c'x^* 
\end{cases}
\quad c'x^* = u'^*b \Rightarrow \text{$x^*$ es óptima en P y $u^*$ en D}
\]
Recíprocramente, supongamos que se tiene $c'x^*=u'^*b$. Por hipótesis, $Ax^*\geq b$ y $u^* \geq 0$, luego:
\[
u'^*(Ax^*-b)\geq 0
\]
Si la desigualdad fuese estricta:
\[
u'^*Ax^* > u'^* b \Rightarrow c'x^* \geq u'^*Ax^*>u'^*b
\]
Esto es absurdo, pues por hipótesis $c'x^*=u'^*b$. Análogamente deducimos que $(u'^*A-c')x^* \leq 0$.
\end{dem}
\section{Análisis de postoptimalidad}
\subsection{Variaciones en la función objetivo}
Dado un problema $(P)$ $\min c'x$, $Ax=b$, $x\geq 0$ consideramos un vector $d\in\R^n$ y la familia de problemas: $P(\Delta)$: $\min c'x + \Delta d'x$, $Ax=b$, $x\geq 0$. Observemos que $P=P(0)$. Supongamos que B es una base asociada a una solución óptima de $x^*(0)$. Es decir, $x^*(0)=[B^{-1}b | 0]' \geq 0$. Queremos determinar el rango de valores de $\Delta$ que mantienen a $x^*(0)$ como solución óptima de $P(\Delta)$. La condición que certifica que $B$ es una base óptima son:
\begin{itemize}
\item $B^{-1}b \geq 0$.
\item $\overline{c_R(0)}'  = c'_N - c_B'B^{-1}N \geq 0$. 
\end{itemize}
Las condiciones en $P(\Delta)$. 
\begin{itemize}
\item $B^{-1}b \geq 0$.
\item $\overline{c_R}'(\Delta) = (c +\Delta d)'_N- (c+\Delta d)'_B B^{-1}N = \overline{c_R}'(0)  + \Delta (d_N'-d'_BB^{-1}N) =  \overline{c_R}'(0)_N + \Delta \overline{d}_N'\geq 0$. 
\end{itemize}
Para una componente genérica $j\in \mathbb{N}$ se tiene que $\overline{c}_j+\Delta \overline{d_j}\geq 0$. Entonces:
\[
\begin{cases}
-\frac{\overline{c}_j}{\overline{d}_j} \geq \Delta & si \quad \overline{d}_j <0\\
-\frac{\overline{c}_j}{\overline{d}_j} \leq \Delta & si \quad \overline{d}_j >0
\end{cases}
\Rightarrow \max_{j:\overline{d}_j >0}-\frac{\overline{c}_j}{\overline{d}_j} \leq \Delta \leq \min_{j:\overline{d}_j <0}-\frac{\overline{c}_j}{\overline{d}_j}
\]

\subsection{Variaciones en términos independientes}

Sea $z(0)= \min c'x$ sujeto a $Ax=b, x\geq 0$. Para cada $g\in\R^m$ queremos estudiar qué le ocurre al problema:
\begin{align*}
z(\Delta)=\min c'x\\
sa: Ax=b+\Delta g\\
\Delta\in\R
\end{align*}
Si $B$ es una base asociada a la solución óptima de $z(0)$, ¿cuál es el rango de $\Delta$ que mantiene $B$ óptima? Las condiciones para que sea óptima son:
\begin{enumerate}
\item Condición de optimalidad. Como nuestra solución es óptima en $\Delta =0$: \[\overline{c_R(0)}'=c'_N-c'_B B^{-1}N\geq 0.\]
Esta condición se sigue manteniendo en el problema modificado, es decir, $\overline{c_R(\Delta)}\geq 0$ $\forall \Delta \in \R$.
\item Factibilidad. En $z(0)$ es $B^{-1}b=\overline{b}\geq 0$. En $z(\Delta)$ es:
\[B^{-1}b(\Delta)=B^{-1}(b+\Delta  g)=B^{-1}b+\Delta B^{-1}g=\overline{b}+\Delta\overline{g}\geq 0\]
Tomemos un índice $j$ arbitrario. Entonces $\overline{b}_j+\Delta\overline{g}_j\geq 0$. Por tanto:
\[
\begin{cases}
-\frac{\overline{b}_j}{\overline{g}_j} \geq \Delta & si \quad \overline{g}_j <0\\
-\frac{\overline{b}_j}{\overline{g}_j} \leq \Delta & si \quad \overline{g}_j >0
\end{cases}
\Rightarrow \max_{j:\overline{g}_j >0}-\frac{\overline{b}_j}{\overline{g}_j} \leq \Delta \leq \min_{j:\overline{g}_j <0}-\frac{\overline{b}_j}{\overline{g}_j}\]
\end{enumerate}
\begin{example}
\begin{align*}
\min & -2x_1-x_2 &\\
sa: & x_1+x_2 &\leq 5\\
    & -x_1+x_2 & =0\\
    & 6x_1+2x_2 & \leq 21\\
    & x_1,x_2 & \geq 0
\end{align*}
Pongamos $g'=(0,0,1)$. La base óptima para $x_1,x_2,x_4$ de este problema era $B=[a_1\ a_2\ a_4]$. Imponemos $B^{-1}(b+\Delta g)\geq 0$ y despejamos $\Delta$ Obtenemos
\[
\begin{pmatrix}
\frac{11}{4}\\
\frac{9}{4}\\
\frac{1}{2}
\end{pmatrix}+\Delta\begin{pmatrix}
\frac{1}{4}\\
\frac{-1}{4}\\
\frac{1}{2}
\end{pmatrix}\geq 0
\]
Por lo tanto $\Delta\in [-1,9]$. Esto quiere decir que $B$ es óptima para $b_3\in[20,30]$. La solución $x(\Delta)=B^{-1}b(\Delta)$ en cada caso será distinta, pero la base será la misma.
\end{example}

\subsection{Añadir una nueva variable $x_{n+1}$ a un problema}
Dado un problema $(P)$ $\min c'x$, $Ax=b$, $x\geq 0$. Queremos estudiar cuánto sigue siendo óptima una base de (P) en el problema $(P')$
\begin{align*}
\min\ & c'x+c_{n+1}x_{n+1}\\
sa:\ & [A|a_{n+1}]\begin{bmatrix}
x\\
x_{n+1}
\end{bmatrix}=b\\
 &  x,x_{n+1}\geq 0
\end{align*}
Dada $B$ una base óptima en $(P)$ queremos determinar una condición para que sea óptima en $(P')$. Las condiciones para $(P')$ son:
\begin{enumerate}
\item Factibilidad, que es la misma que en $(P)$, $B^{-1}b\geq 0$. 
\item Optimalidad. Debe verificarse que:
\[\overline{c}_N(P')=(c'_N, c_{n+1})-c'_B B^{-1}[N| a_{n+1}]\geq 0\]
Lo cual es equivalente a que se cumplan al mismo tiempo
\[
\begin{cases}c'_N-c'_B B^{-1}N\geq 0\\
c_{n+1}-c'_B B^{-1} a_{n+1}\geq 0\end{cases}\]
La primera se verifica trivialmente por las hipótesis. La segunda es la condición que buscábamos.

\end{enumerate}

\subsection{Añadir una restricción}
Partimos de nuevo del problema $(P)$ $\min c'x$, $Ax=b$, $x\geq 0$. y le añadimos la restricción $a'_q x\leq b_q$ con $b_q \geq 0$ para formar el problema $(P_q)$. En tal caso, estandarizamos el problema añadiendo una holgura $x_q$ y nos queda:
\begin{align*}
\min\ & c'x\\
sa:\  & Ax=b\\
& a'_q x+h_q= b_q\\
 & x, b_q, h_q\geq 0
\end{align*}
Suponemos que $B$ es una base óptima en $(P)$. Queremos analizar la base formada por $B$, la ampliación de las columnas y la nueva holgura, es decir:
\[ B_q=\begin{bmatrix}
B & 0\\
(a_q)_B & 1
\end{bmatrix}.\]
La solución asociada a esta base es $x=
\begin{bmatrix}
B^{-1}b & b_q-(a_q)_B B^{-1}b & 0\end{bmatrix}'$. La inversa de nuestra base tiene la siguiente forma:
\[ B_q^{-1}=\begin{bmatrix}
B^{-1} & 0\\
-(a'_q)_B B^{-1} & 1
\end{bmatrix}.\]
Comprobamos las condiciones de optimalidad y factibilidad:
\begin{enumerate}
\item La optimalidad se tiene siempre a partir de la optimalidad de B en (P), pues
\[\overline{c}_N(P_q)=c'_N-(c'_B,0)B_q^{-1}\begin{bmatrix}
N\\
(a'_q)_N
\end{bmatrix}\geq 0 \equiv c'_N-[c'_B B_q^{-1} N + 0]=\overline{c}_N\geq 0\]
\item $B_q^{-1}\begin{bmatrix}
b\\
b_q
\end{bmatrix}\geq 0$. Esto es equivalente a $\begin{bmatrix}
B^{-1} & 0\\
-(a'_q)_B B^{-1} & 1
\end{bmatrix}\begin{bmatrix}
b\\
b_q
\end{bmatrix}=\begin{bmatrix}
B^{-1}b\\
-(a'_q)_B B^{-1}b+b_q
\end{bmatrix}\geq 0$. La primera desigualdad se tiene trivialmente, luego llegamos a la condición es $-(a'_q)_B B^{-1}b+b_q\geq 0$.
\end{enumerate}

\begin{example}
Añadimos al ejemplo anterior la restricción $x_1+x_2\leq 4$ y estandarizamos:
\[\begin{aligned}
\min & -2x_1-x_2 &\\
sa: & x_1+x_2 +x_3 &= 5\\
    & -x_1+x_2 +x_4 & =0\\
    & 6x_1+2x_2  +x_4 & = 21\\
    & x_1+x_2 +x_6  & = 4\\
    & x_1,x_2,x_3,x_4,x_5,x_6 & \geq 0
\end{aligned}\]
Se deja como ejercicio ver que no se verifica la condición obtenida.
\end{example}

\section{Precios Sombra}
Definimos 
\begin{align*}
f: \R^m& \longrightarrow \R\\
 f(b) = & \min\ c'x\\
  sa:\  & Ax=b\\
  & x\geq 0
\end{align*}
Queremos derivar esta función. Supongamos en este problema que las bases factibles son $B_1,\dots, B_r$. Consideramos el problema dual que verifica
\[\begin{aligned}
 f(b) = & \min\ c'x & & = \max u'b & = \max\{c'_{B_1}B_1^{-1}b,\dots, c'_{B_r}B_r^{-1}b\}\\
        & sa: Ax=b & & sa: u'A\leq c\\
        &  x\geq 0 & & u\in\R^m &
\end{aligned}\]
Llamamos $u_i=c'_{B_i}B_i^{-1}$. Entonces, $f(b)=\max_{1\leq i\leq r}u_i'b$. Digamos que este máximo se alcanza en una $u^*$, solución óptima dual.
Entonces $f(b)=(u^*)' b$. Derivando respecto al término independiente de la $i$-ésima restricción $b_i$:
\[\frac{\partial f}{\partial b_i}=\frac{\partial }{\partial b_i}\left(\sum_{j=1}^m u^*_jb_j\right)=u^*_i\]
Por lo tanto, $u^*_i$ es la tasa de crecimiento de la valor objetivo óptimo con una unidad de incremento en $b_i$.

Económicamente, llamamos $u^*$ vector de precios sombra. Para ilustrar, si la $i$-ésima restricción representa una demanda de producción de $b_i$ unidades del $i$-ésimo producto y $c'x$ representa el coste total de producción, $u^*_i$ es el coste incremental de producir una unidad más del $i$-ésimo producto.

\begin{example}
$g'=(0,0,1)=e'_3$. Teníamos que $z(\Delta)$ al variar $b_3\in [20,30]$ mantiene la base óptima.
\begin{gather*}
f(\Delta)= z(\Delta)=c'_B B^{-1}b(\Delta)=c'_B B^{-1}(b+\Delta e_3)=\\
c'_B B^{-1} b +\Delta c'_B B^{-1} e_3 =\frac{31}{4}-\frac{1}{4}\Delta
\end{gather*}
Luego $\frac{\partial z}{\partial\Delta}=-\frac{1}{4}$. Si quisiéramos utilizar el resultado anterior, tendríamos que obtener la solución óptima y multiplicar por $e_3$ (quedarnos con la tercera componente). La solución óptima del dual es
\[(u^*)'=c'_B B^{-1}=[-2\ -1\ 0]B^{-1}=[-\frac{1}{2}\ 0\ -\frac{1}{4}]\]
Por lo tanto. $\frac{\partial z}{\partial\Delta}=\frac{\partial z}{\partial b_3}=u^*_3=-\frac{1}{4}$.
\end{example}
\newpage
\section{Resumen}
Condiciones para el análisis de postóptimalidad
\begin{itemize}
\item En la función objetivo
\[
\max_{j:\overline{d}_j >0}-\frac{\overline{c}_j}{\overline{d}_j} \leq \Delta \leq \min_{j:\overline{d}_j <0}-\frac{\overline{c}_j}{\overline{d}_j}
\]
\item En el término independiente
\[\max_{i, g_i>0}\{\frac{-\overline{b}_i}{\overline{g}_i}\}\leq\Delta \leq \min_{i,g_i<0}\{\frac{-\overline{b}_i}{\overline{g}_i}\}\]
\item Añadir una nueva variable
\[c_{n+1}-c'_B B^{-1} a_{n+1}\geq 0.\]
\item Añadir una nueva restricción
\[-(a'_q)_B B^{-1}b+b_q\geq 0\]
\end{itemize}
%{\large\bf LOGIC INTERLUDE}
%\begin{theorem}
%Si $c \in Res(c_1,c_2)$ entonces $\{ c_1, c_2 \} \models c$.
%\end{theorem}

%\begin{dem}

%Sea $I \models \{ c_1, c_2\}$. Hay que demostrar que $I \models c$. Como $c \in Res(c_1,c_2)$, entonces existe $L \in c_1$ tal que $c = Res_L(c_1, c_2) = (c_1-\{ L \} \cup (c_2 - \{ L^c \})$.

%Consideramos dos casos:
%\begin{itemize}
%	\item Si $I(L) = 1$, entonces $I(L^C)=0$. Por otro lado, sabemos que $I$ es modelo de $c_2$, luego existe $L' \in C_2$ tal que $I(L')=1$. De $I(L^C)=0$ y $I(L')=1$, tenemos que $L^C \neq L'$. Luego $L' \in c_2 - \{ L ^C \}$. Luego $L' \in c$. Por lo tanto, $I \models c$.
%	\item $I(L) = 0$. Como $I \models c_1$, existe $L' \in c_1$ tal que $I(L') = 1$. Entonces $L' \neq L$, luego $L' \in c_1 - \{ L \}$. Entonces $L' \in c$ y $I \models c$. 
%\end{itemize}
%\end{dem}
\end{document}