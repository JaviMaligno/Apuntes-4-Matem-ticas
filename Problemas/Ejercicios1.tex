\documentclass{article}
\usepackage{amsmath,accents}%
\usepackage{amsfonts}%
\usepackage{amssymb}%
\usepackage{comment}
\usepackage{graphicx}
\usepackage{mathrsfs}
\usepackage[utf8]{inputenc}
\usepackage{amsfonts}
\usepackage{amssymb}
\usepackage{graphicx}
\usepackage{mathrsfs}
\usepackage{setspace}
\usepackage{amsthm}
\usepackage{nccmath}
\usepackage[spanish]{babel}
\usepackage{multirow}
\usepackage{tikz-cd}
\usepackage{pgf,tikz}
\usetikzlibrary{arrows}
\usetikzlibrary{cd}
\usetikzlibrary{babel}
\theoremstyle{plain}

\renewcommand{\baselinestretch}{1,4}
\setlength{\oddsidemargin}{0.5in}
\setlength{\evensidemargin}{0.5in}
\setlength{\textwidth}{5.4in}
\setlength{\topmargin}{-0.25in}
\setlength{\headheight}{0.5in}
\setlength{\headsep}{0.6in}
\setlength{\textheight}{8in}
\setlength{\footskip}{0.75in}

\newtheorem{theorem}{Teorema}[section]
\newtheorem{acknowledgement}{Acknowledgement}
\newtheorem{algorithm}{Algorithm}
\newtheorem{axiom}{Axiom}
\newtheorem{case}{Case}
\newtheorem{claim}{Claim}
\newtheorem{propi}[theorem]{Propiedades}
\newtheorem{condition}{Condition}
\newtheorem{conjecture}{Conjecture}
\newtheorem{coro}[theorem]{Corolario}
\newtheorem{criterion}{Criterion}
\newtheorem{defi}[theorem]{Definición}
\newtheorem{example}[theorem]{Ejemplo}
\newtheorem{exercise}{Ejercicio}
\newtheorem{lemma}[theorem]{Lema}
\newtheorem{nota}[theorem]{Nota}
\newtheorem{sol}{Solución}
\newtheorem*{sol*}{Solución}
\newtheorem{prop}[theorem]{Proposición}
\newtheorem{remark}{Remark}

\newtheorem{dem}[theorem]{Demostración}

\newtheorem{summary}{Summary}

\providecommand{\abs}[1]{\lvert#1\rvert}
\providecommand{\norm}[1]{\lVert#1\rVert}
\providecommand{\ninf}[1]{\norm{#1}_\infty}
\providecommand{\numn}[1]{\norm{#1}_1}
\providecommand{\gabs}[1]{\left|{#1}\right|}
\newcommand{\bor}[1]{\mathcal{B}(#1)}
\newcommand{\R}{\mathbb{R}}
\newcommand{\Q}{\mathbb{Q}}
\newcommand{\Z}{\mathbb{Z}}
\newcommand{\F}{\mathbb{F}}
\newcommand{\X}{\chi}
\providecommand{\Zn}[1]{\Z / \Z #1}
\newcommand{\resi}{\varepsilon_L}
\newcommand{\cee}{\mathbb{C}}
\providecommand{\conv}[1]{\overset{#1}{\longrightarrow}}
\providecommand{\gene}[1]{\langle{#1}\rangle}
\providecommand{\convcs}{\xrightarrow{CS}}
% xrightarrow{d}[d]
\setcounter{exercise}{0}
\newcommand{\cicl}{\mathcal{C}}

\newenvironment{ejercicio}[2][Estado]{\begin{trivlist}
\item[\hskip \labelsep {\bfseries Ejercicio}\hskip \labelsep {\bfseries #2.}]}{\end{trivlist}}
%--------------------------------------------------------
\begin{document}

\title{Relación 1 - Geometría y Topología de superficies }
\maketitle
\begin{exercise}
En un espacio $X$ definimos la relación $x \mathcal{R} y$ si existe un subespacio conexo por caminos $C \subseteq X$ con $x,y \in C$. ¿Cuáles son las clases de equivalencia?
\end{exercise}
\begin{sol*}
La clase de equivalencia de un elemento $x \in X$ será todo $y$ tal que existe $C \subseteq X$ con $x, y \in C$, luego:
\[ [x] = \bigcup_{\substack{C \subseteq X \\ x \in C}} C \]
que es el mayor subconjunto conexo por caminos de $X$ que contiene a $x$. Este conjunto, por definición, es la componente conexo por camino de $x$.
\end{sol*}

\newpage
\begin{exercise}
Probar que si $\mathcal{R}$ es una relación de equivalencia sobre un espacio $X$, las siguientes condiciones son equivalentes:
\begin{itemize}
	\item La aplicación $p : X \to X/\mathcal{R}$ es cerrada (abierta).
	\item Para todo cerrado (abierto) $A \subseteq X$, la unión de todas las clases de equivalencia que cortan a $A$ es cerrado (abierto) de $X$.
	\item Para todo abierto (cerrado) $A \subseteq X$, la unión de todas las clases de equivalencias contenidas en $A$ es abierto (cerrado) de $X$.
\end{itemize}
\end{exercise}

\newpage
\begin{exercise}
Sean $A$ y $B$ subconjuntos de $X$ tales que $A$ es cerrado y $A \subseteq B$. Probar que existe una inmersión de $B/A$ en $X/A$.
\end{exercise}
\begin{sol*}[con $B$ cerrado]
Vamos a dar explícitamente dicha inmersión. Esta es la aplicación $i_\pi:B/A\rightarrow X/A$ definida como $i_\pi([b])=[b]$. Está bien definida puesto que la imagen de dos elementos coincide si y solo sí están relacionados. Por ello es además inyectiva. Claramente es continua, ya que si $G$ es un abierto de $X/A$, entonces $i_\pi^{-1}(G)$ es de la forma $G\cap B/A$, que por definición de la topología del subespacio, es abierto. Obsérvese que en $i_\pi([b])$ no hay más elementos que en $[b]$ puesto que si $c\in[b]$, entonces o bien $c=b$ o bien $c,b\in A\subseteq B$. Esto quiere decir que $i_\pi(C)=C\ \forall C\subseteq B$, llevando $C$ la topología del subespacio. Veamos por último que es cerrada. Sea $W$ un cerrado de $B/A$. Entonces $\pi_B^{-1}(W)$ es cerrado por la definición de topología cociente, siendo $\pi_B$ la proyección canónica de $B$ en $B/A$. Observemos el siguiente diagrama conmutativo:
\[
\begin{tikzcd}
B\arrow[r,"\pi_B"]\arrow[d,"i"'] & B/A\arrow[r,"i_\pi"] & X/A\\
X\arrow[urr,"\pi"]
\end{tikzcd}
\]
Esto quiere decir que $i_\pi=\pi\circ i\circ \pi_B^{-1}$. Por lo anterior y la hipótesis añadida, $i\circ \pi_B^{-1}(W)$ es cerrado, luego resta probar que $\pi$ es cerrada. Esto se tiene por el ejercicio anterior, ya que todas las clases de equivalencia contenidas en $A$ forman un solo punto en el cociente, de modo que su unión es cerrada. 
\end{sol*}

\newpage
\begin{exercise}
Sea $X$ un espacio $T_2$ y $K \subseteq X$ compacto. Probar que $X/K$ es un espacio $T_2$.
\end{exercise}
\begin{sol*}
Consideramos dos puntos $x, y \in X$ con $x \neq y$. Supongamos que $y\in K$. Vamos a buscar una separación por entornos de $[x],[y] \in X/K$. Para cada $k \in K$, existe un entorno abierto $V_k$ de $k$ y un entorno abierto $G^k_x$ de $x$ con $G^k_x \cap V_k = \emptyset$. Por lo tanto, definimos:
\[ G_x = \bigcap_{k \in K} G^k_x \]
\[ V = \bigcup_{k \in K} V_k \]
de manera que $G_x \cap V = \emptyset$. Sea $p : X \to X/K$ la proyección a $X/K$. Como $p(G_x)$ y $p(V)$ son entornos abiertos de $[x]$ e $[y]$ respectivamente y $p(G_x) \cap p(V) = \emptyset$, $[x]$ e $[y]$ están separados. Si $x,y \notin K$, basta tomar $H_x, H_y \subset X$ que separen $x$ e $y$, usando que $X$ es $T_2$. Entonces $p(H_x \cap G_x) \cap p(H_y \cap G_y) = \emptyset$. Luego $[x]$ e $[y]$ están separados y, por lo tanto, $X/K$ es espacio $T_2$.
\end{sol*}


\newpage
\begin{exercise}
Indíquense cuáles de los siguientes espacios son homeomorfos entre sí:
\begin{itemize}
	\item $\mathbb{R}/\mathbb{Z}$;
	\item el subespacio $K \subseteq \mathbb{R}^2$ formado por la unión de las cicunferencias $C_k$ con centro $(\pm \frac{1}{k},0)$ y radio $\frac{1}{k}$ con $k \geq 1$;
	\item $[-1,1]/A$ con $A = \{ \pm \frac{1}{k} : k \geq 1\}$
\end{itemize}
\end{exercise}


\newpage
\begin{exercise}
Probar que $\mathbb{R}/\mathbb{Q}$ no es $T_2$ y por tanto no puede ser metrizable.
\end{exercise}
\begin{sol*}
Razonamos por reducción al absurdo. Supongamos que $\mathbb{R}/\mathbb{Q}$ es $T_2$. Entonces, dado $x\in\R$ existen $U,V$ abiertos de $\mathbb{R}/\mathbb{Q}$ tales que $[x]\in U, [\Q]\in V, V\cap U=\emptyset$. Consideremos la proyección canónica $\pi:\R\rightarrow\mathbb{R}/\mathbb{Q}$. Se cumple entonces
\[\pi^{-1}(U\cap V)=\pi^{-1}(U)\cap\pi^{-1}(V)=\emptyset\]
Pero $\Q\subseteq \pi^{-1}(V)$, así que hemos llegado a una contradicción con que $\Q$ es denso en $\R$. 
\end{sol*}


\newpage
\begin{exercise}
Indíquense subespacios de $\mathbb{R}^3$ homeomorfos a los siguientes espacios cocientes:
\begin{itemize}
	\item $\mathbb{R}^2/D^2$, donde $D^2$ es la bola unidad cerrada;
	\item $\mathbb{R}^2/S^1$;
	\item $\mathbb{R}^2/(\mathbb{R}^2-B^2)$, donde $B^2$ es la bola abierta unidad.
\end{itemize}
\end{exercise}


\newpage
\begin{exercise}
Sean $\mathcal{R}_1$ y $\mathcal{R}_2$ dos relaciones de equivalencia sobre $X$. Si $x \mathcal{R}_1 y$ impica que $x \mathcal{R}_2 y$, probar que si $\mathcal{R}$ es la relación en $X/\mathcal{R}_1$ dada por $[x] \mathcal{R} [y]$ si $x \mathcal{R}_2 y$, entonces $(X/\mathcal{R}_1)/\mathcal{R}$ es homeomorfo a $X / \mathcal{R}_2$.
\end{exercise}


\newpage
\begin{exercise}
Sea $\mathcal{R}$ una relación de equivalencia sobre $X$ tal que para todo $x \in X$, $[x]$ está contenida en alguna compomenten conexa de $X$. Probar que las componentes conexa de $X/\mathcal{R}$ son las imágenes por la proyección canónica de las componentes conexas de $X$.
\end{exercise}


\newpage
\begin{exercise} Sea $f : B^n \to S^n$ la función continua dada por:
\[ f(x_1,\dots,x_n) = (2 \sqrt{1-\norm{x}^2}x_1, \dots, 2 \sqrt{1-\norm{x}^2}x_n, 2 \norm{x}^2-1). \]
Sea $\mathcal{R}_f$ la relación de equivalencia genera por $x \mathcal{R}_f y$ si $f(x) = f(y)$. Probar que $B ^n/\mathcal{R}_f$ es $B^n/S^{n-1}$ y deducir que este espacio cociente es homeomorfo a $S^n$.
\end{exercise}


\newpage
\begin{exercise}
En $\mathbb{R}^2$ se define la relación $(x,y) \mathcal{R} (u,v)$ si $y+x^2 = v+u^2$. ¿A qué subespacio euclídeo es homeomorfo $\mathbb{R}^2/\mathcal{R}$?
\end{exercise}


\newpage
\begin{exercise}
Sea $\mathcal{R}$ una relación de equivalencia sobre $(X, \mathcal{T})$. Probar que si $X/\mathcal{R}$ es $T_2$, entonces $\mathcal{R}$ es cerrado en $X \times X$.
\end{exercise}


\newpage
\begin{exercise}
Sea $X = \{x \in \mathbb{R} : x \geq 0\}$. Se considera sobre $X$ la relación de equivalencia $\mathcal{R}$ definida por la partición $\left\{\left\{a, \frac{1}{a}\right\} \right\}_{a > 0} \cup \{0\}$. Demostrar que $p : X \to X/\mathcal{R}$ es abierta, pero que $\mathcal{R}$ no es abierto en $X \times X$.
\end{exercise}


\newpage
\begin{exercise}
Sea $\mathcal{R}$ una relación de equivalencia sobre $(X, \mathcal{T})$ tal que $p : X \to X/\mathcal{R}$ es abierta. Entonces $X/\mathcal{R}$ es $T_2$ si y sólo si $\mathcal{R}$ es cerrado en $X \times X$. Deducir de este resultado que el espacio proyectivo $P_2(\mathbb{R})$ es una espacio $T_2$.
\end{exercise}


\newpage
\begin{exercise}
Probar que para todo espacio topológico $X$ el cono $C X$ y la suspensión $\Sigma X$ son conexos por caminos.
\end{exercise}


\newpage
\begin{exercise}
\begin{itemize}
	\item Probar que $C S^{n-1}$ es homeomorfo a $B^n$, con $n \geq 1$ ($B^n$ es la bola cerrada con centro el origen y radio $1$).
	\item Dado $X \subseteq \mathbb{R}^{n-1} \times \{0\} \subseteq \mathbb{R}^n$ y $v \in \mathbb{R}^n-\mathbb{R}^{n-1} \times \{0\}$ se considera el conjunto $vX = \{ tv + (1-t)x : x \in X, t \in [0,1]\}$. Probar que si $X$ es compacto, entonces $C X$ es homeomorfo a $vX$.
	\item Considérese $X = \mathbb{Z} \subseteq \mathbb{R}$ y demuéstrese que $C X$ no tiene en el vértice una base numerable de abiertos. Concluir que el resultado del apartado anterior no es cierto en general.
\end{itemize}
\end{exercise}


\newpage
\begin{exercise}\mbox{}
\begin{itemize}
	\item Probar que para todo espacio $X$ existe un inmersión cerrada de $X$ en $C X$.
	\item Probar que la aplicación $\phi : X \to \Sigma X$ dada por $\phi(x) = [x, \frac{1}{2}]$ es una inmersión cerrada.
	\item Probar que existen inmersiones cerradas $\phi_1, \phi_2 : C X \to \Sigma X$ cuyas imágenes son los subespacios $C_1 X = \{[x,t] \in \Sigma X : -1 \leq t \leq 0\}$, $C_2 X = \{[x,t] \in \Sigma X : 0 \leq t \leq 1\}$
\end{itemize}
\end{exercise}


\newpage
\begin{exercise}
Probar que $\Sigma S^{n-1}$ es homeomorfo a $S^n$, $n \geq 1$. \textbf{Indicación:} Considérese la aplicación $f : S^{n-1} \times [0,1] \to S^n$ dada por:
\[ f(x,t) = (\sqrt{1-(2t-1)^2}x, 2t-1) \]
\end{exercise}


\newpage
\begin{exercise}
Pruébese que $S^2$ es homeomorfo a ($S^1 \times S^1) / (S^1 \lor S^1)$.
\end{exercise}


\newpage
\begin{exercise}
Probar que si $X$ e $Y$ son conexos por caminos o compactos, $X \cup_f Y$ también lo es.
\end{exercise}


\newpage
\begin{exercise}
Sean $X$, $Y$ y $W$ espacios topológicos, $A \subseteq X$ cerrado y $f : A \to Y$, $g : Y \to W$ aplicaciones continuas. Probar que los espacios $(X \cup_f Y) \cup_g W$ y $X \cup _{g \circ f} W$ son homeomorfos.
\end{exercise}
\end{document}
