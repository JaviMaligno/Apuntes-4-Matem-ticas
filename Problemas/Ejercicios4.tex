\documentclass{article}
\usepackage{amsmath,accents}%
\usepackage{amsfonts}%
\usepackage{amssymb}%
\usepackage{comment}
\usepackage{graphicx}
\usepackage{mathrsfs}
\usepackage[utf8]{inputenc}
\usepackage{amsfonts}
\usepackage{amssymb}
\usepackage{graphicx}
\usepackage{mathrsfs}
\usepackage{setspace}
\usepackage{amsthm}
\usepackage{nccmath}
\usepackage[spanish]{babel}
\usepackage{multirow}
\usepackage{hyperref}
\usepackage{tikz-cd}
\usepackage{pgf,tikz}
\usetikzlibrary{arrows}
\usetikzlibrary{cd}
\usetikzlibrary{babel}
\theoremstyle{plain}
\hypersetup{colorlinks=true,citecolor=red, linkcolor=blue}

\renewcommand{\baselinestretch}{1,4}
\setlength{\oddsidemargin}{0.5in}
\setlength{\evensidemargin}{0.5in}
\setlength{\textwidth}{5.4in}
\setlength{\topmargin}{-0.25in}
\setlength{\headheight}{0.5in}
\setlength{\headsep}{0.6in}
\setlength{\textheight}{8in}
\setlength{\footskip}{0.75in}

\theoremstyle{definition}

\newtheorem{theorem}{Teorema}[section]
\newtheorem{acknowledgement}{Acknowledgement}
\newtheorem{algorithm}{Algorithm}
\newtheorem{axiom}{Axiom}
\newtheorem{case}{Case}
\newtheorem{claim}{Claim}
\newtheorem{propi}[theorem]{Propiedades}
\newtheorem{condition}{Condition}
\newtheorem{conjecture}{Conjecture}
\newtheorem{coro}[theorem]{Corolario}
\newtheorem{criterion}{Criterion}
\newtheorem{defi}[theorem]{Definición}
\newtheorem{example}[theorem]{Ejemplo}
\newtheorem{exercise}{Ejercicio}
\newtheorem{lemma}[theorem]{Lema}
\newtheorem{nota}[theorem]{Nota}
\newtheorem{sol}{Solución}
\newtheorem*{sol*}{Solución}
\newtheorem{prop}[theorem]{Proposición}
\newtheorem{remark}{Remark}

\newtheorem{dem}[theorem]{Demostración}

\newtheorem{summary}{Summary}

\providecommand{\abs}[1]{\lvert#1\rvert}
\providecommand{\norm}[1]{\lVert#1\rVert}
\providecommand{\ninf}[1]{\norm{#1}_\infty}
\providecommand{\numn}[1]{\norm{#1}_1}
\providecommand{\gabs}[1]{\left|{#1}\right|}
\newcommand{\bor}[1]{\mathcal{B}(#1)}
\newcommand{\R}{\mathbb{R}}
\newcommand{\Q}{\mathbb{Q}}
\newcommand{\Z}{\mathbb{Z}}
\newcommand{\F}{\mathbb{F}}
\newcommand{\X}{\chi}
\providecommand{\Zn}[1]{\Z / \Z #1}
\newcommand{\resi}{\varepsilon_L}
\newcommand{\cee}{\mathbb{C}}
\providecommand{\conv}[1]{\overset{#1}{\longrightarrow}}
\providecommand{\gene}[1]{\langle{#1}\rangle}
\providecommand{\convcs}{\xrightarrow{CS}}
% xrightarrow{d}[d]
\setcounter{exercise}{0}
\newcommand{\cicl}{\mathcal{C}}

\newenvironment{ejercicio}[2][Estado]{\begin{trivlist}
\item[\hskip \labelsep {\bfseries Ejercicio}\hskip \labelsep {\bfseries #2.}]}{\end{trivlist}}
%--------------------------------------------------------
\begin{document}
\title{Relación 4 - Geometría y Topología de superficies }
\author{Javi, Rafa, Diego}
\maketitle
\begin{exercise} Probar que la yuxtaposición de caminos satisface la siguiente propiedad de cancelación: si $f_0*g_0 \sim f_1* g_1$ y $g_0 \sim g_1$, entonces $f_0 \sim f_1$.
\end{exercise}
\begin{sol*}
Como $g_0 \sim g_1\Leftrightarrow \overline{g_0}\sim\overline{g_1}$, usamos la compatibilidad de la yuxtaposición con la equivalencia de caminos.
\[
f_0*g_0 \sim f_1* g_1\Leftrightarrow f_0*g_0*\overline{g_0}\sim f_1*g_1*\overline{g_1}\Leftrightarrow f_0\sim f_1.
\]
\end{sol*}

\vspace{0.1cm}

\newpage \begin{exercise} \label{s1}Probar que todo lazo $\alpha$ en $x_0\in X$ define una aplicación continua $f_\alpha: S^1 \to X$ con $f_\alpha((1,0)) = x_0$ y recíprocamente. Más aún, si $\alpha \sim \alpha'$ entonces $f_{\alpha} \simeq f_{\alpha'} \mbox{ rel. } (0,1)$.
\end{exercise}
\begin{sol*}
Observemos que $S^1$ se puede considerar como $[0,1]/\sim$ con la relación generada por $0\sim 1$. Por tanto, si denotamos por $\pi$ a la aplicación cociente e $I=[0,1]$ obtenemos el siguiente diagrama conmutativo:
\[
\begin{tikzcd}
I\arrow[r,"\alpha"]\arrow[d,"\pi"'] & X\\
S^1\arrow[ur,"\alpha\circ\pi^{-1}"']
\end{tikzcd}
\]
Por lo que $f_\alpha=\alpha\circ\pi^{-1}$. Por otro lado, como $\alpha\sim\alpha'$ implica que la homotopía es relativa al $\{0,1\}$, en el cociente esto implica que es relativa a $[0]=[1]=(0,1)$. 
\end{sol*}

\vspace{0.1cm}

\newpage \begin{exercise} Probar que para un espacio conexo por caminos $X$ las tres condiciones siguientes son equivalentes:
(a) Toda aplicación continua $f: S^1\to X$ es homotópica a una constante; (b) toda aplicación continua $f: S ^1 \to X$ se extiende al disco $D^2$;
(c) $\pi_1(X,x_0) = 0$ para todo $x_0\in X$.
\end{exercise}
\begin{sol*}
$\boxed{(a)\Rightarrow(b)}$ Por hipótesis existe una homotopía $H:S^1\times I\to X$ cumpliendo $H(x,0)=f(x)$ y $H(x,1)=x_0\in X$. Podemos pensar en esta homotopía como una aplicación del cilindro a $X$ en la que toda la tapa superior tiene la misma imagen. Por tanto podemos factorizar a través del cono, que sabemos que es homeomorfo mediante un homemomorfismo que llamaremos $g$ a $D^2$.
\[
\begin{tikzcd}
S^\times I\arrow[r,"H"]\arrow[d,"\pi"']& X\\
CS^1\arrow[ur,"\tilde{\pi}"]\arrow[d,"g"',"\cong"]\\
D^2\arrow[uur,"\tilde{\pi}\circ g^{-1}"']
\end{tikzcd}
\]
Por tanto $f$ se extiende mediante $\tilde{\pi}\circ g^{-1}$.\\
$\boxed{(b)\Rightarrow(c)}$ Sea $i:S^1\hookrightarrow D^2$ la inclusión y volvamos a considerar el homeomorfismo $g$ del apartado anterior. Supongamos que $f$ se extiende mediante una función $h$. Entonces obtenemos el siguiente diagrama conmutativo:
\[
\begin{tikzcd}
CS^1\arrow[ddr,"g^{-1}\circ h"]\\
D^2\arrow[u,"g"]\arrow[dr,"h"]\\
S^1\arrow[u,"i",hookrightarrow]\arrow[r,"f"]& X
\end{tikzcd}
\]
Dado un lazo, podemos extenderlo a $D^2$ y llevarlo al cono, contrayéndolo sobre el vértice. Por lo tanto el lazo es homotópico al constante. Como $X$ es conexo por caminos, esta construcción no depende del punto base y por tanto $\pi_1(X,x_0)=0$.\\
$\boxed{(c)\Rightarrow(a)}$ Se tiene que $f:S^1\to X$ identifica a un lazo en $X$, y como $\pi_1(X,x_0)=0$ concluimos que $f\sim x_0$.
\end{sol*}
\newpage 

\begin{exercise}
Probar que $\pi_1(X,x_0) = 0$ si y sólo para todo $x\in X$ dos caminos cualesquiera entre $x_0$ y $x$ son equivalentes.
\end{exercise}

\begin{sol*}
$\boxed{\Rightarrow}$ Sea $\gamma_1$ y $\gamma_2$ dos caminos entre $x_0$ y $x$. Entonces $\gamma_1*\overline{\gamma_2}$ es un lazo en $x_0$. Como $\pi_1(X,x_0)=0$ se tiene
\[
\gamma_1*\overline{\gamma_2}\sim c_{x_0}\Leftrightarrow \gamma_1\sim\gamma_2,
\]
como queríamos demostrar.\\
$\boxed{\Leftarrow}$ Tomando $x=x_0$ tenemos que todos los lazos son equivalentes. En particular son equivalentes al lazo constante $c_{x_0}$, por lo que se deduce que $\pi_1(X,x_0)=0$.
\end{sol*}


\newpage \begin{exercise} Sea $A\subseteq X$ la componente conexa por caminos del punto $x_0\in X$. Probar que la inclusión
$i: A\to X$ induce un isomorfismo $i_*: \pi_1(A,x_0) \to \pi_1(X,x_0)$.
\end{exercise}

\newpage 

\begin{exercise}
Probar que si $X = A\cup B$ con $A$ y $B$ abiertos tales que $A$, $B$ y $A\cap B$ son conexos por caminos
 y $A$ y $B$ simplemente conexos entonces  $X$ es simplemente conexo. Aplicar este resultado para deducir
 que $\pi_1(S^n) = 0$ para todo $n \geq 2$.
\end{exercise}
\newpage 
\begin{exercise}
Dar una familia infinita de espacios conexos por caminos simplemente conexos que no sean homeomorfos.
\end{exercise}
\begin{sol*}
La familia de hiperesferas $\{S^n\}_{n\geq 2}$, que son simplemente conexas como se prueba en el apartado anterior y por el teorema de invariancia del dominio no son homeomorfas. 
\end{sol*}

\vspace{0.1cm}

\newpage \begin{exercise} Un espacio conexo por caminos $X$ se dice que es {\it $1$-simple} si
dados $x_0, x_1\in X$, dos caminos de $X$ cualesquiera entre $x_0$ y $x_1$ inducen el mismo
isomorfismo entre $\pi_1(X,x_1)$ y $\pi_1(X,x_0)$.
Probar que $X$ es 1-simple si y sólo si $\pi_1(X)$ es abeliano.
\end{exercise}

\newpage 
\begin{exercise}
Sea $[S^1,X]$ el conjunto de las clases de homotopía (no necesariamente relativa) de aplicaciones de $S^1$ en $X$. De acuerdo con el problema \ref{s1} hay definida una aplicación $\varphi: \pi_1(X,x_0) \to [S^1,X]$, $[\alpha] \mapsto [f_{\alpha}]$. Probar que si $X$ es conexo por caminos se tiene que $\varphi$ es sobreyectiva. Más aún, si $\varphi ([\alpha]) = \varphi([\beta])$, $[\alpha]$ y $[\beta]$ son elementos conjugados en $\pi_1(X,x_0)$.
Como consecuencia, si $X$ es conexo por caminos existe una biyección entre $[S^1,X]$ y las clases por conjugación de $\pi_1(X,x_0)$.
\end{exercise}





\newpage \begin{exercise} Consideremos el espacio $X = \{a,b,c,d\}$ con la topología $\mathcal{T} = \{\emptyset, X$, $\{c\},\{d\}, \{c,d\}, \{a,c,d\}, \{b,c,d\}\}$.
\par
 Veamos que $\pi_1(X,d)$ no es el grupo trivial. Para ello, sea $\alpha: [0,1]\to X$ la aplicación $\alpha(t) = d$ si $t \in [0,\frac{1}{4}) \cup (\frac{3}{4},1]$, $\alpha(t) = c$ si $t\in (\frac{1}{4},\frac{3}{4})$, $f(\frac{1}{4}) = a$ y $\alpha(\frac{3}{4}) = b$.  Probar que $\alpha$ es un lazo en $d$ que no es equivalente al lazo constante. Para esto último, suponer lo contrario y si $H$ es una homotopía, probar que $t_0 = \mbox{sup }\{t; c\in H(I\times \{t\})\}$ es un máximo y llegar a la contradicción de que debe ser $t_0 = 1$.
\end{exercise}
\begin{sol*}
Suponemos que $\alpha$ es nul-homotópica y consideramos la homotopía $H : I \times I \to X$  de $\alpha$ a la función constante $e(t)=d$ para $t \in I$.
Vamos a demostrar que el conjunto $G = \{t ; c \in H(I \times \{t\}\}$ es un abierto cerrado. Para ver que es abierto, observamos que $\{(i,t) ; c \in H(i,t)\} = H^{-1}({c})$. Como $H$ es continua y $\{c\}$ es un abierto, $H^{-1}({c})$ es también abierto. Como $G$ es la proyección de $H^{-1}(c)$ sobre la segunda coordenada, $G$ es abierto. Por otro lado, existe $t_0$, ya que $G \subseteq I$. Aunque $G$ debe ser una unión de intervalos, podemos suponer sin pérdida de generalidad que es un sólo intervalo. Sea $\{(i_n,t_n)\}\subset I\times G$ una sucesión tal que $H(i_n,t_n) = c$ y $t_n \to t_0$. Por continuidad (tomando si es necesario una subsucesión de manera que $i_n$ converga en $i_0 \in I$):
\[ H(i_0,t_0) = \lim\limits_{(i_n,t_n)\to (\infty,\infty)}H(i_n,t_n) = c \]
Por lo tanto $t_0 \in G$. Entonces $G$ es un abierto cerrado. Basta ver, que por inspección, $X$ es conexo (no se puede expresar como la unión disjunta de dos abiertos de $\mathcal{T}$). Además $G\neq\emptyset$ pues $0\in G$. Por lo tanto $G = I$ y $t_0 = 1$. Hemos llegado entonces a una contradicción, pues $H(I,1) \neq \{d\}$. Luego $\alpha$ no es nul-homotópica y $\pi_1(X,d)$ no es trivial.
\end{sol*}

\newpage \begin{exercise} Sea $H: X\times I \to X$ una homotopía con $H(x,0) =  H(x,1) = x$ para todo $x\in X$. Si $\gamma$ es el lazo en $x_0$ $\gamma(t) = H(x_0,t)$, probar que la clase $[\gamma]$ conmuta con todo elemento de $\pi_1(X,x_0)$.
\end{exercise}
\begin{sol*} Véase que $[\gamma]=[Id_X]$, ya que la homotopía.
\end{sol*}
\end{document}
