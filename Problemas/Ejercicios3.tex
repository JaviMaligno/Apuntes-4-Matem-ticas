\documentclass{article}
\usepackage{amsmath,accents}%
\usepackage{amsfonts}%
\usepackage{amssymb}%
\usepackage{comment}
\usepackage{graphicx}
\usepackage{mathrsfs}
\usepackage[utf8]{inputenc}
\usepackage{amsfonts}
\usepackage{amssymb}
\usepackage{graphicx}
\usepackage{mathrsfs}
\usepackage{setspace}
\usepackage{amsthm}
\usepackage{nccmath}
\usepackage[spanish]{babel}
\usepackage{multirow}
\usepackage{hyperref}
\usepackage{tikz-cd}
\usepackage{pgf,tikz}
\usetikzlibrary{arrows}
\usetikzlibrary{cd}
\usetikzlibrary{babel}
\theoremstyle{plain}

\renewcommand{\baselinestretch}{1,4}
\setlength{\oddsidemargin}{0.5in}
\setlength{\evensidemargin}{0.5in}
\setlength{\textwidth}{5.4in}
\setlength{\topmargin}{-0.25in}
\setlength{\headheight}{0.5in}
\setlength{\headsep}{0.6in}
\setlength{\textheight}{8in}
\setlength{\footskip}{0.75in}

\theoremstyle{definition}

\newtheorem{theorem}{Teorema}[section]
\newtheorem{acknowledgement}{Acknowledgement}
\newtheorem{algorithm}{Algorithm}
\newtheorem{axiom}{Axiom}
\newtheorem{case}{Case}
\newtheorem{claim}{Claim}
\newtheorem{propi}[theorem]{Propiedades}
\newtheorem{condition}{Condition}
\newtheorem{conjecture}{Conjecture}
\newtheorem{coro}[theorem]{Corolario}
\newtheorem{criterion}{Criterion}
\newtheorem{defi}[theorem]{Definición}
\newtheorem{example}[theorem]{Ejemplo}
\newtheorem{exercise}{Ejercicio}
\newtheorem{lemma}[theorem]{Lema}
\newtheorem{nota}[theorem]{Nota}
\newtheorem{sol}{Solución}
\newtheorem*{sol*}{Solución}
\newtheorem{prop}[theorem]{Proposición}
\newtheorem{remark}{Remark}

\newtheorem{dem}[theorem]{Demostración}

\newtheorem{summary}{Summary}

\providecommand{\abs}[1]{\lvert#1\rvert}
\providecommand{\norm}[1]{\lVert#1\rVert}
\providecommand{\ninf}[1]{\norm{#1}_\infty}
\providecommand{\numn}[1]{\norm{#1}_1}
\providecommand{\gabs}[1]{\left|{#1}\right|}
\newcommand{\bor}[1]{\mathcal{B}(#1)}
\newcommand{\R}{\mathbb{R}}
\newcommand{\Q}{\mathbb{Q}}
\newcommand{\Z}{\mathbb{Z}}
\newcommand{\F}{\mathbb{F}}
\newcommand{\X}{\chi}
\providecommand{\Zn}[1]{\Z / \Z #1}
\newcommand{\resi}{\varepsilon_L}
\newcommand{\cee}{\mathbb{C}}
\providecommand{\conv}[1]{\overset{#1}{\longrightarrow}}
\providecommand{\gene}[1]{\langle{#1}\rangle}
\providecommand{\convcs}{\xrightarrow{CS}}
% xrightarrow{d}[d]
\setcounter{exercise}{0}
\newcommand{\cicl}{\mathcal{C}}

\newenvironment{ejercicio}[2][Estado]{\begin{trivlist}
\item[\hskip \labelsep {\bfseries Ejercicio}\hskip \labelsep {\bfseries #2.}]}{\end{trivlist}}
%--------------------------------------------------------
\begin{document}
\title{Relación 3 - Geometría y Topología de superficies }
\author{Javi, Rafa, Diego}
\maketitle
\begin{exercise}
Probar que si $f:X\to S^n$ no es epiyectiva entonces es homotópica a la aplicación constante.
\end{exercise}
\newpage
\begin{exercise}
Probar que todo retracto de un espacio métrico es un subespacio cerrado.
\end{exercise}
\newpage
\begin{exercise}
Probar que $A$ es retracto de $X$ si y sólo si toda función contninua de $A$ es un espacio arbitrario $Z$ admite una extensión continua sobre $X$.
\end{exercise}
\newpage
\begin{exercise}
Probar que si $A$ y $B$ son retractos de deformación de $X$ e $Y$ respectivamente, entonces $A\times B$ es retracto de deformación de $X\times Y$.
\end{exercise}
\newpage
\begin{exercise}
Probar que si $A$ es retracto de deformación de $X$ y $B$ lo es de $A$, entonces $B$ lo es de $X$. 
\end{exercise}
\newpage
\begin{exercise}Dar un ejemplo de un espacio $X$ y de dos subconjuntos $A$ y $B$ homeomorfos tales que A sea retracto de deformación de $X$ y $B$ no lo sea.
\end{exercise}
\newpage
\begin{exercise}
Probar que $S^{n-1}$ es retracto de deformación de $R^n-\{0\}$. Encontrar retractos de deformación (compactos) de los siguientes espacios.
\begin{enumerate}
\item $\R^2-\{p_1,p_2,\dotsc,p_k\}$,
\item $S^1\times S^1-\{*\}$ (el toro menos un punto),
\item $\R^3-OZ$, $\R^3-S^1\lor S^1$,
\item la banda de Möbius y la banda de Möbius menos un punto,
\item $\R^4-(OZ\bigcup S^1)$,
\item el complementario de un disco en el plano proyectivo y
\item el complementario en $\R^3$ de dos rectas paralelas.
\end{enumerate}
\end{exercise}
\newpage
\begin{exercise}
Probar que si X es contractil e Y arcoconexo, dos aplicaciones cualesquiera de X en Y son homotópicas.
\end{exercise}
\newpage
\begin{exercise}
Indicar cuántos elementos tiene el conjunto de las clases de homotopía de aplicaciones continuas de $(S^1,\{1\}$ en $(X,*)$ rel. $\{1\}$, en los casos siguientes:
\begin{enumerate}
\item X tiene la topología discreta.
\item X tiene la topología indiscreta.
\end{enumerate}
\end{exercise}
\newpage
\begin{exercise}
Sea $F:X\to Y$ una aplicación continua. Sobre la unión disjunta $(X\times I)\sqcup Y$ se considera la relación de equivalencia resultante de identificar $(x,1)$ con $f(x)$. El espacio cociente obtenido se llama cilindro de la aplicación $f$, y se denotará por $M_f$. Probar que Y es retracto de deformación de $M_f$.
\end{exercise}
\newpage
\begin{exercise}
Probar que toda aplicación continua $f:Y_0\to Y_1$ da lugar a una aplicación $f_*:[X,Y_0]\to[X,Y_1]$ con las siguientes propiedades.
\begin{itemize}
\item Si $f':Y_0\to Y_1$ es homotópica a $f$ entonces $f_* =f'_*$.
\item Para $id:Y\to Y$ se tiene que $id_*$ es la identidad.
\item Si $g:Y_1\to Y_2$ es continua, entonces $(gf)_* = g_*f_*$.
\item Si $f$ es una equivalencia de homotopía, entonces $f_*$ es biyectiva.
\end{itemize}
\end{exercise}
\newpage
\begin{exercise}
\begin{enumerate}
\item[]
\item Si $F:Y_0\to Y_1$ es tal que $f_*$ es biyectiva para todo X, entonces $f$ es una equivalencia de homotopía.
\item Si $g:X_0\to X_1$ es tal que $g^*$ es la biyectiva para todo Y, entonces g es una equivalencia de homotopía.
\end{enumerate}
\end{exercise}
\newpage
\begin{exercise}
Probar qyue las siguientes aplicaciones son equivalencias de homotopía:
\begin{enumerate}
\item Toda aplicación continua que sea homotópica a una equivalencia de homotopía.
\item Toda aplicación continua entre espacios contráctiles.
\end{enumerate}
\end{exercise}
\newpage
\begin{exercise}
Sean $f_1,g_1:X_1\to Y_1$ y $f_2,g_2:X_2\to Y_2$ aplicaciones tales que $f_1\sim g_1$ y $f_2\sim g_2$. Probar que $f_1\times f_2\sim g_1\times g_2$. Como consecuencia, si $f_1$ y $f_2$ son equivalencias de homotopía, también lo es $f_1\times f_2$. Así pues, $X\times Y$ es contráctil si y solo sí $X$ e $Y$ lo son.
\end{exercise}
\newpage
\begin{exercise}
Se dice que el par $(X,A)$ tiene la Propiedad de Extensión de Homotpía (PEH) si para toda aplicación continua $f:X\to Y$ es un espacio arbitrario Y y toda homotopía $F:A\times I \to Y$ de aplicación $f_{|_A}$, existe homotopía $G:X\times I \to Y$ que extiende a $F\cup f$. Probar que si A es cerrado, $(X,A)$ tiene la PEH si y sólo si $(X\times 0)\bigcup A\times I$ es retracto de $X\times I$.
\end{exercise}
\newpage
\begin{exercise}
Sea $A\subset X$ cerrado y $*$ un punto de A considerado como punto base de X. Supongamos que $(X,A)$ tiene la PEH y que $\{*\}$ es un retracto de deformación fuerte de A. Entonces $p:(X,*)\to(X/A,[A])$ es una equivalencia de homotopía.
\end{exercise}
\newpage
\begin{exercise}
Sea $\Pi_0(X)$ el conjunto de las componentes conexas por caminos de X. Dada una aplicación $f:X\to Y$ sea $f_*:\Pi_0(X)\to\Pi_0(Y)$ la aplicación que lleva de componentes $C_x$ en la componente $C_{f(x)}$. Probar las siguientes propiedades:
\begin{enumerate}
\item $(g\circ f)_* = g_*\circ f_*$.
\item Si $f$ es homotópica a g entonces $f_*=g_*$.
\item Si $f$ es una equivalencai de homotopía entonces $f_*$ es una biyección. Además $C_x$ es del mismo tipo de homotopía que $C_{f(x)}$.
\end{enumerate}
\end{exercise}
\newpage
\begin{exercise}
Probar que un retracto de un espacio contráctil es contráctil.
\end{exercise}
\newpage
\end{document}
