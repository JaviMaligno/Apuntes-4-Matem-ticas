\documentclass{article}
\usepackage{amsmath,accents}%
\usepackage{amsfonts}%
\usepackage{amssymb}%
\usepackage{comment}
\usepackage{graphicx}
\usepackage{mathrsfs}
\usepackage[utf8]{inputenc}
\usepackage{amsfonts}
\usepackage{amssymb}
\usepackage{graphicx}
\usepackage{mathrsfs}
\usepackage{setspace}
\usepackage{amsthm}
\usepackage{nccmath}
\usepackage[spanish]{babel}
\usepackage{multirow}
\usepackage{hyperref}
\usepackage{tikz-cd}
\usepackage{pgf,tikz}
\usetikzlibrary{arrows}
\usetikzlibrary{cd}
\usetikzlibrary{babel}
\theoremstyle{plain}

\renewcommand{\baselinestretch}{1,4}
\setlength{\oddsidemargin}{0.5in}
\setlength{\evensidemargin}{0.5in}
\setlength{\textwidth}{5.4in}
\setlength{\topmargin}{-0.25in}
\setlength{\headheight}{0.5in}
\setlength{\headsep}{0.6in}
\setlength{\textheight}{8in}
\setlength{\footskip}{0.75in}

\theoremstyle{definition}

\newtheorem{theorem}{Teorema}[section]
\newtheorem{acknowledgement}{Acknowledgement}
\newtheorem{algorithm}{Algorithm}
\newtheorem{axiom}{Axiom}
\newtheorem{case}{Case}
\newtheorem{claim}{Claim}
\newtheorem{propi}[theorem]{Propiedades}
\newtheorem{condition}{Condition}
\newtheorem{conjecture}{Conjecture}
\newtheorem{coro}[theorem]{Corolario}
\newtheorem{criterion}{Criterion}
\newtheorem{defi}[theorem]{Definición}
\newtheorem{example}[theorem]{Ejemplo}
\newtheorem{exercise}{Ejercicio}
\newtheorem{lemma}[theorem]{Lema}
\newtheorem{nota}[theorem]{Nota}
\newtheorem{sol}{Solución}
\newtheorem*{sol*}{Solución}
\newtheorem{prop}[theorem]{Proposición}
\newtheorem{remark}{Remark}

\newtheorem{dem}[theorem]{Demostración}

\newtheorem{summary}{Summary}

\providecommand{\abs}[1]{\lvert#1\rvert}
\providecommand{\norm}[1]{\lVert#1\rVert}
\providecommand{\ninf}[1]{\norm{#1}_\infty}
\providecommand{\numn}[1]{\norm{#1}_1}
\providecommand{\gabs}[1]{\left|{#1}\right|}
\newcommand{\bor}[1]{\mathcal{B}(#1)}
\newcommand{\R}{\mathbb{R}}
\newcommand{\Q}{\mathbb{Q}}
\newcommand{\Z}{\mathbb{Z}}
\newcommand{\F}{\mathbb{F}}
\newcommand{\X}{\chi}
\providecommand{\Zn}[1]{\Z / \Z #1}
\newcommand{\resi}{\varepsilon_L}
\newcommand{\cee}{\mathbb{C}}
\providecommand{\conv}[1]{\overset{#1}{\longrightarrow}}
\providecommand{\gene}[1]{\langle{#1}\rangle}
\providecommand{\convcs}{\xrightarrow{CS}}
% xrightarrow{d}[d]
\setcounter{exercise}{0}
\newcommand{\cicl}{\mathcal{C}}

\newenvironment{ejercicio}[2][Estado]{\begin{trivlist}
\item[\hskip \labelsep {\bfseries Ejercicio}\hskip \labelsep {\bfseries #2.}]}{\end{trivlist}}
%--------------------------------------------------------
\begin{document}
\title{Relación 3 - Geometría y Topología de superficies }
\author{Javi, Rafa, Diego}
\maketitle
\begin{exercise}
Probar que si $f:X\to S^n$ no es epiyectiva entonces es homotópica a la aplicación constante.
\end{exercise}
\newpage
\begin{exercise}
Probar que todo retracto de un espacio métrico es un subespacio cerrado.
\end{exercise}
\begin{sol*}
Sea $X$ un espacio métrico y $A\subseteq X$ un retracto. Por hipótesis existe una aplicación continua $r:X\to A$ tal que $r(a)=a\ \forall a\in A$. Sea $\{x_n\}\subset A$ una sucesión que converge a un punto $x\in X$. Tenemos que probar que $x\in A$. Tenemos que $r(x_n)$ converge a $r(x)$. Además, $r(x_n)=x_n\ \forall n$ por ser una sucesión de $A$. Por lo tanto, al tratarse de un espacio métrico (y por tanto Hausdorff) $r(x)=x$, y como $r(x)\in A$ por definición, se tiene el resultado. 
\end{sol*}
\newpage
\begin{exercise}
Probar que $A$ es retracto de $X$ si y sólo si toda función continua de $A$ es un espacio arbitrario $Z$ admite una extensión continua sobre $X$.
\end{exercise}
\newpage
\begin{exercise}
Probar que si $A$ y $B$ son retractos de deformación de $X$ e $Y$ respectivamente, entonces $A\times B$ es retracto de deformación de $X\times Y$.
\end{exercise}
\begin{sol*}
Por hipótesis existen $r_1:X\to A$ y $r_2:Y\to B$ continuas tales que $r_1(a)=a\ \forall a\in A$ y $r_2(b)=b\ \forall b\in B$. Podemos construir entonces la aplicación $r:X\times Y\to A\times B$ definida como $r(x,y)=(r_1(x),r_2(y))$ que cumple claramente $r(a,b)=(a,b)\ \forall (a,b)\in A\times B$ y además es continua por ser el producto de aplicaciones continuas. Hasta ahora hemos probado que $A\times B$ es retraco de $X\times Y$, ahora tenemos que probar que es de deformación. Por hipótesis $i_1\circ r_1\simeq Id_X$ e $i_2\circ r_2\simeq Id_Y$, donde $i_1$ e $i_2$ son las respectivas inclusiones. Sean $H_1$ y $H_2$ las homotopías correspondientes. Vamos a construir una homotopía entre $i\circ r$ y $Id_{X\times Y}$, donde $i=i_1\times i_2$. Basta tomar $H:X\times Y\times I\to X\times Y$ definida como $H((x,y),t)=H_1(x,t)\times H_2(y,t)$, que es continua por ser producto de continuas y cumple por hipótesis
\begin{gather*}
H((x,y),0)=H_1(x,0)\times H_2(y,0)= (x,y)\\
H((x,y),1)=H_1(x,1)\times H_2(y,0)= (i_1(r_1(x)),i_2(r_2(y)).
\end{gather*}
Esto finaliza el ejercicio.
\end{sol*}
\newpage
\begin{exercise}
Probar que si $A$ es retracto de deformación de $X$ y $B$ lo es de $A$, entonces $B$ lo es de $X$. 
\end{exercise}
\begin{sol*}
Por hipótesis existe una función continua $H:X\times X\to X$ que cumple $H(x,0)=x$, $H(x,1)\in A$, y $H(a,t)=a\ \forall a\in A$. También existe $G:A\times A\to A$ cumpliendo $G(a,0)=a$, $G(a,1)\in B$ y $G(b,t)=b\ \forall b\in B$. La idea será retraer con deformación $X$ hasta convertirlo en $A$ y luego continuar hasta convertirlo en $B$. Sea $F:X\times I\to X$ definida como sigue
\[
F(x,t)=\begin{cases}
H(x,2t) & t\in[0,\frac{1}{2}]\\
G(H(x,1),2t-1) & t\in[\frac{1}{2},1]
\end{cases},
\]
que es continua puesto que $F(\frac{1}{2},t)=G(H(x,1),0)=H(x,1)$. Comprobamos que se tiene
\begin{gather*}
F(x,0)=H(x,0)=x\\
F(x,1)=G(H(x,1),1)\in B\\
F(b,t)=H(b,t)=b\ \forall t\in [0,\frac{1}{2}]\ \forall b\in B\\
F(b,t)=G(H(b,1),2t-1)=H(b,1)=b\ \forall t\in[\frac{1}{2},1] \ \forall b\in B.
\end{gather*}
Con lo que ya hemos terminado. 
\end{sol*}
\newpage
\begin{exercise}Dar un ejemplo de un espacio $X$ y de dos subconjuntos $A$ y $B$ homeomorfos tales que $A$ sea retracto de deformación de $X$ y $B$ no lo sea.
\end{exercise}
\begin{sol*}
Sea $X=X_1\cup X_2$ un compacto de $\R^2$ formado por dos componentes conexas $X_1$ y $X_2$ tal como se ve en la figura más abajo. Sea $A=\{a_1,a_2\}$ con $a_1\in X_2$ y $a_2\in X_2$. Sea $B=\{b_1,b_2\}\subset X_1$. Como cada $X_i$ es contráctil, $X_1$ puede retraerse con deformación sobre $a_1$ y lo mismo con $X_2$ sobre $a_2$. Pero $B$ no puede ser retracto de deformación, ya que ambos puntos están en una misma componente conexa. Pero $A\cong B$, pues ambos son dos puntos. 
\definecolor{qqqqff}{rgb}{0.3333333333333333,0.3333333333333333,0.3333333333333333}
\definecolor{zzttqq}{rgb}{0.26666666666666666,0.26666666666666666,0.26666666666666666}
\begin{tikzpicture}[line cap=round,line join=round,>=triangle 45,x=1.0cm,y=1.0cm]
\clip(-8.293333333333335,-0.43333333333333324) rectangle (7.04,6.993333333333332);
\fill[color=zzttqq,fill=zzttqq,fill opacity=0.10000000149011612] (-5.,5.) -- (-2.,5.) -- (-2.,1.) -- (-5.,1.) -- cycle;
\fill[color=zzttqq,fill=zzttqq,fill opacity=0.10000000149011612] (-1.,5.) -- (-1.,1.) -- (2.,1.) -- (2.,5.) -- cycle;
\draw [color=zzttqq] (-5.,5.)-- (-2.,5.);
\draw [color=zzttqq] (-2.,5.)-- (-2.,1.);
\draw [color=zzttqq] (-2.,1.)-- (-5.,1.);
\draw [color=zzttqq] (-5.,1.)-- (-5.,5.);
\draw [color=zzttqq] (-1.,5.)-- (-1.,1.);
\draw [color=zzttqq] (-1.,1.)-- (2.,1.);
\draw [color=zzttqq] (2.,1.)-- (2.,5.);
\draw [color=zzttqq] (2.,5.)-- (-1.,5.);
\draw (-4,5.91) node[anchor=north west] {$\Large{X_1}$};
\draw (0.,5.91) node[anchor=north west] {$\Large{X_2}$};
\draw (-3.6,2.2) node[anchor=north west] {$\Large{a_1}$};
\draw (0.36,2.2) node[anchor=north west] {$\Large{a_2}$};
\draw (-3.6,4.2) node[anchor=north west] {$\Large{b_1}$};
\draw (-3.56,3.2) node[anchor=north west] {$\Large{b_2}$};
\begin{scriptsize}
\draw [fill=qqqqff] (-4.,4.) circle (2.5pt);
\draw [fill=qqqqff] (-4.,3.) circle (2.5pt);
\draw [fill=qqqqff] (-4.,2.) circle (2.5pt);
\draw [fill=qqqqff] (0.,2.) circle (2.5pt);
\end{scriptsize}
\end{tikzpicture}
\end{sol*}
\newpage
\begin{exercise}
Probar que $S^{n-1}$ es retracto de deformación de $R^n-\{0\}$. Encontrar retractos de deformación (compactos) de los siguientes espacios.
\begin{enumerate}
\item $\R^2-\{p_1,p_2,\dotsc,p_k\}$,
\item $S^1\times S^1-\{*\}$ (el toro menos un punto),
\item $\R^3-OZ$, $\R^3-S^1\lor S^1$,
\item la banda de Möbius y la banda de Möbius menos un punto,
\item $\R^4-(OZ\bigcup S^1)$,
\item el complementario de un disco en el plano proyectivo y
\item el complementario en $\R^3$ de dos rectas paralelas.
\end{enumerate}
\end{exercise}
\newpage
\begin{exercise}
Probar que si $X$ es contráctil e $Y$ arcoconexo, dos aplicaciones cualesquiera de $X$ en $Y$ son homotópicas.
\end{exercise}
\begin{sol*}
Por ser $X$ contráctil, $Id_X\simeq x_0$ mediante una homotopía $F:X\times I\to X$. Sea $f:X\to Y$ una aplicación continua y sea $f\circ F: X\times I\to Y$ la aplicación $(x,t)\mapsto f(F(x,t))$, que nos da una homotopía entre $f$ y la aplicación constante $f(x_0)$. Dado $y\in Y$ sea $\gamma: I\to Y$ un camino desde $f(x_0)$ hasta $y$. Definimos entonces la homotopía $G:X\times I\to Y$ como $G(x,t)=\gamma(t)$. Por lo tanto, dadas $f,g:X\to Y$ funciones continuas, podemos partir de $f(x_0)$ y llegar a $y=g(x_0)$ con la siguiente homotopía $H:X\times Y\to Y$,
$$
H(x,t)=\begin{cases}
f(F(x,t)) & t\in[0,\frac{1}{2}]\\
\gamma(t) & t\in[\frac{1}{2},1]
\end{cases},
$$
que claramente es continua.
\end{sol*}
\newpage
\begin{exercise}
Indicar cuántos elementos tiene el conjunto de las clases de homotopía de aplicaciones continuas de $(S^1,\{1\})$ en $(X,*)$ rel. $\{1\}$, en los casos siguientes:
\begin{enumerate}
\item X tiene la topología discreta.
\item X tiene la topología indiscreta.
\end{enumerate}
\end{exercise}
\newpage
\begin{exercise}
Sea $f:X\to Y$ una aplicación continua. Sobre la unión disjunta $(X\times I)\sqcup Y$ se considera la relación de equivalencia resultante de identificar $(x,1)$ con $f(x)$. El espacio cociente obtenido se llama cilindro de la aplicación $f$, y se denotará por $M_f$. Probar que Y es retracto de deformación de $M_f$.
\end{exercise}
\begin{sol*}
Definimos la aplicación $r:(X\times I)\sqcup Y\to Y$ como $r(y)=y\ \forall y\in Y$ y $r(x,t)=f(x)\ \forall x\in X\times I$. Esta aplicación induce una retracción $\tilde{r}$ en el cociente de modo que $\tilde{r}([y])=[y]$ si $y\in Y$ y $\tilde{r}([x,t])=f(x)$. Esta aplicación es continua porque sobre cada componente es continua (son la identidad y una proyección) y al pasar al cociente $\tilde{r}([y])=\tilde{r}([x,t])\ \forall y\in f(X\times I)$. Hemos probado que $Y$ es retracto de $M_f$, ahora vamos a probar que es de deformación. Por ahorrar notación identidicamos $\pi(Y)$ con $Y$, donde $\pi$ es la proyección al cociente. Definimos una homotopía $H:M_f\times I \to M_f$ como $H([x,t],s)=\pi(x,t+(1-t)s)$ en los puntos de la forma $[x,t]$ y $H([y],s)=[y]$ en el resto. Se cumple que $H([x,t],0)=[x,t]$ si $(x,t)\in X\times I$ y $H([y],0)=[y]$ si $y\in Y$. Y por otro lado $H([x,t],1)=[x,1]\in Y$. Por lo tanto es un retracto con deformación. De hecho es fuerte, puesto que los elementos de $Y$ permanecen inmóviles.

\end{sol*}
\newpage
\begin{exercise}
Probar que toda aplicación continua $f:Y_0\to Y_1$ da lugar a una aplicación $f_*:[X,Y_0]\to[X,Y_1]$ con las siguientes propiedades.
\begin{itemize}
\item Si $f':Y_0\to Y_1$ es homotópica a $f$ entonces $f_* =f'_*$.
\item Para $id:Y\to Y$ se tiene que $id_*$ es la identidad.
\item Si $g:Y_1\to Y_2$ es continua, entonces $(gf)_* = g_*f_*$.
\item Si $f$ es una equivalencia de homotopía, entonces $f_*$ es biyectiva.
\end{itemize}
\end{exercise}
\newpage
\begin{exercise}
\begin{enumerate}
\item[]
\item Si $f:Y_0\to Y_1$ es tal que $f_*$ es biyectiva para todo X, entonces $f$ es una equivalencia de homotopía.
\item Si $g:X_0\to X_1$ es tal que $g^*$ es la biyectiva para todo Y, entonces g es una equivalencia de homotopía.
\end{enumerate}
\end{exercise}
\newpage
\begin{exercise}
Probar que las siguientes aplicaciones son equivalencias de homotopía:
\begin{enumerate}
\item Toda aplicación continua que sea homotópica a una equivalencia de homotopía.
\item Toda aplicación continua entre espacios contráctiles.
\end{enumerate}
\end{exercise}
\begin{sol*}
\begin{enumerate}\
\item Sea $f:X\to Y$ una equivalencia de homotopía con $g:Y\to X$ como inversa homotópica. Sea $h:X\to Y$ equivalente a $f$. Entonces, usando las propiedades de la homotopía con la composición, 
\[
h\simeq f\Leftrightarrow h\circ g\simeq f\circ g\simeq Id_Y
\]
y
\[
h\simeq f\Leftrightarrow g\circ h\simeq g\circ f\simeq Id_X.
\]
Por transitividad, tenemos que $h$ es equivalencia de homotopía con la misma inversa que $f$. 
\item Si $f:X\to Y$ es una aplicación continua entre espacios contráctiles, entonces $f\simeq y_0\in Y$. Sea $g:Y\to X$ una aplicación continua, que cumplirá $g\simeq x_0\in X$. 
\[
f\circ g\simeq y_0\circ g = y_0\simeq Id_Y
\]
y,
\[
g\circ f\simeq x_0\circ f =x_0\simeq Id_X.
\]
Luego cualquier función continua es inversa homotópica de cualquier otra cuando los espacios son contráctiles. 
\end{enumerate}
\end{sol*}
\newpage
\begin{exercise}
Sean $f_1,g_1:X_1\to Y_1$ y $f_2,g_2:X_2\to Y_2$ aplicaciones tales que $f_1\simeq g_1$ y $f_2\simeq g_2$. Probar que $f_1\times f_2\simeq g_1\times g_2$. Como consecuencia, si $f_1$ y $f_2$ son equivalencias de homotopía, también lo es $f_1\times f_2$. Así pues, $X\times Y$ es contráctil si y solo sí $X$ e $Y$ lo son.
\end{exercise}
\begin{sol*}
Sea $H_1$ una homotopía entre $f_1$ y $g_1$, y $H_2$ una homotopía entre $f_2$ y $g_2$. Construimos la homotopía $H:X_1\times X_2\times I\to Y_1\times Y_2$ definida como $H((x_1,x_2),t)=(H_1(x_1,t),H_2(x_2,t)$. Se tiene
\begin{gather*}
H((x_1,x_2),0)=(H_1(x_1,0),H_2(x_2,0))=(f_1(x_1),f_2(x_2))\\
H((x_1,x_2),1)=(H_1(x_1,1),H_2(x_2,1))=(g_1(x_1),g_2(x_2))
\end{gather*}
y además es continua por ser producto de continuas. Si $f_1$ y $f_2$ son equivalencias de homotopía basta tomar $g_1$ y $g_2$ sus respectivas inversas homotópicas. En particular, si $f_1=x\in X$, $f_2=y\in Y$, $g_1=Id_X$ y $g_2=Id_Y$, $X\times Y$ es contráctil si y solo si lo son $X$ e $Y$. 
\end{sol*}
\newpage
\begin{exercise}
Se dice que el par $(X,A)$ tiene la Propiedad de Extensión de Homotpía (PEH) si para toda aplicación continua $f:X\to Y$ es un espacio arbitrario Y y toda homotopía $F:A\times I \to Y$ de aplicación $f_{|_A}$, existe homotopía $G:X\times I \to Y$ que extiende a $F\cup f$. Probar que si A es cerrado, $(X,A)$ tiene la PEH si y sólo si $(X\times 0)\bigcup A\times I$ es retracto de $X\times I$.
\end{exercise}
\newpage
\begin{exercise}
Sea $A\subset X$ cerrado y $*$ un punto de A considerado como punto base de X. Supongamos que $(X,A)$ tiene la PEH y que $\{*\}$ es un retracto de deformación fuerte de A. Entonces $p:(X,*)\to(X/A,[A])$ es una equivalencia de homotopía.
\end{exercise}
\newpage
\begin{exercise}
Sea $\Pi_0(X)$ el conjunto de las componentes conexas por caminos de X. Dada una aplicación $f:X\to Y$ sea $f_*:\Pi_0(X)\to\Pi_0(Y)$ la aplicación que lleva de componentes $C_x$ en la componente $C_{f(x)}$. Probar las siguientes propiedades:
\begin{enumerate}
\item $(g\circ f)_* = g_*\circ f_*$.
\item Si $f$ es homotópica a g entonces $f_*=g_*$.
\item Si $f$ es una equivalencai de homotopía entonces $f_*$ es una biyección. Además $C_x$ es del mismo tipo de homotopía que $C_{f(x)}$.
\end{enumerate}
\end{exercise}
\begin{sol*}
\begin{enumerate}
\item $(g\circ f)_*(C_x)=C_{g\circ f(x)}= g_*(C_{f(x)})=g_*\circ f_*(C_x)$.
\item Si $f$ es homotópica a $g$, existe una homotopía $H:X\times I\to Y$ tal que $H(x,0)=f(x)$ y $H(x,1)=g(x)$, luego existe un camino $\gamma(t)=H(x,t)$ que conecta $f(x)$ con $g(x)$, de modo que están en la misma componente conexta. Así pues, $C_{f(x)}=C_{g(x)}\ \forall x\in X\Leftrightarrow f_*=g_*$.
\item Si $f$ es equivalencia de homotopía, existe una apliación continua $g:Y\to X$ tal que $f\circ g\simeq Id_Y$ y $g\circ f\simeq Id_X$. Vamos a ver que $f_*$ es biyectiva comprobado que $g_*$ es su inversa. Como hemos visto en el primer apartado, $(g\circ f)_*(C_x)=C_{g\circ f(x)}$. Como $g\circ f\simeq Id_X$, existe un camino (dado por la homotopía) entre $g\circ f$ y $x$, por lo que  $(g\circ f)_*(C_x)=C_x$. Análogamente, deducimos que $(f\circ g)_*(C_y)=C_y$, por lo que efectivamente $g_*$ es su inversa. Corolario de este resultado es que $C_x$ es del mismo tipo de homotopía que $C_{f(x)}$, puesto que bastaría tomar $f:C_x\to C_f(x)$ y $g:C_f(x)\to C_x$. Por lo anterior $g\circ f\simeq Id_X$, luego en particular $g\circ f\simeq Id_{C_x}$. Del mismo modo $f\circ g\simeq Id_{C_f(x)}$. 
\end{enumerate}
\end{sol*}
\newpage
\begin{exercise}
Probar que un retracto de un espacio contráctil es contráctil.
\end{exercise}
\begin{sol*}
Sabemos que $X$ es contráctil si y solo si $Id_X\simeq cte$ mediante una homotopía $H$. Sea $r:X\to A$ una retracción. Vamos a ver que $A$ es contráctil. Por ser $r$ retracción, se tiene
\[
r\circ i =Id_A,\ \text{con }i: X\hookrightarrow A\text{ la inclusión}
\]
Sea $H:I\times I\to X$ una homotopía que cumpla $H(x,0)=x$ y $H(x,1)=x_0\ \forall x\in X$. Buscamos $G:A\times I\to A$ cumpliendo  $G(a,0)=a$ y $G(a,1)=a_0\in A\ \forall a\in A$. Vamos a definirla como la siguiente composición:
\[
\begin{tikzcd}
G: A\times I \arrow[r,"i\times Id_I"]& X\times I \arrow[r, "H"] &  X \arrow[r,"r"] & A 
\end{tikzcd}
\]
Así pues, dado $a\in A$,
\begin{gather*}
G(a,0)=r\circ H\circ (i\times Id_X)(a,0)=rH(a,0)=r(a)=a\\
G(a,1)=r\circ H\circ (i\times Id_X)(a,1)=rH(a,1)=r(x_0)=a_0\in A
\end{gather*}
Por lo que hemos llegado al resultado. 
\end{sol*}
\newpage
\end{document}
