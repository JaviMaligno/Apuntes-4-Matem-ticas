\documentclass[twoside]{article}
\usepackage{amsmath,accents}%
\usepackage{amsfonts}%
\usepackage{amssymb}%
\usepackage{graphicx}
\usepackage{mathrsfs}
\usepackage[utf8]{inputenc}
\usepackage{amsfonts}
\usepackage{amssymb}
\usepackage{graphicx}
\usepackage{mathrsfs}
\usepackage{setspace}  
\usepackage{amsmath}
\usepackage{nccmath}
\usepackage[spanish]{babel}
\usepackage{multirow}

\renewcommand{\baselinestretch}{1,4}
\setlength{\oddsidemargin}{0.5in}
\setlength{\evensidemargin}{0.5in}
\setlength{\textwidth}{5.4in}
\setlength{\topmargin}{-0.25in}
\setlength{\headheight}{0.5in}
\setlength{\headsep}{0.6in}
\setlength{\textheight}{8in}
\setlength{\footskip}{0.75in}

\newtheorem{theorem}{Teorema}[section]
\newtheorem{acknowledgement}{Acknowledgement}
\newtheorem{algorithm}{Algorithm}
\newtheorem{axiom}{Axiom}
\newtheorem{case}{Case}
\newtheorem{claim}{Claim}
\newtheorem{propi}[theorem]{Propiedades}
\newtheorem{condition}{Condition}
\newtheorem{consec}[theorem]{Consecuencia}
\newtheorem{coro}[theorem]{Corolario}
\newtheorem{criterion}{Criterion}
\newtheorem{defi}[theorem]{Definición}
\newtheorem{example}[theorem]{Ejemplo}
\newtheorem{exercise}{Exercise}
\newtheorem{lemma}[theorem]{Lema}
\newtheorem{nota}[theorem]{Nota}
\newtheorem{problem}{Problem}
\newtheorem{prop}[theorem]{Proposición}
\newtheorem{remark}{Remark}

\newtheorem{dem}[theorem]{Demostración}

\newtheorem{summary}{Summary}
\numberwithin{equation}{section}

\providecommand{\abs}[1]{\lvert#1\rvert}
\providecommand{\norm}[1]{\lVert#1\rVert}
\providecommand{\ninf}[1]{\norm{#1}_\infty}
\providecommand{\numn}[1]{\norm{#1}_1}
\providecommand{\gabs}[1]{\left|{#1}\right|}
\newcommand{\bor}[1]{\mathcal{B}(#1)}
\newcommand{\erre}{\mathbb{R}}
\newcommand{\R}{\mathbb{R}}
\newcommand{\resi}{\varepsilon_L}
\providecommand{\conv}[1]{\overset{#1}{\longrightarrow}}
\providecommand{\convcs}{\xrightarrow{CS}}
\providecommand{\func}[2]{\colon{#1}\longrightarrow{#2}}
\providecommand{\lrg}{\longrightarrow}
%--------------------------------------------------------
\begin{document}

\title{Resumen de Geometría Local de Curvas y Superficies }
\author{Rafael González López}
\maketitle
\section{Curvas}
\subsection{Introducción}
\begin{defi}
Sea $\alpha:(a,b)\longrightarrow \erre$ $\alpha=\alpha(t)$ una CAPR y $\alpha(t_0)$ un punto de la misma. Se denomina parámetro natural (o longitud de arco) de $\alpha$ respecto del punto $\alpha(t_0)$ a la función escalar:
\begin{gather*}
s:(a,b)\longrightarrow \erre\\
s(t)=\int_{t_0}^t \abs{\alpha'(t)}dt
\end{gather*}
\end{defi}
\begin{theorem}[Caracterización del Parámetro Natural] Sea $\alpha:(a,b)\longrightarrow \erre$, $\alpha=\alpha(t)$ una CAPR. Entonces $t$ es el p.n. de $\alpha$ (salvo constante, es decir $t=s+cte$) si y solo sí $\abs{\alpha'(t)}=1$ $\forall t\in(a,b)$.
\end{theorem}
\subsection{Curvatura}
\begin{defi}
Sea $\alpha=\alpha(s)$ una CARPN, $\alpha:(a,b)\longrightarrow \erre^3$. Se denomina \textbf{curvatura} de $\alpha$ en cada punto de $\alpha(s)$ a la función escalar:
\begin{gather*}
k=k(s) \quad k:(a,b)\longrightarrow \erre\\
k(s)=\abs{\ddot{\alpha}(s)} \quad \forall s\in (a,b)
\end{gather*}
\end{defi}
\newpage
\subsection{Referencia de Frenet de curvas alabeadas}
\begin{defi}
Sea $\alpha(s)$ una CARPN. Se define el \textbf{vector normal principal} de $\alpha(s)$ en cada punto y se representa $N(s)$ como un vector unitario en la dirección de $\ddot{\alpha}(s)$. Es decir:
\begin{gather*}
N(s)=\frac{\ddot{\alpha}(s)}{\abs{\ddot{\alpha}(s)}}
\end{gather*}
En las condiciones anteriores se denomina binormal de $\alpha$ en cada punto y se representa por $B(s)$ al vetor:
\begin{gather*}
B(s)=\dot{\alpha}(s)\times\frac{\ddot{\alpha}(s)}{\abs{\ddot{\alpha}(s)}}=T(s)\times N(s)
\end{gather*}
\end{defi}
\begin{theorem} Expresiones de los vectores de la base de Frenet en el caso de que nuestra CAR no esté parametrizada naturalmente.
\begin{gather*}
T(t)=\frac{\alpha'(t)}{\abs{\alpha'(t)}} \qquad B(t)=\frac{\alpha'(t)\times\alpha''(t)}{\abs{\alpha'(t)\times\alpha''(t)}}\\
N(t)=\dfrac{[\alpha'(t)\times\alpha''(t)]\times\alpha'(t)}{\abs{[\alpha'(t)\times\alpha''(t)]\times\alpha'(t)}}
\end{gather*}
\end{theorem}
\subsection{Torsión}
\begin{defi}
Se define la torsión de un CARPN como: $\boxed{\tau(s)=-\dot{B}(s)N(s)}$.
\end{defi}
\begin{nota} No se puede pasar de $(2)$ a $(1)$ multiplicando por $-N(s)$ porque el producto escalar no es asociativo, es decir, $(-\dot{B}(s)N(s))(-N(s))\neq -\dot{B}(s)(N(s)N(s))$. A la inversa sí lo podemos hacer porque $\tau(s)$ es un escalar.
\end{nota}
\subsection{Consecuencias de la definición}
\begin{theorem}
Sea $\alpha(s)$ una CARPN. Entonces $\alpha(s)$ es plana si y solo si $\tau(s)=0$ $\forall s$.
\end{theorem}
\subsection{Ecuación de Frenet-Serret}
\begin{theorem}[Ecuaciones de Frenet-Serret] Sea $\alpha(s)$ una curva CARPN. En cada punto de la curva se tiene:
\begin{gather*}
\dot{T}(s)=k(s)N(s) \qquad \dot{N}(s)=-k(s)T(s)+\tau(s)B(s) \qquad \dot{B}(s)=-\tau(s)N(s)
\end{gather*}
\end{theorem}
\begin{theorem}
Normalmente calcularemos de la siguiente forma la curvatura y la torsión. 
\begin{gather*}
k(s)=\abs{\ddot{\alpha}(s)} \qquad k(t)=\frac{|\alpha'(t)\times\alpha''(t)|}{|\alpha'(t)|^3}\\
\tau(s)=\frac{\dot{\alpha}(s)\cdot(\ddot{\alpha}(s)\times\dddot{\alpha}(s))}{|\ddot{\alpha}(s)|^2} \qquad \tau(t)=\frac{\alpha'(t)\cdot(\alpha''(t)\times\alpha'''(t))}{|\alpha'(t)\times\alpha''(t)|^2}
\end{gather*}
\end{theorem}
\newpage
\section{Superficies}
\subsection{Introducción}
\begin{defi} Se denomina {\em superficie parametrizada regular o superficie inmersa} (SPR) en $\mathbb{R}^3$ a toda aplicación $\chi : U \, (abierto) \, \subseteq \mathbb{R}^2 \mapsto \mathbb{R}^3,$ tal que verifique las dos siguientes condiciones:
\begin{itemize}
\item C1 (Condición de Diferenciabilidad): $\chi \in \mathcal{C}^k$, para algún $k \in \mathbb{N}$.
\item C2 (Condición de Regularidad): $\frac{\partial \chi}{\partial u^1} \wedge \frac{\partial \chi}{\partial u^2} \neq \bar{0},$ \, para todo \, $(u^1, u^2) \in U$.
\end{itemize}
\end{defi}


\begin{defi} Se denomina {\em superficie simple} (SS) a toda superficie parametrizada regular que sea además inyectiva.
\end{defi}


\subsection{Vector Normal y Plano Tangente}

\begin{defi}
Sea $\chi: U \mapsto \mathbb{R}^3$ una superficie simple y $ p\in \chi(U)$ un punto de la misma. Se denomina {\em plano tangente} a $\chi$ en $p$ al plano que pasa por $p$ y tiene como vector normal al $\chi_1 \wedge \chi_2$ en $p.$
\end{defi}

\begin{defi}
Se denomina {\em vector unitario normal} a la superficie simple $\chi = \chi(u^1, u^2)$ en cada punto, y se representa por $ N(u^1, u^2),$ a un vector unitario ortogonal al plano tangente a $\chi$ en ese punto.
\begin{equation*}
N(u^1, u^2) = \frac{\chi_1(u^1, u^2) \wedge \chi_2(u^1, u^2)}{\mid \chi_1(u^1, u^2) \wedge \chi_2(u^1, u^2) \mid}
\end{equation*}
\end{defi}

\subsection{Curvas en una superficie}


\begin{defi} Sea $\chi : U \, (abierto) \subset \mathbb{R}^2 \mapsto \mathbb{R}^3,$ $\chi = \chi(u^1, u^2)$ una superficie simple. Se denomina {\em curva diferenciable} en $\chi$ a toda curva alabeada parametrizada regular, diferenciable, contenida entera\-mente en $\chi$, es decir a toda aplicación

$\alpha : (a,b) \, \subset \mathbb{R} \mapsto \mathbb{R}^3,$ $\alpha =
\alpha(t) = \chi(u^1(t), u^2(t)),$ para todo $ t \in (a,b).$
\end{defi}


\begin{defi} Consideremos el abierto $U$ de $\mathbb{R}^2,$ de coordenadas
$(u^1, u^2)$ y sea $\chi : U \, \mapsto \mathbb{R}^3$ una
superficie simple. Se denominan {\em líneas paramétricas o curvas
paramétricas} de $\chi$ a la imagen mediante la aplicación  $\chi$
de las líneas paramétricas $u^1 = cte$ y $u^2 = cte$ de $U,$ es decir,
las líneas paramétricas de $\chi$ son las curvas sobre la
superficie:
\begin{center} curvas \, $u^1$-param\'etricas: \,\, $\alpha =
\alpha(u^1) = \chi(u^1, u^2_0),$

curvas \, $u^2$-param\'etricas: \,\, $\beta = \beta(u^2) =
\chi(u^1_0, u^2).$
\end{center}
\end{defi}

\subsection{Vectores Tangentes en una superficie}

\begin{defi} Sea $\chi : U \, (abierto) \subset \mathbb{R}^2 \mapsto \mathbb{R}^3,$ $\chi = \chi(u^1, u^2)$ una superficie simple y $P \in \chi(U)$ un punto. Se denomina {\em vector tangente} a $\chi$ en $P$ a todo vector $X \in \mathbb{R}^3$ que sea tangente a una curva alabeada regular cualquiera, contenida en $\chi$ y que pase por $P$.
\end{defi}

\begin{defi}
Al conjunto de todos los vectores tangentes a la superficie $\chi$ en un punto $P$ de la misma lo denotaremos por $T_P(\chi),$ es decir:
\begin{center}
$T_P(\chi) = \{X \in \mathbb{R}^3 \,| \, X \, es \, un \, vector \, tangente \, a \, \chi \, en \, P \}$
\end{center}
\end{defi}

\begin{theorem}
Todo vector tangente $X$ a una superficie simple $\chi$ en un punto $P$ de la misma puede expresarse según $X = X^1 \, \chi_1 + X^2 \, \chi_2.$
\end{theorem}

\begin{dem}
Sea $X \in T_p(\chi)$ y sea $\alpha = \alpha(t) = \chi(u^1(t), u^2(t))$ una curva en $\chi$ tal que $P = \alpha(0)$ y $X = \alpha'(0).$
Entonces:
\begin{gather*}
\alpha'(t) = \frac{\partial \, \chi}{\partial \, u^1} \, \frac{d \, u^1}{d\, t} + \frac{\partial \, \chi}{\partial \, u^2} \, \frac{d \, u^2}{d\, t} = \chi_1 \, 
\frac{d \, u^1}{d\, t} + \chi_2 \,
\frac{d \, u^2}{d\, t}
\end{gather*}
Se tendrá que:

\begin{gather*}
X = \alpha'(0) = \chi_1 \, 
\left.\left(\frac{d \, u^1}{d\, t}\right)\right|_{t=0} + \chi_2 \,
\left.\left(\frac{d \, u^2}{d\, t}\right)\right|_{t=0}
\end{gather*}
Y llamando:

\begin{gather*}
X^1 = \left.\left(\frac{d \, u^1}{d\, t}\right)\right|_{t=0} \qquad  X^2 = \left.\left(\frac{d \, u^2}{d\, t}\right)\right|_{t=0}
\end{gather*}
Finalmente, $X = X^1 \, \chi_1 + X^2 \, \chi_2 = \sum^2_{i = 1} \, X^i \, \chi_1$.
\end{dem}
\subsubsection{Superficies de revolución}

\begin{defi} Se denominan {\em superficies de revoluci\'on} a aquellas superficies engendradas por una curva alabeada plana regular {\bf $\alpha = \alpha(v)$} (denominada curva generatriz) al girar alrededor de una recta contenida en su plano (denominada eje de giro).
\end{defi}
\newpage
\begin{defi} 
\begin{enumerate}
\item[]
\item [a)] Se denominan {\em meridianos} (o generatrices) de una superficie de revoluci\'on a las curvas intersecci\'on de dicha superficie con planos
que pasan por el eje. Los meridianos de una de una superficie de revoluci\'on son curvas del mismo tipo que la generatriz.

\item [b)] Se denominan {\em paralelos} (o c\'irculos de latitud) de una superficie de revoluci\'on a las curvas intersecci\'on de dicha superficie con planos perpendiculares al eje. Los paralelos de una de una superficie de revoluci\'on son siempre circunferencias (de ahí su nombre).
\end{enumerate}
\end{defi}
\begin{nota} De las definiciones se siguen las siguientes equivalencias:
\begin{itemize}
\item Meridianos $\equiv$ generatrices $\equiv$ curvas v-param\'etricas $\equiv$ curvas {\bf $u = cte$}.
\item Paralelos $\equiv$ c\'irculos de latitud $\equiv$ curvas u-param\'etricas $\equiv$ curvas {\bf $v = cte$}.
\end{itemize} 
\end{nota}


\subsection{Ejemplos de superficies de revoluci\'on consideradas como superficies simples}

Sea $\alpha = \alpha(v) = (x(v), 0, z(v))$ una curva plana
regular situada en el plano OXZ, es decir, en el plano $y = 0$. Como
eje de giro se supondr\'a el propio eje OZ.


$$
\begin{array}{lll}
{\underline {Curva \, Generatriz}} & {\underline {Sup. \, Rev.}} &
{\underline {Parametrizaci\acute on \, Sup. Revol.}}  \\

   &   &   \\

	\alpha(v) = (x(v), 0, z(v)) &  & \chi(u,v) =
(x(v) \, cos \, u, x(v)\,  sen\, u, z(v))  \\

    &   &    \\

\alpha(v) = (a \, cos\, v, 0, a \, sen\, v) & Esfera & \chi(u,v) =
(a\,  cos\, v \, cos\, u, a \, cos\, v \, sen\, u, a \, sen\, v)  \\

  &  &  \\

\alpha(v) = (a, 0, v) & Cilindro  & \chi(u,v) =
(a \, cos\, u, \, a \, sen\, u, v)  \\
  & Circular &  \\
  &  &  \\

\alpha(v) = (v, 0, Kv) & Cono & \chi(u,v) =
(v\,  cos\, u, v \, sen\, u, Kv)  \\
  & Circular &  \\
 & & \\

\alpha(v) = (a + b \, cos\, v, 0, b \, sen\, v) & Toro & \chi =
((a + b \, cos\, v)\,  cos\, u, (a + b\,  cos\, v)\,  sen\, u, b \, sen\, v)  \\

  &  &  \\
\end{array}$$
\subsection{Superficies regladas y desarrollables}

\begin{defi} Se denominan
{\bf superficies
regladas} a aquellas superficies que tienen la pro\-piedad
de que por cada uno de sus puntos pasa una recta que est\'a contenida
enteramente en dicha superficie. Estas superficies se componen por tanto
de un n\'umero infinito de rectas (que se denominan {\em generatrices} de la
superficie).
\end{defi}

\begin{theorem} Las superficies regladas est\'an {\bf engendradas por una curva}
{\bf $\alpha = \alpha(u)$}
(deno\-minada {\em curva directriz}) {\bf y un vector} unitario
diferenciable {\bf $\omega = \omega(u)$} (denominado {\em campo de direcciones}), de forma que por cada punto de
$\alpha(u)$ pase una recta cuyo vector director es $\omega(u)$ en ese punto.

\vspace{0.20cm}

La parametrizaci\'on de una superficie reglada viene dada por
$$
\chi = \chi(u,v) = \alpha(u) \, + \, v \,\, \omega(u)
$$
con
$\chi: (a, b) \times \mathbb{R} \mapsto \mathbb{R}^3$, donde (a, b) es el
dominio de la curva  $\alpha = \alpha(u)$.
\end{theorem}

\begin{defi} Una superficie reglada se dice {\em desarrollable} si el plano tangente es constante a lo largo de cada generatriz $u = cte,$ es decir, si el vector normal $N(u,v)$ a lo largo de cada generatriz $u = cte$ no depende de $v.$
\end{defi}

\begin{theorem} Una superficie reglada $\chi(u,v) = \alpha(u) + v \omega(u)$ es desarrollable si y sólo si $(\alpha'(u) \,\, \omega(u) \,\, \omega'(u)) = 0.$
En este resultado no es necesario que $\omega(u)$ sea unitario.
\end{theorem}
\newpage
\section{Geometría Intrínseca}
Partimos de las siguientes hipótesis: Sea $\chi=\chi(u^1,u^2), \chi\func{U}{\R^3}$ una superficie simple, $P\in\chi(u)$ un punto de la misma, $T_P(\chi)$ el plano tangente a $\chi$ en $P$ con $\{\chi_1(P),\chi_2(P)\}$ una base del mismo y $X,Y \in\R^3$ dos vectores tangentes en $P$, lo cuales vienen dados por $X=\sum_{i=1}^{2}X^i\chi_i, Y=\sum_{i=1}^{2}Y^j\chi_j$.

\begin{defi} En las condiciones anteriores, se denomina \textbf{coeficientes métricos} (o coeficientes de la métrica riemanniana o coeficientes de la IFF) de la SS $\chi$ en cada punto a las funciones escalares:
\begin{gather*}
g_{ij}\func{U}{\R}\\
g_{ij}(u^1,u^2)=\chi_i(u^1,u^2)\cdot\chi_j(u^1,u^2) \quad i,j=1,2
\end{gather*}
\end{defi}
\begin{defi} Sea $\chi$ una superficie simple y $(g_{ij})$ la matriz de coeficientes métrico. Denotamos $g^{ij}$ de $\chi$ a las funciones escalares que se corresponden con los elementos $(i,j)$ de la matriz inversa de $(g_{ij})$.
\end{defi}
\begin{theorem}
Se tiene que $g^{11}=\dfrac{g_{22}}{g}$, $g^{12}=g^{21}=\dfrac{-g_{12}}{g}$ y $g^{22}=\dfrac{g_{11}}{g}$
\end{theorem}


\subsection{Curvatura Geodésica}

\begin{defi}Se denomina \textbf{Vector Normal Intrínseca} en cada punto de la curva $\alpha$ sobre $\chi$ y se denota por $S(s)$ al vector $N(s)\wedge\textbf{t}(s)$.
\end{defi}

\begin{theorem}[Expresión de $\ddot{\alpha}$ respecto de las 2 bases anteriores]\
\begin{enumerate}
\item[A] Con respecto de BF: $\ddot{\alpha}(s)=k(s)\textbf{n}(s)$.
\item[B] Con respecto a BD: $\ddot{\alpha}(s)=K_g(s)\textbf{S}(s)+K_\textbf{N}(s)\textbf{N}(s)$. 
\end{enumerate}
Donde $K_g(s)$ es la \textbf{curvatura geodésica} y $K_N(s)$ es la \textbf{curvatura normal} de $\alpha$
\end{theorem}

\begin{defi}[Ecuaciones de Gauss] 
\begin{equation*}
\chi_{ij}=\sum_{k=1}^2\Gamma_{ij}^k\chi_k+L_{ij}N\quad i,j=1,2
\end{equation*}
Los coeficientes $\Gamma_{ij}^k$ se denominan \textbf{símbolos de Christoffel} (de segunda especie) y los $L_{ij}$ son los coeficientes de la IFF de la superficie. Ambos tipo de coeficientes son funciones escalares definidas en el abierto $U$ de $\chi$.
\end{defi}

\begin{theorem}[Teorema de Gauss]
Se verifica que:
\[
\Gamma_{ij}^k = \frac{1}{2}\sum_{h=1}^2 g^{hk}\left(\frac{\partial g_{ih}}{\partial u^j}+\frac{\partial g_{jh}}{\partial u^i}-\frac{\partial g_{ij}}{\partial u^h}\right)
\]
\end{theorem}
\begin{theorem}
En las condiciones anteriores se verifica que:
\[\Gamma^k_{ij} = \sum_{h=1}^2 (\chi_{ij} \chi_h ) g^{hk}\]
\end{theorem}


\begin{theorem}
Sea $\alpha(s)$ una CARPN sobre una superficie $\chi$. Se verifican:
\begin{itemize}
\item $K_g(s) = ({\bf t}(s)\; \dot{\bf t}(s) \; \textbf{N}(s) )$.
\item $K_g(s) = k(s)\cos{\varphi(s)}$ donde $\varphi(s)=\widehat{(\textbf{b}(s) \textbf{N}(s))}$
\end{itemize}
\end{theorem}

\begin{theorem} Sea $\alpha(s)$ una CAR no necesariamente PN. Se tiene que:
\[
K_g(t)=\frac{1}{|\alpha'(t)|^3}(\alpha'(t)\; \alpha''(t)\; N(t))
\]
\end{theorem}
\subsection{Geodésicas}

\begin{defi}
Sea $\chi = \chi(u^1, u^2)$ una superficie simple. Se denomina \textbf{geodésica} en $\chi$ a toda curva alabeada regular p.n. $\alpha = \alpha(s) = \chi(u^1(s), u^2(s))$ sobre $\chi$ tal que $K_g(s) = 0.$
\end{defi}

\begin{theorem} {\em Teorema de Caracterización de las Geodésicas}

Sea $\chi = \chi(u^1, u^2)$ una superficie simple y $\alpha = \alpha(s) = \chi(u^1(s), u^2(s))$ una CARPN sobre $\chi.$ Entonces, $\alpha(s)$ es geodésica sobre $\chi$ si y solo si se cumple una de las siguientes condiciones, equivalentes entre sí:

\begin{enumerate}
\item[C1] \quad $({\bf t}(s) \,\, \dot{{\bf t}}(s) \,\, N(s)) = 0, \quad [\equiv \, K_g(s) = 0].$
\item[C2] \quad $\ddot{u}^k + \sum_{i, j = 1}^2 \, \Gamma_{ij}^k \, \dot{u}^i \, \dot{u}^j \, = 0, \quad i,j = 1,2.$ \quad {\em [Sistema de Ec. Dif. de las geodésicas]}.
\item[C3] \quad ${\bf n}(s) \, \| \, N(s)$ a lo largo de $\alpha.$
\item[C4] \quad ${\bf b}(s) \, \perp \, N(s),$ es decir, el plano osculador a $\alpha$ en cada punto es ortogonal a la superficie en ese \quad punto.
\end{enumerate}
\end{theorem}



\begin{theorem}

En toda superficie de revolución $\chi(u,v) = (f(v) \, cos u, f(v) \, sen u, \, g(v))$ se verifica que:
\begin{enumerate}
\item Los meridianos de la superficie (es decir, las curvas $v$-paramétricas) son siempre geodésicas.
\item Los paralelos de la superficie (es decir, las curvas $u$-paramétricas) son geodésicas si y solo si $f'(v_0) = 0.$
\end{enumerate}
\end{theorem}
\section{Geometría Extrínseca}

\begin{lemma}

En toda superficie simple $\chi$ se verifica que $ \mid \chi_1
\wedge \chi_2 \mid = \sqrt{g}$ y 
$$L_{ij}  = \chi_{ij} \, N = \chi_{ij} \,
\frac{\chi_1 \wedge \chi_2}{\mid \chi_1 \wedge \chi_2  \mid} =
\frac{(\chi_1 \,  \chi_2 \, \chi_{ij})}{\mid \chi_1 \wedge \chi_2
\mid} = \frac{(\chi_1 \,  \chi_2 \, \chi_{ij})}{\sqrt{g}}$$
\end{lemma}

\subsection{Curvaturas Principales y Direcciones Principales}

\begin{defi} Los valores $K_1$ y $K_2$ en los que la función $K_N(X)$ alcanza un máximo y un mínimo en $T_P^*(M)$ se denominan {\em curvaturas principales} de $M$ en $P.$
\end{defi}

\begin{defi} Las direcciones $X_{(1)}$ y $X_{(2)}$ de $T_P^*(M)$ correspondientes a las curvaturas principales $K_1$ y $K_2$ se denominan {\em direcciones principales} de $M$ en $P.$
\end{defi}

\begin{defi}
Un punto $P$ de una superficie $M$ se dice {\textbf{umbílico}} si en él
coinciden las dos curvaturas principales, es decir, si $K_1 = K_2$
en $P.$
\end{defi}

\begin{defi}
Se define la {\em curvatura Media} de $M$ y se representa por $H$
como la media aritmética de las curvaturas principales, es decir,
$H = \dfrac{K_1 + K_2}{2}.$
\end{defi}

\begin{defi}
Se define la {\em curvatura de Gauss o curvatura Total} de $M$ y
se representa por $K$ como el producto de las curvaturas
principales, es decir,  $K = K_1 \, K_2.$
\end{defi}
\begin{prop}
Se verifica que la curvatura de Gauss es:
\begin{itemize}
\item Es invariante por isometrías e isometrías locales.
\item Para la esfera de radio r: $K=\frac{1}{r^2}$
\item Para el plano: $K=0$.
\end{itemize}
Por la primera y la tercera, las superficies desarrollables como el plano, el cono, el cilindro o la banda de Möbius tienen $K=0$.
\end{prop}
\end{document}