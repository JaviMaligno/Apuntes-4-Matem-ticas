\documentclass[twoside]{article}
\usepackage{../estilo-ejercicios}

\usepackage{enumerate}
%--------------------------------------------------------
\begin{document}

\title{Algebra Conmutativa y Geometría Aplicada}
\author{Rafael González López, Diego Pedraza López}
\maketitle

\begin{ejercicio}{1}
Encontrar todos los enteros positivos tales que $\varphi(n)$ no sea divisible por 4.
\begin{sol}
Sea $n\in \N$. Sabemos que $n=p_1^{a_1}\cdots p_k^{a_k}$. En tal caso, 
\[
\varphi(n) = (p_1-1)p_1^{a_1-1}\cdots(p_k-1)p_k^{a_k-1}
\]
Obsérvese que si $n$ tiene más de un divisor primo impar, claramente se tendría que $4\mid \varphi(n)$, pues $p_j -1$ y $p_k-1$ serían dos factores pares de $\varphi(n)$. Por otra parte, si $p_i=2$, debe cumplir que $0\leq a_i \leq 2$. 

Por lo que sabemos hasta ahora, $n = 2^a p^b$ donde $a\in\{0,1,2\}$, $p$ es un primo impar y $b\geq 0$. Distingamos casos en función de $a$ y $b$
\begin{itemize}
\item Si $a \in \{0,1\}$ y $b = 0$, obtenemos $n=1$ y $n=2$, que cumplen que $4 \nmid φ(n)$.
\item Si $a \in \{0,1\}$ y $b > 0$, $\varphi(n) = (p-1)p^{b-1}$. La condición del enunciado se cumple si y solo sí $4 \mid (p-1)$, es decir, si $p \equiv 3 \pmod 4$.
\item Si $a=2$ y $b = 0$, obtenemos $n=4$, que cumple que $4 \nmid φ(4)$.
\item Si $a=2$ y $b > 0$, $\varphi(n)=2(p-1)p^{b-1}$, que es siempre divisible por 4.
\end{itemize}
Tenemos entonces que las soluciones son:
\[ \{1,2,4\} \cup \{p^k \mid p\text{ primo con } p \equiv 3 \pmod 4, k≥1\} \cup \{2p^k \mid p\text{ primo impar}, k ≥ 1\} \].
\end{sol}
\end{ejercicio}

\newpage


\begin{ejercicio}{2}
Probar que para todo $k\geq 0$
\[
\gcd \left\{\binom{2k}{k},\binom{2k+1}{k}\right\}=\binom{2k}{k}\frac{1}{k+1}, \quad \gcd \left\{\binom{2k+1}{k},\binom{2k+2}{k+1}\right\}=\binom{2k+1}{k+1}
\]
\begin{sol}
Vamos a ver que si dividimos los elementos el candidato a máximo común divisor nos quedan elementos claramente coprimos.
\begin{align*}
\frac{\binom{2k}{k}}{\binom{2k}{k}\frac{1}{k+1}}&=k+1\\
\dfrac{\binom{2k+1}{k}}{\binom{2k}{k}\frac{1}{k+1}}&=\frac{\frac{(2k+1)!}{k!(k+1)!}(k+1)}{\frac{(2k)!}{k!k!}} = \frac{(2k+1)!k!^2(k+1)}{k!(k+1)!(2k)!}=2k+1
\end{align*}
Claramente $k+1$ y $2k+1$ son coprimos, pues si $\exists s$ que divide a ambos, entonces se tiene que $s\mid (2k+1)-(k+1) = k$, por lo que $s \mid (k+1)-k =1$. Análogamente para el segundo caso
\begin{align*}
\frac{\binom{2k+1}{k}}{\binom{2k+1}{k+1}}&=\frac{\frac{(2k+1)!}{k!(k+1)!}}{\frac{(2k+1)!}{(k+1)!k!}}=1\\
\dfrac{\binom{2k+2}{k+1}}{\binom{2k+1}{k+1}}&=\frac{\frac{(2k+2)!}{(k+1)!(k+1)!}}{\frac{(2k+1)!}{(k+1)!k!}} = \frac{(2k+2)!k!(k+1)!}{(k+1)!^2(2k+1)!}=\frac{2k+2}{k+1}=2
\end{align*}
Claramente coprimos.
\end{sol}
\end{ejercicio}


\newpage


\begin{ejercicio}{3}
Probar que para todo $n$
\[
\sum_{d\mid n} \sigma(d)\mu(n/d)=n 
\]
\begin{sol}
Dado que las funciones $\sigma(n)$ y $\mu(n)$ son multiplicativas, su convolución también lo es. Por tanto, como $n\mapsto n$ también es multiplicativa, basta probar la igualdad para los números de la forma $p^a$ con $p$ primo y $a\geq 1$ (el caso $n=1$ es trivial). Sea pues $n=p^a$ con $p$ primo, entonces
\begin{gather*}
\sum_{d\mid n} \sigma(d)\mu(n/d) = \sum_{k=0}^a \sigma(p^k)\mu(p^{a-k}) = \sigma(p^{a-1})\mu(p) + \sigma(p^a)\mu(1) = \\
=  -\frac{p^a - 1}{p-1} +\frac{p^{a+1}-1}{p-1} = \frac{p^{a+1}-p^a}{p-1} = p^a
\end{gather*}
Tal y como queríamos ver. Hemos usado que $\mu(p^a)=0$ $\forall a \geq 2$.
\end{sol}
\end{ejercicio}
\newpage
\begin{ejercicio}{4}
Definamos $$c(n)=\sum\limits_{k\perp n} e^{2k\pi i/n},$$ donde la suma se extiende a los restos de módulo $n$, es decir, la suma de las raíces primitivas $n-$ésimas de la unidad.

\begin{enumerate}[a)]
\item Probar que $c$ es una función aritmética multiplicativa.
\item Probar que $c(n)=\mu (n)$.
\end{enumerate}

\begin{sol}
Vamos a hacer ambos apartados a la vez, ya que si $c(n)=\mu(n)$ resulta que $c(n)$ es multiplicativa.\\
Usamos la misma función $\mu(n)$ para definir $c(n)$ ya que $\sum\limits_{d\mid n}\mu(d)=\left\lbrace\begin{array}{ll}
1 & \text{Si } n=1\\
0 & \text{Si } n>1
\end{array}\right.$\\ Tenemos pues:$$c(n)=\sum\limits_{k\perp n} e^{2k\pi i/n}=\sum\limits_{k=1}^n e^{2k\pi i/n}\sum\limits_{d\mid gcd(k,n)}\mu(d)$$ Reordenando la suma $$\sum\limits_{k=1}^n e^{2k\pi i/n}\sum\limits_{d\mid gcd(k,n}\mu(d)=\sum\limits_{d\mid n} \mu(d)\sum\limits_{d\mid k,k\leq n}e^{2k\pi i/n}=\sum\limits_{d\mid n} \mu(d)\sum\limits_{k=1}^{n/d}e^{2kd\pi i/n}$$ Ahora bien, la segunda suma es geométrica y por tanto $$\sum\limits_{k=1}^{n/d}e^{2kd\pi i/n}=\frac{1-e^{(2d\pi i/n)(n/d+1)}}{1-e^{2d\pi i/n}}-1=\frac{1-e^{2d\pi i/n}}{1-e^{2d\pi i/n}}-1=0\text{ si }n/d>1$$ Es decir, si y sólo si $d=n$ la suma se hará 1, por tanto $$c(n)=\sum\limits_{d\mid n} \mu(d)\sum\limits_{k=1}^{n/d}e^{2kd\pi i/n}=\mu(n)\qed$$
\end{sol}
\end{ejercicio}
\newpage

\begin{ejercicio}{5}
\begin{itemize}
\item[]
\item Demostrar que 
\[
\sum_{d\mid n} \mu(d)^2 = 2^{\omega(n)}
\]
donde $\omega(n)=k$ si $n=p_1^{a_1}\cdots p_k^{a_k}$.
\item Sabiendo que
\[
\frac{\zeta^2(s)}{\zeta(2s)}=\sum_{n=1}^\infty \frac{2^{\omega(n)}}{n^s}
\]
calcular la suma de la serie de Dirichlet
$$
\sum_{n=1}^\infty \frac{\mu(n)^2}{n^s}
$$
\end{itemize}
\begin{sol}
\begin{itemize}
\item[]
\item Primeramente, observamos que $\sum_{d\mid n} \mu(d)^2$ es la convolución de las funciones $f\equiv 1$ y $\mu^2$, ambas multiplicativas, por lo que también es multiplicativa. Además, $2^{\omega(n)}$ es claramente multiplicativa. Por tanto, solo tenemos que probar el resultado para los números de la forma $n=p^a$, $p$ primo y $a\geq 1$.
\[
\sum_{d\mid n} \mu(d)^2 = \sum_{n=0}^a \mu(p^n)^2 = \mu(1)^2 + \mu(p)^2 = 2^1 = 2^{\omega(n)}
\]
\item Basta realizar unas sencillas manipulaciones
\begin{gather*}
\frac{\zeta^2(s)}{\zeta(2s)}=\sum_{n=1}^\infty \frac{1}{n^s}\sum_{d\mid n} \mu(d)^2 = \left(\sum_{n=1}^\infty \frac{\mu(n)^2}{n^s}\right)\left(\sum_{n=1}^\infty \frac{1}{n^s}\right) = \zeta(s)\sum_{n=1}^\infty \frac{\mu(n)^2}{n^s}\\
\sum_{n=1}^\infty \frac{\mu(n)^2}{n^s} = \frac{\zeta(s)}{\zeta(2s)}
\end{gather*}
Alternativamente:
\[ \frac{ζ(s)}{ζ(2s)} = \prod_p \frac{1-1/p^{2s}}{1-1/p^s} = \prod_p \frac{(1-1/p^s)(1+1/p^s)}{1-1/p^s} = \prod_p (1+1/p^s) = \sum_{n=1}^{∞} \frac{μ(n)^2}{n^s} \]
\end{itemize}
\end{sol}
\end{ejercicio}
\newpage

\begin{ejercicio}{6}
Una función aritmética $f$ se dice \textit{completamente multiplicativa} si $f(mn)=f(m)f(n)$.
\begin{enumerate}[(a)]
\item Probar que la función de Liouville
	\[ λ(n) = (-1)^{Ω(n)} \]
	es completamente multiplicativa. ($Ω(n) = a_1+a_2+\cdots+a_k$ si $n=p_1^{a_1}p_2^{a_2}\cdots p_k^{a_k}$ es la factorzación canónica de $n$).
\item Demostrar que
	\[ \sum_{d|n} λ(d) = \begin{cases}
		1 &\text{ si }n\text{ es un cuadrado}\\
		0 &\text{ en otro caso}
	\end{cases}\]
\end{enumerate}
\end{ejercicio}
\begin{solucion}
Pasemos a la demostración.
\begin{enumerate}[(a)]
\item Sean $m,n\in \N$ entonces $mn = p_1^{a_1}\cdots p_k^{a_k}$. Entonces podemos escribir $m=p_{i_1}^{a_{i_1}}\cdots p_{i_s}^{a_{i_s}}$ y $n=p_{j_1}^{a_{j_1}}\cdots p_{j_t}^{a_{j_t}}$ de manera que si $i_q=j_l = t$ entonces $a_{i_q}+a_{j_l}=a_t$. Es claro, por tanto, que 
$$ \Omega(mn) = \Omega(n)+\Omega(m) $$ $$ \lambda(mn) = (-1)^{\Omega(mn)}=(-1)^{ \Omega(n)+\Omega(m)} = (-1)^{\Omega(n)}(-1)^{\Omega(m)} = \lambda(n)\lambda(m)
$$
Por lo que $\lambda$ es completamente multiplicativa.
\item  El miembro izquierdo de la igualdad es la convolución $\lambda(n)$ y $g(n)=1$, ambas multiplicativas, luego la convolución también lo será. Sea $h(n)$ la función descrita en el miembro derecho, entonces sean $n,m\in \N$ coprimos. Entonces $mn$ es un cuadrado si y solo si $m,n$ son cuadrados. Por tanto,
 $$
 h(nm)=\begin{cases}
 		1 &\text{ si }mn\text{ es un cuadrado}\\
 		0 &\text{ en otro caso}
 	\end{cases}
 $$
 $$h(n)h(m)=\begin{cases}
 		1 &\text{ si $m$ y $n$ son cuadrados}\\
 		0 &\text{ en otro caso}
 	\end{cases}
 	=\begin{cases}
 		1 &\text{ si $mn$ es un cuadrado}\\
 		0 &\text{ en otro caso}
	\end{cases}
 $$
 \newpage
 Como $h$ es multiplicativa, basta que probemos la igualdad para $n=p^a$ con $p$ primo y $a\geq 0$, aunque $a=0$ es trivial. 
 $$h(p^a) = \begin{cases}
 		1 &\text{ si $a$ es par}\\
 		0 &\text{ en otro caso}
 	\end{cases}	$$
 	$$
 	\sum_{d\mid n}\lambda(d) = \sum_{n=0}^a \lambda(p^n) = \sum_{n=0} (-1)^a = \begin{cases}
 		1 &\text{ si $a$ es par}\\
 		0 &\text{ en otro caso}
 	\end{cases}	$$
 	\end{enumerate}

\end{solucion}
\newpage

\begin{ejercicio}{7}
Se sabe que una función aritmética $f(n)$ cumple $\sum_{d|n} f(d) = n^2$.
\begin{enumerate}[(a)]
\item Demostrar que $f$ es multiplicativa.
\item Demostrar que
\[ f(n) = n^2 \prod_{p|n} \left(1-\frac{1}{p^2}\right)\]
\item Expresar $\sum_{n=1}^{∞} \frac{f(n)}{n^s}$ en términos de la función $ζ(s)$.
\end{enumerate}
\end{ejercicio}
\begin{solucion} Pasemos a demostrar cada uno de los apartados.
 \begin{enumerate}[(a)]
 \item Vamos aplicar la primera fórmula de inversión de Möbius.
 $$
 f(n)=\sum_{d\mid n}\mu(n/d)d^2
 $$
 Por tanto, $f$ es la convolución de $h(n)=n^2$ y $\mu(n)$, ambas claramente multiplicativas, por lo que también lo será $f$.
 +\item Sea $t(n) = n^2 \prod_{p|n} \left(1-\frac{1}{p^2}\right)$ entonces si $m$ y $n$ son coprimos, un primo $p$ divide o bien a uno o bien a otro, luego es claro que
 $$
 t(nm) = (nm)^2\prod_{p|nm} \left(1-\frac{1}{p^2}\right) = n^2m^2\prod_{p|m} \left(1-\frac{1}{p^2}\right)\prod_{p|n} \left(1-\frac{1}{p^2}\right)=t(n)t(m)
 $$
 Por tanto, $t(n)$ es multiplicativa. Para ver la igualdad basta probarla para los elementos de la forma $p^a$ con $p$ primo y $a\geq 0$, aunque el caso $a=0$ es trivial.
 \begin{align*}
 f(p^a)&= \sum_{d\mid n}\mu(n/d)d^2 = \sum_{k=0}^a\mu(p^k)(p^{a-k})^2 = \mu(1)p^{2a} + \mu(p)p^{2(a-1)}\\
 &=p^{2a}-p^{2a-2}\\
 t(p^a)&= p^a \left(1-\frac{1}{p^2}\right) = p^a - p^{2a-2}
 \end{align*}
 \item Consideremos
 $$
 \sum_{n=1}^{∞} \frac{f(n)}{n^s} = \sum_{n=1}^{∞} \frac{1}{n^s} \sum_{d\mid n}\mu(n/d)d^2 = \left( \sum_{n=1}^{∞} \frac{n^2}{n^s} \right)\left( \sum_{n=1}^{∞} \frac{\mu(n)}{n^s} \right) = \zeta(s-2)\zeta(s)^{-1} = \frac{\zeta(s-2)}{\zeta(s)}
 $$
 \end{enumerate}
\end{solucion}
\newpage

\begin{ejercicio}{8}
Demostrar que para todo $x > 0$ se tiene
\[ 2(e^x-1)e^{-2x} = \sum_{n=1}^{∞} \frac{μ(n)}{\senh nx} \]
\end{ejercicio}
\begin{solucion} Hagamos la siguiente consideración
\begin{align*}
\sum_{n=1}^{∞} \frac{μ(n)}{\sinh (nx)} &= 2 \sum_{n=1}^{∞} \frac{μ(n)}{e^{nx} - e^{-nx}} = 2 \sum_{n=1}^{∞} \frac{μ(n)}{e^{nx}} \frac{1}{1-e^{-2nx}}\\
&= 2 \sum_{n=1}^{∞} μ(n) e^{-nx}(1+e^{-2nx}+e^{-4nx}+\dots)\\
& = 2 \sum_{n=1}^{∞} μ(n) \sum_{k=1}^{∞} e^{-(2k-1)nx}= 2 \sum_{n=1}^{∞} \sum_{k=1}^{∞} μ(n) e^{-(2k-1)nx}
\end{align*}
ya que estas series son absolutamente convergente (compruébalo):
\[ \sum_{n=1}^{∞} \frac{μ(n)}{\sinh (nx)} = 2 \sum_{N=1}^{∞} e^{-Nx} \left(\sum_{n|N \text{ y }N/n\text{ es impar}} μ(n) \right)\]
Tenemos que si $N$ es impar:
\[ \sum_{\substack{n|N \\ N/n\text{ es impar}}} μ(n) = \begin{cases}
	1, &\text{ si }N = 1\\
	0, &\text{ si }N > 1
\end{cases}\]
Si $N$ es par y $N=2^am$:
\[ \sum_{2^ak|2^am} μ(n) = \sum_{k|m} μ(2^a k) = \sum_{k|m} μ(2^a) μ(k) = \begin{cases}
	0, &\text{ si }a ≥ 2\\
	-1 &\text{ si }m=1\\
	0 &\text{ c.c.}
\end{cases}\]
Luego:
\[ \sum_{n=1}^{∞} \frac{μ(n)}{\sinh (nx)} = 2(e^{-x}-e^{-2x})\]
\end{solucion}
\newpage

\begin{ejercicio}{9}
Sea $f(n)$ el mayor número natural $m$ tal que $m^2\mid n$. definimos
\[
g(n) = \sum_{d\mid n} f(d)\mu(n/d)
\]
\begin{itemize}
\item Demostrar que $g$ es una función multiplicativa.
\item Calcular $g(p)$, $g(p^2)$ y $g(p^3)$, siendo $p$ un número primo.
\item ¿Para qué valores de $n\in\N$ se tiene que $g(n)=0$?
\end{itemize}
\begin{sol}
\begin{itemize}
\item[]
\item Dado que $g(n) = (f\ast \mu)(n)$ y $\mu$ es una función multiplicativa, basta probar que $f$ también lo es para demostrar que $g$ lo es (propiedades de la convolución). Sean $n,m\in \N$ coprimos y $q=f(mn)$. Sabemos que podemos escribir $q=p_1^{a_1},\dotsc,p_k^{a_k}$. Sea $p_i$, como $(n,m)=1$, entonces o bien $p_i \mid n$, o bien $p_i \mid m$, pero no puede dividir a ambos. En tal caso, podemos hacer una partición entre los factores primos que dividen a $n$ y los que dividen a $m$, de forma que podemos escribir $q=p_np_m$, cumpliéndose $(n,p_m)=1$ y $(m,p_n)=1$. Resta ver que $f(n)=p_n$ y $f(m)=p_m$, como queríamos probar.

Supongamos que $f(n)>p_n$. Sabemos que $p_n^2\mid n$, luego $p_n \mid f(n)$, por lo que $f(n)=p_n k$. Ya que $k^2 \mid n$, se tiene que $k \mid n$, $(p_m,k)=1$ y $(p_n k)^2 \mid n \mid mn$. Pero esto implica que $p_n k \mid q = p_n p_m$, de donde $k \mid p_m$, por lo que $k=1$.
\item Veamos los cálculos.
\begin{gather*}
g(p)=f(1)\mu(p) + f(p)\mu(1) = -1+1 = 0 \\
g(p^2) = f(1)\mu(p^2) + f(p)\mu(p) + f(p^2)\mu(1) = p-1\\
g(p^3) = f(1)\mu(p^3) + f(p)\mu(p^2) + f(p^2)\mu(p) + f(p^3)\mu(1) = -p+p=0
\end{gather*}
Es claro que para $n\geq 1$ se tiene
\begin{gather*}
g(p^n) = f(p^{n-1})\mu(p) + f(p^n)\mu(1) = f(p^n)-f(p^{n-1})\\
g(p^n) = 
\begin{cases}
p^{n/2}(p-1) & \text{si $n$ es par}\\
0 			& \text{si $n$ es impar}
\end{cases}
\end{gather*}
\item A partir del apartado anterior tenemos prácticamente la respuesta. $g(n)=0$ si $\exists p$ primo tal que $\nu_p(n)$ es impar, es decir, si y solo si $n$ no es un cuadrado perfecto.
\end{itemize}
\end{sol}
\end{ejercicio}

\newpage

\begin{ejercicio}{10}
Sea $f : [0,+∞) \to \C$ una función nula para $0 ≤ x < \varepsilon$, para cirto $\varepsilon > 0$.
Sea
\[ R(x) = f\left(\frac{x}{1}\right) - f\left(\frac{x}{3}\right) + f\left(\frac{x}{5}\right) - f\left(\frac{x}{7}\right) + f\left(\frac{x}{9}\right) - \cdots \]
Demostrar que $R(x)$ está bien definida y que
\[ f(x) = μ(1)R\left(\frac{x}{1}\right) - μ(3)R\left(\frac{x}{3}\right) + μ(5)R\left(\frac{x}{5}\right) - μ(7)R\left(\frac{x}{7}\right) + μ(9)R\left(\frac{x}{9}\right) - \cdots\]
\end{ejercicio}
\begin{solucion}
Podemos escribir
$$
R(x)=\sum_{n=1}^\infty (-1)^{n+1} f\left(\frac{x}{2n-1}\right)
$$
Sabemos que $\forall x >0$ $\exists n>0$ tal que $\frac{x}{2n-1}<\varepsilon$. Por tanto, sabemos que $\forall x\in[0,+\infty)$, $R(x)$ se puede escribir como una suma finita, por lo que está bien definida. Aplicamos ahora la segunda fórmula de inversión
\end{solucion}

\newpage
\begin{ejercicio}{11}
Probar la siguiente identidad	
\[
\sigma(n)^2 = \sum_{d\mid n} \frac{n}{d}\sigma(d^2) 
\]
\begin{sol}
Dado que $\sigma(n)$ es una función multiplicativa, sean $m,n$ coprimos, tenemos que
\[
\sigma(nm)^2 = (\sigma(n)\sigma(m))^2 = \sigma(n)^2\sigma(m)^2
\]
Por otra parte, la otra igualdad es claramente la segunda parte de la igualdad es la convolución de las funciones $f(x)=x$ (claramente multiplicativa) y $g(x)=\sigma(x^2)$. Esta, en principio, no parece tan claro que sea multiplicativa, pero veamos que sí lo es. Sean $n,m\in\N$ coprimos. Si $(n,m)=1$, entonces es claro que $(n^2,m^2)=1$, luego
$$
g(nm)=\sigma((nm)^2) = \sigma(n^2 m^2)= \sigma(n^2)\sigma(m^2)=g(n)g(m)$$
Por tanto, basta probar la identidad para los números de la forma $p^a$ con $p$ primo y $a\geq 1$ (el caso $n=1$ es trivial). Sea $n=p^a$,
\begin{gather*}
\sigma(n)^2 = \sigma(p^a)^2 = \left(\frac{p^{a+1}-1}{p-1}\right)^2\\  \sum_{d\mid n} \frac{n}{d}\sigma(d^2) = \sum_{k=0}^a \frac{p^a}{p^k}\sigma(p^{2k})= \sum_{k=0}^a p^{a-k}\frac{p^{2k+1}-1}{p-1} = \frac{p^a}{p-1} \sum_{k=0}^a p^{k+1}- p^{-k} = \\
 = \frac{p^a}{p-1} \left(\frac{p^{a+2}-p}{p-1}-\frac{p^{-(a+1)}-1}{p^{-1}-1}\right) = \frac{p^a}{(p-1)^2}(p^{a+2}-p+ p^{-a}-p) = \\
 = \frac{{p^{a+1}}^2-2p^{a+1}+1}{(p-1)^2} =  \left(\frac{p^{a+1}-1}{p-1}\right)^2
\end{gather*}
Como queríamos probar.
\end{sol}
\end{ejercicio}

\newpage

\begin{ejercicio}{12}
Demostrar que para $|x| < 1$ se tiene
\[ \sum_{n=1}^{∞} \frac{φ(n) x^n}{1-x^n} = \frac{x}{(1-x)^2} \]
\end{ejercicio}
\begin{solucion}
Si $|x|<1$ podemos escribir
$$
\sum_{n=1}^{∞} φ(n)\frac{ x^n}{1-x^n} = \sum_{n=1}^{∞} \left(φ(n)\sum_{k=1}^\infty x^{kn} \right) \overset{(1)}{=} \sum_{m=1}^\infty \left(x^m\sum_{n \mid m}\varphi(n)\right) = \sum_{m=1}^\infty m x^m \overset{(2)}{=}\frac{x}{(1-x)^2}
$$
Vamos a ver más claramente las ecuaciones señaladas.
Para $(1)$, basta fijar el exponente de $x$ y tomar todos los $n$ tales que existe una $k$ tal que $nk=m$, es decir, que $n \mid m$.

La igualdad $(2)$ sale a partir de la siguiente consideración. 
\begin{align*}
\frac{x}{1-x} = \sum_{m=1}^\infty x^m
\end{align*}
Como es una serie de potencias, podemos derivar término a término en $|x|<1$, obteniendo así:
\[ \frac{1}{(1-x)^2} = \sum_{m=1}^\infty mx^{m-1} \]
Multiplicando por $x$ en ambos lados obtenemos el resultado.
\end{solucion}

\newpage

\begin{ejercicio}{13}
Probar que	
\[
\sum_{d\mid n} (-1)^{n/d}\varphi(d) = 
\begin{cases}
0 & \text{si $n$ es par}\\ 
-n & \text{si $n$ es impar}
\end{cases}
\]
\begin{sol}
Sea $n \in \N$ impar, entonces $\forall d$ tal que $d \mid n$, $\frac{n}{d}$ también es impar, lo cual se traduce en que
\[
\sum_{d\mid n} (-1)^{n/d}\varphi(d) = \sum_{d\mid n} (-1)\varphi(d) = -\sum_{d\mid n}\varphi(d) = -n
\]
Sea $n$ par, de manera que $n=2^sm$ para un $m$ impar y $s ≥ 1$. Entonces:
\[ \sum_{d\mid n}(-1)^{n/d}φ(d) =  \sum_{2^s\mid d \mid n}(-1)^{n/d}φ(d) +  \sum_{2^s\nmid d \mid n}(-1)^{n/d}φ(d) = -\sum_{2^s\mid d \mid n}φ(d) + \sum_{2^s\nmid d \mid n}φ(d) \]

Por un lado, como todo $d$ tal que $2^s\mid d \mid n$ debe ser de la forma $2^se$ con $e$ impar:
\[ \sum_{d=2^se \mid n} φ(d) = \sum_{d=2^se \mid n} φ(2^se) = 2^{s-1}\sum_{d=2^se|n} φ(e) = 2^{s-1}m = \frac{n}{2}\]
Por otro lado, todos los divisores $d$ de $n$ que no son divididos por $2^s$ son precisamente todos los divisores de $n/2$, luego:
\[ \sum_{2^s\nmid d \mid n}φ(d) = \frac{n}{2} \]
Por lo tanto:
\[ \sum_{d|n} (-1)^{n/d}φ(d) = -\sum_{2^s\mid d\mid n}φ(d)+\sum_{2^s\nmid d\mid n}φ(d) = -\frac{n}{2}+\frac{n}{2}=0\]
\end{sol}

\newpage

\begin{ejercicio}{14}
Definimos el radical de un número natural $R(n)$ como $R(1) = 1$ y para $n > 1$ sea $R(n) = \prod_{p|n} p$; es decir, el producto de los primos que dividen a $n$. Demostrar que
\[ \sum_{d|n} μ(d) σ(d) = (-1)^{ω(n)} R(n) \]
donde $ω(n)$ denota el número de factores primos distintos de $n$.
\end{ejercicio}
\begin{solucion}
El resultado se da trivialmente para $n=1$. De aquí en adelante suponemos que $n>1$.
Supongamos que $n=p_1^{a_1}p_2^{a_2}\cdots p_k^{a_k}$. Tomamos $N=p_1p_2\cdots p_k$. Tenemos entonces que:
\[ \sum_{d|n} μ(d) σ(d) = \sum_{d|N} μ(d) σ(d) \]
y
\[ (-1)^{ω(n)} R(n) = (-1)^k N \]
pues $μ(d)=0$ para todo $d$ divisible por un cuadrado.
Observamos además que:
\[ \sum_{d|N} μ(d) σ(d) = (-1)^k\sum_{d|N} μ(N/d) σ(d) \]
Como $μ$ y $σ$ son multiplicativas, la convolución $μ * σ$ es multiplicativa. Podemos quitar $(-1)^{k}$ de ambos lados de la ecuación a demostrar. Basta comprobar la igualdad para $N=p$ (pues ya tenemos que $N$ es libre de cuadrados).
\[ \sum_{d|p} μ(p/d) σ(d) = -1+(1+p) = p \]
Lo que demuestra la fórmula.
\end{solucion}

\newpage

\begin{ejercicio}{15}
Probar que
\[ \prod_{d|n} d = n^{d(n)/2} \]
\end{ejercicio}

\begin{sol}
Como $d$ es un divisor de $n$ si y sólo si $n/d$ es divisor de $n$:
\[ \prod_{d|n} d = \prod_{d|n} n/d \]
Luego:
\[ \prod_{d|n} d = \sqrt{\left(\prod_{d|n} d\right)^2} = \sqrt{\prod_{d|n} d \cdot \prod_{d|n} n/d} = \sqrt{\prod_{d|n} n} = \sqrt{n^{d(n)}} = n^{d(n)/2} \]
\end{sol}
\end{ejercicio}
\end{document}