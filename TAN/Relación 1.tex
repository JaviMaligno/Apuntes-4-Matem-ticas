\documentclass[twoside]{article}
\usepackage{../estilo-ejercicios}

%--------------------------------------------------------
\begin{document}

\title{Algebra Conmutativa y Geometría Aplicada}
\author{Rafael González López, Diego Pedraza López}
\maketitle

\begin{ejercicio}{1}
Encontrar todos los enteros positivos tales que $\varphi(n)$ no sea divisible por 4.
\begin{sol}
Sea $n\in \N$. Sabemos que $n=p_1^{a_1}\cdots p_k^{a_k}$. En tal caso, 
\[
\varphi(n) = (p_1-1)p_1^{a_1-1}\cdots(p_k-1)p_k^{a_k-1}
\]
Obsérvese que si $n$ tiene más de un divisor primo impar, claramente se tendría que $4\mid \varphi(n)$, pues $p_j -1$ y $p_k-1$ serían dos factores pares de $\varphi(n)$. Por otra parte, si $p_i=2$, debe cumplire que $0\leq a_i \leq 2$. 

Por lo que sabemos hasta ahora, $n = 2^a p^b$ donde $a\in\{0,1,2\}$, $p$ es un primo impar y $b\geq 0$. Distingamos casos en función de $a$ y $b$
\begin{itemize}
\item Si $a \in \{0,1\}$ y $b = 0$, obtenemos $n=1$ y $n=2$, que cumplen que $4 \nmid φ(n)$.
\item Si $a \in \{0,1\}$ y $b > 0$, $\varphi(n) = (p-1)p^{b-1}$. La condición del enunciado se cumple si y solo sí $4 \mid (p-1)$, es decir, si $p \equiv 3 \pmod 4$.
\item Si $a=2$ y $b = 0$, obtenemos $n=4$, que cumple que $4 \nmid φ(4)$.
\item Si $a=2$ y $b > 0$, $\varphi(n)=2(p-1)p^{b-1}$, que es siempre divisible por 4.
\end{itemize}
Tenemos entonces que las soluciones son:
\[ \{1,2,4\} \cup \{p^k \mid p\text{ primo con } p \equiv 3 \pmod 4, k≥1\} \cup \{2p^k \mid p\text{ primo impar}, k ≥ 1\} \].
\end{sol}
\end{ejercicio}

\newpage


\begin{ejercicio}{2}
Probar que para todo $k\geq 0$
\[
\gcd \left\{\binom{2k}{k},\binom{2k+1}{k}\right\}=\binom{2k}{k}\frac{1}{k+1}, \quad \gcd \left\{\binom{2k+1}{k},\binom{2k+2}{k+1}\right\}=\binom{2k+1}{k+1}
\]
\begin{sol}
\end{sol}
\end{ejercicio}


\newpage


\begin{ejercicio}{3}
Probar que para todo $n$
\[
\sum_{d\mid n} \sigma(d)\mu(n/d)=n 
\]
\begin{sol}
Dado que las funciones $\sigma(n)$ y $\mu(n)$ son multiplicativas, su convolución también lo es. Por tanto, como $n\mapsto n$ también es multiplicativa, basta probar la igualdad para los números de la forma $p^a$ con $p$ primo y $a\geq 1$ (el caso $n=1$ es trivial). Sea pues $n=p^a$ con $p$ primo, entonces
\begin{gather*}
\sum_{d\mid n} \sigma(d)\mu(n/d) = \sum_{k=0}^a \sigma(p^k)\mu(p^{a-k}) = \sigma(p^{a-1})\mu(p) + \sigma(p^a)\mu(1) = \\
=  -\frac{p^a - 1}{p-1} +\frac{p^{a+1}-1}{p-1} = \frac{p^{a+1}-p^a}{p-1} = p^a
\end{gather*}
Tal y como queríamos ver. Hemos usado que $\mu(p^a)=0$ $\forall a \geq 2$.
\end{sol}
\end{ejercicio}
\newpage


\begin{ejercicio}{5}
\begin{itemize}
\item[]
\item Demostrar que 
\[
\sum_{d\mid n} \mu(d)^2 = 2^{\omega(n)}
\]
donde $\omega(n)=k$ si $n=p_1^{a_1}\cdots p_k^{a_k}$.
\item Sabiendo que
\[
\frac{\zeta^2(s)}{\zeta(2s)}=\sum_{n=1}^\infty \frac{2^{\omega(n)}}{n^s}
\]
calcular la suma de la serie de Dirichlet
$$
\sum_{n=1}^\infty \frac{\mu(n)^2}{n^s}
$$
\end{itemize}
\begin{sol}
\begin{itemize}
\item[]
\item Primeramente, observamos que $\sum_{d\mid n} \mu(d)^2$ es la convolución de las funciones $f\equiv 1$ y $\mu^2$, ambas multiplicativas, por lo que también es multiplicativa. Por tanto, solo tenemos que probar el resultado para los números de la forma $n=p^a$, $p$ primo y $a\geq 1$.
\[
\sum_{d\mid n} \mu(d)^2 = \sum_{n=0}^a \mu(p^n)^2 = \mu(1)^2 + \mu(p)^2 = 2^1 = 2^{\omega(n)}
\]
\item Basta realizar unas sencillas manipulaciones
\begin{gather*}
\frac{\zeta^2(s)}{\zeta(2s)}=\sum_{n=1}^\infty \frac{1}{n^s}\sum_{d\mid n} \mu(d)^2 = \left(\sum_{n=1}^\infty \frac{\mu(n)^2}{n^s}\right)\left(\sum_{n=1}^\infty \frac{1}{n^s}\right) = \zeta(s)\sum_{n=1}^\infty \frac{\mu(n)^2}{n^s}\\
\sum_{n=1}^\infty \frac{\mu(n)^2}{n^s} = \frac{\zeta(s)}{\zeta(2s)}
\end{gather*}
Alternativamente:
\[ \frac{ζ(s)}{ζ(2s)} = \prod_p \frac{1-1/p^{2s}}{1-1/p^s} = \prod_p \frac{(1-1/p^s)(1+1/p^s)}{1-1/p^s} = \prod_p (1+1/p^s) = \sum_{n=1}^{∞} \frac{μ(n)^2}{n^s} \]
\end{itemize}
\end{sol}
\end{ejercicio}
\newpage

\begin{ejercicio}{9}
Sea $f(n)$ el mayor número natural $m$ tal que $m^2\mid n$. definimos
\[
g(n) = \sum_{d\mid n} f(d)\mu(n/d)
\]
\begin{itemize}
\item Demostrar que $g$ es una función multiplicativa.
\item Calcular $g(p)$, $g(p^2)$ y $g(p^3)$, siendo $p$ un número primo.
\item ¿Para qué valores de $n\in\N$ se tiene que $g(n)=0$?
\end{itemize}
\begin{sol}
\begin{itemize}
\item[]
\item Dado que $g(n) = (f\ast \mu)(n)$ y $\mu$ es una función multiplicativa, basta probar que $f$ también lo es para demostrar que $g$ lo es (propiedades de la convolución). Sean $n,m\in \N$ coprimos y $q=f(mn)$. Sabemos que podemos escribir $q=p_1^{a_1},\dotsc,p_k^{a_k}$. Sea $p_i$, como $(n,m)=1$, entonces o bien $p_i \mid n$, o bien $p_i \mid m$, pero no puede dividir a ambos. En tal caso, podemos hacer una partición entre los factores primos que dividen a $n$ y los que dividen a $m$, de forma que podemos escribir $q=p_np_m$, cumpliéndose $(n,p_m)=1$ y $(m,p_n)=1$. Resta ver que $f(n)=p_n$ y $f(m)=p_m$, como queríamos probar.

Supongamos que $f(n)>p_n$. Sabemos que $p_n^2\mid n$, luego $p_n \mid f(n)$, por lo que $f(n)=p_n k$. Ya que $k^2 \mid n$, se tiene que $k \mid n$, $(p_m,k)=1$ y $(p_n k)^2 \mid n \mid mn$. Pero esto implica que $p_n k \mid q = p_n p_m$, de donde $k \mid p_m$, por lo que $k=1$.
\item Veamos los cálculos.
\begin{gather*}
g(p)=f(1)\mu(p) + f(p)\mu(1) = -1+1 = 0 \\
g(p^2) = f(1)\mu(p^2) + f(p)\mu(p) + f(p^2)\mu(1) = p-1\\
g(p^3) = f(1)\mu(p^3) + f(p)\mu(p^2) + f(p^2)\mu(p) + f(p^3)\mu(1) = -p+p=0
\end{gather*}
Es claro que para $n\geq 1$ se tiene
\begin{gather*}
g(p^n) = f(p^{n-1})\mu(p) + f(p^n)\mu(1) = f(p^n)-f(p^{n-1})\\
g(p^n) = 
\begin{cases}
p^{n/2}(p-1) & \text{si $n$ es par}\\
0 			& \text{si $n$ es impar}
\end{cases}
\end{gather*}
\item A partir del apartado anterior tenemos prácticamente la respuesta. $g(n)=0$ si $\exists p$ primo tal que $\nu_p(n)$ es impar, es decir, si y solo si $n$ no es un cuadrado perfecto.
\end{itemize}
\end{sol}
\end{ejercicio}

\newpage


\begin{ejercicio}{11}
Probar la siguiente identidad	
\[
\sigma(n)^2 = \sum_{d\mid n} \frac{n}{d}\sigma(d^2) 
\]
\begin{sol}
Dado que $\sigma(n)$ es una función multiplicativa, sean $m,n$ coprimos, tenemos que
\[
\sigma(nm)^2 = (\sigma(n)\sigma(m))^2 = \sigma(n)^2\sigma(m)^2
\]
Por otra parte, la otra igualdad es claramente la segunda parte de la igualdad es la convolución de las funciones $f(x)=x$ (claramente multiplicativa) y $g(x)=\sigma(x^2)$. Esta, en principio, no parece tan claro que sea multiplicativa, pero veamos que sí lo es. Sean $n,m\in\N$ coprimos. Si $(n,m)=1$, entonces es claro que $(n^2,m^2)=1$, luego
$$
g(nm)=\sigma((nm)^2) = \sigma(n^2 m^2)= \sigma(n^2)\sigma(m^2)=g(n)g(m)$$
Por tanto, basta probar la identidad para los números de la forma $p^a$ con $p$ primo y $a\geq 1$ (el caso $n=1$ es trivial). Sea $n=p^a$,
\begin{gather*}
\sigma(n)^2 = \sigma(p^a)^2 = \left(\frac{p^{a+1}-1}{p-1}\right)^2\\  \sum_{d\mid n} \frac{n}{d}\sigma(d^2) = \sum_{k=0}^a \frac{p^a}{p^k}\sigma(p^{2k})= \sum_{k=0}^a p^{a-k}\frac{p^{2k+1}-1}{p-1} = \frac{p^a}{p-1} \sum_{k=0}^a p^{k+1}- p^{-k} = \\
 = \frac{p^a}{p-1} \left(\frac{p^{a+2}-p}{p-1}-\frac{p^{-(a+1)}-1}{p^{-1}-1}\right) = \frac{p^a}{(p-1)^2}(p^{a+2}-p+ p^{-a}-p) = \\
 = \frac{{p^{a+1}}^2-2p^{a+1}+1}{(p-1)^2} =  \left(\frac{p^{a+1}-1}{p-1}\right)^2
\end{gather*}
Como queríamos probar.
\end{sol}
\end{ejercicio}


\newpage


\begin{ejercicio}{13}
Probar que	
\[
\sum_{d\mid n} (-1)^{n/d}\varphi(d) = 
\begin{cases}
0 & \text{si $n$ es par}\\ 
-n & \text{si $n$ es impar}
\end{cases}
\]
\begin{sol}
Sea $n \in \N$ impar, entonces $\forall d$ tal que $d \mid n$, $\frac{n}{d}$ también es impar, lo cual se traduce en que
\[
\sum_{d\mid n} (-1)^{n/d}\varphi(d) = \sum_{d\mid n} (-1)\varphi(d) = -\sum_{d\mid n}\varphi(d) = -n
\]
Sea $n$ par, de manera que $n=2^sm$ para un $m$ impar y $s ≥ 1$. Entonces:
\[ \sum_{d\mid n}(-1)^{n/d}φ(d) =  \sum_{2^s\mid d \mid n}(-1)^{n/d}φ(d) +  \sum_{2^s\nmid d \mid n}(-1)^{n/d}φ(d) = -\sum_{2^s\mid d \mid n}φ(d) + \sum_{2^s\nmid d \mid n}φ(d) \]

Por un lado, como todo $d$ tal que $2^s\mid d \mid n$ debe ser de la forma $2^se$ con $e$ impar:
\[ \sum_{d=2^se \mid n} φ(d) = \sum_{d=2^se \mid n} φ(2^se) = 2^{s-1}\sum_{d=2^se|n} φ(e) = 2^{s-1}m = \frac{n}{2}\]
Por otro lado, todos los divisores $d$ de $n$ que no son divididos por $2^s$ son precisamente todos los divisores de $n/2$, luego:
\[ \sum_{2^s\nmid d \mid n}φ(d) = \frac{n}{2} \]
Por lo tanto:
\[ \sum_{d|n} (-1)^{n/d}φ(d) = -\sum_{2^s\mid d\mid n}φ(d)+\sum_{2^s\nmid d\mid n}φ(d) = -\frac{n}{2}+\frac{n}{2}=0\]
\end{sol}

\newpage

\begin{ejercicio}{15}
Probar que
\[ \prod_{d|n} d = n^{d(n)/2} \]
\end{ejercicio}

\begin{sol}
Como $d$ es un divisor de $n$ si y sólo si $n/d$ es divisor de $n$:
\[ \prod_{d|n} d = \prod_{d|n} n/d \]
Luego:
\[ \prod_{d|n} d = \sqrt{\left(\prod_{d|n} d\right)^2} = \sqrt{\prod_{d|n} d \cdot \prod_{d|n} n/d} = \sqrt{\prod_{d|n} n} = \sqrt{n^{d(n)}} = n^{d(n)/2} \]
\end{sol}
\end{ejercicio}

\[ \sum_{n=1}^{∞} \frac{μ(n)}{\sinh (nx)} = 2 \sum_{n=1}^{∞} \frac{μ(n)}{e^{nx} - e^{-nx}} = 2 \sum_{n=1}^{∞} \frac{μ(n)}{e^{nx}} \frac{1}{1-e^{-2nx}} = 2 \sum_{n=1}^{∞} μ(n) e^{-nx}(1+e^{-2nx}+e^{-4nx}+\dots) \]
\[ = 2 \sum_{n=1}^{∞} μ(n) \sum_{k=1}^{∞} e^{-(2k-1)nx} = 2 \sum_{n=1}^{∞} \sum_{k=1}^{∞} μ(n) e^{-(2k-1)nx}\]
ya que estas series son absolutamente convergente (compruébalo):
\[ \sum_{n=1}^{∞} \frac{μ(n)}{\sinh (nx)} = 2 \sum_{N=1}^{∞} e^{-Nx} \left(\sum_{n|N \text{ y }N/n\text{ es impar}} μ(n) \right)\]
Tenemos que si $N$ es impar:
\[ \sum_{\substack{n|N \\ N/n\text{ es impar}}} μ(n) = \begin{cases}
	1, &\text{ si }N = 1\\
	0, &\text{ si }N > 1
\end{cases}\]
Si $N$ es par y $N=2^am$:
\[ \sum_{2^ak|2^am} μ(n) = \sum_{k|m} μ(2^a k) = \sum_{k|m} μ(2^a) μ(k) = \begin{cases}
	0, &\text{ si }a ≥ 2\\
	-1 &\text{ si }m=1\\
	0 &\text{ c.c.}
\end{cases}\]
Luego:
\[ \sum_{n=1}^{∞} \frac{μ(n)}{\sinh (nx)} = 2(e^{-x}-e^{-2x})\]
\end{document}