\documentclass[twoside]{article}
\usepackage{../estilo-ejercicios}
\providecommand{\bo}[1]{\mathcal{O}\left(#1\right)}
%--------------------------------------------------------
\begin{document}

\title{Teoría Analítica de Números}
\author{Rafael González López\\Diego Pedraza López}
\maketitle

\begin{ejercicio}{1}
\begin{enumerate}
	\item Demostrar que para $\Re s > 1$
	\[ -\frac{ζ'(s)}{ζ(s)} = s \int_1^{+∞} \frac{ψ(x)}{x^{s+1}} dx \]
	\item Probar que para $\Re s > 1$ se tiene
	\[ -\frac{ζ'(s)}{ζ(s)} = \frac{s}{s-1} + s \int_1^{+∞} \frac{ψ(x)-x}{x^{s+1}}dx \]
	\item Demostrar que si $ψ(x) = x + \mathcal{O}(x^a)$ con $a > 1/2$, entonces $ζ(s)$ no se anula en $\Re s > a$.
\end{enumerate}
\end{ejercicio}
\begin{solucion}
\begin{enumerate}
	\item Para $s=σ+it$ con $σ>1$, tenemos que:
\[ - \frac{ζ'(s)}{ζ(s)} = \int_{n=1}^{∞} \frac{Λ(n)}{n^s} \]
Por la fórmula de Abel:
\[ \sum_{n≤x} \frac{Λ(n)}{n^s} = \frac{ψ(x)}{x^s} + s \int_1^{x} \frac{ψ(u)}{u^{s+1}} du \]
Tomando el límite cuando $x\to∞$ se da la fórmula que buscabamos si $\frac{ψ(x)}{x^s}$ tiende a $0$.
\[ \lim_{x \to ∞} \left|\frac{ψ(x)}{x^s}\right| ≤ \lim_{x \to ∞} \frac{Ax}{x^σ} = 0\]
	\item Sumando y restando $s\int_1^{+∞} \frac{x}{x^{s+1}}dx$ en el resultado anterior:
	\[ -\frac{ζ'(s)}{ζ(s)} = s\int_1^{+∞} \frac{x}{x^{s+1}} + s \int_1^{+∞} \frac{ψ(x)-x}{x^{s+1}} dx = \frac{s}{1-s} + s \int_1^{+∞} \frac{ψ(x)-x}{x^{s+1}} dx \]
	\item Sea $f(s) = \int_1^{+∞} \frac{ψ(x)-x}{x^{s+1}} dx$. Veamos que $f$ es analítica aplicando el teorema de convergencia dominada. Tenemos que usando que $ψ(x) = x+\mathcal{O}(x^a)$:
	\[ \left|\frac{ψ(x)-x}{x^{s+1}}\right| ≤ \frac{cx^a}{x^{σ+1}} ≤ \frac{cx^a}{x^{a'+1}} = \frac{c}{x^{a'-a+1}} \]
	con $σ > a' > a$. Como $a'-a+1 > 1$, esta fración es integrable en $[1,∞)$. Las otras condiciones del teorema de convergencia dominada son claras, luego $f(s)$ es analítica. En consecuencia, $-\frac{-ζ'(s)}{ζ(s)} = \frac{s}{s-1}+f(s)$ es meromorfa con un polo en $1$.
\end{enumerate}
\end{solucion}

\newpage
\begin{ejercicio}{2}
\begin{enumerate}[(a)]
	\item Probar que para todo $x > 0$ y todo $s$ con $\Re s > 0$ se tiene
	\[ ζ(s) - \frac{1}{s-1} = \sum_{n≤x} \frac{1}{n^s} + \frac{x^{1-s}-1}{s-1} + \frac{\{x\}-\frac{1}{2}}{x^s} - s \int_x^{+∞} \frac{\{u\}-\frac{1}{2}}{u^{s+1}} du \]
	\item Demostrar que existe y calcular el límite
	\[ \lim_{s \to 1} \left(\sum_{n≤x} \frac{1}{n^s} + \frac{x^{1-s}-1}{s-1} + \frac{\{x\}-\frac{1}{2}}{x^s} - s \int_x^{+∞} \frac{\{u\}-\frac{1}{2}}{u^{s+1}} du\right) \]
	\item Demostrar que
	\[ ζ(s) = \frac{1}{s-1} + γ + \sum_{n=1}^{∞} \frac{(-1)^n}{n!} γ_n (s-1)^n \]
	donde la serie de potencias tiene radio infinito, y $γ$ es la constante de Euler.
\end{enumerate}
\end{ejercicio}
\begin{solucion}
\begin{enumerate}[(a)]
	\item Consecuencia directa de los apuntes.
	\item Veamos que ocurre en cada término:
	\[ \lim_{s \to 1} \sum_{n≤x} \frac{1}{n^s} = \sum_{n≤x} \frac{1}{n} \]
	\[ \lim_{s \to 1} \frac{x^{1-s}-1}{s-1} = - \log x \]
	\[ \lim_{s \to 1} \frac{\{x\}-\frac{1}{2}}{x^s} = \frac{\{x\}-\frac{1}{2}}{x} \]
	\[ \lim_{s \to 1} s \int_x^{∞} \frac{\{u\}-\frac{1}{2}}{u^{s+1}} du = \int_x^{∞} \frac{\{u\}-\frac{1}{2}}{u^2} du \]
	Por lo tanto el límite es:
	\[ \sum_{n≤x}\frac{1}{n} - \log x + \frac{\{x\}-\frac{1}{2}}{x} - \int_x^{∞} \frac{\{u\}-\frac{1}{2}}{u^2} du \]
	\item Se tiene que $ζ(s)-\frac{1}{s-1}$ es analítica en todo el plano complejo. Tomamos como su desarrollo en serie con centro en $1$. $a_n=\frac{(-1)^n}{n!} γ_n$ para $n ≥ 1$. El valor de $a_0$ será igual al resultado del anterior apartado. Como no depende de $x$, podemos límite cuando $x$ tiende a infinito y obtenemos:
	\[ ζ(s) - \frac{1}{s-1} = γ + \sum_{n=1}^{∞} \frac{(-1)^n}{n!} γ_n (s-1)^n \]
\end{enumerate}
\end{solucion}
\newpage

\begin{ejercicio}{3}
\begin{enumerate}
	\item Demostrar que para $x > 1$ $\sum_{n=1}^{∞} e^{-πn^2 x} ≤ \frac{21}{20} e^{-πx}$.
	\item Demostrar que para $0 < σ < 1$ se tiene $π^{-\frac{σ}{2}}Γ(σ/2) ζ(ο) ≤ -1.9$.
	\item Deducir que $ζ(σ) < 0$ para $0 < σ < 1$.
	\item La función zeta no se anula en el segmento $[0,1]$.
\end{enumerate}
\end{ejercicio}
\begin{solucion}
\end{solucion}
\newpage

\begin{ejercicio}{4}
\begin{itemize}
\item[]
\item[a)] Demostrar que
$$
-\frac{\Gamma'(1+s)}{\Gamma(1+s)} = \gamma + \sum_{n=1}^\infty (-1)^n\zeta(n+1)s^n
$$
\item[b)] Deducir que
$$
x\cot x = 1 - 2\sum_{n=1}^\infty\zeta(2n)\left(\frac{x}{\pi}\right)^{2n}
$$
\end{itemize}
\end{ejercicio}
\begin{sol}
\begin{itemize}
\item[]
\item[a)] Sabemos por teoría que podemos expresar $\gamma$ es una función meromorfa definida en todo el plano complejo salvo los enteros no positivos, sin ceros y que verifica en este dominio la expresión:
\[ \frac{1}{Γ(s)} = s e^{γs} \prod_{n=1}^{∞} \left(1 + \frac{s}{n}\right) e^{-s/n} \]
Cambiando de parámetro $s=z+1$ y renombrando $z$ como $s$, por lo anterior podemos considerar la siguiente derivada logarítica en todo el plano complejo salvo los enteros negativos
\begin{align*}
\frac{d}{ds}\left(-\log \Gamma(s+1)\right) &=  \frac{d}{ds}\log\left((s+1) e^{γ(s+1)} \prod_{n=1}^{∞} \left(1 + \frac{(s+1)}{n}\right) e^{-(s+1)/n}\right)\\
&= \frac{d}{ds}\left( \log(s+1)+\gamma(s+1)+\sum_{n=1}^\infty \log\left(1+\frac{s+1}{n}\right)-\frac{s+1}{n}\right)\\
&= \frac{1}{s+1} +\gamma + \sum_{n=1}^\infty \left(\frac{1}{n+s+1}-\frac{1}{n}\right)= -\frac{\Gamma'(s+1)}{\Gamma(s+1)}
\end{align*}
Pasamos a denotar $\psi(s) = \dfrac{\Gamma'(s)}{\Gamma(s)}$. Por tanto, hemos llegado a la expresiones
\begin{align*}
-\psi(s+1) &= \gamma+\frac{1}{s+1} + \sum_{n=1}^\infty \left(\frac{1}{n+s+1}-\frac{1}{n}\right)\\
& = \gamma -\sum_{n=1}^\infty \left(\frac{1}{n} - \frac{1}{n+s}\right) = \gamma -\sum_{n=1}^\infty \left(\frac{s}{n(n+s)}\right) 
\end{align*}
Donde $h(s)=\sum_{n=1}^\infty \left(\frac{s}{n(n+s)}\right)$ define una función meromorfa definida en todo $\C$ salvo los enteros negativos, ya que la serie converge trivialmente en ese dominio. Vamos a estudiar la serie de Taylor de $h(s)$.
\begin{align*}
h'(s) & = \sum_{n=1}^\infty\frac{1}{(n+s)^2} & h''(s)&=  -2\sum_{n=1}^\infty\frac{1}{(n+s)^3}\\
h'''(s) &= 2 \cdot 3 \, \sum_{n=1}^\infty\frac{1}{(n+s)^4} &  h^{4)}(s) &= -2 \cdot 3 \cdot 4 \,\sum_{n=1}^\infty\frac{1}{(n+s)^5}
\end{align*}
Continuando podemos probar por inducción de manera trivial que
$$h^{n)}(s) = (-1)^{n+1}n!\,\sum_{k=1}^\infty\frac{1}{(k+s)^{n+1}}$$
Por tanto, como $h$ es analítica en $0$ y $h(0)=0$, podemos escribir
$$
h(s)=\sum_{n=0}^\infty \frac{h^{n)}(0)}{n!}s^n = \sum_{n=1}^\infty\sum_{k=1}^\infty\frac{(-1)^{n+1}}{k^{n+1}}  s^n = \sum_{n=1}^\infty(-1)^{n+1}\sum_{k=1}^\infty\frac{1}{k^{n+1}}  s^n 
$$
Utilizando lo que hemos visto anteriormente tenemos en $|s|<1$ la siguiente expresión
$$
-\psi(s+1) = \gamma -  \sum_{n=1}^\infty(-1)^{n+1}\sum_{k=1}^\infty\frac{1}{k^{n+1}}  s^n = \gamma + \sum_{n=1}^\infty(-1)^{n}\zeta(n+1) s^n
$$
Como queríamos demostrar.
\newpage
\item[b)] Sabemos por teoría que si $s\notin \Z$
$$\Gamma(1+s)=s\Gamma(s)  \qquad \frac{1}{s}\Gamma(1+s)\Gamma(1-s)= 	\Gamma(s)\Gamma(1-s)=\frac{\pi}{\sin(\pi s)} \qquad 
$$
Utilizando la derivada logarítmica aplicando a $s/\pi$ vamos a deducir lo siguiente:
\begin{align*}
\left(\frac{\pi}{s}\Gamma\left(1+\frac{s}{\pi}\right)\Gamma\left(1-\frac{s}{\pi}\right)\right)' & = \frac{1}{\pi}\frac{\Gamma'\left(1+\frac{s}{\pi}\right)}{\Gamma\left(1+\frac{s}{\pi}\right)} - \frac{1}{\pi}\frac{\Gamma'(1-\frac{1}{\pi})}{\Gamma(1-\frac{1}{\pi})} - \frac{1}{s}\\
 &= \frac{1}{\pi}\psi(1+s/\pi)-\frac{1}{\pi}\psi(1-s/\pi) - \frac{1}{s}\\
&=\left(\log\frac{\pi}{\sin(s)}\right)'  = -\cot(s)
\end{align*}
Si $0<|s|<1$ podemos utilizar la expresión obtenida en el apartado anterior
\begin{align*}
\frac{1}{\pi}\left(\psi(1+s/\pi)-\psi(1-s/\pi)\right) &=  -\frac{1}{\pi}\left(\gamma \sum_{n=1}^\infty(-1)^{n}\zeta(n+1) \left(\frac{s}{\pi}\right)^n -  \gamma - \sum_{n=1}^\infty\zeta(n+1) \left(\frac{s}{\pi}\right)^n\right)\\
&=2\sum_{n=1}^\infty \zeta(2n)\frac{s^{2n-1}}{\pi^{2n}}\\
&=-\cot(s)+\frac{1}{s}
\end{align*}
Si ahora multiplicamos por $s$ a ambos lados
\begin{gather*}
2\sum_{n=1}^\infty \zeta(2n)\left(\frac{s}{\pi}\right)^{2n}=-s\cot(s)+1 
\end{gather*}
De donde basta una simple manipulación algebraica para obtener
$$
1-2\sum_{n=1}^\infty \zeta(2n)\left(\frac{s}{\pi}\right)^{2n}=s\cot(s)
$$
Una fórmula que puede extenderse a todo $|s|<1$, pues $s\cot(s)\to 1$ cuando $s\to 0$.
\end{itemize}
\end{sol}
\newpage

\begin{ejercicio}{7}
\begin{itemize}
\item[]
\item[a)] Sea $0<\theta<1$. Demostrar que existe una constante $A$ tal que para $t>1$ u $\sigma>\theta$ se tiene
$$
|\zeta(\sigma +it)|\leq A\frac{t^{1-\theta}}{\theta(1-\theta)}
$$
\item[b)] Demostrar que existe una constante positiva $B$ tal que para $t>e$ se tiene
$$
|\zeta(1+it)|\leq B\log t
$$
\end{itemize}
\end{ejercicio}
\end{document}