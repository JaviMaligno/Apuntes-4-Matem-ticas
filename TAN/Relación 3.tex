\documentclass[twoside]{article}
\usepackage{../estilo-ejercicios}
\providecommand{\bo}[1]{\mathcal{O}\left(#1\right)}
%--------------------------------------------------------
\begin{document}

\title{Teoría Analítica de Números}
\author{Rafael González López\\Diego Pedraza López}
\maketitle

\begin{ejercicio}{4}
\begin{itemize}
\item[]
\item[a)] Demostrar que
$$
-\frac{\Gamma'(1+s)}{\Gamma(1+s)} = \gamma + \sum_{n=1}^\infty (-1)^n\zeta(n+1)s^n
$$
\item[b)] Deducir que
$$
x\cot x = 1 - 2\sum_{n=1}^\infty\zeta(2n)\left(\frac{x}{\pi}\right)^{2n}
$$
\end{itemize}
\end{ejercicio}
\begin{sol}
\begin{itemize}
\item[]
\item[a)] Sabemos por teoría que podemos expresar $\gamma$ es una función meromorfa definida en todo el plano complejo salvo los enteros no positivos, sin ceros y que verifica en este dominio la expresión:
\[ \frac{1}{Γ(s)} = s e^{γs} \prod_{n=1}^{∞} \left(1 + \frac{s}{n}\right) e^{-s/n} \]
Cambiando de parámetro $s=z+1$ y renombrando $z$ como $s$, por lo anterior podemos considerar la siguiente derivada logarítica en todo el plano complejo salvo los enteros negativos
\begin{align*}
\frac{d}{ds}\left(-\log \Gamma(s+1)\right) &=  \frac{d}{ds}\log\left((s+1) e^{γ(s+1)} \prod_{n=1}^{∞} \left(1 + \frac{(s+1)}{n}\right) e^{-(s+1)/n}\right)\\
&= \frac{d}{ds}\left( \log(s+1)+\gamma(s+1)+\sum_{n=1}^\infty \log\left(1+\frac{s+1}{n}\right)-\frac{s+1}{n}\right)\\
&= \frac{1}{s+1} +\gamma + \sum_{n=1}^\infty \left(\frac{1}{n+s+1}-\frac{1}{n}\right) = -\frac{\Gamma'(s+1)}{\Gamma(s+1)}
\end{align*}
Pasamos a denotar $\psi(s) = \dfrac{\Gamma'(s)}{\Gamma(s)}$. Por tanto, hemos llegado a la expresiones
\begin{align*}
-\psi(s+1) &= \gamma+\frac{1}{s+1} + \sum_{n=1}^\infty \left(\frac{1}{n+s+1}-\frac{1}{n}\right)\\
& = \gamma -\sum_{n=1}^\infty \left(\frac{1}{n} - \frac{1}{n+s}\right) = \gamma -\sum_{n=1}^\infty \left(\frac{s}{n(n+s)}\right) 
\end{align*}
Donde $h(s)=\sum_{n=1}^\infty \left(\frac{s}{n(n+s)}\right)$ define una función meromorfa definida en todo $\C$ salvo los enteros negativos, ya que la serie converge trivialmente en ese dominio. Vamos a estudiar la serie de Taylor de $h(s)$.
\begin{align*}
h'(s) & = \sum_{n=1}^\infty\frac{1}{(n+s)^2} & h''(s)&=  -2\sum_{n=1}^\infty\frac{1}{(n+s)^3}\\
h'''(s) &= 2 \cdot 3 \, \sum_{n=1}^\infty\frac{1}{(n+s)^4} &  h^{4)}(s) &= -2 \cdot 3 \cdot 4 \,\sum_{n=1}^\infty\frac{1}{(n+s)^5}
\end{align*}
Continuando podemos probar por inducción de manera trivial que
$$h^{n)}(s) = (-1)^{n+1}n!\,\sum_{k=1}^\infty\frac{1}{(k+s)^{n+1}}$$
Por tanto, como $h$ es analítica en $0$ y $h(0)=0$, podemos escribir
$$
h(s)=\sum_{n=0}^\infty \frac{h^{n)}(0)}{n!}s^n = \sum_{n=1}^\infty\sum_{k=1}^\infty\frac{(-1)^{n+1}}{k^{n+1}}  s^n = \sum_{n=1}^\infty(-1)^{n+1}\sum_{k=1}^\infty\frac{1}{k^{n+1}}  s^n 
$$
Utilizando lo que hemos visto anteriormente tenemos en $|s|<1$ la siguiente expresión
$$
-\psi(s+1) = \gamma -  \sum_{n=1}^\infty(-1)^{n+1}\sum_{k=1}^\infty\frac{1}{k^{n+1}}  s^n = \gamma + \sum_{n=1}^\infty(-1)^{n}\zeta(n+1) s^n
$$
Como queríamos demostrar.
\newpage
\item[b)] Sabemos por teoría que 
$$\Gamma(s)\Gamma(1-s)=\frac{\pi}{\sin(\pi s)}
$$
Utilizando la derivada logarítmica vamos a deducir lo siguiente:
\begin{align*}
(\log{\Gamma(s)\Gamma(1-s))}' & = \frac{\Gamma'(s)}{\Gamma(s)} - \frac{\Gamma'(1-s)}{\Gamma(1-s)} = \psi(s)-\psi(1-s) = \\
&=\left(\log\frac{\pi}{\sin(\pi s)}\right)'  = -\pi\cot(\pi s)
\end{align*}
$$
\psi(1-s)-\psi(s) = \pi\cot(\pi s) \Longrightarrow s\psi(1-s)-s\psi(s) = s\pi\cot(\pi s)
$$
Vamos a analizar la función $q(s) = s\psi(1-s)-s\psi(s)$. Obviamente es una función meromorfa definida en el plano complejo salvo en los enteros. Además, por igualdad anterior, deducimos que $q(s)$ es impar, luego en sus series de Taylor tiene los coeficientes impares nulos.
\begin{align*}
s\psi(1-s)-s\psi(s) & = \gamma + sh(-s) -\gamma - sh(s-1)\\
&=s\sum_{n=1}^\infty(-1)^{n}\zeta(n+1) (-s)^n - s\sum_{n=1}^\infty(-1)^{n}\zeta(n+1) (s-1)^n
\end{align*}
\end{itemize}
\end{sol}
\newpage

\begin{ejercicio}{7}
\begin{itemize}
\item[]
\item[a)] Sea $0<\theta<1$. Demostrar que existe una constante $A$ tal que para $t>1$ u $\sigma>\theta$ se tiene
$$
|\zeta(\sigma +it)|\leq A\frac{t^{1-\theta}}{\theta(1-\theta)}
$$
\item[b)] Demostrar que existe una constante positiva $B$ tal que para $t>e$ se tiene
$$
|\zeta(1+it)|\leq B\log t
$$
\end{itemize}
\end{ejercicio}
\end{document}