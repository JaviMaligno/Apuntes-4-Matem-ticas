\documentclass[TAN.tex]{subfiles}
\begin{document}

\chapter{Funciones aritméticas}
\section{Divisibilidad}
Consideramos conocidos los conjuntos:
\[ \N = \{1,2,3,4,\dots\} \text{ números naturales}\]
\[ \Z = \{\dots,-3,-2,-1,0,1,2,3,\dots,\} \text{ números enteros}\]
y las operaciones de suma y producto definidas en ellos con las propiedades usuales. El conjunto $\Z$, dotado con las operaciones usuales es un \textbf{anillo conmutativo}. El anillo $\Z$ es un \textbf{dominio de integridad}.

\section{Las funciones $d(n)$ y $σ(n)$}

\begin{prop} Si $f$ y $g$ son multiplicativas la función $f * g$ definida por
\[ f * g (n) = \sum_{d|n} f(d)g(n/d) \]
es también multiplicativa

\begin{dem}
\[ f*g(nm) = \sum_{c|nm}f(c)g\left(\frac{nm}{c}\right) = \sum_{a|n,b|m}f(ab)g\left(\frac{nm}{ab}\right) \]
Usando que $f$ y $g$ son multiplicativas:
\begin{align*}
	f*g(nm) & = \sum_{a|n,b|m} f(a)f(b)g(n/a)g(m/b)  = \sum_{a|n}f(a)g(n/a) \sum_{b|m}f(b)g(m/b) \\
	& = (f*g)(m) \cdot (f*g)(n)
\end{align*}
\QED
\end{dem}
\end{prop}

\section{Las funciones $φ(n)$ de Euler y $μ(n)$ de Möbius}
\begin{prop}\mbox{}
\begin{enumerate}[(a)]
	\item La funcion φ es multiplicativa
	\item $φ(n) = n \displaystyle\prod_{p|n} \left(1-\dfrac{1}{p}\right)$
\end{enumerate}
\end{prop}

\begin{dem}
Usando que $φ$ es multiplicativa:
\[ φ(p^a) = p^a - p^{a-1} = p^a(1-1/p) \]
Sea $n = p_1^{a_1}\cdots p_k^{a_k}$:
\[ φ(n) = φ(p_1^{a_1})\cdots φ(p_k^{a_k}) = p_1^{a_1}(1-1/p_1)\cdots p_k^{a_k}(1-1/p_k) = n \prod_{p|n} \left(1-\dfrac{1}{p}\right)\]
\QED
\end{dem}

\begin{prop}
Para todo $n \in \N$ se tiene $\displaystyle\sum_{d|n} φ(d) = n$.
\end{prop}

\begin{dem}
Sea $f(n) = \sum_{d|n} φ(d)$ y $g(n) = n$. Sea $h(n) = 1$. Obsérvese que $f = φ * h$. Como $φ$ y $h$ son multiplicativas, $f$ es multiplicativa. Como $g$ también es mulitplicativa, para probar que $f = g$ basta ver que $f(p^a)=g(p^a)$ para un $p$ primo y $a ≥ 1$.

\[ f(p^a) = \sum_{b|p^a} φ(b) = \sum_{k=0}^a φ(p^k) = 1+(p-1)+(p²-p)+\cdots+(p^a-p^{a-1}) = p^a = g(p^a) \]

Luego $f = g$.
\qed

Como demostración alternativa:
\[ n = \left|\left\{\frac{a}{n} \mid 1 ≤ a ≤ n\right\}\right|
= \left|\bigcup_{b|n} \left\{\frac{a}{b} \mid 1≤a≤b, a \perp b\right\}\right|
= \sum_{b|n} \left|\left\{\frac{a}{b} \mid 1 ≤ a ≤ b, a \perp b\right\}\right|
= \sum_{b|n} φ(b) \]\QED
\end{dem}


\begin{prop}
\[ \sum_{a|n} μ(a) = \begin{cases}
	1 & \text{ si } n = 1\\
	0 & \text{ si } n > 1
\end{cases}\]
\end{prop}

\begin{prop}
\begin{enumerate}[(A)]
	\item $\forall n$, $g(n) = \sum_{a|n} f(a)$
	\item $\forall n$, $f(n) = \sum_{a|n}μ(a) g(n/a)$
\end{enumerate}

\end{prop}
\begin{dem}\mbox{}
\begin{itemize}
	\item[[$(A) \Rightarrow (B)$]]
	\[
	\sum_{a|n} μ(a)g(n/a) = \sum_{a|n}μ(n/a)g(a) = \sum_{a|n}μ(a) \sum_{b|n/a} f(b) = \sum_{b|n}f(b)\left(\sum_{a|n/b} μ(a)\right) = f(n)
	\]
	\item[[$(B) \Rightarrow (A)$]]
	\[
	\sum_{a|n}f(a) = \sum_{a|n} \sum_{b|a} μ(b)g(a/b) = \sum_{a|n}\sum_{b|a} μ(a/b)g(b) = \sum_{b|n}g(b)\left(\sum_{b|a|n}μ(a/b)\right) = g(n)
	\]
\end{itemize}
\end{dem}
\section{Series de Dirichlet}
\[ \sum \frac{d(n)}{n^s} = \left(\sum \frac{1}{n^s}\right) \cdot \left(\sum \frac{1}{n^s}\right) = ζ(s)^2 \]

\[ \sum_{n=1}^\infty \frac{σ(n)}{n^s} = ζ(s) \sum_{n=1}^\infty \frac{n}{n^s} = ζ(s) \sum \frac{1}{n^{s-1}} = ζ(s)ζ(s-1) \]
\[ \sum_{n=1}^\infty \frac{μ(n)}{n^s} \sum_{n=1}^\infty \frac{1}{n^s} = 1 \Rightarrow \sum_{n=1}^\infty \frac{μ(n)}{n^s} = \frac{1}{ζ(s)} \]

donde $s \in \C$, $s = σ + it$. Si $σ > 1$, $ζ(s)$ converge, $1/ζ(s)$ también converge, pues:
\[ \left|\frac{μ(n)}{n^s}\right| ≤ \frac{1}{n^σ} \]

\[ \left(\sum\frac{μ(n)}{n^s}\right)\left(\sum\frac{1}{m^s}\right) = \sum_{n,m=1}^\infty \frac{μ(n)}{(n\cdot m)^s} = \sum_{N=1}^\infty \frac{1}{N^s}\left(\sum_{n|N} μ(n)\right) = 1 \]
\section{Convergencia de series de Dirichlet}
\begin{teorema}
SI la serie de Dirichlet $\sum_{n=1}^\infty \frac{a_n}{n^s}$ converge en un punto $s = s_0$, entonces converge uniformemente en el ángulo $Γ_α = \{s \in \C : Re(s) > Re(s_0), |arg(s-s_0)|< α\}$ para cada $0 < α ≤ \pi/2$.
\end{teorema}

\begin{dem}
En el caso reducido $s_0=0$, $\sum_{n=1}^\infty a_n = 0$. Definimos $s_n = \sum_{k=1}^n a_k$. Entonces, para $n>m$:
\[ \left|\sum_{k=1}^n\frac{a_k}{k^s}-\sum_{k=1}^m\frac{a_k}{k^s}\right| = \left|\sum_{k=m+1}^n\frac{a_k}{k^s}\right| = \left|\frac{s_{m+1}-s_m}{(m+1)^s} + \frac{s_{m+2}-s_{m+1}}{(m+2)^s}+\dots+\frac{s_{n}-s_{n-1}}{n^s}\right|\]
\[ = \left|\frac{s_n}{n^s}-\frac{s_m}{(m+1)^s}+\sum_{k=m+1}^{n-1}s_k\left(\frac{1}{k^s}-\frac{1}{(k+1)^s}\right)\right| ≤ \frac{ϵ}{n^σ} + \frac{ϵ}{(m+1)^σ} + \sum_{k=m+1}^{n-1} ϵ \left|\frac{1}{k^s}-\frac{1}{(k+1)^s}\right|\]

Usando que:
\[ \left|\frac{1}{k^s}-\frac{1}{(k+1)^s}\right| = \left|\int_k^{k+1}-st^{-s-1}dt\right| ≤ |s|\int_k^{k+1}\left|t^{-σ-1}\right|dt ≤ \frac{|s|}{σ} \left(\frac{1}{k^σ}-\frac{1}{(k+1)^σ}\right)\]

y que $n^σ ≥ 1$ y $(m+1)^σ ≥ 1$:

\[ \left|\sum_{k=1}^n\frac{a_k}{k^s}-\sum_{k=1}^m\frac{a_k}{k^s}\right| ≤ 2ϵ + ϵ \sum_{k=m+1}^{n-1} \frac{|s|}{σ}\left(\frac{1}{k^σ}-\frac{1}{(k+1)^σ}\right) = 2ϵ+ϵ\frac{|s|}{σ}\left(\frac{1}{(m+1)^σ}-\frac{1}{n^σ}\right) ≤ 2ϵ+ϵ\frac{|s|}{σ} \]
Como $σ/|s| = \cos(θ)$ donde $θ = \arg(s) ≤ α$, $|s|/σ ≤ (\sin α)^{-1}$, luego:
\[ \left|\sum_{k=1}^n\frac{a_k}{k^s}-\sum_{k=1}^m\frac{a_k}{k^s}\right| ≤ ϵ\left(2+\frac{1}{\sin α}\right) \]
\end{dem}

\begin{coro}
Si una serie de Dirichlet converge en algún punto, existe $σ_0 \in [-\infty,\infty)$, tal que la serie de Dirichlet define una función analitica en el semiplano $Re(s) > σ_0$ y divirge para todo $Re(s) < σ_0$. Decimos que $σ_0$ es la abcisa de convergencia de la serie de Dirichlet.
\end{coro}

\begin{ej}
La serie de Dirichlet $\sum_{n=1}^\infty \frac{(-1)^n}{n^s}$ tiene una abcisa de convergencia en $σ_0=0$. Obsérvese que no converge absolutamente en $0<σ<1$. De hecho, hay una abcisa de convergencia absoluta en $σ_0=1$.
\end{ej}

\begin{prop} Sea $f(s) = \sum_{n=1}^\infty a_n/n^s$ una función definida por una serie de Dirichlet. La función $f(s)$ es idénticamente $0$ si y sólo si todos los coeficientes $a_n$ son nulos.
\end{prop}

\begin{prop} Sea $f$ multiplicativa y la serie de Dirichlet $\sum_{n=1}^\infty \frac{f(n)}{n^s}$ con abcisa de convergencia absoluta en $σ_0$. Entonces:
\[ \sum_{n=1}^\infty \frac{f(n)}{n^s} = \lim_{x \to \infty} \prod_{p≤x} \left(1+\frac{f(p)}{p^s} + \frac{f(p^2)}{p^{2s}+\dots}\right) \]
La serie dentro del producto del segundo miembro converge por ser una subserie de la serie de Dirichlet donde converge absolutamente. Si $p_1,\dots,p_a$ son todos los primos menores que $x$:
\[ \prod_{p≤x} \left(1+\frac{f(p)}{p^s} + \frac{f(p^2)}{p^{2s}}+\dots\right) = \sum_{P^+(n)≤x} \frac{f(n)}{n^s} \]
\end{prop}

\begin{coro}

Si $f$ es además completamente multiplicativa:
\[ \sum_{n=1}^\infty \frac{f(n)}{n^s} = \prod_{p}\left(1-\frac{f(p)}{p^s}\right)^{-1}\]
\end{coro}
\section{Crecimiento de funciones multiplicativas}

\section{Sumación parcial}

\section{El orden medio de $d(n)$}

\section{Método de exclusión-inclusión}

\section{Algunas otras funciones aritméticas}
\end{document}
