\documentclass[TAN.tex]{subfiles}
\begin{document}

\chapter{Funciones aritméticas}
\section{Divisibilidad}
Consideramos conocidos los conjuntos:
\[ \N = \{1,2,3,4,\dots\} \text{ números naturales}\]
\[ \Z = \{\dots,-3,-2,-1,0,1,2,3,\dots,\} \text{ números enteros}\]
y las operaciones de suma y producto definidas en ellos con las propiedades usuales. El conjunto $\Z$, dotado con las operaciones usuales es un \textbf{anillo conmutativo}. El anillo $\Z$ es un \textbf{dominio de integridad}.

\section{Las funciones $d(n)$ y $σ(n)$}

\begin{prop} Si $f$ y $g$ son multiplicativas la función $f * g$ definida por
\[ f * g (n) = \sum_{d|n} f(d)g(n/d) \]
es también multiplicativa

\begin{dem}
\[ f*g(nm) = \sum_{c|nm}f(c)g\left(\frac{nm}{c}\right) = \sum_{a|n,b|m}f(ab)g\left(\frac{nm}{ab}\right) \]
Usando que $f$ y $g$ son multiplicativas:
\begin{align*}
	f*g(nm) & = \sum_{a|n,b|m} f(a)f(b)g(n/a)g(m/b)  = \sum_{a|n}f(a)g(n/a) \sum_{b|m}f(b)g(m/b) \\
	& = (f*g)(m) \cdot (f*g)(n)
\end{align*}
\QED
\end{dem}
\end{prop}

\section{Las funciones $φ(n)$ de Euler y $μ(n)$ de Möbius}
\begin{prop}\mbox{}
\begin{enumerate}[(a)]
	\item La funcion φ es multiplicativa
	\item $φ(n) = n \displaystyle\prod_{p|n} \left(1-\dfrac{1}{p}\right)$
\end{enumerate}
\end{prop}

\begin{dem}
Usando que $φ$ es multiplicativa:
\[ φ(p^a) = p^a - p^{a-1} = p^a(1-1/p) \]
Sea $n = p_1^{a_1}\cdots p_k^{a_k}$:
\[ φ(n) = φ(p_1^{a_1})\cdots φ(p_k^{a_k}) = p_1^{a_1}(1-1/p_1)\cdots p_k^{a_k}(1-1/p_k) = n \prod_{p|n} \left(1-\dfrac{1}{p}\right)\]
\QED
\end{dem}

\begin{prop}
Para todo $n \in \N$ se tiene $\displaystyle\sum_{d|n} φ(d) = n$.
\end{prop}

\begin{dem}
Sea $f(n) = \sum_{d|n} φ(d)$ y $g(n) = n$. Sea $h(n) = 1$. Obsérvese que $f = φ * h$. Como $φ$ y $h$ son multiplicativas, $f$ es multiplicativa. Como $g$ también es mulitplicativa, para probar que $f = g$ basta ver que $f(p^a)=g(p^a)$ para un $p$ primo y $a ≥ 1$.

\[ f(p^a) = \sum_{b|p^a} φ(b) = \sum_{k=0}^a φ(p^k) = 1+(p-1)+(p²-p)+\cdots+(p^a-p^{a-1}) = p^a = g(p^a) \]

Luego $f = g$.
\qed

Como demostración alternativa:
\[ n = \left|\left\{\frac{a}{n} \mid 1 ≤ a ≤ n\right\}\right|
= \left|\bigcup_{b|n} \left\{\frac{a}{b} \mid 1≤a≤b, a \perp b\right\}\right|
= \sum_{b|n} \left|\left\{\frac{a}{b} \mid 1 ≤ a ≤ b, a \perp b\right\}\right|
= \sum_{b|n} φ(b) \]\QED
\end{dem}


\begin{prop}
\[ \sum_{a|n} μ(a) = \begin{cases}
	1 & \text{ si } n = 1\\
	0 & \text{ si } n > 1
\end{cases}\]
\end{prop}

\begin{prop}
\begin{enumerate}[(A)]
	\item $\forall n$, $g(n) = \sum_{a|n} f(a)$
	\item $\forall n$, $f(n) = \sum_{a|n}μ(a) g(n/a)$
\end{enumerate}

\end{prop}
\begin{dem}\mbox{}
\begin{itemize}
	\item[[$(A) \Rightarrow (B)$]]
	\[
	\sum_{a|n} μ(a)g(n/a) = \sum_{a|n}μ(n/a)g(a) = \sum_{a|n}μ(a) \sum_{b|n/a} f(b) = \sum_{b|n}f(b)\left(\sum_{a|n/b} μ(a)\right) = f(n)
	\]
	\item[[$(B) \Rightarrow (A)$]]
	\[
	\sum_{a|n}f(a) = \sum_{a|n} \sum_{b|a} μ(b)g(a/b) = \sum_{a|n}\sum_{b|a} μ(a/b)g(b) = \sum_{b|n}g(b)\left(\sum_{b|a|n}μ(a/b)\right) = g(n)
	\]
\end{itemize}
\end{dem}
\section{Series de Dirichlet}
\[ \sum \frac{d(n)}{n^s} = \left(\sum \frac{1}{n^s}\right) \cdot \left(\sum \frac{1}{n^s}\right) = ζ(s)^2 \]

\[ \sum_{n=1}^\infty \frac{σ(n)}{n^s} = ζ(s) \sum_{n=1}^\infty \frac{n}{n^s} = ζ(s) \sum \frac{1}{n^{s-1}} = ζ(s)ζ(s-1) \]
\[ \sum_{n=1}^\infty \frac{μ(n)}{n^s} \sum_{n=1}^\infty \frac{1}{n^s} = 1 \Rightarrow \sum_{n=1}^\infty \frac{μ(n)}{n^s} = \frac{1}{ζ(s)} \]

donde $s \in \C$, $s = σ + it$. Si $σ > 1$, $ζ(s)$ converge, $1/ζ(s)$ también converge, pues:
\[ \left|\frac{μ(n)}{n^s}\right| ≤ \frac{1}{n^σ} \]

\[ \left(\sum\frac{μ(n)}{n^s}\right)\left(\sum\frac{1}{m^s}\right) = \sum_{n,m=1}^\infty \frac{μ(n)}{(n\cdot m)^s} = \sum_{N=1}^\infty \frac{1}{N^s}\left(\sum_{n|N} μ(n)\right) = 1 \]
\section{Convergencia de series de Dirichlet}
\begin{teorema}
Si la serie de Dirichlet $\sum_{n=1}^\infty \frac{a_n}{n^s}$ converge en un punto $s = s_0$, entonces converge uniformemente en el ángulo $Γ_α = \{s \in \C : Re(s) > Re(s_0), |arg(s-s_0)|< α\}$ para cada $0 < α ≤ \pi/2$.
\end{teorema}

\begin{dem}
En el caso reducido $s_0=0$, $\sum_{n=1}^\infty a_n = 0$. Definimos $s_n = \sum_{k=1}^n a_k$. Entonces, para $n>m$:
\[ \left|\sum_{k=1}^n\frac{a_k}{k^s}-\sum_{k=1}^m\frac{a_k}{k^s}\right| = \left|\sum_{k=m+1}^n\frac{a_k}{k^s}\right| = \left|\frac{s_{m+1}-s_m}{(m+1)^s} + \frac{s_{m+2}-s_{m+1}}{(m+2)^s}+\dots+\frac{s_{n}-s_{n-1}}{n^s}\right|\]
\[ = \left|\frac{s_n}{n^s}-\frac{s_m}{(m+1)^s}+\sum_{k=m+1}^{n-1}s_k\left(\frac{1}{k^s}-\frac{1}{(k+1)^s}\right)\right| ≤ \frac{ϵ}{n^σ} + \frac{ϵ}{(m+1)^σ} + \sum_{k=m+1}^{n-1} ϵ \left|\frac{1}{k^s}-\frac{1}{(k+1)^s}\right|\]

Usando que:
\[ \left|\frac{1}{k^s}-\frac{1}{(k+1)^s}\right| = \left|\int_k^{k+1}-st^{-s-1}dt\right| ≤ |s|\int_k^{k+1}\left|t^{-σ-1}\right|dt ≤ \frac{|s|}{σ} \left(\frac{1}{k^σ}-\frac{1}{(k+1)^σ}\right)\]

y que $n^σ ≥ 1$ y $(m+1)^σ ≥ 1$:

\[ \left|\sum_{k=1}^n\frac{a_k}{k^s}-\sum_{k=1}^m\frac{a_k}{k^s}\right| ≤ 2ϵ + ϵ \sum_{k=m+1}^{n-1} \frac{|s|}{σ}\left(\frac{1}{k^σ}-\frac{1}{(k+1)^σ}\right) = 2ϵ+ϵ\frac{|s|}{σ}\left(\frac{1}{(m+1)^σ}-\frac{1}{n^σ}\right) ≤ 2ϵ+ϵ\frac{|s|}{σ} \]
Como $σ/|s| = \cos(θ)$ donde $θ = \arg(s) ≤ α$, $|s|/σ ≤ (\sin α)^{-1}$, luego:
\[ \left|\sum_{k=1}^n\frac{a_k}{k^s}-\sum_{k=1}^m\frac{a_k}{k^s}\right| ≤ ϵ\left(2+\frac{1}{\sin α}\right) \]
\end{dem}

\begin{coro}
Si una serie de Dirichlet converge en algún punto, existe $σ_0 \in [-\infty,\infty)$, tal que la serie de Dirichlet define una función analitica en el semiplano $Re(s) > σ_0$ y divirge para todo $Re(s) < σ_0$. Decimos que $σ_0$ es la abcisa de convergencia de la serie de Dirichlet.
\end{coro}

\begin{ej}
La serie de Dirichlet $\sum_{n=1}^\infty \frac{(-1)^n}{n^s}$ tiene una abcisa de convergencia en $σ_0=0$. Obsérvese que no converge absolutamente en $0<σ<1$. De hecho, hay una abcisa de convergencia absoluta en $σ_0=1$.
\end{ej}

\begin{prop} Sea $f(s) = \sum_{n=1}^\infty a_n/n^s$ una función definida por una serie de Dirichlet. La función $f(s)$ es idénticamente $0$ si y sólo si todos los coeficientes $a_n$ son nulos.
\end{prop}
\begin{dem}
Sea $a_N$ el primer coeficiente distinto de 0. Se tiene entonces que:
\[ -\frac{a_N}{N^s} = \sum_{n=N+1}^{∞} \frac{a_n}{n^s} \]
Sea $α$ la parte real de $s$ y $σ > α+1$:
\[ \left|-\frac{a_N}{N^σ}\right| = \left|\sum_{n=N+1}^{∞} \frac{a_n}{n^s}\right| ≤ \sum_{n=N+1}^{∞} \frac{|a_n|}{n^σ} ≤ c \sum_{n=N+1}^{∞} \frac{1}{n^{σ-α}} ≤ c \int_N^{∞} t^{α-σ} dt = c \left.\frac{t^{α-σ+1}}{α-σ+1}\right|_N^{∞} = \frac{c}{σ-α-1} \frac{1}{N^{σ-α-1}}\]
\[ |a_N| ≤ N^σ \frac{c}{σ-α-1} \frac{1}{N^{σ-α-1}} = \frac{CN^{α+1}}{σ-α-1} \]
\end{dem}

\begin{prop} Sea $f$ multiplicativa y la serie de Dirichlet $\sum_{n=1}^\infty \frac{f(n)}{n^s}$ con abcisa de convergencia absoluta en $σ_0$. Entonces:
\[ \sum_{n=1}^\infty \frac{f(n)}{n^s} = \lim_{x \to \infty} \prod_{p≤x} \left(1+\frac{f(p)}{p^s} + \frac{f(p^2)}{p^{2s}+\dots}\right) \]
La serie dentro del producto del segundo miembro converge por ser una subserie de la serie de Dirichlet donde converge absolutamente. Si $p_1,\dots,p_a$ son todos los primos menores que $x$:
\[ \prod_{p≤x} \left(1+\frac{f(p)}{p^s} + \frac{f(p^2)}{p^{2s}}+\dots\right) = \sum_{P^+(n)≤x} \frac{f(n)}{n^s} \]
\end{prop}

\begin{coro}

Si $f$ es además completamente multiplicativa:
\[ \sum_{n=1}^\infty \frac{f(n)}{n^s} = \prod_{p}\left(1-\frac{f(p)}{p^s}\right)^{-1}\]
\end{coro}
\section{Crecimiento de funciones multiplicativas}
\begin{teorema}
Sea $f(n)$ una función multiplicativa. Si $\lim_{p^k \to +∞} f(p^k) = 0$, entonces
\[ \lim_{n\to +∞} f(n) = 0 \]
(Ver Hardy Wright p.261 Nathanson p.255)
\end{teorema}
\begin{prop}
\[ d(n) = \mathcal{O}(n^δ) \]
para todo $δ > 0$.
\end{prop}
\begin{dem}
\[ \frac{d(n)}{n^δ} = \prod_{p|n} \frac{1+k}{p^{kδ}} = \prod_{p^δ≥2}\frac{1+k}{p^{kδ}} \prod_{p^δ<2}\frac{1+k}{p^{kδ}} ≤ \prod_{p^δ<2} \frac{1+k}{p^{kδ}}\]
usando que:
\[ \frac{1+k}{p^{kδ}} = \frac{1}{p^{kδ}}+\frac{k}{p^{kδ}} ≤ 1 + \frac{k}{p^{kδ}} ≤ 1 + \frac{k}{kδ\log p} = 1 + \frac{1}{δ\log p} ≤ 1 + \frac{1}{δ\log 2} ≤ \exp\left(\frac{1}{δ\log 2}\right)\]
llegamos a que:
\[ \frac{d(n)}{n^δ} ≤ \exp\left(\frac{|\{p : p^δ<2\}|}{δ\log2}\right) ≤ \exp\left(\frac{2^{1/δ}}{δ\log 2}\right)\]
\end{dem}

\begin{prop}
Para todo $ϵ > 0$ se tiene que
\[ \lim_{n \to +∞} φ(n)/n^{1+ϵ} =0, \quad \lim_{n \to +∞} φ(n)/n^{1-ϵ} = +∞ \]
(Ver Nathanson p.227)
\end{prop}
El \textbf{orden máximo} y el \textbf{orden mínimo} de una función aritmética $f(n)$ se definen como funciones monótonas crecientes $g(n)$ tales que
\[ \limsup_{n\to+∞} \frac{f(n)}{g(n)} = 1, \quad \text{respectivamente }\liminf_{n\to+∞}\frac{f(n)}{g(n)} = 1 \]
\begin{prop}
El orden mínimo de $\log d(n)$ es $\log 2$ y el orden máximo es
\[ \frac{\log 2 \log n}{\log \log n} \]
\end{prop}
\begin{dem}
Vamos a demostrar sólo el orden máximo. Para el orden mínimo ver Hardy \& Wright, p.262. Vamos a ver que
\[ \limsup \frac{\log d(n)}{\frac{\log 2 \log n}{\log \log n}} = 1\]
Aplicando logaritmo a la expresión $\frac{d(n)}{n^δ} ≤ \exp\left(\frac{2^{1/δ}}{δ\log 2}\right)$.
\[
	\log d(n) ≤ δ\log n + \frac{2^{1/δ}}{δ\log 2}
\]
Sea $δ = (1+ϵ)\frac{\log 2}{\log \log n}$, entonces:
\[
	\log d(n) ≤ (1+ϵ) \frac{\log 2 \log n}{\log \log n} + \frac{(\log n)^{1/(1+ϵ)}}{(\log 2)^2} \frac{\log \log n}{1+ϵ} =: (1+ϵ)f(n) + g(n)
\]
Dividiendo por $f(n)$:
\[ 
	\frac{\log d(n)}{f(n)} ≤ 1 + ϵ + \frac{g(n)}{f(n)}
\]
Se observa además que $\frac{g(n)}{f(n)} \xrightarrow{n \to ∞} 0$, luego:
\[ 
	\limsup \frac{\log d(n)}{f(n)} ≤ 1+ϵ
\]
Como esto se da para todo $ϵ$, se cumple que:
\[ 
	\limsup \frac{\log d(n)}{\frac{\log 2 \log n}{\log \log n}} ≤ 1
\]

Por otro lado, tomando $n = \prod_{p≤x} p$, entonces:
\[ \log n = \sum_{p≤x} \log p =: Θ(x) ≤ \sum_{p≤x} \log x = π(x) \log x \]
\[ d(n) = 2^{π(x)} \]
se llega a que:
\[ \frac{\log d(n)}{\frac{\log 2 \log n}{\log \log n}} ≥ 1 \]
Luego:
\[
	\limsup \frac{\log d(n)}{\frac{\log 2 \log n}{\log \log n}} = 1 
\]
\end{dem}

\section{Sumación parcial}
La idea básica que queremos desarrollar es la aproximación de una suma por una integral, y se puede plasmar en una figura.

\begin{teorema}[Comparación de una suma y una integral]
Sea $f : (0,+\infty) \to [0,+\infty)$ una función positiva y decreciente. Existe una constante $γ(f)$ tal que para todo $x>1$ se tiene
\[ \sum_{n≤x} f(n) = \int_1^x f(t)dt + γ(f) + \mathcal{O}(f(x)) \]
\end{teorema}
\begin{teorema}
Existe una constante $γ$ tal que para todo $x>1$
\[ \sum_{n≤x} \frac{1}{n} = \log x + γ + \mathcal{O}\left(\frac{1}{x}\right) \]
\end{teorema}
La constante $γ$ se denomina constante de Euler, su valor aproximado es $γ \approx 0.577215663...$ Es un problema abierto decidir si es racional o irracional.

Clásicamente se llama sumación de Abel al proceso en que convertimos una suma $\sum_n a_n b_n$ mediante las sumas parciales $\sum_n a_n$. Esto es, definiendo
\[ A_n = \sum_{j=1}^n a_j, \qquad A_0 = 0 \]
la transformación es
\[ \sum_{j=1}^n a_j b_j = \sum_{j=1}^n (A_j-A_{j-1})b_j = \sum_{j=1}^{n-1} A_j(b_j-b_{j+1}) + A_n b_n \]
Este tipo de argumento es análogo al de la integración por partes. Si $A(x)=\int_a^x a(t)dt$
\[ \int_a^x a(t)f(t) dt = \int_a^x f(t) dA(t) = \left.f(t)A(t)\right|_a^x - \int_a^x A(t) df(t) = A(x)f(x) - \int_a^x A(t) df(t) \]
en el siguiente teorema lo exttendemos al caso en que $A$ es una funcón de salto.

\begin{teorema}[Fórmula de Abel] Sea $(a_n)$ una sucesión de números complejos. Pongamos
\[ A(x) = \sum_{n≤x} a_n \quad (x > 0) \]
Sea $f(t)$ una función con derivada continua en $[1,x]$. Entonces tenemos
\[ \sum_{1≤n≤x} a_nf(n) = A(x)f(x)-\int_1^x A(t)f'(t)dt \]
\end{teorema}
\section{El orden medio de $d(n)$}
Como vemos, hemos tenido que pasar a $\log d(n)$ para poder determinar el orden máximo. El tomar logaritmos obviamente hace la función más manejable. Otra forma de resolver e problema es estudiar el \textbf{orden medio}, s llama así a una función $g(n)$ monótona y tal que
\[ f(1) + f(2) + \cdots + f(n) \sim g(1) + g(2) + \cdots + g(n) \]
\begin{teorema}[Dirichlet]
Cuando $x$ tiende a infinito, tenemos
\[ \sum_{n≤x} d(n) = x \log x + (2γ -1)x + \mathcal{O}(\sqrt{x}) \]
\end{teorema}

\begin{dem}
\[ \sum_{n≤x} d(n) = \sum_{n≤x}\sum_{a|n}1 = \sum_{a≤x}\lfloor \frac{x}{a}\rfloor = \sum_{ab≤x} 1 \]
Consideramos el retículo $\mathcal{R}=\{(n,m) | n \in \N,m \in \N, nm≤x\}$. Se tiene que $\sum_{ab≤x} 1 = \#\mathcal{R}$. Si particionamos $\mathcal{R}$ en los conjuntos: $A=\{(n,m) \mid n \in \N,m \in \N, nm≤x, m>\sqrt{x}\}, B=\{(n,m) \mid n \in \N,m \in \N, nm≤x, n>\sqrt{x}\}$ y $C=\{(n,m) \mid n \in \N,m \in \N, nm≤\sqrt{x}\}$. Se tiene que $\#A=\#B$, luego $\#\mathcal{R}=2\#A+\#C$:
\begin{align*}
	\sum_{ab≤x} 1 & = 2 \# A - \lfloor \sqrt{x} \rfloor^2 = 2 \sum_{a≤\sqrt{x}} \lfloor \frac{x}{a}\rfloor - \lfloor \sqrt{x} \rfloor^2 = 2 \sum_{a≤x} \left(\frac{x}{a}-\left\{\frac{x}{a}\right\}\right) - (\sqrt{x}-\{\sqrt{x}\})^2 =\\
	& = 2x \sum_{a ≤ \sqrt{x}} \frac{1}{a} + \mathcal{O}(\sqrt{x}) - (x-2\{\sqrt{x}\} \sqrt{x} + \{\sqrt{x}\}^2) = \\
	& = 2x(\log\sqrt{x} + γ + \mathcal{O}(\frac{1}{x})) + \mathcal{O}(\sqrt{x}) -x+\mathcal{O}(\sqrt{x})+\mathcal{O}(1) = x \log x + (2γ-1)x+\mathcal{O}(\sqrt{x})
\end{align*}
donde $\{x\} = x-\lfloor x \rfloor$ es la parte fraccionaria de $x$. Luego $\sum_{n≤x} d(n) = x \log x + (2γ-1)x + \mathcal{O}(\sqrt{x})$.
\end{dem}
La tecnica usada en la prueba se debe a Dirichlet y se denomina \textbf{método de la hipérbola}.

El término de error $\mathcal{O}(\sqrt{x})$ puede mejorarse. La determinación del ínfimo de los $θ$ tales que podamos substituir $\mathcal{O}(\sqrt{x})$ por $\mathcal{O}(x^{θ})$ en el teorema anterior, constituye un problema abierto que se onoce con el nombre de \textbf{problema del divisor de Dirichlet}. En 1915 Hardy y Landau probaron que $\inf θ ≥ 1/4$.

Por un método más simple podemos obtener el siguiente resultado.
\begin{teorema}
Para $x≥1$
\[ Φ(x) = \sum_{n≤x} φ(n) = \frac{3x^2}{π^2} + \mathcal{O}(x\log x) \]
\end{teorema}
\begin{dem}
\begin{align*}
	\sum_{n≤x} φ(n) & = \sum_{n≤x}\sum_{a|n} μ(a) \frac{n}{a} = \sum_{a≤x} \frac{μ(a)}{a}\left(\sum_{a|n,n≤x}n\right) = \frac{1}{2} \sum_{a≤x} μ(a)\left(\lfloor \frac{x}{a}\rfloor^2 + \lfloor \frac{x}{a}\rfloor \right)\\
	& = \frac{1}{2} \sum_{a≤x} μ(a) \left(\left(\frac{x}{a} - \left\{ \frac{x}{a}\right\} \right)^2 + \frac{x}{a}-\left\{\frac{x}{a}\right\}\right)\\
	& = \frac{1}{2} \sum_{a≤x} μ(a) \left(\frac{x^2}{a^2} - 2\left\{\frac{x}{a}\right\}\frac{x}{a} + \left\{\frac{x}{a}\right\}^2+\frac{x}{a}-\left\{\frac{x}{a}\right\}\right)\\
	&  \frac{x^2}{2} \sum_{a≤x} \frac{μ(a)}{a^2} - \frac{x}{2} \sum_{a≤x} \frac{μ(a)}{a} - x \sum_{a≤x} \frac{μ(a)}{a}\left\{\frac{x}{a}\right\} + \mathcal{O}(x)
\end{align*}
Como:
\[ \left|x\sum_{a≤x} \frac{μ(a)}{a} \left\}\frac{x}{a}\right\}\right| ≤ x \sum_{a≤x} \frac{1}{a} = x (\log x + γ + \mathcal{O}\left(\frac{1}{x}\right)) ≤ 2x \log x \]
Luego:
\begin{align*}
	\sum_{n≤x} φ(n) = \frac{x^2}{2} \sum_{a≤x} \frac{μ(a)}{a^2} + \mathcal{O}(x\log x) = \frac{x^2}{2}\left(S - \sum_{a>x} \frac{μ(a)}{a^2}\right) + \mathcal{O}(x \log x)
\end{align*}
donde $S = \sum_a \frac{μ(a)}{a^2}$. Como
\[ \left|\sum_{a>x} \frac{μ(a)}{a^2}\right| ≤ \sum_{a>x} \frac{1}{a^2} ≤ \frac{1}{x^2} + \int_x^\infty \frac{dt}{t^2} = \frac{1}{x}+\frac{1}{x^2} ≤ \frac{2}{x} \]
Luego $\sum_{a>x} \frac{μ(a)}{a^2} \in \mathcal{O}(1/x)$
\begin{align*}
	\sum_{n≤x} φ(n) = \frac{x^2}{2}S + \mathcal{O}(x \log x)
\end{align*}
Por otro lado, como $ζ(s)^{-1} = \sum \frac{μ(n)}{n^s}$
\[ \sum_a \frac{μ(a)}{a^2} ζ(2) = 1 \Rightarrow S = \frac{1}{ζ(2)} = \frac{6}{π^2}\]
Finalmente:
\[ \sum_{n≤x} φ(n) = \frac{6}{π}\frac{x^2}{2} + \mathcal{O}(x\log x)\]
\end{dem}

El teorema se debe a Mertens (1874). El mejor término de error que se conoce es $\mathcal{O}(x(\log x)^{2/3}(\log \log x)^{4/3})$ que se debe a Walfisz en 1963.

\begin{coro} Dado un entero $x$, la probabilidad de que dos eneteros positivos $≤x$ sea relativamente primos es: $\dfrac{6}{π^2}+ \mathcal{O}\left(\frac{\log x}{x}\right)$.
\end{coro}

\begin{prop}
El orden medio de $σ(n)$ es $π^2n/6$. Esto es
\[ \sum_{n≤x} σ(n) = \frac{1}{12} π^2x^2 + \mathcal{O}(x\log x) \]
\end{prop}
El mejor término de error que se conoce es $\mathcal{O}(x(\log x)^{2/3})$ que se debe a Walfisz en 1963.

\section{Método de exclusión-inclusión}
\section{Algunas otras funciones aritméticas}
La \textbf{función de Liouville} se define de manera parecida a la de Möbius
\[ λ(n) = \begin{cases}
	1 &\text{ si }n=1\\
	(-1)^{α_1+\dots+α_r} &\text{ si }n=p_1^{α_1}\cdot p_r^{α_r}
\end{cases} \]
$ω(n)$ denota el número de factores primos distintos de $n$ y $Ω(n)$ el número total de factores primos:
\[ ω(n) = \sum_{p|n} 1, \qquad Ω(n) = \sum_{p^r|n} 1 \]
Finalmente es de uso frecuenta la función
\[ r(n) = \text{card}\{(x,y) \in \Z^2 : x^2+y^2=n\} \]
Para el capítulo en general son recomendables los libros de \href{https://leonettipaolo.files.wordpress.com/2012/07/ebook-english-g-h-hardy-an-introduction-to-the-theory-of-numbers.pdf}{Hardy \& Wright}, Tenenbaum o Nathanson.
\end{document}
