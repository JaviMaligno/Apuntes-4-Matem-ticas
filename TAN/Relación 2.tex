\documentclass[twoside]{article}
\usepackage{../estilo-ejercicios}
\providecommand{\bo}[1]{\mathcal{O}\left(#1\right)}
%--------------------------------------------------------
\begin{document}

\title{Teoría Analítica de Números}
\author{Rafael González López\\Diego Pedraza López}
\maketitle

\begin{ejercicio}{1}
Demostrar que
\[ \sum_{n≤x} \frac{1}{\sqrt{n}} =  2 \sqrt{x} + A + \mathcal{O}(x^{-1/2}) \]
\end{ejercicio}
\begin{sol}
Basta de aplicar la fórmula:
\[ \sum_{n≤x} \frac{1}{\sqrt{n}} = \int_1^x \frac{1}{\sqrt{t}}dt + A + \mathcal{O}\left(\frac{1}{\sqrt{n}}\right) = 2 \sqrt{x}-2+A + \mathcal{O}(x^{-1/2}) \]
\end{sol}

\newpage

\begin{ejercicio}{2}\mbox{}
\begin{enumerate}[(a)]
	\item Demostrar que para $x > 2$ se tiene
	\[ \sum_{n≤x} \frac{\log n}{n} = \frac{(\log x)^2}{2} + \mathcal{O}(1) \]
	\item Probar que podemos mejorar (a) y probar
	\[ \sum_{n≤x} \frac{\log n}{n} = \frac{(\log x)^2}{2} + A + \mathcal{O}\left(\frac{\log x}{x}\right) \]
\end{enumerate}
\end{ejercicio}
\begin{sol}
Aplicando la Fórmula de Abel con $a_n=1/n$ y usando que $\sum_{n≤x} a_n = \log x + γ + \mathcal{O}(1/x)$:
\begin{align*}
	\sum_{n≤x} \frac{\log n}{n} & = \log x (\log x + γ + \mathcal{O}(1/x)) - \int_1^x \frac{1}{t}(\log t + γ + \mathcal{O}(1/t))\\
	& = \left((\log x)^2 + γ\log x + \mathcal{O}\left(\frac{\log x}{x}\right)\right) - \left(\frac{(\log x)^2}{2} + γ\log x - A - \mathcal{O}\left(\frac{1}{x}\right)\right)\\
	& = \frac{(\log x)^2}{2} + A + \mathcal{O}\left(\frac{\log x}{x}\right)
\end{align*}
donde $A$ se corresponde con el término $-\int \mathcal{O}(1/t^2)$ en $t=1$. De esta forma demostramos (b). Para demostrar (a) basta con tomar una forma más debil de la expresión:
\[ \sum_{n≤x} \frac{\log n}{n} = \frac{(\log x)^2}{x} + A + \mathcal{O}(1) = \frac{(\log x)^2}{x} + \mathcal{O}(1) \]
\end{sol}

\newpage

\begin{ejercicio}{3}
Demostrar que para $x ≥ 2$
\[ \sum_{n≤x} (\log n)^2 = x(\log x)^2 - 2x \log x + 2x + \mathcal{O}((\log x)^2) \]
\end{ejercicio}
\begin{sol}
Tomando $f(n)=\log^2 n$ y $a_n=1$ tenemos que 
\begin{align*}
	\sum_{n≤x} (\log n)^2 = x\log^2 x - \int_1^x \frac{2\log t}{t}tdt = x\log^2 x - 2x\log x +2x -2
\end{align*}
\end{sol}

\newpage

\begin{ejercicio}{4}
Demostrar que
\[ \sum_{n≤x} \left\{ \frac{x}{n} \right\} = (1-γ)x + \mathcal{O}(x^{1/2}) \]
donde $\{x\}=x-\lfloor x \rfloor$ denota la parte fraccionaria de $x$.
\end{ejercicio}
\begin{sol}
Recordemos que $\sum_{n≤x} \lfloor x/n \rfloor = \sum_{n≤x} d(n)$ donde $d(n)$ es el número de divisores de $n$\footnote{\url{https://math.stackexchange.com/questions/338432/how-to-prove-the-relation-between-the-floor-function-and-the-number-of-divisors}}.
\begin{align*}
	\sum_{n≤x} \left\{ \frac{x}{n} \right\} & = x\sum_{n≤x} \frac{1}{n} - \sum_{n≤x} \lfloor \frac{x}{n}\rfloor = x(\log x + γ + \mathcal{O}(1/x))- \sum_{n≤x} d(n)\\
	& = x(\log x + γ + \mathcal{O}(1/x))- (x \log x + (2γ-1)x + \mathcal{O}(\sqrt{x})\\
	& = (1-γ)x + \mathcal{O}(\sqrt{x})
\end{align*}
\end{sol}

\newpage

\begin{ejercicio}{5}
Demostrar que
\[ \sum_{n≤x} \frac{d(n)}{n} = \frac{1}{2} (\log x)^2 + 2γ\log x + B + \mathcal{O}(x^{-1/2}) \]
\end{ejercicio}
\begin{sol}
Aplicando la fórmula de Abel con $a_n=d(n)$ aplicando el teorema 1.8.1:
\begin{align*}
	\sum_{n≤x} \frac{d(n)}{n} & = \frac{x \log x + (2γ-1)x + \mathcal{O}(\sqrt{x})}{x} + \int_1^t \frac{x \log x + (2γ-1)x + \mathcal{O}(\sqrt{x})}{x^2}\\
	& = \log x + (2γ-1) + \mathcal{O}(x^{-1/2}) + \frac{(\log x)^2}{2} + (2γ-1)\log x - \mathcal{O}(x^{-1/2}) + A\\
	& = \frac{(\log x)^2}{2} + 2γ\log x + (2γ-1 + A) + \mathcal{O}(x^{-1/2})
\end{align*}
\end{sol}

\newpage

\begin{ejercicio}{6}
\begin{itemize}
\item[]
\item Demostrar que para todo $x ≥ 1$ se tiene
\[ \sum_{n≤x} μ(n) \left\lfloor \frac{x}{n} \right\rfloor = 1 \]
\item Probar que para todo $x ≥ 1$ se tiene
\[ \left|\sum_{n≤x} \frac{μ(n)}{n}\right| ≤ 1 \]

\end{itemize}

\end{ejercicio}
\begin{sol}
\begin{itemize}
\item Por el mismo argumento del ejercicio 4:
\[ \sum_{n≤x} \sum_{d|n} μ(d) = \sum_{n≤x} μ(n) \left\lfloor \frac{x}{n} \right\rfloor \]
Como $\sum_{d|n} μ(d) = 0$ si $n > 1$ y $\sum_{d|n} μ(d) = 1$ si $n = 1$:
\[ \sum_{n≤x} μ(n) \left\lfloor \frac{x}{n} \right\rfloor = 1 \]

\item Por el ejecicio anterior:
\[ 1 = \sum_{n≤x} μ(n) \left\lfloor \frac{x}{n} \right\rfloor = x\sum_{n≤x} μ(n) \frac{1}{n} - \sum_{n≤x} μ(n) \left\{ \frac{x}{n} \right\}  \]
Luego:
\[ \gabs{\sum_{n≤x} \frac{μ(n)}{n}} = \gabs{\frac{1}{x} \left(1+\sum_{n≤x}μ(n)\left\{\frac{x}{n}\right\}\right)}
≤ \frac{1}{x} \left(1+\sum_{n≤x}\gabs{μ(n)\left\{\frac{x}{n}\right\}}\right) \]
Como para $n=x$, $\{x/n\}=0$, podemos prescindir del último término:
\[ \left|\sum_{n≤x} \frac{μ(n)}{n}\right| ≤ \frac{1}{x} \left(1+\sum_{n≤x-1}\gabs{μ(n)\left\{\frac{x}{n}\right\}}\right) ≤ \frac{1}{x} \left(1+\sum_{n≤x-1}1\right) ≤ \]
\[\leq \frac{1}{x} \left(1+\lfloor x \rfloor-1\right) = \frac{\lfloor x \rfloor}{x} ≤ 1\]
\end{itemize}

\newpage

\begin{ejercicio}{7}
Probar que para $n ≥ 2$ se tiene
\[ \sum_{n≤x} Λ(n) \log n = ψ(x) \log x + \mathcal{O}(x) \]
\end{ejercicio}
\begin{sol}
Aplicamos la fórmula de Abel con $a_n=Λ(n)$ recordando que, por definición, $ψ(x) = \sum_{n≤x} Λ(n)$:
\[ \sum_{n≤x} Λ(n) \log n = ψ(x)\log x + \int_1^x \frac{ψ(t)}{t}dt \]
Por el teorema de la Valleé Poussin, existe una constante absoluta $c > 0$ tal que:
\[ ψ(x) = x + \mathcal{O}(xe^{-c\sqrt{\log x}}) \]
Tomando la expresión más débil $x + \mathcal{O}(x)$:
\[ \int_1^x \frac{ψ(t)}{t}dt = \int_1^x (1 + \mathcal{O}(1)) dt = x + A + \mathcal{O}(x) = \mathcal{O}(x) \]
Luego:
\[ \sum_{n≤x} Λ(n) \log n = ψ(x)\log x + \mathcal{O}(x) \]

\end{sol}
\newpage

\begin{ejercicio}{8}
Demostrar que
\[ \sum_{n≤x} \frac{σ(n)}{n} = \frac{π^2}{6} x + \mathcal{O}(\log^2 x) \]
\end{ejercicio}
\begin{sol}
Aplicando la fórmula de Abel con $a_n = σ(n)$:
\begin{align*}
	\sum_{n≤x} \frac{σ(n)}{n} & = \frac{1}{x}\left(\frac{1}{12}π^2x^2 + \mathcal{O}(x \log x)\right) + \int_1^x \frac{1}{t^2}\left(\frac{1}{12}π^2t^2 + \mathcal{O}(t \log t)\right)dt\\
	& = \frac{1}{12}π^2x + \mathcal{O}(\log x) + \frac{1}{12}π^2x + \mathcal{O}(\log^2 x) = \frac{π^2}{6}x + \mathcal{O}(\log^2 x)
\end{align*}
\end{sol}

\newpage

\begin{ejercicio}{9}
Demostrar que existen constantes positivas $A$ y $B$ tales que para todo $x$ suficientemente grande
\[ A\sqrt{x} ≤ \sum_{p≤x} \frac{\log p}{\sqrt{p}} ≤ B \sqrt{x}\]
\end{ejercicio}
\begin{sol}
\end{sol}

\newpage

\begin{ejercicio}{10}
Demostrar que
\[ \sum_{n≤x} r(n) = πx + \mathcal{O}(\sqrt{x}) \]
\end{ejercicio}

\newpage

\begin{ejercicio}{11}
Demostar que para todo número natural $k$ y todo $x > 0$ se tiene
\[ \frac{π(x)}{x} ≤ \frac{φ(k)}{k} + \frac{2k}{x} \]
Deducir de lo anterior y del teorema de Chebyshe que existe una constante $C > 0$ tal que
\[ φ(k) ≥ \frac{Ck}{\log k} \quad \forall k≥2 \]
\end{ejercicio}

\newpage

\begin{ejercicio}{12}
Suponiendo la hipótesis de Riemann puede probarse que
$$
\theta(x):=\sum_{p\leq x}\log p = x + \mathcal{O}(x^{1/2}\log^2x).
$$
Admitiendo este resultado probar que si es cierta la hipótesis de Riemann, entonces existe una constante A tal que
$$
\sum_{p\leq x}\frac{\log p}{p} = \log x + A + \mathcal{O}(x^{-1/2}\log^2 x)
$$
\begin{sol}
Sea $a_n = I_{\mathbb{P}}(n)\log n$, donde $I_{\mathbb{P}}(n)$ es la funcion indicatriz de los números primos. Entonces, suponiendo cierta la hipótesis de Riemann y el resultado anterior, además de aplicar la fórmula de sumación de Abel, obtenemos
\begin{align*}
\sum_{p\leq x}\frac{\log p}{p} &= \sum_{n\leq x}\frac{a_n}{n} = \frac{\theta(x)}{x} - \int_1^x \frac{-\theta(t)}{t^2}dt =\\
&= \frac{x + \mathcal{O}(x^{1/2}\log^2x)}{x} + \int_1^x \frac{t + \mathcal{O}(t^{1/2}\log^2t)}{t^2}dt\\
 &= 1+\mathcal{O}(x^{-1/2}\log^2 x) + \int_1^x \frac{1}{t} + \frac{\mathcal{O}(t^{1/2}\log^2 t)}{t^2}dt =\\
&=\log x + 1 + \mathcal{O}(x^{-1/2}\log^2 t)  + \int_1^x \frac{\mathcal{O}(t^{1/2}\log^2 t)}{t^2}dt
\end{align*}
Estamos muy cerca del resultado que buscamos, pero todavía nos falta trabajo. Tenemos que analizar el último término integral. Primeramente, $\mathcal{O}(t^{1/2}\log^2 t)$ es una función $U(t)$ de la que sabemos existen $t_0 \in \R$ y $C\in\R_{>0}$ tales que $|U(t)|\leq Ct^{1/2}\log^2 t$ $\forall t>t_0$. De hecho, $U(t)=\theta(t)-t$. Consideremos entonces
\begin{gather*}
\int_1^x \frac{\mathcal{O}(t^{1/2}\log^2 t)}{t^2}dt = \int_1^x \frac{U(t)}{t^2}dt = \int_1^\infty \frac{U(t)}{t^2}dt - \int_x^\infty \frac{U(t)}{t^2}dt
\end{gather*}
De hecho
$$
\gabs{\int_1^\infty \frac{U(t)}{t^2}dt} \leq \int_1^\infty \frac{\gabs{U(t)}}{t^2}dt \leq \int_1^\infty \frac{U(t)}{t^2}dt =\int_1^\infty  Ct^{-3/2}\log^2 t dt =16C<\infty
$$ Si tomamos $A = 1 + \int_1^\infty \frac{U(t)}{t^2}dt$ llegamos a que
$$
\sum_{p\leq x}\frac{\log p}{p} = \log x + A + \mathcal{O}(x^{-1/2}\log^2 t) - \int_x^\infty \frac{U(t)}{t^2}dt
$$
Entonces tenemos qué ver el orden del último sumando, pero
\begin{gather*}
S(x)= \int_x^\infty \frac{U(t)}{t^2}dt = \int_x^\infty \frac{\mathcal{O}(t^{1/2}\log^2 t)}{t^2}dt = \bo{\int_x^\infty  {t^{-3/2}\log^2 t}dt}=\\
=\bo{\left[-\frac{2(\log^2 t+4\log t + 8}{t^{1/2}}\right]_x^\infty }= \bo{\frac{2(\log^2 x+4\log x + 8)}{x^{1/2}}} = \mathcal{O}(t^{-1/2}\log^2 t)
\end{gather*}
Como queríamos probar.
\end{sol}
\end{ejercicio}

\newpage

\begin{ejercicio}{13}
Suponiendo la hipótesis de Riemann puede probarse que $θ(x) = x + \mathcal{O}(\sqrt{x} \log^2x)$. Admitiendo la hipótesis de Riemann, probar que:
\[ \sum_{p≤x} \sqrt{p}\log p = \frac{2}{3} x^{3/2} + \mathcal{O}(x\log^2 x) \]
\end{ejercicio}

\newpage

\begin{ejercicio}{14}
¿Para qué valores de $α$ y $β$ $\in \R$ es convergente la serie $\sum\limits_p \dfrac{1}{p^α(\log p)^β}$?
\end{ejercicio}
\end{document}