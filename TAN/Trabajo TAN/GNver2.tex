\documentclass[a4paper,12pt]{article}
\usepackage{makeidx}
\usepackage[T1]{fontenc}
\usepackage{amsmath,amscd,amsthm}
\usepackage{amssymb}
\usepackage{tabularx}
\usepackage{amssymb,eucal,bezier,graphicx}
\usepackage{times,amssymb}
\usepackage[ansinew]{inputenc}
\usepackage{multicol}
\usepackage{hyperref}
\usepackage{multicol}
\usepackage{verbatim}
\usepackage{tikz-cd}
\usetikzlibrary{graphs}
\usetikzlibrary{arrows.meta}

\newtheorem{thm}{Theorem}[section]
\newtheorem{cor}[thm]{Collorary}
\newtheorem{lemma}[thm]{Lemma}
\newtheorem{prop}[thm]{Proposition}
\newtheorem{prob}[thm]{Problema}
\newtheorem{defi}[thm]{Definici�n}
\newtheorem{conj}[thm]{Conjectura}
\newtheorem{nota}[thm]{Note}
\newtheorem{ejem}[thm]{Ejemplo}
\providecommand{\gilb}[1]{\mathcal{G}_{#1}}
\providecommand{\abs}[1]{\left|{#1}\right|}
\newcommand{\Pri}{\mathbb{P}}
\newcommand{\Z}{\mathbb{Z}}
\newcommand{\N}{\mathbb{N}}
\newtheorem{demo}[thm]{Proof}
\providecommand{\gabs}[1]{\left|{#1}\right|}
\begin{document}

%%Los tres siguientes comandos estaban inicialmente desinsertados, pero no funcionaban en concordancia en el texto

\renewcommand{\figurename}{Figure}

\renewcommand\thefigure{\arabic{section}.\arabic{figure}} 

\numberwithin{figure}{section} 

\renewcommand\refname{Bibliografia}


\begin{center} {\large \bf Estudio de la Conjetura de Gilbreath}
\end{center}


\begin{center}
{\bf Eleazar Duarte Aponte, Rafael Gonz�lez L�pez,\\
Roc�o Palacios Cantillo, Luis Palma Blanco,\\
Diego Pedraza L�pez, Mar�a de los Remedio Boza Ru�z}
\end{center}

\begin{center}
\small{ Teor�a Anal�tica de N�meros \\
\small Universidad de Sevilla. }
\end{center}

\vspace{0.4cm}

\begin{center}
{\bf Abstract}

\end{center}

\begin{quotation}
\noindent En este trabajo vamos a TAL TAL TAL
\end{quotation}

\vspace{0.1cm}



\section*{Introducci�n}
Durante el a�o 1958, Norman Gilbreath buscaba, como muchos otros matem�ticos a lo largo de la historia, una forma de generar los primos. Para ello comenz� a estudiar las diferencias entre primos consecutivos. Si volvemos a repetir las diferencias en valor absoluto una y otra vez sobre la sucesi�n que nos queda, obtenemos algo de esta forma
$$
\begin{array}{c c c c c}
2 & 3 & 5 & 7 & 11\\
\hline
1 & 2 & 2 & 4 &\\
1 & 0 &  2 &  &\\
1 & 2 & & & \\
1 &   &  & & 
\end{array}
$$
Si denotamos $d^k_n$ al $n$-�simo elemento de la $k$-�sima fila, entonces tenemos que
\begin{gather*}
d^1_n = p_{n+1}-p_n\\
d^k_n = \abs{d_{n+1}^{k-1}-d_{n}^{k-1}}
\end{gather*}
\section{Preliminares}


\section{Estudio computacional de la conjetura de Gilbreath}


\section{�rbol de Gilbreath}

Consideremos todas las sucesiones $\{a_n\}_{n\in I}$ tales que $a_1=2$, $a_2=3$, $a_n$ es impar y mayor que $3$ para todo $n$. Si no hay confusi�n sobre la sucesi�n de la que estamos hablando, notaremos 
\begin{gather*}
d^1_n = a_{n+1}-a_n\\
d^k_n = \abs{d_{n+1}^{k-1}-d_{n}^{k-1}}
\end{gather*}
Normalmente $I=\N$ o $I=\N_k$.
\begin{defi}
En las condiciones anteriores, diremos que $\{a_n\}_{n\in I}$ es una sucesi�n de Gilbreath si $d^k_1 =1$ para todo $k\in I$.
\end{defi}
Una consecuencia trivial de esta definici�n es que
\begin{prop}
La conjetura de Gilbreath es equivalente a que $\{p_n\}$ sea una sucesi�n de Gilbreath. 
\end{prop}

\subsection{Crecimiento y distribuci�n}
Aunque estamos particularmente interesados en estudiar las sucesiones de Gilbreath crecientes, sin embargo hemos damos una definici�n m�s relajada con el fin de agilizar su estudio mediante una serie de nuevos conceptos.
\begin{defi}
Sea $S=\{a_n\}_{n\in\N_k}$ una sucesi�n finita de Gilbreath. Definimos $\mathcal{G}_S$ como el conjunto de todos los elementos tales que si $m\in \mathcal{G}_S$ entonces la sucesi�n definida como $b_n = a_n$ si $n\leq k$ y $b_{k+1}=m$ sigue siendo una sucesi�n de Gilbreath.
\end{defi}
\begin{defi}
En las condiciones de la definici�n anterior denotamos $\mathcal{G}^+_S$ al conjunto $\{s \in \mathcal{G}_S \mid s > a_n\}$.
\end{defi}
Vamos a ilustrar este concepto con algunos ejemplos. 
\begin{ejem}
Si $S=\{2,3\}$ entonces $\gilb{S}=\{5\}$.
\end{ejem}
\begin{conj} Sean $S=\{s_i\}_{i\in \N_k}$ y $K=\{k_i\}_{i\in \N_k}$ dos conjuntos Gilbreath de la misma longitud. Si $s_i \geq k_i$ $\forall i=1,\dotsc,n$, entonces $\mathcal{G}_k \subset \mathcal{G}_S$.
\end{conj}

\subsection{Estudio probabil�stico}



\newpage






\begin{thebibliography}{99}


\bibitem{odly}  Odlyzko, A. (1993). Iterated Absolute Values of Differences of Consecutive Primes. Mathematics of Computation, 61(203), 373-380. doi:10.2307/2152962
\end{thebibliography}

\end{document}
