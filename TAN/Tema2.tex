\documentclass[TAN.tex]{subfiles}
\begin{document}
\chapter{Distribución de los números primos\\Métodos elemntales}

\section{Funciones de Chebyshev}
\begin{nota}
\[ π(x)=\sum_{p≤x} 1, \quad θ(x) = \sum_{p≤x} \log p, \quad ψ(x)=\sum_{p^k≤x} \log p \]
\end{nota}
Observar que $ψ(x)$ es el logaritmo del mínimo común múltiplo de los $n$ primeros números naturales.

\begin{prop} Se tienen las relaciones
\end{prop}

\section{Teoremas de Chebyshev}
\begin{teorema}
Para $x ≥ 1$ se tiene:
\[ \prod_{p≤x} p ≤ 4^{x-1} \]
\end{teorema}
\begin{dem}
\url{https://amakelov.github.io/2015/12/23/A-neat-bound-on-the-primorial-due-to-Erdos.html}
Podemos considerar que $x$ es natural. Vamos a distinguir los casos $x$ par e impar. Supongamos primero el caso impar $x=2n+1$. Lo demostraremos por inducción. Después de comprobar unos casos base, supongamos que tengamos que la hipótesis para $2n-1$. Como $n+1≤2n$:
\[ \prod_{p≤2n+1} p = \prod_{p≤n+1} p \prod_{n+1<p≤2n+1} p ≤ 4^n \prod_{n+1<p≤2n+1} p ≤ 4^n \cdot \binom{2n+1}{n+1}≤ 4^n\cdot 2^{2n}=4^{2n}=4^x \]
Como $\prod_{n+1<p≤2n+1} p $

Para $x=2n$:
\[ \prod_{p≤2n} p = \prod_{p≤n} p \prod {n<p≤2n}p ≤ 4^{n-1}\]
\end{dem}

\begin{teorema}[Legendre]
Si $n$ es un número natural y $p$ un primo, el exponente $ν_p(n!)$ de $p$ en $n!$ es igual a
\[ ν_p(n!) = \lfloor \frac{n}{p}\rfloor + \lfloor \frac{n}{p^2}\rfloor + \lfloor \frac{n}{p^3}\rfloor + \dots \]
\end{teorema}

\begin{coro}
Hay $\min(v_2(n!),v_5(5!))$ ceros al final de la representación decimal de $n!$.
\end{coro}

\begin{lemma}
Sea $p$ un primo. Si $p^a$ es la mayor potencia de $p$ que divide a $\binom{n}{k}$, entonces $p^a ≤ n$.
\end{lemma}
\begin{dem}
Podemos extender $v_p(a/b) = v_p(a)-v_p(b)$. Entonces:
\[ v_p\binom{n}{k} = v_p(n!) - v_p(k!) - v_p((n-k)!) = \sum_{a=1} \left(\lfloor \frac{n}{p^a} \rfloor - \lfloor \frac{k}{p^a}\rfloor - \lfloor \frac{n-k}{p^a} \rfloor\right) \]
Como se tiene que: $\lfloor x+y \rfloor -\lfloor x\rfloor - \lfloor y \rfloor ≤ 1$. Entonces $ν_p\binom{n}{k}$ es un entero no negativo. Además, de que $n ≥ k$, luego los términos son ceros una vez que $p^a>n$. Sea $b$ el entero que cumple que $p^b ≤ n < p^{b+1}$, entonces:
\[ a ≤ v_p\binom{n}{k} ≤ b \]
Entonces $p^a≤p^b≤n$.
\end{dem}

\begin{coro}
Cada coeficiente binomial $\binom{n}{k} ≤ n^{π(n)}$.
\end{coro}
\begin{dem}
\[ \binom{n}{k} = \prod_{p≤n} p^{v_p\binom{n}{k}} ≤ \prod_{p≤n} n = n^{π(n)} \]
\end{dem}

No sé a cuento de qué pero voy a copiar lo que viene en la pizarra.

Si $x\geq 1$ entonces $\sum_{p\leq x} \log p \leq (x-1) \log 4$. Si definimos $a_n$ como la indicatriz de los primos, entonces 
$$\sum_{p\leq x} \log p = \sum_{n\leq x}a_n \log n = \pi(x)\log(x)-\int_1^x \pi(t)\frac{dt}{t}$$
$$\pi(x)\log(x) \leq \int_1^x \pi(t)\frac{dt}{t} + (x-1)\log 4
\leq \int_1^x t\frac{dt}{t} + (x-1)\log4 = (x-1)(\log4 +1)$$
$$ \pi(x)\leq (\log 4 +1) \frac{x-1}{\log x}\leq (1+\log 4)\frac{x}{\log x}$$
Si $x\geq 3$, $n=\lfloor x \rfloor$, $n\leq x < n+1$. Entonces
\begin{gather*}
2^x < 2^{n+1} = \sum_{k=0}^{n+1}\binom{n+1}{k} = 2+ \sum_{k=1}^n \binom{n+1}{k} \leq 2 + \sum_{k=1}^n 2n^{\pi(n)} \leq\\
\leq 2+2n^{\pi(n)+1} \leq 4n^{\pi(n)+1}\leq 4x^{\pi(x)+1}
\end{gather*}
\begin{gather*}
x \log 2 \leq 2\log 2 + (\pi(x)+1)\log x\\
 \pi(x) \geq \log 2 \frac{x}{\log x} -1 -\frac{\log 4}{\log x} = \frac{1}{2}\frac{x}{\log x} +  (\log 2 - \frac{1}{2})\frac{x}{\log x} -1 - \frac{\log4}{\log x} \geq \frac{x}{2\log{x}}
	\end{gather*}
	
Esto ya es otro resultado. Vamos a ver que $p_n \sim n \log n$. Si aplicamos la desigualdad anterior $x=p_n$ entonces:
$$
A\frac{p_n}{\log p_n} \leq \pi(p_n) = n \Rightarrow p_n \leq \frac{1}{A}n\log p_n 
$$
Tomando logaritmos es fácil ver que para algún $\delta >0$
$$
\log n \geq \log p_n - \log \log p_n - \log A \geq (1-\delta) \log p_n$$ 
$$
p_n \leq \frac{1}{A}\frac{1}{1-\delta}n \log n$$
Por otra parte, $n = \pi(p_n) \leq B \dfrac{p_n}{\log p_n}$.
$$p_n \geq \frac{1}{B}n\log p_n \geq \frac{1}{B}n \log n$$

\begin{dem}[Teorema de Martens]
Sabemos que $n! = \prod_{p\leq n} p^{\nu_p(n!)}$. Si $x\geq 1$ y $\lfloor x \rfloor = n$
\begin{gather*}
\log n! = \sum_{k\leq x} \log k = \sum_{k \leq x} 1 \cdot \log k = \lfloor x \rfloor\log x - \int_1^x \lfloor t \rfloor \frac{dt}{t} = \\ =
x\log x + O(\log x) - (x-1) + \int_1^x\frac{\{t\}}{t}dt =  x\log x - x  + O(\log x)
\end{gather*}
Ya que 
$$
 \int_1^x\frac{\{t\}}{t}dt  \leq  \int_1^x\frac{1}{t} = \log xdt 
$$
\end{dem}
\end{document}
