\documentclass[twoside]{article}
\usepackage{../estilo-ejercicios}
\newcommand{\x}{\underline{X}}
\renewcommand{\X}{\overline{\underline{X}}}

\usepackage{enumerate}
%--------------------------------------------------------
\begin{document}

\title{Análisis de Datos Multivariantes}
\author{Rafael, Diego}
\maketitle
\begin{ejercicio}{1} Sea $\x_1,\dotsc,\x_n$ una muestra aleatoria, entonces
$$
\sum_{i=1}^n (\x_i-\X) = 0
$$
\end{ejercicio}
\begin{solucion}
Basta tener en cuenta que 
$$
\sum_{i=1}^n (\x_i-\X) = \sum_{i=1}^n \x_i - \sum_{i=1}^n \X = \sum_{i=1}^n \x_i  - n \X = \sum_{i=1}^n \x_i  - n \frac{1}{n} \sum_{i=1}^n \x_i  =0
$$
\end{solucion}

\begin{proof}
Sea $t \in \R^p$. Tomemos $\hat{e}_1,\dots,\hat{e}_p$ que forman una base ortonormal de $\R^p$.
Entonces existen unos coeficientes $c_1,\dots,c_p \in \R$ tal que $t = \sum_{i=1}^p c_i \hat{e}_i = E c$.
Si $t' \cdot t = 1$, entonces, como $E$ debe ser ortogonal:
\[ 1 = c' E' E c = c' c \]
Tenemos que:
\[ \hat{\sigma}^2_{t'x} = t' \Sigma_x t = c' E' \Sigma_x E c = (c_1,\dots,c_p) \begin{pmatrix}\lambda_1 & \dots & 0\\\vdots & \vdots & \vdots\\0 & \dots & \lambda_p\end{pmatrix}\begin{pmatrix}c_1\\\vdots\\c_p\end{pmatrix} = \sum_{i=1}^p \lambda_i c_i^2 \]
Por otro lado:
\begin{equation}\label{cov-autov} \hat{\sigma}_{t'x;\hat{e}_k'x} = t' \Sigma_x \hat{e}_k = t' \lambda_k \hat{e}_k = \lambda_k c' E' \hat{e}_k = \lambda_k c' \underline{v} = \lambda_k c_k \end{equation}
donde $\underline{v} = E' \hat{e}_k$ es un vector que tiene todo ceros excepto en la posición $k$, donde tiene un $1$.

Una vez visto esto, procedemos a demostrar el teorema:
\begin{enumerate}
\item Sea $t$ tal que $t't=1$, entonces partiendo que hemos ordenado los $\lambda_1 \geq \dots \geq \lambda_p$:
\[ \hat{\sigma}^2_{t'x} = \sum_{i=1}^p \lambda_i c_i^2 \leq \lambda_1 \sum_{i=1}^p c_i^2 = \lambda_1 c' c = \lambda_1 = \hat{\sigma}_{\hat{e}_1'x}^2\]
Además, como $e_1'e_1 = 1$, esta cota se alcanza.
\qedhere

\item Sea $t \in \R^p$ tal que $t't=1$ y $t' \hat{\Sigma}\hat{e}_1 = \hat{\sigma}_{t'x;\hat{e}_1'x}=0$. Esta segunda condición es equivalente a que $c_1 \lambda_1 = 0$ por \eqref{cov-autov}.
Entonces $\lambda_1 = 0 = \lambda_2 = \dots = \lambda_p$ ó $c_1 = 0$.
El primer caso es trivial. Miramos el caso $c_1 = 0$.
Obsérvese que:
\[ \hat{\sigma}_{t'x}^2 = \sum_{i=1}^p \lambda_i c_i^2 = \sum_{i=2}^p \lambda_i c_i^2 \leq \lambda_2 \sum_{i=2}^p c_i = \lambda_2 c'c = \lambda_2 = \hat{\sigma}_{\hat{e}_2'x}^2\]
\qed

\item Inducción (queda como ejercicio).
\end{enumerate}
\end{proof}

\begin{nota}[Tema 6]
En un estadístico ji-cuadrado para la hipótesis de independencia:
\[ \chi^2 \xrightarrow{\mathcal{L}} \chi_{M-N} \]
donde $M = np-1$ es el número de parámetros linealmente independientes y $N = (n-1) + (p-1)$ es el número de parámetros linealmente independenientes bajo la hipótesis nula.

En el caso de hipotesis de homogeneidad de poblaciones, el estadístico es el mismo, pero $M = n(p-1)$ y $N=p-1$.
\end{nota}
\end{document}