\documentclass[twoside]{article}
\usepackage{../../estilo-ejercicios}
\newcommand{\media}[1]{{\overline{#1}}}
\newcommand{\muestra}[1]{{\underline{#1}}}
\newcommand{\m}[1]{{\muestra{#1}}}
\newcommand{\mX}{{\muestra{X}}}
\usepackage{csquotes}
\newcommand{\ev}{\m{\hat{e}}}
%--------------------------------------------------------
\begin{document}

\title{Ejercicios de Análisis de Datos Multivariantes}
\author{Diego Pedraza López, Javier Aguilar Martín, Rafael González López}
\maketitle

\section{Análisis de Componentes Principales}
Sea $X$ una matriz de datos $n \times p$ y $\widehat{\Sigma}$ la matriz de varianzas asociada.

\begin{enumerate}
\item Describe cual es el objetivo fundamental del ACP.

Determinar un espacio de dimensión \enquote{reducida} que represente adecuadamente un conjunto de n observaciones $p$-dimensionales. Para ello sustituimos las variables originales por un número pequeño de combinaciones lineales de las variables originales, incorreladas y \enquote{perdiendo} poca información. En ocasiones el ACP sirve como input para posteriores análisis: regresión, ANOVA, conglomerados,...

Su construcción no requiere supuesto de normalidad. No obstante, en poblaciones normales se pueden realizar tests de hipótesis y proporcionan interpretaciones útiles de los elipsoides de densidad constante.

\item Define las puntuaciones de los $n$ puntos sobre las $p$ CP. 

Definimos la puntuación de los $n$ puntos sobre las $k$-ésima componente principal como
$$
\muestra{y}_{(k)} = \begin{pmatrix}
{\m{\hat{e}}}_k'\m{x}_1\\
\vdots\\
{\m{\hat{e}}}_k'\m{x}_n
\end{pmatrix} = 
X\m{\hat{e}}_k
$$
Por tanto, las puntuaciones sobre las $p$ componentes principales son
$$
Y = \begin{pmatrix}
{\m{\hat{e}}}_1'\m{x}_1 & \cdots & {\m{\hat{e}}}_p'\m{x}_1\\\
\vdots  &  & \vdots\\
{\m{\hat{e}}}_1'\m{x}_n  & \cdots &	{\m{\hat{e}}}_p'\m{x}_n
\end{pmatrix} 
=X\hat{E}
$$

\item Demuestra que $d(\m{x}_i, \m{x}_j) = d(\muestra{y_i}, \m{y}_j)$ con $\muestra{y}_i$ el vector de puntuaciones sobre las componentes principales correspondiente al individuo $i$. 

Sabemos que $\muestra{y_i} = \hat{E}'\m{x}_i$ con $E$ una matriz ortogonal, luego se tiene de manera inmediata que 
\begin{align*}
d^2(\m{y}_i,\m{y}_j)&=(\m{y}_j-\m{y}_i)'(\m{y}_j-\m{y}_i)\\
& = (\hat{E}'\m{x}_j-\hat{E}'\m{x}_i)'(\hat{E}'\m{x}_j-\hat{E}'\m{x}_i)\\
& = (\m{x}_j-\m{x}_i)'\hat{E}\hat{E}'(\m{x}_j-\m{x}_i)\\
& = (\m{x}_j-\m{x}_i) '(\m{x}_j-\m{x}_i) = d^2(\m{x}_i,\m{x}_j)
\end{align*}

\item Demuestra que la varianza de la $j$−ésima CP es el $j$−ésimo mayor autovalor de $\widehat{\Sigma}$.

Sabemos que $$\Sigma_{\muestra{Y}} = Cov(\hat{E}'\muestra{X})= \hat{E}'Cov(\muestra{X})\hat{E} =\hat{E}' \hat{\Sigma}\hat{E}= \hat{\Lambda}$$
La varianza de la $j$-ésima componente principal es el $j$-ésimo elemento diagonal de $\hat{\Lambda}$, que es precisamente el $j$-ésimo mayor autovalor de $\hat{\Sigma}$.

\item Determina el coeficiente de correlación lineal entre $\muestra{y}_{{(i)}}$ ($i$−ésima CP) y $\muestra{x}_{(j)}$ .($j$−ésima variable).

Sea $c_j$ el $j$-ésimo vector de la base canónica de $\R^p$.
\begin{align*}
Cov(\muestra{y}_{{(i)}},\muestra{x}_{{(j)}}) &= Cov(\muestra{x}_{{(j)}},\muestra{y}_{{(i)}})\\
& = Cov(c_j'\muestra{x},\ev_i' \muestra{x})\\
&= c_j'Cov(\muestra{x},\muestra{x})\ev_i \\
&= c_j'\hat{\Sigma}\ev_i \\
&= c_j' \hat{\lambda}_i\ev_i \\
&= \hat{\lambda}_i \ev_{ij}
\end{align*} 
Por tanto, tenemos
$$
r(\muestra{y}_{{(i)}},\muestra{x}_{{(j)}}) = \frac{\hat{\lambda}_i \hat{e}_{ij}}{\hat{\sigma}_{j}\sqrt{\hat{\lambda_j}}} = \frac{\hat{e}_{ij}\sqrt{\hat{\lambda}_i}}{\hat{\sigma}_j}
$$
\newpage
\item Demuestra que la primera CP es la combinación lineal (normalizada) de máxima varianza. 

Sea $H=\{t\in \R^p \mid t't=1\}$. El enunciado del problema es equivalente a demostrar la igualdad
$$
\hat{\sigma}_{\muestra{y}_{(1)}}=\sup_{t\in H}\hat{\sigma}^2_{t'\muestra{x}}
$$

Como las columnas de $\hat{E}$ forman una base (ortonormal) de $\R^p$, para cada $t$ existe $c\in \R^p$ tal que $t = \sum_{i=1}^p c_i \hat{e}_i$. Es decir, $t=\hat{E}c$. Además, $c'c = t'EE't=t't=1$ y
$$
\hat{\sigma}^2_{t'\muestra{x}} = t'\hat{\Sigma}t =  c'\hat{E}'\hat{\Sigma}\hat{E}c = c'\hat{\Lambda}c = \sum_{i=1}^p \hat{\lambda}_i c_i^2 \leq \hat{\lambda}_1 \sum_{i=1}^p  c_i^2 = \hat{\lambda}_1 = \hat{\sigma}^2_{\muestra{y}_{(1)}}
$$
\item Demuestra que la suma de las varianzas de las variables originales es igual a la suma de las varianzas de las CP.

Sabemos que $trace(AB)=trace(BA)$. Además, es conocido que 
$$
trace(\hat{\Sigma}) = \sum_{i=1}^p \sigma_i^2 \qquad trace(\hat{\Lambda}) = \sum_{i=1}^p \hat{\lambda}_i
$$
Luego basta tener en cuenta que
$$
trace(\hat{\Sigma})=trace(\hat{E}\hat{\Lambda}\hat{E}')=trace((\hat{E}\hat{\Lambda})\hat{E}') = trace(\hat{E}'(\hat{E}\hat{\Lambda})) = trace(\hat{\Lambda})
$$
\item Demuestra que las CP están incorreladas.

Es trivial, pues la matriz de varianzas covarianzas es diagonal, luego las correlaciones entre distintas variables son nulas.
\item Considérese el conjunto de datos muestrales.
\[ \begin{pmatrix}5 & 3 & ? & ? & ?\\4 & a & ? & ? & ?\\4 & a & ? & ? & ?\\1 & a & ? & ? & ?\end{pmatrix}\]
\begin{enumerate}
	\item Determina el mayor autovalor de $\widehat{\Sigma}$ sabiendo que el autovector asociado es $(1,0,0,0,0)$.
	
	Claramente el autovalor asociado será el elemento $\hat{\Sigma}_{11}$. Si utilizamos las fórmulas de $\hat{\sigma}_{11}$ obtenemos
$$
\hat{\sigma}_{11}=\hat{\sigma}_1^2=\frac{1}{3}\sum_{i=1}^4(X_{i1}-\frac{5+4+4+1}{4})^2 = \frac{1}{3}\left[(5-\frac{14}{4})^2 +  2(4-\frac{14}{4})^2 + (1-\frac{14}{4})^2\right]= 3
$$

	\item Calcula el valor de $a$ sabiendo que el segundo autovector es $(0,1,0,0,0)$. 
	
Sabemos que la $\hat{\sigma}_{\muestra{y}_{(1)}\muestra{y}_{(2)}}=0$. Tenemos
$$
\hat{\sigma}_{\muestra{y}_{(1)},\muestra{y}_{(2)}} = \hat{\sigma}_{\hat{e}_1'\muestra{x},\hat{e}_2'\muestra{x}} = \hat{e}_1'\Sigma \hat{e}_2 = \hat{\sigma}_{12}
$$
Haciendo el cálculo de dicho estimador obtenemos
\begin{align*}
\hat{\sigma}_{12}&=\frac{1}{3}\sum_{i=1}^4(X_{i1}-\frac{14}{4})(X_{i2}-\frac{3+3a}{4})\\
&=\frac{1}{3}\left[(5-\frac{14}{4})(3-\frac{3+3a}{4})+2(4-\frac{14}{4})(a-\frac{3+3a}{4})+(1-\frac{14}{4})(a-\frac{3+3a}{4})\right]\\
&=\frac{3-a}{2}
\end{align*}
Por tanto $a=3$.

\end{enumerate}
\end{enumerate}
\newpage
\section{Análisis Factorial}
\begin{enumerate}
\item Describe cual es el objetivo básico del Análisis Factorial.

El objetivo del Análisis Factorial (AF) es describir, si es posible, la estructura de covarianzas entre diversas variables en términos de un número reducido de variables latentes, no observables, denominadas factores. Supuesto que las variables están correladas, el objetivo es reducir la redundancia usando un menor número de factores. 

Si las variables observables pueden agruparse a través de sus correlaciones (altas correlaciones dentro de los grupos y
bajas entre variables de distintos grupos) parece lógico pensar que cada grupo de variables representa una variable simple subyacente (o factor) que es responsable de
las correlaciones observadas, aunque dicho factor no sea observable.

\item Describe el Modelo Factorial Ortogonal y las hipótesis del mismo.

Sea $\muestra{X}$ un vector aleatorio (observable) $p$−dimensional, $\mX \sim (\mu,\Sigma)$. El modelo factorial postula que el vector aleatorio $\muestra{X}$ depende linealmente de \enquote{un número reducido} de variables aleatorias no observables $F_1,\dotsc,F_m$, denominadas factores comunes y $p$ fuentes de variación adicionales $\varepsilon_1, ..., ε_p$ denominados errores aleatorios o
factores específicos:
$$
\mX-\m{\mu} = L\m{F}+\m{\varepsilon}
$$
La matriz $L$ es la matriz de cargas factoriales. $l_{ij}$ es la carga factorial de la variable $i$-ésima sobre el $j$-ésimo factor. Los factores comunes pueden estar relacionados con todas las variables originales, mientras que cada factor específico está relacionado con una única variable original. Tanto los factores comunes como los factores específicos son no observables. 

Además, asumimos $\E[F]=\E[\varepsilon]=0$. $Cov[F]=I_m$, $Cov(\varepsilon)=\Psi$ matriz diagonal y $Cov(F,\varepsilon)=0$.
\item Demuestra que bajo las hipótesis del modelo:
\begin{gather}
\Sigma = L L' + \Psi \Leftrightarrow
\begin{cases}
\sigma_{ii} = l_{i1}^2 + \dots + l_{im}^2 + \psi_i\\
\sigma_{ik} = \sum_{j=1}^m l_{ij} \cdot l_{kj}
\end{cases}\\
Cov(\m{X}, \m{F}) = L \Leftrightarrow l_{ij} = Cov(X_i, F_j)
\end{gather}

La doble implicación es clara, así que demostramos la igualdad izquierda.
$$
\Sigma = Cov(\mX-\mu) = Cov(LF+\varepsilon) = LCov(F)L' + Cov(\varepsilon) = LL'+\Psi
$$
Para la segunda igualdad
$$
Cov(\mX,F) = Cov(\mX-\mu,F) = Cov(LF+\varepsilon,F) = Cov(LF,F)+Cov(\varepsilon,F) = LI_m + 0 = L
$$
\item Demuestra que si las variables originales están normalizadas

\[ \rho(X_i, F_j) = l_{ij}. \]

Si las variables están estandarizadas $\hat{\sigma}_{X_i}=1$. Además, $\sigma_{F_j}=Cov(F)_{jj} = 1$ por hipótesis del modelo. Por tanto
$$
\rho(X_i,F_j) = \frac{Cov(X_i,F_j)}{\sigma_{X_i}\sigma_{F_j}} = L_{ij} = l_{ij}
$$
\item Describe las (3) etapas básicas, y el objetivo de cada una de ellas, cuando se realiza un análisis factorial.

Las tres etapas básicas son la estimación, la rotación (interpretación) y la puntuación. Expliquemos en qué consisten:

\begin{enumerate}
\item \textbf{Estimación}. Sea $\muestra{x_1},\dotsc,\muestra{x_n}$ una muestra aleatoria de $\mX\sim (\mu,\Sigma)$. Dado $(\hat{\Sigma},m)$ nuestro objetivo es determinar $(\hat{L},\hat{\Psi})$. Para este cometido los métodos más usuales son: CP, factor principal y máxima verosimilitud.
\item \textbf{Rotación}. Las cargas factoriales obtenidas a través de una transformación ortogonal de las cargas iniciales tienen la misma capacidad para reproducir la matriz de covarianza (o correlación). Dado que una transformación ortogonal realiza una rotación de los ejes coordenados, a las transformaciones ortogonales de las cargas factoriales, y por tanto de los factores, se denomina rotación factorial o rotación de los factores.

Especialmente interesante es el caso de la rotación varimax. La interpretación de los factores se facilita si los que `afectan' (o correlacionan) a algunas variables no lo hacen a otras. La rotación varimax es aquella maximiza una medida de la variabilidad de los cuadrados de las cargas factoriales, es decir, un forma de alcanzar el propóstio anterior.

\item \textbf{Puntuación}. Teniendo en cuenta que los factores son variables no observable, un problema de interés
es estimar los valores de los factores (puntuaciones factoriales) en cada una de los $n$ elementos muestrales, o bien, sobre un nuevo individuo. Con este fin hemos visto los métodos de mínimos cuadrados ponderados y regresión.
\end{enumerate}

\item Describe el Método de las Componentes Principales para estimar la matriz de cargas factoriales y la matriz de varianzas específicas.

Sea $\muestra{x_1},\dotsc,\muestra{x_n}$ una muestra aleatoria de $\mX\sim (\mu,\Sigma)$. Dado $(\hat{\Sigma},m)$ nuestro objetivo es determinar $(\hat{L},\hat{\Psi})$. Para ello consideramos $(\hat{\lambda_i},\hat{e_i})$ los autovalores y atovectores unitarios (ordenados decrecientemente) de $\hat{\Sigma}$. Fijamos un número $m<p$. Sabemos que si los $p-m$ últimos autovalores son muy pequeños, podemos aproximar
$$
\Sigma  \approx \begin{pmatrix}
\sqrt{\lambda_1}e_1 & \cdots & \sqrt{\lambda_m}e_m 
\end{pmatrix}
\begin{pmatrix}
\sqrt{\lambda_1}e_1' \\ \vdots \\ \sqrt{\lambda_m}e_m'
\end{pmatrix} + \Psi
$$
$$
\psi_i = \sigma_{ii}-\sum_{j=1}^m l_{ij}^2
$$
Nuestra aproximación muestral consistirá en $$\hat{L}_m=\begin{pmatrix}
\sqrt{\hat{\lambda_1}}\hat{e_1} & \cdots & \sqrt{\hat{\lambda_m}}\hat{e_m}\end{pmatrix} \qquad 	\hat{\psi}_i = \hat{\sigma}_{ii}-\sum_{j=1}^m \hat{l}_{ij}^2
$$
\item Demuestra que en el Método de las Componentes Principales, las cargas factoriales estimadas no cambian cuando se incrementa el número de factores.

El resultado es claro, pues para un $m$ fijo, la estimación las cargas factoriales (elementos de $\hat{L}_m$ son $\hat{l}_{ij}=\sqrt{\hat{\lambda}_j}\hat{e}_{ji}$. Aumentar $m$ tan solo implica añadir nuevas columnas a la matriz $\hat{L}_m$, de manera que las ya existentes se mantienen inmutables. Se sigue, por tanto, que las estimaciones de las cargas factoriales ya estimadas no cambian.
\item Describe
\begin{enumerate}
	\item En qué consiste la rotación de factores.

	Dado que una transformación ortogonal realiza una rotación de los ejes coordenados, a las transformaciones ortogonales de las cargas factoriales, y por tanto de los factores, se denomina rotación factorial o rotación de los factores. Sabemos que si tenemos una estimación de $\hat{L}$ y $T$ una matriz ortogonal, entonces la matriz de $\hat{L}^* = \hat{L}T$ es una matriz de cargas rotada. En particular, mantiene la factorización de la matriz de covarianzas (o correlaciones)
	$$
	\hat{\Sigma} = \hat{L}\hat{L}' + \hat{\Psi} = \hat{L}TT'\hat{L}' + \hat{\Psi} = \hat{L}^*(\hat{L}^*)'+\Psi
	$$ 
Por lo que la matriz de residuos tampoco se ve modificada
$$
\hat{\Phi} = \hat{\Sigma} -( \hat{L}\hat{L}' + \hat{\Psi}) =  \hat{\Sigma} -(\hat{L}^*(\hat{L}^*)' + \hat{\Psi})
$$
	\item En que consiste la rotación varimax y qué pretende.
	
	La interpretación de los factores se facilita si los que afectan (o correlacionan) a algunas variables no lo hacen a otras. Buscamos una represenetación de manera que, si $(\hat{l}_{ij}^*)^2$ crece en valor absoluto, entonces $(\hat{l}_{kj}^*)^2$ decrece en valor absoluto para $k\neq i$. Una forma de alcanzar el objetivo anterior es buscar la representación con máxima variabilidad (criterio varimax) en los conjuntos de valores. Para ello, utilizamos una medida de variabilidad de las cargas factoriales asociados al factor $j$-ésimo viene dada por la varianza de $\{(\hat{l}_{kj}^*)^2\mid i=1,\dotsc,p\}$:
	$$
	V_j =\frac{1}{p}\sum_{i=1}^p \left[(\hat{l}_{kj}^*)^2-\nu_j\right]^2 \qquad \nu_j = \frac{1}{p}\sum_{i=1}^p (\hat{l}_{kj}^*)^2
	$$
La rotación varimax es la que maximiza la suma de las variabilidades de todos los factores: 
$$
V = \sum_{j=1}^m V_j
$$
Las rotaciones varimax obtenidas a partir de dos estimaciones distintas de  cargas factoriales resultan distintas. El aumento del número de factores comunes puede provocar grandes cambios en las rotación varimax
\end{enumerate}
\end{enumerate}

\newpage
\section{Análisis de Correspondencia}
\begin{enumerate}
\item 
\begin{enumerate}
\item Plantea los problemas de independencia de caracteres y de homgeneidad de poblaciones.

\begin{enumerate}
\item[] \textbf{Independencia}. Sean $(A,B)$ dos variables categóricas con $n$ y $p$ modalidades respectivamente. El problema de la independencia consiste en determinar si ambas variables son independencias, es decir, si se verifica la hipótesis
$$
H_0\colon P(A_i\cap B_j) = P(A_i)P(B_j), \; \forall i,j
$$
Seleccionamos una muestra $m$ de tamaño $N$ y consideramos $N_{ij}$ el número de elementos de $m$ que presentan $A_i\cap B_j$. Se tiene que $N_{ij}\sim Bi(N,P(A_i\cap B_j)$. En estas condiciones tenemos que el estadístico ji-cuadrado para el contraste es
$$
\chi^2 = \sum_{i=1}^n \sum_{j=1}^p \frac{(N_{ij}-e_{ij})^2}{e_{ij}}
$$
Donde $e_{ij} = \E_{H_0}(N_{ij}) = NP_{H_0}(A_i)P_{H_0}(B_j) = \dfrac{N_{i.}N_{.j}}{N}$. Nuestro estadístico $\chi^2 \conv{L}\chi^2_{(n-1)(p-1)}$. Por tanto, una región crítica para un test de tamaño $\alpha$ es $\chi^2 \geq \chi^2_{(n-1)(p-1),1-\alpha}$.
\item[] \textbf{Homogeneidad} Consideremos una variable $B$ con $p$ modalidades $B_1,\dotsc,B_p$. El problema de la homogeoneidad en $n$ poblacioens $(P_1,\dotsc,P_n)$ consiste en determinar si podemos aceptar la hipótesis
$$
H_0\colon P_1(B_j) = \cdots P_n(B_j),\;\forall j
$$
Para ello seleccionamos de forma independiente una muestra $m_i$ de tamaño $N_i$ en cada población. Sean $N_{ij}$ el número de elementos de $m_i$ que presentan $B_j$, se tiene que $N_{ij}\sim Bi(N_i,P(B_j)$. En estas condiciones tenemos que el estadístico ji-cuadrado para el contraste es
$$
\chi^2 = \sum_{i=1}^n \sum_{j=1}^p \frac{(N_{ij}-e_{ij})^2}{e_{ij}}
$$
Donde $e_{ij} = \E_{H_0}(N_{ij}) = N_{i.}P_{H_0}(B_j) = \dfrac{N_{i.}N_{.j}}{N}$. Verifica $\chi^2 \conv{L}\chi^2_{(n-1)(p-1)}$. Una región crítica para un test de tamaño $\alpha$ es $\chi^2 \geq \chi^2_{(n-1)(p-1),1-\alpha}$.
\end{enumerate}

\item Demuestra que los estadísticos $\chi^2$ asociados a ambos problemas y sus distribuciones coinciden.

Lo demostraría, pero no sé cómo. ¿¿¿¿¿¿????????????¿¿¿¿¿¿¿¿¿¿¿?????
\end{enumerate}
\item Sean $r_1', \dots, r_n'$ las distribuciones condicionadas por filas asociadas a un tabla de contingencia
\[\begin{matrix}
    & B_1    & \dots & B_j    & \dots & B_p\\
A_1 & N_{11} & \dots & N_{1j} & \dots & N_{1p} & N_{1.}\\
\vdots & \vdots & & \vdots & & \vdots & \vdots\\
A_i & N_{i1} & \dots & N_{ij} & \dots & N_{ip} & N_{i.} = \sum_{j=1}^p N_{ij}\\
\vdots & \vdots & & \vdots & & \vdots & \vdots\\
A_n & N_{n1} & \dots & N_{nj} & \dots & N_{np} & N_{n.}\\
    & N_{.1} & \dots & N_{.j} = \sum_{i=1}^n N_{ij} & \dots & N_{.p} & N
\end{matrix}\]
\begin{enumerate}
	\item Define el centro de gravedad de las filas: $m_r'$
	
	El centro de gravedad $m_r'$ de las filas de $M_r$ viene dado por 
	$$
	m_r'=\sum_{i=1}^n f_{i.}r_i' = \sum_{i=1}^n  \frac{N_{i.}}{N}\left[\frac{N_{ij}}{N_{i.}}\right]_j = \frac{1}{N}\sum_{i=1}^n \left[{N_{ij}}\right]_j = \left[\frac{N_{.j}}{N}\right]_{j}
	$$
	\item Define la distancia ji-cuadrado entre dos distribuciones.  PREGUNTAR ¿Las distribuciones son las filas de Mr? ¿O de la tabla de contigencia? 
	
	Sea $D_p = diag(f_{.1},\dotsc,f_{.p})$. Definimos la distancia $d_{\chi^2}(r_i,r_j)$ como
	$$
	d_{\chi^2}(r_i,r_j) = (r_i - r_j)'D_p^{-1}(r_i - r_j)
	$$
	\item Demuestra que $\chi^2 = N\sum_{i=1}^n f_{i.} d_{\chi^2} (r_i, m_r)$. 
	
	
	\begin{align*}
	\chi^2 &= \sum_{i=1}^n \sum_{j=1}^p \frac{(N_{ij}-e_{ij})^2}{e_{ij}} = \sum_{i=1}^n \sum_{j=1}^p \frac{(N_{ij}-\frac{N_{i.}N_{.j}}{N})^2}{\frac{N_{i.}N_{.j}}{N}}\\
	&=  \sum_{i=1}^n \sum_{j=1}^p \frac{N_{i.}^2(\frac{N_{ij}}{N_{i.}}-\frac{N_{.j}}{N})^2}{\frac{N_{i.}N_{.j}}{N}} = N \sum_{i=1}^n \sum_{j=1}^p \frac{N_{i.}(r_{ij}-f_{.j})^2}{N_{.j}}\\
	&= N \sum_{i=1}^n N_{i.}/N\sum_{j=1}^p \frac{(r_{ij}-f_{.j})^2}{N_{.j}/N} = N \sum_{i=1}^n f_{i.}\sum_{j=1}^p \frac{(r_{ij}-f_{.j})^2}{f_{.j}} \\
	&= N\sum_{i=1}^n f_{i.}d_{\chi^2}(r_i,m_r)
	\end{align*}
\end{enumerate}
\item Describe el objetivo del análisis de correspondencias (por filas).
\item Analogías y diferencias entre el ACP y el AC.
\item Plantea y resuelve el problema de optimización al que conduce el AC. 
\item Demuestra que el cuadrado de la distancia euclídea entre las proyecciones de dos filas $P_{r_i}$ y $P_{r_i'}$ coincide con la distancia \emph{ji−cuadrado} entre las filas $r_i$ y $r_i'$.
\item Concepto de vértice en el problema del AC por filas. Interpretación. 
\item Determina las coordenadas de $m_r'$ en la nueva base. 
\item Demuestra que $(0, m_r)$ es un autovalor-vector de la matriz
\[ S_r = \sum_{i=1}^n f_{i.} (r_i - m_r) (r_i - m_r)^t D_p^{-1} \]
\item Determina las coordenadas de las filas correspondientes al autovector $m_r$. ¿Qué evidencian?
\item Demuestra que la varianza ponderada de las coordenadas correspondientes al autovector $u_k$ se puede expresar como
\[ \sum_{i=1}^n f_{i.} (r_i^t D_p^{-1} u_k)^2 \]
\item Define la inercia total y demuestra que es igual a la suma de los autovalores de $S_r$.
\end{enumerate}
\newpage
\section{Análisis Discriminante}
\begin{enumerate}
\item Discriminación en dos poblaciones con distribuciones conocidas
\begin{enumerate}
	\item Define la probabilidad total de error de clasificación de una regla discriminante.
	\item Determina la regla discriminante que minimiza la probabilidad total de error de clasificación 
	\item Plantea y resuelve el problema original de Fisher. 
	\item Demuestra que supuesto que las poblaciones se distribuyen según $N(\mu_i, \Sigma)$, $i = 1, 2$, la regla discriminante que minimiza la proba- bilidad total de error coincide con la resultante de la propuesta de Fisher.
\end{enumerate}
\item Coordenadas discriminantes canónicas 
\begin{enumerate}
	\item Planteamiento y solución.
	\item Describe el criterio de clasificación basado en las coordenadas canónicas.
\end{enumerate}
\end{enumerate}
\end{document}
