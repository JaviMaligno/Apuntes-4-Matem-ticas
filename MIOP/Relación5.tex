\documentclass[twoside]{article}
\usepackage{../estilo-ejercicios}

%--------------------------------------------------------
\begin{document}

\title{Modelos de Investigación Operativa\\ Relación 5}
\author{Rafael González López, Javier Aguilar Martín}
\date{}
\maketitle

\begin{ejercicio}{1}Considerar el modelo $Y=a+bx+c^2+\varepsilon$, donde $z$ es la variable independiente, $y$ es la variable dependiente, $a,b$ y $c$ con parámetros desconocidos y $\varepsilon$ es el error experimental. La tabla siguiente da los valores de $x,y$:
\begin{center}
\begin{tabular}{|c|c|c|c|c|c|c|}
\hline
x & 0 &1 & 2 	& 3 	& 4 	& 5\\
\hline
y & 2 &2 	& -12 	& -27 & -60 &-90\\
\hline
\end{tabular}
\end{center}
Hallar los mejores valores para $a,b$ y $c$ minimizando
\begin{enumerate}
\item La suma de los errores al cuadrado.
\item La suma de los valores absolutos de los errores.
\item El máximo valor absoluto de los errores.
\end{enumerate}
\begin{solucion}
\begin{enumerate}
Buscamos 
$$
\min \sum_{i=1}^6 \varepsilon_i^2 = \min \sum_{i=0}^5(y_i -a-bx_i-c x_i)^2 = \psi(a,b,c)
$$
\begin{align*}
\frac{\partial \psi}{\partial a} &= -2\left(\sum_i y_i - 6a - \sum_i b x_i - \sum_i cx_i^2\right) = 0\\
\frac{\partial \psi}{\partial b} &= -2\sum_i x_i\left(\sum_i y_i - 6a - \sum_i b x_i - \sum_i cx_i^2\right) = 0\\
\frac{\partial \psi}{\partial c} &= -2\sum_i x_i^2\left(\sum_i y_i - 6a - \sum_i b x_i - \sum_i cx_i^2\right) = 0
\end{align*}
Resolviendo el sistema obtenemos $a=2,928$, $b=1,292$ y $c=-4,035$.
\end{enumerate}
\end{solucion}
\end{ejercicio}

\newpage 
\end{document}