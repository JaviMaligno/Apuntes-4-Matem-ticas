\documentclass[twoside]{article}
\usepackage{../estilo-ejercicios}

%--------------------------------------------------------
\begin{document}

\title{Modelos de Investigación Operativa\\ Relación 5}
\author{Rafael González López, Javier Aguilar Martín}
\date{}
\maketitle

\begin{ejercicio}{1}Considerar el modelo $y=a+bx+c^2+\varepsilon$, donde $z$ es la variable independiente, $y$ es la variable dependiente, $a,b$ y $c$ con parámetros desconocidos y $\varepsilon$ es el error experimental. La tabla siguiente da los valores de $x,y$:
\begin{center}
\begin{tabular}{|c|c|c|c|c|c|c|}
\hline
x & 0 &1 & 2 	& 3 	& 4 	& 5\\
\hline
y & 2 &2 	& -12 	& -27 & -60 &-90\\
\hline
\end{tabular}
\end{center}
Hallar los mejores valores para $a,b$ y $c$ minimizando
\begin{enumerate}
\item La suma de los errores al cuadrado.
\item La suma de los valores absolutos de los errores.
\item El máximo valor absoluto de los errores.
\end{enumerate}
\begin{solucion}
\begin{enumerate}
\item[]
\item Buscamos 
$$
\min \sum_{i=1}^6 \varepsilon_i^2 = \min \sum_{i=0}^5(y_i -a-bx_i-c x_i^2)^2 = \psi(a,b,c)
$$
Derivamos e igualamos a cero las parciales
\begin{align*}
\frac{\partial \psi}{\partial a} &= -2\left(\sum_i y_i - 6a - \sum_i b x_i - \sum_i cx_i^2\right) = 0\\
\frac{\partial \psi}{\partial b} &= -2\sum_i x_i\left(\sum_i y_i - 6a - \sum_i b x_i - \sum_i cx_i^2\right) = 0\\
\frac{\partial \psi}{\partial c} &= -2\sum_i x_i^2\left(\sum_i y_i - 6a - \sum_i b x_i - \sum_i cx_i^2\right) = 0
\end{align*}
Resolviendo el sistema obtenemos $a=2,928$, $b=1,292$ y $c=-4,035$.
\item Buscamos
$$
\min \sum_{i=1}^6 |\varepsilon_i|= \min \sum_{i=0}^5|y_i -a-bx_i-c x_i^2| 
$$
Para resolverlo consideramos el siguiente problema de programación lineal
\begin{gather*}
\min \sum_i z_i\\
z_i\geq y_i -a-bx_i-c x_i^2 \quad \forall i \\
z_i\geq -y_i +a+bx_i+c x_i^2 \quad \forall i\\
z_i \in \R \quad \forall i
\end{gather*}
\item Buscamos
$$
\min \max_i |\varepsilon_i| = \min \max_i |y_i -a-bx_i-c x_i^2|
$$
Definimos una nueva variable $z$ y resolvemos
\begin{gather*}
\min \sum z\\
z\geq |y_i -a-bx_i-c x_i^2| \quad \forall i\\
z_i \in \R \quad \forall i
\end{gather*}
\end{enumerate}
\end{solucion}
\end{ejercicio}

\newpage 

\begin{ejercicio}{2}
Supongamos que $x_{k+1}$ y $x_k$ son dos puntos consecutivos generados por el método del descenso máximo con la regla de minimización. Probar que $\nabla f(x_{k})'\nabla f(x_{k+1})=0$.
\end{ejercicio}
\begin{solucion}
\end{solucion}

\newpage

\begin{ejercicio}{3}
Sea $f:\R^n\to\R$. Considerar el problema $\min_{\lambda\in\R}f(x+\lambda d)$. Probar que una condición necesaria para que $\overline{\lambda}$ sea un mínimo es que $\nabla f(y)'d=0$, donde $y=x+\overline{\lambda}d$. ¿Cuándo es esa condición suficiente?
\end{ejercicio}

\newpage

\begin{ejercicio}{4}
Considerar el problema:
$$\min f(x,y)=100(y-x^2)^2+(1-x)^2.$$
Aplique algún método de gradiente y el método de Newton con $x_0=(-1,1)$.
\end{ejercicio}
\begin{solucion}

\newpage
\begin{ejercicio}{5}
Considere el problema irrestringido siguiente:
$$\min f(x,y)=3x^2-2xy+y^2+3e^{-x}.$$
¿Es esta función cóncava o convexa? Aplique el método de gradiente con $x^0=(0,0)$ y longitud inicial de paso $\alpha^0=2$. Calcule el mínimo aplicando el método de Newton. 
\end{ejercicio}
\begin{solucion}
\end{solucion}

\newpage
\begin{ejercicio}{6}
Considerar el problema $\min f(x,y)=3x^2+y^4$.
\begin{enumerate}
\item Aplicar una iteración del método de descenso máximo con $x^0=(1,-2)$ y longitud de paso dada por la regla de Armijo con $s=1,\sigma=0.1$ y $\beta=0.5$. 
\item Repetir con $s=1,\sigma=0.1$ y $\beta=0.1$. Comparar los resultados.
\item Aplicar una iteración del método de Newton con el mismo punto de partida.
\end{enumerate}
\end{ejercicio}
\begin{solucion}
\end{solucion}

\newpage

\begin{ejercicio}{7}
Considerar la función $f:\R^n\to\R$ dada por $f(x)=||x||^{3/2}$, y el método de descenso máximo con longitud de paso constante en cada iteración. Probar que la condicción $||\nabla f(x)-\nabla f(y)||\leq L||x-y||$ para todo $x,y$ no se verifica para ningún $L$. Además probar que este algoritmo converge a la solución óptima en un número finito de pasos o no converge a él, en este problema.
\end{ejercicio}
\begin{solucion}
\end{solucion}

\newpage

\begin{ejercicio}{8}
Sea $f$ dos veces diferenciable con continuidad. Supongamos que $x^*$ es un mínimo local tal que para toda bola abierta $S$ centrada en $x^*$ se verifica para algún $m>0$,
$$m||d||^2\leq d'\nabla^2f(x)d,\ \forall d\in\R^n.$$
Probar que para todo $x\in S$:
$$||x-x^*||\leq\frac{||\nabla f(x)||}{m},\quad f(x)-f(x^*)\leq\frac{||\nabla f(x)||^2}{m}.$$
\end{ejercicio}
\begin{solucion}
\end{solucion}

\end{solucion}



\end{document}