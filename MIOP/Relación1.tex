\documentclass[twoside]{article}
\usepackage{../estilo-ejercicios}

%--------------------------------------------------------
\begin{document}

\title{Modelos de Investigación Operativa}
\author{Rafael González López, Javier Aguilar Martín}
\maketitle

\begin{ejercicio}{1}
Una persona desea invertir 4000 u.m. y se le presentan tres opciones. Cada opción requiere
depósitos en cantidades de 1000 u.m. El inversionista puede colocar todo el dinero entre las tres.
Las ganancias esperadas se presentan en la siguiente tabla:

 \hskip 15em Inversión\
\begin{center}
\begin{tabular}{c |c c c c}
 & $1000$ & $2000$ & $3000$ & $4000$\\
 \hline
Inv. 1 & $2000$ & $5000$ & $6000$ & $7000$\\
Inv. 2 & $1000$ & $3000$ & $6000$ & $7000$\\
Inv. 3 & $1000$ & $4000$ & $5000$ & $8000$

\end{tabular}
\end{center}
Determinar la política óptima.
\begin{solucion}
Hay 3 etapas. En cada etapa se decide qué cantidad se invierte. Estados $(t,k)$ donde $t$ es la etapa en la que nos encontramos y $k$ es la cantidad de dinero que todavía podemos invertir. $f(t,k)$ es la máxima ganancia esperada desde la etapa $t$ hasta el final si se dispone de $k$ u.m.
$f(t,k)=\max_{x_t\leq k,1000|x_t}\{g(t,x_t)+f(t+1,k-x_t)\}$ y $f(3,k)=\max_{x_3\leq k}\{g(3,x_3)\}=g(3,k)$. La función $g$ representa la ganancia, es decir, el valor de la tabla en la posición $(t,x_t)$. Vamos a empezar con la tabla de la segunda etapa, pues la de la tercera es trivial.
\begin{center}
\begin{tabular}{|c| c| c| c| c|}
\hline
$k$ & $x$ & $g(2,x)$ & $f(3,k-x)$ & $f(2,k)$\\
\hline
0   &  0  & 0        &  0      &  0\\
\hline
\hline
1000    &  0  & 0       &   1000 & 1000\\
     &  1000  & 1000 & 0         & 1000\\
     \hline
     \hline
2000 &  0     & 0        &  4000 & $\boxed{4000}$\\
  &  1000     & 1000     & 1000  & 2000\\
  &  2000     & 3000     &  0    & 3000\\
  \hline
  \hline
3000 & 0	& 0	& 5000 & 5000\\
	 & 1000 & 1000 & 4000 & 5000\\
	 & 2000 & 3000 & 1000 & 5000\\
	 & 3000 & 6000 & 0  & $\boxed{6000}$\\
\hline
\hline
4000 & 0   & 0 & 8000 & $\boxed{8000}$\\
	 & 1000 & 1000 & 5000 & 6000\\
	 & 2000 & 3000 & 4000 & 7000\\
	 & 3000 & 6000 & 1000 & 7000\\
	 & 4000 & 7000 & 0 & 7000\\
	 \hline
\end{tabular}

\vspace{0.5em}

\begin{tabular}{|c|c|c|c|c|}
\hline
$k$ & $x$ & $g(1,x)$ & $f(2,k-x)$ &$f(1,k)$\\
\hline
4000 & 0   & 0       &  8000 & 8000\\
	 & 1000 & 2000   &   6000 & 8000 \\
	 & 2000 &  5000 &   4000 & $\boxed{9000}$\\
	 & 3000 & 6000  &   1000 & 7000\\
	 & 4000 & 7000 &   0     & 7000\\
	 \hline
\end{tabular}

\vspace{0.5em}
\end{center}

$x_1=2000, x_2=0, x_3=2000$.
\end{solucion}
\end{ejercicio}

\newpage 
\begin{ejercicio}{2}
Un transportista posee $8$ $m^3$ de espacio disponible en un vehículo que saldría hacia Madrid.
Un distribuidor que tiene grandes cantidades de tres artículos diferentes, todos destinados para
esa ciudad, ha ofrecido al transportista los siguientes pagos por transportar tantos artículos como
quepan en su vehículo:
\begin{center}

\begin{tabular}{c| c| c}
Artículo & Pago (u.m./art.) & Volumen ($m^3$/art.)\\
\hline
I & $11$ & $1$\\
II& $32$& $3$\\
III& $58$ & $5$\\

\end{tabular}
\end{center}

\begin{solucion}
Hay $3$ etapas, en cada una decidimos cuántos artículos de cada tipo transportamos. Los estados serán del tipo $(t,k)$, donde $t$ es la etapa en la que nos encontramos y $k$ es el espacio disponible. Entonces $f(t,k)$ es la máxima ganancia que puede obtener el transportista desde la etapa $t$ hasta la etapa $3$ si dispone de $k$ $m^3$ de espacio. Llamamos $p_t$ al pago en la etapa $t$ y $v_t$ al volumen del producto en la etapa $t$. 
\begin{align*}
f(3,k)=&\max_{x_3}\{58x_3: 5x_3\leq k\}\\
f(t,k)=&\max_{x_t:v_tx_t\leq k}\{p_tx_t+f_{t+1}(k-v_tx_t)\}
\end{align*}
\underline{Etapa 3:} $p_3=58$, $v_3=5$\
\begin{center}
\begin{tabular}{c| c c c c c c c c c}
$k$ & 8 & 7 & 6 & 5 & 4 & 3 & 2 & 1 & 0\\
\hline
$f(3,k)$ & $58$ & 58 & 58 & 58 & 0 & 0 & 0 & 0 & 0\\
$x_3$ &  $1$ & 1 & 1 & 1& 0 & 0 & 0 & 0 & 0
\end{tabular}
\end{center}
\underline{Etapa 2}: $p_2=32$, $v_2=5$, $f_2(k)=\max\{32x_2+f_3(k-3x_2):3x_2\leq k\}$.

$$f(2,8)=\begin{cases}
32\cdot 0+f(3,8)=58\\
32\cdot 1+f(3,5)=90\\
32\cdot 2+f(3,2)=64
\end{cases}\quad f(2,7)=\begin{cases}
32\cdot 0+f(3,7)=58\\
32\cdot 1+f(3,4)=32\\
32\cdot 2+f(3,1)=64
\end{cases}$$

$$f(2,6)=\begin{cases}
32\cdot 0+f(3,6)=58\\
32\cdot 1+f(3,3)=32\\
32\cdot 2+f(3,0)=64
\end{cases}\quad f(2,5)=\begin{cases}
32\cdot 0+f(3,5)=58\\
32\cdot 1+f(3,2)=32\\
\end{cases}$$
\begin{center}
\begin{tabular}{c| c c c c c c c c c}
$k$ & 8 & 7 & 6 & 5 & 4 & 3 & 2 & 1 & 0\\
\hline
$f_2(k)$ & 90 & 64 & 64 & 58 & 32 & 32 & 0 & 0 & 0\\
$x_2$ &  $1$ & 2 & 2 &    1&  0 & 1 &   1 & 0 & 0
\end{tabular}\
\end{center}
\underline{Etapa 1:}
$$f_1(8)=\max_{x_1\leq 8}\{11x_1+f_2(8-x_1)\}=\begin{cases}
f(2,8)=90\\
11+f(2,7)=75\\
22+f(2,6)=86\\
33+f(2,5)=\boxed{91}\\
44+f(2,4)=76\\
55+f(2,3)=87\\
66+f(2,2)=66\\
77+f(2,1)=77\\
88+f(2,0)=88
\end{cases}$$

Hemos llegado a que la ganancia óptima es $91$, con la configuración $x_1=3,x_2=0, x_3=1$.
\end{solucion}

\end{ejercicio}

\newpage 
\begin{ejercicio}{3}
Una pequeña compañía puede fabricar hasta 4 ordenadores por semana y se ha comprometido
a entregar en cada una de las siguientes 4 semanas tres, dos, cuatro y dos ordenadores, respectivamente.
Los costos de producción están en función del número de ordenadores fabricados y se dan
(en miles de u.m.) como sigue
\begin{center}
\begin{tabular}{c| c c c c c}
Unidades producidas & 0 & 1 & 2 & 3 & 4 \\
\hline
Costo & 4 & 13 & 19 & 27 & 32
\end{tabular}
\end{center}
Los ordenadores pueden entregarse a los consumidores al final de la misma semana en que se
fabrican o pueden almacenarse para su entrega futura, con un costo de 400 u.m. por semana.
Debido a la capacidad limitada de almacenamiento, la compañía no puede almacenar más de tres
ordenadores a un tiempo. El inventario actual es cero y la compañía no desea ningún inventario
al final de la semana 4. Determinar el número de ordenadores a fabricar en cada semana, para
cumplir la demanda a un costo mínimo total.
\begin{solucion}
Las demandas son $d_1=3, d_2=2, d_3=4, d_4=2$. Hay 4 etapas, en cada una de las cuales se decide el número de ordenadores que se fabrican esa semana. Los estados son $(t, I)$ donde $t$ es la etapa e $I$ es el espacio en el inventario. El coste de almacenamiento, en miles, es 0.4. $f(t,I)$ es el mínimo coste de producción y almacenamiento desde la etapa $t$ hasta el final si tenemos en inventario $I$ ordenadores.
$$f(t,I)=\min_{0\leq I+x-d_t\leq 3,x\leq 4}\{c(x)+0.4(I-d_t+x)+f(t+1,I-d_t+x)\}\quad t=1,2,3$$
Se podría interpretar que el inventario se paga al principio de la etapa y entonces el coste de inventario sería simplemente $0.4I$ añadiéndole esa cantidad también a la función siguiente. 
$$f(4,I)=\begin{cases}
+\infty & I>2\\
c(d_4-I) & I\leq 2
\end{cases}$$
Otra opción es añadir una etapa extra de forma que 
$$f(5,I)=\begin{cases}
0 & I=0\\
+\infty & I\neq 0
\end{cases}$$
En tal caso, para $t=4$ seguiríamos usando la función recursiva. Continuemos con la resolución del problema.
\newpage
\underline{Etapa 4}
\begin{center}
\begin{tabular}{|c| c| c|}
\hline
$I$ & $x$ & $f(4,I)$\\
\hline
0 & 2 & 19\\
\hline
1 & 1 & 13\\
\hline
2 & 0 & 4\\
\hline
\end{tabular}
\end{center}
\underline{Etapa 3}
\begin{center}
\begin{tabular}{|c| c| c|c|c|}
\hline
$I$ & $x$ & $c(x)+0.4(I-4+x)$ & $f(4,I-4+x)$ & $f(3,I)$\\
\hline
0 & 4 & 32 & 19 & 32+19\\
\hline
\hline
1 & 4 & 32+0.4 & 13 & $\boxed{32.4+13}$\\
\hline
 & 3 & 27 & 19 & 27+19\\
\hline
\hline
2 & 4 & 32+0.4 $\cdot$ 2 & 4 & $\boxed{32.8+4}$\\
\hline
  & 3 & 27+0.4 & 13 & 27.4+13\\
  \hline
  & 2 & 19+0.4 & 19 & 19+19\\
  \hline
  \hline
3 & 3 & 27+0.8 & 4 & $\boxed{31.8}$\\
\hline
 &  2 & 19+0.4 & 13 & 19.4+13\\
 \hline
 & 1 & 13 & 19 & 13.4+19\\
 \hline
\end{tabular}
\end{center}
\underline{Etapa 2}
\begin{center}
\begin{tabular}{|c|c|c|c|c|}
\hline
$I$ & $x$ & $c(x)+0.4(I-2+x)$ & $f(3,I-2+x)$ & $f(2,I)$\\
\hline
0 & 4 & 32+0.4$\cdot$2 & 36.8 & $\boxed{69.6}$\\
  & 3 &     27+0.4          &    45.4     &72.8 \\
  & 2 &                &        & \\
\hline
\hline
1 & 4  &   &    & \\
  & 3 &   &     & \\
  & 2 &   &    & \\
  & 1 & 4 & 51 & $\boxed{51}$\\
  \hline
\end{tabular}
\end{center}
\underline{Etapa 1}
\begin{center}
\begin{tabular}{|c|c|c|c|c|}
\hline
$I$ & $x$ & $c(x)+0.4(I-3+x)$ & $f(2,I-3+x)$ & $f(1,0)$\\
\hline
0 & 4 &  & 64 & $\boxed{96.4}$\\
  & 3 &                &       & \\
\hline
\end{tabular}
\end{center}
$x_1=4, x_2=1, x_3=4, x_4=2$.
\end{solucion}
\end{ejercicio}	


\newpage 
\begin{ejercicio}{4}
Un estudiante tiene siete días para preparar sus exámenes finales de cuatro asignaturas y
quiere asignar el tiempo que tiene de la manera más eficiente posible. Necesita por lo menos un
día para cada asignatura por lo que quiere asignar uno,dos,tres o cuatro días a cada asignatura.
Utilizar programación dinámica para realizar estas asignaciones de forma que se maximice el total
de puntos obtenidos en los cuatro exámenes. La siguiente tabla muestra la estimación de los puntos
obtenidos en cada asignatura según el número de días de estudio asignados.
\begin{center}
 Asignaturas\\
\begin{tabular}{c|c c c c}
Nº días & 1 & 2 & 3 & 4\\
\hline
1 & 3 & 5 & 2 & 6\\
2 & 5 & 5 & 4 & 7\\
3 & 6 & 6 & 7 & 9\\
4 & 7 & 9 & 8 & 9
\end{tabular}
\end{center}
Resolver el problema mediante programación dinámica.
\begin{nota}
¿Cómo se haría si quisiera aprobar todas como condición inicial?
\end{nota}
\begin{solucion}
Tenemos 4 etapas, una para cada asignatura, en la que decidimos cuántos días le dedicamos. Por tanto tenemos los estados $(t,k)$, donde $t$ es la etapa en la que nos encontramos y $k$ es la cantidad de días disponible para asignar (teniendo en cuenta que ya se ha asignado un día previamente). $f(t,k)$ representaría la máxima cantidad de puntos que podemos obtener desde la etapa $t$ hasta el final si disponemos de $k$ días libres. Entonces definimos
$$f(t,k)=\max_{1\leq x_t\leq k}\{g(t,x_t)+f(t+1,k-x_t)\}$$ y $f(4,k)=\max_{1\leq x_3\leq k}\{g(4,x_3)\}=g(4,k)$, donde $g(t,x)$ representa los puntos obtenidos, es decir, el valor de la tabla en la posición $(t,x)$.
\underline{Etapa 4}
\begin{center}
\begin{tabular}{|c|c|c|}
\hline
$k$ & $x$ & $f(4,k)$\\
\hline
1 & 1 & 6\\
2 & 2 & 7\\
3 & 3 & 9\\
4 & 4 & 9\\
\hline
\end{tabular}
\end{center}
\newpage
\underline{Etapa 3}
\begin{center}
\begin{tabular}{|c|c|c|c|c|c|}
\hline
$k$ & $x$ & $g(3,x)$ & $f(4,k-x)$ & $f(3,k)$\\
\hline
2 & 1 & 2 & 6 & $\boxed{8}$\\
\hline
3 & 1 & 2 & 7 & 9\\ 
  & 2 & 4 & 6 & $\boxed{10}$\\
\hline
4 & 1 & 2 & 9 & 11\\
  & 2 & 4 & 7 & 11\\
  & 3 & 7 & 6 & $\boxed{13}$\\
  \hline
5 & 1 & 2 & 9 & 11\\
  & 2 & 4 & 9 & 13\\
  & 3 & 7 & 7 & $\boxed{14}$\\
  & 4 & 8 & 6 & 14\\
  \hline
\end{tabular}
\end{center}
\underline{Etapa 2}
\begin{center}
\begin{tabular}{|c|c|c|c|c|}
\hline
$k$ & $x$ & $g(2,x)$ & $ f(3,k-x)$ & $f(2,k)$\\
\hline 
3 & 1 & 5 & 8 & $\boxed{13}$\\
\hline
4 & 1 & 5 & 10 & $\boxed{15}$\\
  & 2 & 5 & 8 & 13\\
  \hline
5 & 1 & 5 & 13 & $\boxed{18}$\\
  & 2 & 5 & 10 & 15\\
  & 3 & 6 & 8 & 14\\
  \hline
6 & 1 & 5 & 14 & $\boxed{19}$\\
  & 2 & 5 & 13 & 18\\
  & 3 & 6 & 10 & 16\\
  & 4 & 9 & 8  & 17\\
  \hline
\end{tabular}
\end{center}
\underline{Etapa 1}
\begin{center}
\begin{tabular}{|c|c|c|c|c|}
\hline
$k$ & $x$ & $g(1,x)$ & $f(2,k-x)$ & $f(1,k)$\\
\hline
7 & 1 & 3 & 19 & 22\\
  & 2 & 5 & 18 & $\boxed{23}$\\
  & 3 & 6 & 15 & 21\\
  & 4 & 7 & 13 & 20\\
  \hline
\end{tabular}
\end{center}
$x_1=2, x_2=1, x_3=3, x_4=1$.
\end{solucion}
\end{ejercicio}

\newpage 
\begin{ejercicio}{5}
En una empresa se estiman las necesidades de mano de obra para las cuatro estaciones del año.
Estas son 225, 220, 240 y 200 hombres respectivamente en primavera, verano, otoño e invierno.
Si se emplean a más hombres de los necesarios se produce un coste de 2000 u.m. por persona
y periodo. Los costos debidos a la contratación o al despido de trabajadores entre periodos se
estiman en 200 veces el cuadrado de la diferencia entre el número de trabajadores en uno y otro
periodo.
Determinar el número de empleados que deben estar en plantilla en cada estación del año; para
ello asumir que la cantidad de trabajadores es continuamente divisible.
\begin{solucion}
\end{solucion}
\end{ejercicio}

\newpage 
\begin{ejercicio}{6}\label{6}
Descomponer utilizando la programación dinámica una cantidad $C > 0$ en tres partes de forma
que cada una de ellas sea mayor o igual que $C/4$ y que su producto sea mínimo. Definir las etapas,
estados, decisiones y recursividad de este proceso.
\end{ejercicio}
\begin{solucion}
Primero vamos a plantearlo como un problema de programación lineal. Sean $x_1,x_2,x_3$ cada una de las 3 partes, las cuales cumplen $x_1+x_2+x_3=C$ y $x_1,x_2,x_3\geq\frac{C}{4}$. Con esta restricción tendríamos que calcular $\min x_1x_2x_3$. 

Vamos a simplificarlo tomando $C=1$. Para programación dinámica tendríamos 3 etapas, cada una correspondiente a cada $x_i$. En los estados tenemos el índice $i$ y $d=1-\sum_{j=1}^ix_j$. La función recursiva por lo tanto sería
\begin{align*}
f(i,d)=&\min_{\frac{1}{4}\leq d-\frac{3-i}{4}} x_if(i+1,d-x_i)\\
f(3,d)=&\begin{cases}
d & d\geq\frac{1}{4}\\
+\infty & d<\frac{1}{4}
\end{cases}
\end{align*}

Así, pues $f(2,d)=\min_{\frac{1}{4}\leq d-\frac{1}{4}}\begin{cases}
x(d-x) & d-x\geq\frac{1}{4}\\
+\infty & d-x <\frac{1}{4}
\end{cases}$
Por tanto, el mínimo se alcanza en  se alcanza tanto en $\frac{1}{4}$ como en $d-\frac{1}{4}$. Elegimos por simplicidad $\frac{1}{4}$, con lo que sustituyendo nos da $f(2,d)=\frac{1}{4}(d-\frac{1}{4})$ si $d\geq\frac{1}{2}$ e infinito en caso contrario. Por último, en el primer paso tenemos la cantidad completa, por lo tanto calculamos
$$f(1,1)=\min_{\frac{1}{4}\leq x\leq\frac{1}{2}}xf(2,1-x)=\begin{cases}
x\frac{1}{4}(1-x-\frac{1}{4}) & 1-x\geq\frac{1}{2}\\
+\infty & cc
\end{cases}$$
Como $x\leq\frac{1}{2}$ era una de las condiciones, el segundo caso podemos descartarlo. El mínimo se alcanza tanto en $\frac{1}{2}$ como en $\frac{1}{4}$, cuyo resultado es $\frac{1}{32}$. Vamos a elegir la solución $x_1=\frac{1}{4}$. Habíamos obtenido también $x_2=\frac{1}{4}$, por lo que obtenemos $x_3=\frac{1}{2}$.
\end{solucion}

\newpage 
\begin{ejercicio}{7}
Un concurso de televisión consiste en un juego bipersonal en el que cada participante debe
eliminar números dispuestos en una tabla. Inicialmente la tabla consta de tres filas, en la primera
hay ocho números, en la segunda cinco y en la tercera tres. El primer participante elige una fila de
las tres existentes y elimina uno, dos o tres números; a continuación es el turno del otro participante
que repite la elección de fila y la cantidad de números que elimina de la misma. El juego continua
hasta que uno de los participantes elimina el último número, en cuyo caso pierde la partida.
Formular el sistema de ecuaciones recursivas que determina la estrategia ganadora óptima.
\end{ejercicio}
\begin{solucion}

\end{solucion}

\newpage
\begin{ejercicio}{8} 
Consideremos la matriz 4$\times$5 que se muestra abajo. Definimos una vez fijada la posición $(i, j)$,
un movimiento elemental como aquel que pasa a la posición $(i + 1, j)$ o $(i, j + 1)$. Suponiendo
que estamos en el córner inferior derecho, determinar la secuencia de movimientos que minimiza la
suma de los valores almacenados en las posiciones recorridas desde la posición superior izquierda.
\begin{center}
\begin{tabular}{c c c c c}
0 & 4 & 3 & 6 & 4\\
7 & 8 & 6 & 8 & 8\\
2 & 3 & 1 & 8 & 7\\
6 & 2 & 9 & 3 & 0
\end{tabular}
\end{center}
\end{ejercicio}
\begin{solucion}
En Xpress
\begin{verbatim}
model hoja1ejercicio8
uses "mmxprs"; !gain access to the Xpress-Optimizer solver

!optional parameters section
parameters
n=4
m=5
end-parameters

!sample declarations section
declarations
	c,f:array(1..n,1..m) of integer
end-declarations
c::[0, 4, 3, 6, 4,
	7, 8, 6, 8, 8,
	2, 3, 1, 6, 7,
	6, 2, 9, 3, 0]
writeln(c) !muestra por pantalla
writeln("c::")
forall(i in 1..n) do
	forall(j in 1..m) write (c(i,j)," ")
	writeln !Salto de linea
end-do

f(1,1):=c(1,1)
forall(i in 2..n)f(i,1):=c(i,1)+f(i-1,1)
forall(j in 2..m)f(1,j):=c(1,j)+f(1,j-1)
forall(i in 2..n, j in 2..m) do
	f(i,j):=c(i,j)+f(i-1,j)
	if(c(i,j)+f(i,j-1)<f(i,j))then
	f(i,j):=c(i,j)+f(i,j-1)
	end-if
end-do
end-model

\end{verbatim}
\end{solucion}

\newpage

\begin{ejercicio}{9}
Sea $p$ la probabilidad de obtener cara al lanzar una moneda. Calcular la probabilidad de
obtener un número par de caras en $n$ lanzamientos mediante un procedimiento recursivo.
\end{ejercicio}
\begin{solucion}
Consideramos los sucesos 

$A_n=$ obtener un número par de caras en $n$ lanzamientos.\\
$B_n=$ obtener un número impar de caras en $n$ lanzamientos.

Tenemos que calcular $P(A_n)$ de la siguiente forma
$$P(A_n)=P(\text{cara}|B_{n-1})P(B_{n-1})+P(\text{cruz}|A_{n-1})P(A_{n-1})=pP(B_{n-1})+(1-p)P(A_{n-1}).$$
En primer lugar necesitamos $P(B_{n-1})$, que la calculamos de forma similar
$$P(B_{n-1})=P(\text{cara}|A_{n-2})P(A_{n-2})+P(\text{cruz}|B_{n-2})P(B_{n-2})=pP(A_{n-2})+(1-p)P(B_{n-2}).$$
Solo nos queda añadir las condiciones iniciales, $P(A_1)=1-p$ y $P(B_1)=p$. 

En resumen, el proceso recursivo sería el siguiente:
$$\begin{cases}
P(B_1)=p\\
P(A_1)=1-p\\
P(B_{n-1})=pP(A_{n-2})+(1-p)P(B_{n-2})\\
P(A_n)=pP(B_{n-1})+(1-p)P(A_{n-1})
\end{cases}$$
\end{solucion}

\newpage

\begin{ejercicio}{10}
Una comunidad autónoma tiene cuatro provincias. Una fracción $f_i$ de los electores viven en
la provincia $i = 1,\dots,4$ con $f_1 = 0.25$, $f_2 = 0.2$, $f_3 = 0.4 $y $f_4 = 0.15$. La cámara autónimca consta
de $R = 28$ escaños. Cada provincia $i$ debería obtener $D_i = Rf_i$ escaños. Sin embargo, el número
de escaños asignados debe ser entero. Si se desea minimizar la máxima diferencia entre $D_i$ y el
número de escaños de la provincia $i$,
\begin{enumerate}
\item Formular el problema como un modelo de programación dinámica.
\item Resolver el problema.
\end{enumerate}
\end{ejercicio}
\begin{solucion}
Tengamos en cuenta que $D_1=7$, $D_2=5.6$, $D_3=11.2$ y $D_4=4.2$. Los estados $(i,d)$ contienen el índice de la provincia y los escaños disponibles.
$f(i,d)$ sería el mínimo valor de la máxima diferencia entre los escaños asignados y los que en teoría le corresponderían, desde la provincia $i$ hasta la 4, si en el momento hay disponibles $d$ escaños. 

$$f(i,d)=\min_{x_i\leq d}\max\{|D_i-x_i|,f(i+1,d-x_i)\}$$
$$f(4,d)=\begin{cases}
0.2 & d\geq 4,x_4=4\\
4.2-d & d<4, x_4=d
\end{cases}$$

$$f(3,d)=\min_{x_i\leq d}\max\{|11.2-x_3|,f(4,d-x_3)\}$$

$f(3,0)=\min\max\{11.2,4.2\}=11.2;\quad f(3,1)=\min\begin{cases}
\max\{11'2,3'2\}=11'2\\
\max\{10'2,4'2\}=10'2
\end{cases}=10'2\quad x_3=1;$

$f(3,2)=\min\begin{cases}
\max\{11'2,2'2\}=11.2 & x_3=0\\
\max\{10'2,3'2\}=10.2 & x_3=1\\
\max\{9'2,4'2\} & x_3=2
\end{cases}=9'2, x_3=2$

Los siguiente siguen una progresión similar, así que pasamos directamente a 
$$f(3,8)=\min_{x_3\leq 8}\left.\begin{cases}
11'2 & x_3=0\\
10'2 & x_3=1\\
\vdots & \\
\max\{4'2,3'2\}=4'2 & x_3=7\\
\max\{3'2,3'2\}=3'2 & x_3=8\\
\max\{2'2,4'2\}=4'3 & x_3=9
\end{cases}\right\}= 3.2\quad x_3=8$$

La función sigue aumentando hasta llegar a 15 y se mantiene constante. La solución será 
$$f(1,28)=\min_{x_1\leq 28}\max\{|7-x_1|,f(2,28-x_1)\}=\{0,0'4\}=0'4$$
que se obtiene en $f(2,21)=0'4$, por lo que la solución óptima es 
$$\boxed{x_1=7, x_2=6,x_3=11,x_4=4}$$

\end{solucion}

\newpage

\begin{ejercicio}{extra}
Aplicar un procedimiento de programación dinámica para resolver el problema:
\begin{align*}
\min\sum_{i=1}^N (x_i-bx_i)^2\\
s.a. \sum_{i=1}^N x_i=C\\
x\geq 0
\end{align*}
\end{ejercicio}
\begin{solucion}
Obsérvese en primer lugar que $$\sum_{i=1}^N (x_i-bx_i)^2=\sum_{i=1}^N (1-b)^2x_i^2=(1-b)^2\sum_{i=1}^N x_i^2,$$ por lo que basta minimizar la suma de los cuadrados con las restricciones anteriores. Vamos a identificar $N$ etapas, cada una correspondiente al valor asignado a la variable $x_i$. Definimos los estados $(i,d)$, donde $i$ es el índice de la variable en la que nos encontramos y $d=C-\sum_{j=1}^i x_j$ es la cantidad que queda disponible para asignar a la variable. En nuestro modelo, $f(i,d)$ representará el valor mínimo de $\sum_{j=i}^N x_i^2$ si en la etapa $i$ hay una cantidad $d$ para asignar. La función recursiva será $f(i,d)=\min_{0\leq x_i\leq d}  x_i^2+f(i+1,d-x_i)$, teniendo en cuenta que $f(N,d)=\begin{cases}
C^2 & d\geq C\\
d^2 & C\geq d\geq 0\\
+\infty & d<0
\end{cases}$. 

Vamos a probar por inducción que $f(N-i,d)=\frac{d^2}{i+1}$, alcanzando dicho valor en $x_i=\frac{d}{i+1}$, $\forall i=0,\dots, N-1$ y $0\leq d\leq C$. Tenemos como caso inicial que $f(N-0,d)=\frac{d^2}{0+1}$, valor alcanzado en $x_N=\frac{d}{0+1}$. Supongamos como hipótesis de inducción que

$$f(N-i,d)=\min_{0\leq x_{N-i}\leq d}x_{N-i}^2+(d-x_{N-i})^2$$
obtiene su mínimo en $x_{N-i}=\frac{d}{i+1}$, donde vale $\frac{d^2}{i+1}$. Entonces

$$f(N-(i+1),d)=\min_{0\leq x_{N-(i+1)}\leq C}x_{N-(i+1)}^2+\frac{(d-x_{N-(i+1)})^2}{i+1}$$
La función es una parábola que alcanza su mínimo en $x_{N-(i+1)}=\frac{d}{i+2}$, donde vale $\frac{d^2}{i+2}$, lo que prueba nuestra tesis.

Como la solución es $f(1,C)$ basta sutituir estos valores y obtenemos $f(1,C)=\frac{C^2}{N}$, para $x_1=\frac{C}{N}$. Para recuperar los demás $x_i$ probaremos por inducción que $x_i=\frac{C}{N}\ \forall i=1,\dots,N$. El caso inicial es el que acabamos de calcular. Fijemos $2\leq j\leq N$ y supongamos entonces que $x_i=\frac{C}{N}\ \forall i=1,\dots,j-1$. Tenemos que $j=N-(N-j)$ y que queda una cantidad $d=C-\sum_{i=1}^{j-1}\frac{C}{N}=\frac{N-j+1}{N}C$. Por tanto sustituimos en la fórmula obtenida en la anterior inducción y resulta
$$x_j=\frac{\frac{N-j+1}{N}C}{N-j+1}=\frac{C}{N}$$
como queríamos demostrar.
\end{solucion}


\end{document}