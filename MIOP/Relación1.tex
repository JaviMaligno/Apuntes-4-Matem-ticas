\documentclass[twoside]{article}
\usepackage{../estilo-ejercicios}

%--------------------------------------------------------
\begin{document}

\title{Modelos de Investigación Operativa}
\author{Rafael González López, Javier Aguilar Martín}
\maketitle

\begin{ejercicio}{1}
Una persona desea invertir 4000 u.m. y se le presentan tres opciones. Cada opción requiere
depósitos en cantidades de 1000 u.m.. el inversionista puede colocar todo el dinero entre las tres.
Las ganancias esperadas se presentan en la siguiente tabla:

 \hskip 5em Inversión\

\begin{tabular}{c |c c c c}
 & $1000$ & $2000$ & $3000$ & $4000$\\
 \hline
Inv. 1 & $2000$ & $5000$ & $6000$ & $7000$\\
Inv. 2 & $1000$ & $3000$ & $6000$ & $7000$\\
Inv. 3 & $1000$ & $4000$ & $5000$ & $8000$

\end{tabular}

Determinar la política óptima.
\begin{solucion}
Hay 3 etapas. En cada etapa se decide qué cantidad se invierte. Estados $(t,k)$ donde $t$ es la etapa en la que nos encontramos y $k$ es la cantidad de dinero que todavía podemos invertir. $f(t,k)$ es la máxima ganancia esperada desde la etapa $t$ hasta el final si se dispone de $k$ u.m.
$f(t,k)=\max_{x_t\leq k,1000|x_t}\{g(t,x_t)+f(t+1,k-x_t)\}$ y $f(3,k)=\max_{x_3\leq k}\{g(3,x_3)\}=g(3,k)$. La función $g$ representa la ganancia, es decir, el valor de la tabla en la posición $(t,x_t)$. Vamos a empezar con la tabla de la segunda etapa, pues la de la tercera es trivial.

\begin{tabular}{|c| c| c| c| c|}
\hline
$k$ & $x$ & $g(2,x)$ & $f(3,k-x)$ & $f(2,k)$\\
\hline
0   &  0  & 0        &  0      &  0\\
\hline
\hline
1000    &  0  & 0       &   1000 & 1000\\
     &  10000  & 1000 & 0         & 1000\\
     \hline
     \hline
2000 &  0     & 0        &  4000 & $\boxed{4000}$\\
  &  1000     & 1000     & 1000  & 2000\\
  &  2000     & 3000     &  0    & 3000\\
  \hline
  \hline
3000 & 0	& 0	& 5000 & 5000\\
	 & 1000 & 1000 & 4000 & 5000\\
	 & 2000 & 3000 & 1000 & 5000\\
	 & 3000 & 6000 & 0  & $\boxed{6000}$\\
\hline
\hline
4000 & 0   & 0 & 8000 & $\boxed{8000}$\\
	 & 1000 & 1000 & 5000 & 6000\\
	 & 2000 & 3000 & 4000 & 7000\\
	 & 3000 & 6000 & 1000 & 7000\\
	 & 4000 & 7000 & 0 & 7000\\
	 \hline
\end{tabular}

\vspace{0.5em}

\begin{tabular}{|c|c|c|c|c|}
\hline
$k$ & $x$ & $g(1,x)$ & $f(2,k-x)$ &$f(1,k)$\\
\hline
4000 & 0   & 0       &  8000 & 8000\\
	 & 1000 & 2000   &   6000 & 8000 \\
	 & 2000 &  5000 &   4000 & $\boxed{9000}$\\
	 & 3000 & 6000  &   1000 & 7000\\
	 & 4000 & 7000 &   0     & 7000\\
	 \hline
\end{tabular}

\vspace{0.5em}

$x_1=2000, x_2=0, x_3=2000$.
\end{solucion}
\end{ejercicio}

\newpage 
\begin{ejercicio}{2}
Un transportista posee $8$ $m^3$ de espacio disponible en un vehículo que saldría hacia Madrid.
Un distribuidor que tiene grandes cantidades de tres artículos diferentes, todos destinados para
esa ciudad, ha ofrecido al transportista los siguientes pagos por transportar tantos artículos como
quepan en su vehículo:

\begin{tabular}{c| c| c}
Artículo & Pago (u.m./art.) & Volumen ($m^3$/art.)\\
\hline
I & $11$ & $1$\\
II& $32$& $3$\\
III& $58$ & $5$\\

\end{tabular}

\begin{solucion}
Hay $3$ etapas, en cada una decidimos cuántos artículos de cada tipo transportamos. Los estados serán del tipo $(t,k)$, donde $t$ es la etapa en la que nos encontramos y $k$ es el espacio disponible. Entonces $f_t(K)$ es la máxima ganancia que puede obtener el transportista desde la etapa $t$ hasta la etapa $3$ si dispone de $k$ $m^3$ de espacio. Llamamos $p_t$ al pago en la etapa $t$ y $v_t$ al volumen del producto en la etapa $t$. 
\begin{align*}
f_3(k)=&\max_{x_3}\{58x_3: 5x_3\leq k\}\\
f_t(k)=&\max_{x_t:v_tx_t\leq k}\{p_tx_t+f_{t+1}(k-v_tx_t)\}
\end{align*}
\underline{Etapa 3:} $p_3=58$, $v_3=5$\

\begin{tabular}{c| c c c c c c c c c}
$k$ & 8 & 7 & 6 & 5 & 4 & 3 & 2 & 1 & 0\\
\hline
$f_3(k)$ & $58$ & 58 & 58 & 58 & 0 & 0 & 0 & 0 & 0\\
$x_3$ &  $1$ & 1 & 1 & 1& 0 & 0 & 0 & 0 & 0
\end{tabular}

\underline{Etapa 2}: $p_2=32$, $v_2=5$, $f_2(k)=\max\{32x_2+f_3(k-3x_2):3x_2\leq k\}$.

$f(2,8)=\begin{cases}
32\cdot 0+f(3,8)=58\\
32\cdot 1+f(3,5)=90\\
32\cdot 2+f(3,2)=64
\end{cases}$\quad $f(2,7)=\begin{cases}
32\cdot 0+f(3,7)=58\\
32\cdot 1+f(3,4)=32\\
32\cdot 2+f(3,1)=64
\end{cases}$

$f(2,6)=\begin{cases}
32\cdot 0+f(3,6)=58\\
32\cdot 1+f(3,3)=32\\
32\cdot 2+f(3,0)=64
\end{cases}$\quad $f(2,5)=\begin{cases}
32\cdot 0+f(3,5)=58\\
32\cdot 1+f(3,2)=32\\
\end{cases}$\quad etc.

\begin{tabular}{c| c c c c c c c c c}
$k$ & 8 & 7 & 6 & 5 & 4 & 3 & 2 & 1 & 0\\
\hline
$f_2(k)$ & 90 & 64 & 64 & 58 & 32 & 32 & 0 & 0 & 0\\
$x_2$ &  $1$ & 2 & 2 &    1&  0 & 1 &   1 & 0 & 0
\end{tabular}\

\underline{Etapa 1:}
$f_1(8)=\max_{x_1\leq 8}\{11x_1+f_2(8-x_1)\}=\begin{cases}
f_2(8)=90\\
11+f_2(7)=75\\
22+f_2(6)=86\\
33+f_2(5)=\boxed{91}\\
44+f_2(4)=76\\
55+f_2(3)=87\\
66+f_2(2)=66\\
77+f_2(1)=77\\
88+f_2(0)=88
\end{cases}$

Hemos llegado a que la ganancia óptima es $91$, con la configuración $x_1=3,x_2=0, x_3=1$.
\end{solucion}

\end{ejercicio}

\newpage 
\begin{ejercicio}{3}
Una pequeña compañía puede fabricar hasta 4 ordenadores por semana y se ha comprometido
a entregar en cada una de las siguientes 4 semanas tres, dos, cuatro y dos ordenadores, respectivamente.
Los costos de producción están en función del número de ordenadores fabricados y se dan
(en miles de u.m.) como sigue

\begin{tabular}{c| c c c c c}
Unidades producidas & 0 & 1 & 2 & 3 & 4 \\
\hline
Costo & 4 & 13 & 19 & 27 & 32
\end{tabular}

Los ordenadores pueden entregarse a los consumidores al final de la misma semana en que se
fabrican o pueden almacenarse para su entrega futura, con un costo de 400 u.m. por semana.
Debido a la capacidad limitada de almacenamiento, la compañía no puede almacenar más de tres
ordenadores a un tiempo. El inventario actual es cero y la compañía no desea ningún inventario
al final de la semana 4. Determinar el número de ordenadores a fabricar en cada semana, para
cumplir la demanda a un costo mínimo total.
\begin{solucion}
Las demandas son $d_1=3, d_2=2, d_3=4, d_4=2$. Hay 4 etapas, en cada una de las cuales se decide el número de ordenadores que se fabrican esa semana. Los estados son $(t, I)$ donde $t$ es la etapa e $I$ es el espacio en el inventario. El coste de almacenamiento, en miles, es 0.4. $f(t,I)$ es el mínimo coste de producción y almacenamiento desde la etapa $t$ hasta el final si tenemos en inventario $I$ ordenadores.
$f(t,I)=\min_{d_t\leq I+x, I+x-d_t\leq 3,x\leq 4}\{c(x)+0.4(I-d_t+x)+f(t+1,I-d_t+x)\}$ $t=1,2,3$ (Se podría interpretar que el inventario se paga al principio de la etapa y entonces el coste de inventario sería simplemente $0.4I$ añadiéndole esa cantidad también a la función siguiente). 

$f(4,I)=\begin{cases}
+\infty & I>2\\
c(d_4-I) & I\leq 2
\end{cases}$

Otra opción es añadir una etapa extra de forma que 

$f(5,I)=\begin{cases}
0 & I=0\\
+\infty & I\neq 0
\end{cases}$
En tal caso, para $t=4$ seguiríamos usando la función recursiva. Continuemos con la resolución del problema.

\underline{Etapa 4}

\begin{tabular}{|c| c| c|}
\hline
$I$ & $x$ & $f(4,I)$\\
\hline
0 & 2 & 19\\
\hline
1 & 1 & 13\\
\hline
2 & 0 & 4\\
\hline
\end{tabular}

\underline{Etapa 3}

\begin{tabular}{|c| c| c|c|c|}
\hline
$I$ & $x$ & $c(x)+0.4(I-4+x)$ & $f(4,I-4+x)$ & $f(3,I)$\\
\hline
0 & 4 & 32 & 19 & 32+19\\
\hline
\hline
1 & 4 & 32+0.4 & 13 & $\boxed{32.4+13}$\\
\hline
 & 3 & 27 & 19 & 27+19\\
\hline
\hline
2 & 4 & 32+0.4 $\cdot$ 2 & 4 & $\boxed{32.8+4}$\\
\hline
  & 3 & 27+0.4 & 13 & 27.4+13\\
  \hline
  & 2 & 19+0.4 & 19 & 19+19\\
  \hline
  \hline
3 & 3 & 27+0.8 & 4 & $\boxed{31.8}$\\
\hline
 &  2 & 19+0.4 & 13 & 19.4+13\\
 \hline
 & 1 & 13 & 19 & 13.4+19\\
 \hline
\end{tabular}

\underline{Etapa 2}

\begin{tabular}{|c|c|c|c|c|}
\hline
$I$ & $x$ & $c(x)+0.4(I-2+x)$ & $f(3,I-2+x)$ & $f(2,I)$\\
\hline
0 & 4 & 32+0.4$\cdot$2 & 36.8 & $\boxed{69.6}$\\
  & 3 &                &       & \\
  & 2 &                &        & \\
\hline
\hline
1 & 4  &   &    & \\
  & 3 &   &     & \\
  & 2 &   &    & \\
  & 1 & 4 & 51 & $\boxed{51}$\\
  \hline
\end{tabular}

\underline{Etapa 1}

\begin{tabular}{|c|c|c|c|c|}
\hline
$I$ & $x$ & $c(x)+0.4(I-3+x)$ & $f(2,I-3+x)$ & $f(1,0)$\\
\hline
0 & 4 &  & 64 & $\boxed{96.4}$\\
  & 3 &                &       & \\
\hline
\end{tabular}

$x_1=4, x_2=1, x_3=4, x_4=2$.
\end{solucion}
\end{ejercicio}	


\newpage 
\begin{ejercicio}{4}

\begin{solucion}

\end{solucion}
\end{ejercicio}

\newpage 
\begin{ejercicio}{5}
\begin{solucion}
\end{solucion}
\end{ejercicio}

\newpage 
\begin{ejercicio}{6}

\end{ejercicio}
\begin{solucion}

\end{solucion}

\newpage 
\begin{ejercicio}{7}

\end{ejercicio}
\begin{solucion}

\end{solucion}

\newpage
\begin{ejercicio}{12} 
\end{ejercicio}
\begin{solucion}

\end{solucion}

\end{document}