\documentclass[MIOP.tex]{subfiles}
\usepackage{mathtools}
%\usepackage{sagetex}
\begin{document}

\chapter{Programación entera.}

\section{Desigualdades válidas.}

Dado un poliedro $P=\{x\in\R^n:Ax\leq b,x\geq 0\}$. La desigualdad $\Pi'x\leq\Pi_0$ es válida para $P$ si $\forall x\in P$ se verifica la desigualdad. Por ejemplo, todas las desigualdades de $Ax\leq b$ son válidas para $P$. La definición se extiende a cualquier conjunto que no sea un poliedro. Por ejemplo a $P_E=P\cap\Z^n$. En ese caso $conv(P_E)=\{x\in\R^n:\overline{A}x\leq\overline{b},x\geq 0\}$ y se tiene
\begin{enumerate}
\item $Ax\leq b$ son desigualdades válidas para $P_E$.
\item $\overline{A}x\leq\overline{b}$ son desigualdades válidas de $P_E$. 
\end{enumerate} 
\begin{ejs}\
\begin{enumerate}
\item $X=\{x\in\{0,1\}^5: 3x_1-4x_2+2x_3-3x_4+x_5\leq-2\}$. Si $x_2=x_4=0\Rightarrow 3x_1+2x_3+x_5\leq -2$, pero todas las variables son no negativas, luego no se puede conseguir la igualdad. Así que los puntos de la forma $(x_1,0,x_3,0,x_5)\notin X$, es decir, $x_2+x_4\geq 1$, que es una desigualdad válida.
\item $X=\{(x,y):0\leq x\leq 5, x\leq 9999y,y\in\{0,1\}\}$. Entonces $x\leq\min\{5,9999y\}=\min\{5,9999\}\Rightarrow x\leq 5$ es una desigualdad válida. Como además $y\in\{0,1\}$ y $x\in[0,5]$, se tiene que $x\leq 5y$.
\item $X:\{(x,y): x\leq 10y, 0\leq x\leq 14, y\in\Z^+\}$.
La recta definida por $(10,1)$ y $(14,2)$ define una desigualdad válida para $X$, $x\leq 6+4y$. En general, si $X=\{(x,y):x\leq Cy, 0\leq x\leq b,y\in\Z^+\}$, si $C$ no divide a $b$ entonces estamos en la misma situación anterior. La desigualdad válida estará definida por $(C,1),(b,\lceil C/b\rceil)$. En la gráfica a continuación se puede ver el ejemplo.

\definecolor{ffqqqq}{rgb}{1.,0.,0.}
\definecolor{qqqqff}{rgb}{0.,0.,1.}
\begin{tikzpicture}[line cap=round,line join=round,>=triangle 45,x=1.0cm,y=1.0cm]
\draw[color=black] (-0.4,0.) -- (14.933333333333337,0.);
\foreach \x in {,1.,2.,3.,4.,5.,6.,7.,8.,9.,10.,11.,12.,13.,14.}
\draw[shift={(\x,0)},color=black] (0pt,2pt) -- (0pt,-2pt) node[below] {\footnotesize $\x$};
\draw[color=black] (0.,-1.0266666666666635) -- (0.,6.4);
\foreach \y in {-1.,1.,2.,3.,4.,5.,6.}
\draw[shift={(0,\y)},color=black] (2pt,0pt) -- (-2pt,0pt) node[left] {\footnotesize $\y$};
\draw[color=black] (0pt,-10pt) node[right] {\footnotesize $0$};
\clip(-0.4,-1.0266666666666635) rectangle (14.933333333333337,6.4);
\draw [line width=2.pt,domain=-0.4:14.933333333333337] plot(\x,{(-0.-1.*\x)/-10.});
\draw [line width=2.pt,color=qqqqff] (0.,1.)-- (10.,1.);
\draw [line width=2.pt,color=qqqqff] (0.,2.)-- (14.,2.);
\draw [line width=2.pt,color=qqqqff] (0.,3.)-- (14.,3.);
\draw [line width=2.pt,color=qqqqff] (0.,4.)-- (14.,4.);
\draw [line width=2.pt,color=qqqqff] (0.,5.)-- (14.,5.);
\draw [line width=2.pt,color=ffqqqq,domain=-0.4:14.933333333333337] plot(\x,{(-6.--1.*\x)/4.});
\draw [line width=2.pt,color=qqqqff] (0.,6.)-- (14.,6.);
\begin{scriptsize}
\draw [fill=qqqqff] (0.,0.) circle (2.0pt);
\draw [fill=ffqqqq] (10.,1.) circle (2.5pt);
\draw [fill=ffqqqq] (14.,2.) circle (2.5pt);
\end{scriptsize}
\end{tikzpicture}

\item En el problema de emparejamiento, sea $G=(V,E)$. Si $i\in V$ se define $\delta(i)=\{e\in E:i\in e\}$. Un emparejamiento (\emph{matching}) es un conjunto de arista que no tiene vértices comunes. El politopo del matching viene descrito por $\sum_{e\in\delta(i)}x_e\leq 1\ \forall i\in V, x_e\in\{0,1\}$. $x_e=1$ si y solo si $e\in E$ está en el matching.  

Sea $T$ un conjunto de vértices $T\subseteq V$ de cardinalidad impar mayor o igual que 3. Entonces en cualquier solución de emparejamiento el conjunto de las aristas que tiene ambos vértices en $T$, $E(T)$, debe verificar que $\sum_{e\in E(T)}x_e\leq\lfloor |T|/2\rfloor$, es decir solo se pueden elegir $\frac{|T|-1}{2}$ aristas en el emparejamiento.
\item $X=P\cap\Z^4, P=\{x\in\R^4_+:13x_1+20x_2+11x_3+6x_4\geq 72\}$. Si dividimos la inecuación por 11 se sigue verificando. Y como buscamos enteros no negativos podemos tomar las partes enteras superiores de los coeficientes, dando lugar a 
$$2x_1+2x_2+x_3+x_4\geq 7$$
\end{enumerate}
\end{ejs}

\begin{prop}
Sea $A\in\R^{n\times m}$. $\Pi'x\leq\Pi_0$ es una desigualdad válida para $P=\{x\in\R^n:Ax\leq b,x\geq 0\}$ si y solo si $\exists u\in\R^m_+,u'A\geq\Pi,u'b\leq\Pi_0$.
\end{prop}
\begin{dem}
Consideremos el siguiente problema
\begin{align*}
z_p=& \max \Pi'x\\
   &sa: Ax\leq b\\
   &x\geq 0
\end{align*}
La desigualdad es válida si y solo si $z_p\leq\Pi_0$. Para probar el resultado escribimos el dual, que en caso de tener solución tiene el mismo valor
\begin{align*}
z_p=& \min u'b\\
   &sa: u'A\geq \Pi\\
   &u\geq 0
\end{align*}
Entonces la desigualdad es válida si y solo si $z_p\leq\Pi_0$, de donde se deduce el resultado. 
$\QED$
\end{dem}
\begin{coro}
El mismo resultado se tiene si y solo si $\exists v\geq 0, u\geq 0, u'A-v= \Pi, u'b\leq\Pi_0$. La $v$ no es más que una holgura.
\end{coro}

\subsection{Desigualdades válidas para conjuntos de puntos enteros.}
\begin{prop}
Sea $X=\{y\in\Z:y\leq b\}$, entonces $y\leq\lfloor b\rfloor$ es válida para $X$.
\end{prop}
\begin{dem}
Trivialérrimo. $\QED$
\end{dem}
\begin{ej}
Consideremos $P$ dado por las desigualdades
\begin{align*}
7x_1-2x_2\leq 14\\
x_2\leq 3\\
2x_1-2x_2\leq 3\\
x\geq 0
\end{align*}
Hacemos una combinación lineal de las ecuaciones con coeficientes $(2/7,27/36, 0)$. Se obtiene $2x_1-\frac{1}{63}x_2\leq\frac{121}{21}$. Buscamos desigualdades válidas para $P_E$. Así que entonces también se verfica $2x_1\leq 5\Rightarrow x_1\leq 2$.
\end{ej}

\subsection{Procedimiento de Chvatal-Gomory}
Es aplicable a problemas enteros de la forma $X=P\cap\Z^n$ con $P=\{x\in\R^n:Ax\leq b,x\geq 0\}$, siendo $A=[a_1,\dots,a_m]\in\R^{m\times n}$ y $u\in\R^m_+$. 
\begin{enumerate}
\item La desigualdad $\sum_{j=1}^nu'a_jx_j\leq u'b$ es válida para $P$.
\item La desigualdad $\sum_{j=1}^n\lfloor u'a_j\rfloor x_j\leq \lfloor u'b\rfloor$ es válida para $X$.
\end{enumerate}
\begin{teorema}
Toda desiguald válida para $X$ se puede obtener por el procedimiento de Chvatal-Gomory (C-G) siendo aplicado un número finito de veces.
\end{teorema}
\begin{ej}
Volvamos al ejemplo del matching. Tenemos el poliedro $P=\{x\geq 0.\sum_{e\in\delta(i)}x_e\leq 1\ \forall i\in E\}$. Fijado $T$, tomamos coeficientes $u_i=1/2$ si $i\in T, u_i=0$ si $i\notin T$. 
$$\sum_{i\in T}\frac{1}{2}(\sum_{e\in\delta(i)}x_e)\leq\frac{|T|}{2}$$
Cada $x_e$ aparece 2 veces en $E(T)$ y 1 cuando la arista tiene un vértice en $T$ y otra en el complementario.
$$\sum_{e\in E(T)}\frac{1}{2}(2x_e)+\sum_{e\in T\times V\setminus T}\frac{1}{2}x_e=\sum_{e\in E(T)}x_e+\frac{1}{2}\sum_{e\in T\times V\setminus T}x_e\geq \sum_{e\in E(T)}x_e.$$
Luego tenemos que $\sum_{e\in E(T)}\leq\frac{|T|}{2}$, que redondeando se convierte en $\sum_{e\in E(T)}\leq\frac{|T|-1}{2}$ 

\end{ej}













\section{Desigualdades válidas para poliedros enteros}
\begin{prop}
 Sea $X^{\geq}=\{(x,y)\in\R_+\times \Z \mid x+y \geq b\}$ y $f= b-\suelo{b}>0$. Entonces la desigualdad $x\geq f(\ceil{b}-y)$ es válida.
\end{prop}

\begin{dem}\
\begin{enumerate}
\item Si $y\geq\lceil b\rceil$, como $0<f<1$ entonces $x\geq 0\geq \lceil b\rceil-y$ y $0\geq f(\lceil b\rceil-y)$, luego $x\geq f(\lceil b\rceil-y).$
\item Si $y<\lceil b\rceil$, entonces $x\geq b-y=\lfloor b\rfloor+f-y\geq f+f(\lfloor b\rfloor-y)\Rightarrow \frac{x}{f}\geq 1+\lfloor b\rfloor-y=\ceil{b}-y.$
\end{enumerate}
$\QED$
\end{dem}

\begin{coro}
Para el caso $X^\leq =\{(x,y)\in\R_+\times \Z \mid y\leq b+x\}$ con $f=b-\suelo{b}>0$ entonces $y\leq \suelo{b}+\frac{x}{1-f}$ es válida para $X^\leq$.
\end{coro}
\begin{dem}
$x+b\geq y \equiv x-y\geq -b$. Denotamos $f'=-b-\suelo{-b}$. La desigualdad válida es 
$$
\frac{x}{f'}-y\geq \ceil{-b}=-\suelo{b} \equiv \frac{x}{1-f}-y \geq -\suelo{b}
$$
$\QED$
\end{dem}
\begin{propi}
Sea $X^{MIR}=\{(x,y)\in\R_+\times \Z_+^2 \mid a_1y_2+a_2y_2\leq b+x\}$, $a_1,a_2,b\in\R$, $b\notin \Z$. Supongamos que $f=b-\suelo{b}$, $f_i = a_i -\suelo{a_i}$, $i=1,2$ y $f_1\leq f\leq f_2$. Entonces 
$$
\suelo{a_1}y_1+\left(\suelo{a_2}+\frac{f_2-f}{1-f}\right)y_2\leq \suelo{b}+\frac{x}{1-f}$$
es válida para $X^{MIR}$.
\end{propi}
\begin{dem}
Observamos que $\suelo{a_1}\leq a_1$ y $a_2=\ceil{a_2}-(1-f_2)=\suelo{a_2}+f_2$. Entonces
$$
\suelo{a_1}y_1+(\ceil{a_2}-(1-f_2))y_2 \leq a_1y_1 + a_2y_2\leq b+x
$$
$$\suelo{a_1}y_1+(\ceil{a_2}-(1-f_2))y_2 \leq b+x+(1-f_2)y_2$$
Por el Corolario
$$
\suelo{a_1}y_1+\ceil{a_2}y_2 \leq \suelo{b}+\frac{x+(1-f_2)y_2}{1-f}
$$
$$
\suelo{a_1}y_1+\left(\ceil{a_2}-\frac{1-f_2}{1-f}\right)y_2 \leq \suelo{b}+\frac{x}{1-f}
$$
$$
\suelo{a_1}y_1+\left(\suelo{a_2}+\frac{1-f-1+f_2}{1-f}\right)y_2\leq \suelo{b}+\frac{x}{1-f}
$$
$\QED$
\end{dem}
\begin{prop}
$X=X_1\cup X_2$ se denomina disyunción si sus elementos están en $X_1$ o $X_2$, $X_i\subset \R^n_+$. Supongamos que ${\Pi^i}'x\leq \Pi^i_0$ es válida para $X^i$, $i=1,2$. Entonces $\Pi' x\leq \Pi_0$ tal que $\Pi_j=\min\{\Pi^1_j,\Pi^2_j\}$, $\Pi_0 =\max\{\Pi^1_0,\Pi^2_0\}$ es válida para $X$.
\end{prop}
\begin{dem}
Sea $x\in X$. Evaluamos
$$
\Pi' x = \sum_{j=1}^n \Pi_j x_j = \sum_{j=1}^n \min\{\Pi_j^1,\Pi_j^2\}x_j
$$
Si $x\in X_i$ entonces $\Pi'x \leq \Pi^i x \leq \Pi_0$. 
$\QED$
\end{dem}
\begin{prop}
Si $P^i = \{x\in\R^n_+\mid A^i x \leq b^i\}$, $i=1,2$ son poliedros no vacíos entonces $\Pi x\leq \Pi_0$ es una desigualdad válida para $Co(P^1\cup P^2)$ sii $\exists u^1,u^2$ tales que $u^1A^1\geq \Pi$, $u^2A^2 \geq \Pi$, $u^1 b^1 \leq \Pi_0$, $u^2b^2\leq \Pi_0$
\end{prop}
\begin{dem}
Supongamos que $\Pi x \leq \Pi_0$ es válida para $ Co(P^1\cup P^2)$. Sea $x \in Co(P^1\cup P^2)$ entonces $\Pi x \leq \Pi_0$. En particular si $x\in P^1 \subset Co(P^1\cup P^2)$ también es válida para $P^1$ sii $\exists u^1\mid u^1 A^1 \geq \Pi$, $u^1 b^1 \leq \Pi_0$. Análogo para $x\in P^2$.

Recíprocamente supongamos que $\Pi x \leq \Pi_0$ verifica que existen los $u^i$ del enunciado. Sea $x\in Co(P^1\cup P^2)$, $x=\lambda x^1 +(1-\lambda)x^2$, $\lambda \in [0,1]$, $x^i\in P^i$. Entonces 
$$
\Pi x = \Pi (\lambda x^1 +(1-\lambda)x^2) = \lambda \Pi x^1 +(1-\lambda)\Pi x^2
$$
Como se cumple $u^i A^i \geq \Pi$, $u^i b^i \leq \Pi_0$ es equivalente a que la igualdad se cumpla para cada poliedro, de donde se deduce el resultado aplicando la desigualdad sobre lo anterior. $\QED$
\end{dem}

\begin{ej}
$P^1=\{x\in\R^2_+:-x_1+x_1\leq 1, x_1+x_2\leq 5\}; P^2=\{x\in\R^2_+:x_2\leq 4,-2x_1+x_2\leq -6, -3x_2\leq -2\}; u^1=(2,1);u^2=(5/2,1/2,0)$. Aplicamos la  proposición. 
$$
u^1\begin{pmatrix}
-1 & 1\\
1 & 1
\end{pmatrix}=(-1,3)
$$
$$u^1b^1=(2,1)(1,5)'=7$$
$$u^2A^2=(5/2,1/2,0)\begin{pmatrix}
0 & 1\\
-2 & 1\\
0 & -3
\end{pmatrix}=(-1,3)$$
$$
u^2b^2=(5/2, 1/2,0)(4, -6,-2)'=7$$
Tomamos $\Pi=(-1,3), \Pi_0=7$, entonces, $\Pi x\leq\Pi_0\equiv -x_1+3x_2\leq 7$ es válida.
\end{ej}
Consideremos el poliedro $P=\{x\in\R^n \mid Ax\leq b, 0\leq x \leq 1\}$. Si suponemos que $x_j\in\{0,1\}$ entonces podemos considerar los poliedros 
$$P^0=\{x\in\R^n \mid x_j =0\}\cap P \qquad P^1 =\{x\in\R^n \mid x_j=1\}\cap P
$$
\begin{prop}
La desigualdad $\Pi x\leq \Pi_0$ es válida para $Co(P^0 \cup P^1)$ si existen $u^i \in \R^m_+$, $v^i \in \R^n_+$, $w^i\in\R_+$ para $i=1,2$ tales que 
\begin{align*}
\Pi\leq u^0A + v^0 + w^0e_j && \Pi \leq u^1 A + v^1-w^1 e_j\\
\Pi_0 \geq u^0b+1^tv^0 && \Pi_0\geq u^1 b +1^t v^1 -w^1
\end{align*}
\end{prop}
\begin{dem}
Observamos que
$$
P^0=\{x\in\R^n_+\mid Ax \leq b, x\leq 1, x_j \leq 0\} \quad P^1=\{x\in\R^n_+\mid Ax \leq b, x\leq 1, -x_j \leq -1\}
$$
$E_j$ es la matriz 0 pero poniéndole un 1 en la diagonal.
$$
\begin{pmatrix}
u_0\\
v_0\\
w_0
\end{pmatrix}:=\begin{pmatrix}
A\\
I\\
E_j
\end{pmatrix}x \leq 
\begin{pmatrix}
b\\
1\\
0
\end{pmatrix}
\qquad
\begin{pmatrix}
u^1\\
v^1\\
w^1
\end{pmatrix} :=
\begin{pmatrix}
A\\
I\\
-E_j
\end{pmatrix}x \leq 
\begin{pmatrix}
b\\
1\\
-e_j
\end{pmatrix}
$$
$\QED$
\end{dem}

\begin{ej}
\begin{align*}
\max\ & 12x_1+14x_2+7x_3+12x_4\\
sa\ & 4x_1+5x_2+3x_3+6x_4\leq 8\\
& x_i\in\{0,1\}
\end{align*}
La solución de la relajación lineal (i.e. $x_i\in[0,1]$) es $x^*=(1, 0.8, 0, 0)$. Elegimos $x_2$: $\Pi x\leq \Pi_0$, $\Pi x^*-\Pi_0>0$. 
\begin{align*}
\max\ & \Pi_1x^*_1+\Pi_2x^*_2+\Pi_3x^*_3+\Pi_4x^*_4-\Pi_0\\
sa\ & \Pi_1\leq 4u^0+v_1^0+w^0\cdot 0\\
& \Pi_2\leq 5u^0+v_2^0+w^0\cdot 1\ (\text{porque estamos ramificando en }x_2)\\
& \Pi_3\leq 3u^0+v_3^0+w^0\cdot 0\\
&\Pi_4\leq 6u^0+v_4^0+w^0\cdot 0\\
\\
& \Pi_1\leq 4u^1+v_1^1-w^1\cdot 0\\
& \Pi_2\leq 5u^1+v_2^1-w^1\cdot 1\\
& \Pi_3\leq 3u^1+v_3^1-w^1\cdot 0\\
&\Pi_4\leq 6u^1+v_4^1-w^1\cdot 0\\
\\
&\Pi_0\geq 8u^0+v_1^0+v_2^0+v_3^0+v_4^0\\
\\
&\Pi_0\geq 8u^1+v_1^1+v_2^1+v_3^1+v_4^1-w^1
\end{align*}
\end{ej}
\end{document}
