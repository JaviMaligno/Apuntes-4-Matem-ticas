\documentclass[MIOP.tex]{subfiles}
\usepackage{mathtools}
%\usepackage{sagetex}
\begin{document}

\chapter{Programación entera.}

\section{Desigualdades válidas.}

Dado un poliedro $P=\{x\in\R^n:Ax\leq b,x\geq 0\}$. La desigualdad $\Pi'x\leq\Pi_0$ es válida para $P$ si $\forall x\in P$ se verifica la desigualdad. Por ejemplo, todas las desigualdades de $Ax\leq b$ son válidas para $P$. La definición se extiende a cualquier conjunto que no sea un poliedro. Por ejemplo a $P_E=P\cap\Z^n$. En ese caso $conv(P)=\{x\in\R^n:\overline{A}x\leq\overline{b},x\geq 0\}$ y se tiene
\begin{enumerate}
\item $Ax\leq b$ son desigualdades válidas para $P_E$.
\item $\overline{A}x\leq\overline{b}$ son desigualdades válidas de $P_E$. 
\end{enumerate} 
\begin{ejs}\
\begin{enumerate}
\item $X=\{x\in\{0,1\}^5: 3x_1-4x_2+2x_3-3x_4+x_5\leq-2\}$. Si $x_2=x_4=0\Rightarrow 3x_1+2x_3+x_5\leq -2$, pero todas las variables son no negativas, luego no se puede conseguir la igualdad. Así que los puntos de la forma $(x_1,0,x_3,0,x_5)\notin X$, es decir, $x_2+x_4\geq 1$, que es una desigualdad válida.
\item $X=\{(x,y):0\leq x\leq 5, x\leq 9999y,y\in\{0,1\}\}$. Entonces $x\leq\min\{5,9999y\}=\min\{5,9999\}\Rightarrow x\leq 5$ es una desigualdad válida. Como además $y\in\{0,1\}$ y $x\in[0,5]$, se tiene que $x\leq 5y$.
\item $X:\{(x,y): x\leq 10y, 0\leq x\leq 14, y\in\Z^+\}$. (Hacer dibujo)
La recta definida por $(10,1)$ y $(14,2)$ define una desigualdad válida para $X$, $x\leq 6+4y$. En general, si $X=\{(x,y):x\leq Cy, 0\leq x\leq b,y\in\Z^+\}$, si $C$ no divide a $b$ entonces estamos en la misma situación anterior. La desigualdad válida estará definida por $(C,1),(b,\lceil C/b\rceil)$.
\item En el problema de emparejamiento, sea $G=(V,F)$. Si $i\in V$ se define $\delta(i)=\{e\in E:i\in e\}$. Un emparejamiento (\emph{matching}) es un conjunto de arista que no tiene vértices comunes. El politopo del matching viene descrito por $\sum_{e\in\delta(i)}x_e\leq 1\ \forall i\in V, x_e\in\{0,1\}$. $x_e=1$ si y solo si $e\in E$ está en el matching.  

Sea $T$ un conjunto de vértices $T\subseteq V$ de cardinalidad impar mayor o igual que 3. Entonces en cualquier solución de emparejamiento el conjunto de las aristas que tiene ambos vértices en $T$, $E(T)$, debe verificar que $\sum_{e\in E(T)}x_e\leq\lfloor |T|/2\rfloor$, es decir solo se pueden elegir $\frac{|T|-1}{2}$ aristas en el emparejamiento.
\item $X=P\cap\Z^4, P=\{x\in\R^4_+:13x_1+20x_2+11x_3+6x_4\geq 72\}$. Si dividimos la inecuación por 11 se sigue verificando. Y como buscamos enteros no negativos podemos tomar las partes enteras superiores de los coeficientes, dando lugar a 
$$2x_1+2x_2+x_3+x_4\geq 7$$
\end{enumerate}
\end{ejs}

\begin{prop}
$\Pi'x\leq\Pi_0$ es una desigualdad válida para $P=\{x\in\R^n:Ax\leq b,x\geq 0\}$ si y solo si $\exists u\in\R^m_+,u'A\geq\Pi,u'b\leq\Pi_0$.
\end{prop}
\begin{dem}
Consideremos el siguiente problema
\begin{align*}
z_p=& \max \Pi'x\\
   &sa: Ax\leq b\\
   &x\geq 0
\end{align*}
La desigualdad es válida si y solo si $z_p\leq\Pi_0$. Para probar el resultado escribimos el dual, que en caso de tener solución tiene el mismo valor
\begin{align*}
z_p=& \min u'b\\
   &sa: u'A\geq \Pi\\
   &u\geq 0
\end{align*}
Entonces la desigualdad es válida si y solo si $z_p\leq\Pi_0$, de donde se deduce el resultado. 
$\QED$
\end{dem}
\begin{coro}
El mismo resultado se tiene si y solo si $\exists v\geq 0, u\geq 0, u'A-v= \Pi, u'b\leq\Pi_0$. La $v$ no es más que una holgura.
\end{coro}

\subsection{Desigualdades válidas para conjuntos de puntos enteros.}
\begin{prop}
Sea $X=\{y\in\Z:y\leq b\}$, entonces $y\leq\lfloor b\rfloor$ es válida para $X$.
\end{prop}
\begin{dem}
Trivialérrimo. $\QED$
\end{dem}
\begin{ej}
Consideremos $P$ dado por las desigualdades
\begin{align*}
7x_1-2x_2\leq 14\\
x_2\leq 3\\
2x_1-2x_2\leq 3\\
x\geq 0
\end{align*}
Hacemos una combinación lineal de las ecuaciones con coeficientes $(2/7,27/36, 0)$. Se obtiene $2x_1-\frac{1}{63}x_2\leq\frac{121}{21}$. Buscamos desigualdades válidas para $P_E$. Así que entonces también se verfica $2x_1\leq 5\Rightarrow x_1\leq 2$.
\end{ej}

\subsection{Procedimiento de Chvatal-Gomory}
Es aplicable a problemas enteros con $X=P\cap\Z^n$ con $P=\{x\in\R^n:Ax\leq b,x\geq 0\}$, siendo $A=[a_1,\dots,a_m]\in\R^{m\times n}$ y $u\in\R^m_+$. 
\begin{enumerate}
\item La desigualdad $\sum_{j=1}^nu'a_jx_j\leq u'b$ es válida para $P$.
\item La desigualdad $\sum_{j=1}^n\lfloor u'a_j\rfloor x_j\leq \lfloor u'b\rfloor$ es válida para $X$.
\end{enumerate}
\begin{teorema}
Toda desiguald válida para $X$ se puede obtener por el procedimiento de Chvatal-Gomory (C-G) siendo aplicado un número finito de veces.
\end{teorema}
\begin{ej}
Volvamos al ejemplo del matching. Tenemos el poliedro $P=\{x\geq 0.\sum_{e\in\delta(i)}x_e\leq 1\ \forall i\in E\}$. Fijado $T$, tomamos coeficientes $u_i=1/2$ si $i\in T, u_i=0$ si $i\notin T$. 
$$\sum_{i\in T}\frac{1}{2}(\sum_{e\in\delta(i)}x_e)\leq\frac{|T|}{2}$$
Cada $x_e$ aparece 2 veces en $E(T)$ y 1 cuando la arista tiene un vértice en $T$ y otra en el complementario.
$$\sum_{e\in E(T)}\frac{1}{2}(2x_e)+\sum_{e\in T\times V\setminus T}\frac{1}{2}x_e=\sum_{e\in E(T)}x_e+\frac{1}{2}\sum_{e\in T\times V\setminus T}x_e\geq \sum_{e\in E(T)}x_e.$$
Luego tenemos que $\sum_{e\in E(T)}\leq\frac{|T|}{2}$, que redondeando se convierte en $\sum_{e\in E(T)}\leq\frac{|T|-1}{2}$ 

\end{ej}
\end{document}
