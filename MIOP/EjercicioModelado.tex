\documentclass[twoside]{article}
\usepackage{../estilo-ejercicios}

%--------------------------------------------------------
\begin{document}

\title{Modelos de Investigación Operativa}
\author{Rafael González López, Javier Aguilar Martín}
\maketitle

\begin{ejercicio}{de modelado 1}
Hay un presupuesto $b$ disponible para invertir en proyectos durante el próximo año y $n$ proyectos a considerar, donde $a_j$ es el gasto del proyecto $j$, y $c_j$ es el retorno esperado del proyecto $j$. El objetivo es elegir un conjunto de proyectos de forma que no se exceda el presupuesto y el retorno esperado sea máximo.
\end{ejercicio}
\begin{solucion}
Se definen las variables $x_j\in\{0,1\}$ para todo $j=1,\dots, n$, que representa si se decide invertir en el proyecto $j$. El problema es el siguiente
\begin{align*}
\max\ & \sum x_jc_j\\
sa:\ & \sum_{j=1}^n x_j\leq n\\
& x_j\in\{0,1\}
\end{align*}
\end{solucion}

\newpage

\begin{ejercicio}{de modelado 2}
Hay $n$ personas para realizar $n$ trabajos. A cada persona se le asigna un trabajo. Algunos individuos están mejor preparados para un trabajo en particular que otros, por lo que hay un coste estimado $c_{ij}$ si la persona $i$ es asignada al trabajo $j$. El problema es encontrar la asignación de coste mínimo.
\end{ejercicio}
Se definen las variables $x_{ij}\in\{0,1\}$ que representarán si el trabajador $i$ es asignado al trabajo $j$. 
\begin{align*}
\min\ & \sum_{i,j}c_{ij}x_{ij}\\
& \sum_i x_{ij}=1, \forall j\\
& \sum_j x_{ij}=1, \forall i\\
& x_{ij}\in\{0,1\}
\end{align*}
\begin{solucion}
\end{solucion}

\newpage

\begin{ejercicio}{de modelado 3}
Sea $G=(V,E)$ un grafo con matriz de adyacencia $(a_{ij})$ simétrica (el grafo no es dirigido) y con coeficientes binarios. Buscamos el máximo conjunto independiente, es decir, el conjunto de nodos de mayor cardinal tal que ninguno de ellos esté conectado a otro del conjunto.
\end{ejercicio}
\begin{solucion}
Definimos variables $x_i\in\{0,1\}$ para decidir si el vértice $i$ se incluye en el conjunto.
\begin{align*}
\max& \ \sum_i x_i\\
sa:\ &  x_ia_{ij}+x_ja_{ij}\leq 1\ \forall i,j\\
& x_i\in\{0,1\} 
\end{align*}
\end{solucion}

\newpage

\begin{ejercicio}{de modelado 4}
Similar al anterior, pero buscando el máximo subgrafo completo o \emph{cliqué}.
\end{ejercicio}
\begin{solucion}
Definimos variables $x_i\in\{0,1\}$ para decidir si el vértice $i$ se incluye en el conjunto.
\begin{align*}
\max& \ \sum_i x_i\\
sa:\ &(1-a_{ij})(x_i+x_j)\leq 1\ \forall i,j\\
& x_i\in\{0,1\} 
\end{align*}
\end{solucion}

\newpage

\begin{ejercicio}{de modelado 5}
La búsqueda del número cromático. 
\end{ejercicio}
\begin{solucion}
Variable $x_{ik}\in\{0,1\}$ si asignamos el color $k$ al vértice $i$, con $1\leq i,k\leq n$.  Definimos también $y_k$ si se usa el color $k$. 
\begin{align*}
\min & \sum_k y_k\\
sa:\ & a_{ij}(x_{ik}+x_{jk})\leq 1\ \forall j,k\\
     & \sum_k x_{ik}=1\ \forall i\\
    & y_k\geq x_{ij}\ \forall i, k\\
    & y_k\in\{0,1\}\\
    & x_{ik}\in\{0,1\}
\end{align*}
\end{solucion}
Podemos hacerlo más compacta sustituyendo la primera y la tercera restricción por $a_{ij}(x_{ik}+x_{jk})\leq y_k$. 
\end{document}