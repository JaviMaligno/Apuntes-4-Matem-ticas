\documentclass[twoside]{article}
\usepackage{../estilo-ejercicios}

%--------------------------------------------------------
\begin{document}

\title{Modelos de Investigación Operativa\\ Relación 4}
\author{Rafael González López, Javier Aguilar Martín}
\date{}
\maketitle

\begin{ejercicio}{1}
Un consumidor tiene una función de satisfacción $f(x_1,x_2)=10-(x_1-1)^2-(x_2-2)^2$ donde $x_i$ es la cantidad consumida del producto $i$-ésimo, $i=1,2$.
\begin{enumerate}
\item Encuentre los niveles de consumo óptimos si no hay restricciones en el dinero que se puede gastar.
\item ¿Cuál sería la decisión si los precios unitarios de los productos fuesen 2 y 1 respectivamente y el dinero disponible por el consumidor $m=4$ um?
\end{enumerate}
\begin{solucion}\
\begin{enumerate}
\item $\nabla f(x_1,x_2)=\begin{pmatrix}
-2(x_1-1)\\
-2(x_2-2)
\end{pmatrix}=0\Rightarrow x_1=1, x_2=2$. 

$\nabla^2f(x_1,x_2)=\begin{pmatrix}
-2 & 0\\
0  & -2
\end{pmatrix}$. El menor de orden 2 es positivo y el de orden 1 es negativo, por lo que el punto crítico es un máximo. Como la matriz es definida negativa tenemos un máximo local. Además, la función es claramente decreciente en todas las direcciones pues es un paraboloide, por lo que el máximo es global. 
\item Tendríamos que maximizar la función sujeta a la restricción $2x_1+x_2\leq 4$. Como el $(1,2)$ verifica la restricción, sigue siendo óptimo.
\end{enumerate}
\end{solucion}
\end{ejercicio}

\newpage 
\begin{ejercicio}{2}%\label{2}
Resolver el siguiente sistema de ecuaciones
\[
2x_1-x_2+x_3=-1,\ x_1+2x_2=0,\ 3x_1+x_2+2x_3=3
\]
utilizando el método de gradiente partiendo del punto $x^0=(0,0,0)'$.
\begin{solucion}
Sea $g(x_1,x_2,x_3)=\begin{pmatrix}
2x_1-x_2+x_3+1\\
 x_1+2x_2\\
 3x_1+x_2+2x_3-3
\end{pmatrix}$ y pensamos que es el gradiente de alguna función $F$. Aplicamos el método de la suma divergente para elegir la distanciad de paso. Entonces $\alpha^k=\frac{1}{k+1}$. Calculamos iteraciones hasta que damos con un punto que verifique el sistema.
\begin{align*}
x^1&=0-\nabla g(x^0)=\begin{pmatrix}
-1\\
0\\
3
\end{pmatrix}\\
x^2&=x^1-\frac{1}{2}g(x^1)=\begin{pmatrix}
-2\\
\frac{1}{2}\\
3
\end{pmatrix}\\
&\vdots
\end{align*}
La convergencia es muy lenta, así que es mejor resolverlo en Xpress.
\begin{verbatim}
model Ejercicio2Hoja4
uses "mmxprs"; !gain access to the Xpress-Optimizer solver

!optional parameters section
parameters
 n=3
 K=100000
end-parameters

!sample declarations section
declarations
 x:array(1..K,1..3) of real
 Fx:array(1..K-1,1..3) of real
end-declarations
(!
x(1,1):=0
x(1,2):=0
x(1,3):=0
!)
forall(i in 1..n)x(1,i):=0

forall(k in 1..K-1)do
 Fx(k,1):=2*x(k,1)-x(k,2)+x(k,3)+1
 Fx(k,2):=x(k,1)+2*x(k,2)
 Fx(k,3):=3*x(k,1)+x(k,2)+2*x(k,3)-3
 forall(i in 1..n)x(k+1,i):=x(k,i)-(1/k)*Fx(k,i)
end-do

end-model
\end{verbatim}
\end{solucion}

\end{ejercicio}

\newpage 

\begin{ejercicio}{3}
Sea $f:\R^n\to\R$ una función diferenciable que verifica que $x^*$ es un mínimo local de $f$ a lo largo de cada línea que pasa por $x^*$. Probar que 
\begin{enumerate}
\item $\nabla f(x^*)=0$
\item Considerar la función $f(y,z)=(z-py^2)(z-qy^2)$, donde $0<p<q$. Probar que $(0,0)$ es un mínimo en toda recta, pero no es un mínimo local de $f$.
\end{enumerate}

\begin{solucion}\
\begin{enumerate}
\item Que $x^*$ sea mínimo local a lo largo de cada línea que pasa por $x^*$ es equivalente a decir que las derivadas direccionales de $f$ se anulan en $x^*$ para toda dirección. En particular se anulan en las direcciones de la base canónica, por lo que $\nabla f(x^*)=0$. 
\item Consideremos aparte las rectas $y=0$ y $z=0$. $f(0,z)=z^2$, $f(y,0)=pqy^4$ luego el resultado es trivial en ambos casos. Consideremos ahora la recta $z=my, m\in\R^*$. Entonces $f(y)=(my-py^2)(my-qy^2)= pqy^4-m(q+p)y^3 +m^2y^2$. $f'(y)=4pqy^3-3m(q+p)y^2+2m^2y$, claramente $y=0$ es un punto crítico. $f''(y)=12pqy^2-6m(q+p)y+2m^2$, por lo que $f''(0)>0$, así que $y=0$ es un mínimo. 

Para ver que $(0,0)$ no es mínimo local veamos que no lo es en la dirección $y=x$, $z=x^2(p+q)/2$. Obviamente es una dirección válida que pasa por el $(0,0)$. Tenemos entonces 
$$
f(x) = -\frac{x^2(q-p)}{2}\frac{x^2(q-p)}{2} = -\frac{x^4(p-q)^2}{4}
$$
Obviamente en $x=0$ tenemos un máximo, luego no puede ser mínimo.
\end{enumerate}
\end{solucion}
\end{ejercicio}

\newpage 

\begin{ejercicio}{4}
Considere el siguiente problema:
Dados $n$ puntos $a_1,\dots,a_n$ en el plano encontrar un punto $x$ que minimice la distancia ponderada a dichos puntos. Esto es,
$$\min_{x\in\R^2}\sum_{i=1}^nw_i||x-a_i||$$
donde $w_1,\dots,w_n$ son escalares positivos. Probar que existe un mínimo global para este problema. 
\begin{solucion}\
Sea $f(x)=\sum_{i=1}^nw_i||x-a_i||$ continua. Si encontramos un conjunto compacto que contenga a todos los puntos podremos asegurar que existe ese mínimo. Si existe $x^*$ mínimo, debería verificar que $\forall x\in\R^2, \exists i_x\mid ||x^*-a_{i_x}||\leq ||x-a_{i_x}||$, en particular $$||x^*-a_{i_x}||\leq ||a_{i_0}||$$
Por tanto, como $||x^*||-||a_{i_0}||\leq ||x^*-a_{i_x}||\leq ||a_{i_0}||$, $||x^*||\leq 2||a_{i_0}||\leq 2\max_i{||a_i||}$. Así, basta minimizar en la bola de centro 0 y radio $2\max_i{||a_i||}$. Ya hemos dado con el compacto.
\end{solucion}
\end{ejercicio}

\newpage 

\begin{ejercicio}{5}

Para cada valor de $\beta$ buscar el conjunto de puntos estacionarios de la función: $$f(x,y)=x^2+y^2+\beta xy+x+2y.$$
¿Cuáles son mínimos globales y cuáles locales?
\begin{solucion}\
Iremos tratando el caso $\beta=0$ sobre la marcha. $\nabla f(x,y)=\begin{pmatrix}
2x+\beta y +1\\
2y+\beta x+2
\end{pmatrix}=0$. Para el caso $\beta=0$ tenemos la solución $x=-\frac{1}{2},y=-1$. En ese caso $\nabla^2 f(x,y)=\begin{pmatrix}
2 & 0\\
0 & 2
\end{pmatrix}$, con lo que el punto es un mínimo local.  Además es mínimo global, pues es un paraboloide.

Si $\beta=\pm 2$ entonces no hay puntos críticos puesto que el sistema no tiene solución. En otro caso, la solución es 
$$x=-\frac{2(b-1)}{b^2-4}, y=\frac{4-b}{b^2-4}$$
$\nabla^2 f(x,y)=\begin{pmatrix}
2 & \beta\\
\beta & 2
\end{pmatrix}$, cuyos menores son $4-\beta^2$ y $2$. Para $\beta\in (-2,2)$ tenemos un mínimo local y para el resto de casos no hay mínimos locales. Para los casos en los que hay mínimo local es global por lo mismo de antes. 
\end{solucion}
\end{ejercicio}

\newpage 

\begin{ejercicio}{6}
Probar que para todo $x>0$ se verifica que $\frac{1}{x}+x\geq 2$.

\begin{solucion}\
Sea $f(x)=\frac{1}{x}+x$. $f'(x)=-\frac{1}{x^2}+1=0\Rightarrow x=1, f''(x)=\frac{2}{x^3},f''(1)=1>0, f(1)=2$. Como la función es convexa, además el mínimo es global.
\end{solucion}
\end{ejercicio}

\newpage 

\begin{ejercicio}{7}
Buscar el paralepípedo de volumen unidad que tiene mínima superficie. 

\begin{solucion}\
$Vol(P)=1=xyz$. La superficie es $f(x,y,z)=2(xy+xz+yz)$. Despejamos $x=\frac{1}{yz}$ y obtenemos $f(y,z)=2(\frac{1}{z}+\frac{1}{y}+zy)$. 
$$\nabla f(y,z)=2\begin{pmatrix}
-\frac{1}{y^2}+z\\
-\frac{1}{z^2}+y
\end{pmatrix}$$
Obtenemos el punto crítico $y=1,z=1$.
$$\nabla^2f(y,z)=2\begin{pmatrix}
\frac{2}{y^4} & 1\\
1 & \frac{2}{z^4}
\end{pmatrix}\Rightarrow \nabla^2f(1,1)=2\begin{pmatrix}
2 & 1\\
1 & 2
\end{pmatrix}$$
Los dos menores principales son positivos, por lo que la matriz es definida positiva, lo cual quiere decir que el  punto crítico es un mínimo. En conclusión, el óptimo es el cubo unidad.
\end{solucion}
\end{ejercicio}

\newpage 

\begin{ejercicio}{8}
Probar las siguientes cuestiones:
\begin{enumerate}
\item $f(x,y)=(x^2-4)^2+y^2$ tiene dos mínimos globales y un punto estacionario no mínimo, ni máximo.
\item $f(x,y)=(y-x^2)^2-x^2$ tiene un punto estacionario no mínimo ni máximo.
\item Buscar todos los mínimos locales de $f(x,y)=\frac{1}{2}x+xcosy$.
\end{enumerate}

\begin{solucion}\
\begin{enumerate}
\item Calculamos los puntos estacionarios.
$$\nabla f(x,y)=\begin{pmatrix}
2(x^2-4)2x\\
2y
\end{pmatrix}=0\Rightarrow (0,0),(2,0),(-2,0).$$
$$\nabla^2 f(x,y)=\begin{pmatrix}
12x^2-8 & 0\\	
0 & 2
\end{pmatrix}$$
Observamos al sustituir que $(0,0)$ es punto de silla y los otros dos punto son mínimos locales. En los mínimos la función vale 0, y por su definición es siempre no negativa, por lo que son mínimos globales.
\item Veamos qué puntos son estacionarios.
$$\nabla f(x,y)=\begin{pmatrix}
-4x(y-x^2)\\
2(y-x^2)
\end{pmatrix}=0\Rightarrow x=y=0, y=x^2.$$
$$\nabla^2 f(x,y)=\begin{pmatrix}
12x^2-4y-2 & -4x\\
-4x & 2
\end{pmatrix}$$
Es claro que $(0,0)$ no es ni mínimo ni máximo. 
\item Calculamos el gradiente. $$\nabla f(x,y)=\begin{pmatrix}
\frac{1}{2}+\cos{y}\\
-x\sen{y}
\end{pmatrix}=0\Rightarrow \begin{cases}
(0,2(3\pi n -\pi)/3)\\
(0,2(3\pi n +\pi)/3)
\end{cases}\; \forall n\in\Z$$
No hay más puntos estacionarios, así que veamos si este es mínimo local.
$$\nabla^2 f(x,y)=\begin{pmatrix}
0 & -\sen{y}\\
-\sen{y} & -x\cos{y}
\end{pmatrix}\Rightarrow \nabla^2f(0,0)=\begin{pmatrix}
0 & -\frac{\sqrt{3}}{2}\\
-\frac{\sqrt{3}}{2}& 0
\end{pmatrix}$$
Como el determinante es negativo tenemos un punto de silla.
\end{enumerate}
\end{solucion}
\end{ejercicio}

\newpage 

\begin{ejercicio}{9}
Resolver el siguiente sistema de ecuaciones
\[
2x_1-x_2+x_3=-1,\ x_1+2x_2=0,\ 3x_1+x_2+2x_3=3
\]
utilizando el método de gradiente partiendo del punto $x^0=(0,0,0)'$.
\begin{solucion}\
Por no hacerlo igual que el \ref{ejer:2} aplicamos el método de Newton. Tenemos que
$$(\nabla^2 F(x^k))^{-1}=\begin{pmatrix}
4/5 & 3/5 & -2/5\\
-2/5 & 1/5 & 1/5\\
 -1 & -1 & 1
\end{pmatrix}\qquad 
\nabla F(x^k) = \begin{pmatrix}
2x_1-x_2+x_3+1\\
x_1+2x_2\\
3x_1+x_2+2x_3-3
\end{pmatrix}
$$
Iteramos hasta que obtengamos un $x^k$ que anule el gradiente.
$$
x^{k+1}=x^k - (\nabla^2 F(x^k))^{-1}\nabla F(x^k) \qquad x^1 = \begin{pmatrix}
-2\\
1\\
4
\end{pmatrix} \qquad \nabla F(x^1) = 0
$$
\end{solucion}
\end{ejercicio}

\newpage 

\begin{ejercicio}{10}
Considérese la función $f(x,y)=y^3+axy+12y$, siendo $a\in\R$.
\begin{enumerate}
\item Demuestre que independientemente del valor de $a$, la función anterior no tiene ni máximos ni mínimos.
\item Sea $a=1$. Pruebe que $(x^*,y^*)=(1,1)$ es un máximo local de $f$ en $S=\{(x,y)\in\R^2:x+16y=17\}$. ¿Es máximo local en $S_1=\{(x,y)\in\R^2:x+16y\geq 17\}$?
\end{enumerate}

\begin{solucion}\
\begin{enumerate}
\item Para $a=0$, $f(x,y)=y^3+12y$, luego $\nabla f(x,y)=\begin{pmatrix}
0\\
3y^2+12
\end{pmatrix}=\begin{pmatrix}
0\\
0
\end{pmatrix}$, que no tiene solución real. Por tanto no hay puntos críticos.
Para $a\neq 0$ $\nabla f(x,y)=\begin{pmatrix}
ay\\
3y^2+ax+12
\end{pmatrix}=\begin{pmatrix}
0\\
0
\end{pmatrix}\Leftrightarrow y=0,x=-\frac{12}{a}$. La matriz hessiana es $\begin{pmatrix}
0 & a\\
a & 6y
\end{pmatrix}\Rightarrow \nabla^2 f(-\frac{12}{a},0)=\begin{pmatrix}
0 & a\\
a & 0
\end{pmatrix}$. El menor de orden 2 es negativo, luego no hay ni máximos ni mínimos.
\item En $S_1$ podemos expresar la función en una sola variable, por ejemplo, $f(y)=y^3-16y^2+29y, f'(y)=3y^2-32y+29, f'(1)=0, f''(y)=6y-32,f''(1)<0$.
Además $f(1,1)=14$ y $f(1+\varepsilon,1+\varepsilon)>14$, luego no será máximo en $S_1$.
\end{enumerate}
\end{solucion}
\end{ejercicio}

\newpage 

\begin{ejercicio}{11}
En un mercado existen dos firman que venden gasolina. El coste de la primera firma para producir $q_1$ litros es $q_1$, mientras que para la segunda firma producir $q_1$ litros le cuesta $q_2^2$. Si se producen un total de $q$ litros, los consumidores pagarán cada litro a $200-q$.
\begin{enumerate}
\item Supongamos que las dos firmas deciden cooperar para maximizar sus beneficios. ¿Cuántos libros debe producir cada firma?
\item Supongamos ahora que la solución $(\overline{q}_1,\overline{q}_2)$ que se quiere imlementar consiste en lo siguiente:

\emph{La primera firma producirá $\overline{q}_1$ si este valor verifica que cualquier cambio del mismo supuesto que la segunda firma se mantiene en $\overline{q}_2$ supone una disminución de sus beneficios. Análogamente, si la segunda firma cambia su nivel de producción de $\overline{q}_2$ a otro valor mientras que la primera firma mantiene su producción en $\overline{q}_1$ entonces la segunda disminuye su nivel de beneficio.}

Hallar $(\overline{q}_1,\overline{q}_2)$.
\end{enumerate}
\begin{solucion}\
\begin{enumerate}


\item La función de beneficios conjuntos es $b(q_1,q_2)=(q_1+q_2)(200-(q_1+q_2))-q_1-q_2^2$, que desarrollándola obtenemos
$$b(q_1,q_2)=-q_1^2-2q_2^2-2q_1q_2+199q_1+200q_2$$
$$\nabla b(q_1,q_2)=\begin{pmatrix}
-2q_1-2q_2+199\\
-4q_2-2q_1+200
\end{pmatrix}$$
Tras igualar a 0 obtenemos la solución $q_1=99,q_2=\frac{1}{2}$. 
$$\nabla^2b(q_1,q_2)=\begin{pmatrix}
-2 & -2\\
-2 & -4
\end{pmatrix}$$
El primer menor es negativo y el segundo positivo, luego tenemos un máximo.
\item 
\begin{gather*}
b_1(q_1,q_2)=q_1(200-q_1-q_2)-q_1\\
b_2(q_1,q_2)=q_2(200-q_1-q_2)-q_2
\end{gather*}
El punto $(\overline{q}_1,\overline{q}_2)$ es tal que $b_1(\overline{q}_1,\overline{q}_2)\geq b_1(q_1,\overline{q}_2)\ \forall q_1$ y $b_2(\overline{q}_1,\overline{q}_2)\geq b_1(\overline{q}_1,q_2)\ \forall q_2$. Tenemos que hacer entonces las derivadas parciales.
$$\frac{\partial b_1(q_1,\overline{q}_2)}{\partial q_1}=199-2q_1-\overline{q}_2=0\Rightarrow \overline{q}_1=\frac{199-\overline{q}_2}{2}$$
Análogamente $\overline{q}_2=\frac{200-\overline{q}_1}{4}$. La solución al sistema de ecuaciones es $\overline{q}_1=\frac{596}{7},\overline{q}_2=\frac{201}{7}$.
\end{enumerate}
\end{solucion}
\end{ejercicio}

\end{document}