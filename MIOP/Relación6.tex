\documentclass[twoside]{article}
\usepackage{../estilo-ejercicios}

%--------------------------------------------------------
\begin{document}

\title{Modelos de Investigación Operativa\\ Relación 6}
\author{Rafael González López, Javier Aguilar Martín}
\date{}
\maketitle

\begin{ejercicio}{1}
Suponga que está interesado en elegir entre un conjunto de inversiones $I=\{1,\dots,7\}$ mediante variables binarias. Modele las siguientes restricciones:
\begin{enumerate}
\item No se puede invertir en todas.
\item Existe la obligación de invertir al menos en una de ellas.
\item La inversión 1 no puede ser escogida si la inversión 3 ha sido escogida.
\item La inversión 4 puede ser escogida solamente en el caso de que la inversión 2 también lo sea.
\item En cuanto a las inversiones 1 y 5, bien pueden ser escogidas conjuntamente o bien no se puede escoger ninguna de ellas.
\item Se debería escoger al menos una de las inversiones 1,2,3 o al menos dos entre las inversiones 2,4,5,6.
\end{enumerate}
\end{ejercicio}
\begin{solucion}\
\begin{enumerate}
\item $\sum_{i=1}^7x_i\leq 6$. 
\item $\sum_{i=1}^7\geq 1$.
\item $x_1+x_3\leq 1$.
\item $x_2\geq x_4$. 
\item $x_1-x_5=0$.
\item Usamos una variable auxiliar $z\in\{0,1\}$. Entonces $x_1+x_2+x_3\geq z$ y $x_2+x_4+x_5+x_6\geq 2(1-z)$. 
\end{enumerate}
\end{solucion}

\newpage

\begin{ejercicio}{2}
Supongamos que se está en invitar al máximo número de amigos de entre $\{A,B,C,D\}$ a una fiesta. Modele las siguientes restricciones utilizando variables binarias:
\begin{enumerate}
\item El anfitrión no hace la fiesta si van menos de 2.
\item A va si va D.
\item Si va A a la fiesta, no van ni B ni C.
\item Si van A y B, C no va.
\item Si A y B van, C no va a menos que vaya D.
\end{enumerate}
\end{ejercicio}
\begin{solucion}\
\begin{enumerate}
\item $x_A+x_B+x_C+x_D\geq 2$. 
\item $x_A\geq x_D$.
\item $x_A+ x_B\leq 1$ y $x_A+ x_C\leq 1$.  
\item $x_A+x_B+x_C\leq 2$. 
\item $x_C\leq 2-x_A-x_B+x_D$.  
\end{enumerate}
\end{solucion}

\newpage

\begin{ejercicio}{3}
Modele las siguientes situaciones:
\begin{enumerate}
\item $z\leq\min\{x,y\}, x,y,z\in\{0,1\}$.
\item $z\geq\max\{x,y\}, x,y,z\in\{0,1\}$.
\item $z\geq\min\{x,y\}, x,y,z\in\{0,1\}$.
\item $z\leq\max\{x,y\},x,y,z\in\{0,1\}$.
\item $z=\max\{x,y\}, x,y,z\in\{0,1\}$.
\item $z=\min\{x,y\}, x,y,z\in\{0,1\}$.
\item $z=|x-y|, x,y,z\in\{0,1\}$.
\end{enumerate}
\end{ejercicio}
\begin{solucion}\
\begin{enumerate}
\item $z\leq x,z\leq y$. 
\item $z\geq x,z\geq y$.
\item $z\geq x+y-1$.
\item $z\leq x+y$
\item Apartados 2 y 4.
\item Apartados 1 y 3.
\item 
\end{enumerate}
\end{solucion}

\newpage

\begin{ejercicio}{6}
Considere el siguiente problema de dos variables
\begin{align*}
\min\ & 13x_1+8x_2\\
s.a.\   & x_1+2x_2\leq 10\\
       & 5x_1+2x_2\leq 20\\
       & x_1,x_2\in\Z^+
\end{align*}
Resolver el problema por ramificación y acotación.
\end{ejercicio}
\begin{solucion}
Resuelto en la hoja. Hacer gráficamente.
\end{solucion}

\newpage

\begin{ejercicio}{7}
Considere el siguiente árbol de ramificación para un problema de minimizar:

DIBUJO DE GRAFO

Determine las cotas globales del problema en esta situación y determine los nodos para los cuales la ramificación ha finalizado explicando los motivos.
\end{ejercicio}
\begin{solucion}\

\begin{tabular}{l l}
Cota inferior: & 25 (solución del problema relajado) \\
Cota superior: & 31 (única solución entera encontrada) \\
Nodos cerrados & 7 (solución entera), 8 (infactible) y 6 (la cota superior es menor)
\end{tabular}
\end{solucion}



\end{document}