\documentclass[CV.tex]{subfiles}

\begin{document}


%\hyphenation{equi-va-len-cia}\hyphenation{pro-pie-dad}\hyphenation{res-pec-ti-va-men-te}\hyphenation{sub-es-pa-cio}

\chapter{Integración en variedades}






\section{Definición de la integral}

Sea $M$ una $n$-variedad diferenciable orientable. Se va a definir una integral
\[
\int_M:\Omega_c^n(M)\to\R
\]
el espacio vectorial de $n$-formas con soporte compacto. En el caso especial de que $M=\R^n$ (con la orientación estándar), podemos escribir $w\in\Omega_c^n(M)$ de forma única como $w=f(x)dx_1\land\dots\land dx_n$, donde $f\in C^{\infty}(\R^n;\R)$ tiene soporte compacto. Definimos entonces
\[
\int_{\R^n}f(x)dx_1\land\dots\land dx_n=\int_{\R^n} f(x)d\mu_n,
\]
donde $d\mu_n$ es la medida de Lebesgue usual de $\R^n$. La misma definición puede ser usada cuando $w\in\Omega_c^n(V)$ para un abierto $V\subseteq\R^n$, pues $w$ y $f$ son extensibles con diferenciabilidad a todo $\R^n$ haciendo que sean 0 en $\R^n-supp_V(w)$. 

\begin{lemma}
Sea $\phi:V\to W$ entre abiertos de $\R^n$ y supongamos que $\det(D_x\phi)$ tiene signo constante $\delta=\pm 1$ para todo $x\in V$. Para $w\in\Omega_c^n(W)$ tenemos que
\[
\int_V\phi^*(w)=\delta\int_Ww.
\]
\end{lemma}
\begin{proof}
Si $w=f(x)dx_1\land\dots\land dx_n$ se sigue que
\begin{align*}
\phi^*(w)=f(\phi(x))\det(D_x\phi)dx_1\land\dots\land dx_n=\delta f(\phi(x))|\det(D_x\phi)|dx_1\land\dots\land dx_n.
\end{align*}
El resultado se sigue entonces del teorema de reparametrización de integrales que afirma que
\[
\int_W f(x)d\mu_n=\int_V f(\phi(x))|\det(D_x\phi)|d\mu_n.
\]
\end{proof}

\begin{prop}
Para una $n$-variedad diferenciable orientable cualquiera $M$, existe un único operador lineal 
\[
\int_M:\Omega_c^n(M)\to\R
\]
con la propiedad de que si $w\in\Omega_c^n(\R)$ tiene soporte contenido en $U$, donde $(U,h)$ es una carta positivamente orientada, entonces
\[
\int_M w=\int_{h(U)}(h^{-1})^*w.
\]
\end{prop}
\begin{dem}
Si $w\in\Omega^n_c(M)$ y tiene $supp(w)\subseteq U$, con $(U,h)$ una carta orientada positivamente, definimos
\[
\int_M w:=\int_{h(U)}(h^{-1})^*w.
\]
Veamos que este valor no depende de la carta. Sea $(\tilde{U}, \tilde{h})$ otra carta orientada positivamente que contenga al soporte de $w$, entonces tenemos el cambio de cartas $\phi:V=h(U\cap\widetilde{U})\to W=\widetilde{h}(U\cap\widetilde{U})$ es un difeomorfismo con jacobiano positivo. Si escribimos $\tau=(\widetilde{h}^{-1})^*(w)$, aplicando el lema anterior
\[
\int_M w=\int_{h(U)}(h^{-1})^*(w)=\int_V\phi^*(\tau)=\int_W\tau.
\] 
Para el caso general, fijado un atlas positivo de $M$, elegimos una partición de la unidad subordinada al atlas $\{\rho_\alpha\}_{\alpha\in A}$. Entonces, si $w\in\Omega_c^n(M)$, $w=\sum_{\alpha\in A}\rho_\alpha w$. Para cada $\alpha\in A$, $supp(\rho_\alpha w)\subseteq U_\alpha$ y es compacto. Definimos
\[
\int_M w=\sum_{\alpha\in A}\int_M\rho_\alpha w. 
\]
Además, por las propiedades de la partición de la unidad, en cada punto tenemos una suma finita. En particular, si $w$ tiene soporte pequeño, las definiciones coinciden. Claramente el operador definido es lineal y continuo. La unicidad se sigue de la propia demostración, pues si hubiera otro homomorfismo verificando la definición, entonces podríamos escribirlo, eligiendo una partición de la unidad, podemos escribirlo como la suma anterior, y cada sumando debe coincidir pues debe ser el mismo número en las formas de soporte pequeño.
\QED
\end{dem}
\begin{lemma}
Sea $M$ una variedad diferenciable orientable de dimensión $n$.
\begin{enumerate}
\item  $\int_M w$ cambia de signo cuando se recierte la orientación de $M$.
\item Si $w\in \Omega_c^n(M)$ tiene soporte contenido en un abierto $W\subseteq M$, entonces
\[
\int_M w=\int_W w.
\]
\item Si $\phi:M\to N$ es un difeomorfismo que preserva la orientación, entonces se tiene que
\[
\int_M w=\int_N\phi^*(w)
\]
para $w\in\Omega_c^n(M)$. 
\end{enumerate}
\end{lemma}

\begin{nota}
Extendemos la definición a $n$-formas continuas. Supongamos $M$ dotada de una forma de orientación $\sigma$. Consideramos $\{f\sigma:f\in C^0(M;\R)\}$. Se tiene $supp(f\sigma)=supp(f)\subseteq M$. Podemos definir 
\[
I_\sigma:C_c^0(M;\R)\to\R
\]
como $I_\sigma(f)=\int_M f\sigma$, que es lineal y de hecho $I_\sigma(f)\geq 0$ si $f\geq 0$. 

Usando particiones de la unidad subornidada a un atlas positivo orientado, basta definid $I_\sigma(f)$ sobre aquellas $f$ de soporte pequeño en una carta positiva $(U,h)$. En tal caso
\[
I_\sigma(f)=\int_M f\sigma:=\int_{h(U)}f\circ h^{-1}\phi(x) d\mu_n
\]
donde $\phi$ está determinado por $(h^{-1})^*(\sigma)=\phi(x)dx_1\land\dots\land dx_n$. Como $\phi$ es positiva, $I_\sigma(f)\geq 0$.

De acuerdo con el teorema de representación de Riesz que enunciaremos a continuación, $I_\sigma$ determina una medida positiva $\mu_\sigma$ en $M$ satisfaciendo
\[
\int_M f(x)d\mu_\sigma=\int_M f\sigma, \quad f\in C_c^0(M;\R)
\]
con lo que se podrán utilizar los resultados de la integral de Lebesgue. 

\begin{teorema}[de Representación de Riesz]
Sea $X$ un espacio topológico localmente compacto y Hausdorff. Sea $\Lambda:C_c^0(X)\to\R$ un homeomorfismo lineal tal que $\Lambda(f)\geq 0$ para toda $f\geq 0$. Entonces existe un $\sigma$-álgebra $\mathcal{M}$ en $X$ conteniendo a los conjuntos de Borel de $X$ y existe una única medida positiva $\mu$ que representa a $\sigma$, es decir, $\Lambda(f)=\int_X f d\mu$.
\end{teorema}

Si $M$ es una $n$-variedad Riemanniana orientada, el volumen $vol_M$ determinará una medida $\mu_M$ en $M$ análoga a la de Lebesgue en $\R^n$. Para un conjunto compacto $K$, el volumen de $K$ puede definirse como 
\[
Vol(K)=\int_K vol_M\in\R.
\]
\end{nota}

\begin{ej}
Si quisiéramos calcular $vol(S^1)=\int_{S^1} (dx- dy)$ necesitaríamos hacer una partición de la unidad, lo cual sería muy complicado. Entonces se calcula $\int_{S^1} f d\mu_{vol_{S^1}}=\int_{S^1-\{N,S\}} f d\mu_{vol_{S^1}}=\int_{S^1-\{N,S\}} vol$, que no es más que la suma de las integrales en los dos abiertos, que sabemos calcularla.
\end{ej}

Lo siguiente que queremos es definir la integral en variedades con borde diferenciable. Sea $M$ una variedad diferenciable de dimensión $n$. Si $N\subseteq M$ es un dominio con borde diferenciable, entonces $\partial N$ es una subvariedad diferenciable de $M$ de dimensión $n-1$. Si $n\geq 2$ y $M$ es orientable, entonces existe una orientación sobre $\partial N$ tal que para $p\in\partial N$ y $v_1\in T_pM$ ``apuntando hacia fuera'' entonces $\{v_2,\dots, v_n\}$ es base de $T_p\partial N$ positivamente orientada si y solo si $\{v_1,v_2,\dots, v_n\}$ es base de $T_pM$ positivamente orientada. 

Vamos a dar más rigor a lo descrito anterior mente. Dado $v_1\in T_pM$ con $p\in\partial N$, se dice que $v_1$ \textbf{apunta hacia fuera} si existe una carta $(U,h)$ alrededor de $p$ (adaptada a $N$, es decir, $h(U\cap N)=h(U)\cap \R^n_-$) tal que $D_ph(v_1)=w_1$ tiene la primera coordenada positiva. 

Dada una base $\{e_1,e_2,\dots, e_n\}\subset T_{h(p)}\R^n$ positivamente orientada, como $D_ph$ es isomorfismo y la carta es positiva, las preimágenes de esta base forman una base positivamente orientada de $T_pM$. Bastará llamar $v_1,v_2,\dots, v_n$ a estas preimágenes, y de modo que $D_ph(v_i)=e_i$. 

\begin{nota}
Si $w\in\Omega_c^n(M)$ y $N\subseteq M$ es un dominio diferenciable con borde podemos definir
\[
\int_N w=\int_M\chi_N w=\int_M \chi_N f\sigma=\int_M \chi_N f d\mu_{\sigma}
\]
pues $supp(w)$ es compacto en $M$, $\chi_N$ es integrable y $supp(\chi_Nw)$ es compacto en $N$. Aquí $\sigma$ representa la orientación de $M$. Podemos por tanto integrar las formas compactas de $M$ sobre dominios de $N$.

En el caso $n=1$, si $N\subseteq M$, entonces $\dim(N)=1$ y $\dim(\partial N)=0$, por lo que el borde será una colección de puntos. Una orientación sobre $\partial N$ consiste en asignar signo $\pm$ a cada uno de los puntos de $\partial N$. Dado $v_1\in T_pM$ dirigido hacia fuera: a $p$ le asignamos signo positivo si $v_1$ define una orientación positiva de $T_pM$ y negativo en caso contrario. Nótese que las únicas variedades diferenciables conexas y cerradas de dimensión 1 son $\R$ y $S^1$, por lo que solo habrá una cantidad finita de puntos en el borde de cualquier subvariedad del mismo tipo. Si $f\in\Omega^0(\partial N)$ entonces 
\[
\int_{\partial N} f=\sum_{p\in\partial N} sgn(p)f(p)
\]
\end{nota}

\begin{teorema}[Stokes]
Sea $N\subseteq M$ un dominio diferenciable con borde de la $n$-variedad orientada $M$ y supongamos que $\partial N$ tiene la orientación inducida. Sea $w\in\Omega^{n-1}(M)$ de modo que $supp(w)\cap N$ sea compacto. Entonces
\[
\int_{\partial N} i^*(w)=\int_N dw.
\]
donde $i:\partial N\to M$ es la inclusión (y $w$ se entiende también como la restricción a $N$).
\end{teorema}
\begin{dem}
La demostración se hará para $n=2$, pues para $n=1$ es el teorema fundamental del cálculo. Como $i^*(w)$ tiene soporte compacto, podemos elegir una función $f\in\Omega_c^0(M)$ que valga 1 en $N\cap supp(w)$ para considerar $fw$ que vale lo mismo que $w$ sobre $N$. Las integrales valen lo mismo cuando reemplazamos $w$ por $fw$, asíq ue podemos suponer que $w$ tiene soporte compacto desde el principio. 

Tomamos un atlas sobre $M$ con cartas adaptadas a $N$ positivamente orientadas y una partición de la unidad $\{\rho_{\alpha}\}_{\alpha\in A}$ subordinada al atlas. Por un lado
\[
\int_{\partial N} i^*(w)=\sum_{\alpha\in A}\int_{\partial N}\rho_{\alpha} i^*(w)
\]
y por otro
\[
\int_N dw=\sum_{\alpha\in A}\int_N d(\rho_\alpha w)
\]
así que podemos reducir al caso:
\begin{enumerate}[a.]
 \item $w\in\Omega_c^{n-1}(M)$
 \item $supp(w)\subseteq U$, con $(U,h)$ una carta adaptada orientada positivamente. 
\end{enumerate}

Tomamos $k\in\Omega_c^{n-1}(\R^n)$ tal que $k|_{h(U)}= (h^{-1})^*(w)$ y $k|_{\R^n-h(U)}=0$. Entonces por la invarianza de la integral
\[
\int_{\partial N} w=\int_{h(U)\cap \partial \R^n_-} (h^{-1})^*(w)=\int_{\partial\R^n_-} k
\]
y 
\[
\int_N dw=\int_{h(U)\cap\R^n_-}(h^{-1})^*(dw)=\int_{\R^n_-}dk.
\]
Nos reducimos al caso $M=\R^n$ y $N=\R^n_-$ con $w\in\Omega_c^{n-1}(\R^n)$. Tenemos que $w=\sum_{i=1}^n f_i dx_1\land\cdots\land \hat{dx}_i\land\cdots\land dx_n$ con $f_i\in\Omega_c^0(\R^n)$. Existe $b>0$ tal que $supp(f_i)\subsetneq [-b,b]^n$. Ahora, 
\[
i^*(w)=w_{\partial\R^n_-}=f_1(0,x_2,\dots, x_n)dx_2\land\cdots\land dx_n.
\]
Dado $p\in\partial\R^n_-$ y $v_2,\dots, v_n\in T_p(\partial\R^n_-)$, $i^*(w)_p(v_2,\dots, w_n)=w_{i(p)}(D_pi(v_2),\dots, D_pi(v_n))$, donde $i:\partial\R^n_-=\R^{n-1}\to \R^n$ es la aplicación $(x_2,\dots, x_n)\mapsto (0,x_2,\dots, x_n)$, por lo que $D_p i:T_p\R^{n-1}\to R_p\R^n$ es la aplicación $(t_2,\dots, t_n)\mapsto (0,t_2,\dots, t_n)$. Como empiezan por 0, cada vez que aparezca $dx_1$, el término se anulará. Por tanto
\[
\int_{\partial\R^n_-}i^*(w)=\int_{\partial\R^n_-} f_1(0,x_2,\dots, x_n)dx_1\land\cdots\land dx_n=\int_{\partial\R^n_-}f_1(0,x_2,\dots, x_n) d\mu_{n-1}
\]
Por otra parte
\[
dw=\sum_{i=1}^n df_i\land dx_1\land\cdots\land \hat{dx}_i\land\cdots\land dx_n=\sum_{i=1}^n(-1)^{i-1}\parcial{f}{x_i}dx_1\land\cdots\land \hat{dx}_i\land\cdots\land dx_n
\]
Así que
\[
\int_{\R^n_-}dw=\sum_{i=1}^n(-1)^{i-1}\int_{\R^n_-}\parcial{f}{x_i}d\mu_n
\]
Ahora, para $2\leq i\leq n$, 
\[
\int_{-\infty}^{+\infty}\parcial{f_i}{x_i}(x_1, \dots, t,\dots, x_n)dt =\int_{-b}^b\parcial{f_i}{x_i}(x_1,\dots, t,\dots, x_n)dt=f_i(x,\dots, b,\dots, x_n)-f_i(x_1,\dots, -b, \dots, x_n)=0
\]
pues al estar $supp(f_i)\subsetneq [-b,b]^n$ ambos términos se anulan. Entonces, aplicando Fubini
\[
\int_{\R^n_-}\parcial{f_i}{x_i}d\mu_n=\int_{-\infty}^0\int_{-\infty}^{+\infty}\cdots\int_{-\infty}^{+\infty} (\parcial{f_i}{x_i}dx_n)\cdots dx_2dx_1=0.
\]
Para $i=1$, 
\[
\int_{-\infty}^0\parcial{f_1}{x_1}(t,x_2,\dots, x_n)dt=\int_{-b}^0(t,x_2,\dots, x_n)dt=f_1(0,x_2,\dots, x_n)-f_1(-b,x_2,\dots, x_n)=f_1(0,x_2,\dots, x_n).
\]
Por lo que usando Fubini
\[
\int_{\R^n_-}\parcial{f_1}{x_1}d\mu_n=\int_{-\infty}^{+\infty}\cdots\int_{-\infty}^{+\infty}\int_{-\infty}^0\parcial{f_1}{x_1}(t,x_2,\dots, x_n)dt)dx_2\cdots dx_n=
\]
\[
\int_{\partial\R^{n-1}}f_1(0,x_2,\dots, x_n)dx_2\land\cdots\land dx_n=\int_{\partial\R^n_-}f_1(0,x_2,\dots, x_n)d\mu_{n-1}.
\]
Con lo que hemos probado el resultado.
\QED
\end{dem}

\begin{coro}
Si $M$ es una $n$-variedad diferenciable orientada y $w\in \Omega_c^{n-1}(M)$, entonces
\[
\int_M dw=0.
\]
\end{coro}

\begin{teorema}\label{aumento}
Sea $M$ una variedad diferenciable orientada conexa de dimensión $n$. Entonces
\[
\cdots\to\Omega_c^{n-1}(M)\xrightarrow{d}\Omega_c^{n}(M)\xrightarrow{\int_M}\R\to 0
\]
es exacta.
\end{teorema}

\begin{coro}
Si $M$ es además compacta, 
\[
\cdots\to\Omega^{n-1}(M)\xrightarrow{d}\Omega^{n}(M)\xrightarrow{\int_M}\R\to 0
\]
es exacta e induce un isomorfismo $\int_M:H^n(M)\to\R$. 
\end{coro}
\begin{dem}
Si $\tau\in\Omega^{n-1}(M)$, $d\tau\in\Omega^n(M)$, luego $\int_M d\tau=0$, así que $\int_M$ pasa al cociente.  El homomorfismo $\varphi=\int_M:H^n(M)\to\R$,$[z]\mapsto \int_M z=\varphi(z)$:
\begin{itemize}
\item es sobreyectivo por exactitud.
\item es inyectivo, si $\varphi([w])=\int_M w=0$, por exactitud, $w=d\rho$ para $\rho\in\Omega^{n-1}(M)$, así que $[w]=[d\rho]=0\in H^n(M)$.
\end{itemize}
\QED
\end{dem}

En particular, $\int_M vol\neq 0$ porque $vol_M$ no se anula, así que $[vol_M]$ es un generador de $H^n(M)$. 

Para probar el teorema \ref{aumento} hacen falta varios lemas probados en el libro.
\begin{lemma}\label{A}
El teorema \ref{aumento} es válido para $M=\R^n$ ($n\geq 1$).
\end{lemma}
\begin{lemma}
Dado un recubrimiento por abiertos $\{U_{\alpha}\}$ de una variedad diferenciable conexa $M$ y dados $p,q\in M$, existen $\alpha_1,\dots, \alpha_r\in A$ tales que $p\in U_{\alpha_1}$, $q\in U_{\alpha_r}$ y $U_{\alpha_i}\cap U_{\alpha_{i+1}}\neq\emptyset$ para $1\leq i\leq r-1$.
\end{lemma}
\begin{lemma}
Sea $U\subseteq M$ un abierto de una variedad diferenciable difeomorfo a $\R^n$ y $W\subseteq U$ abierto. Entonces para toda $w\in\Omega_c^n(M)$ con $supp(w)\subseteq U$, existe $k\in\Omega_c^{n-1}(M)$  con $supp(k)\subseteq U$ tal que $supp(w-dk)\subseteq W$. 
\end{lemma}
\begin{lemma}\label{D}
Si $M$ es una variedad diferenciable conexa y $\emptyset\neq W\subseteq M$ es abierto, entonces para toda $w\in\Omega_c^n(M)$ existe $k\in\Omega_c^{n-1}(M)$ con $supp(w-dk)\subseteq W$. 
\end{lemma}

\begin{dem}[del teorema \ref{aumento}]

Supongamos que tenemos $w\in\Omega^n_c(M)$ con $\int_M=0$. Elegimos $\R^n\cong W\subseteq M$ abierto. Por el lema \ref{D} existe $k\in\Omega_c^{n-1}(M)$ con $supp(w-dk)\subseteq W$. Así que
\[
\int_W w-dk=\int_M w-dk=\int_M w-\int_M dk=0-0=0
\]
Ahora, aplicando el lema \ref{A}, existe $\tau_0\in\Omega_c^{n-1}(W)$ tal que $d\tau_0=w-dk$ en $W$. Extendemos $\tau_0$ a todo $M$ haciendo que valga 0 fuera de $W$, por lo que obtenemos $\tau\in\Omega_c^{n-1}(M)$. Entonces $w-dk=d\tau$, luego $w=d(\tau+k)$. \QED
\end{dem}
\end{document}

