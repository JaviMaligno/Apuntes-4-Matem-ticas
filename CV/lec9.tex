\documentclass[CV.tex]{subfiles}

\begin{document}


%\hyphenation{equi-va-len-cia}\hyphenation{pro-pie-dad}\hyphenation{res-pec-ti-va-men-te}\hyphenation{sub-es-pa-cio}

\chapter{Integración en variedades}






\section{Definición de la integral}

Sea $M$ una $n$-variedad diferenciable orientable. Se va a definir una integral
\[
\int_M:\Omega_c^n(M)\to\R
\]
el espacio vectorial de $n$-formas con soporte compacto. En el caso especial de que $M=\R^n$ (con la orientación estándar), podemos escribir $w\in\Omega_c^n(M)$ de forma única como $w=f(x)dx_1\land\dots\land dx_n$, donde $f\in C^{\infty}(\R^n;\R)$ tiene soporte compacto. Definimos entonces
\[
\int_{\R^n}f(x)dx_1\land\dots\land dx_n=\int_{\R^n} f(x)d\mu_n,
\]
donde $d\mu_n$ es la medida de Lebesgue usual de $\R^n$. La misma definición puede ser usada cuando $w\in\Omega_c^n(V)$ para un abierto $V\subseteq\R^n$, pues $w$ y $f$ son extensibles con diferenciabilidad a todo $\R^n$ haciendo que sean 0 en $\R^n-supp_V(w)$. 

\begin{lemma}
Sea $\phi:V\to W$ entre abiertos de $\R^n$ y supongamos que $\det(D_x\phi)$ tiene signo constante $\delta=\pm 1$ para todo $x\in V$. Para $w\in\Omega_c^n(W)$ tenemos que
\[
\int_V\phi^*(w)=\delta\int_Ww.
\]
\end{lemma}
\begin{proof}
Si $w=f(x)dx_1\land\dots\land dx_n$ se sigue que
\begin{align*}
\phi^*(w)=f(\phi(x))\det(D_x\phi)dx_1\land\dots\land dx_n=\delta f(\phi(x))|\det(D_x\phi)|dx_1\land\dots\land dx_n.
\end{align*}
El resultado se sigue entonces del teorema de reparametrización de integrales que afirma que
\[
\int_W f(x)d\mu_n=\int_V f(\phi(x))|\det(D_x\phi)|d\mu_n.
\]
\end{proof}

\begin{prop}
Para una $n$-variedad diferenciable orientable cualquiera $M$, existe un único operador lineal 
\[
\int_M:\Omega_c^n(M)\to\R
\]
con la propiedad de que si $w\in\Omega_c^n(\R)$ tiene soporte contenido en $U$, donde $(U,h)$ es una carta positivamente orientada, entonces
\[
\int_M w=\int_{h(U)}(h^{-1})^*w.
\]
\end{prop}
\begin{dem}
Si $w\in\Omega^n_c(M)$ y tiene $supp(w)\subseteq U$, con $(U,h)$ una carta orientada positivamente, definimos
\[
\int_M w:=\int_{h(U)}(h^{-1})^*w.
\]
Veamos que este valor no depende de la carta. Sea $(\tilde{U}, \tilde{h})$ otra carta orientada positivamente que contenga al soporte de $w$, entonces tenemos el cambio de cartas $\phi:V=h(U\cap\widetilde{U})\to W=\widetilde{h}(U\cap\widetilde{U})$ es un difeomorfismo con jacobiano positivo. Si escribimos $\tau=(\widetilde{h}^{-1})^*(w)$, aplicando el lema anterior
\[
\int_M w=\int_{h(U)}(h^{-1})^*(w)=\int_V\phi^*(\tau)=\int_W\tau.
\] 
Para el caso general, fijado un atlas positivo de $M$, elegimos una partición de la unidad subordinada al atlas $\{\rho_\alpha\}_{\alpha\in A}$. Entonces, si $w\in\Omega_c^n(M)$, $w=\sum_{\alpha\in A}\rho_\alpha w$. Para cada $\alpha\in A$, $supp(\rho_\alpha w)\subseteq U_\alpha$ y es compacto. Definimos
\[
\int_M w=\sum_{\alpha\in A}\int_M\rho_\alpha w. 
\]
Además, por las propiedades de la partición de la unidad, en cada punto tenemos una suma finita. En particular, si $w$ tiene soporte pequeño, las definiciones coinciden. Claramente el operador definido es lineal y continuo. La unicidad se sigue de la propia demostración, pues si hubiera otro homomorfismo verificando la definición, entonces podríamos escribirlo, eligiendo una partición de la unidad, podemos escribirlo como la suma anterior, y cada sumando debe coincidir pues debe ser el mismo número en las formas de soporte pequeño.
\QED
\end{dem}
\begin{lemma}
Sea $M$ una variedad diferenciable orientable de dimensión $n$.
\begin{enumerate}
\item  $\int_M w$ cambia de signo cuando se recierte la orientación de $M$.
\item Si $w\in \Omega_c^n(M)$ tiene soporte contenido en un abierto $W\subseteq M$, entonces
\[
\int_M w=\int_W w.
\]
\item Si $\phi:M\to N$ es un difeomorfismo que preserva la orientación, entonces se tiene que
\[
\int_M w=\int_N\phi^*(w)
\]
para $w\in\Omega_c^n(M)$. 
\end{enumerate}
\end{lemma}

\begin{nota}
Extendemos la definición a $n$-formas continuas. Supongamos $M$ dotada de una forma de orientación $\sigma$. Consideramos $\{f\sigma:f\in C^0(M;\R)\}$. Se tiene $supp(f\sigma)=sup(f)\subseteq M$. Podemos definir 
\[
I_\sigma:C_c^0(M;\R)\to\R
\]
como $I_\sigma(f)=\int_M f\sigma$, que es lineal y de hecho $I_\sigma(f)\geq 0$ si $f\geq 0$. 

Usando particiones de la unidad subornidada a un atlas positivo orientado, basta definid $I_\sigma(f)$ sobre aquellas $f$ de soporte pequeño en una carta positiva $(U,h)$. En tal caso
\[
I_\sigma(f)=\int_M f\sigma:=\int_{h(U)}f\circ h^{-1}(x)\phi(x) d\mu_n
\]
donde $\phi$ está determinado por $(h^{-1})^*(\sigma)=\phi(x)dx_1\land\dots\land dx_n$. Como $\phi$ es positiva, $I_\sigma(f)\geq 0$.

De acuerdo con el teorema de representación de Riesz, $I_\sigma$ determina una medida positiva $\mu_\sigma$ en $M$ satisfaciendo
\[
\int_M f(x)d\mu_\sigma=\int_M f\sigma, \quad f\in C_c^0(M;\R)
\]
con lo que se podrán utilizar los resultados de la integral de Lebesgue. 

Si $M$ es una $n$-variedad Riemanniana, el volumen $vol_M$ determinará una medida $\mu_M$ en $M$ análoga a la de Lebesgue en $\R^n$. Para un conjunto (compacto) $K$, el volumen de $K$ puede definirse como 
\[
Vol(K)=\int_K vol_M\in\R.
\]
\end{nota}

\end{document}
