\documentclass[CV.tex]{subfiles}

\begin{document}


%\hyphenation{equi-va-len-cia}\hyphenation{pro-pie-dad}\hyphenation{res-pec-ti-va-men-te}\hyphenation{sub-es-pa-cio}

\chapter{La sucesión de Mayer-Vietoris}

\section{Definición de la sucesión de Mayer-Vietoris}
Aquí desarrollaremos la llamada sucesión exacta larga (de cohomología) de Mayer-Vietoris que nos permitirá calcular la cohomología de de Rham de abiertos euclídeos.

\begin{teorema}
Sean $U_1,U_2\subseteq\R^n$ abiertos y $U=U_1\cup U_2$. Sean $i_{\nu}:U_{\nu}\to U$ y $j_{\nu}:U_1\cap U_2\to U_{\nu}$ ($\nu=1,2$) las correspondientes inclusiones. Entonces, la sucesión
\[
0\to \Omega^p(U)\overset{I^p}{\to}\Omega^p(U_1)\oplus\Omega^p(U_2)\overset{J^p}{\to}\Omega^p(U_1\cap U_2)\to 0
\]
es exacta, donde $I^p(w)=(i_1^*(w), i_2^*(w))$ y $J^p(w_1,w_2)=j_1^*(w_1)-j_2^*(w_2)$.
\end{teorema}
\begin{dem}
Dada una aplicación diferenciable $\phi:V\to W$ y una $p$-forma $w\sum f_Idx_I\in\Omega^p(W)$, 
\[
\Omega^p(\phi)(w)=\phi^*(w)=\sum(f_I\circ\phi)d\phi_{i_1}\land\dots\land d\phi_{i_p}.
\]
En particular, si $\phi$ es una inclusión de abiertos euclídeos, esto es, $\phi_i(x)=x_i$, entonces $d\phi_{i_1}\land\dots\land d\phi_{i_p}=dx_{i_1}\land\dots\land dx_{i_p}$. Por tanto, 
\begin{equation}\label{eq}
\phi^*(w)=\sum(f_I\circ\phi)dx_I.
\end{equation}
Esto será usado para $\phi= i_{\nu},j_{\nu}$ ($\nu=1,2$). Se sigue de \ref{eq} que $I^p$ es inyectiva, pues si $I^p(w)=0$, entonces $i_1^*(w)=0=i_2^*(w)$, es decir,
\[
i_{\nu}^*(w)=\phi^*(w)=\sum(f_I\circ i_{\nu})dx_I=0
\]
si y solo si $f_I\circ i_{\nu}=0$ para todo $I$. Sin embargo, $f_I\circ i_{1}= f_I\circ i_{2}=0$ implica que $f_I=0$ en todo $U$, ya que $U_1$y $U_2$ cubren $U$. Probamos ahora que $\ker{J^p}=\Ima{I^p}$.  En primer lugar, 
\[
J^p\circ I^p(w)=j_2^*i_2^*(w)-j_1^*i_1^*(w)=j^*(w)-j^*(w)=0,
\]
donde $j:U_1\cap U_2\to U$ es la inclusión. Por tanto, $\Ima{I^p}\subseteq \ker{J^p}$. Para probar la inclusión contraria, tomamos dos $p$-formas $w_{\nu}\in\Omega^p(U_{\nu})$,
\[
w_1=\sum f_Idx_I,\quad w_2=\sum g_Idx_I.
\]
Si $J^p(w_1,w_2)=0$, entonces $j^*_1(w_1)=j_2^*(w_2)$, lo cual por \ref{eq} implica que $f_I\circ i_1= g_I\circ i_2$, o lo que es lo mismo, $f_I=g_I$ en $U_1\cap U_2$. Definimos la función diferenciable $h_I:U\to\R^n$ como
\[
h_I(x)=\begin{cases}
f_I(x) & x\in U_1,\\
g_I(x) & x\in U_2.
\end{cases}
\]
Entonces, $I^p(\sum h_Idx_I)=(w_1,w_2)$. Por último, probamos que $J^p$ es sobreyectiva. Para ello usamos una partición de la unidad $\{p_1,p_2\}$ con soporte en $\{U_1, U_2\}$, esto es, funciones diferenciables
\[
p_{\nu}:U\to [0,1]
\]
para las que $supp_U(p_{\nu})\subseteq U_{\nu}$ y tales que $p_1(x)+p_2(x)=1$ para todo $x\in U$ (ver apéndice A del libro). Sea $f:U_1\cap U_2\to \R$ una función diferenciable. Usamos la partición para extenderla a $U_1$ y $U_2$. Como $supp_U(p_1)\cap U_2\subseteq U_1\cap U_2$, podemos definir una función diferenciable mediante
\[
f_2(x)=\begin{cases}
-f(x)p_1(x) & x\in U_1\cap U_2,\\
0 & x\in U_2-supp_U(p_1).
\end{cases}
\]
Efectivamente, es diferenciable pues en la intersección de los dominios la función vale 0, ya que $p_1(x)=0$ en $U_2-supp_U(p_1)$.  Análogamente definismo
\[
f_1(x)=\begin{cases}
f(x)p_2(x) & x\in U_1\cap U_2,\\
0 & x\in U_1-supp_U(p_2).
\end{cases}
\]
Nótese que $f_1(x)-f_2(x)=f(x)$ para todo $x\in U_1\cap U_2$, pues $p_1(x)+p_2(x)=1$. Para una forma diferencial $w\in \Omega^p(U_1\cap U_2)$, $w=\sum f_Idx_I$, podemos aplicarle el procedimiento anterior a cada $f_I:U_1\cap U_2\to\R$, dando lugar a funciones $f_{I,\nu}:U_{\nu}$, y por tanto, a formas diferenciales $w_{\nu}=\sum f_{I,\nu}dx_I\in\Omega^p(U_{\nu})$. Con esta elección, $J^p(w_1,w_2)=w$. 
\QED
\end{dem}

Claramente, $I^*$ y $J^*$ son aplicaciones de complejos de cadenas, por lo que obtenemos una sucesión exacta
\[
0\to \Omega^*(U)\overset{I^*}{\to}\Omega^*(U_1)\oplus\Omega^*(U_2)\overset{J^*}{\to}\Omega^*(U_1\cap U_2)\to 0
\]
de complejos de cadenas, a la cual hay una sucesión exacta larga de cohomología asociada
\[
\cdots\to H^p(U)\overset{I^p}{\to}H^p(\Omega^*(U_1)\oplus \Omega^*(U_2))\overset{J^p}{\to}H^p(U_1\cap U_2)\overset{\partial^*}{\to}H^{p+1}(U)\to \cdots
\]
Además, sabemos que $H^p(\Omega^*(U_1)\oplus \Omega^p(U_2))=H^p(\Omega^*(U_1))\oplus H^p(\Omega^*(U_2))=H^p(U_1)\oplus H^p(U_2)$. Así que se tiene el siguiente teorema.

\begin{teorema}[sucesión de Mayer-Vietoris] Sean $U_1,U_2\in\R^n$ abiertos y $U=U_1\cup U_2$. Entonces existe una sucesión exacta larga en cohomología
\[
\cdots\to H^p(U)\overset{I^p}{\to}H^p(U_1)\oplus H^p(U_2)\overset{J^p}{\to}H^p(U_1\cap U_2)\overset{\partial^*}{\to}H^{p+1}(U)\to \cdots
\]
donde $I^p([w])=(i_1^*([w]), i_2^*([w]))$ y $J^p([w_1],[w_2])=j_1^*([w_1])-j_2^*([w_2])$.
\end{teorema}

\begin{coro}\label{coro}
Si $U_1$ y $U_2$ son abiertos disjuntos de $\R^n$ entonces
\[
I^*: H^p(U_1\cup U_2)\to H^p(U_1)\oplus H^p(U_2)
\]
es isomorfismo.
\end{coro}

La demostración es muy sencilla, así que se deja como ejercicio.

\begin{ej}
Vamos a calcular $H^*(\R^2-\{0\})$. Sean
\begin{align*}
&U_1=\R^2-\{(x_1,x_2)\mid x_1\geq 0, x_2=0\}\\
&U_2=\R^2-\{(x_1,x_2)\mid x_1\leq 0, x_2=0\}.
\end{align*}
Estos son conjuntos abiertos estrellados, así que $H^p(U_1)=H^p(U_2)=0$ para todo $p>0$ y $H^0(U_1)=H^0(U_2)=\R$. De hecho $H^0(U_{\nu})=\langle [\chi_{U_{\nu}}]\rangle$. Su intersección es $\R^2_+\sqcup\R^2_-$, así que por el corolario \ref{coro} aplicado a los semiplanos, $H^p(U_1\cap U_2)=0$ para todo $p>0$ y $H^0(U_1\cap U_2)=\R\oplus\R$, con generadores $[\chi_{\R^2_+}],[\chi_{\R^2_-}]$. En la sucesión de Mayer-Vietoris, para $p>0$ tenemos
\[
0\to H^p(U_1\cap U_2)\overset{\partial^*}{\to} H^{p+1}(\R^2-\{0\})\to 0,
\]
lo cual indica que $\partial^*$ es un isomorfismo, así que $H^{q}(\R^2-\{0\})=0$ para $q\geq 2$. Para $p=0$ obtenemos
\[
\begin{tikzcd}
0\arrow[r] & H^0(\R^2-\{0\})\arrow[r,"I^0"] & H^0(U_1)\oplus H^0(U_2)\arrow[r, "J^0"]\arrow[d,equals] & H^0(U_1\cap U_2)\arrow[r,"\partial^*"]\arrow[d, equals] & H^1(\R^2-\{0\})\arrow[r,"I^1"] & 0\\
 & & \R\oplus\R &\R\oplus\R & & 
\end{tikzcd}
\]
Como $\R^2-\{0\}$ es conexo, $H^0(\R^2-\{0\})\cong\R$ generador por $[\chi_{\R^2-\{0\}}]$. Así que como $I^0$ es inyectiva por exactitud, $\Ima{I^0}\cong\R$. Esto nos da a su vez $\ker{J^0}\cong\R$. Por el teorema del rango-nulidad, esto implica que $\Ima{J^0}\cong\R$. Utilizando el primer teorema de isomorfía teniendo en cuenta que $\partial^*$ es sobreyectiva y la exactitud, tenemos
\[
H^0(U_1\cap U_2)/\Ima{J^0}\cong H^1(\R^2-\{0\}).
\]
Pero $H^0(U_1\cap U_2)/\Ima{J^0}\cong\R$, por lo que concluimos
\[
H^p(\R^2-\{0\})=\begin{cases}
0 & p\geq 2\\
\R & p=1\\
\R & p=0.
\end{cases}
\]

Además, que $\partial^*$ sea sobreyectiva implica que $H^1(\R^2-\{0\})$ está generado por la imagen del algún elemento de $H^0(U_1\cap U_2)$. Además, por exactitud debe ser un elemento que no esté en la imagen de $J^0$. Siconsideramos $[\chi_{U_1}]$ y $[\chi_{U_2}]$, entonces $J^0([\chi_{U_1}])=j_2^*[\chi_{U_1}]=[\chi_{U_1}\circ j_1]=[\chi_{U_1\cap U_2}]=[\chi_{\R^2_+}]+[\chi_{\R^2_-}]$. Análogamente, $J^0[\chi_{U_2}]=-[\chi_{U_1\cap U_2}]=-[\chi_{\R^2_+}]-[\chi_{\R^2_-}]$. Tomando coordenadas, podemos decir que $\Ima{J^0}=\langle (1,1)\rangle$. Por tanto, dando un generador del cociente $H^0(U_1\cap U_2)/\Ima{J^0}=\langle (1,0),(0,1)\rangle/\langle (1,1)\rangle$ tenemos un elemento cuya imagen mediante $\partial^*$ genera $H^1(\R^2-\{0\})$. Esto es sencillo, pues podemos tomar, por ejemplo $(1,0)+\langle (1,1)\rangle$, que se corresponde con $[\chi_{\R^2_+}]+\Ima{J^0}$. También podríamos haber tomado de forma análoga como generador $[\chi_{\R^2_-}]+\Ima{J^0}$, o cualquier combinación de ambos.
\end{ej}

\begin{teorema}
Sea $U$ un abierto cubierto por abiertos convexos $U_1,\dots, U_r$. Entonces, para todo $p$, $H^p(U)$ es finitamente generado.
\end{teorema}
\begin{dem}
Lo probamos por inducción en el número de convexos. Si $r=1$ se sigue del lema de Poincaré. Supongamos que el resultado es cierto para $r-1$ y sea $V=U_1\cup\dots\cup U_{r-1}$ de tal forma que $U=V\cup U_r$. Por la sucesión de Mayer-Vietoris tenemos la sucesión exacta
\[
H^{p-1}(U)(V\cap U_r)\overset{\partial^*}{\to} H^p(U)\overset{I^*}{\to} H^p(V)\oplus H^p(U_r),
\]
lo cual implica, por el lema \ref{sum} usando que $\ker{I^*}=\Ima{\partial^*}$, que $H^p(U)\cong \Ima{\partial^*}\oplus\Ima{I^*}$. Ahora, tanto, $V$ como $V\cap U_r=(U_1\cap U_r)\cup\dots\cup (U_{r-1}\cap U_r)$ son uniones de $r-1$ abiertos convexos. Así que $H^{p-1}(U)(V\cap U_r)$, $H^p(V)$ y $H^p(U_r)$ son finitamente generados, lo que implica el resultado para $H^p(U)$. \QED
\end{dem}

\end{document}

