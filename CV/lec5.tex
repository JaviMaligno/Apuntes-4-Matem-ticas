\documentclass[CV.tex]{subfiles}

\begin{document}


%\hyphenation{equi-va-len-cia}\hyphenation{pro-pie-dad}\hyphenation{res-pec-ti-va-men-te}\hyphenation{sub-es-pa-cio}

\chapter{Homotopía}

\section{Definiciones}

lema 2: 

$X\sim_H Y\sim_K Z$, entonces cogemos $g1 f1\sim g1f2$ con $g1H$ y $g1f2\sim g2f2$ con $K(f2 x I)$. Por transitividad $g1f1\simeq g2f2$.

En la prueba de 6.11 el segmento $\phi$ es ir de $x$ a $x+(0,1)$. 

\end{document}

