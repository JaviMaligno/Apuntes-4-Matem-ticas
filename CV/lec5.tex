\documentclass[CV.tex]{subfiles}

\begin{document}


%\hyphenation{equi-va-len-cia}\hyphenation{pro-pie-dad}\hyphenation{res-pec-ti-va-men-te}\hyphenation{sub-es-pa-cio}

\chapter{Homotopía}

\section{Definiciones}

\begin{defi}
Dos aplicaciones $f_i:X\to Y$, $i=1,2$, continuas entre espacios topológicos se dicen \textbf{homotópicas} si existe $H:X\times [0,1]\to Y$ continua con $H(x,0)=f_1(x)$ y $H(x,1)=f_2(x)$ para todo $x\in X$. Escribiremos entonces $f_1\simeq f_2$ o $f_1\simeq_H f_2$.
\end{defi}

\begin{lemma}
Ser homotópicas es una relación de equivalencia.
\end{lemma}
\begin{proof}
En primer lugar, $f\simeq f$ mediante $H(x,t)=f(x)$. Si $f_0\simeq f_1$ mediante $H$, entonces $f_1\simeq f_0$ mediante $G(x,t)=H(x,1-t)$. Si $f_0\simeq f_1$ mediante $H$ y $f_1\simeq f_2$ mediante $G$, entonces $f_0\simeq f_2$ mediante
\[
F(x,t)=\begin{cases}
H(x,2t) & 0\leq t\leq\frac{1}{2},\\
G(x,2t-1) & \frac{1}{2}\leq t\leq 1.
\end{cases}
\]
\end{proof}

\begin{lemma}
Si $X,Y$ y $Z$ son espacios topológicos y $f_1,f_2:X\to Y$ y $g_1,g_2:Y\to Z$ son continuas con $f_1\simeq f_2$ y $g_1\simeq g_2$, entonces $g_1\circ f_1\simeq g_2\circ f_1$. 
\end{lemma}
\begin{proof}
Dada una homotopía $F$ de $f_1$ a $f_2$ y una homotopía $G$ de $g_1$ a $g_2$, definimos $H(x,t)=G(F(x,t),t)$, con la que se tiene el resultado.
\end{proof}

%lema 2: 
%
%$X\sim_H Y\sim_K Z$, entonces cogemos $g1 f1\sim g1f2$ con $g1H$ y $g1f2\sim g2f2$ con $K(f2 x I)$. Por transitividad $g1f1\simeq g2f2$.

\begin{defi}
Una aplicación continua $f:X\to Y$ se dice que es \textbf{equivalencia de homotopía} si existe una aplicación continua $g:Y\to X$ tal que $g\circ f\simeq Id_X$ y $f\circ g\simeq Id_Y$. En tal caso diremos que $g$ es \textbf{inversa homotópica} de $f$ y que $X$ e $Y$ son \textbf{homotópicamente equivalentes}.
\end{defi}

\begin{defi}
Si $X$ es homotópicamente equivalente al espacio topológico $\{\star\}$ se dice que es \textbf{contráctil}. Esto es equivalente a que $Id_X$ es homotópica a una aplicación constante.
\end{defi}

\begin{ej}
Sean $Y⊂\R^m$ y $f,g:X→Y$ aplicaciones continuas tales que $∀x∈X$, el segmento $[f(x),g(x)]⊂Y$, entonces $f≃g$. Para probarlo, basta tomar la homotopía $H(x,t)=(1-t)f(x)+tg(x)$. En particular, si $Y$ es estrellado, entonces $Y$ es contráctil.
\end{ej}

\begin{lemma}
Sean $U,V\subseteq\R^n$ abiertos euclídeos, entonces:
\begin{enumerate}
\item $\forall h:U\to V$ continua, existe una aplicación $C^\infty$-diferenciable homotópica a $h$.
\item Si $f,g:U\to V$ son aplicaciones diferenciables $f\simeq g$, entonces existe $H:U\times\mathbb R\to V$ diferenciable tal que $H(x,0)=f(x)$, $H(x,1)=g(x)$ para todo $x\in U$.
\end{enumerate}
\end{lemma}
\begin{proof}
\ref{te} (el teorema donde aparece la descomposición de la forma diferencial, que tengo que añadirle la demostración)
\end{proof}

\begin{teorema}
Si $f,g:U→V$ son aplicaciones diferenciables entre abiertos euclídeos con $f≃g$, entonces
$$f^*,g^*:Ω^*(V)→Ω^*(U)$$
son homotópicas como complejos de cadenas.
\end{teorema}
\begin{dem}
\QED
\end{dem}

Se sigue de lo anterior que
$$f^*=g^*:H^p(V)→H^p(U)\ ∀p.$$
Así, dada una aplicación continua $ϕ:U→V$ podemos encontrar $f≃Φ:U→V$ con $f$ diferenciable induciendo
$$ϕ^*=f^*:H^p(V)→H^p(U),$$
que es independiente de la elección de $f$ y que denotaremos por $ϕ^*$ (i.e., $ϕ^*:=f^*$ donde $f≃ϕ$).

\begin{teorema}\label{6.8}
Sean $U$, $V$ y $W$ abiertos euclídeos.
\begin{enumerate}
\item Si $\phi_0,\phi_1:U\to V$ son continuas con $\phi_0\simeq \phi_1$, entonces $\phi_0^*=\phi_1^*:H^p(V)\to H^p(U)$.
\item Si $U\overset{\phi}{\to}V\overset{\psi}{\to}W$ son continuas, entonces $(\psi\circ\phi)^*=\phi^*\circ\psi^*$.
\item Si $U\overset{\phi}{\to}V$ es una equivalencia de homotopía, entonces $\phi^*:H^p(V)\to H^p(U)$ es isomorfismo $\forall p$.
\end{enumerate}
\end{teorema}
\begin{dem}\
\begin{enumerate}
\item Sea $f:U\to V$ diferenciable con $f\simeq\phi_0$. Por transitividad, $f\simeq\phi_1$, de donde se sigue el apartado.
\item La fórmula se cumple para aplicaciones diferenciables, así que elegimos $f:U\to V$ y $g:V\to W$, de modo que obtenemos 
\[
(\psi\circ\phi)^*=(g\circ f)^*=f^*\circ g^*=\phi^*\circ\psi^*.
\]
\item Si $\psi:V\to U$ es su inversa homotópica, entonces $\psi\circ\phi\simeq Id_U$ y $\phi\circ\psi\simeq Id_V$, de donde se sigue el resultado aplicando el apartado anterior.
\end{enumerate}
\QED
\end{dem}

Este resultado muestra la invarianza homotópica de la topología, en particular:

\begin{coro}[Invarianza topológica de la cohomología] 
Si $h:U\to V$ es un homeomorfismo, entonces $h^*:H^p(V)\to H^p(U)$ es isomorfismo para todo $p$.
\end{coro}

\begin{coro}
Si $U\subseteq\R^n$ es un abierto contráctil, entonces $H^p(U)=0$ para todo $p>0$ y $H^0(U)=\R$. 
\end{coro}

En la proposición siguiente, se identificará $\R^n$ con el subespacio $\R^n\times\{0\}\subseteq\R^{n+1}$ y $\R\cdot 1$ denota el espacio vectorial de dimensión 1 consistente en las funciones constantes.

\begin{prop}
Si $A⊊\R^n$ es un cerrado se tiene que
$$H^p(\R^{n+1}−A)=\begin{cases}
H^{p−1}(\R^n−A) & p≥2,\\
H^0(\R^n−A)/\R⋅1 & p=1,\\
\R & p=0.
\end{cases}$$
\end{prop}
\begin{dem}
Definimos los abiertos de $\R^{n+1}=\R^n\times\R$
\begin{align*}
U_1=\R^n\times (0,\infty)\cup (\R^n-A)\cup (-1,\infty),\\
U_1=\R^n\times (\infty,0)\cup (\R^n-A)\cup (\infty, 1).
\end{align*}
Se tiene que $U_1\cup U_2=\R^{n+1}-A$ y $U_1\cap U_2=(\R^n-A)\times(-1,1)$. Sea $\phi_1:U_1\to U_1$ dado por $\phi(x)=x+(0,1)$, es decir, colocar un 1 en la $(n+1)$-ésima coordenada (el primer $U_1$ está identificado con su proyección sobre $\R^n$). Para todo $x\in U_1$, $U_1$ contiene el segmento que une $x $ con $\phi(x)$ y el que une $\phi(x)$ con un punto fijo de $\R^n\times (0,\infty)$ (por ejemplo, $(0,1)$). Esto nos da homotopías de $Id_{U_1}$ a $\phi$ y de $\phi$ a una aplicación constante. Se sigue que $U_1$ es contráctil. Análogamente, añadiendo un -1 en lugar de un 1, se obtiene que $U_2$ es contráctil. 

Sea la proyección $pr:U_1\cap U_2=(\R^n-A)\times(-1,1)\to\R^n-A$. Se define $i:\R^n-A\to U_1\cap U_2$ como $i(y)=(y,0)$. Se tiene $pr\circ i=Id_{\R^n-A}$ y $i\circ pr\simeq Id_{U_1\cap U_2}$. Del teorema \ref{6.8} concluimos que $$pr^*:H^p(\R^n-A)\to H^p(U_1\cap U_2)$$ es un isomorfismo para todo $p$. En la sucesión de Mayer-Vietoris para $p\geq 1$ obtenemos que que $$\partial^*:H^p(U_1\cap U_2)\to H^{p+1}(\R^{n+1}-A)$$ es isomorfismo. Componiendo con $pr^*$ se obtiene la primera parte del teorema.

Consideremos ahora la sucesión exacta
\[
0\to H^0(\R^{n+1}-A)\overset{I^*}{\to}H^0(U_1)\oplus H^0(U_2)\overset{J^*}{\to}H^0(U_1\cap U_2)\overset{\partial^*}{\to}H^1(\R^{n+1}-A)\to 0.
\]
Un elemento de $H^0(U_1)\oplus H^0(U_2)$ viene dado por un par de funciones constantes en $U_1$ y $U_2$ con valores $a_1$ y $a_2$. Su imagen mediante $J^*$ es la función constante $a_1-a_2$ en $U_1\cap U_2$. Esto demuestra que $\ker{\partial^*}=\Ima{J^*}=\R\cdot 1$ y obtenemos isomoformismos
\[
H^1(R^{n+1}-A)\cong H^0(U_1\cap U_2)/\R\cdot 1=H^0(\R^n-A)/\R\cdot 1.
\]
Tenemos también $\dim(\Ima{I^*})=\dim(\ker{\partial^*})=1$, por lo que $H^0(R^{n+1}-A)\cong\R$. 
\QED
\end{dem}

\end{document}

