\documentclass[cursovd_portada.tex]{subfiles}

\begin{document}


%\hyphenation{equi-va-len-cia}\hyphenation{pro-pie-dad}\hyphenation{res-pec-ti-va-men-te}\hyphenation{sub-es-pa-cio}

\chapter{Introducción}
\section{Álgebra alternada}
\begin{defi}
Sea $V$ un $\R$-e.v. Deremos que $f:\underbrace{V\times\dots\times V}_{r}\to\R$ es \textbf{multilineal} o $r$-lineal si lineal en cada una de sus coordenadas.

Una aplicación $r$-lineal $w:\V^r\to\R$ se dirá \textbf{alternada} si $w(v_1,\dots, v_r)=0$ cuando $v_i=v_j$ para algunos $i\neq j$. 
\end{defi}

El conjunto de aplicaciones multilineales es un $\R$-e.v., $Alt^r(V)=\{w:V^r\to\R: w$ es multilineal y alternado$\}$. 
\begin{ej}
$Alt^0(V)\cong\R$, $Alt^1(V)=V^*$.
\end{ej}

\begin{lemma}
Sea $V$ tal que $\dim(V)=n$ y consideramos $Alt^r(V)$ con $r>n$. Sea $w\in Alt^r(V)$. Entonces $w\equiv 0$.
\end{lemma}
\begin{proof}
Sea $B=\{e_1,\dots, e_n\}$ base de $V$. Entonces $v_i=\sum_{j=1}^n\lambda_{ij}e_j$. Entonces
$$
w(v_1,\dots, v_r)=w\left(\sum_{j=1}^n\lambda_{1j}e_j,\dots,\sum_{j=1}^n\lambda_{rj}e_j\right)=\sum\lambda_Jw(e_{j_1},\dots, e_{j_r}),$$
donde $\lambda_J=\lambda_{j_1,1}\cdots\lambda_{j_r,r}$. Como $r>n$, entonces habrá alguna repetición entre los $e_{j_1},\dots, e_{j_r}$, por lo que se tiene el resultado.
\end{proof}

El grupo de permutaciones del conjunto $\{1,\dots, r\}$ se denotará por $S(r)$. Recordemos que toda permutación se puede escribir como composición de trasposiciones $(i,j)$. Podemos definir entonces el siguiente homomorfismo:
\begin{align*}
sgn: & S(r)\to \{1,-1\}\\
& \sigma\to sign(\sigma)
\end{align*}
tal que $sgn((i,j))=-1$, que se extiende de forma natural al resto de permutaciones.
\begin{lemma}
Si $w\in Alt^r(V), \sigma\in S(r)$, entonces
$$w(v_{\sigma(1)},\dots, v_{\sigma(r)})=sgn(\sigma)w(v_1,\dots, v_r).$$
\end{lemma}
\begin{proof}
Es suficiente demostrarlo para $\sigma=(i,j)$. Fijamos $(v_1,\dots, v_r)$ y definimos 
$$w_{i,j}(v,v')=(v_1,\dots,v_{i-1},v,v_{i+1},\dots, v_{j-1},v',v_{j+1},\dots, v_r).$$ 
Se tiene que $w_{i,j}\in Alt^2(V)$, de donde $w_{i,j}(v_i+v_j,v_i+v_j)=0$. Desarrollando
$$
0=w_{i,j}(v_i,v_i)+w_{i,j}(v_i,v_j)+w_{i,j}(v_j,v_i)+w_{i,j}(v_j,v_j)=w_{i,j}(v_i,v_j)+w_{i,j}(v_j,v_i).$$
De donde se deduce el resultado.
\end{proof}
\begin{ej}
Sea $V=\R^r$ y $r$ vectores $v_i=(v_{i,1},\dots, v_{i,r})$. Entonces $w(v_1,\dots, v_r)=\det(v_{i,j})$ es alternado. 
\end{ej}

\begin{defi}\label{ext}
Se define el \textbf{producto exterior} de alternados como 
$$\land:Alt(V)\times Alt(V)\to Alt^2(V)$$
dada por $w_1\land w_2(v_1,v_2)=w_1(v_1)w_2(v_2)-w_1(v_2)w_2(v_1)$. A continuación definiremos para espacios alternados de dimensión mayor.
\end{defi}
\begin{defi}
Dentro de $S(p+q)$, llamaremos $(p+q)$-\textbf{barajas} a las permutaciones de $S(p+q)$ que verifican
\begin{gather*}
\sigma(1)<\sigma(2)<\cdots<\sigma(p)\\
\sigma(p+1)<\sigma(p+2)<\cdots<\sigma(q)
\end{gather*}
Al conjunto de tales permutaciones lo denotaremos $S(p,q)$. Una $(p,q)$-baraja queda totalmente determinada por $\{\sigma(1),\dots, \sigma(p)\}$ (o por los otros). Por tanto, $Card(S(p,q))=\binom{p+1}{p}=\binom{p+q}{q}$.
\end{defi}
\begin{defi}[Producto exterior]
Dados $w_1\in Alt^p(V),w_2\in Alt^q(V)$
$$
(w_1\land w_2)(v_1,\dots, v_{p+q})=\sum_{\sigma\in S(p,q)} sgn(\sigma)w_1(v_{\sigma(1)},\dots, v_{\sigma(p)})w_2(v_{\sigma(p+1)},\dots, v_{\sigma(p+q)}).$$
\end{defi}
Faltaría probar que $w_1\land w_2\in Alt^{p+q}(V)$, pero antes:
\begin{nota}\
\begin{enumerate}
\item Para $p=q=1$ es justamente la definición \ref{ext}. $S(1,1)=\{1,(1,2)\}$.
\item $w_1\in Alt^0(V), w_2\in Alt^1(V)\Rightarrow w_1\land w_2\in Alt^1(V)$. $S(0,1)=\{1\}$. $w_1\land w_2(v)=w_1 w_2(v)$.
\end{enumerate}
Probamos ahora el resultado. 

\begin{lemma}
Sean $w_1\in Alt^p,w_2\in Alt^{q}(V)$. Entonces, $w_1\land w_2\in Alt^{p+q}$, es decir,
\begin{enumerate}
\item $w_1\land w_2$ es $(p+q)$-lineal.
\item $w_1\land w_2$ es alternado.
\end{enumerate}
\end{lemma}
\begin{proof}\
\begin{enumerate}
\item Se tiene por definición.
\item Veamos que $w_1\land w_2(v_1,v_2,\dots, v_{p+q})=0$ si $v_1=v_2$ (para cualquier otro caso será análogo). Definimos
\begin{itemize}
\item $S_{12}=\{\sigma\in S(p,q)\mid \sigma(1)=1,\sigma(p+1)=2\}$.
\item $S_{21}=\{\sigma\in S(p,q)\mid \sigma(1)=2,\sigma(p+1)=1\}$.
\item $S_0=S(p,q)-(S_{12}\cup S_{21})$.
\end{itemize}
Si $\sigma\in S_0$, entonces o bien $w_1(v_{\sigma(1)},\dots, v_{\sigma(p)})=0$ o bien $w_2(v_{\sigma(p+1)},\dots, v_{\sigma(p+q)})=0$, pues o bien $v_{\sigma(1)}=v_{\sigma(2)}$ o bien $v_{\sigma(p+1)}=v_{\sigma(p+2)}$.
Observamos que existe una biyección $S_{12}\to S_{21}$ dada por composición por la izquierda con $\tau=(1,2)$. Por lo comentado antes, tenemos que
\begin{gather*}
w_1\land w_2(v_1,\dots, v_{p+1})=\sum_{\sigma\in S_{12}}sgn(\sigma)w_1(v_{\sigma(1)},\dots, v_{\sigma(p)})w_2(v_{\sigma(p+1)},\dots,v_{\sigma(p+q)})+\\
\sum_{\sigma\in S_{21}}sgn(\sigma)w_1(v_{\sigma(1)},\dots, v_{\sigma(p)})w_2(v_{\sigma(p+1)},\dots,v_{\sigma(p+q)})=\\
\sum_{\sigma\in S_{12}}sgn(\sigma)w_1(v_{\sigma(1)},\dots, v_{\sigma(p)})w_2(v_{\sigma(p+1)},\dots,v_{\sigma(p+q)})-\\
\sum_{\tau\sigma\in S_{12}}sgn(\sigma)w_1(v_{\tau\sigma(1)},\dots, v_{\tau\sigma(p)})w_2(v_{\tau\sigma(p+1)},\dots,v_{\tau\sigma(p+q)})=0
\end{gather*}
Pues si $\sigma\in S_{12}$, entonces $\sigma(1)=1$, $\sigma(p+1)=2$, $\tau\sigma(1)=2$, $\tau\sigma(p+1)=1$, $\tau\sigma(i)=\sigma(i)\ \forall i\neq 1,p+1$. Como $v_1=v_2$, entonces $$w_1(v_{\sigma(1)},\dots,v_{\sigma(p)})w_2(v_{\sigma(p+1)},\dots, v_{\sigma(p+1)})=w_1(v_{\tau\sigma(1)},\dots, v_{\tau\sigma(p)})w_2(v_{\tau\sigma(p+1)},\dots, v_{\tau\sigma(p+q)}).$$
\end{enumerate}
\end{proof}
\end{nota}

\end{document}