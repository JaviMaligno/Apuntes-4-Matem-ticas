	\documentclass[twoside]{article}
\usepackage{../../estilo-ejercicios}

%--------------------------------------------------------
\begin{document}

\title{Ejercicios de From Calculus to Cohomology, Capítulo 5}
\author{Javier Aguilar Martín}
\maketitle


\begin{ejercicio}{5.1}
Usando la notación del ejemplo 5.4, un punto $(x,y)\in U_1$ está unívocamente determinado por sus coordenadas polares $(r,\theta)\in (0,\infty)\times (0,2\pi)$. Sea $arg_1\in\Omega^0(U_1)$ la función $(x,y)\mapsto \theta\in(0,2\pi)$ (¿por qué es $arg_1$ diferenciable?). Definir similarmente $arg_2\in\Omega^0(U_2)$ usando coordenadas polares con $\theta\in (-\pi,\pi)$ y probar la existencia de una 1-forma cerrada $\tau\in\Omega^1(\R^2-\{0\})$ tal que
\[
\eta_{U_{\nu}}=i^*_{\nu}(\tau)=d(arg_{\nu})\quad (\nu=1,2).
\]
Probar que el homomorfismo 
\[
\partial^0:H^0(U_1\cap U_2)\to H^1(\R^2-\{0\})
\]
lleva funciones localmente constantes con valores $\{0,2\pi\}$ en los semiplanos superior e inferior respectivamente en $[\tau]$. 
\end{ejercicio}
\begin{solucion}
$arg_1(x,y)=\arctan(y/x)$, que es diferenciable en $x\neq 0$, así que $arg_1$ es diferenciable. Definimos $arg_2=arg_1-\pi$. Tenemos que 
\[
d(arg_1)=d(arg_2)=-\frac{y}{x^2+y^2}dx+\frac{x}{x^2+y^2}dy.
\]
Por lo que basta definir $\tau$ como la 1-forma obtenida. Para la última parte, sea $c=0\chi_{\R^2_+}+2\pi\chi_{\R^2_-}$. Vamos a buscar una preimagen de $c$ por $J^0$ utilizando la misma técnica que en la demostración de la sucesión de Mayer-Vietoris. Sean $p_1,p_2:U=\R^2-\{0\}\to[0,1]$ tales que $supp_U(p_{\nu})\subseteq U_{\nu}$ y $p_1+p_2=1$. Definimos
\[
c_2(x)=\begin{cases}
-c(x)p_1(x) & x\in U_1\cap U_2,\\
0 & x\in U_2-supp_U(p_1),
\end{cases}
\]
y 
\[
c_1(x)=\begin{cases}
c(x)p_2(x) & x\in U_1\cap U_2,\\
0 & x\in U_1-supp_U(p_2).
\end{cases}
\]

Se tiene entonces que $c=c_1(x)-c_2(x)$. Por tanto, $J^0(c_1,c_2)=c$. 
 Entonces, por definición
\[
\partial^0([c])=[(I^1)^{-1}d_{\Omega^0(U_1)\oplus\Omega^0(U_2)}((J^0)^{-1})(c)]=[(I^1)^{-1}d_{\Omega^0(U_1)\oplus\Omega^0(U_2)}(c_1,c_2)]=
\]
ESTO NO ME LLEVA A NINGUNA PARTE, TENDRÍA QUE VER QUE LO QUE ME DA AL HACER LA DIFERENCIA CON TAU ESTÁ EN LA IMAGEN DE J0
\end{solucion}

\newpage

\begin{ejercicio}{5.2}
Probar que la 1-forma $\tau\in\Omega^1(\R^2-\{0\})$ del ejercicio \ref{ejer:5.1} y $\Ima(z^{-1}dz)$ del ejercicio 3.6 son la misma. %\ref{3.6}
\end{ejercicio}
\begin{solucion}

Con la expresión explícita que hemos dado es claro que son la misma.

\end{solucion}
\newpage

\begin{ejercicio}{5.3}
¿Puede ser $\R^2$ cubierto por dos abiertos conexos $U,V$ tales que $U\cap V$ es disconexo? 
\end{ejercicio}
\begin{solucion}
Sí, puede. Sea $U=\{(x,y)\in\R^2\mid x^2+y^2>1\}$ y $V=\{(x,y)\in\R^2\mid -1<y<1\}$. Claramente $\R^2=U\cup V$, pero $U\cap V=\{(x,y)\in\R^2\mid x^2+y^2>1, x>0\}\cup \{(x,y)\in\R^2\mid x^2+y^2>1, x<0\}$, que es unión de dos abiertos disjuntos y por tanto, disconexo. 
\end{solucion}
\newpage

\begin{ejercicio}{5.4}[Propiedad Phragmen-Brouwer de $\R^n$]  Supongamos que $p\neq q$ en $\R^n$. Un subconjunto cerrado $A\subseteq\R^n$ se dice que \emph{separa} $p$ y $q$ si $p$ y $q$ pertenecen a distintas componentes conexas de $\R^n-A$. 

Sean $A$ y $B$ dos subcojuntos cerrados disjuntos de $\R^n$. Dados dos puntos distintos $p$ y $q$ en $\R^n-(A\cup B)$. Probar que si ni $A$ ni $B$ separan $p$ y $q$, entonces $A\cup B$ no separa $p$ y $q$. (Aplicar Mayer-Vietoris a $U_1=\R^n-A$, $U_2=\R^n-B$).

\end{ejercicio}
\begin{solucion}
No vamos a seguir exactamente la indicación aunque se va a fundamentar en lo mismo.

Recordemos que en $\R^n$, abierto y conexo implica conexo por caminos, por lo que las componentes conexas lo serán por caminos. Sea $V$ un entorno del segmento $[p,q]$ en $\R^n$ (podemos suponer que es contráctil). Como ni $A$ ni $B$ separan a $p$ y $q$, sabemos que podemos escoger $V$ de forma que $V-A$ y $V-B$ son conexos por caminos. 

Así que, en Mayer-Vietoris para $p=0$
\[
\begin{tikzcd}
0\arrow[r] & H^0(V)\arrow[r,"I^0"] & H^0(V-A)\oplus H^0(V-B)\arrow[r, "J^0"] & H^0(V-(A\cup B))\arrow[r,"\partial^*"] & H^1(V).
\end{tikzcd}
\]
Como $H^1(V)=0$, $H^0(V)=H^0(V-A)=H^0(V-B)=\R$,
\[
\begin{tikzcd}
0\arrow[r] & \R\arrow[r,"I^0"] & \R\oplus \R\arrow[r, "J^0"] & H^0(V-(A\cup B))\arrow[r]& 0
\end{tikzcd}
\]
de donde $\R\oplus\R\cong \R\oplus  H^0(V-(A\cup B))$, por lo que $H^0(V-(A\cup B))\cong\R$. Eso significa que $V-(A\cup B)$ es conexo por caminos, por lo que $A\cup B$ no separa $p$ y $q$. 

{\bf\large ALTERNATIVA USANDO TAL CUAL LA INDICACIÓN}

Recordemos que todo abierto de $\R^n$ puede ser expresado como unión numerable de componentes conexas (basta tomar bolas con centro racional). Sean $U_1=\R^n-A$ y $U_2=\R^n-B$. Entonces $U_1\cap U_2=\R^n-(A\cup B)$ y $U_1\cup U_2=\R^n$. Supongamos que $H^0(U_1)=\oplus_{i\in I}\R$ y $H^0(U_2)=\oplus_{j\in J}\R$. En Mayer-Vietoris, para $p=0$
\[
\begin{tikzcd}
0\arrow[r] & H^0(\R^n)\arrow[r,"I^0"] & H^0(U_1)\oplus H^0(U_2)\arrow[r, "J^0"] & H^0(U_1\cap U_2)\arrow[r,"\partial^*"] & H^1(\R^n)
\end{tikzcd}
\]
Como $H^0(\R^n)=\R$ y $H^1(\R^n)=0$, tenemos
\[
\begin{tikzcd}
0\arrow[r] & \R \arrow[r,"I^0"] & H^0(U_1)\oplus H^0(U_2)\arrow[r, "J^0"] & H^0(U_1\cap U_2)\arrow[r] & 0
\end{tikzcd}
\]
de donde $ H^0(U_1)\oplus H^0(U_2)=\R\oplus H^0(U_1\cap U_2)$, es decir,
\[
\oplus_{i\in I}\R\bigoplus \oplus_{j\in J}\R =\R\oplus H^0(U_1\cap U_2)
\]
\end{solucion}



\end{document}