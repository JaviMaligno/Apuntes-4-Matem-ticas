	\documentclass[twoside]{article}
\usepackage{../../estilo-ejercicios}

%--------------------------------------------------------
\begin{document}

\title{Cálculo en Variedades, Diciembre de 2015}
\author{Javier Aguilar Martín}
\maketitle


\begin{ejercicio}{1}
Demuestra que si $V$ es un $\R$-espacio vectorial con $\dim(v)=3$, entonces para todo $w\in Alt^q(V), q>0$ se verifica que $w\land w=0$.
\end{ejercicio}
\begin{solucion}
Solo es necesario considerar el caso $q=1$, pues para $q\geq 2$, $w\land w\in Alt^{2q}(V)$ con $2q\geq 4>3$, por lo que se tiene trivialmente el resultado. En el caso $q=1$ usamos que $w\land w=(-1)^{q^2}w\land w=-w\land w$, por lo que $w\land w=0$.


\end{solucion}

\newpage

\begin{ejercicio}{2}
Sea $M$ una $n$-variedad diferenciable, compacta y orientable, $\theta\in\Omega^{n-1}(M)$. Demuestra que $d\theta$ debe anularse en algún punto de $M$.
\end{ejercicio}
\begin{solucion}

\end{solucion}
\newpage

\begin{ejercicio}{3}
Sea $M$ una $n$-variedad diferenciable conexa y orientable. Considera la aplicación identidad sobre $M$:
\[
1_M:M\to -M
\]
entre las dos orientaciones posibles $M$ y $-M$. Demuestra que entonces $1_M$ invierta la orientación.
\end{ejercicio}
\begin{solucion}

\end{solucion}
\newpage

\begin{ejercicio}{4}

\end{ejercicio}
\begin{solucion}

\end{solucion}




\end{document}