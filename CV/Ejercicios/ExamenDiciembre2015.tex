	\documentclass[twoside]{article}
\usepackage{../../estilo-ejercicios}

%--------------------------------------------------------
\begin{document}

\title{Cálculo en Variedades, Diciembre de 2015}
\author{Javier Aguilar Martín}
\maketitle


\begin{ejercicio}{1}
Demuestra que si $V$ es un $\R$-espacio vectorial con $\dim(v)=3$, entonces para todo $w\in Alt^q(V), q>0$ se verifica que $w\land w=0$.
\end{ejercicio}
\begin{solucion}
Solo es necesario considerar el caso $q=1$, pues para $q\geq 2$, $w\land w\in Alt^{2q}(V)$ con $2q\geq 4>3$, por lo que se tiene trivialmente el resultado. En el caso $q=1$ usamos que $w\land w=(-1)^{q^2}w\land w=-w\land w$, por lo que $w\land w=0$.


\end{solucion}

\newpage

\begin{ejercicio}{2}
Sea $M$ una $n$-variedad diferenciable, compacta y orientable, $\theta\in\Omega^{n-1}(M)$. Demuestra que $d\theta$ debe anularse en algún punto de $M$.
\end{ejercicio}
\begin{solucion}

\end{solucion}
\newpage

\begin{ejercicio}{3}
Sea $M$ una $n$-variedad diferenciable conexa y orientable. Considera la aplicación identidad sobre $M$:
\[
1_M:M\to -M
\]
entre las dos orientaciones posibles $M$ y $-M$. Demuestra que entonces $1_M$ invierta la orientación.
\end{ejercicio}
\begin{solucion}
Sean $w\in\Omega^n(M)$ la forma de orientación que determina la orientación de $M$ (y por tanto $-w$ la de $-M$). Por funtorialidad de la estrella, $1_M^*=Id:\Omega^*(-M)\to\Omega^*(M)$, por lo que $1_M^*(-w)=-w$, así que $1_M$ revierte la orientación por definición.
\end{solucion}
\newpage

\begin{ejercicio}{4}
Sea $M$ una $n$-variedad diferenciable orientable y compacta. Considera el siguiente morfismo
\[
\begin{tikzcd}
\mathcal{P}: & \Omega^k(M)\arrow[r]    & (\Omega^{n-k}(M))^*  &                  & \\
             & \alpha\arrow[r, mapsto] & \mathcal{P}(\alpha): & \Omega^{n-k}(M)\arrow[r]  & \R\\
             &		                   &					  &  \beta\arrow[r]  &\mathcal{P}(\alpha)(\beta)=\int_M \alpha\land\beta
\end{tikzcd}
\]
\begin{enumerate}[a)]
\item Demuestra que $\mathcal{P}$ está bien definido y que además describe un homomorfismo de $\R$-espacios vectoriales.
\item Diremos que las $k$-formas $\delta\equiv\gamma$ son equivalentes si $\delta-\gamma=d\phi$ para alguna $(k-1)$-forma $\phi$. Demuestra que si $\alpha_1\equiv \alpha_2$, $\beta_1\equiv\beta_2$ y son formas exactas, entonces $\mathcal{P}(\alpha_1)(\beta_1)=\mathcal{P}(\alpha_2)(\beta_2)$. 
\item Utilizando b), demuestra que $\mathbb{P}$ induce un homomorfismo de $\R$-espacios vectoriales $\overline{P}:H^k(M)\to (H^{n-k}(M))^*$.
\end{enumerate}
\begin{nota}
Recuerda que si $V$ es un $R$-espacio vectorial, también se denota por $(V)^*=Hom_\R(V,\R)=Alt^1(V)$ al conocido como espacio dual de $V$.
\end{nota}
\end{ejercicio}
\begin{solucion}

\end{solucion}




\end{document}