	\documentclass[twoside]{article}
\usepackage{../../estilo-ejercicios}

%--------------------------------------------------------
\begin{document}

\title{Cálculo en Variedades, Diciembre de 2015}
\author{Javier Aguilar Martín}
\maketitle


\begin{ejercicio}{1}
Demuestra que si $V$ es un $\R$-espacio vectorial con $\dim(v)=3$, entonces para todo $w\in Alt^q(V), q>0$ se verifica que $w\land w=0$.
\end{ejercicio}
\begin{solucion}
Solo es necesario considerar el caso $q=1$, pues para $q\geq 2$, $w\land w\in Alt^{2q}(V)$ con $2q\geq 4>3$, por lo que se tiene trivialmente el resultado. En el caso $q=1$ usamos que $w\land w=(-1)^{q^2}w\land w=-w\land w$, por lo que $w\land w=0$.


\end{solucion}

\newpage

\begin{ejercicio}{2}

\end{ejercicio}
\begin{solucion}

\end{solucion}
\newpage

\begin{ejercicio}{3}

\end{ejercicio}
\begin{solucion}

\end{solucion}
\newpage

\begin{ejercicio}{4}

\end{ejercicio}
\begin{solucion}

\end{solucion}




\end{document}