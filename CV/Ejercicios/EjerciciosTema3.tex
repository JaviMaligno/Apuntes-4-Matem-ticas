	\documentclass[twoside]{article}
\usepackage{../../estilo-ejercicios}

%--------------------------------------------------------
\begin{document}

\title{Ejercicios de From Calculus to Cohomology, Capítulo 4}
\author{Javier Aguilar Martín}
\maketitle


\begin{ejercicio}{4.1}
Consideremos el diagrama conmutativo de espacios vectoriales y aplicaciones lineales en el que las filas son exactas
\[
\begin{tikzcd}
A_1\arrow[d,"f_1"]\arrow[r, "\alpha_1"] & A_2\arrow[d,"f_2"]\arrow[r, "\alpha_2"] & A_3\arrow[d,"f_3"]\arrow[r, "\alpha_3"] & A_4\arrow[d,"f_4"]\arrow[r, "\alpha_4"] & A_5\arrow[d,"f_5"]\\
B_1\arrow[r, "\beta_1"] & B_2\arrow[r, "\beta_2"] & B_3\arrow[r, "\beta_3"] & B_4\arrow[r, "\beta_4"] & B_5
\end{tikzcd}
\]
Supongamos que $f_4$ es inyectiva, $f_1$ es sobreyectiva y $f_2$ es inyectiva. Probar que $f_3$ es inyectiva. Suponer que $f_2$ es sobreyectiva, $f_4$ es sobreyectiva y $f_5$ es inyectiva. Probar que $f_3$ es sobreyectiva. 
\end{ejercicio}
\begin{solucion}
\underline{Inyectividad}

Sea $a_3\in A_3$ tal que $f_3(a_3)=0\in B_3$. Por conmutatividad del diagrama, $f_4(\alpha_3(a_3))=\beta_3(f_3(a_3))=\beta_3(0)=0$. Por inyectividad de $f_4$ se sigue que $\alpha_3(a_3)=0$, es decir, $a_3\in\ker{\alpha_3}=\Ima{\alpha_2}$, luego existe $a_2\in A_2$ con $\alpha_2(a_2)=a_3$. Denotamos $b_2=f_2(a_2)$. Por conmutatividad, $\beta_2(b_2)=f_3(\alpha_2(a_2))=f_3(a_3)=0$. Por tanto, $b_2\in\ker{\beta_2}=\Ima{\beta_1}$, así que existe $b_1\in B_1$ con $\beta_1(b_1)=b_2$. Por la sobreyectividad de $f_1$, existe $a_1\in A_1$ tal que $f_1(a_1)=b_1$, luego aplicando conmutatividad, $f_2(\alpha_1(a_1))=\beta_1(f_1(a_1))=\beta_1(b_1)=b_2=f_2(a_2)$. Esto quiere decir que $\alpha_1(a_1)-a_2\in\ker{f_2}$, pero como $f_2$ es inyectiva llegamos a que $\alpha_1(a_1)=a_2$. Ahora, por exactitud, $0=\alpha_2(\alpha_1(a_1))=\alpha_2(a_2)=a_3$, con lo que se llega al resultado.

\underline{Sobreyectividad}

Sea $b_3\in B_3$. Denotamos $b_4=\beta_3(b_3)\in B_4$. Por sobreyectividad de $f_4$, existe $a_4\in A_4$ tal que $f_4(a_4)=b_4$. Aplicando la conmutatividad del diagrama, $f_5(\alpha_4(a_4))=\beta_4(b_4)=\beta_4(\beta_3(b_4))=0$ por exactitud. Así que de la inyectividad de $f_5$ deducimos que $\alpha_4(a_4)=0$, o lo que es lo mismo, $a_4\in\ker{\alpha_4}=\Ima{\alpha_3}$, de modo que existe $a_3\in A_3$ con $\alpha_3(a_3)=a_4$. Usando la conmutatividad del diagrama, $\beta_3(f_3(a_3))=f_4(\alpha_3(a_3))=f_4(a_4)=b_4=\beta_3(b_3)$. Por lo tanto, $f_3(a_3)-b_3\in\ker{\beta_3}=\Ima{\beta_2}$, luego existe $b_2\in B_2$ tal que $\beta_2(b_2)=f_3(a_3)-b_3$. Por sobreyectividad de $f_2$, se tiene que existe $a_2\in A_2$ con $f_2(a_2)=b_2$, de donde se sigue por conmutatividad que $f_3(\alpha_2(a_2))=\beta_2(f_2(a_2))=\beta_2(b_2)=f_3(a_3)-b_3$. Concluimos pues, que $b_3=f_3(\alpha_2(a_2)-a_3)$.
\end{solucion}

\newpage

\begin{ejercicio}{4.2}
Consideremos el siguiente diagrama conmutativo
\[
\begin{tikzcd}
0\arrow[r] & A_1\arrow[d,"f_1"]\arrow[r, "\alpha_1"] & A_2\arrow[d,"f_2"]\arrow[r, "\alpha_2"] & A_3\arrow[d,"f_3"]\arrow[r] & 0\\
0\arrow[r] & B_1\arrow[r, "\beta_1"] & B_2\arrow[r, "\beta_2"] & B_3\arrow[r] & 0
\end{tikzcd}
\]
donde las filas son sucesiones exactas. Probar que existe una sucesión exacta
\[
0\to \ker{f_1}\to \ker{f_2}\to \ker{f_3}\to \coker{f_1}\to \coker{f_2}\to \coker{f_3}\to 0.
\]
(Pista: utilizar la sucesión exacta larga de cohomología).
\end{ejercicio}
\begin{solucion}
Vamos a ver primero la exactitud de
\[
0\to \ker{f_1}\to \ker{f_2}\to \ker{f_3}.
\]
Denotamos $a_i=\alpha_i|_{\ker{f_i}}$. Estas aplicaciones están bien definidas, pues si $n\in\ker{f_i}$, entonces $f_{i+1}a_i(n)=\beta_if_i(n)=0$, así que $a_i(n)\in\ker{f_{i+1}}$. Claramente $a_1$ es inyectiva. Claramente, $\Ima{a_1}\subseteq\ker{a_2}$. Además, si $n\in\ker{f_2}$ con $a_2(n)=0$. En particular, $n\in\ker{\alpha_2}$, luego $n\in\Ima{\alpha_1}$. Así que $n=\alpha_1(m)$ para $m\in A_1$. Tenemos que probar que de hecho $m\in\ker{f_1}$. Usando la conmutatividad del diagrama, $\beta_1(f_1(m))=f_2(\alpha_1(m))=f_2(n)=0$. Como $\beta_1$ es inyectiva, $f_1(m)=0$, así que $m\in\ker{f_1}$.

Probamos ahora la exactitud de
\[
\coker{f_1}\to \coker{f_2}\to \coker{f_3}\to 0.
\]
%\[
%B_1/\Ima{f_1}\to B_2/\Ima{f_2}\to B_3/\Ima{f_3}\to 0
%\]
Definimos $b_i(n+\Ima{f_i})=\beta_i(n)+\Ima{f_{i+1}}$. Veamos que están bien definidas. Sean $n+\Ima{f_i}=n'+\Ima{f_i}$, entonces $n-n'\in\Ima{f_i}$, es decir, existe $a\in A_i$ con $f_i(a)=n-n'$. Por tanto, $\beta_i(n)-\beta(n')=\beta_if_i(a)=f_{i+1}\alpha_i(a)$, así que $\beta_i(n)-\beta(n')\in\Ima{f_{i+1}}$. Veamos que $b_2$ es sobreyectiva. Sea $m+\Ima{f_3}\in\coker{f_3}$. Como $\beta_2$ es sobreyectiva, existe $n\in B_2$ con $\beta_2(n)=m$. Entonces, $b_2(n+\Ima{f_2})=\beta_2(n)+\Ima{f_3}=m+\Ima{f_3}$. Por otro lado, claramente $b_2b_1=0$, por lo que $\Ima{b_1}\subseteq\ker{b_2}$. Recíprocamente, si $n+\Ima{f_2}\in\ker{b_2}$, es decir, $\beta_2(n)\in\Ima{f_3}$, por lo que sea $m\in A_3$ con $f_3(m)=\beta_2(n)$. Como $\alpha_2$ es sobreyectiva, sea $p\in A_2$ con $\alpha_2(p)=m$. Tenemos por conmutatividad, $\beta_2(n)=f_3\alpha_2(p)=\beta_2f_2(p)$, con lo que $n-f_2(p)\in\ker{\beta_2}=\Ima{\beta_1}$. Sea pues, $r\in B_1$ con $\beta_1(r)=n-f_2(p)$. Consideremos la clase $r+\Ima{f_1}$. Entonces, $$b_1(r+\Ima{f_1})=\beta_1(r)+\Ima{f_2}=n-f_2(p)+\Ima{f_2}=n+\Ima{f_2}.$$

Por último, probamos la exactitud de 
\[
\ker{f_2}\to \ker{f_3}\to \coker{f_1}\to \coker{f_2}.
\]
Como ya tenemos $a_2$ y $b_1$ definidas, tenemos que definir $\delta:\ker{f_3}\to\coker{f_1}$. Dado $c\in\ker{f_3}$, como $\alpha_2$ es sobreyectiva, existe $a\in A_2$ con $\alpha_2(a)=c$. Por conmutatividad, $\beta_2(f_2(a_2))=f_3(c)=0$, luego $f_2(a)\in\ker{\beta_2}=\Ima{\beta_1}$. Así que sea $b\in B_1$ con $\beta_1(b)=f_2(a)$ ($b$ es único por inyectividad). Entonces, definimos $\delta(c)=b+\Ima{f_1}$, es decir, $$\delta(c)=\beta_1^{-1}(f_2(\alpha_2^{-1}(c)))+\Ima{f_1}.$$ Tenemos que ver que está bien definida. Sean $\alpha_2(a)=\alpha_2(a')=c$. Existe un único $b'\in B_1$ con $\beta_1(b')=f_2(a')$ de la misma forma que se calculó en la definición original. Tenemos que $a-a'\in\ker{\alpha_2}=\Ima{\alpha_1}$, así que $a-a'=\alpha_1(z)$ para algún $z\in A_1$. Así que $\beta_1(f_1(z))=f_2(\alpha_1(z))=f_2(a-a')=\beta_1(b-b')$. Por inyectividad de $\beta_1$, $f_1(z)=b-b'\in\Ima{f_1}.$

Por la definición de $\delta$ es claro que $\Ima{a_2}\subseteq\ker{\delta}$. Recíprocamente, sea $p\in\ker{\delta}$ estando $p\in\ker{f_3}$. Tenemos que $p=\alpha_2(n)$ y $f_2(n)=\beta_1(m)$ con $m\in\Ima{f_1}$. Sea $m=f_1(x)$ con $x\in A_1$. Entonces $f_2\alpha_1(x)=\beta_1 f_1(x)=f_2(n)$.  Entonces, $n-\alpha_1(x)\in\ker{f_2}$. Además, $a_2(n-\alpha_1(x))=\alpha_2(n)-\alpha_2\alpha_1(x)=\alpha_2(n)=p$, así que $p\in\Ima{a_2}$.

Por último, sea $p\in\ker{f_3}$, entonces $p=\alpha_2(b)$ y $f_2(a)=\beta_1(b)$, así que $\delta(p)=b+\Ima{f_1}$. Por tanto. $b_1(\delta(p))=\beta_1(b)+\Ima{f_2}=0$. Recíprocamente, supongamos que $b+\Ima{f_1}\in\ker{b_1}$. Entonces, $\beta_1(b)\in\Ima{f_2}$, digamos $\beta_1(b)=f_2(a)$ para $a\in A_2$. Sea $p=\alpha_2(a)$. Entonces, $f_3(\alpha_2(a))=\beta_2(f_2(a))=\beta_2(\beta_1(a))=0$. Por tanto, $p\in\ker{f_3}$, y $\delta(p)=b+\Ima{f_1}$.

\vspace{0.5cm}

{\bf\large ALTERNATIVA}

Sean $C_1=\{0\to A_1\to B_1\to 0\}$, $C_2=\{0\to A_2\to B_2\to 0\}$ y $C_3=\{0\to A_3\to B_3\to 0\}$ complejos de cadenas inducidos por el diagrama. Entonces entonces podemos ver el diagrama como una sucesión exacta corta de complejos de cadenas
\[
0\to C_1\to C_2\to C_3\to 0,
\]
que tiene asociada
\[
0\to H^0(C_1)\to H^0(C_2)\to H^0(C_3)\overset{\partial^*}{\to} H^1(C_1)\to H^1(C_2)\to H^1(C_3)\to 0,
\]
que es igual a 
\[
0\to \ker{f_1} \to\ker{f_2} \to \ker{f_3}\to\coker{f_1}\to\coker{f_2}\to\coker{f_3}\to 0.
\]


\end{solucion}
\newpage

\begin{ejercicio}{4.3}
En el diagrama conmutativo
\[
\begin{tikzcd}
& 0\arrow[d] & 0\arrow[d] & 0\arrow[d] & 0\arrow[d] \\
0\arrow[r] & A^{0,0}\arrow[r]\arrow[d] & A^{1,0}\arrow[r]\arrow[d]& A^{2,0}\arrow[r]\arrow[d]  & A^{3,0}\arrow[r]\arrow[d] &\cdots\\
0\arrow[r] & A^{0,1}\arrow[r]\arrow[d] & A^{1,1}\arrow[r]\arrow[d]& A^{2,1}\arrow[r]\arrow[d]  & A^{3,1}\arrow[r]\arrow[d] &\cdots\\
0\arrow[r] & A^{0,2}\arrow[r]\arrow[d] & A^{1,2}\arrow[r]\arrow[d]& A^{2,2}\arrow[r]\arrow[d]  & A^{3,2}\arrow[r]\arrow[d] &\cdots\\
0\arrow[r] & A^{0,3}\arrow[r]\arrow[d] & A^{1,3}\arrow[r]\arrow[d]& A^{2,3}\arrow[r]\arrow[d]  & A^{3,3}\arrow[r]\arrow[d] &\cdots\\
& \vdots & \vdots & \vdots & \vdots
\end{tikzcd}
\]
la horizontal $(A^{*,q})$ y la vertical $(A^{p,*})$ son complejos de cadenas donde $A^{p,q}=0$ si o bien $p<0$ o bien $q<0$. Supongamos que
\begin{gather*}
H^p(A^{*,q})=0\text{ para }q\neq 0\text{ y todo }p,\\
H^q(A^{p,*})=0\text{ para }p\neq 0\text{ y todo }q.
\end{gather*}
Construir isomorfismos $H^p(A^{*,0})\to H^p(A^{0,*})$ para todo $p$.
\end{ejercicio}
\begin{solucion}
De las hipótesis tenemos que para todo $p$ y para todo $q\neq 0$
\[
H^p(A^{*,q})=\frac{\ker(A^{p,q}\to A^{p+1,q})}{\Ima(A^{p-1}\to A^{p,q})}=0,
\]
de donde $\ker(A^{p,q}\to A^{p+1,q})=\Ima(A^{p-1,q}\to A^{p,q})$. Análogamente, para todo $q$ y para todo $p\neq 0$, $\ker(A^{p,q}\to A^{p,q+1})=\Ima(A^{p,q-1}\to A^{p,q})$. De esta forma, tenemos exactitud en todos los complejos salvo en $(A^{0,*})$ y $(A^{*,0})$. En particular, $A^{p,0}\to A^{p,1}$ y $A^{0,q}\to A^{0,q+1}$ son inyectivas para todo $p,q\neq 0$. Veamos que $H^0(A^{*,0})= H^0(A^{0,*})$. Por definición, 
\begin{align*}
& H^0(A^{*,0})=\frac{\ker(A^{0,0}\to A^{1,0})}{\Ima(0\to A^{0,0})}=\ker(A^{0,0}\to A^{1,0}),\\
& H^0(A^{0,*})= \frac{\ker(A^{0,0}\to A^{0,1})}{\Ima(0\to A^{0,0})}=\ker(A^{0,0}\to A^{0,1}).
\end{align*}
Dado $c\in \ker(A^{0,0}\to A^{1,0})$, utilizando la conmutatividad del diagrama podemos llevarlo a $A^{1,1}$ por dos caminos, ambos llegando al 0. La inyectividad de $A^{0,1}\to A^{1,1}$ nos garantiza entonces que $c\in \ker(A^{0,0}\to A^{0,1})$ y análogamente se tiene la otra inclusión. 

Vayamos ahora al $H^1$. Tenemos
$$
H^1(A^{*,0})=\frac{\ker(A^{1,0}\to A^{2,0})}{\Ima(A^{0,0}\to A^{1,0})}
$$

Sea entonces $[a_{1,0}]\in H^1(A^{*,0})$. Utilizando la conmutatividad del diagrama y la inyectividad de $A^{1,0}\to A^{1,1}$, tenemos una única imagen de $a_{1,0}$: $a_{1,1}\in \ker(A^{1,1}\to A^{2,1})=\Ima(A^{0,1}\to A^{1,1})$. Por inyectividad de esta última aplicación existe un único $a_{0,1}\in A^{0,1}$ que definiremos como la imagen de $a_{1,0}$ en $H^1(A^{0,*})$. En primer, lugar veamos que $a_{0,1}$ define un ciclo. Basta enviarlo a $A^{1,2}$ utilizando conmutatividad e inyectividad, pues la imagen de $a_{1,1}$ en $A^{1,2}$ es cero pues $a_{1,1}$ era un borde, así que efectivamente $a_{0,1}\in\ker(A^{1,1}\to A^{0,2})$. Tenemos que ver entonces que está bien definida. Vamos a añadir algo de notación para facilitarlo. Llamamos $g:A^{0,0}\to A^{1,0}$, $f:A^{1,0}\to A^{1,1}$, $g'=A^{0,0}\to A^{0,1}$, $f':A^{0,1}\to A^{1,1}$. Tenemos entonces que $a_{0,1}=(f')^{-1}(f(a_{1,0}))$. Sean entonces $a,a'$ tales que $a-a'\in \Ima(g)$, es decir, existe $\alpha$ con $g(\alpha)=a-a'$. Por conmutatividad, $f(a-a')=f(g(\alpha))=g'(g'(\alpha))$, luego $(f')^{-1}(f(a-a'))=g'(\alpha)$, de modo que  $(f')^{-1}(f(a-a'))=g'(\alpha)\in\Ima(g')$, así que las imágenes de $a$ y $a'$ definen la misma clase de cohomología. En la dirección opuesta se define análogamente, y se comprueba fácilmente que las definiciones son inversas la una de la otra, pues la expresión de la inversa será $f^{-1}f'$. En definitiva, hemos establecido $ H^1(A^{*,0})\cong  H^1(A^{0,*})$.

Para el resto de casos es similar pero con más pasos intermedios usando que aplicar borde dos veces es trivial, así que se deja como ejercicio para quienes quieran malgastar su vida chasing diagrams.
\end{solucion}
\newpage

\begin{ejercicio}{4.4}
Sea $0\to A^0\overset{d_0}{\to}A^1\overset{d_1}{\to}\cdots\overset{d_{n-1}}{\to}A^n\to 0$ un complejo de cadenas y supongamos que $\dim A^i<\infty$. La \emph{característica de Euler} está definida como
\[
\chi(A^*)=\sum_{i=0}^n (-1)^i\dim A^i.
\]
Probar que $\chi(A^*)=0$ si $A^*$ es exacta. Probar que la sucesión 
\[
0\to H^i(A^*)\to A^i/\Ima d^{i-1}\overset{d^i}{\to}\Ima d^i\to 0
\]
es exacta y concluir que
\[
\dim A^i-\dim\Ima d^{i-1}=\dim H^i(A^*)+\dim\Ima d^i.
\]
Probar que $\chi(A^*)=\sum_{i=0}^n\dim H^i(A^*)$.
\end{ejercicio}
\begin{solucion}
Recordemos que si $A^*$ es exacta, a partir de $A^{i-1}\overset{d^{i-1}}{\to} A^i\overset{d^i}{\to} A^{i+1}$ podemos obtener para todo $i$ sucesiones exactas $\ker{d^i}=\Ima{d^{i-1}}\to A^i\to \Ima{d^i}$. Para $i=0$ tendríamos $A^0=\Ima{d^0}$ y para $i=n$ tenemos $A^n=\Ima{d^{n-1}}$. Como son sucesiones exactas de espacios vectoriales, escinden. Así que $A^i\cong \Ima{d^{i-1}}\oplus\Ima{d^i}$. Entonces, sustituyendo,
\[
\chi(A^*)=\dim(\Ima{d^0})-(\dim(\Ima{d^0}+\dim(\Ima{d^1}))+\cdots +(-1)^{n-1}(\dim(\Ima{d^{n-2}})+
\]
\[
+\dim(\Ima{d^{n-1}}))+(-1)^n\dim(\Ima{d^{n-1}})=0
\]

Observemos ahora el siguiente diagrama
\[
\begin{tikzcd}
 & & & 0 &  \\
 0\arrow[r] & \ker{d^i}\arrow[d]\arrow[r] & A^i\arrow[d]\arrow[r] & \Ima{d^{i-1}}\arrow[r]\arrow[u] & 0\\
0 \arrow[r] &  H^i(A^*)\arrow[r] & A^i/ \Ima{d^{i-1}}\arrow[ur] & & 
\end{tikzcd}
\]

La fila completa sabemos que es exacta.  Además, $^i(A^*)\to A^i/ \Ima{d^{i-1}}$ es inyectiva pues es la proyección al cociente de una aplicación inyectiva. También se puede ver porque si $[a]=[0]\in  A^i/ \Ima{d^{i-1}}$, entonces $a\in  \Ima{d^{i-1}}$, luego también representa la clase del 0 en la cohomología. Por otra parte, tanto $A^i\to \Ima{d^{i-1}}$ como $A^i\to A^i/ \Ima{d^{i-1}}$ son sobreyectivas, así que por conmutatividad del diagrama, necesariamente $ A^i/ \Ima{d^{i-1}}\to \Ima{d^{i-1}}$ es sobreyectiva. Además se tiene la igualdad entre núcleo e imagen trivialmente, con lo que tenemos la exactitud buscada. Entonces, como la sucesión escinde, tomando dimensión $\dim(A^i)-\dim(\Ima{d^{i-1}})=\dim(A^i/\Ima{d^{i-1}})=\dim(H^i(A^ *))+\dim(\Ima{d^i})$. Entonces, sustituyendo 
\[
\chi(A^*)=\sum_{i=0}^n(-1)^i\dim(A^i)=\sum_{i=0}^n(-1)^i(\dim(\Ima{d^{i-1}})+\dim(H^i(A^*))+\dim(\Ima{d^i}))=\sum_{i=0}^n\dim(H^i(A^*)).
\]
\end{solucion}

\newpage

\begin{ejercicio}{4.5}
Asociar a dos aplicaciones lineales componibles 
\[
f:V_1\to V_2,\quad g:V_2\to V_3
\]
una sucesión exacta
\[
0\to \ker{f}\to\ker{g\circ f}\to\ker{g}\to\coker{f}\to\coker{g\circ f}\to\coker{g}\to 0.
\]
\end{ejercicio}
\begin{solucion}
Sea
\[
\begin{tikzcd}
& 0\arrow[d] & 0 \arrow[d] & 0\arrow[d]&\\
0\arrow[r] & V_1\arrow[d,"f"]\arrow[r,"1\oplus f"] & V_1\oplus V_2\arrow[r, "f-1"] \arrow[d, "(g\circ f)\oplus 1"]& V_2 \arrow[d,"g"]\arrow[r] & 0\\
0\arrow[r] &V_2\arrow[d]\arrow[r, "g\oplus 1"]& V_3\oplus V_2\arrow[d] \arrow[r, "1-g"]& V_3\arrow[d]\arrow[r] & 0\\
& 0 & 0 & 0 &
\end{tikzcd}
\]

Se comprueba fácilmente que los cuadrados conmutan y que las filas son exactas (incluyendo la inyecticidad de la izquierda y la sobreyectividad de la derecha). De la sucesión exacta larga en cohomología se obtiene el resultado.
\end{solucion}

\end{document}