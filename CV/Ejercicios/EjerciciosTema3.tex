	\documentclass[twoside]{article}
\usepackage{../../estilo-ejercicios}

%--------------------------------------------------------
\begin{document}

\title{Ejercicios de From Calculus to Cohomology, Capítulo 4}
\author{Javier Aguilar Martín}
\maketitle


\begin{ejercicio}{4.1}
Consideremos el diagrama conmutativo de espacios vectoriales y aplicaciones lineales en el que las filas son exactas
\[
\begin{tikzcd}
A_1\arrow[d,"f_1"]\arrow[r, "\alpha_1"] & A_2\arrow[d,"f_2"]\arrow[r, "\alpha_2"] & A_3\arrow[d,"f_3"]\arrow[r, "\alpha_3"] & A_4\arrow[d,"f_4"]\arrow[r, "\alpha_4"] & A_5\arrow[d,"f_5"]\\
B_1\arrow[r, "\beta_1"] & B_2\arrow[r, "\beta_2"] & B_3\arrow[r, "\beta_3"] & B_4\arrow[r, "\beta_4"] & B_5
\end{tikzcd}
\]
Supongamos que $f_4$ es inyectiva, $f_1$ es sobreyectiva y $f_2$ es inyectiva. Probar que $f_3$ es inyectiva. Suponer que $f_2$ es sobreyectiva, $f_4$ es sobreyectiva y $f_5$ es inyectiva. Probar que $f_3$ es sobreyectiva. 
\end{ejercicio}
\begin{solucion}
\underline{Inyectividad}

Sea $a_3\in A_3$ tal que $f_3(a_3)=0\in B_3$. Por conmutatividad del diagrama, $f_4(\alpha_3(a_3))=\beta_3(f_3(a_3))=\beta_3(0)=0$. Por inyectividad de $f_4$ se sigue que $\alpha_3(a_3)=0$, es decir, $a_3\in\ker{\alpha_3}=\Ima{\alpha_2}$, luego existe $a_2\in A_2$ con $\alpha_2(a_2)=a_3$. Denotamos $b_2=f_2(a_2)$. Por conmutatividad, $\beta_2(b_2)=f_3(\alpha_2(a_2))=f_3(a_3)=0$. Por tanto, $b_2\in\ker{\beta_2}=\Ima{\beta_1}$, así que existe $b_1\in B_1$ con $\beta_1(b_1)=b_2$. Por la sobreyectividad de $f_1$, existe $a_1\in A_1$ tal que $f_1(a_1)=b_1$, luego aplicando conmutatividad, $f_2(\alpha_1(a_1))=\beta_1(f_1(a_1))=\beta_1(b_1)=b_2=f_2(a_2)$. Esto quiere decir que $\alpha_1(a_1)-a_2\in\ker{f_2}$, pero como $f_2$ es inyectiva llegamos a que $\alpha_1(a_1)=a_2$. Ahora, por exactitud, $0=\alpha_2(\alpha_1(a_1))=\alpha_2(a_2)=a_3$, con lo que se llega al resultado.

\underline{Sobreyectividad}

Sea $b_3\in B_3$. Denotamos $b_4=\beta_3(b_3)\in B_4$. Por sobreyectividad de $f_4$, existe $a_4\in A_4$ tal que $f_4(a_4)=b_4$. Aplicando la conmutatividad del diagrama, $f_5(\alpha_4(a_4))=\beta_4(b_4)=\beta_4(\beta_3(b_4))=0$ por exactitud. Así que de la inyectividad de $f_5$ deducimos que $\alpha_4(a_4)=0$, o lo que es lo mismo, $a_4\in\ker{\alpha_4}=\Ima{\alpha_3}$, de modo que existe $a_3\in A_3$ con $\alpha_3(a_3)=a_4$. Usando la conmutatividad del diagrama, $\beta_3(f_3(a_3))=f_4(\alpha_3(a_3))=f_4(a_4)=b_4=\beta_3(b_3)$. Por lo tanto, $f_3(a_3)-b_3\in\ker{\beta_3}=\Ima{\beta_2}$, luego existe $b_2\in B_2$ tal que $\beta_2(b_2)=f_3(a_3)-b_3$. Por sobreyectividad de $f_2$, se tiene que existe $a_2\in A_2$ con $f_2(a_2)=b_2$, de donde se sigue por conmutatividad que $f_3(\alpha_2(a_2))=\beta_2(f_2(a_2))=\beta_2(b_2)=f_3(a_3)-b_3$. Concluimos pues, que $b_3=f_3(\alpha_2(a_2)-a_3)$.
\end{solucion}

\newpage

\begin{ejercicio}{4.2}
Consideremos el siguiente diagrama conmutativo
\[
\begin{tikzcd}
0\arrow[r] & A_1\arrow[d,"f_1"]\arrow[r, "\alpha_1"] & A_2\arrow[d,"f_2"]\arrow[r, "\alpha_2"] & A_3\arrow[d,"f_3"]\arrow[r] & 0\\
0\arrow[r] & B_1\arrow[r, "\beta_1"] & B_2\arrow[r, "\beta_2"] & B_3\arrow[r] & 0
\end{tikzcd}
\]
donde las filas son sucesiones exactas. Probar que existe una sucesión exacta
\[
0\to \ker{f_1}\to \ker{f_2}\to \ker{f_3}\to \coker{f_1}\to \coker{f_2}\to \coker{f_3}\to 0.
\]
(Pista: utilizar la sucesión exacta larga de cohomología).
\end{ejercicio}
\begin{solucion}
Vamos a ver primero la exactitud de
\[
0\to \ker{f_1}\to \ker{f_2}\to \ker{f_3}.
\]
Denotamos $a_i=\alpha_i|_{\ker{f_i}}$. Estas aplicaciones están bien definidas, pues si $n\in\ker{f_i}$, entonces $f_{i+1}a_i(n)=\beta_if_i(n)=0$, así que $a_i(n)\in\ker{f_{i+1}}$. Claramente $a_1$ es inyectiva. Claramente, $\Ima{a_1}\subseteq\ker{a_2}$. Además, si $n\in\ker{f_2}$ con $a_2(n)=0$. En particular, $n\in\ker{\alpha_2}$, luego $n\in\Ima{\alpha_1}$. Así que $n=\alpha_1(m)$ para $m\in A_1$. Tenemos que probar que de hecho $m\in\ker{f_1}$. Usando la conmutatividad del diagrama, $\beta_1(f_1(m))=f_2(\alpha_1(m))=f_2(n)=0$. Como $\beta_1$ es inyectiva, $f_1(m)=0$, así que $m\in\ker{f_1}$.

Probamos ahora la exactitud de
\[
\coker{f_1}\to \coker{f_2}\to \coker{f_3}\to 0.
\]
%\[
%B_1/\Ima{f_1}\to B_2/\Ima{f_2}\to B_3/\Ima{f_3}\to 0
%\]
Definimos $b_i(n+\Ima{f_i})=\beta_i(n)+\Ima{f_{i+1}}$. Veamos que están bien definidas. Sean $n+\Ima{f_i}=n'+\Ima{f_i}$, entonces $n-n'\in\Ima{f_i}$, es decir, existe $a\in A_i$ con $f_i(a)=n-n'$. Por tanto, $\beta_i(n)-\beta(n')=\beta_if_i(a)=f_{i+1}\alpha_i(a)$, así que $\beta_i(n)-\beta(n')\in\Ima{f_{i+1}}$. Veamos que $b_2$ es sobreyectiva. Sea $m+\Ima{f_3}\in\coker{f_3}$. Como $\beta_2$ es sobreyectiva, existe $n\in B_2$ con $\beta_2(n)=m$. Entonces, $b_2(n+\Ima{f_2})=\beta_2(n)+\Ima{f_3}=m+\Ima{f_3}$. Por otro lado, claramente $b_2b_1=0$, por lo que $\Ima{b_1}\subseteq\ker{b_2}$. Recíprocamente, si $n+\Ima{f_2}\in\ker{b_2}$, es decir, $\beta_2(n)\in\Ima{f_3}$, por lo que sea $m\in A_3$ con $f_3(m)=\beta_2(n)$. Como $\alpha_2$ es sobreyectiva, sea $p\in A_2$ con $\alpha_2(p)=m$. Tenemos por conmutatividad, $\beta_2(n)=f_3\alpha_2(p)=\beta_2f_2(p)$, con lo que $n-f_2(p)\in\ker{\beta_2}=\Ima{\beta_1}$. Sea pues, $r\in B_1$ con $\beta_1(r)=n-f_2(p)$. Consideremos la clase $r+\Ima{f_1}$. Entonces, $$b_1(r+\Ima{f_1})=\beta_1(r)+\Ima{f_2}=n-f_2(p)+\Ima{f_2}=n+\Ima{f_2}.$$

Por último, probamos la exactitud de 
\[
\ker{f_2}\to \ker{f_3}\to \coker{f_1}\to \coker{f_2}.
\]
Como ya tenemos $a_2$ y $b_1$ definidas, tenemos que definir $\delta:\ker{f_3}\to\coker{f_1}$. Dado $c\in\ker{f_3}$, como $\alpha_2$ es sobreyectiva, existe $a\in A_2$ con $\alpha_2(a)=c$. Por conmutatividad, $\beta_2(f_2(a_2))=f_3(c)=0$, luego $f_2(a)\in\ker{\beta_2}=\Ima{\beta_1}$. Así que sea $b\in B_1$ con $\beta_1(b)=f_2(a)$ ($b$ es único por inyectividad). Entonces, definimos $\delta(c)=b+\Ima{f_1}$, es decir, $$\delta(c)=\beta_1^{-1}(f_2(\alpha_2^{-1}(c)))+\Ima{f_1}.$$ Tenemos que ver que está bien definida. Sean $\alpha_2(a)=\alpha_2(a')=c$. Existe un único $b'\in B_1$ con $\beta_1(b')=f_2(a')$ de la misma forma que se calculó en la definición original. Tenemos que $a-a'\in\ker{\alpha_2}=\Ima{\alpha_1}$, así que $a-a'=\alpha_1(z)$ para algún $z\in A_1$. Así que $\beta_1(f_1(z))=f_2(\alpha_1(z))=f_2(a-a')=\beta_1(b-b')$. Por inyectividad de $\beta_1$, $f_1(z)=b-b'\in\Ima{f_1}.$

Por la definición de $\delta$ es claro que $\Ima{a_2}\subseteq\ker{\delta}$. Recíprocamente, sea $p\in\ker{\delta}$ estando $p\in\ker{f_3}$. Tenemos que $p=\alpha_2(n)$ y $f_2(n)=\beta_1(m)$ con $m\in\Ima{f_1}$. Sea $m=f_1(x)$ con $x\in A_1$. Entonces $f_2\alpha_1(x)=\beta_1 f_1(x)=f_2(n)$.  Entonces, $n-\alpha_1(x)\in\ker{f_2}$. Además, $a_2(n-\alpha_1(x))=\alpha_2(n)-\alpha_2\alpha_1(x)=\alpha_2(n)=p$, así que $p\in\Ima{a_2}$.

Por último, sea $p\in\ker{f_3}$, entonces $p=\alpha_2(b)$ y $f_2(a)=\beta_1(b)$, así que $\delta(p)=b+\Ima{f_1}$. Por tanto. $b_1(\delta(p))=\beta_1(b)+\Ima{f_2}=0$. Recíprocamente, supongamos que $b+\Ima{f_1}\in\ker{b_1}$. Entonces, $\beta_1(b)\in\Ima{f_2}$, digamos $\beta_1(b)=f_2(a)$ para $a\in A_2$. Sea $p=\alpha_2(a)$. Entonces, $f_3(\alpha_2(a))=\beta_2(f_2(a))=\beta_2(\beta_1(a))=0$. Por tanto, $p\in\ker{f_3}$, y $\delta(p)=b+\Ima{f_1}$.
\end{solucion}
\newpage

\begin{ejercicio}{4.3}
En el diagrama conmutativo
\[
\begin{tikzcd}
& 0\arrow[d] & 0\arrow[d] & 0\arrow[d] & 0\arrow[d] \\
0\arrow[r] & A^{0,0}\arrow[r]\arrow[d] & A^{1,0}\arrow[r]\arrow[d]& A^{2,0}\arrow[r]\arrow[d]  & A^{3,0}\arrow[r]\arrow[d] &\cdots\\
0\arrow[r] & A^{0,1}\arrow[r]\arrow[d] & A^{1,1}\arrow[r]\arrow[d]& A^{2,1}\arrow[r]\arrow[d]  & A^{3,1}\arrow[r]\arrow[d] &\cdots\\
0\arrow[r] & A^{0,2}\arrow[r]\arrow[d] & A^{1,2}\arrow[r]\arrow[d]& A^{2,2}\arrow[r]\arrow[d]  & A^{3,2}\arrow[r]\arrow[d] &\cdots\\
0\arrow[r] & A^{0,3}\arrow[r]\arrow[d] & A^{1,3}\arrow[r]\arrow[d]& A^{2,3}\arrow[r]\arrow[d]  & A^{3,3}\arrow[r]\arrow[d] &\cdots\\
& \vdots & \vdots & \vdots & \vdots
\end{tikzcd}
\]
la horizontal $(A^{*,q})$ y la vertical $A^{p,*})$ son complejos de cadenas donde $A^{p,q}=0$ si o bien $p<0$ o bien $q<0$. Supongamos que
\begin{gather*}
H^p(A^{*,q})=0\text{ para }q\neq 0\text{ y todo }p,
H^q(A^{p,*})=0\text{ para }p\neq 0\text{ y todo }q.
\end{gather*}
Construir isomorfismos $H^p(A^{*,0})\to H^p(A^{0,*})$ para todo $p$.
\end{ejercicio}
\begin{solucion}

\end{solucion}
\newpage

\begin{ejercicio}{4.4}
Sea $0\to A^0\overset{d_0}{\to}A^1\overset{d_1}{\to}\cdots\overset{d_{n-1}}{\to}A^n\to 0$ un complejo de cadenas y supongamos que $\dim A^i<\infty$. La \emph{característica de Euler} está definida como
\[
\chi(A^*)=\sum_{i=0}^n (-1)^i\dim A^i.
\]
Probar que $\chi(A^*)=0$ si $A^*$ es exacta. Probar que la sucesión 
\[
0\to H^i(A^*)\to A^i/\Ima d^{i-1}\overset{d^i}{\to}\Ima d^i\to 0
\]
es exacta y concluir que
\[
\dim A^i-\dim\Ima d^{i-1}=\dim H^i(A^*)+\dim\Ima d^i.
\]
Probar que $\chi(A^*)=\sum_{i=0}^n\dim H^i(A^*)$.
\end{ejercicio}
\begin{solucion}

\end{solucion}

\newpage

\begin{ejercicio}{4.5}
Asociar a dos aplicaciones lineales componibles 
\[
f:V_1\to V_2,\quad g:V_2\to V_3
\]
una sucesión exacta
\[
0\to \ker{f}\to\ker{g\circ f}\to\ker{g}\to\coker{f}\to\coker{g\circ f}\to\coker{g}\to 0.
\]
\end{ejercicio}
\begin{solucion}

\end{solucion}

\end{document}