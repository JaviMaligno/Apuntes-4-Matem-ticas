	\documentclass[twoside]{article}
\usepackage{../../estilo-ejercicios}

%--------------------------------------------------------
\begin{document}

\title{Ejercicios de From Calculus to Cohomology, Capítulo 4}
\author{Javier Aguilar Martín}
\maketitle


\begin{ejercicio}{4.1}
Consideremos el diagrama conmutativo de espacios vectoriales y aplicaciones lineales en el que las filas son exactas
\[
\begin{tikzcd}
A_1\arrow[d,"f_1"]\arrow[r] & A_2\arrow[d,"f_2"]\arrow[r] & A_3\arrow[d,"f_3"]\arrow[r] & A_4\arrow[d,"f_4"]\arrow[r] & A_5\arrow[d,"f_5"]\\
B_1\arrow[r] & B_2\arrow[r] & B_3\arrow[r] & B_4\arrow[r] & B_5
\end{tikzcd}
\]
Supongamos que $f_4$ es inyectiva, $f_1$ es sobreyectiva y $f_2$ es inyectiva. Probar que $f_3$ es inyectiva. Suponer que $f_2$ es sobreyectiva, $f_4$ es sobreyectiva y $f_5$ es inyectiva. Probar que $f_3$ es sobreyectiva. 
\end{ejercicio}
\begin{solucion}

\end{solucion}

\newpage

\begin{ejercicio}{4.2}
Consideremos el siguiente diagrama conmutativo
\[
\begin{tikzcd}
0\arrow[r] & A_1\arrow[d,"f_1"]\arrow[r] & A_2\arrow[d,"f_2"]\arrow[r] & A_3\arrow[d,"f_3"]\arrow[r] & 0\\
0\arrow[r] & B_1\arrow[r] & B_2\arrow[r] & B_3\arrow[r] & 0
\end{tikzcd}
\]
donde las filas son sucesiones exactas. Probar que existe una sucesión exacta
\[
0\to \ker{f_1}\to \ker{f_2}\to \ker{f_3}\to \coker{f_1}\to \coker{f_2}\to \coker{f_2}\to 0.
\]
(Pista: utilizar la sucesión exacta larga de cohomología).
\end{ejercicio}
\begin{solucion}

\end{solucion}
\newpage

\begin{ejercicio}{4.3}
En el diagrama conmutativo
\[
\begin{tikzcd}
& 0\arrow[d] & 0\arrow[d] & 0\arrow[d] & 0\arrow[d] \\
0\arrow[r] & A^{0,0}\arrow[r]\arrow[d] & A^{1,0}\arrow[r]\arrow[d]& A^{2,0}\arrow[r]\arrow[d]  & A^{3,0}\arrow[r]\arrow[d] &\cdots\\
0\arrow[r] & A^{0,1}\arrow[r]\arrow[d] & A^{1,1}\arrow[r]\arrow[d]& A^{2,1}\arrow[r]\arrow[d]  & A^{3,1}\arrow[r]\arrow[d] &\cdots\\
0\arrow[r] & A^{0,2}\arrow[r]\arrow[d] & A^{1,2}\arrow[r]\arrow[d]& A^{2,2}\arrow[r]\arrow[d]  & A^{3,2}\arrow[r]\arrow[d] &\cdots\\
0\arrow[r] & A^{0,3}\arrow[r]\arrow[d] & A^{1,3}\arrow[r]\arrow[d]& A^{2,3}\arrow[r]\arrow[d]  & A^{3,3}\arrow[r]\arrow[d] &\cdots\\
& \vdots & \vdots & \vdots & \vdots
\end{tikzcd}
\]
la horizontal $(A^{*,q})$ y la vertical $A^{p,*})$ son complejos de cadenas donde $A^{p,q}=0$ si o bien $p<0$ o bien $q<0$. Supongamos que
\begin{gather*}
H^p(A^{*,q})=0\text{ para }q\neq 0\text{ y todo }p,
H^q(A^{p,*})=0\text{ para }p\neq 0\text{ y todo }q.
\end{gather*}
Construir isomorfismos $H^p(A^{*,0})\to H^p(A^{0,*})$ para todo $p$.
\end{ejercicio}
\begin{solucion}

\end{solucion}
\newpage

\begin{ejercicio}{4.4}
Sea $0\to A^0\overset{d_0}{\to}A^1\overset{d_1}{\to}\cdots\overset{d_{n-1}}{\to}A^n\to 0$ un complejo de cadenas y supongamos que $\dim A^i<\infty$. La \emph{característica de Euler} está definida como
\[
\chi(A^*)=\sum_{i=0}^n (-1)^i\dim A^i.
\]
Probar que $\chi(A^*)=0$ si $A^*$ es exacta. Probar que la sucesión 
\[
0\to H^i(A^*)\to A^i/\Ima d^{i-1}\overset{d^i}{\to}\Ima d^i\to 0
\]
es exacta y concluir que
\[
\dim A^i-\dim\Ima d^{i-1}=\dim H^i(A^*)+\dim\Ima d^i.
\]
Probar que $\chi(A^*)=\sum_{i=0}^n\dim H^i(A^*)$.
\end{ejercicio}
\begin{solucion}

\end{solucion}

\newpage

\begin{ejercicio}{4.5}
Asociar a dos aplicaciones lineales componibles 
\[
f:V_1\to V_2,\quad g:V_2\to V_3
\]
una sucesión exacta
\[
0\to \ker{f}\to\ker{g\circ f}\to\ker{g}\to\coker{f}\to\coker{g\circ f}\to\coker{g}\to 0.
\]
\end{ejercicio}
\begin{solucion}

\end{solucion}

\end{document}