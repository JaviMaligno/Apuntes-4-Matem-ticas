	\documentclass[twoside]{article}
\usepackage{../../estilo-ejercicios}

%--------------------------------------------------------
\begin{document}

\title{Ejercicios de From Calculus to Cohomology, Capítulo 3}
\author{Javier Aguilar Martín}
\maketitle


\begin{ejercicio}{3.1}
Probar para un abierto de $\R^2$ que el complejo de deRham
\[
0\to\Omega^0(U)\to\Omega^1(U)\to\Omega^2(U)\to 0
\]
es isomorfo al complejo
\[
0\to C^{\infty}(U;\R)\overset{grad}{\to}C^{\infty}(U;\R^2)\overset{rot}{\to}C^{\infty}(U;\R)\to 0.
\]
Análogamente, probar que para un abierto de $\R^3$ el complejo de deRham es isomorfo a
\[
0\to C^{\infty}(U;\R)\overset{grad}{\to}C^{\infty}(U;\R^3)\overset{rot}{\to}C^{\infty}(U;\R^3)\overset{div}{\to}C^{\infty}(U;\R)\to 0.
\]
\end{ejercicio}
\begin{solucion}
Será suficiente ver quién es la derivada exterior en cada caso. Sea $U\subseteq\R^2$, veamos en qué consiste $d:\Omega^0(U)\to\Omega^0(U)$. Si $f\in\Omega^0(U)$, $df=\parcial{f}{x}\varepsilon_1+\parcial{f}{y}\varepsilon_2$, que es isomorfo a $grad(f)$. Además, el conjunto imagen $grad(\Omega^0(U))\subseteq C^{\infty}(U;\R^2)$, por lo que es correcto elegir como siguiente conjunto como $C^{\infty}(U;\R^2)$, que es isomorfo como módulo a $\Omega^1(U)$. Sea ahora $w\in\Omega^1(U)$, por lo que $w=f_1\varepsilon_1+f_2\varepsilon_2$. 
\[dw=d(f_1\varepsilon_1+f_2\varepsilon_2)=df_1\land\varepsilon_1+df_2\land\varepsilon_2=\left(\parcial{f_2}{x_1}-\parcial{f_1}{x_2}\right)\varepsilon_1\land\varepsilon_2\cong rot(f_1,f_2)
\]
Con la misma observación el conjunto imagen está bien definido. Y por el capítulo 1, la sucesión es exacta. Como bien dice el enunciado, el caso de $\R^3$ es análogo.
\end{solucion}

\newpage

\begin{ejercicio}{3.2}

\end{ejercicio}
\begin{solucion}
 
\end{solucion}
\newpage

\begin{ejercicio}{3.3}
\end{ejercicio}
\begin{solucion}
\end{solucion}
\newpage

\begin{ejercicio}{3.4}

\end{ejercicio}
\begin{solucion}

\end{solucion}

\newpage

\begin{ejercicio}{3.5}

\end{ejercicio}
\begin{solucion}


\end{solucion}

\newpage

\begin{ejercicio}{3.6}

\end{ejercicio}
\begin{solucion}

\end{solucion}

\newpage

\begin{ejercicio}{3.7}

\end{ejercicio}
\begin{solucion}

\end{solucion}

\end{document}