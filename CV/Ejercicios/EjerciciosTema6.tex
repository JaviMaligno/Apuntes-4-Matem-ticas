	\documentclass[twoside]{article}
\usepackage{../../estilo-ejercicios}

%--------------------------------------------------------
\begin{document}

\title{Ejercicios de From Calculus to Cohomology, Capítulo 7}
\author{Javier Aguilar Martín}
\maketitle


\begin{ejercicio}{7.1}
 Probar que $\R^n$ no contiene ningún subconjunto homeomorfo a $D^m$ cuando $m>n$.
\end{ejercicio}
\begin{solucion}
Si $D^m$ fuera homeomorfo a un cierto subconjunto $A\subseteq\R^n$ entonces también sus interiores son homeomorfos, con lo que habría un abierto no vacío n $\R^n$ homeomorfo a un abierto de $\R^m$, lo cual contradice la invariancia de dimensión.


\end{solucion}

\newpage

\begin{ejercicio}{7.2}
Sea $\Sigma\subseteq\R^n$ homeomorfo a $S^k$ ($1\leq k\leq n-2$). Probar que
\[
H^p(\R^n-\Sigma)\cong\begin{cases}
\R& p=0,n-k-1,n-1\\
0 & c.c.
\end{cases}
\]
\end{ejercicio}
\begin{solucion}
%\S sale el símbolo de "capítulo"
Basta hacer el cálculo para $\Sigma=S^k\subseteq\R^{k+1}\subseteq\R^n$. Como $\R^n-S^k$ es conexo para todo $1\leq k\leq n-2$, tenemos que $H^0(\R^n-S^k)\cong\R$. Primero calculamos
\[
H^p(\R^{k+1}-S^k)=H^p(\mathring{D}^{k+1})\oplus H^p(\R^{k+1}-D^{k+1}).
\]
Entonces $H^0(\R^{k+1}-S^k)\cong \R\oplus\R$, así que $H^1(\R^{k+2}-S^k)\cong\R$, lo cual, para $n=k+2$ prueba que $H^{n-k-1}(\R^n-S^k)\cong\R$. Podemos extender este resultado a todo $n\geq k+2$ usando que $H^{1+m}(\R^{n+m}-S^k)\cong H^1(\R^n-S^k)$. 

Por otro lado, para $p>0$, $H^k(\R^{k+1}-S^k)\cong \R$ y $H^p(\R^{k+1}-S^k)=0$ en cualquier otro caso. Por tanto $H^{k+1}(\R^{k+2}-S^k)\cong \R$ y $H^p(\R^{k+2}-S^k)=0$ en otro caso que no haya sido cubierto anteriormente.  Esto nos da los resultados que faltaban para $n=k+2$, y del mismo modo que en el párrafo anterior se puede extender para todo $n\geq k+2$.
\end{solucion}
\newpage

\begin{ejercicio}{7.3}
Probar que no hay ninguna aplicación continua $g:D^n\to S^{n-1}$ con $g|_{S^{n-1}}\simeq Id|_{S^{n-1}}$.
\end{ejercicio}
\begin{solucion}
En la demostración del lema 7.2 en lugar de $g(t\cdot r(x))$ cogeríamos $G(x,t)=g(2tr(x))$ cuando $t\in [0,1/2]$ y $G(x,t)=H(r(x), 2t-1)$ cuando $t\in [1/2,0]$, donde $H$ es una homotopía entre la $g$ y $Id|_{S^{n-1}}$. Esto nos daría de nuevo una homotopía entre una constante y $r$.
\end{solucion}
\newpage

\begin{ejercicio}{7.4}
Sea $f:D^n\to\R^n$ una aplicación continua, y sea $r\in(0,1)$ dado. Supongamos para todo $x\in S^{n-1}$ que $||f(x)-x||\leq 1-r$. Probar que $f(D^n)$ contiene el disco cerrado de radio $r$ y centro 0.

(Pista: Modificar la prueba del teorema del punto fijo de Brouwer y usar ejercicios 6.4 y 7.3)
\end{ejercicio}
\begin{solucion}
Supongamos que hay algún punto $y$ del disco cerrado de radio $r$ y centro 0 que no está en $f(D^n)$. Para cada punto $z=f(x)\in f(D^n)$ definimos $g:B[0,1]\to S^{n-1}$ donde $g(x)$ es la intersección con $S^{n-1}$ de la semirrecta que comienza en $z$ y pasa por $y$. 

Sea $h(y)$ la distancia con signo desde $y$ hasta el hiperplano $\pi$ orientado de modo que $h(p)>0$ donde $p$ tiene coordenadas no negativas (está por encima del hiperplano). Esto implica que para todo $y,\in B(x,1-r)$, $h(y)>0$. Además, $h(y)=y\cdot x$ siendo $x\perp \pi$ y $|x|=1$.  También se tiene $h(g(x))=\frac{f(x)-x_0}{|f(x)-x_0|}\cdot x=\frac{1}{|f(x)-x_0|}(f(x)\cdot x-x_0\cdot x)$, pues $g(x)=\frac{f(x)-x_0}{|f(x)-x_0|}$. Aquí todo es positivo, ya que la distancia dirigida $f(x)\cdot x=h(f(x))>h(x_0)=x_0\cdot x$, siendo $h(f(x))>0$, mientras que $h(-x)<0$, por lo que $g(x)\neq -x$. Esto implica que $g$ es homotópica a la identidad por el 6.4, lo cual contradice el ejercicio \ref{ejer:7.3}.
\end{solucion}

\newpage

\begin{ejercicio}{7.5}
Supongamos dadas dos aplicaciones continuas inyectivas $\alpha,\beta:[0,1]\to D^2$ tales que 
\begin{align*}
&\alpha(0)=(-1,0), \quad \alpha(1)=(1,0)\\
&\beta(0)=(0,-1), \quad \beta(1)=(0,1).
\end{align*}
Probar que las curvas $\alpha$ y $\beta$ intersecan (aplicar ambas partes del teorema 7.10). 
\end{ejercicio}
\begin{solucion}
Si no es inyectiva el resultado también es válido, bastará omitir los ciclos, lo que ocurre es que además también se cortan a sí mismas.

Consideremos el arco de circunferencia $C$ que une $(-1,0)$ con $(1,0)$ (por ejemplo, el superior), modificándola ligeramente para que no corte a $\beta(1)$. Definimos $\Sigma=\alpha\cup C$ homeomorfa a $S^1$. Podemos considerar $D^2$ embebido en $\R^2$, de modo que, por el teorema de la curva de Jordan, $\Sigma$ divide a $\R^2$ en dos abiertos disjuntos $U_1$ enteramente contenido en $D^2$ y $U_2$ no acotado, siendo $\Sigma$ la frontera entre $U_1$ y $U_2$. Si $\beta$ no corta a $\alpha$, como está enteramente contenida en $D^2$ (es decir, no cortará a $C$ para salirse de $D^2$), entonces pasaría de $U_1$ y a $U_2$ sin cortar la frontera entre ellos, lo cual contradice el hecho de que $\beta$ sea continua.
\end{solucion}




\end{document}