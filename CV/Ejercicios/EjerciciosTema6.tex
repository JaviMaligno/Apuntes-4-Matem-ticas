	\documentclass[twoside]{article}
\usepackage{../../estilo-ejercicios}

%--------------------------------------------------------
\begin{document}

\title{Ejercicios de From Calculus to Cohomology, Capítulo 7}
\author{Javier Aguilar Martín}
\maketitle


\begin{ejercicio}{7.1}
 Probar que $\R^n$ no contiene ningún subconjunto homeomorfo a $D^m$ cuando $m>n$.
\end{ejercicio}
\begin{solucion}
Sea $A\subseteq \R^n$ homeomorfo a $D^m\subseteq\R^m$. Como $A$ es compacto y conexo por ser $D^m$ compacto y conexo. En particular, $A$ es cerrado y no es $\R^n$. En el caso $n=1$ es trivial porque los compactos conexos de $\R$ son los intervalos cerrados y acotados, que no son homeomorfos a ningún $D^m$ con $m>1$.

Sea $\phi:A\to D^m$ el homeomorfismo de la hipótesis. Entonces, por el lema 7.6 tenemos un homeomorfismo $h:\R^{n+m}\to\R^{n+m}$ con $h(x,0_m)=(0_n,\phi(x))$. 


\end{solucion}

\newpage

\begin{ejercicio}{7.2}
Sea $\Sigma\subseteq\R^n$ homeomorfo a $S^k$ ($1\leq k\leq n-2$). Probar que
\[
H^p(\R^n-\Sigma)\cong\begin{cases}
\R& p=0,n-k-1,n-1\\
0 & c.c.
\end{cases}
\]
\end{ejercicio}
\begin{solucion}

\end{solucion}
\newpage

\begin{ejercicio}{7.3}
Probar que no hay ninguna aplicación continua $g:D^n\to S^{n-1}$ con $g|_{S^{n-1}}\simeq Id|_{S^{n-1}}$.
\end{ejercicio}
\begin{solucion}
En la demostración del lema 7.2 en lugar de $g(t\cdot r(x))$ cogeríamos $H(t\cdot r(x), t)$, donde $H$ es una homotopía entre la $g$ y $Id|_{S^{n-1}}$. Esto nos daría de nuevo una homotopía entre una constante y $r$.
\end{solucion}
\newpage

\begin{ejercicio}{7.4}
Sea $f:D^n\to\R^n$ una aplicación continua, y sea $r\in(0,1)$ dado. Supongamos para todo $x\in S^{n-1}$ que $||f(x)-x||\leq 1-r$. Probar que $f(D^n)$ contiene el disco cerrado de radio $r$ y centro 0.

(Pista: Modificar la prueba del teorema del punto fijo de Brouwer y usar ejercicios 6.4 y 7.3)
\end{ejercicio}
\begin{solucion}

\end{solucion}

\newpage

\begin{ejercicio}{7.5}
Supongamos dadas dos aplicaciones continuas inyectivas $\alpha,\beta:[0,1]\to D^2$ tales que 
\begin{align*}
&\alpha(0)=(-1,0), \quad \alpha(1)=(1,0)\\
&\beta(0)=(0,-1), \quad \beta(1)=(0,1).
\end{align*}
Probar que las curvas $\alpha$ y $\beta$ intersecan (aplicar ambas partes del teorema 7.10). 
\end{ejercicio}
\begin{solucion}
Si no es inyectiva el resultado también es válido, bastará omitir los ciclos, lo que ocurre es que además también se cortan a sí mismas.

Consideremos el arco de circunferencia $C$ que une $(-1,0)$ con $(1,0)$ (por ejemplo, el superior). Definimos $\Sigma=\alpha\cup C$ homeomorfa a $S^1$. Podemos considerar $D^2$ embebido en $\R^2$, de modo que, por el teorema de la curva de Jordan, $\Sigma$ divide a $\R^2$ en dos abiertos disjuntos $U_1$ enteramente contenido en $D^2$ y $U_2$ no acotado, siendo $\Sigma$ la frontera entre $U_1$ y $U_2$. Si $\beta$ no corta a $\alpha$, como está enteramente contenida en $D^2$ (es decir, no cortará a $C$ para salirse de $D^2$), entonces pasaría de $U_1$ y a $U_2$ sin cortar la frontera entre ellos, lo cual contradice el hecho de que $\beta$ sea continua.
\end{solucion}




\end{document}