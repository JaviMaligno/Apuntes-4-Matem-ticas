	\documentclass[twoside]{article}
\usepackage{../../estilo-ejercicios}

%--------------------------------------------------------
\begin{document}

\title{Ejercicios de From Calculus to Cohomology, Capítulo 10}
\author{Javier Aguilar Martín}
\maketitle

\begin{ejercicio}{10.1}
Sea $\pi:\R^2\to T^2$ como en el ejercicio 8.4 y sean
\[
U_1=\pi(\R\times (0,1)),\quad U_2=\pi(\R\times (-\frac{1}{2},\frac{1}{2})).
\]
Probar que $U_1$ y $U_2$ son difeomorfos a $S^1\times\R$, y que $U_1\cap U_2$ tiene 2 componentes conexas, ambas isomorfas a $S^1\times\R$. Nótese que $U_1\cup U_2=T^2$.

Usar la sucesión exacta de Mayer-Vietoris y el corolario 10.14 para probar que
\[
H^0(T^2)\cong H^2(T^2)\cong\R\text{ y }H^1(T^2)\cong\R^2.
\]
\end{ejercicio}
\begin{solucion}
FALTA PROBAR EL DIFEOMORFISMO

Como el toro es orientable conexo y compacto, el corolario 10.14 nos da $H^2(T^2)\cong\R$. Por otra parte, $H^p(S^1\times\R)\cong H^p(S^1)$ para todo $p$ por la invarianza homotópica de la cohomología. Por ser en particular conexo, $H^0(T^2)\cong\R$. En Mayer-Vietoris
\[
0\to\R\xrightarrow{I^*}\R\oplus\R\xrightarrow{J^*}\R\oplus\R\xrightarrow{\partial_*}H^1(T^2)\xrightarrow{I^*}\R\oplus\R\xrightarrow{J^*}\R\oplus\R\xrightarrow{\partial^*}\R\xrightarrow{} 0
\]
Reiterando el teorema del rango-nulidad llegamos a que $\Ima{\partial^*}=\ker{I^*}=\R=\Ima{I^*}$. Por el primer teorema de isomorfía, $H^1(T^2)=\R\oplus\R$. 

USO QUE LA COHOMOLOGÍA DE UN CONEXO TIENE DIMENSIÓN 1

\end{solucion}
\newpage


\begin{ejercicio}{10.2}
Con la notación del ejercicio anterior tenemos variedades diferenciables
\[
C_1=\pi(\R\times\{a\}),\quad C_2=\pi(\{b\}\times\R)\quad (a,b\in\R)
\]
de $T^2$ que son difeomorfas a $S^1$ y tienen la orientación inducida por $\R$. Probar que la aplicación
\[
\Omega^1(T^2)\to\R^2; w\mapsto (\int_{C_1}w, \int_{C_2}w)
\]
induce un isomorfismo $H^1(T^2)\to \R^2$. Probar que este isomorfismo es independiente de $a$ y $b$.
\end{ejercicio}
\begin{solucion}
Primero probamos que la aplicación está bien definida. Si $\tau=w+d\rho$, entonces como $\partial C_i=\emptyset$, se sigue de la linealidad de la integral y el teorema de Stokes. Claramente la aplicación es $\R$-lineal. 
FALTA MUCHO
\end{solucion}
\newpage

\begin{ejercicio}{10.3}
Usando la notación del ejercicio 8.4, tenemos subvariedades diferenciables en el toro $n$ dimensional $T^n=\R^n/\Z^n$, que son difeomorfas a $S^1$
\[
C_j=\{\pi(0,\dots, 0,s,0,\dots, 0)\mid s\in\R\}
\]
donde $s$ está en la $j$-ésima posición, $1\leq j\leq n$. Se las dota de la orientación canónica. Sea $w\in\Omega^1(T^n)$ una 1-forma cerrada con 
\[
\int_{C_j}w=0,\quad 1\leq j\leq n.
\]
Probar que $w$ es exacta. 

(Pista: Encontrar $f\in C^{\infty}(\R^n;\R)$ tal que $df=\pi^*(w)$, y probar que $f$ es periódica con periodo 1 en las $n$ variables. )

Probar también que la aplicación
\[
\Omega^1(T^n)\to\R^n; w\mapsto (\int_{C_1}w,\dots, \int_{C_n}w)
\]
induce un isomorfismo $H^1(T^n)\to\R^n$. 
\end{ejercicio}
\begin{solucion}
La mayor parte es análoga al ejercicio anterior.
FALTA MUCHO
\end{solucion}
\newpage

\begin{ejercicio}{10.4}
Probar que para toda $n$-variedad diferenciable conexa compacta y no orientable $M$ se tiene que $H^n(M)=0$. (Pista: usar ejercicio 9.18).
\end{ejercicio}
\begin{solucion}
Por el ejercicio 9.18 tenemos un isomorfismo entre $H^n(M)\cong H^n(\widetilde{M})_+$, y por otro lado tenemos el isomorfismo entre $H^n(\widetilde{M})\cong\R$. Consideremos $w\in\Omega^n(M)$ una $n$-forma cerrada y sea $\tilde{w}=\pi^*(w)$. Como $A$ revierte la orientación
\[
\int_{\widetilde{M}} A^*\tilde{w}=-\int_{\widetilde{M}} \tilde{w}
\]
Pero $A^*\tilde{w}=A^*\pi^*(w)=(\pi\circ A)^*(w)=\pi^*(w)=\tilde{w}$, por lo que $\int_{\widetilde{M}} \tilde{w}=0$. Esto quiere decir que existe $\rho\in\Omega^{n-1}(\widetilde{M})$ con $\tilde{w}=d\rho$. Consideramos ahora $\eta=\frac{1}{2}(\rho+A^*\rho)$. Es claro que $A^*\eta=\eta$ y además $d\eta=\frac{1}{2}(d\rho+A^*d\rho)=\tilde{w}$, así que podemos reemplazar $\rho$ por $\eta$. Como es invariante por $A^*$ AHORA FALTARÍA BAJARLA DE NUEVO PARA CONSEGUIR QUE TODAS SEAN EXACTAS, PERO NO LO VEO CLARO. $\pi^*$ es difeomorfismo local, así que alomejor podría bajarlo con $(\pi^{-1})^*$ pero no estoy seguro si con eso me valdría o tengo que hacer algún razonamiento extra por el hecho de que es local.
\end{solucion}

\newpage

\begin{ejercicio}{10.5}
Calcular la cohomología de deRham de la botella de Klein. (Pista: el oriented double covering se puede identificar con la aplicación $T^2\to K^2$ del ejercicio 8.5)
\end{ejercicio}
\begin{solucion}
Por la pista y por el ejercicio anterior tenemos que $H^2(K^2)=0$ y por conexión $H^0(K^2)=\R$, así que solo falta calcular $H^1(K^2)$. 
¿NO HARÍA FALTA LA COHOMOLOGÍA DEL TORO PARA ESTO?
\end{solucion}
\newpage

\begin{ejercicio}{10.6}
Sea $R$ un dominio compacto con borde diferenciable en una $n$-variedad diferenciable orientada $M$. Probar que para $w\in\Omega^{p-1}(M)$, $\tau\in\Omega^{n-p}(M9$ se tiene
\[
\int_R dw\land \tau=\int_{\partial R}w\land\tau+(-1)^p\int_R w\land d\tau
\]
\end{ejercicio}
\begin{solucion}
Recordemos que 
\[
d(w\land\tau)=dw\land \tau +(-1)^{p-1} w\land d\tau
\]
por lo que
\[
\int_R d(w\land\tau)=\int_R dw\land \tau +(-1)^{p-1}\int_R w\land d\tau.
\]
Usando Stokes en el miembro de la izquierda y despejando
\[
\int_R dw\land \tau=\int_{\partial R}w\land\tau+(-1)^p\int_R w\land d\tau.
\]
\end{solucion}

\newpage

\begin{ejercicio}{10.}

\end{ejercicio}
\begin{solucion}

\end{solucion}
\newpage

\begin{ejercicio}{10.}

\end{ejercicio}
\begin{solucion}

\end{solucion}
\newpage

\begin{ejercicio}{10.}

\end{ejercicio}
\begin{solucion}


\end{solucion}
\newpage

\begin{ejercicio}{10.}

\end{ejercicio}
\begin{solucion}

 
\end{solucion}
\newpage

\begin{ejercicio}{10.}

\end{ejercicio}
\begin{solucion}
Recordemos del ejercicio \ref{ejer:9.14} que para $x\in V-M$

\end{solucion}
\newpage

\begin{ejercicio}{10.}

\end{ejercicio}
\begin{solucion}

\end{solucion}
\end{document}