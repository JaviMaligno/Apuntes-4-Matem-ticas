\documentclass[twoside]{report}
\usepackage{../estilo-apuntes}
\addto\captionsspanish{\renewcommand{\chaptername}{Lección}}
\setcounter{chapter}{0}
\def\a{\alpha}
\def\b{\beta}
\def\om{\omega}
\def\fl{\longrightarrow}
\def\vp{\varphi}
\def\hs{\hspace*{1.5 em}}
\def\dif{{\rm d}}
\newcommand {\At} {{\mathcal{A}}}

\newcommand{\de}[1]{{\rm d} #1}
\newcommand{\dep}[1]{\displaystyle{\frac{\partial}{\partial #1}}}
\newcommand{\deri}[1]{\displaystyle{\frac{{\rm d}}{{\rm d} #1}}}
\newcommand{\dderi}[2]{\displaystyle{\frac{{\rm d} #1}{{\rm d} #2}}}
\newcommand{\esp}[1]{{\cal #1}}
\newcommand{\ddep}[2]{\displaystyle{\frac{\partial #1}{\partial #2}}}
\newtheorem{teoap}{Teorema}
\newtheorem{propoap}{Proposición}
\newtheorem{lemaap}{Lema}
\newtheorem{coroap}{Corolario}
\newtheorem{defiap}{Definición}

%\usepackage{pb-diagram}

\rhead{Notas para Álgebra, Combinatoria y Computación}
\lhead{Curso 2017/2018}

\begin{document}
\chapter{Geometría algebraica}
\section{Variedades}
\begin{defi}
Decimos que un espacio topológico $X$ es irreducibe si no es unión de dos cerrados (no necesariamente disjuntos).
\end{defi}

\begin{propi}
Un espacio topológico $X$ es irreducible sii todo abierto no vacío es denso.
\end{propi}

\begin{dem}
$X$ no es unión de dos cerrados propios sii $\emptyset$ no es intersección de dos abiertos sii $\not\exists$ abiertos disjuntos sii todo abierto no vacío corta todo abierto no vacío sii todo abierto no vacío es denso.
\end{dem}

\begin{propi}
Una variedad algebraica (o conjunto algebraico) $W$ es irreducible si y sólo si $\I(W)$ es primo.
\end{propi}
\begin{dem}
Supongamos que $\I(W)$ es primo y sea $W = U \cup V$. Basta ver que si $V \neq W$, entonces $U = V$.
Supongamos que $W$ es reducible, entonces $W = U \cup V$ para algún cerrado $U$ y $V$, entonces $U=\V(I)$ y $V=\V(J)$ para ideales radicales $I$ y $J$ distintos.
Como $U\cup V = \V(I)\cup\V(J) = \V(IJ)$, deducimos que $\V(IJ)=W$.
Vemos que $\sqrt{IJ}\subset I \cap J$.
Si $a \in \sqrt{IJ}$, entonces existe $a^p \in I$ e $a^q \in J$ tal que $p+q=n$.
Como $I$ y $J$ son radicales, entonces $a \in I$ y $a \in J$, luego $a \in I\cap J$.
Entonces $\I(W) = I \cap J$.

Supongamos que $\I(W)$ no fuera primo. Sea $fg \in \I(W)$, si $f$ y $g$ no están en $\I(W)$, entonces $\V(f)\cup\V(g)=\V(fg)\supset W$. Entonces $(\V(f) \cap W) \cup (\V(g) \cap W) = W$ y $W$ no es irreducible.
\end{dem}

\begin{lemma}
Una variedad algebraica finita $X\subset k^n$ tiene cardinal $d$ si y solo si la dimensión como espacio vectorial de $k[x_1,\dots,x_n]/\I(X)$ es $d$.
\end{lemma}
\chapter{Bases de Gröbner}
\section{Monomios}
\begin{lemma}
Un monomio $x^{α} \in \gene{LT(I)}$ si y sólo si $\exists f \in I$ tal que $LT(f) = x^{α}$.
\end{lemma}
\begin{dem}
\begin{itemize}
\item[($\Leftarrow$)] Definición de $\gene{LT(I)}$.
\item[($\Rightarrow$)] Si $x^{α} \in \gene{LT(I)}$, entonces existe $f \in I$ tal que $LT(f) \mid x^{α}$ por el lema 2 de la sección 2.4. Digamos que $LT(f) = cx^{β}$. Tenemos que $α-β \in \Z_{≥0}^n$. Entonces $g:=\frac{1}{c}x^{α-β}f \in I$ y $LT(g) = x^{α}$.
\end{itemize}
\end{dem}
\begin{lemma} $LT(f)$ no divide a $LT(f-LT(f))$.
\end{lemma}
\begin{dem}
Por el Ejercicio 2.2.7, sabemos que para cualquier orden monomial si $\alpha>\beta$ entonces $x^\alpha$ no divide a $^\beta$. Por definición de orden monomial, si $LT(f)$ es el monomio cuyo multigrado es mayor que cualquier otro $f$, de donde se tiene el resultado.
\end{dem}
\begin{lemma} Si $r,r'$ son tales que ningún monomio de $\gene{LT(I)}$ divide a ninguno de sus términos, entonces ningún monomio de $\gene{LT(I)}$ divide a ningún término de $r+r'$.
\end{lemma}
\begin{dem}
Si el resultado no fuera cierto, existiría algún término divisible por algún monomio de $\gene{LT(I)}$. Sin embargo, dicho monomio sería un monomio de $r$ o un monomio de $r'$ o una combinación de ambos, que solo cambiaría el coeficiente pero no el multigrado, luego ninguno puede ser divisible por hipótesis.
\end{dem}
\section{Anillos noetherianos}
\begin{defi}
Un anillo se dice \textbf{noetheriano} si todo ideal es finitamente generado.
\end{defi}
\begin{prop}
Son anillos noetherianos:
\begin{itemize}
\item Los anillos finitos
\item Los dominios de ideales principales (y por tanto los cuerpos).
\end{itemize}
\end{prop}
\begin{prop}
Un anillo A es noetheriano si y solo si verifica la condición de cadena ascendentes para ideales (ACC). 
\end{prop}
\begin{dem}
\begin{itemize}
\item[]
	\item Supongamos que A es noetheriano y consideremos la cadena creciente de ideales:
	\[ I_1 \subseteq I_2 \subseteq I_3 \subseteq \dots \subseteq I_r \subseteq \dots \]
	Tomamos $I = \bigcup_{n\geq 1} I_n$. Tenemos que $I$ es un ideal, por lo tanto está finitamente generado como $I = \langle a_1, \dots, a_r \rangle$. Cada $a_i \in I_{n_i}$ para algún $n_i$. Entonces $a_i \in I_n$  $\forall i$, si $n > \max\{n_1,\dots,n_r\}$. Llegamos entonces a que $I \subseteq I_n \subseteq I$, es decir $I_n = I$. Por lo tanto $I_{n+k} = I$ $\forall k \geq 0$. 

	\item Supongamos que se da la ACC. Sea $I \subseteq A$ y $x_1 \in I$. Entonces $\langle x_1 \rangle \subseteq I \subseteq A$. Si $I = \langle x_1 \rangle$, hemos acabado. Si no, tomamos $x_2 \in I \setminus \langle x_1 \rangle$ y $\langle x_1, x_2 \rangle \subseteq I \subseteq A$. Si $I = \langle x_1, x_2 \rangle$, hemos acabado. Si no, repetimos el argumento tomando un $x_3 \in I \setminus \langle x_1,x_2 \rangle$. Como la cadena creciente debe ser estacionaria, llegamos a que en algún momento, la cadena de ideales $\langle x_1,\dots,x_n\rangle$ debe estabilizarse y deducimos que $I = \langle x_1,\dots,x_n\rangle$. Por lo tanto $A$ es noetheriano. 

\end{itemize}
\end{dem}
\begin{teorema}[Teorema de la Base de Hilbert] Sea $A$ un anillo conmutativo noetheriano, entonces $A[x]$ también es noetheriano.
\end{teorema}
\begin{dem}
Sea $I$ idean de $A[x]$. Para $i\geq 1$ tomamos $f_i \in I\setminus \gene{f_1,\dotsc,f_{i-1}}$ elemento de grado mínimo. Si $\gene{f_1,\dotsc,f_i} = I$, entonce hemos acabado. En caso contrario, para cada $f_i$ consideramos $a_i$ su coeficiente líder y el ideal $\gene{a_1,\dotsc,a_i,\dotsc}\subset A$. Como $A$ es noetheriano, $\exists m\in \N$ tal que $\gene{a_1,\dotsc,a_i,\dotsc}=\gene{a_1,\dotsc,a_m}$. Vamos a probar que $I=\gene{f_1,\dotsc,f_m}$. Si no es así, entonces tomamos $f_{m+1}$ elemento de grado mínimo en $ I\setminus \gene{f_1,\dotsc,f_{m}}$. Sabemos que $a_{m+1}=\sum_{j=1}^m u_j a_j$ para ciertos $u_j\in A$. Dado que el $\deg(f_{m+1})\geq \deg(f_i)$ para $i=1,\dotsc,m$ el siguiente polinomio está bien definido
$$
g = \sum_{j=1}^m u_jf_jx^{\deg(f_{m+1})-\deg(f_j)} \in \gene{f_1,\dotsc,f_m}
$$
y además tiene el mismo coeficiente líder que $f_{m+1}$.  Obviamente $f_{m+1}-g \notin \gene{f_1,\dotsc,f_m}$, pues en ese caso $f_{m+1}$ también pertencería, pero $f_{m+1}-g$ tiene grado menor estricto que $f_{m+1}$, lo cuál es una contradicción con que $f_{m+1}$ sea de grado mínimo.
\end{dem}

\section{Demostración automátiac de teoremas}
Vamos a demostrar el \textbf{teorema de las medianas}.
Sean $A=(a_1,a_2)$, $B=(b_1,b_2)$ y $C=(c_1,c_2)$ puntos de $\R^2$.
Consideramos el triángulo $\triangle ABC$.
Sean $M_{AB}=(m_{11},m_{12})$, $M_{BC}=(m_{21},m_{22})$ y $M_{CA}=(m_{31},m_{32})$ las medianas del triángulo entre los correspondientes puntos.
Los puntos $M$ están definidos por las raíces de los siguientes polinomios:
\begin{align*}
	h_1 \equiv 2m_{11}-a_1-b_1\\
	h_2 \equiv 2m_{12}-a_2-b_2\\
	h_3 \equiv 2m_{21}-b_1-c_1\\
	h_4 \equiv 2m_{22}-b_2-c_2\\
	h_5 \equiv 2m_{31}-1_1-c_1\\
	h_6 \equiv 2m_{32}-a_2-c_2\\
\end{align*}
Sea $G=(g_1,g_2)$ el punto de corte de $AM_{AB}$ y $BM_{CA}$, es decir, el baricéntro.
Se tiene que $G \in AM_{BC}$ si y sólo si $(g_1,g_2)$ es raíz de:
\[ h_7 \equiv \begin{vmatrix}1 & 1 & 1\\a_1 & m_{21} & g_1\\\\a_2 & m_{22} & g_2\end{vmatrix} \]
$G \in BM_{CA}$ si y sólo si $(g_1,g_2)$ es raíz de:
\[ h_8 \equiv \begin{vmatrix}1 & 1 & 1\\b_1 & m_{31} & g_1\\\\b_2 & m_{32} & g_2\end{vmatrix} \]

El teorema consiste en demostrar que $G \in CM_{AB}$, es decir, que:
\[ c \equiv \begin{vmatrix}1 & 1 & 1\\c_1 & m_{11} & g_1\\\\c_2 & m_{12} & g_2\end{vmatrix} = 0 \]

Tenemos un conjunto de polinomios en $\mathbb{K}[a_1,a_2,\dots,g_1,g_2]$.
Tenemos que:
\[ h_1=h_2=\dots=h_8 = 0 \Rightarrow c=0 \]
es equivalente a:
\[ \V(h_1,\dots,h_8) \subset \V(c) \]
o, alternativamente:
\[ c \in \sqrt{\gene{h_1,\dots,h_8}} \]

Hemos transformado el problema geométrico en un problema algebraico de pertenencia en ideal.
En este caso estamos en la pertenencia al radical de un ideal, esto es equivalente a comprobar la pertenencia en un ideal con una variable más por el siguiente lema.

\begin{lemma}
Para un ideal $I \subset K[\underline{x}]$, son equivalentes:
\begin{enumerate}
\item $c \in \sqrt{I}$
\item $\gene{I, ct-1} = \gene{1}$, con $\gene{I, ct-1} \subset K[\underline{x},t]$.
\end{enumerate}
\end{lemma}
\begin{proof}\mbox{}
\begin{itemize}
\item[$(1)\Rightarrow(2)$] Sea $c \in \sqrt{I}$ y llamemos $\overline{I} = \gene{I, ct-1}$.
Queremos probar que $\V(\overline{I})=\emptyset$.
Supongamos que $(\underline{a},b) \in \V(\overline{I})$.
Entonces $f_i(\underline{a}) = 0$ para $1\leq i \leq r$, donde $I = \gene{f_1,\dots,f_r}$.
Entonces $c \in \sqrt{I}$, implica que $c(\underline{a})=0$.
Luego $c(\underline{a})\cdot b - 1 = -1 \neq 0$.
Esto es una contradicción. 
\item[$(2)\Rightarrow(1)$] 
Para todo $\underline{a} \in \V(I) = \V(f_1,\dots,f_r)$, se tiene que $f_i(\underline{a})$ para $1\leq i \leq r$.
Existe un polinomio $h$ tal que:
\[ 1 = h(\underline{a},t)(1-c(\underline{a})\cdot t)\]
o equivalentemente:
\[ h(\underline{a},t) = \frac{1}{1-c(\underline{a})\cdot t} \]
Como $h$ debe ser un polinomio, el denominador debe ser trivial, luego $c(\underline{a}) = 0$.
\end{itemize}
\end{proof}

No siempre el métdo funciona.
Como ejemplo, sea $G$ el baricentro, $H$ el ortocentro y $O$ el circuncentro.
Veamos que están alineados (la recta que une estos puntos se llama \emph{recta de Euler}).
Fijando un sistema de referencia adecuado: $A=(0,0)$, $B=(1,0)$ y $C=(c_1,c_2)$.

Tenemos que la mediana desde $C$ da la ecuación:
\[ (1-2c_1)y+2c_2x-c_2 = 0 \]
La mediana desde $A$:
\[ (c_1+1)y+c_2x = 0 \]

\begin{itemize}
\item Condiciones sobre $G=(x_1,x_2)$
\[ h_1 \colon (1-2c_1)x_2 + 2c_2x_1 - c_2 = 0 \]
\[ h_2 \colon (c_1+1)x_2 - c_2x_1 = 0 \]
\item Condiciones sobre $H = (y_1,y_2)$
\[ h_3 \colon y_1 - c_1 = 0 \]
\[ h_4 \colon -c_2 y - c_1 y_1 + y_1 = 0 \]
\item Condiciones sobre $O = (z_1,z_2)$
\[ h_5 \colon d(A,O)^2 - d(B,O)^2 = z_1^2+z_2^2 - \left((z_1-1)^2+z_2^2\right) = 0 \]
\[ h_6 \colon d(A,O)^2 - d(C,O)^2 = z_1^2+z_2^2 - \left((z_1-c_1)^2+(z_2-c_2)^2\right) = 0 \]
\end{itemize}
Tenemos que demostrar que:
\[ c \colon \begin{vmatrix}1 & 1 & 1\\x_1 & y_1 & z_1\\x_2 & y_2 & z_2\end{vmatrix} = 0\]
Definiendo $I = \gene{h_1,\dots,h_6}$, el teorema quedaría probado si $c \in \sqrt{I}$, es decir, si $\gene{I,ct-1}=\gene{1}$.
Sin embargo, la base de Gröbner de $\gene{I,ct-1}$ es:
\[ \mathcal{G} = \{x_1y_2t-x_1z_2t+z_2c_1t-\frac{1}{2}y_2t-1, c_1^2-c_1, x_2, y_1-c_1, z_1-\frac{1}{2},c_2 \}\]
Obsérvese ese $c_2$ al final, eso no puede estar bien.
Hay que añadir la \emph{condición degenerada} $c_2 \neq 0$ al ideal, lo hacemos añadiendo $c_2t-1$.
Con este polinomio añadido a $I$, ya tenemos que $\gene{I,ct-1} = \gene{1}$.
\end{document}
