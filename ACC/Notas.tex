\documentclass[twoside]{report}
\usepackage{../estilo-apuntes}
\addto\captionsspanish{\renewcommand{\chaptername}{Lección}}
\setcounter{chapter}{0}
\def\a{\alpha}
\def\b{\beta}
\def\om{\omega}
\def\fl{\longrightarrow}
\def\vp{\varphi}
\def\hs{\hspace*{1.5 em}}
\def\dif{{\rm d}}
\newcommand {\At} {{\mathcal{A}}}

\newcommand{\de}[1]{{\rm d} #1}
\newcommand{\dep}[1]{\displaystyle{\frac{\partial}{\partial #1}}}
\newcommand{\deri}[1]{\displaystyle{\frac{{\rm d}}{{\rm d} #1}}}
\newcommand{\dderi}[2]{\displaystyle{\frac{{\rm d} #1}{{\rm d} #2}}}
\newcommand{\esp}[1]{{\cal #1}}
\newcommand{\ddep}[2]{\displaystyle{\frac{\partial #1}{\partial #2}}}
\newtheorem{teoap}{Teorema}
\newtheorem{propoap}{Proposición}
\newtheorem{lemaap}{Lema}
\newtheorem{coroap}{Corolario}
\newtheorem{defiap}{Definición}

%\usepackage{pb-diagram}

\rhead{Notas para Álgebra, Combinatoria y Computación}
\lhead{Curso 2017/2018}

\begin{document}
\begin{defi}
Decimos que un espacio topológico $X$ es irreducibe si no es unión de dos cerrados (no necesariamente disjuntos).
\end{defi}

\begin{propi}
Un espacio topológico $X$ es irreducible sii todo abierto no vacío es denso.
\end{propi}

\begin{dem}
$X$ no es unión de dos cerrados propios sii $\emptyset$ no es intersección de dos abiertos sii $\not\exists$ abiertos disjuntos sii todo abierto no vacío corta todo abierto no vacío sii todo abierto no vacío es denso.
\end{dem}

\begin{dem}
Supongamos que $\I(W)$ es primo y sea $W = U \cup V$. Basta ver que si $V \neq W$, entonces $U = V$.
Supongamos que $W$ es reducible, entonces $W = U \cup V$ para algún cerrado $U$ y $V$, entonces $U=\V(I)$ y $V=\V(J)$ para ideales radicales $I$ y $J$ distintos.
Como $U\cup V = \V(I)\cup\V(J) = \V(IJ)$, deducimos que $\V(IJ)=W$.
Vemos que $\sqrt{IJ}\subset I \cap J$.
Si $a \in \sqrt{IJ}$, entonces existe $a^p \in I$ e $a^q \in J$ tal que $p+q=n$.
Como $I$ y $J$ son radicales, entonces $a \in I$ y $a \in J$, luego $a \in I\cap J$.
Entonces $\I(W) = I \cap J$.

Sea $fg \in \I(W)$, si $f$ y $g$ no están en $\I(W)$, entonces $\V(f)\cup\V(g)=\V(fg)\supset W$. Entonces $(\V(f) \cap W) \cup (\V(g) \cap W) = W$ y $W$ no es irreducible.
\end{dem}
\end{document}
