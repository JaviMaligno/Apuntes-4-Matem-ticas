\documentclass[twoside]{article}
\usepackage{../../estilo-ejercicios}

%--------------------------------------------------------
\begin{document}

\title{Ejercicios de Ideals, Varieties, and Algorithms (4ª Edición)}
\author{Diego Pedraza López, Javier Aguilar Martín, Rafael González López}
\maketitle

\begin{ejercicio}{1.3.1}
Parametrize all solutions of linear equations
\begin{align*}
x+2y-2z+w=-1\\
x+y+z-w=2
\end{align*}
\end{ejercicio}
\begin{solucion}
Simplemente consideramos $(t,u,v,-1-t-2u+2v)$ y $(t,u,v,-2+t+u+v)$
\end{solucion}

\newpage

\begin{ejercicio}{1.3.2}
Use a trigonometric identity to show that
\begin{align*}
x&=\cos(t)\\
y&=\cos(2t)
\end{align*}
parametrizes a portion of a parabola. Indicate exactly what portion of the parabola is covered.
\end{ejercicio}
\begin{solucion}
Sabemos que $\cos(2t)=\cos^2(t)-\sin^2(t) = 2\cos^2(t)-1$. Por tanto, es claro que se parametriza la parábola $y=x^2$ en el intervalo $[-1,1]$.
\end{solucion}

\newpage

\begin{ejercicio}{1.3.3}
Given $f\in k[x]$, find a parametrization of $\V(y-f(x))$.
\end{ejercicio}
\begin{solucion}
Trivialérrimo. $x=t$ y $y=f(t)$.
\end{solucion}

\newpage

\begin{ejercicio}{1.3.4}
Consider parametric representation
\begin{align*}
x&=\frac{t}{1+t}\\
y&=1-\frac{1}{t^2}
\end{align*}
\begin{enumerate}[a)]
\item Find the equation of the affine variety determined by the above parametric equations.
\item Show that the above equations parametrize all points of the variety found in part
(a) except for the point $(1, 1)$.
\end{enumerate}
\end{ejercicio}
\begin{solucion}
\begin{enumerate}[a)]
\item[]
\item Lo que pega es hacer 
$$
x=\frac{t}{1+t} = \frac{1+t}{1+t}-\frac{1}{1+t} = 1 -\frac{1}{1+t}\qquad 1+t= \frac{1}{1-x} \qquad
t = \frac{x}{1-x}
$$
$$
y = 1-\frac{1}{t^2} = 1 - \frac{(1-x)^2}{x^2} = \frac{2x-1}{x^2}
$$
Luego tenemos $\V(x^2y-2x+1)$. Efectivamente es fácil ver que la parametrización está contenida en $\V(x^2y-2x+1)$. \item Si $x=1$ entonces $y-2+1=0$, luego $y=1$. Si $x\neq 1$ entonces podemos escribir 
$t=x(1-x)^{-1}$ y procedemos inversamente al apartado anterior para ver que verifican la parametrización. Para ver que la parametrizació no cubre al $(1,1)$, vemos que no hay soluciones si impones que
$$
1=y=1-\frac{1}{t^2} \qquad 0 = \frac{1}{t^2}
$$
\end{enumerate}
\end{solucion}

\newpage


\begin{ejercicio}{1.3.5}
This problem will be concerned with the hyperbola $x^2 − y^2 = 1$.
\begin{enumerate}[a)]
\item Just as trigonometric functions are used to parametrize the circle, hyperbolic functions are used to parametrize the hyperbola. Show that the point
\begin{align*}
x&=\cosh(t)\\
y&=\sinh(t)
\end{align*}
always lies on $x^2 − y^2 = 1$. What portion of the hyperbola is covered?
\item Show that a straight line meets a hyperbola in $0$, $1$, or $2$ points, and illustrate your
answer with a picture. Hint: Consider the cases $x = a$ and $y = mx + b$ separately.
\item Adapt the argument given at the end of the section to derive a parametrization of the hyperbola. Hint: Consider nonvertical lines through the point $(−1, 0)$ on the hyperbola.
\item The parametrization you found in part (c) is undefined for two values of t. Explain how this relates to the asymptotes of the hyperbola.
\end{enumerate}
\end{ejercicio}
\begin{solucion}
Veamos la solución
\begin{enumerate}[a)]
\item Es inmediato a partir de la igualdad $\cosh^2(t)
-\sinh^2(t)=1$. De hecho, la cubre completamente.
\item Si $x=a$ entonces tenemos los casos donde $a\in(-1,1)$ donde corta en 0 puntos, los casos $a=-1,1$, donde corta en un punto, y el resto de casos donde se cortan en dos puntos. Si $y=mx+b$ entonces $(1-m^2)x^2-2mbx-(b^2+1)=0$. El discriminante es $4m^2b^2+4(1-m^2)(b^2+1) = 4(b^2-m^2+1)$. 
\begin{itemize}
\item Si el discriminante es negativo no tenemos ninguna solución.
\item Si el discriminante es nulo, obtenemos una única solución para la variable $x$, que será $x=\frac{mb}{1-m^2}$. Como $b^2-m^2 =1$ entonces $x= \frac{m}{b}$. En este caso $-y^2 = 1 + \frac{m}{b}$. Luego tendremos a lo sumo dos soluciones.
\item Si tenemos dos soluciones para $x$.  
\end{itemize}
\end{enumerate}
\end{solucion}

\newpage
\begin{ejercicio}{1.3.6}
The goal of this problem is to show that the sphere $x^2 + y^2 + z^2 = 1$ in $3$-dimensional space can be parametrized by
\begin{align*}
x&=\frac{2u}{u^2+v^2+1}\\
y&=\frac{2v}{u^2+v^2+1}\\
z&=\frac{u^2+v^2-1}{u^2+v^2+1}
\end{align*}
The idea is to adapt the argument given at the end of the section to 3-dimensional space.
\begin{enumerate}[a)]
\item Given a point $(u, v, 0)$ in the $(x, y)$-plane, draw the line from this point to the “north pole” $(0, 0, 1)$ of the sphere, and let $(x, y, z)$ be the other point where the line meets the sphere. Draw a picture to illustrate this, and argue geometrically that mapping
$(u, v)$ to $(x, y, z)$ gives a parametrization of the sphere minus the north pole.
\item Show that the line connecting $(0, 0, 1)$ to $(u, v, 0)$ is parametrized by $(tu, tv, 1 − t)$,
where $t$ is a parameter that moves along the line.
\item Substitute $x = tu$, $y = tv$ and $z = 1−t$ into the equation for the sphere $x^2+y^2+z^2 = 1$.
Use this to derive the formulas given at the beginning of the problem.
\end{enumerate}
\end{ejercicio}
\begin{solucion}
Es trivial, pues es la proyección estereográfica.
\end{solucion}

\newpage

\begin{ejercicio}{1.3.7} Adapt the argument of the previous exercise to parametrize the “sphere” $x^2_1
+\cdots+x^{2}_n = 1$ in n-dimensional affine space. Hint: There will be n − 1 parameters.
\end{ejercicio}
\begin{solucion}
Análogo al anterior.
\end{solucion}

\newpage

\begin{ejercicio}{1.2.8}
It can take some work to show that something is not an affine variety. For example,
consider the set
$$X = \{(x, x) \mid x ∈ \R, x \neq 1\} ⊆ \R^2,$$
which is the straight line $x = y$ with the point $(1, 1)$ removed. To show that $X$ is not
an affine variety, suppose that $X = \V( f_1,\dots, f_s)$. Then each $f_i$ vanishes on $X$, and if
we can show that $f_i$ also vanishes at $(1, 1)$, we will get the desired contradiction. Thus,
here is what you are to prove: if $f ∈ \R[x, y]$ vanishes on $X$, then $f (1, 1) = 0$. Hint: Let
$g(t) = f (t, t)$, which is a polynomial $\R[t]$. Now apply the proof of Proposition 5 of §1.
\end{ejercicio}
\begin{solucion}
Con lo que nos dice el enunciado, tenemos que $g(t)$ es un polinomio en $\R[t]$ que se anula en infinitos puntos, luego es el polinomio idénticamente 0, por lo que $f(1,1)=g(1)=0$. Esto, aplicado a los polinomios que generarían $X$, da lugar a la contradicción de que $(1,1)\in X$.
\end{solucion}

\newpage

\begin{ejercicio}{1.2.9}
Let $R = \{(x, y) ∈ \R^2 \mid  y > 0\}$ be the upper half plane. Prove that $R$ is not an affine
variety.
\end{ejercicio}
\begin{solucion}
Supongamos que $R=\V(f_1,\dots, f_s)$. Sea $M$ la mayor potencia de $x$ o de $y$ que aparece en los polinomios. Entonces $f_1,\dots, f_s$ se anulan en todo el plano superior, en particular en el lattice $\Z^2_{M+1}$. Así que por el Ejercicio 1.1.6, los polinomios se anulan en todo el plano (de hecho en todo $\C^2$). 
\end{solucion}

\newpage

\begin{ejercicio}{1.2.10}
Let $\Z^n ⊆ \C^n$ consist of those points with integer coordinates. Prove that $\Z^n$ is not an
affine variety. Hint: See Exercise 6 from §1.
\end{ejercicio}
\begin{solucion}
Por el ejercicio 1.1.6 tenemos que si $\V(f_1,\dots, f_s)=\Z^n$, al ser $f_1=\dots=f_s=0$ en $\Z^n$, estos polinomios se anulan en todo $\C^n$, luego $\V(f_1,\dots, f_s)=\C^n$. Contradicción. 
\end{solucion}

\newpage

\begin{ejercicio}{1.2.11}
So far, we have discussed varieties over $\R$ or $\C$. It is also possible to consider varieties
over the field $\Q$, although the questions here tend to be much harder. For example, let n
be a positive integer, and consider the variety $F_n ⊆ \Q^2$ defined by
$$x^n + y^n = 1.$$
Notice that there are some obvious solutions when $x$ or $y$ is zero. We call these \emph{trivial
solutions}. An interesting question is whether or not there are any nontrivial solutions.
\begin{itemize}
\item[a.] Show that $F_n$ has two trivial solutions if $n$ is odd and four trivial solutions if $n$ is even.
\item[b.] Show that $F_n$ has a nontrivial solution for some $n ≥ 3$ if and only if Fermat’s Last
Theorem were false.
Fermat’s Last Theorem states that, for $n ≥ 3$, the equation
$$x^n + y^n = z^n$$
has no solutions where $x$, $y$, and $z$ are nonzero integers. The general case of this conjecture
was proved by Andrew Wiles in 1994 using some very sophisticated number theory. The
proof is extremely difficult.
\end{itemize}
\end{ejercicio}
\begin{solucion}
\begin{itemize}
\item[a.] Las soluciones triviales vienen dadas por $y^n=1$ y $x^n=1$. Para $n$ impar tenemos las soluciones $y=1$ por un lado y $x=1$ por otro. En el caso de $n$ impar añadimos en la primera $y=-1$ y en la segunda $x=-1$.
\item[b.] Si $F_n$ tiene una solución no trivial, entonces, eliminando denominadores, tendríamos una solución para el teorema de Fermat. Por otro lado, si el teorema de Fermat fuera falso, podríamos dividir por $z^n$ y nos daría una una solución de la forma $(x,y,1)$ con $(x,y)\in\Q^2$.
\end{itemize}
\end{solucion}

\newpage

\begin{ejercicio}{1.2.15}
In Lemma 2, we showed that if $V$ and $W$ are affine varieties, then so are their union $V∪W$
and intersection $V ∩W$. In this exercise we will study how other set-theoretic operations
affect affine varieties.
\begin{itemize}
\item[a.] Prove that finite unions and intersections of affine varieties are again affine varieties.
Hint: Induction.
\item[b.] Give an example to show that an infinite union of affine varieties need not be an
affine variety. Hint: By Exercises 8–10, we know some subsets of kn that are not
affine varieties. Surprisingly, an infinite intersection of affine varieties is still an affine
variety. This is a consequence of the Hilbert Basis Theorem, which will be discussed
in Chapters 2 and 4.
\item[c.] Give an example to show that the set-theoretic difference $V \ W$ of two affine varieties
need not be an affine variety.
\item[d.] Let $V ⊆ k^n$ and $W ⊆ k^m$ be two affine varieties, and let
$$V × W = \{(x_1,\dots , x_n, y_1,\dots , y_m) ∈ k^{n+m} \mid  (x_1, . . . , x_n) ∈ V, (y_1, . . . , y_m) ∈ W\}$$
be their Cartesian product. Prove that $V ×W$ is an affine variety in $k^{n+m}$. Hint: If $V$ is
defined by $f_1,\dots , f_s ∈ k[x_1,\dots , x_n]$, then we can regard $f_1,\dots, f_s$ as polynomials in
$k[x_1,\dots , x_n, y_1,\dots, y_m]$, and similarly for $W$. Show that this gives defining equations
for the Cartesian product.
\end{itemize}
\end{ejercicio}
\begin{solucion}
\begin{itemize}
\item[a.] Trivial.
\item[b.] $\{y>0\}=\cup_{c>0}\{y=c\}$. Por el ejercicio \ref{ejer:1.2.9} el conjunto no es una variedad afín, pero cada recta sí lo es. 
\item[c.] $\{y=x\}\setminus \{(1,1)\}$. Por el ejercicio \ref{ejer:1.2.8} no es variedad afín, aunque la recta y el punto sí lo son.
\item[d.] Sean $V=\V(f_1,\dots, f_r)$ con $f_1,\dots, f_r\in k[x_1,\dots, x_n]$ y $W=\V(g_1,\dots, g_s)$ con $g_1,\dots, g_s\in k[y_1,\dots, y_m]$. Entonces $V\times W=\{(x_1,\dots, x_n,y_1,\dots, y_m)\in k^{n+m}\mid f_i(x_1,\dots, x_n)=0, g_j(y_1,\dots, y_m)=0, 1\leq i\leq n, 1\leq j\leq m\}$, luego por definción es una variedad afín.
\end{itemize}
\end{solucion}

\end{document}