\documentclass[twoside]{article}
\usepackage{../../estilo-ejercicios}

%--------------------------------------------------------
\begin{document}

\title{Ejercicios de Ideals, Varieties, and Algorithms (4ª Edición)}
\author{Diego Pedraza López, Javier Aguilar Martín}
\maketitle

\begin{ejercicio}{1.2.1}
Sketch the following affine varieties in $\R^2$.
\begin{enumerate}[a.]
\item $\V(x^2+4y^2+2x-16y+1)$.
\item $\V(x^2-y^2)$.
\item $\V(2x+y-1,3x-y+2)$.
\end{enumerate}
\end{ejercicio}
\begin{solucion}
\begin{enumerate}[a.]
\item[]
\item Elipse. Véase $x^2+4y^2+2x-16y+1=0$ es equivalente a $(x+1)^2+(2y-4)^2=4^2$.
\item Par de rectas $x+y=0$ y $x-y=0$.
\item Punto $(-0.2,1.4)$.
\end{enumerate}
\end{solucion}

\newpage

\begin{ejercicio}{1.2.2}
In $\R^2$, sketch $\V(y^2-x(x-1)(x-2))$.
For which $x$'s is it possible to solver for $y$?
How many $y$'s correspond to each $x$?
What symmetry does the curve have?.
\end{ejercicio}
\begin{solucion}
Véase que para $x<0$, $-x(x-1)(x-2)>0$, luego no existe solución en $x<0$.
Lo mismo ocurre para $x$ en $(1,2)$.
Además, obsérvese que la curva tiene simetría respecto del eje de abcisa.
Estamos ante una curva elíptica.
\end{solucion}

\newpage

\begin{ejercicio}{1.2.3}
In the plane $\R^2$, draw a picture to illustrate
\[ \V(x^2+y^2-4) \cap \V(xy-1) = \V(x^2+y^2-4,xy-1) \]
and determine the points of intersection. Note that this is a special case of Lemma 2.
\end{ejercicio}
\begin{solucion}
De $xy-1=0$, obtenemos $x=1/y$. Entonces $x^2+1/x^2=4$, luego $x^4-4x^2+1=0$.
\[ x^2 = \frac{4 \pm \sqrt{16-4}}{2} = 2 \pm \sqrt{3} \]
\end{solucion}

\newpage

\begin{ejercicio}{1.2.4}
Sketch the following affine varieties in $\R^3$.
\begin{enumerate}[a.]
\item $\V(x^2+y^2+z^2-1)$.
\item $\V(x^2+y^2-1)$.
\item $\V(x+2,y-1.5,z)$.
\item $\V(xz^2-xy)$.
\item $\V(x^4-zx,x^3-yx)$.
\item $\V(x^2+y^2+z^2-1, x^2+y^2+(z-1)^2-1)$.
\end{enumerate}
\end{ejercicio}
\begin{solucion}
\begin{enumerate}[a.]
\item[]
\item Esfera centrada en el origen de radio $1$.
\item Cilindro con eje en el eje $z$ de radio $1$.
\item Punto $(-2,1.5,0)$.
\item Unión de plano $x=0$ y cilindro parabólico $y=z^2$.
\item Unión del plano $x=0$ y la curva $(t,t^2,t^3)$.
\item Circunferencia intersección de dos esferas.
\end{enumerate}
\end{solucion}

\newpage


\begin{ejercicio}{1.2.5}
Use the proof of Lemma 2 to sketch $\V((x-2)(x^2-y),y(x^2-y),(z+1)(x^2-y))$ in $\R^3$.
\end{ejercicio}
\begin{solucion}
Tenemos que:
\[ \V((x-2)(x^2-y),y(x^2-y),(z+1)(x^2-y)) = \V(x^2-y) \cup \V(x-2,y,z+1) \]
Luego tenemos un cilindro parabólico $y=x^2$ y el punto $(2,0,-1)$.
\end{solucion}

\newpage

\begin{ejercicio}{1.2.6}
Let us show that all finite subsets of $k^n$ are affine varieties.
\begin{enumerate}[a.]
\item Prove that a single point $(a_1,\dots,a_n) \in k^n$ is an affine variety.
\item Prove that every finite subset of $k^n$ is an affine variety.
\end{enumerate}
\end{ejercicio}
\begin{solucion}
Tenemos que $\{(a_1,\dots,a_n)\} = \V(x_1-a_1,x_2-a_2,\dots,x_n-a_n)$, luego todo punto es una variedad afín.
Todo subconjunto finito de $k^n$ es unión de puntos.
Como los puntos son variedades afines y la unión de variedades afines es variedad afín, todo subconjunto finito de $k^n$ es variedad afín.
\end{solucion}

\newpage

\begin{ejercicio}{1.2.7}
One of the pretties examples from polar coordinates is the four-leaved rose.
This curve is defined by the polar equation $r = \sin(2θ)$. We will show that this curve is an affine variety.
\begin{enumerate}[a.]
\item Using $r^2=x^2+y^2$, $x=r \cos θ$ and $y = r \sin θ$, show that the four-leaved rose is contained in the affine variety $\V((x^2+y^2)^3-4x^2y^2)$.
\item Now argue carefully that $\V((x^2+y^2)^3-4x^2y^2)$ is contained in the four-leaved rose.
This is trickier that it seems since $r$ can be negative in $r = \sin(2θ)$.
\end{enumerate}
Combining parts $(a)$ and $(b)$, we have proved that the four-leaved rose is the affine variety $\V((x^2+y^2)^3-4x^2y^2)$.
\end{ejercicio}
\begin{solucion}
\begin{enumerate}[a.]
\item Tenemos que
\[ (x^2+y^2)^3-4x^2y^2=r^6-4(r\cosθ)^2(r\sinθ)^2=r^6-4r^4\cos^2θ\sin^2θ \]
Como $\sin(2θ)=2\sinθ\cosθ$:
\[ r^6-4r^4\cos^2θ\sin^2θ = r^6-r^4 (\sin2θ)^2 = r^6 - r^4 r^2 = 0 \]
Luego la rosa de cuatro hojas está contenida en $\V((x^2+y^2)^3-4x^2y^2)$. Recíprocamente, sea $(x,y)\in \V((x^2+y^2)^3-4x^2y^2)$. Vamos a tomar las coordenadas polares de este punto
$$
r=\sqrt{x^2+y^2} \qquad \theta = \tan^{-1}(y/x)
$$
Como $(x^2+y^2)=(2xy)^{2/3}$ entonces $r=(2xy)^{1/3}$. Además, sabemos que $$\sin(2\tan^{-1}(x)) = \frac{2x}{{x^2+1}}$$
De esto, deducimos que
$$
\sin(2\theta) = \sin(2\tan^{-1}(y/x)) = \frac{2xy}{y^2+x^2} = \frac{2xy}{(2xy)^{2/3}} = (2xy)^{1/3} = r  
$$
\end{enumerate}
\end{solucion}

\newpage

\begin{ejercicio}{1.2.8}
It can take some work to show that something is not an affine variety. For example,
consider the set
$$X = \{(x, x) \mid x ∈ \R, x \neq 1\} ⊆ \R^2,$$
which is the straight line $x = y$ with the point $(1, 1)$ removed. To show that $X$ is not
an affine variety, suppose that $X = \V( f_1,\dots, f_s)$. Then each $f_i$ vanishes on $X$, and if
we can show that $f_i$ also vanishes at $(1, 1)$, we will get the desired contradiction. Thus,
here is what you are to prove: if $f ∈ \R[x, y]$ vanishes on $X$, then $f (1, 1) = 0$. Hint: Let
$g(t) = f (t, t)$, which is a polynomial $\R[t]$. Now apply the proof of Proposition 5 of §1.
\end{ejercicio}
\begin{solucion}
Con lo que nos dice el enunciado, tenemos que $g(t)$ es un polinomio en $\R[t]$ que se anula en infinitos puntos, luego es el polinomio idénticamente 0, por lo que $f(1,1)=g(1)=0$. Esto, aplicado a los polinomios que generarían $X$, da lugar a la contradicción de que $(1,1)\in X$.
\end{solucion}

\newpage

\begin{ejercicio}{1.2.9}
Let $R = \{(x, y) ∈ \R^2 \mid  y > 0\}$ be the upper half plane. Prove that $R$ is not an affine
variety.
\end{ejercicio}
\begin{solucion}
Supongamos que $R=\V(f_1,\dots, f_s)$. Sea $M$ la mayor potencia de $x$ o de $y$ que aparece en los polinomios. Entonces $f_1,\dots, f_s$ se anulan en todo el plano superior, en particular en el lattice $\Z^2_{M+1}$. Así que por el Ejercicio 1.1.6, los polinomios se anulan en todo el plano (de hecho en todo $\C^2$). 
\end{solucion}

\newpage

\begin{ejercicio}{1.2.10}
Let $\Z^n ⊆ \C^n$ consist of those points with integer coordinates. Prove that $\Z^n$ is not an
affine variety. Hint: See Exercise 6 from §1.
\end{ejercicio}
\begin{solucion}
Por el ejercicio 1.1.6 tenemos que si $\V(f_1,\dots, f_s)=\Z^n$, al ser $f_1=\dots=f_s=0$ en $\Z^n$, estos polinomios se anulan en todo $\C^n$, luego $\V(f_1,\dots, f_s)=\C^n$. Contradicción. 
\end{solucion}

\newpage

\begin{ejercicio}{1.2.11}
So far, we have discussed varieties over $\R$ or $\C$. It is also possible to consider varieties
over the field $\Q$, although the questions here tend to be much harder. For example, let n
be a positive integer, and consider the variety $F_n ⊆ \Q^2$ defined by
$$x^n + y^n = 1.$$
Notice that there are some obvious solutions when $x$ or $y$ is zero. We call these \emph{trivial
solutions}. An interesting question is whether or not there are any nontrivial solutions.
\begin{itemize}
\item[a.] Show that $F_n$ has two trivial solutions if $n$ is odd and four trivial solutions if $n$ is even.
\item[b.] Show that $F_n$ has a nontrivial solution for some $n ≥ 3$ if and only if Fermat’s Last
Theorem were false.
Fermat’s Last Theorem states that, for $n ≥ 3$, the equation
$$x^n + y^n = z^n$$
has no solutions where $x$, $y$, and $z$ are nonzero integers. The general case of this conjecture
was proved by Andrew Wiles in 1994 using some very sophisticated number theory. The
proof is extremely difficult.
\end{itemize}
\end{ejercicio}
\begin{solucion}
\begin{itemize}
\item[a.] Las soluciones triviales vienen dadas por $y^n=1$ y $x^n=1$. Para $n$ impar tenemos las soluciones $y=1$ por un lado y $x=1$ por otro. En el caso de $n$ impar añadimos en la primera $y=-1$ y en la segunda $x=-1$.
\item[b.] Si $F_n$ tiene una solución no trivial, entonces, eliminando denominadores, tendríamos una solución para el teorema de Fermat. Por otro lado, si el teorema de Fermat fuera falso, podríamos dividir por $z^n$ y nos daría una una solución de la forma $(x,y,1)$ con $(x,y)\in\Q^2$.
\end{itemize}
\end{solucion}

\newpage

\begin{ejercicio}{1.2.13}
Consider a robot arm in $\R^2$ that consists of three arms of length $3$, $2$ and $1$, respectively.
The arm of length $3$ is anchored at the origin, the arm of length $2$ is attached to the free end of the arm of length $3$, and the arm of length $1$ is attached to the free end of the arm of length $2$.
The ``hand'' of the robot arm is attached to the end of the arm of length $1$.
\begin{enumerate}[a.]
\item Draw a picture of the robot arm.
\item How many variables does it take to determine the ``state'' of the robot arm?
\item Give the equations for the variety of possible states.
\item Using the intuitive notion of dimension discussed in this section, guess what the dimension of the variety of states should be.
\end{enumerate}
\end{ejercicio}
\begin{solucion}
Sean $(x,y), (z,w)$ y $(u,v)$ las coordenadas de las posición del final de los tres brazos robóticos.
Es decir, el estado del brazo robótico viene dado por $(x,y,z,w,u,v) \in \R^6$.
Los estados posibles son los que obedecen la siguientes ecuaciones:
\begin{align*}
x^2 + y^2 & = 3^2\\
(z-x)^2 + (w-y)^2 & = 2^2\\
(u-z)^2 + (v-w)^2 & = 1^2
\end{align*}
Ingenuamente diríamos que la variedad de estados, como está definida por $3$ ecuaciones en $\R^6$, tiene dimensión $6-3=3$.
\end{solucion}

\newpage

\begin{ejercicio}{1.2.14}
This exercise will study the possible ``hand'' positions of the robot arm described in Exercise \ref{ejer:1.2.13}.
\begin{enumerate}[a.]
\item If $(u,v)$ is the position of the hand, explain why $u^2+v^2≤36$.
\item Suppose we ``lock'' the joint between the length $3$ and length $2$ arms to form a straight angle, but allow the other joint to move freely. Draw a picture to show that in these configurations, (u,v) can be any point of the annulus $16≤u^2+v^2≤36$.
\item Draw a picture to show that $(u,v)$ can be any point in the disk $u^2+v^2 ≤ 36$.
\end{enumerate}
\end{ejercicio}
\begin{solucion}\mbox{}
\begin{enumerate}[a.]
\item 
\begin{align*}
 \norm{(u,v)} & = \norm{(u,v)-(z,w) + (z,w)} ≤ \norm{(u,v)-(z,w)} + \norm{(z,w)} = 1 + \norm{(z,w)}\\
 & = 1 + \norm{(z,w)-(x,y)+(x,y)} ≤ 1 + \norm{(z,w)-(x,y)} + \norm{(x,y)} = 1 + 2 + 3 = 6
\end{align*}
Luego $u^2+v^2 = \norm{(u,v)}^2 ≤ 36$.
\item Si imponemos que los dos primeros brazos formen un ángulo llano, tenemos que $\norm{(z,w)} = 5$. Es decir, el tercer brazo tiene base en la circunferencia de radio $5$. Todo punto en la corona circuar de radios $4$ y $6$ está a distancia $1$ de algún punto de esta circunferencia, entonces $(u,v)$ puede ser cualquier punto de la corona circular.
\item Basta hacer el mismo argumento imponiendo que los dos primeros brazos formen un ángulo recto para recubrir la corona circular de radios $2$ y $4$ y ángulo nulo, para recubrir el círculo de radio $2$. Entonces $(u,v)$ cubre cualquier solución de $u^2+v^2 ≤ 6^2$.
\end{enumerate}
\end{solucion}
\newpage

\begin{ejercicio}{1.2.15}
In Lemma 2, we showed that if $V$ and $W$ are affine varieties, then so are their union $V∪W$
and intersection $V ∩W$. In this exercise we will study how other set-theoretic operations
affect affine varieties.
\begin{itemize}
\item[a.] Prove that finite unions and intersections of affine varieties are again affine varieties.
Hint: Induction.
\item[b.] Give an example to show that an infinite union of affine varieties need not be an
affine variety. Hint: By Exercises 8–10, we know some subsets of kn that are not
affine varieties. Surprisingly, an infinite intersection of affine varieties is still an affine
variety. This is a consequence of the Hilbert Basis Theorem, which will be discussed
in Chapters 2 and 4.
\item[c.] Give an example to show that the set-theoretic difference $V \ W$ of two affine varieties
need not be an affine variety.
\item[d.] Let $V ⊆ k^n$ and $W ⊆ k^m$ be two affine varieties, and let
$$V × W = \{(x_1,\dots , x_n, y_1,\dots , y_m) ∈ k^{n+m} \mid  (x_1, . . . , x_n) ∈ V, (y_1, . . . , y_m) ∈ W\}$$
be their Cartesian product. Prove that $V ×W$ is an affine variety in $k^{n+m}$. Hint: If $V$ is
defined by $f_1,\dots , f_s ∈ k[x_1,\dots , x_n]$, then we can regard $f_1,\dots, f_s$ as polynomials in
$k[x_1,\dots , x_n, y_1,\dots, y_m]$, and similarly for $W$. Show that this gives defining equations
for the Cartesian product.
\end{itemize}
\end{ejercicio}
\begin{solucion}
\begin{itemize}
\item[a.] Trivial.
\item[b.] $\{y>0\}=\cup_{c>0}\{y=c\}$. Por el ejercicio \ref{ejer:1.2.9} el conjunto no es una variedad afín, pero cada recta sí lo es. 
\item[c.] $\{y=x\}\setminus \{(1,1)\}$. Por el ejercicio \ref{ejer:1.2.8} no es variedad afín, aunque la recta y el punto sí lo son.
\item[d.] Sean $V=\V(f_1,\dots, f_r)$ con $f_1,\dots, f_r\in k[x_1,\dots, x_n]$ y $W=\V(g_1,\dots, g_s)$ con $g_1,\dots, g_s\in k[y_1,\dots, y_m]$. Entonces $V\times W=\{(x_1,\dots, x_n,y_1,\dots, y_m)\in k^{n+m}\mid f_i(x_1,\dots, x_n)=0, g_j(y_1,\dots, y_m)=0, 1\leq i\leq n, 1\leq j\leq m\}$, luego por definción es una variedad afín.
\end{itemize}
\end{solucion}

\end{document}