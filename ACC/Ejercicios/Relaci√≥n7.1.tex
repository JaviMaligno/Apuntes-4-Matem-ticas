\documentclass[twoside]{article}
\usepackage{../../estilo-ejercicios}

%--------------------------------------------------------
\begin{document}

\title{Ejercicios de Ideals, Varieties, and Algorithms (4ª Edición)}
\author{Diego Pedraza López, Javier Aguilar Martín, Rafael González López}
\maketitle

\begin{ejercicio}{7.1.1}
Prove that $f ∈ k[x, y, z]$ is symmetric if and only if $f (x, y, z) = f (y, x, z) = f (y, z, x)$.
\end{ejercicio}
\begin{solucion}
La implicación directa es trivial. Para la segunda basta probar que las permutaciones del enunciado generan $S_3$, pues podemos ver los polinomios que son iguales tras reordenar las variables como elementos de cada órbita por la acción de $S_3$ en las 3-uplas de variables, por tanto, como esta acción es transitiva, si son iguales los polinomios evaluados sobre la imagen mediante unos generadores, lo serán sobre todos los reordenamientos de variables.

En este caso, las permutaciones dadas son $(1\ 2)$,  $(2\ 3)$ y $(1\ 3)$. Estas son todas las permutaciones de $S_3$, por lo tanto sabemos que generan todo el grupo, con lo que ya lo tenemos.  
\end{solucion}

\newpage

\begin{ejercicio}{7.1.2}
(Requires abstract algebra) Prove that $f ∈ k[x_1, \dots , x_n]$ is symmetric if and only if
$f (x_1, x_2, x_3, \dots , x_n) = f (x_2, x_1, x_3, \dots , x_n) = f (x_2, x_3, \dots, x_n, x_1)$. Hint: Show that the
cyclic permutations $(1\ 2)$ and $(1\ 2\ \dots \ n)$ generate the symmetric group $S_n$. See Exercise
4 in Section 3.5 of DUMMIT and FOOTE (2004).
\end{ejercicio}
\begin{solucion}

\end{solucion}

\newpage

\begin{ejercicio}{7.1.3}
Let $σ^{(i)}_j$ denote the $j$-th elementary symmetric polynomial in $x_1,\dots, x_{i−1}, x_{i+1},\dots, x_n$
for $j < n$. The superscript “$(i)$” tells us to omit the variable $x_i$. Also set $σ^{(i})_n = 0$ and
$σ_0 = 1$. Prove that $σ_j = σ^{(i)}_j + x_iσ^{(i)}_{j−1}$ for all $i, j$. This identity is useful in induction
arguments involving elementary symmetric polynomials.
\end{ejercicio}
\begin{solucion}

\end{solucion}
\newpage

\begin{ejercicio}{7.1.4}
As in (2), let $f (X) = (X − x_1)(X − x2) \cdots (X − x_n)$. Prove that $f = X_n − σ_1X^{n−1} +
σ_2X^{n−2} +\cdots+(−1)^{n−1}σ_{n−1}X +(−1)^nσ_n$. Hint: You can give an induction proof using
the identities of Exercise \ref{ejer:7.1.3}.
\end{ejercicio}
\begin{solucion}

\end{solucion}

\newpage

\begin{ejercicio}{7.1.5}

\end{ejercicio}
\begin{solucion}

\end{solucion}
\newpage

\begin{ejercicio}{7.1.6}

\end{ejercicio}
\begin{solucion}

\end{solucion}
\newpage

\begin{ejercicio}{7.1.7}

\end{ejercicio}
\begin{solucion}

\end{solucion}
\end{document}
