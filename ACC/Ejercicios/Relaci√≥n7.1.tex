\documentclass[twoside]{article}
\usepackage{../../estilo-ejercicios}

%--------------------------------------------------------
\begin{document}

\title{Ejercicios de Ideals, Varieties, and Algorithms (4ª Edición)}
\author{Diego Pedraza López, Javier Aguilar Martín, Rafael González López}
\maketitle

\begin{ejercicio}{7.1.1}
Prove that $f ∈ k[x, y, z]$ is symmetric if and only if $f (x, y, z) = f (y, x, z) = f (y, z, x)$.
\end{ejercicio}
\begin{solucion}
La implicación directa es trivial. Para la segunda basta probar que las permutaciones del enunciado generan $S_3$, pues podemos ver los polinomios que son iguales tras reordenar las variables como elementos de cada órbita por la acción de $S_3$ en las 3-uplas de variables, por tanto, como esta acción es transitiva, si son iguales los polinomios evaluados sobre la imagen mediante unos generadores, lo serán sobre todos los reordenamientos de variables.

En este caso, las permutaciones dadas son $(1\ 2)$,  $(2\ 3)$ y $(1\ 3)$. Estas son todas las permutaciones de $S_3$, por lo tanto sabemos que generan todo el grupo, con lo que ya lo tenemos.  
\end{solucion}

\newpage

\begin{ejercicio}{7.1.2}
(Requires abstract algebra) Prove that $f ∈ k[x_1, \dots , x_n]$ is symmetric if and only if
$f (x_1, x_2, x_3, \dots , x_n) = f (x_2, x_1, x_3, \dots , x_n) = f (x_2, x_3, \dots, x_n, x_1)$. Hint: Show that the
cyclic permutations $(1\ 2)$ and $(1\ 2\ \dots \ n)$ generate the symmetric group $S_n$. See Exercise
4 in Section 3.5 of DUMMIT and FOOTE (2004).
\end{ejercicio}
\begin{solucion}
Sabemos que $S_n$ está generado por las trasposiciones de la forma $(i\ i+1)$, luego basta ver que podemos generar estas permutaciones a partir de $\tau=(1\ 2)$ y $\sigma=(1\ 2\ \dots \ n)$. Para ello, basta observar que $\sigma^{i-1}\tau\sigma^{-i+1}=(i\ i+1)$. Efectivamente
\[
\sigma^{i-1}\tau\sigma^{-i+1}(k)=\sigma^{i-1}\tau(k-i+1\mod n)=\begin{cases}
\sigma^{i-1}(2) & k-i+1\equiv 1\mod n\\
\sigma^{i-1}(1) & k-i+1\equiv 2\mod n\\
\sigma^{i-1}(k-i+1\mod n) & c.c.
\end{cases}
\]
Observamos que las congruencias de las condiciones son respectivamente equivalentes a $k\equiv i\mod n$ y $k\equiv 1\mod n$ respectivamente, luego obtenemos finalmente
\[
\sigma^{i-1}\tau\sigma^{-i+1}(k)=\begin{cases}
1+i & k\equiv i\mod n\\
i& k\equiv i+1\mod n\\
k & c.c.
\end{cases}
\]
como queríamos demostrar.
\end{solucion}

\newpage

\begin{ejercicio}{7.1.3}
Let $σ^{(i)}_j$ denote the $j$-th elementary symmetric polynomial in $x_1,\dots, x_{i−1}, x_{i+1},\dots, x_n$
for $j < n$. The superscript “$(i)$” tells us to omit the variable $x_i$. Also set $σ^{(i)}_n = 0$ and
$σ_0 = 1$. Prove that $σ_j = σ^{(i)}_j + x_iσ^{(i)}_{j−1}$ for all $i, j$. This identity is useful in induction
arguments involving elementary symmetric polynomials.
\end{ejercicio}
\begin{solucion}
Para $\sigma_1$ y $\sigma_n$ es trivial. 

\end{solucion}
\newpage

\begin{ejercicio}{7.1.4}
As in (2), let $f (X) = (X − x_1)(X − x_2) \cdots (X − x_n)$. Prove that $f = X_n − σ_1X^{n−1} +
σ_2X^{n−2} +\cdots+(−1)^{n−1}σ_{n−1}X +(−1)^nσ_n$. Hint: You can give an induction proof using
the identities of Exercise \ref{ejer:7.1.3}.
\end{ejercicio}
\begin{solucion}

\end{solucion}

\newpage

\begin{ejercicio}{7.1.5}
Consider the polynomial
$f = (x^2 + y^2)(x^2 + z^2)(y^2 + z^2) ∈ k[x, y, z]$.
\begin{enumerate}[a.]
\item Use the method given in the proof of Theorem 3 to write f as a polynomial in the
elementary symmetric polynomials $σ_1$, $σ_2$, $σ_3$.
\item Use the method described in Proposition 4 to write $f$ in terms of $σ_1$, $σ_2$, $σ_3$.
\end{enumerate}

 You can use a computer algebra system for both parts of the exercise. Note that by
stripping off the coefficients of powers of $X$ in the polynomial $(X − x)(X − y)(X − z)$,
you can get the computer to generate the elementary symmetric polynomials.
 
\end{ejercicio}
\begin{solucion}



\end{solucion}
\newpage

\begin{ejercicio}{7.1.6}
If the variables are $x_1, \dots , x_n$, show that $\sum_{i\neq j} x^2_i
x_j = σ_1σ_2 −3σ_3$. Hint: If you get stuck,
see Exercise \ref{ejer:7.1.13}. Note that a computer algebra system cannot help here!
\end{ejercicio}
\begin{solucion}
\begin{gather*}
\sigma_1\sigma_2=\sum_{i=1}^nx_i\sum_{i_1<i_2}x_{i_1}x_{i_2}=\sum_{1\leq i\leq n, i_1<i_2}x_ix_{i_1}x_{i_2}=\\
\sum_{i_2<i_2}x_{i_1}^2x_{i_2}+\sum_{i_2<i_2}x_{i_1}x_{i_2}^2+\sum_{i<i_1<i_2}x_ix_{i_1}x_{i_2}+\sum_{i_1<i<i_2}x_ix_{i_1}x_{i_2}+\sum_{i_1<i_2<i}x_ix_{i_1}x_{i_2}=\\
\sum_{i\neq j} x_i^2x_j+3\sum_{j_1<j_2<j_3}x_{j_1}x_{j_2}x_{j_3}=\sum_{i\neq j} x_i^2x_j+3\sigma_3.
\end{gather*}
\end{solucion}
\newpage

\begin{ejercicio}{7.1.7}
Let $f = x_n + a_1x_{n−1} + \cdots + a_n ∈ k[x]$ have roots $α_1,\dots, α_n$, which lie in some bigger
field $K$ containing $k$.
\begin{enumerate}[a.]
\item Prove that any symmetric polynomial $g(α_1, \dots , α_n)$ in the roots of $f$ can be expressed
as a polynomial in the coefficients $a_1, \dots , a_n$ of $f$.
\item In particular, if the symmetric polynomial $g$ has coefficients in $k$, conclude that
$g(α_1, \dots , α_n) ∈ k$.
\end{enumerate}
 
\end{ejercicio}
\begin{solucion}

\end{solucion}

\newpage

\begin{ejercicio}{7.1.9}
Given a cubic polynomial $f = x^3+a_1x^2+a_2x+a_3$, what condition must the coefficients
of $f$ satisfy in order for one of its roots to be the average of the other two? Hint: If $α_1$ is
the average of the other two, then $2α_1 − α_2 − α_3 = 0$. But it could happen that $α_2$ or
$α_3$ is the average of the other two. Hence, you get a condition stating that the product of
three expressions similar to $2α_1 − α_2 − α_3$ is equal to zero. Now use Exercise \ref{ejer:7.1.7}.
\end{ejercicio}
\begin{solucion}
\end{solucion}
\newpage

\begin{ejercicio}{7.1.13}

\end{ejercicio}
\begin{solucion}

\end{solucion}
\newpage
\begin{ejercicio}{7.1.16}
Consider the field $\F_2 = \{0, 1\}$ consisting of two elements. Show that it is impossible
to express the symmetric polynomial $xy ∈ \F_2[x, y]$ as a polynomial in $s_1$ and $s_2$ with
coefficients in $\F_2$. Hint: Show that $s_2 = s^2_1$!
\end{ejercicio}
\begin{solucion}
Por definición $s_1=x+y$, $s_2=x^2+y^2$. Como estamos en $\F_2$, $s_1^2=x^2+y^2=s_2$. Supongamos que $xy$ se expresa como un polinomio de grado $n$ en $s_1$ y $s_2$. Utilizando la observación anterior tendríamos que
\[
xy=a_1(x+y)+\sum_{i=1}^nb_i(x^2+y^2)^i+(x+y)\sum_{j=1}^{n-1}(x^2+y^2)^j,\quad a_1, b_i,c_j\in \F_2
\]
Observamos que en los dos sumandos de la derecha, tanto $x$ como $y$ aparecen siempre con grando mayor o igual que 2, mientras que en $xy$ tienen ambos grado 1. Por tanto, $xy=a_1(x+y)$, lo cual no es posible.
\end{solucion}
\end{document}
