\documentclass[twoside]{article}
\usepackage{../../estilo-ejercicios}

%--------------------------------------------------------
\begin{document}

\title{Ejercicios de Ideals, Varieties, and Algorithms (4ª Edición)}
\author{Diego Pedraza López, Javier Aguilar Martín, Rafael González López}
\maketitle

\begin{ejercicio}{7.1.1}
Prove that $f ∈ k[x, y, z]$ is symmetric if and only if $$f (x, y, z)= f (y, x, z) = f (y, z, x)$$
\end{ejercicio}
\begin{solucion}
La implicación directa es trivial. Para la segunda basta probar que las permutaciones del enunciado generan $S_3$, pues podemos ver los polinomios que son iguales tras reordenar las variables como elementos de cada órbita por la acción de $S_3$ en las 3-uplas de variables, por tanto, como esta acción es transitiva, si son iguales los polinomios evaluados sobre la imagen mediante unos generadores, lo serán sobre todos los reordenamientos de variables.

En este caso, las permutaciones dadas son $(1\ 2)$,  $(2\ 3)$ y $(1\ 3)$. Estas son todas las permutaciones de $S_3$, por lo tanto sabemos que generan todo el grupo, con lo que ya lo tenemos. 
\end{solucion}

\begin{solucion}
Una implicación es trivial. Sean $\sigma,\mu$ dos permutaciones. Notemos por $f_\sigma(x,y,z) =  f(\sigma(x),\sigma(y),\sigma(z))$. Si $\sigma$ y $\mu$ son tales que
$$f(x,y,z)=f(\sigma(x),\sigma(y),\sigma(z))=f(\mu(x),\mu(y),\mu(z))$$
Entonces es claro que $f(x,y,z)=f_{\sigma \circ \mu}(x,y,z)=f_{ \mu\circ \sigma}(x,y,z)$. Es decir, el conjunto de permutaciones que mantienen la simetría de $f$ forman un grupo. En este caso, sabemos que $(1,2)$ y $(1,3)$. Sabemos que $\gene{(1,2),(1,3)}=S_3$, por lo que se verifica para toda permutación y $f$ es simétrica.
\end{solucion}

\newpage

\begin{ejercicio}{7.1.2}
(Requires abstract algebra) Prove that $f ∈ k[x_1, \dots , x_n]$ is symmetric if and only if
$f (x_1, x_2, x_3, \dots , x_n) = f (x_2, x_1, x_3, \dots , x_n) = f (x_2, x_3, \dots, x_n, x_1)$. Hint: Show that the
cyclic permutations $(1\ 2)$ and $(1\ 2\ \dots \ n)$ generate the symmetric group $S_n$. See Exercise
4 in Section 3.5 of DUMMIT and FOOTE (2004).
\end{ejercicio}
\begin{solucion}
Sabemos que $S_n$ está generado por las trasposiciones de la forma $(i\ i+1)$, luego basta ver que podemos generar estas permutaciones a partir de $\tau=(1\ 2)$ y $\sigma=(1\ 2\ \dots \ n)$. Para ello, basta observar que $\sigma^{i-1}\tau\sigma^{-i+1}=(i\ i+1)$. Efectivamente
\[
\sigma^{i-1}\tau\sigma^{-i+1}(k)=\sigma^{i-1}\tau(k-i+1\mod n)=\begin{cases}
\sigma^{i-1}(2) & k-i+1\equiv 1\mod n\\
\sigma^{i-1}(1) & k-i+1\equiv 2\mod n\\
\sigma^{i-1}(k-i+1\mod n) & c.c.
\end{cases}
\]
Observamos que las congruencias de las condiciones son respectivamente equivalentes a $k\equiv i\mod n$ y $k\equiv 1\mod n$ respectivamente, luego obtenemos finalmente
\[
\sigma^{i-1}\tau\sigma^{-i+1}(k)=\begin{cases}
1+i & k\equiv i\mod n\\
i& k\equiv i+1\mod n\\
k & c.c.
\end{cases}
\]
como queríamos demostrar.
\end{solucion}	

\newpage

\begin{ejercicio}{7.1.3}
Let $σ^{(i)}_j$ denote the $j$-th elementary symmetric polynomial in $x_1,\dots, x_{i−1},$ $x_{i+1},\dots, x_n$
for $j < n$. The superscript “$(i)$” tells us to omit the variable $x_i$. Also set $σ^{(i)}_n = 0$ and
$σ_0 = 1$. Prove that $σ_j = σ^{(i)}_j + x_iσ^{(i)}_{j−1}$ for all $i, j$. This identity is useful in induction
arguments involving elementary symmetric polynomials.
\end{ejercicio}
\begin{solucion}
En realidad la simetría de los polinomios elementales nos permite colegir que si probamos el resultado para un $i$ fijo y para todo $j$ entonces está probado para todo $i$, pues el papel que cumple la variable $i$ es fácilmente intercambiable por simetría. Sea $i=1$ y $j$ fijo. Veamos que el resultado se tiene directamente. Podemos suponer que $n\geq 2$, pues otro caso es trivial.
\begin{align*}
\sigma_j &= \sum_{1\leq i_1<\cdots<i_j\leq n}x_{i_1}\cdots x_{i_j}\\
&= \sum_{2\leq i_2<\cdots <i_j\leq n}x_{1}x_{i_2}\cdots x_{i_j} + \sum_{2\leq i_1<\cdots<i_j\leq n}x_{i_1}\cdots x_{i_j} \\
& = x_1 \sigma_{j-1}^{(1)}+\sigma_j^{(1)}
\end{align*}
\end{solucion}
\newpage

\begin{ejercicio}{7.1.4}
As in (2), let $f (X) = (X − x_1)(X − x_2) \cdots (X − x_n)$. Prove that $f = X^n − σ_1X^{n−1} +
σ_2X^{n−2} +\cdots+(−1)^{n−1}σ_{n−1}X +(−1)^nσ_n$. Hint: You can give an induction proof using
the identities of Exercise \ref{ejer:7.1.3}.
\end{ejercicio}
\begin{solucion}
Para $n=1$ tenemos el resultado de manera trivial. Por ilustrar, para $n=2$ tenemos
$$
(X-X_1)(X-x_2) = X^2-(x_1+x_2)X+x_1x_2
$$
Supongamos que el resultado es cierto para $n-1$ y probémoslo para $n$. Esto significa
\begin{align*}
f(X)&=(X − x_1)(X − x_2) \cdots (X − x_n)\\
&= (X-x_1)[(X − x_2) \cdots (X − x_n)]\\
&=(X-x_1)[X^{n-1} − σ_1^{(1)}X^{n−2} +\cdots+(−1)^{n−2}σ_{n−2}^{(1)}X +(−1)^{n-1}σ_{n-1}^{(1)}]\\
&=X^n + (-\sigma_1^{(1)} + x_1\sigma_0^{(1)})X^{n-1}+ \cdots +(-1)^{n-1}(\sigma_{n-1}^{(1)}-x_1\sigma_{n-1}^{(1)})X+(-1)^nx_1\sigma_{n-1}^{(1)}\\
&= X^n − σ_1X^{n−1} +σ_2X^{n−2} +\cdots+(−1)^{n−1}σ_{n−1}X +(−1)^nσ_n\\
\end{align*}
Como queríamos probar.
\end{solucion}

\newpage

\begin{ejercicio}{7.1.5}
Consider the polynomial
$f = (x^2 + y^2)(x^2 + z^2)(y^2 + z^2) ∈ k[x, y, z]$.
\begin{enumerate}[a.]
\item Use the method given in the proof of Theorem 3 to write f as a polynomial in the
elementary symmetric polynomials $σ_1$, $σ_2$, $σ_3$.
\item Use the method described in Proposition 4 to write $f$ in terms of $σ_1$, $σ_2$, $σ_3$.
\end{enumerate}

 You can use a computer algebra system for both parts of the exercise. Note that by
stripping off the coefficients of powers of $X$ in the polynomial $(X − x)(X − y)(X − z)$,
you can get the computer to generate the elementary symmetric polynomials.
 
\end{ejercicio}
\begin{solucion}
Será útil tener $f$ desarrollado como $x^4y^2 + x^4z^2 + x^2y^4 + 2x^2y^2z^2 + x^2z^4 + y^4z^2 + y^2z^4$.
\begin{enumerate}[a.]
\item $LT(f)=x^4y^2=LT(\sigma_2^2\sigma_1^2)$
\begin{align*}
f_1=f-\sigma_2^2\sigma_1^2&=-2x^4yz - 2x^3y^3 - 8x^3y^2z - 8x^3yz^2 - 2x^3z^3 - 8x^2y^3z - 13x^2y^2z^2\\ &- 8x^2yz^3 - 2xy^4z - 8xy^3z^2 - 8xy^2z^3 - 2xyz^4 - 2y^3z^3
\end{align*}
$LT(f_1)=-2x^4yz=-2LT(\sigma_3\sigma_1^3)$
\begin{align*}
f_2=f_1+2\sigma_3\sigma_1^3&=-2x^3y^3 - 2x^3y^2z - 2x^3yz^2 - x^3z^3 - 2x^2y^3z - x^2y^2z^2\\
& - 2x^2yz^3 - 2xy^3z^2 - 2xy^2z^3 - 2y^3z^3
\end{align*}
$LT(f_2)=-2x^3y^3=-2LT(\sigma_2^3)$
\[
f_3=f_2+2\sigma_2^3=4x^3y^2z + 4x^3yz^2 + 4x^2y^3z + 11x^2y^2z^2 + 4x^2yz^3 + 4xy^3z^2 + 4xy^2z^3
\]
$LT(f_3)=4x^3y^2z=4LT(\sigma_1\sigma_2\sigma_3)$
\[
f_4=f_3-4\sigma_1\sigma_2\sigma_3=-x^2y^2z^2=-\sigma_3^2
\]
Por tanto
\[
f=-\sigma_3^2+4\sigma_1\sigma_2\sigma_3-2\sigma_2^3-2\sigma_3\sigma_1^3.
\]
\item Consideramos el ideal $I=\gene{\sigma_1-t_1,\sigma_2-t_2,\sigma_3-t_3}\subset k[x,y,z,t_1,t_2,t_3]$, con base de Gröbner
\[
G=\{x + y + z - t_1, y^2 + yz - yt_1 + z^2 - zt_1 + t_2, z^3 - z^2t_1 + zt_2 - t_3\}
\]
Si denotamos $g\in k[t_1,t_2,t_3]$ al resto de dividir $f$ por $G$, sabemos que $g(\sigma_1,\sigma_2,\sigma_3)$. De hecho
$$
g(t_1,t_2,t_3)= -2t_1^3t_3 + t_1^2t_2^2 + 4t_1t_2t_3 - 2t_2^3 - t_3^2
$$
Notemos que hemos encontrado la misma expresión usando ambos métodos, como no podría ser de otra forma.
\end{enumerate}



\end{solucion}
\newpage

\begin{ejercicio}{7.1.6}
If the variables are $x_1, \dots , x_n$, show that $\sum_{i\neq j} x^2_i
x_j = σ_1σ_2 −3σ_3$. Hint: If you get stuck,
see Exercise \ref{ejer:7.1.13}. Note that a computer algebra system cannot help here!
\end{ejercicio}
\begin{solucion}
\begin{gather*}
\sigma_1\sigma_2=\sum_{i=1}^nx_i\sum_{i_1<i_2}x_{i_1}x_{i_2}=\sum_{1\leq i\leq n, i_1<i_2}x_ix_{i_1}x_{i_2}=\\
\sum_{i_2<i_2}x_{i_1}^2x_{i_2}+\sum_{i_2<i_2}x_{i_1}x_{i_2}^2+\sum_{i<i_1<i_2}x_ix_{i_1}x_{i_2}+\sum_{i_1<i<i_2}x_ix_{i_1}x_{i_2}+\sum_{i_1<i_2<i}x_ix_{i_1}x_{i_2}=\\
\sum_{i\neq j} x_i^2x_j+3\sum_{j_1<j_2<j_3}x_{j_1}x_{j_2}x_{j_3}=\sum_{i\neq j} x_i^2x_j+3\sigma_3.
\end{gather*}
\end{solucion}
\newpage

\begin{ejercicio}{7.1.7}
Let $f = x_n + a_1x_{n−1} + \cdots + a_n ∈ k[x]$ have roots $α_1,\dots, α_n$, which lie in some bigger
field $K$ containing $k$.
\begin{enumerate}[a.]
\item Prove that any symmetric polynomial $g(α_1, \dots , α_n)$ in the roots of $f$ can be expressed
as a polynomial in the coefficients $a_1, \dots , a_n$ of $f$.
\item If the symmetric polynomial $g$ has coefficients in $k$, conclude that
$g(α_1, \dots , α_n) ∈ k$.
\end{enumerate}
 
\end{ejercicio}
\begin{solucion}\
\begin{enumerate}[a.]
\item Por el Ejercicio \ref{ejer:7.1.4} sabemos que los coeficientes de $f$ son, salvo signo, los polinomios simétricos en las raíces de $f$. Es decir, cada $a_i$ es la forma $(-1)^{n-i}\sigma_i(\alpha_1,\dotsc,\alpha_n)$ o, equivalentemente, $\sigma_i = (-1)^{n-i}a_i$. Además, dado que $g(\alpha_1,\dotsc,\alpha_n)$ es un polinomio simétrico, por el Teorema Fundamental de los Polinomios simétricos con coeficientes en $(\alpha_1,\dotsc,\alpha_n)$, $g=h(\sigma_1,\dotsc,\sigma_n)$. Dado que las $\sigma_i$ se pueden escribir como funciones de $a_i$ tenemos el resultado.
\item No más que tener en cuenta que para probar el resultado anterior, supuesto que $g$ tiene coeficientes en $k$, la expresión de $g$ como polinomio en los coeficientes de $a_i$ también tiene coeficientes en $k$. Por tanto, $g$ es el resultado de sumas y productos de elementos de $k$ que, por ser cuerpo, es cerrado para ambas operaciones.
\end{enumerate}

\end{solucion}

\newpage




\begin{ejercicio}{7.1.8}
As in Exercise 7, let $f = x_n +a_1x_{n−1} +\cdots + a_n ∈ k[x]$ have roots $α_1, \dotsc , α_n$, which lie
in some bigger field K containing k. The discriminant of $f$ is defined to be
$$
D( f) = \prod_{i\neq j}(\alpha_i - \alpha_j)
$$
\begin{enumerate}[a.]
\item Use Exercise 7 to show that $D( f )$ is a polynomial in $a_1, \dotsc, a_n$.
\item When $n = 2$, express $D( f )$ in terms of $a_1$ and $a_2$. Does your result look familiar?
\item  When $n = 3$, express $D( f )$ in terms of $a_1, a_2, a_3$.
\item Explain why a cubic polynomial $x^3 +a_1x^2 +a_2x+a_3$ has a multiple root if and only if $−4a^3_1a_3 + a^2_1a^2_2+ 18a_1a_2a_3 - 4a^3_2 - 27a^2_3= 0.$
\end{enumerate}
\end{ejercicio}

\begin{solucion}\
\begin{enumerate}[a.]
\item Claramente es un polinomio simétrico en las raíces de $f$, luego el primer apartado del ejercicio anterior deriva inmediatamente el resultado.
\item Si $n=2$ tenemos una parábola y tan solo dos raíces dadas por $(a_1^2\pm\sqrt{a_1^2-4a_2})/2$. Por tanto
$$
D(f)=(\alpha_1 - \alpha_2)(\alpha_2 - \alpha_1) =-\sqrt{a_1^2-4a_2}\sqrt{a_1^2-4a_2} = 4a_2 - a_1^2
$$
\item Tenemos
\begin{align*}
-D(f)&=(\alpha_1-\alpha_2)^2(\alpha_1-\alpha_3)^2(\alpha_2-\alpha_3)^2\\
&=-4\sigma_1^3\sigma_3 + \sigma_1^2\sigma_2^2 + 18\sigma_1\sigma_2\sigma_3 - 4\sigma_2^3 - 27\sigma_3^2
\end{align*}
Ahora basta tener en cuenta que en el caso cúbico $\sigma_1 = -a_1$, $\sigma_2 = a_2$ y $\sigma_3 = -a_3$.
\item Irresoluble sin conocer resultantes.
\end{enumerate}
\end{solucion}

\newpage
\begin{ejercicio}{7.1.9}
Given a cubic polynomial $f = x^3+a_1x^2+a_2x+a_3$, what condition must the coefficients
of $f$ satisfy in order for one of its roots to be the average of the other two? Hint: If $α_1$ is
the average of the other two, then $2α_1 − α_2 − α_3 = 0$. But it could happen that $α_2$ or
$α_3$ is the average of the other two. Hence, you get a condition stating that the product of
three expressions similar to $2α_1 − α_2 − α_3$ is equal to zero. Now use Exercise \ref{ejer:7.1.7}.
\end{ejercicio}
\begin{solucion}
Tenemos como dice el enunciado la ecuación
\begin{equation}\label{1}
(2α_1 − α_2 − α_3)(-α_1 + 2α_2 − α_3)(-α_1 − α_2 + 2α_3)=0.
\end{equation}
Por otro lado tenemos
\begin{align*}
a_3=&-\alpha_1\alpha_2\alpha_3\\
a_2= & \alpha_1\alpha_2+\alpha_1\alpha_3+\alpha_2\alpha_3\\
a_1=&-\alpha_1-\alpha_2-\alpha_3
\end{align*}
Tenemos, a partir de última ecuación que \ref{1} se reescribe como
\[
(a_1+3\alpha_1)(a_1+3\alpha_2)(a_1+3\alpha_3)=0
\]
que desarrollando se convierte en
\[
a_1^3 + 3a_1^2\alpha_1 + 3a_1^2\alpha_2 + 3a_1^2\alpha_3 + 9a_1\alpha_1\alpha_2+ 9a_1\alpha_1\alpha_3 + 9a_1\alpha_2\alpha_3 + 27\alpha_1\alpha_2\alpha_3=
\]
\[
a_1^3-3a_1^3+9a_1a_2-27a_3=-2a_1^3+9a_1a_2-27a_3=0
\]
\end{solucion}
\newpage

\begin{ejercicio}{7.1.10}
As in Proposition 5, let $h_j(x_1 , \dots , x_n)$ be the sum of all monomials of total degree $j$ in $x_1 ,\dots , x_n$.
Also, let $σ_0 = 1$ and $σ_i = 0$ if $i > n$.
The goal of this exercise is to show that if $j > 0$, then
\[ 0 = \sum_{i=0}^j (-1)^i h_{j-i}(x_1,\dots,x_n) \sigma_i(x_1,\dots,x_n) \]
In Exercise \ref{ejer:7.1.11}, we will use this to prove the closely related identity (5) that appears in the text.
To prove the above identity, we will compute the coefficients of the monomials $x^α$ that appear in $h_{j-i} σ_i$.
Since every term in $h_{j-i} σ_i$ has total degree $j$, we can assume that $x^α$ has total degree $j$.
We will let $a$ denote the number of variables that actually appear in $x^α$.
\begin{enumerate}
\item If $x^α$ appears in $h_{j-i} σ_i$, show that $i ≤ a$. Hint: How many variables appear in each term of $σ_i$?
\item If $i ≤ a$, show that exactly $\binom{a}{i}$ terms of $σ_i$ involve only variables that appear in $x^α$.
Note that all of these terms have total degree $i$.
\item If $i ≤ a$, show that $x^α$ appears in $h_{j-i} σ_i$ with coefficient $\binom{a}{i}$.
Hint: This follows from part (b) because $h_{j-i}$ is the sum of all monomials of total degree $j - i$, and each monomial has coefficient $1$.
\item Conclude that the coefficient of $x^α$ in $\sum_{i=0}^j (-1)^i h_{j-i}σ_i$ is $\sum_{i=0}^a (-1)^i \binom{a}{i}$.
Then use the binomial theorem to show that the coefficient of $x^α$ is zero.
This will complete the proof of our identity.
\end{enumerate}
\end{ejercicio}
\begin{solucion}
\mbox{}
\begin{enumerate}
\item En cada término de $σ_i$ hay $i$ variables únicas.
En consecuencia, cada término de $h_{j-i} σ_i$ tiene al menos $i$ variables únicas.
Si $x^α$ es el monomio de uno de estos términos y tiene $a$ variables únicas, entonces necesariamente $i ≤ a$.
\item Combinatoria básica. Estamos eligiendo grupos de $i$ elementos entre $a$ opciones sin repetición ni relevancia de orden.
\item Sea $x^β$ uno de los $\binom{a}{i}$ monomios de $σ_i$ de los que hablamos en (b).
Tenemos que $|β|=i$, que $β$ está formado por $0$ y $1$, y que $α_k ≥ β_k$ para todo $k$.
Podemos tomar $γ = α-β$.
Tenemos que $|γ| = j-i$, luego $x^γ$ aparece en $h_{j-i}$.
Entonces $x^γ x^β = x^{γ+β} = x^α$ aparece en $h_{j-i} σ_i$.

En resumen, por cada uno de los $\binom{a}{i}$ monomios de $σ_i$, hay un único monomio de $h_{j-i}$ que al multiplicar nos da $x^α$.
Al sumarlos todos, el coeficiente de $x^α$ es $\binom{a}{i}$.
\item Por un lado sabemos que $x^α$ no aparecerá en los términos donde $i > a$ (por (a)).
Por otro lado, el coeficiente de $x^α$ en de cada término del sumatorio es $(-1)^i \binom{a}{i}$ (por (c)).
Entonces el coeficiente en el sumatorio es $\sum_{i=0}^a (-1)^i \binom{a}{i}$.

Sin embargo, por la fórmula del binomio
\[
\sum_{i=0}^a (-1)^i \binom{a}{i} = \sum_{i=0}^a \binom{a}{i} (-1)^i 1^{a-i} = (-1+1)^a = 0
\]
Luego el coeficiente de todo monomio $x^α$ es $0$, lo que demuestra la identidad.
\end{enumerate}
\end{solucion}

\newpage

\begin{ejercicio}{7.1.11}
In this exercise, we will prove the identity
\[ 0 = h_k(x_k, \dots, x_n) + \sum_{i=1}^k (-1)^i h_{k-i}(x_k,\dots,x_n) σ_i(x_1,\dots,x_n) \]
used in the proof of Proposition 5.
As in Exercise 10, let $σ_0 = 1$, so that the identity can be written more compactly as
\[ 0 = \sum_{i=0}^k (-1)^i h_{k-i}(x_k,\dots,x_n) σ_i(x_1,\dots,x_n) \]
The idea is to separate out the variables $x_1,\dots,x_{j-1}$.
TO this end, if $S \subset \{1,\dots,k-1\}$, let $x^S$ be the product of the coresponding variables and let $|S|$ denote the number of elements in $S$.
\begin{enumerate}
\item Prove that
\[ σ_i(x_1,\dots,x_n) = \sum_{S \subset \{1,\dots,k-1\} x^S σ_{i-|S|}}(x_k,\dots,x_n), \]
where we set $σ_m = 0$ if $m < 0$.
\item Prove that
\[ \sum_{i=0}^k (-1)^i h_{k-i}(x_k,\dots,x_n) σ_i(x_1,\dots,x_n) \]
\[ = \sum_{S \subset \{1,\dots,k-1\}} x^S \left(\sum_{i=|S|} (-1)^i h_{k-i}(x_k,\dots,x_n) σ_{i-|S|}(x_k,\dots,x_n)\right) \]
\item Use Exercise \ref{ejer:7.1.10} to conclude that the sum inside the parentheses is zero for every $S$.
This proves the desired identity.
Hint: Let $l = i-|S|$.
\end{enumerate}

\end{ejercicio}

\newpage

\begin{ejercicio}{7.1.14}
In this exercise, you will prove the Newton identities used in the proof of Theorem 8. Let the variables be $x_1,\dotsc,x_n$
\begin{enumerate}[a.]
\item Set $σ_0 = 1$ and $σ_i = 0$ whenever $i < 0$ or $i > n$. Then show that the Newton identities are equivalent to
$$
\sum_{k=0}^{j-1} (-1)^{k}\sigma_{k}s_{j-k} + (-1)^j j \sigma_j = 0, \quad j\geq 1 
$$
\item Prove the identity of part (a) by induction on $n$. Hint: Similar to Exercise 3, let $s^{(n)}_j$ be
the $j$th power sum of $x_1,\dotsc,x_{n-1}$, i.e., all variables except $x_n$. Then use Exercise 3 and note that $s_j = s^{(n)}_j+x_n^j$.
\end{enumerate} 
\end{ejercicio}
\begin{solucion}
\begin{enumerate}[a.]
\item[]
\item Si $1\leq j \leq n$ es claro, pues coincide con la identidad de Newton habitual. Si $j>n$ entonces tenemos que comprobar que
$$
\sum_{k=0}^{j-1} (-1)^{k}\sigma_{k}s_{j-k} + (-1)^j j \sigma_j  = \sum_{k=0}^n \sigma_k s_{j-k}(-1)^k
$$
Como $\sigma_j=0$ para $j>n$, entonces en la igualdad de la izquierda podemos eliminar el primer sumando y dejar la suma hasta $n$, donde tenemos la coincidencia.  
\item La igualdad $s_j = s_j^{(n)} + x^j_n$ es trivial, no resiste mayor análisis. Pasemos a probar $a.$ por inducción sobre $n$. Si $n = 1$ entonces $\forall j >1$ $\sigma_j =0$. La igualdad es clara para $j=$. Si $\forall j \geq 2$ todos los términos de la suma se anulan salvo los dos primeros (que se anulan), quedando
$$
\sum_{k=0}^{j-1} (-1)^{k}\sigma_{k}s_{j-k} = \sigma_0 s_j - \sigma_1 s_{j-1} = x^j - x x^{j-1} = 0
$$
Supongamos que para todo $n\leq s-1$ verifica la igualdad para todo $j$ y veamos qué ocurre para $n=s$. Si $j=1$ es trivial. Supongamos hipótesis de inducción para $j\leq m$.
\begin{align*}
\sum_{k=0}^{j-1} (-1)^{k}\sigma_{k}s_{j-k} + (-1)^j j \sigma_j   &= (-1)^{j-1}\sigma_{j-1}s_1 + (-1)^{j-1}(j-1)\sigma_{j-1}+(-1)^j j \sigma_j\\
&=(-1)^{j-1}\sigma_{j-1}(s_1 + (j-1))+(-1)^j j \sigma_j
\end{align*}
Si $j\geq n+2$ entonces el resultado es trivial. Si $j=n+1$ entonces nos queda, dado que $\sigma_{n+1}=0$, $(-1)^n\sigma_n (s_1+n)$
\end{enumerate}
\end{solucion}

\newpage
\begin{ejercicio}{7.1.16}
Consider the field $\F_2 = \{0, 1\}$ consisting of two elements. Show that it is impossible
to express the symmetric polynomial $xy ∈ \F_2[x, y]$ as a polynomial in $s_1$ and $s_2$ with
coefficients in $\F_2$. Hint: Show that $s_2 = s^2_1$!
\end{ejercicio}
\begin{solucion}
Por definición $s_1=x+y$, $s_2=x^2+y^2$. Como estamos en $\F_2$, $s_1^2=x^2+y^2=s_2$. Supongamos que $xy$ se expresa como un polinomio de grado $n$ en $s_1$ y $s_2$. Utilizando la observación anterior tendríamos que
\[
xy=a_1(x+y)+\sum_{i=1}^nb_i(x^2+y^2)^i+(x+y)\sum_{j=1}^{n-1}(x^2+y^2)^j,\quad a_1, b_i,c_j\in \F_2
\]
Observamos que en los dos sumandos de la derecha, tanto $x$ como $y$ aparecen siempre con grando mayor o igual que 2, mientras que en $xy$ tienen ambos grado 1. Por tanto, $xy=a_1(x+y)$, lo cual no es posible.
\end{solucion}
\end{document}
