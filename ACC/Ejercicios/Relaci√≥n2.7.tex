\documentclass[twoside]{article}
\usepackage{../../estilo-ejercicios}
\newcommand{\lex}{<_{lex}}
\newcommand{\grlex}{<_{grlex}}
\newcommand{\grevlex}{<_{grevlex}}

\newcommand{\PhantC}{\phantom{\colon}}%
\newcommand{\CenterInCol}[1]{\multicolumn{1}{c}{#1}}%

%--------------------------------------------------------
\begin{document}

\title{Ejercicios de Ideals, Varieties, and Algorithms (4ª Edición)}
\author{Diego Pedraza López, Javier Aguilar Martín, Rafael González López}
\maketitle

\begin{ejercicio}{2.7.1}
Check that $\overline{S( f_i , f_j )}^F = 0$ for all pairs $1 ≤ i < j ≤ 5$ in Example 1.
\end{ejercicio}

\begin{solucion}
Copio del libro:
\begin{align*}
f_1 & = x^3-2xy & f_2 & = x^2y-2y^2+x\\
f_3 & = -x^2 & f_4 & = -2xy\\
f_5 & = -2y^2+x & &
\end{align*}
Notemos que hay que realizar $(n-1)n/2$ comprobaciones (en este caso $n=5$). Obviamente no vamos a realizar 10 comprobaciones.

\[ S(f_1, f_3) = \frac{x^3}{x^3}f_1 - \frac{x^3}{-x^2}f_3 = -2xy\]
\[ -2xy = f_4
\]


\[ S(f_1, f_5) = \frac{x^3y^2}{x^3}f_1 - \frac{x^3y^2}{-2y^2}f_5 = y^2(x^3-2xy)+\frac{x^3}{2}(-2y^2+x) = -2xy^3+\frac{1}{2}x^4 = \frac{1}{2}x^4-2xy^3\]
\[ \frac{1}{2}x^4-2xy^3 = \frac{x}{2}f_1+f_2 -y^2f_4-f_5 
\]

\[ S(f_2, f_4) = \frac{x^2y}{x^2y}f_2 - \frac{x^2y}{-2xy}f_4 = -2y^2+x\]
\[ -2y^2+x = f_5
\]


\end{solucion}

\newpage

\begin{ejercicio}{2.7.2}
Use the algorithm given in Theorem 2 to find a Gröbner basis for each of the following
ideals. You may wish to use a computer algebra system to compute the $S$-polynomials
and remainders. Use the lex, then the grlex order in each case, and then compare your
results.
\begin{enumerate}[a.]
\item $I = \gene{x^2 y − 1, xy^2 − x}$ .
\item $I = \gene{x^2 + y, x^4 + 2x^2 y + y^2 + 3}$ . [What does your result indicate about the variety
$\V(I)$?]
\item $I = \gene{x − z^4 , y − z^5}$ .
\end{enumerate}
\end{ejercicio}
\begin{solucion}
\begin{enumerate}[a.]
\item[]
\item Distingamos según el orden que utilicemos. 
\begin{itemize}
\item Para el lex tenemos
\[ S(f_1, f_2) = \frac{x^2y^2}{x^2y}f_1 - \frac{x^2y^2}{xy^2}f_2 = x^2y^2-y-x(xy^2-x) = x^2-y\]
Como no es divisible por ninguno, añadimos $f_3=x^2-y$.
Calculamos
\[ S(f_1, f_3) = \frac{x^2y}{x^2y}f_1 - \frac{x^2y}{x^2}f_3 = -1+y^2\]
Como no es divisible, consideramos $f_4=y^2-1$. 
\[ S(f_2, f_3) = \frac{x^2y^2}{xy^2}f_2 - \frac{x^2y^2}{x^2}f_3 =-x^2+y^3 = -f_3 -yf_4\]
Comprobamos el resto de casos en los que no añadiremos más
$$
S(f_1,f_4) =  \frac{x^2y^2}{x^2y}f_1 - \frac{x^2y^2}{y^2}f_4 = -x^2+y^3 = -f_3 + yf_4
$$
$$
S(f_2,f_4) =  \frac{xy^2}{xy^2}f_2 + \frac{xy^2}{y^2}f_4 =-x+x = 0
$$
$$
S(f_3,f_4) =  \frac{x^2y^2}{x^2}f_3 - \frac{x^2y^2}{y^2}f_4 = -y^3+x^2 = f_3-yf_4 
$$
Por tanto, $G=\{y^2-1,x^2-y,x^2y-1,xy^2-x\}$.
\newpage
\item Pasamos a resolver con grlex. 
\[ S(f_1, f_2) = \frac{x^2y^2}{x^2y}f_1 - \frac{x^2y^2}{xy^2}f_2 = x^2y^2-y-x(xy^2-x) = x^2-y\]
Como no es divisible por ninguno, añadimos $f_3=x^2-y$.
\[ S(f_1, f_3) = \frac{x^2y}{x^2y}f_1 - \frac{x^2y}{x^2}f_3 = -1+y^2\]
Como no es divisible, consideramos $f_4=y^2-1$. 
\[ S(f_2, f_3) = \frac{x^2y^2}{xy^2}f_2 - \frac{x^2y^2}{x^2}f_3 =y^3-x^2 = yf_4 -f_3\\
\]
Análogamente comprobamos que el resto de la divisiones también son cero.
\end{itemize}
\item Distingamos los órdenes
\begin{itemize}
\item Para el orden lex, el resto de dividir $S(f_1,f_2)$ entre $\{f_1,f_2\}$ es $-3$, por lo que $I=k[x,y]$ y $\V(I)=\emptyset$.
\item El hecho de que $I=k[x,y]$ no cambia por que utilicemos otro orden para dividir. 
\end{itemize}
\item Distingamos los órdenes
\begin{itemize}
\item Comencemos con el orden lex
$$
S(f_1,f_2) = xz^5-yz^4 = z^5f_1 -z^4f_2
$$
Por lo tanto $I$ es una base de Groebner.
\item He comprobado en SAGE este caso y es estúpidamente largo, así que no lo voy a hacer.
\end{itemize}
\end{enumerate}
\end{solucion}
\newpage

\begin{ejercicio}{2.7.3}
Find reduced Gröbner bases for the ideals in Exercise \ref{ejer:2.7.2} with respect to lex and grlex.
\end{ejercicio}
\begin{solucion}
\begin{enumerate}[a.]
\item[]
\item $G=\{y^2-1,x^2-y,x^2y-1,xy^2-x\}$. Si consideramos los términos líderes, tenemos $y^2,x^2,x^2y,xy^2$, luego es claro que podemos desechar los dos útimos y quedarnos con $\{y^2-1,x^2-y\}$. Claramente se cumplen las condiciones de base reducida tanto para lex como para grlex.
\item[c.] Tenemos que $\{x-z^4,y-z^5\}$ es claramente una base de Grobner reducida. El caso para el orden grlex se deja como ejercicio para el lector.
\end{enumerate}
\end{solucion}

\newpage

\begin{ejercicio}{2.7.4}
Use the result of Exercise 7 of §4 to give an alternate proof that Buchberger’s algorithm
will always terminate after a finite number of steps.
\end{ejercicio}
\begin{solucion}
En la prueba, una vez que sabemos que $LT(r)\notin \gene{LT(G')}$ pero $LT(r)\in\gene{LT(G)}$, en lugar de usar la cadena ascendente, podemos notar simplemente que por el lema de Dikcson, $LT(G)$ está finitamente generado, y como en cada paso nos mantenemos dentro de $LT(G)$ pero añadiendo polinomios que no están en el $LT(G')$ resultante, el algoritmo debe parar en un número finito de pasos.
\end{solucion}
\newpage

\begin{ejercicio}{2.7.5}
Let $G$ be a Gröbner basis of an ideal $I$ with the property that $LC (g) = 1$ for all $g ∈ G$.
Prove that $G$ is a minimal Gröbner basis if and only if no proper subset of $G$ is a Gröbner
basis of $I$.
\end{ejercicio}
\begin{solucion}
La definición de base de Gröbner minimal es que $LC (g) = 1$ para todo $g ∈ G$  y no existe $p\in G$ tal que $LT(p)\in\gene{LT(G\setminus\{p\})}$, o equivalentemente, $LC (g) = 1$ para todo $g ∈ G$ y para todo $p\in G$, $LT(p)\notin \gene{LT(G\setminus\{p\})}$.

$\boxed{\Rightarrow}$ Sea $G=\{g_1,\dots, g_t\}$ y supongamos que existe un subconjunto propio $G'\subsetneq G$ que es base de Gröbner de $I$. Podemos suponer sin pérdida de generalidad que $G'=\{g_2,\dots, g_t\}$. Por ser base de Gröbner, $\gene{LT(G')}=LT(I)=\gene{LT(G)}$. Pero entonces, $g_1\in G$ y $LT(g_1)\in \gene{LT(G')}=\gene{LT(G\setminus\{g_1\})}$, lo cual contradice la minimalidad.

$\boxed{\Leftarrow}$ Supongamos que $G$ no es minimal. Entonces existe $p\in G$ con $LT(p)\in\gene{LT(G\setminus\{p\})}$. Si llamamos $G'=G\setminus\{p\}$, entonces $LT(I)=\gene{LT(G)}=\gene{LT(G')}$, lo cual contradice que ningún subconjunto propio de $G$ sea base de Gröbner.
\end{solucion}

\newpage

\begin{ejercicio}{2.7.6}
The minimal basis of a monomial ideal was introduced in Proposition 7 of §4. Show that
a Gröbner basis $G$ of $I$ is minimal if and only if $LC (g) = 1$ for all $g ∈ G$ and $LT (G)$ is
the minimal basis of the monomial ideal $LT (I)$.
\end{ejercicio}
\begin{solucion}
$G$ es una base de Gröbner minimal si y solo si $LC (g) = 1$ para todo $g ∈ G$  y para todo $p\in G$, $LT(p)\notin \gene{LT(G\setminus\{p\})}$. Así que ya tenemos por definición $LC(g)=1$ para todo $g\in G$. La otra parte, por el Lemma 2 de §2, es equivalente a que para todo $p\in G$, $LT(p)$ no sea divisible por ningún $LT(g)$ con $g\in G$. Esta es precisamente la definición de ser base minimal de un ideal monomial, luego se tienen ambas implicaciones.
\end{solucion}

\newpage

\begin{ejercicio}{2.7.7}
Fix a monomial order, and let $G$ and $\tilde{G}$ be minimal Gröbner bases for the ideal $I$.
\begin{enumerate}[a.]
\item Prove that $LT (G) = LT (\tilde{G})$.
\item Conclude that $G$ and $\tilde{G}$ have the same number of elements.
\end{enumerate}
\end{ejercicio}
\begin{solucion}\
\begin{enumerate}[a.]
\item Por el ejercicio anterior (\ref{ejer:2.7.6}) sabemos que $LT(G)$ y $LT(\tilde{G})$ son bases minimales de $LT(I)$, pero por la Proposición 7 de §4, esta base minimal es única, así que necesariamente $LT (G) = LT (\tilde{G})$.
\item El cardinal de $LT(G)$ es igual al de $G$, así que es trivial. 
\end{enumerate}
\end{solucion}

\newpage

\begin{ejercicio}{2.7.8}
Develop an algorithm that produces a reduced Gröbner basis (see Definition 4) for an
ideal $I$, given as input an arbitrary Gröbner basis for $I$. Prove that your algorithm works.
\end{ejercicio}
\begin{solucion}
\begin{enumerate}
\item[]
\item Entrada: $G=\{g_1,\dotsc,g_t\}$ base de Gröbner.
\item Dividimos por los términos líderes para que cada polinomio sea mónico. 
\item Utilizando el Lema 3, comprobamos si $LT(g_i)\in \gene{LT(G\setminus\{g_i\}}$. Si está, removemos $g_i$ de la base, en caso contrario la dejamos.
\item Procedemos como en la demostración de la Proposición 6, es decir, intercambiamos cada $g_i$ que permanezca en nuestro conjunto por $\overline{g_i}^{G\setminus \{g_i\}}$.
\end{enumerate}
La demostración de la Proposición 6 garantiza que el algoritmo funciona.
\end{solucion}

\newpage

\begin{ejercicio}{2.7.9}
Consider the ideal
\[
I = \gene{3x − 6y − 2z, 2x − 4y + 4w, x − 2y − z − w} ⊆ k[x, y, z, w]
\]
mentioned in the text. We will use lex order with $x > y > z > w$.
\begin{enumerate}
\item Show that the linear polynomials determined by the row echelon matrix (3) give a
minimal Gröbner basis $I =  \gene{x − 2y − z − w, z + 3w}$ . Hint: Use Theorem 6 of §6.
\item Show that the linear polynomials from the reduced row echelon matrix (4) give the
reduced Gröbner basis $I = \gene{x − 2y + 2w, z + 3w}$.
\end{enumerate}
\end{ejercicio}
\begin{solucion}
\begin{enumerate}
\item[]
\item Tenemos que una base $G=\{f_1,f_2\}=\{x-2y-z-w,z+3w\}$. ¿Es una base de Groebner? Calculamos
$$
S(f_1,f_2) = -3xw-2yz-z^2-zw = -3wf_1 -(2y+z+w)f_2
$$
Por tanto $G$ es base de Grobner. Además, es minimal pues $x\notin \gene{z}$ y $z\notin \gene{x}$. Sin embargo, es claro que no es reducida, $z$ es un monomio de $f_1$ que está en $\gene{z}$. 
\item Como los términos líderes coinciden con los de la base de Gröbner anteriormente calculada, tenemos que esta también es una base de Gröbner. Además, es claramente reducida pues $x,y,w\notin \gene{z}$ y $z,w\notin\gene{x}$.
\end{enumerate}
\end{solucion}

\newpage

\begin{ejercicio}{2.7.10}
Let $A = (a_{ij} )$ be an $n × m$ matrix with entries in $k$ and let $f_i = a_{i1} x_1 +\cdots + a_{im} x_m$ be the
linear polynomials in $k[x_1 ,\dots, x_m ]$ determined by the rows of $A$. Then we get the ideal
$I = \gene{f_1 ,\dots , f_n}$ . We will use lex order with $x_1 > \cdots > x_m$ . Now let $B = (b_{ij} )$ be the
reduced row echelon matrix determined by $A$ and let $g_1 ,\dots , g_t$ be the linear polynomials
coming from the nonzero rows of $B$ (so that $t ≤ n$). We want to prove that $g_1 ,\dots , g_t$
form the reduced Gröbner basis of $I$.
\begin{enumerate}[a.]
\item Show that $I =\gene{ g_1 ,\dots , g_t}$ . Hint: Show that the result of applying a row operation to
$A$ gives a matrix whose rows generate the same ideal.
\item Use Theorem 6 of §6 to show that $g_1 , \dots, g_t$ form a Gröbner basis of $I$. Hint: If the
leading 1 in the ith row of $B$ is in the $s$-th column, we can write $g_i = x_s + C$, where $C$
is a linear polynomial involving none of the variables corresponding to leading 1’s.
If $g_j = x_t + D$ is written similarly, then you need to divide $S(g_i , g_j ) = x_t C − x_s D$ by
$g_1 ,\dots , g_t$. Note that you will use only $g_i$ and $g_j$ in the division.
\item Explain why $g_1 ,\dots , g_t$ form the reduced Gröbner basis of $I$.
\end{enumerate}
\end{ejercicio}
\begin{solucion}
\begin{enumerate}[a.]
\item Aplicar operaciones por filas no es más que hacer combinaciones lineales de los $f_i$, luego claramente $\gene{f_1,\dots, f_n}\supseteq \gene{ g_1 ,\dots , g_t}$. Para la inclusión contraria basta recordar que estas operaciones son reversibles, pues se corresponden con el producto por una matriz regular, así que el mismo razonamiento sirve para probar  $\gene{f_1,\dots, f_n}\subseteq \gene{ g_1 ,\dots , g_t}$.

\item Usaremos que $G=\{g_1 , \dots, g_t\}$ forma una base de Gröbner si y solo si el resto de dividir $S(g_i,g_j)$ por $G$ es 0 para todo $i\neq j$. Al estar en forma row echelon podemos escribir $g_i=x_s+C$ donde $C$ es un polinomio en variables que no son las correspondientes a ninguno de los 1's pivotes de la matriz, y análogamente $g_j=x_r+D$. Entonces $\lcm(LM(g_i),LM(g_j))=x_sx_r$, de modo que 
\[
S(g_i,g_j)=\frac{x_sx_r}{x_s}(x_s+C)-\frac{x_sx_r}{x_r}(x_r+D)=x_rx_s +x_rC-x_sx_r-x_sD=x_rC-x_sD
\]
Al dividir $x_rC-x_sD$ por $G$ solo intervienen $g_i$ y $g_j$ por construcción. De hecho, $x_rC-x_sD=-Dg_i+Cg_j$, por lo que el resto será 0.

\item En primer lugar, $LC(p)=1$ para todo $p\in G$. Por otra parte, dado $p=x_i+C$ como antes, como ni $x_i$ ni ninguna variable de $C$ coincide con ninguno de los términos líderes del resto de polinomios de $G$, no puede ocurrir que un monomio de $p$ esté en $\gene{LT(G\setminus\{p\})}$. 
\end{enumerate}
\end{solucion}

\newpage

\begin{ejercicio}{2.7.11}
Show that the result of applying the Euclidean Algorithm in $k[x]$ to any pair of polynomials $f , g$ is a reduced Gröbner basis for $\gene{f , g}$ (after dividing by a constant to make the leading coefficient equal to 1). Explain how the steps of the Euclidean Algorithm can be seen as special cases of the operations used in Buchberger’s algorithm.
\end{ejercicio}
\begin{solucion}
Sabemos que aplicando sucesivamente el algoritmo de división euclídea obtenemos el máximo común divisor $h$ de $f$ y $g$ que es precisamente una base de $\gene{f,g}$ como probamos en la Sección 1.5. Podemos suponer que $LC(h)=1$. Sea $p$ el grado de $h$. Como vimos en la demostración de la Proposición 6 de la Seccion 1.5, $h$ es un elemento de grado mínimo en $I$, luego es una base de Gröbner, $LT(h)$ divide a $LT(q)$ para cualquier otro $q\in I$. Por tanto, $\gene{LT(h)}=\gene{x^p}=\gene{LT(I)}$.  Cualquier monomio de $h$ distinto de su término líder tiene, por definición, grado menor que $p$, luego no puede pertenecer a $\gene{x^p}$, pues $x^p$ no puede dividirlo.
\end{solucion}

\newpage

\begin{ejercicio}{2.7.12}
Fix $F = \{ f_1 , \dots , f_s \}$ and let $r = f$. Since dividing $f$ by $F$ gives $r$ as remainder, adding
$r$ to the polynomials we divide by should reduce the remainder to zero. In other words,
we should have $\overline{f}^{F∪\{r\}}
= 0$ when $r$ comes last. Prove this as follows.
\begin{enumerate}[a.]
\item When you divide $f$ by $F ∪\{r\}$, consider the first place in the division algorithm where
the intermediate dividend $p$ is not divisible by any $LT ( f_i )$. Explain why $LT (p) =
LT (r)$ and why the next intermediate dividend is $p − r$.
\item From here on in the division algorithm, explain why the leading term of the inter-
mediate dividend is always divisible by one of the $LT ( f_i )$. Hint: If this were false,
consider the first time it fails. Remember that the terms of $r$ are not divisible by any
$LT ( f_i )$.
\item Conclude that the remainder is zero, as desired.
\item (For readers who did Exercise 11 of §3.) Give an alternate proof of $\overline{f}^{F∪\{r\}}
= 0$ using
Exercise 11 of §3.
\end{enumerate}
\end{ejercicio}
\begin{solucion}
\begin{enumerate}[a.]
\item Al dividir $f$ por $F$, que el resto sea $r$ significa que en el momento en que ningún $LT(f_i)$ divide al dividendo intermedio, $LT(p)=LT(r)$, ya que a partir de ese momento el término líder del resto no se altera y los demás términos del resto son de grado menor. Por tanto, al incluir $r$, pasaríamos a dividir $p$ por $r$, y como $LT(p)=LT(r)$, multiplicaríamos $r$ por 1 y restaríamos, es decir, haríamos $p-r$. 
\item
\item
\item 
\end{enumerate}
\end{solucion}

\newpage

\begin{ejercicio}{2.7.13}
In the discussion following the proof of Theorem 2, we commented that if $\overline{S( f , g)}^{G'} = 0$,
then remainder stays zero when we enlarge $G'$. More generally, if $\overline{f}^F = 0$ and $F'$ is
obtained from $F$ by adding elements at the end, then $\overline{f}^{F'} = 0$. Prove this.
\end{ejercicio}
\begin{solucion}
Como $\overline{f}^F = 0$ no hay necesidad de añadir ningún término al resto al dividir, luego menos aún si añadimos algún polinomio.
\end{solucion}

\newpage

\begin{ejercicio}{2.7.14}
Suppose we have $n$ points $V = \{(a_1 , b_1 ), \dots , (a_n , b_n )\} ⊆ k^2$ where $a_1 , \dots , a_n$ are
distinct. This exercise will study the Lagrange interpolation polynomial defined by
\[
h(x) =\sum^n_{i=1}b_i\prod_{i\neq j}
\frac{ x − a_j}{a_i − a_j}
∈ k[x].
\]


We will also explain how $h(x)$ relates to the reduced Gröbner basis of $\I(V) ⊆ k[x, y]$.
\begin{enumerate}[a.]
\item Show that $h(a_i ) = b_i$ for $i = 1, \dots , n$ and explain why $h$ has degree $≤ n − 1$.
\item Prove that $h(x)$ is the unique polynomial of degree $≤ n − 1$ satisfying $h(a_i ) = b_i$ for
$i = 1,\dots , n$.
\item Prove that $\I(V) = \gene{f (x), y − h(x)}$ , where $f (x) =\prod_{i=1}^n (x − a_i )$. Hint: Divide
$g ∈ \I(V)$ by $f (x), y − h(x)$ using lex order with $y > x$.
\item Prove that $\{ f (x), y − h(x)\}$ is the reduced Gröbner basis for $\I(V) ⊆ k[x, y]$ for lex
order with $y > x$.
\end{enumerate}
\end{ejercicio}
\begin{solucion}
\begin{enumerate}[a.]
\item Sea $a_k$ con $1\leq k\leq n$. Entonces el producto $\prod_{i\neq j}
\frac{ x − a_j}{a_i − a_j}$ vale 1 cuando $k=i$ y 0 cuando $k\neq i$. Así que en la suma solo nos queda $b_i$, por lo que $h(a_i)=b_i$. Además $h$ tiene grado menor o igual que $n-1$ porque en cada producto hay $n-1$ factores y la suma de polinomios no aumenta el grado. 

\item Si hubiera un polinomio $f$ de grado menor o igual que $n-1$ con la misma propiedad, $h-f$ sería un polinomio de grado a lo sumo $n-1$ que se anula en $n$ puntos $a_1,\dots, a_n$. Por el teorema fundamental del álgebra, $h-f=0$, luego $h=f$. 
\item
\item 
\end{enumerate}
\end{solucion}


\end{document}
