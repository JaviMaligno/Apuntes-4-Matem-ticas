\documentclass[twoside]{article}
\usepackage{../../estilo-ejercicios}
\newcommand{\lex}{<_{lex}}
\newcommand{\grlex}{<_{grlex}}
\newcommand{\grevlex}{<_{grevlex}}

\newcommand{\PhantC}{\phantom{\colon}}%
\newcommand{\CenterInCol}[1]{\multicolumn{1}{c}{#1}}%

%--------------------------------------------------------
\begin{document}

\title{Ejercicios de Ideals, Varieties, and Algorithms (4ª Edición)}
\author{Diego Pedraza López, Javier Aguilar Martín, Rafael González López}
\maketitle

\begin{ejercicio}{2.7.1}
Check that $\overline{S( f_i , f_j )}^F = 0$ for all pairs $1 ≤ i < j ≤ 5$ in Example 1.
\end{ejercicio}

\begin{solucion}

\end{solucion}

\newpage

\begin{ejercicio}{2.7.2}
Use the algorithm given in Theorem 2 to find a Gröbner basis for each of the following
ideals. You may wish to use a computer algebra system to compute the $S$-polynomials
and remainders. Use the lex, then the grlex order in each case, and then compare your
results.
\begin{enumerate}[a.]
\item $I = \gene{x^2 y − 1, xy^2 − x}$ .
\item $I = \gene{x^2 + y, x^4 + 2x^2 y + y^2 + 3}$ . [What does your result indicate about the variety
\V(I)?]
\item $I = \gene{x − z^4 , y − z^5}$ .
\end{enumerate}
\begin{solucion}

\end{solucion}
\newpage

\begin{ejercicio}{2.7.3}
Find reduced Gröbner bases for the ideals in Exercise \ref{ejer:2.7.2} with respect to lex and grlex.
\end{ejercicio}
\begin{solucion}

\end{solucion}

\newpage

\begin{ejercicio}{2.7.4}
Use the result of Exercise 7 of §4 to give an alternate proof that Buchberger’s algorithm
will always terminate after a finite number of steps.
\end{ejercicio}
\begin{solucion}

\end{solucion}
\newpage

\begin{ejercicio}{2.7.5}
Let $G$ be a Gröbner basis of an ideal $I$ with the property that $LC (g) = 1$ for all $g ∈ G$.
Prove that $G$ is a minimal Gröbner basis if and only if no proper subset of $G$ is a Gröbner
basis of $I$.
\end{ejercicio}
\begin{solucion}

\end{solucion}

\newpage

\begin{ejercicio}{2.7.6}
The minimal basis of a monomial ideal was introduced in Proposition 7 of §4. Show that
a Gröbner basis $G$ of $I$ is minimal if and only if $LC (g) = 1$ for all $g ∈ G$ and $LT (G)$ is
the minimal basis of the monomial ideal $LT (I)$.
\end{ejercicio}
\begin{solucion}

\end{solucion}

\newpage

\begin{ejercicio}{2.7.7}
Fix a monomial order, and let $G$ and $\tilde{G}$ be minimal Gröbner bases for the ideal $I$.
\begin{enumerate}[a.]
\item Prove that $LT (G) = LT (\tilde{G})$.
\item Conclude that $G$ and have the same number of elements.
\end{enumerate}
\end{ejercicio}
\begin{solucion}

\end{solucion}

\newpage

\begin{ejercicio}{2.7.8}
Develop an algorithm that produces a reduced Gröbner basis (see Definition 4) for an
ideal $I$, given as input an arbitrary Gröbner basis for $I$. Prove that your algorithm works.
\end{ejercicio}
\begin{solucion}

\end{solucion}

\newpage

\begin{ejercicio}{2.7.9}
Consider the ideal
\[
I = \gene{3x − 6y − 2z, 2x − 4y + 4w, x − 2y − z − w} ⊆ k[x, y, z, w]
\]
mentioned in the text. We will use lex order with $x > y > z > w$.
\begin{enumerate}
\item Show that the linear polynomials determined by the row echelon matrix (3) give a
minimal Gröbner basis $I =  \gene{x − 2y − z − w, z + 3w}$ . Hint: Use Theorem 6 of §6.
\item Show that the linear polynomials from the reduced row echelon matrix (4) give the
reduced Gröbner basis $I = \gene{x − 2y + 2w, z + 3w}$.
\end{ejercicio}
\begin{solucion}

\end{solucion}

\newpage

\begin{ejercicio}{2.7.10}
Let $A = (a_{ij} )$ be an $n × m$ matrix with entries in $k$ and let $f_i = a_{i1} x_1 +\cdots + a_{im} x_m$ be the
linear polynomials in $k[x_1 ,\dots, x_m ]$ determined by the rows of $A$. Then we get the ideal
$I = \gene{f_1 ,\dots , f_n}$ . We will use lex order with $x_1 > \cdots > x_m$ . Now let $B = (b_{ij} )$ be the
reduced row echelon matrix determined by $A$ and let $g_1 ,\dots , g_t$ be the linear polynomials
coming from the nonzero rows of $B$ (so that $t ≤ n$). We want to prove that $g_1 ,\dots , g_t$
form the reduced Gröbner basis of $I$.
\begin{enumerate}[a.]
\item Show that $I =\gene{ g_1 ,\dots , g_t}$ . Hint: Show that the result of applying a row operation to
$A$ gives a matrix whose rows generate the same ideal.
\item Use Theorem 6 of §6 to show that $g_1 , \dots, g_t$ form a Gröbner basis of $I$. Hint: If the
leading 1 in the ith row of $B$ is in the $s$-th column, we can write $g_i = x_s + C$, where $C$
is a linear polynomial involving none of the variables corresponding to leading 1’s.
If $g_j = x_t + D$ is written similarly, then you need to divide $S(g_i , g_j ) = x_t C − x_s D$ by
$g_1 ,\dots , g_t$. Note that you will use only $g_i$ and $g_j$ in the division.
\item Explain why $g_1 ,\dots , g_t$ form the reduced Gröbner basis of $I$.
\end{enumerate}
\end{ejercicio}
\begin{solucion}
\end{solucion}

\newpage

\begin{ejercicio}{2.7.11}
Show that the result of applying the Euclidean Algorithm in $k[x]$ to any pair of polyno-
mials $f , g$ is a reduced Gröbner basis for $f , g$ (after dividing by a constant to make the
leading coefficient equal to 1). Explain how the steps of the Euclidean Algorithm can be
seen as special cases of the operations used in Buchberger’s algorithm.
\end{ejercicio}
\begin{solucion}

\end{solucion}

\newpage

\begin{ejercicio}{2.7.12}
Fix $F = \{ f_1 , \dots , f_s \}$ and let $r = f$. Since dividing $f$ by $F$ gives $r$ as remainder, adding
$r$ to the polynomials we divide by should reduce the remainder to zero. In other words,
we should have $\overline{f}^{F∪\{r\}}
= 0$ when $r$ comes last. Prove this as follows.
\begin{enumerate}[a.]
\item When you divide $f$ by $F ∪\{r\}$, consider the first place in the division algorithm where
the intermediate dividend $p$ is not divisible by any $LT ( f_i )$. Explain why $LT (p) =
LT (r)$ and why the next intermediate dividend is $p − r$.
\item From here on in the division algorithm, explain why the leading term of the inter-
mediate dividend is always divisible by one of the $LT ( f_i )$. Hint: If this were false,
consider the first time it fails. Remember that the terms of $r$ are not divisible by any
$LT ( f_i )$.
\item Conclude that the remainder is zero, as desired.
\item (For readers who did Exercise 11 of §3.) Give an alternate proof of $\overline{f}^{F∪\{r\}}
= 0$ using
Exercise 11 of §3.
\end{ejercicio}
\begin{solucion}

\end{solucion}

\newpage

\begin{ejercicio}{2.7.13}
In the discussion following the proof of Theorem 2, we commented that if $\overline{S( f , g)}^{G'} = 0$,
then remainder stays zero when we enlarge $G'$. More generally, if $\overline{f}^F = 0$ and $F'$ is
obtained from $F$ by adding elements at the end, then $f = 0$. Prove this.
\end{ejercicio}
\begin{solucion}

\end{solucion}

\newpage

\begin{ejercicio}{2.7.14}
Suppose we have $n$ points $V = \{(a_1 , b_1 ), \dots , (a_n , b_n )\} ⊆ k^2$ where $a_1 , \dots , a_n$ are
distinct. This exercise will study the Lagrange interpolation polynomial defined by
\[
h(x) =\sum^n_{i=1}b_i\prod_{i\neq j}
\frac{ x_j − a_j}{a_i − a_j}
∈ k[x].
\]


We will also explain how $h(x)$ relates to the reduced Gröbner basis of $\I(V) ⊆ k[x, y]$.
\begin{enumerate}[a.]
\item Show that $h(a_i ) = b_i$ for $i = 1, \dots , n$ and explain why $h$ has degree $≤ n − 1$.
\item Prove that $h(x)$ is the unique polynomial of degree $≤ n − 1$ satisfying $h(a_i ) = b_i$ for
$i = 1,\dots , n$.
\item Prove that $\I(V) = \gene{f (x), y − h(x)}$ , where $f (x) =\prod_{i=1}^n (x − a_i )$. Hint: Divide
$g ∈ \I(V)$ by $f (x), y − h(x)$ using lex order with $y > x$.
\item Prove that $\{ f (x), y − h(x)\}$ is the reduced Gröbner basis for $\I(V) ⊆ k[x, y]$ for lex
order with $y > x$.
\end{ejercicio}
\begin{solucion}

\end{solucion}


\end{document}
