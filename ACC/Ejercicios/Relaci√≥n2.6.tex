\documentclass[twoside]{article}
\usepackage{../../estilo-ejercicios}
\newcommand{\lex}{<_{lex}}
\newcommand{\grlex}{<_{grlex}}
\newcommand{\grevlex}{<_{grevlex}}

\newcommand{\PhantC}{\phantom{\colon}}%
\newcommand{\CenterInCol}[1]{\multicolumn{1}{c}{#1}}%

%--------------------------------------------------------
\begin{document}

\title{Ejercicios de Ideals, Varieties, and Algorithms (4ª Edición)}
\author{Diego Pedraza López, Javier Aguilar Martín, Rafael González López}
\maketitle

\begin{ejercicio}{2.6.1}
Show that Proposition 1 can be strengthened slightly as follows. Fix a monomial ordering
and let $I ⊆ k[x_1,\dots , x_n]$ be an ideal. Suppose that $f ∈ k[x_1,\dots , x_n]$.
\begin{enumerate}
\item[a.] Show that $f$ can be written in the form $f = g + r$, where $g ∈ I$ and no term of $r$ is
divisible by any element of $LT(I)$.
\item[b.] Given two expressions $f = g + r = g'
+ r'$ as in part (a), prove that $r = r'$ Thus, $r$
and $g$ are uniquely determined.
\end{enumerate}
\end{ejercicio}

\begin{solucion}
\begin{enumerate}
\item[]
\item[a.] Sea $\{g_1,\dotsc,g_t\}$ una base de Gröbner. Por la Proposición 1 tenemos que $f$ puede escribirse como $f=g+r$ con $g\in I$ y $r$, resto de la división, no es divisible por ningún $LT(g_i)$. Sea $h\in \gene{LT(I)}=\gene{LT(g_1),\dotsc,LT(g_t)}$. Luego, por ser ideal monomial, $\exists i \in\{1,\dotsc,t\}$ tal que $LT(g_i)$ divide a $LT(h)$, por lo que si $h$ divide a $r$, también lo haría $g_i$, lo cuál es una contradicción con el hecho de que $r$ sea resto de la división. 

\item[b.] Este apartado se prueba igual que en el libro, puesto que basta que ninguno de los términos de $r$ y $r'$ sea divisibles por ninguno de los términos líderes de los generadores.
\end{enumerate}
\end{solucion}

\newpage

\begin{ejercicio}{2.6.2}
In §5, we showed that $G = \{x + z, y − z\}$ is a Gröbner basis for lex order. Let us use
this basis to study the uniqueness of the division algorithm.
\begin{enumerate}[a.]
\item Divide $xy$ by $x + z$, $y − z$.
\item Now interchange the two polynomials and divide $xy$ by $y − z$, $x + z$.
\end{enumerate}
\end{ejercicio}
\begin{solucion}
Vamos a fijar el orden lex habitual
\begin{enumerate}[a.]
\item $xy=y(x+z)-z(y-z)-z^2.$
\item $xy=x(y-z)+z(x+z)-z^2.$
\end{enumerate}
\end{solucion}
\newpage

\begin{ejercicio}{2.6.3}
In Corollary 2, we showed that if $I =\gene{ 
g_1,\dots , g_t}$ and if $G = \{g_1,\dots , g_t\}$ is a Gröbner
basis for $I$, then $\overline{f}^G = 0$ for all $f ∈ I$. Prove the converse of this statement. Namely, show
that if $G$ is a basis for $I$ with the property that $\overline{f}^G = 0$ for all $f ∈ I$, then $G$ is a Gröbner
basis for I.

\end{ejercicio}
\begin{solucion}
Sea $G$ una base de $I$ tal que $\overline{f}^G=0$ para todo $f\in I$. Supongamos que no es una base de Gröbner, es decir, que existe $f\in I$ con $LT(f)\notin \gene{LT(G)}$. En particular, por el Ejercicio 2.5.5, $LT(f)$ no es divisible por $LT(g_i)$ para ningún $i$. Pero entonces $\overline{f}^G \neq 0$. Contradicción.
\end{solucion}

\newpage

\begin{ejercicio}{2.6.4}
Let $G$ and $G'$ be Gröbner bases for an ideal $I$ with respect to the same monomial order
in $k[x_1,\dots , x_n]$. Show that $\overline{f}^G = \overline{f}^{G'}$
for all $f ∈ k[x_1,\dots , x_n]$. Hence, the remainder on
division by a Gröbner basis is even independent of which Gröbner basis we use, as long
as we use one particular monomial order. Hint: See Exercise \ref{ejer:2.6.1}.
\end{ejercicio}
\begin{solucion}
Es consecuencia del Ejercicio \ref{ejer:2.6.1}, pues al dividir por $G$ tendríamos $f=g+r$ y al dividir por $G'$ tendríamos $f=g'+r'$, con $g,g',r,r'$ cumpliendo las condiciones de dicho ejercicio. Pero por el apartado b, $r=r'$, luego, usando la notación del enunciado, $\overline{f}^G = \overline{f}^{G'}$.
\end{solucion}
\newpage

\begin{ejercicio}{2.6.5}
Compute $S( f , g)$ using the lex order.
\begin{enumerate}[a.]
\item $f = 4x^2z − 7y^2$, $g = xyz^2 + 3xz^4$.
\item $f = x^4y − z^2$, $g = 3xz^2 − y$.
\item $f = x^7y^2z + 2ixyz$, $g = 2x^7y^2z + 4$.
\item $f = xy + z^3$, $g = z^2 − 3z$.
\end{enumerate}
\end{ejercicio}
\begin{solucion}\mbox{}
\begin{enumerate}
\item \[ S(f,g) = \frac{x^2yz^2}{4x^2z}f - \frac{x^2yz^2}{xyz^2}g = \frac{1}{4}yz(4x^2z-7y^2)-x(xyz^2+3xz^4) =  -3x^2z^4 - \frac{7}{4}y^3z. \]
\item \[ S(f,g) = \frac{x^4yz^2}{x^4y}f - \frac{x^4yz^2}{3xz^2}g = z^2f-\frac{1}{3}x^3yg = \frac{1}{3}x^3y^2-z^4 \]
\item \[ S(f,g) = \frac{x^7y^2z}{x^7y^2z}f-\frac{x^7y^2z}{2x^7y^2z}g = f- \frac{1}{2}g = 2ixyz-2\]
\item \[ \frac{xyz^2}{xy}f-\frac{xyz^2}{z^2}g = z^2f-xyg = 3xyz+z^5 \]
\end{enumerate}
\end{solucion}

\newpage

\begin{ejercicio}{2.6.6}
Does $S( f , g)$ depend on which monomial order is used? Illustrate your assertion with
examples.
\end{ejercicio}
\begin{solucion}
Sí depende. Sean $f=x+y$ y $g=x+y^2$, entonces con lex siendo $x>y$ obtenemos $S(f,g)=-y^2+y$, mientras que si $y>x$ nos da $S(f,g)=yx-x$.
\end{solucion}

\newpage

\begin{ejercicio}{2.6.7}
Prove that $\mathrm{multideg}(S( f , g)) < γ$, where $x^γ = \lcm (LM( f ), LM(g))$. Explain why this
inequality is a precise version of the claim that $S$-polynomials are designed to produce
cancellation.
\end{ejercicio}
\begin{solucion}
Por definición, $$S(f,g)=f\frac{x^{\gamma}}{LT(f)}-g\frac{x^{\gamma}}{LT(g)}$$ 
Tenemos que $$LT\left(f\frac{x^{\gamma}}{LT(f)}\right)=LT(f)\frac{x^{\gamma}}{LT(f)}=x^{\gamma}=LT(g)\frac{x^{\gamma}}{LT(g)}=LT\left(g\frac{x^{\gamma}}{LT(g)}\right)$$
Por tanto, el término de multigrado $\gamma$ se cancela en $S(f,g)$. Como los demás términos tienen grado estrictamente menor, se tiene el resultado.
\end{solucion}

\newpage

\begin{ejercicio}{2.6.8}
As in the proof of Theorem 6, suppose that $c_ix^{α(i)}g_i$ and $c_jx^{α(j)}g_j$ have multidegree $δ$.
Prove that
$$S(x^{α(i)}g_i, x^{α(j)}g_j) = x^{δ−γ_{ij}}S(g_i, g_j),$$
where $x^{γ_{ij}} = \lcm (LM(g_i), LM(g_j))$.
\end{ejercicio}
\begin{solucion}
En primer, lugar, tenemos $LT(x^{\alpha(i)}g_i)=x^{\alpha(i)}LT(g_i)$ puesto que los órdenes monomiales preservan la suma, y del mismo modo $LM(x^{\alpha(i)}g_i)=x^{\alpha(i)}LM(g_i)$. Así que si denotamos $x^{\gamma}=\lcm (x^{α(i)}LM(g_i),x^{α(j)}LM(g_j))$, entonces 
\begin{align*}S(x^{α(i)}g_i, x^{α(j)}g_j)&=\frac{x^{\gamma}}{x^{\alpha(i)}LT(g_i)}x^{α(i)}g_i-\frac{x^{\gamma}}{x^{\alpha(j)}LT(g_j)}x^{α(j)}g_j\\
&=\frac{x^{\gamma}}{LT(g_i)}g_i-\frac{x^{\gamma}}{LT(g_j)}g_j\\
&=x^{\gamma-\gamma_{ij}}\left(\frac{x^{\gamma_{ij}}}{LT(g_i)}g_i-\frac{x^{\gamma_{ij}}}{LT(g_j)}g_j\right)\\
&=x^{\gamma-\gamma_{ij}} S(g_i,g_j)
\end{align*}
Es fácil ver que $\gamma-\gamma_{ij}\geq 0$. Por tanto, tenemos que probar que $\gamma-\gamma_{ij}=\delta-\gamma_{ij}$, es decir, que $\gamma = \delta$.  Esto se deduce de que $\delta$ es el multigrado de ambos polinomios y si $f$ y $g$ tienen multigrado $\nu$ entonces $\lcm (f,g)$ también tiene multigrado $\nu$.
\end{solucion}

\newpage

\begin{ejercicio}{2.6.9}
Show that $\{y − x^2, z − x^3\}$ is not a Gröbner basis for lex order with $x > y > z$.
\end{ejercicio}
\begin{solucion}
Denotando $f=y − x^2,g=z − x^3$ tenemos que $\gene{LT(f),LT(g)}=\gene{-x^2,-x^3}=\gene{x^2}$. Sea $h=g-xf=z-y\in I$, pero $LT(h)=-y\notin\gene{LT(f),LT(g)}$.
\end{solucion}

\newpage

\begin{ejercicio}{2.6.10}
Using Theorem 6, determine whether the following sets $G$ are Gröbner bases for the
ideal they generate. You may want to use a computer algebra system to compute the
$S$-polynomials and remainders.
\begin{enumerate}[a.]
\item $G = \{x^2 − y, x^3 − z\}$ for grlex order.
\item $G = \{x^2 − y, x^3 − z\}$ for invlex order (see Exercise 6 of §2).
\item $G = \{xy^2 − xz + y, xy − z^2, x − yz^4\}$ for lex order.
\end{enumerate}
\end{ejercicio}
\begin{solucion}
Denotando en cada caso $f_i$ al i-ésimo elemento de $G$.
\begin{enumerate}[a.]
\item Tenemos que $S(f_1,f_2)=yx$, que no es divisible por $LT(f_i)$. 
\item Es trivial que sí lo es, pues invlex order equivale al lex order con $z>y>x$.
\item Tenemos que $S(f_2,f_3)=-z^2+y^2z^4$, que no es divisible por $LT(f_i)$.
\end{enumerate}
\end{solucion}

\newpage

\begin{ejercicio}{2.6.11}
Let $f , g ∈ k[x_1,\dots , x_n]$ be polynomials such that $LM( f )$ and $LM(g)$ are relatively prime
monomials and $LC( f) = LC(g) = 1$. Assume that $f$ or $g$ has at least two terms.
\begin{enumerate}
\item Show that $S( f , g) = −(g − LT(g))f + (f − LT( f ))g$.
\item Deduce that $S( f , g) \neq 0$ and that the leading monomial of $S( f , g)$ is a multiple of
either $LM( f )$ or $LM(g)$ in this case.
\end{enumerate}
\end{ejercicio}
\begin{solucion}
\begin{enumerate}
\item[]
\item Sabemos que, por las hipótesis del enunciado, $x^\gamma=\lcm(f,g)=LT(f)LT(g)$. Por definición
$$S(f,g)= S=f\frac{x^{\gamma}}{LT(f)}-g\frac{x^{\gamma}}{LT(g)} = LT(g)f-LT(f)g
$$
Si sumamos y restamos $fg$ en la expresión anterior, tenemos
\begin{align*}
S(f,g)&=-gf+LT(g)f +fg-LT(f)g\\
&=(-g+LT(g))f+(f-LT(f))g \\
&=-(g-LT(g))f+(f-LT(f))g
\end{align*}
\item Sabemos que $f$ o $g$ tiene al menos dos términos. Supongamos que uno de los dos tiene un único término. Podemos suponer que es $g$ sin pérdida de generalidad, entonces
$$
S=S(f,g)=-(g-LT(g))f+(f-LT(f))g = (f-LT(f))g 
$$
Luego $S\neq 0$ porque $g\neq 0$ y $LT(S)$ es múltiplo de $LT(g)$. El otro caso es análogo. Supongamos ahora que ambos tienen dos términos al menos. 
% Sean 
%$$
%f^* = LT(f-LT(f)) \qquad g^* = LT(g-LT(g))
%$$
En ese caso, $g=LT(g)+r$ y $f=LT(f)+s$ con $r,s\in k[x_1,\dots, x_n]$ no nulos con $multideg(r)<multideg(g)$ y $multideg(s)<multideg(f)$. Tenemos del apartado anterior que $S(f,g)=LT(g)f-LT(f)g=LT(g)s-LT(f)r\neq 0$. Esto se tiene debido a que para anularse, el polinomio de la izquierda debe tener los mismos monomios que el de la derecha. En particular, $LT(f)$ debería dividir a $LT(g)s$. Como $LT(g)$ es primo con $LT(f)$ esto implicaría que $LT(f)$ divide a $s$, lo cual no es posible porque $multideg(s)<multideg(f)$. 

Por otro lado, si el monomio líder está en $LT(g)f$ o en $LT(f)g$ entonces claramente el monomio líder es múltiplo de $LM(g)$ o de $LM(f)$. Supongamos entonces que estuviera en una combinación de $LT(g)s-LT(f)r$. En concreto, existirían $s',r'$ monomios de $s$ y $r$ respectivamente tales que $multideg(LT(g)s')=multideg(LT(f)r')$. Esto no es posible siguiendo un razonamiento análogo al utilizado para ver que $S\neq 0$, ya que si dos monomios tienen el mismo multigrado, entonces son iguales salvo constante. 

\end{enumerate}
\end{solucion}

\newpage

\begin{ejercicio}{2.6.12}
Let $f , g ∈ k[x_1,\dots , x_n]$ be nonzero and $x^α, x^β$ be monomials. Verify that
$$S(x^αf , x^βg) = x^γS( f , g)$$
where
$$x^γ =
\frac{\lcm (x^αLM( f ), x^βLM(g))}{\lcm (LM( f ), LM(g))}
.$$
Be sure to prove that $x^{γ}$ is a monomial. Also explain how this relates to Exercise \ref{ejer:2.6.8}.
\end{ejercicio}
\begin{solucion}
Es fácil probar que el multigrado  de $\lcm(x^αLM( f ), x^βLM(g))$ es mayor que el $LM(f)$ y $LM(g)$ (con cualquier orden monomial), de donde se puede deducir que es mayor que el $\lcm$ de ambos. Se tiene además que
\begin{align*}
x^\gamma S(f,g)&=\frac{\lcm (x^αLM( f ), x^βLM(g))}{\lcm (LM( f ), LM(g))}\left(\frac{\lcm (LM( f ), LM(g))}{LT(f)}f-\frac{\lcm (LM( f ), LM(g))}{LT(g)} g\right)\\
&=\frac{\lcm (x^αLM( f ), x^βLM(g))}{LT(f)}f-\frac{\mathrm{lcm}(x^αLM( f ), x^βLM(g))}{LT(g)}g\\
&=\frac{\lcm (x^αLM( f ), x^βLM(g))}{LT(f)x^\alpha}fx^\alpha-\frac{\mathrm{lcm}(x^αLM( f ), x^βLM(g))}{LT(g)x^\beta}g
x^\beta\\
&=\frac{\lcm (x^αLM( f ), x^βLM(g))}{LT(fx^\alpha)}fx^\alpha-\frac{\mathrm{lcm}(x^αLM( f ), x^βLM(g))}{LT(gx^\beta)}g
x^\beta\\
&=S(x^\alpha f, x^\beta g)
\end{align*}
\end{solucion}

\newpage

\begin{ejercicio}{2.6.13}
Let $I ⊆ k[x_1,\dots , x_n]$ be an ideal, and let $G$ be a Gröbner basis of $I$.
\begin{enumerate}[a.]
\item Show that $\overline{f}^G = \overline{g}^G$ if and only if $f − g ∈ I$. Hint: See Exercise \ref{ejer:2.6.1}.
\item Use Exercise \ref{ejer:2.6.1} to show that
$$\overline{f + g}^G = \overline{f}^G + \overline{g}^G.$$
\item Deduce that
$$\overline{fg}^G = \overline{\overline{f}^G \cdot  \overline{g}^G}^G
.$$
We will return to an interesting consequence of these facts in Chapter 5.
\end{enumerate}
\end{ejercicio}
\begin{solucion}\
\begin{enumerate}[a.]
\item Fijemos un orden monomial y sean $f=h+r$ y $g=h'+r'$ las descomposiciones obtenidas según la Proposición 1. 
\begin{itemize}
\item Si los restos coinciden, entonces $f-g=h-h'$, pero $h,h'\in I$, luego $h-h'\in I$. 
\item Si $f-g\in I$ entonces $f-g=f-g+0 = h-h' + r-r'$. Por unicidad de descomposición según la Proposición 1 se tiene el resultado.
\end{itemize} 
\item Es suficiente utilizar la Proposición 1. Consideremos las descomposiciones $f=h+r$ y $g=h+r'$. Naturalmente, $f+g=h+h'+r+r'$. Se tiene que $h+h'\in I$. Además, si $f+g\neq 0$ (en cuyo caso se tendría el resultado de manera trivial), 
$$
multideg(f+g)\leq \max\{multideg(r),multideg(r')\}<multideg(g) \quad \forall g \in G
$$
De donde deducimos que $LT(g)$ no divide a $LT(r-r')$ para todo $g\in G$. Por la unicidad que garantiza la Proposición 1 tenemos el resultado.
\item Tenemos que $fg = hh'+hr'+h'r+rr'$. Sabemos que $hh'+hr'+h'r = p \in I$. Sin embargo, es posible que $rr'$ no cumpla primera condición de la Proposición 1, en ese caso escribimos $rr'=p'+r''$, según dicha proposición. Tomando $q=p+p'$ tenemos que $fg=q+r''$, ahora sí en las condiciones de la Proposición 1. Por la unicidad de $r''$ tenemos el resultado.
\end{enumerate}
\end{solucion}



\end{document}
