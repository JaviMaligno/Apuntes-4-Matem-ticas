\documentclass[twoside]{article}
\usepackage{../../estilo-ejercicios}
\newcommand{\lex}{<_{lex}}
\newcommand{\grlex}{<_{grlex}}
\newcommand{\grevlex}{<_{grevlex}}

\newcommand{\PhantC}{\phantom{\colon}}%
\newcommand{\CenterInCol}[1]{\multicolumn{1}{c}{#1}}%

%--------------------------------------------------------
\begin{document}

\title{Ejercicios de Ideals, Varieties, and Algorithms (4ª Edición)}
\author{Diego Pedraza López, Javier Aguilar Martín, Rafael González López}
\maketitle

\begin{ejercicio}{2.6.1}
Show that Proposition 1 can be strengthened slightly as follows. Fix a monomial ordering
and let $I ⊆ k[x_1,\dots , x_n]$ be an ideal. Suppose that $f ∈ k[x_1,\dots , x_n]$.
\begin{enumerate}
\item[a.] Show that $f$ can be written in the form $f = g + r$, where $g ∈ I$ and no term of $r$ is
divisible by any element of $LT(I)$.
\item[b.] Given two expressions $f = g + r = g'
+ r'$ as in part (a), prove that $r = r'$ Thus, $r$
and $g$ are uniquely determined.
\end{enumerate}
\end{ejercicio}

\begin{solucion}
\begin{enumerate}
\item[a.] Si algún término de $r$ fuera divisible por algún elemento $h=h_1LT(g_1)+\dots+h_tLT(g_t)$, en particular sería divisible por cualquiera de los $LT(g_i)$, lo cual contradice la Proposición 1.

\item[b.] Este apartado se prueba igual que en el libro, puesto que basta que ninguno de los términos de $r$ y $r'$ sea divisibles por ninguno de los términos líderes de los generadores.
\end{enumerate}
\end{solucion}

\newpage

\begin{ejercicio}{2.6.2}
In §5, we showed that $G = \{x + z, y − z\}$ is a Gröbner basis for lex order. Let us use
this basis to study the uniqueness of the division algorithm.
\begin{enumerate}[a.]
\item Divide $xy$ by $x + z$, $y − z$.
\item Now interchange the two polynomials and divide $xy$ by $y − z$, $x + z$.
\end{enumerate}
\end{ejercicio}
\begin{solucion}
Vamos a fijar el orden lex habitual
\begin{enumerate}[a.]
\item $xy=y(x+z)-z(y-z)-z^2.$
\item $xy=x(y-z)+z(x+z)-z^2.$
\end{enumerate}
\end{solucion}
\newpage

\begin{ejercicio}{2.6.3}
In Corollary 2, we showed that if $I =\gene{ 
g_1,\dots , g_t}$ and if $G = \{g_1,\dots , g_t\}$ is a Gröbner
basis for $I$, then $\bar{f}^G = 0$ for all $f ∈ I$. Prove the converse of this statement. Namely, show
that if $G$ is a basis for $I$ with the property that $\bar{f}^G = 0$ for all $f ∈ I$, then $G$ is a Gröbner
basis for I.

\end{ejercicio}
\begin{solucion}
Supongamos que existe $f\in I$ con $LT(f)\notin LT(G)$. En particular, $LT(f)$ no es divisible por $LT(g_i)$ para ningún $i$. Pero entonces $\bar{f}^G \neq 0$. Contradicción.
\end{solucion}

\newpage

\begin{ejercicio}{2.6.4}
Let $G$ and $G'$ be Gröbner bases for an ideal $I$ with respect to the same monomial order
in $k[x_1,\dots , x_n]$. Show that $\bar{f}^G = \bar{f}^{G'}$
for all $f ∈ k[x_1,\dots , x_n]$. Hence, the remainder on
division by a Gröbner basis is even independent of which Gröbner basis we use, as long
as we use one particular monomial order. Hint: See Exercise \ref{ejer:2.6.1}.
\end{ejercicio}
\begin{solucion}
Es consecuencia de \ref{ejer:2.6.1} pues al dividir por $G$ tendríamos $f=g+r$ y al dividir por $G'$ tendríamos $f=g'+r'$, con $g,g',r,r'$ cumpliendo las condiciones de dicho ejercicio. Pero por el apartado b, $r=r'$, luego, usando la notación del enunciado, $\bar{f}^G = \bar{f}^{G'}$.
\end{solucion}
\newpage

\begin{ejercicio}{2.6.5}
Compute $S( f , g)$ using the lex order.
\begin{enumerate}[a.]
\item $f = 4x^2z − 7y^2$, $g = xyz^2 + 3xz^4$.
\item $f = x^4y − z^2$, $g = 3xz^2 − y$.
\item $f = x^7y^2z + 2ixyz$, $g = 2x^7y^2z + 4$.
\item $f = xy + z^3$, $g = z^2 − 3z$.
\end{enumerate}
\end{ejercicio}
\begin{solucion}

\begin{enumerate}[a.]
\item $S(f,g)=3xz^2-y.$
\item $S(f,g)=\frac{1}{3}x^3y^2 - z^4.$
\item $S(f,g)=2ixyz+z^2.$
\item $S(f,g)=3xyz+z^5.$
\end{enumerate}
\end{solucion}

\newpage

\begin{ejercicio}{2.6.6}
Does $S( f , g)$ depend on which monomial order is used? Illustrate your assertion with
examples.
\end{ejercicio}
\begin{solucion}
Sí depende. Sean $f=x+y$ y $g=x+y^2$, entonces con lex siendo $x>y$ obtenemos $S(f,g)=-y^2+y$, mientras que si $y>x$ nos da $S(f,g)=yx-x$.
\end{solucion}

\newpage

\begin{ejercicio}{2.6.7}
Prove that $\mathrm{multideg}(S( f , g)) < γ$, where $x^γ = \mathrm{lcm}(LM( f ), LM(g))$. Explain why this
inequality is a precise version of the claim that $S$-polynomials are designed to produce
cancellation.
\end{ejercicio}
\begin{solucion}
Por definición, $S(f,g)=f\frac{x^{\gamma}}{LT(f)}-g\frac{x^{\gamma}}{LT(g)}$. Tenemos que $LT\left(f\frac{x^{\gamma}}{LT(f)}\right)=LT(f)\frac{x^{\gamma}}{LT(f)}=x^{\gamma}=LT(g)\frac{x^{\gamma}}{LT(g)}=LT\left(g\frac{x^{\gamma}}{LT(g)}\right)$. Por tanto, el término de multigrado $\gamma$ se cancela en $S(f,g)$. Como los demás términos tienen grado estríctamente menor, tenemos lo que buscábamos.
\end{solucion}

\newpage

\begin{ejercicio}{2.6.8}
As in the proof of Theorem 6, suppose that $c_ix^{α(i)}g_i$ and $c_jx^{α(j)}g_j$ have multidegree $δ$.
Prove that
$$S(x^{α(i)}g_i, x^{α(j)}g_j) = x^{δ−γ_{ij}}S(g_i, g_j),$$
where $x^{γ_{ij}} = \mathrm{lcm}(LM(g_i), LM(g_j))$.
\end{ejercicio}
\begin{solucion}

\end{solucion}

\newpage

\begin{ejercicio}{2.6.9}
Show that $\{y − x^2, z − x^3\}$ is not a Gröbner basis for lex order with $x > y > z$.
\end{ejercicio}
\begin{solucion}
Denotando $f=y − x^2,g=z − x^3$ tenemos que $\gene{LT(f),LT(g)}=\gene{-x^2,-x^3}=\gene{x^2}$. Sea $h=g-xf=z-y\in I$, pero $LT(h)=-y\notin\gene{LT(f),LT(g)}$.
\end{solucion}

\newpage

\begin{ejercicio}{2.6.10}
Using Theorem 6, determine whether the following sets $G$ are Gröbner bases for the
ideal they generate. You may want to use a computer algebra system to compute the
$S$-polynomials and remainders.
\begin{enumerate}[a.]
\item $G = \{x^2 − y, x^3 − z\}$ for grlex order.
\item $G = \{x^2 − y, x^3 − z\}$ for invlex order (see Exercise 6 of §2).
\item $G = \{xy^2 − xz + y, xy − z^2, x − yz^4\}$ for lex order.
\end{enumerate}
\end{ejercicio}
\begin{solucion}\
\begin{enumerate}[a.]
\item No lo es, pues $LT(G)=\gene{x^2}$, mientras que para $f=-xy+xz$, $LT(f)=-xy\notin LT(G)$. 
\item 
\item 
\end{enumerate}
\end{solucion}

\newpage

\begin{ejercicio}{2.6.11}
Let $f , g ∈ k[x_1,\dots , x_n]$ be polynomials such that $LM( f )$ and $LM(g)$ are relatively prime
monomials and $LC( f) = LC(g) = 1$. Assume that $f$ or $g$ has at least two terms.
\begin{enumerate}
\item Show that $S( f , g) = −(g − LT(g))f + (f − LT( f ))g$.
\item Deduce that $S( f , g) \neq 0$ and that the leading monomial of $S( f , g)$ is a multiple of
either $LM( f )$ or $LM(g)$ in this case.
\end{enumerate}
\end{ejercicio}
\begin{solucion}

\end{solucion}

\newpage

\begin{ejercicio}{2.6.12}
Let $f , g ∈ k[x_1,\dots , x_n]$ be nonzero and $x^α, x^β$ be monomials. Verify that
$$S(x^αf , x^βg) = x^γS( f , g)$$
where
$$x^γ =
\mathrm{lcm}(x^αLM( f ), x^βLM(g))
\mathrm{lcm}(LM( f ), LM(g))
.$$
Be sure to prove that xγ is a monomial. Also explain how this relates to Exercise \ref{ejer:2.6.8}.
\end{ejercicio}
\begin{solucion}

\end{solucion}

\newpage

\begin{ejercicio}{2.6.13}
Let $I ⊆ k[x_1,\dots , x_n]$ be an ideal, and let $G$ be a Gröbner basis of $I$.
\begin{enumerate}[a.]
\item Show that $\bar{f}^G = \bar{g}^G$ if and only if $f − g ∈ I$. Hint: See Exercise \ref{ejer:2.6.1}.
\item Use Exercise \ref{ejer:2.6.1} to show that
$$\overline{f + g}^G = \bar{f}^G + \bar{g}^G.$$
\item Deduce that
$$\overline{fg}^G = \overline{\bar{f}^G  \bar{g}^G}^G
.$$
We will return to an interesting consequence of these facts in Chapter 5.
\end{enumerate}
\end{ejercicio}
\begin{solucion}\
\begin{enumerate}[a.]
\item Sean $f=h+r$ y $g=h'+r'$ tras dividir ambos polinomios por $G$. $f-g=(h-h')+(r-r')$. Si $r=r'$ entonces $f-g\in I$ puesto que $h,h'\in I$. Si $f-g\in I$, entonces $(r-r')=(f-g)-(h-h')\in I$, luego $LT(r-r')\in LT(I)$, pero ninguno de los términos de $r$ ni de $r'$ es divisible por ningún elemento de $G$, que es una base de Gröbner, luego $r-r'=0$. 
\item
\item
\end{enumerate}
\end{solucion}



\end{document}
