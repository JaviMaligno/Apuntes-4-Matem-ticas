\documentclass[twoside]{article}
\usepackage{../../estilo-ejercicios}

%--------------------------------------------------------
\begin{document}

\title{Ejercicios de Ideals, Varieties, and Algorithms (4ª Edición)}
\author{Diego Pedraza López, Javier Aguilar Martín, Rafael González López}
\maketitle

\begin{ejercicio}{7.3.1}\
\begin{enumerate}[a.]
\item Explain why $1 ∈ k[x^2]$ but $1 \not∈ 
\gene{x^2}$.
\item Explain why $x^3 \not∈ k[x^2]$ but $x^3 ∈ 
\gene{x^2}$.
\end{enumerate}
\end{ejercicio}
\begin{solucion}\
\begin{enumerate}[a.]
\item $1 ∈ k[x^2]$ por definición de anillo de polinomios, ya que $1\in k$. $1 \not∈ 
\gene{x^2}$ porque eso implicaría que existe un polinomio $f\in k[x]$ de modo que $f(x)x^2=1$, con lo que $x^2$ sería unidad, lo cual no es cierto.
\item Los polinomios de $k[x^2]$ son expresiones finitas de la forma $\sum_i a_i(x^2)^i$, por lo que todos los términos tienen grado par, por lo que $x^3 \not∈ k[x^2]$. Sin embargo, $x^3=x\cdot x^2$, por lo que $x^3 ∈ 
x^2$.
\end{enumerate}
\end{solucion}

\newpage

\begin{ejercicio}{7.3.2}
Let $G$ be a finite matrix group in $GL(n, k)$. Prove that the Reynolds operator $R_G$ has the
following properties:
\begin{enumerate}[a.]
\item If $a, b ∈ k$ and $f , g ∈ k[x_1, \dots, x_n]$, then $R_G(af + bg) = aR_G( f) + bR_G(g)$.
\item $R_G$ maps $k[x_1, \dots, x_n]$ to $k[x_1, \dots , x_n]^G$ and is onto.
\item $R_G \circ R_G = R_G$.
\item If $f ∈ k[x_1, \dots, x_n]^G$ and $g ∈ k[x_1, \dots , x_n]$, then $R_G( fg) = f  R_G(g)$.
\end{enumerate}
\end{ejercicio}
\begin{solucion}
Tengamos presenta la definición del operador de Reynolds
\[
R_G( f )(x) =
\frac{1}{|G|}\sum_{A∈G}f (A \cdot x)
\]
\begin{enumerate}[a.]
\item 
\begin{align*}
R_G( af + bg )(x) &= \frac{1}{|G|}\sum_{A∈G}(af + bg) (A \cdot x)\\
&=a\frac{1}{|G|}\sum_{A∈G}f (A \cdot x)+b\frac{1}{|G|}\sum_{A∈G}g (A \cdot x)\\
&=aR_G( f)(x) + bR_G(g)(x)
\end{align*}

\item Por un lado, $\forall A \in G$
$$
R_G((f(Ax)) = \frac{1}{|G|}\sum_{B \in G} f(BAx) = \frac{1}{|G|}\sum_{B \in G} f(Bx) = \frac{1}{|G|}\sum_{B \in G} f(x) = R_G(f(x))
$$
Si $f\in k[x_1,\dotsc,x_n]^G$ entonces $R_G(f)=f$, por lo que basta tomar como preimagen de cada $f$ a ella misma.
\item Por el apartado anterior, $R_G$ envía $k[x_1, \dots, x_n]$ en $k[x_1, \dots, x_n]^G$ y es la identidad sobre este último, por lo que se tiene el resultado.
\item Vemos directamente que 
\[
R_G( fg )(x) =
\frac{1}{|G|}\sum_{A∈G}f (A \cdot x)g(A\cdot x)=f(x)\sum_{A∈G}g(A\cdot x)=f(x)R_G(g)(x).
\]

\end{enumerate}
\end{solucion}

\newpage

\begin{ejercicio}{7.3.3}
In this exercise, we will work with the cyclic group $C_4 ⊆ GL(2, k)$ from Example 4 in
the text.
\begin{enumerate}[a.]
\item Prove that the Reynolds operator of $C_4$ is given by
\[
R_{C_4} ( f )(x, y) =
\frac{1}{4}
( f (x, y) + f (−y, x) +f (−x,−y) + f (y,−x)).
\]
\item Compute $R_{C_4}(x^iy^j)$ for all $i + j ≤ 4$. Note that some of the computations are done in
Example 4. You can check your answers against the table in Example 6.
\end{enumerate}
\end{ejercicio}
\begin{solucion}\
\begin{enumerate}[a.]
\item El grupo $C_4$ está formado por las matrices
\[
\begin{pmatrix}
0 & -1\\
1 & 0
\end{pmatrix}, \begin{pmatrix}
-1 & 0\\
0 & -1
\end{pmatrix},\begin{pmatrix}
0 & 1\\
-1 & 0
\end{pmatrix}, \begin{pmatrix}
0 & 1\\
1 & 0
\end{pmatrix}.
\]
Con esto y con la definición del operador de Reynolds el resultado se sigue inmediatamente.
\item Están hechos en la página 368 del libro.
\end{enumerate}

\end{solucion}


\newpage
\begin{comment}
\begin{ejercicio}{7.3.4}
We will use
the multinomial coefficients, which are defined as follows. For $α = (α_1, \dots , α_n) ∈ \Z^n_{≥0}$,
let $|α| = m$ and define 
\[
\binom{m}{α}
=
\frac{m!}{α_1!α_2! \cdots α_n!}.
\]
\begin{enumerate}[a.]
\item Prove that $\binom{m}{\alpha}$
is an integer. Hint: Use induction on $n$ and note that when $n = 2$, $\binom{m}{\alpha}$
is a binomial coefficient.

\item Prove that
\[
(x_1 + \cdots + x_n)^m =
\sum_{|α|=m}\binom{m}{α}x^α.
\]
In particular, the coefficient $a_α$ in equation (2) is the positive integer $\binom{m}{α}$. Hint: Use
induction on $n$ and note that the case $n = 2$ is the binomial theorem.
\end{enumerate}
\end{ejercicio}
\begin{solucion}\
\begin{enumerate}[a.]
\item Para $n=1$ tendríamos el número 1. Para $n=2$ se comprueban los casos $(1,3)$ y $(2,2)$. Es claro que si se cumple para $\alpha$ se cumple para $reverse(\alpha)$.  Supongámoslo cierto para $n-1$, es decir, es cierto para $|\alpha|=\alpha_1+\cdots\alpha_{n-1}=m$. Entonces, dado $\alpha_n\in\Z$,
\[
\binom{m+\alpha_n}{\alpha\cup \alpha_n}=\frac{(m+\alpha_n)!}{α_1! \cdots \alpha_{n-1}!α_n!}=\frac{(m+\alpha_n)\cdots(m+1)}{\alpha_n!}\frac{m!}{α_1! \cdots \alpha_{n-1}!}=\A\binom{m}{\alpha}\in\Z
\]
Tenemos que el producto de una sucesión de $k$ enteros es divisible por $k!$, por lo que $A\in \N$.
\item Vamos a probarlo para un $m$ arbitrario para dos casos. Para $n=1$
$$
x^m = \binom{m}{m}x^m = \sum_{|\alpha|}x^\alpha
$$
Para $n=2$ tenemos, por el binomio de Newton
$$
(x_1+x_2)^m = \sum_{k=0}^m \binom{m}{k}x_1^kx_2^{m-k} = \sum_{k=0}^m \binom{n}{(n-k,k)}(x_1,x_2)^{(n-k,k)}
$$
Para $m=1$ y $n$ arbitrario también es fácil de comprobar. Por doble inducción, supongamos que se cumple para $m-1$ y $n-1$. Supongamos la hipótesis de inducción para $n-1$. Entonces para $n$ tenemos
\begin{align*}
(x_1+\dotsc+x_n)^m &= \sum_{k=0}^m \binom{m}{k}(x_1+\dotsc x_{n-1})^kx_n^{m-k}\\
&= \sum_{k=0}^m \binom{m}{k}(x_1+\dotsc x_{n-1})^kx_n^{m-k} \\
&=(x_1+\dotsc+x_{n-1})^m + \sum_{k=0}^{m-1} \sum_{|\beta|=k}(x_1,\dotsc,x_{n-1})^\beta x^{m-k}\\
&=(x_1+\dotsc+x_{n-1})^m + \sum_{k=0}^{m-1} \sum_{|\beta|=k}\binom{m}{\beta}(x_1,\dotsc,x_{n-1})^\beta x^{m-k}\\
&=(x_1+\dotsc+x_{n-1})^m + \sum_{k=0}^{m-1} \sum_{|\beta|=k}\binom{m}{(\beta,m-k)}(x_1,\dotsc,x_{n-1},x_n)^{(\beta,m-k)}
\end{align*}

\end{enumerate}
\end{solucion}
\newpage
\end{comment}
\begin{ejercicio}{7.3.5}
Let $G ⊆ GL(n, k)$ be a finite matrix group. In this exercise, we will give Hilbert’s proof
that $k[x_1, \dots , x_n]^G$ is generated by finitely many homogeneous invariants. To begin the
argument, let $I ⊆ k[x_1, \dots , x_n]$ be the ideal generated by all homogeneous invariants of
positive total degree.
\end{ejercicio}

\begin{enumerate}[a.]
\item Explain why there are finitely many homogeneous invariants $f_1, \dots , f_m$ such that
$I = 
 \gene{f_1, \dots , f_m}$. The strategy of Hilbert’s proof is to show that $k[x_1, \dots , x_n]^G =
k[ f_1, \dots , f_m]$. Since the inclusion $k[ f_1, \dots , f_m] ⊆ k[x_1, \dots , x_n]^G$ is obvious, we must
show that $k[x_1, \dots x_n]^G \not⊆ k[ f_1, \dots , f_m]$ leads to a contradiction.

\item Prove that $k[x_1, \dots , x_n]^G \not⊆ k[ f_1, \dots , f_m]$ implies there is a homogeneous invariant $f$
of positive degree which is not in $k[ f_1, \dots , f_m]$.

\item For the rest of the proof, pick $f$ as in part (b) with minimal total degree $d$. By definition, $f ∈ I$, so that $f = \sum^m_{i=1} h_i f_i$ for $h_1, \dots , h_m ∈ k[x_1, \dots , x_n]$. Prove that for each
$i$, we can assume that $h_i f_i$ is either 0 or homogeneous of total degree $d$.

\item Use the Reynolds operator to show that $f =\sum^m_{i=1} R_G(h_i) f_i$. Hint: Use Proposition 3
and Exercise \ref{ejer:7.3.2}. Also show that for each $i$, $R_G(h_i) f_i$ is either 0 or homogeneous of
total degree $d$.

\item Since $f_i$ has positive total degree, conclude that $R_G(h_i)$ is a homogeneous invariant of
total degree $< d$. By the minimality of $d$, $R_G(h_i) ∈ k[ f_1, \dots, f_m]$ for all $i$. Prove that
this contradicts $f \not∈ k[ f_1, \dots , f_m]$.
\end{enumerate}
\begin{solucion}
\begin{enumerate}[a.]
\item[]
\item Sabemos por ACGA que $I$ es un ideal homogéneo y que, por ser $k[x_1,\dotsc,x_n]$ noetheriano, existe un conjunto finito de polinomios homogéneos tales que $I=\gene{f_1,\dotsc,f_m}$ [No encuentro la prueba].
\item Dado que las constantes siempre están contenidas, si no se da la contención del enunciado es porque existe $f$ no constante tal que $f\in k[x_1,\dotsc,x_n]^g$ que no está en $k[f_1,\dotsc,f_m]$.
\item Consideremos con grado total minimal $d$. Si descomponemos $h_i = \sum_{j} h_{ij}$ en sus componentes homogéneas, entonces
$$
f = \sum_i f_i h_i = \sum_i f_i \sum_j  h_{ij}
$$
Como $f$ es homogéneo de grado $d$, para cada $f_i$ existe a lo sumo un $h_{ij}$ tal que el producto tenga grado $d$. Después de agrupar los $f_i h_{ij}$ por grado, los que no sean de grado $d$ se anularán. Por tanto, si denotamos por $m_i$ al $h_{ij}$ tal que $f_i h_{ij}=d$ (o $0$ en caso de no existir), podemos escribir
$$
f = \sum_i f_i m_i 
$$
con $f_i m_i$ es $0$ o de grado total $d$.

\item Se deduce de que lo siguiente. Como $f$ está en el ideal, entonces también es invariante, luego $R_G(f)=f$. Además, el operador de Reynolds es aditivo, por lo que $R_G$ de la suma, es la suma de las $R_G$. Terminamos la prueba usando que si $f_i$ es invariante, entonces $R_G(h_if_i)=R_G(h_i)f_i$. $R_G(h_i)f_i$ es $0$ si $h_i f_i = 0$ (si $h_i = 0$). En otro caso, como $R_G(h_i)$ no cambia el grado total $h_i$, se tiene el resultado.
\item Sabemos que $f$ tiene grado mínimo entre los invariantes que no están en $k[f_1,\dotsc,f_n]$. Como podemos suponer que $R_G(h_i)f_i$ tiene grado $d$, necesariamente, si no es un término nulo, $R_G(h_i)$ tiene grado total $<d$ (pues el de $f_i$ es no nulo). Como $R_G(h_i)$ es invariante y tiene grado $<d$, implica que $R_G(h_i)\in k[f_1,\dotsc,f_m]$ $\forall i$. Entonces $R_G(h_i) = g_i(f_1,\dotsc,f_m)$, luego
$$
f = \sum_{i=1}^m f_i g_i(f_1,\dotsc,f_m) = p(f_1,\dotsc,f_m) \in k[f_1,\dotsc,f_m]
$$
De donde se sigue la contradicción.
\end{enumerate}
\end{solucion}
\newpage

\begin{ejercicio}{7.3.6}
If we have two finite matrix groups $G$ and $H$ such that $G ⊆ H ⊆ GL(n, k)$, prove that
$k[x_1, \dots , x_n]^H ⊆ k[x_1, \dots , x_n]^G$.
\end{ejercicio}
\begin{solucion}
Dado $f\in k[x_1, \dots , x_n]^H$, $f(Ax)=f(x)$ para toda $A\in H$. En particular, para $A\in G$, por lo que se cumple el resultado.
\end{solucion}
\newpage

\begin{ejercicio}{7.3.7}
Consider the matrix
$$A =
\begin{pmatrix}
0 &−1\\
1 &−1
\end{pmatrix}∈ GL(2, k).$$
\begin{enumerate}[a.]
\item Show that $A$ generates a cyclic matrix group $C_3$ of order 3.
\item Use Theorem 5 to find finitely many homogeneous invariants which generate $k[x, y]^{C_3}$.
\item Can you find fewer invariants that generate $k[x, y]^{C_3}$? Hint: If you have invariants
$f_1, \dots , f_m$, you can use Proposition 7 to determine whether $f_1 ∈ k[ f_2, \dots , f_m]$.
\end{enumerate}
\end{ejercicio}
\begin{solucion}\
\begin{enumerate}[a.]
\item 
\[
A^2=\begin{pmatrix}
-1 & 1\\
-1 & 0
\end{pmatrix}, A^3=I.
\]
\item Como el grupo tiene orden 3, tenemos que calcular $R_{C_3}(x^iy^j)$ para todo $1\leq i+j\leq 3$.
\begin{align*}
R_{C_3}(x)=\frac{1}{3}(-y+y-x+x) \Rightarrow  &0\\
R_{C_3}(x^2)=\frac{1}{3}((-y)^2+(y-x)^2+x^2)\Rightarrow & 2x^2-2xy+2y^2\\
R_{C_3}(x^3)=\frac{1}{3}(-y^3+(y-x)^3+x^3)\Rightarrow & 3x^2y-3y^2\\
R_{C_3}(y)=\frac{1}{3}(x-y-x+y)\Rightarrow & 0\\
R_{C_3}(y^2)=\frac{1}{3}((x-y)^2+x^2+y^2)\Rightarrow & 2x^2+2y^2-2xy\\
R_{C_3}(y^3)=\frac{1}{3}((x-y)^3-x^3+y^3)\Rightarrow &-3x^2y+3xy^2\\
R_{C_3}(xy)=\frac{1}{3}()\\
R_{C_3}(x^2y)=\frac{1}{3}()\\
R_{C_3}(xy^2)=\frac{1}{3}()\\
\end{align*}
TERMINARLO O HACER UN PROGRAMA EN SAGE
\end{enumerate}


\end{solucion}
\newpage

\begin{ejercicio}{7.3.8}
Let $A$ be the matrix of Exercise \ref{ejer:7.3.7}.
\begin{enumerate}[a.]
\item Show that $−A$ generates a cyclic matrix group $C_6$, of order 6.
\item Show that $−I_2 ∈ C_6$. Then use Exercise \ref{ejer:7.3.6} and §2 to show that $k[x, y]^{C_6} ⊆ k[x^2, y^2, xy]$.
Conclude that all nonzero homogeneous invariants of $C_6$ have even total degree.
\item Use part (b) and Theorem 5 to find $k[x, y]^{C_6}$. Hint: There are still a lot of Reynolds
operators to compute. You should use a computer algebra program to design a procedure
that has $i, j$ as input and $R_{C_6} (x^iy^j)$ as output.
\end{enumerate}
\end{ejercicio}
\begin{solucion}\
\begin{enumerate}[a.]
\item $(-A)(-A)=A^2$, $-AA^2=-A^3=-I_2$, $-A(-A)^3=A^4=A$, $-AA=-A^2$, $-A(-A)^2=A^3=I_2$. 
\item $-I_2\in C_6$ por el apartado anterior. Ahora, esto implica que $C_2\subseteq C_6$, por lo que usando el ejercicio \ref{ejer:7.3.6}, $k[x, y]^{C_6} ⊆ k[x,y]^{C_2}=k[x^2, y^2, xy]$. Esto implica evidentemente que los invariantes homogéneos tienen grado total par.
\item Tendremos que calcular el operador de Reynolds sobre los monomios $x^iy^j$ con $1\leq i+j\leq 6$ y $i+j\equiv 0\mod 2$, es decir, sobre $x^2,y^2,xy$ y cualquier producto de ellos de grado total inferior a 6. FALTAN LOS CÁLCULOS.
\end{enumerate}
\end{solucion}

\newpage

\begin{ejercicio}{7.3.9}
Let $A$ be the matrix
\[
A =
\frac{1}{\sqrt{2}}\begin{pmatrix}
1 &−1\\
1 &1\\
\end{pmatrix}
∈ GL(2, k).
\]
\begin{enumerate}[a.]
\item Show that $A$ generates a cyclic matrix group $C_8 ⊆ GL(2, k)$ of order 8.
\item Give a geometric argument to explain why $x^2 + y^2 ∈ k[x, y]^{C_8}$ . Hint: $A$ is a rotation
matrix.
\item As in Exercise \ref{ejer:7.3.8}, explain why all homogeneous invariants of $C_8$ have even total
degree.
\item Find $k[x, y]^{C_8}$ . Hint: Do not do this problem unless you know how to design a procedure
(on some computer algebra program) that has $i$, $j$ as input and $R_{C_8} (x^iy^j)$ as
output.
\end{enumerate}
\end{ejercicio}
\begin{solucion}
\begin{enumerate}[a.]
\item HACIENDO LOS CÁLCULOS NO ME SALE.
\item La ecuación $x^2+y^2=0$, luego si $k$ es algebraicamente cerrado $x^2=-y^2\Rightarrow x=iy$. El conjunto de puntos de este tipo son invariantes por rotaciones. En un cuerpo contenido en $\R^2$ es más claro, pues $x^2+y^2=r^2$ es una circunferencia, que claramente es invariante por rotaciones para cualquier $r$.
\item
\item 
\end{enumerate}
\end{solucion}
\newpage

\begin{ejercicio}{7.3.10}
Consider the finite matrix group
\[
G =
\left\{\begin{pmatrix}
±1& 0 &0\\
0 &±1 &0\\
0 &0 &±1
\end{pmatrix}
\right\}
⊆ GL(3, k).
\]
Note that $G$ has order 8.
\begin{enumerate}[a.]
\item If we were to use Theorem 5 to determine $k[x, y, z]^G$, for how many monomials would
we have to compute the Reynolds operator?
\item Use the method of Example 12 in §2 to determine $k[x, y, z]^G$.
\end{enumerate}
 
\end{ejercicio}
\begin{solucion}\
\begin{enumerate}[a.]
\item $9+8+7+6+5+4+3+2=44$. 
\item Obsérvese que este grupo está generado por las 3 matrices que son como la identidad pero con una de las entradas de la diagonal cambiada de signo. Entonces un polynomio $f\in k[x, y, z]^G$ si y solo si $f(x,y,z)=f(-x,y,z)=f(x,-y,z)=f(x,y,-z)$. Siguiendo la analogía del ejemplo 12, llegamos a que $k[x, y, z]^G=k[x^2,y^2,z^2]$.
\end{enumerate}
\end{solucion}
\newpage


\begin{ejercicio}{7.3.11}
Let $f$ be the polynomial (4) in the text.
\begin{enumerate}[a.]
\item Verify that $f ∈ k[x, y]^{C_4} = k[x^2 + y^2, x^3y − xy^3, x^2y^2]$.
\item Use Proposition 7 to express $f$ as a polynomial in $x^2 + y^2$, $x^2y − xy^3$, $x^2y^2$.
\end{enumerate}
\end{ejercicio}
\begin{solucion} 
El polinomio en cuestión es $$f (x, y) = x^8 + 2x^6y^2 − x^5y^3 + 2x^4y^4 + x^3y^5 + 2x^2y^6 + y^8.$$
\begin{enumerate}[a.]
\item Se tiene $f(x,y)=f(-y,x)=f(-x,-y)=f(y,-x)$, que es por definición ser invariante por $C_4$.
\item 
\end{enumerate}
\end{solucion}


\end{document}

