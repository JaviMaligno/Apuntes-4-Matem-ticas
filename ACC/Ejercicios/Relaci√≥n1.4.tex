\documentclass[twoside]{article}
\usepackage{../../estilo-ejercicios}

%--------------------------------------------------------
\begin{document}

\title{Ejercicios de Ideals, Varieties, and Algorithms (4ª Edición)}
\author{Diego Pedraza López, Javier Aguilar Martín, Rafael González López}
\maketitle

\begin{ejercicio}{1.4.1}
Consider the equations
\begin{align*}
x^2 + y^2 -1 & = 0\\
xy - 1 & = 0
\end{align*}
which describe the intersection of a circle and a hyperbola.
\begin{enumerate}[a.]
\item Use algebra to eliminate $y$ from the above equations.
\item Show how the polynomial found in part (a) lies in $\gene{x^2+y^2-1, xy-1}$.
\end{enumerate}
\end{ejercicio}

\begin{solucion}
Multiplicando la segunda ecuación por $xy+1$ obtenemos $x^2y^2-1=0$, luego $y^2 = 1/x^2$.
Desde aquí eliminamos $y$ de la primer ecuación: $x^2+1/x^2-1=0$, que es equivalente a $x^4-x^2+1=0$.
De como hemos obtenido este polinomio, deducimos que: $x^4-x^2+1 = (x^2)(x^2+y^2-1)-(xy+1)(xy-1) \in \gene{x^2+y^2-1, xy-1}$.
\end{solucion}

\newpage

\begin{ejercicio}{1.4.2}
Let $I \subseteq k[x_1,\dots,x_n]$ be an ideal, and let $f_1,\dots,f_s \in k[x_1,\dots,x_n]$. Prove that the following statements are equivalent:
\begin{enumerate}[(i)]
\item $f_1,\dots,f_s \in I$
\item $\langle f_1,\dots,f_s\rangle \in I$
\end{enumerate}
This fact is useful when you want to show that one ideal is contained in another.
\end{ejercicio}

\begin{solucion}
La implicación $(ii)\Rightarrow(i)$ es evidente, pues $f_1,\dots,f_s \in \gene{f_1,\dots, f_s}$.

Véamos la otra implicación.
Sea $f \in \gene{f_1,\dots,f_s}$.
Entonces existen $g_1,\dots,g_n \in k[x_1,\dots,x_n]$ tal que:
\[ f = g_1 f_1 + g_2 f_2 + \dots + g_n f_n \]
Por las propiedades de los ideales: como $f_i \in I$, $g_i f_i \in I$, luego $g_1 f_1 + g_2 f_2 + \dots + g_n f_n \in I$.
\end{solucion}

\newpage

\begin{ejercicio}{1.4.3}
Use the previous exercise to prove the following equalities of ideals in $\Q[x,y]$:
\begin{enumerate}[a.]
\item $\gene{x+y,x-y} = \gene{x,y}$.
\item $\gene{x+xy, y+xy, x^2, y^2} = \gene{x, y}$.
\item $\gene{2x^2+3y^2-11, x^2-y^2-3} = \gene{x^2-4, y^2-1}$.
\end{enumerate}
This illustrates that the same ideal can have many different bases and that different bases may have different numbers of elements.
\end{ejercicio}
\begin{solucion}\mbox{}
\begin{enumerate}[a.]
\item $x+y \in \gene{x,y}$ y $x-y \in \gene{x,y}$, luego $\gene{x+y,x-y} \subseteq \gene{x,y}$.
$x = \frac{1}{2} (x+y) + \frac{1}{2}  (x-y) \in \gene{x+y,x-y}$ e $y = \frac{1}{2} (x+y) - \frac{1}{2} (x-y) \in \gene{x+y, x-y}$, luego $\gene{x,y} \subseteq \gene{x+y,x-y}$.
\item $\gene{x+xy, y+xy, x^2, y^2} \subseteq \gene{x, y}$ es evidente.
\[ x = (1+x)(x + xy) - x(y+xy) - x^2 \in \gene{x+xy, y+xy, x^2, y^2} \]
\[ y = (1+y)(y + xy) - y(x+xy) - y^2 \in \gene{x+xy, y+xy, x^2, y^2} \]
Luego $\gene{x,y} \subseteq \gene{x+xy, y+xy, x^2, y^2}$.
\item 
\[ 2x^2+3y^2 - 11 = 2(x^2-4)+3(y^2-1) \in \gene{x^2-4, y^2-1} \]
\[ x^2-y^2-3 = (x^2-4) - (y^2-1) \in \gene{x^2-4, y^2-1} \]
Luego $\gene{2x^2+3y^2-11, x^2-y^2-3} \subseteq \gene{x^2-4, y^2-1}$.
Podemos usar estas ecuaciones para despejar la otra base:
\[ x^2-4 = \frac{1}{5} (2x^2+3y^2-11) - \frac{3}{5} (x^2-y^2-3) \]
\[ y^2-1 = \frac{1}{5}(2x^2+3y^2-11) - \frac{2}{5}(x^2-y^2-3) \]
Luego $\gene{2x^2+3y^2-11, x^2-y^2-3} \supseteq \gene{x^2-4, y^2-1}$.
\end{enumerate}
\end{solucion}

\newpage

\begin{ejercicio}{1.4.4}
Prove Proposition 4: If $f_1,\dots,f_s$ and $g_1,\dots,g_t$ are bases of the same ideal in $k[x_1,\dots,x_n]$, so that $\langle f_1,\dots,f_s\rangle = \langle g_1,\dots,g_t\rangle$, then we have $\V(f_1,\dots,f_s) = \V(g_1,\dots,g_t)$.
\end{ejercicio}
\begin{solucion}
Evidente cuando recordamos que $\V(f_1,\dots,f_s) = \V(\gene{f_1,\dots,f_s})$.
\end{solucion}

\newpage

\begin{ejercicio}{1.4.5}
Show that $\V(x+xy,y+xy,x^2,y^2) = \V(x,y)$.
\end{ejercicio}
\begin{solucion}
Hemos visto en el ejercicio \ref{ejer:1.4.3} que $\gene{x+xy,y+xy,x^2,y^2} = \gene{x,y}$.
Este hecho unido con el ejercicio \ref{ejer:1.4.4} demuestra el resultado.
\end{solucion}

\newpage

\begin{ejercicio}{1.4.6}
The word ``basis'' is used in various ways in mathematicas.
In this exercise, we will see that ``a basis of an ideal'', as defined in this section, is quite different from ``a basis of a subspace'', which is studied in linear algebra.
\begin{enumerate}[a.]
\item Firrst, consider the ideal $I = \gene{x} \subseteq k[x]$.
As an ideal, $I$ has a basis consisting of the one element $x$.
But $I$ can be regarded as a subspace of $k[x]$, which is a vector space of $k$.
Prove that any vector space basis of $I$ over $k$ is infinite.
\item In linear algebra, a basis must span and be linearly independent over $k$, whereas for an ideal, a basis is concerned only with spanning -- there is no mention of any sort of independence.
The reason is that once we allow polynomial coefficients, no independence is possible.
To see this, consider the ideal $\gene{x,y} \subseteq k[x,y]$.
Show that zero can be be written as a linear combination of $y$ and $x$ with nonzero polynomial coefficients.
\item More generally, suppose that $f_1,\dots,f_s$ is the basis of an ideal $I \subseteq k[x_1,\dots,x_n]$.
If $s ≥ 2$ and $f_i \neq 0$ for all $i$, then show that for any $i$ and $j$, zero can be rewritten as a linear combination of $f_i$ and $f_j$ with nonzero polynomial coefficients.
\item As a consequence of the lack of independence is that when we write an element $f \in \gene{f_1,\dots,f_s}$ as $f = \sum_{i=1}^s h_i f_i$, the coefficients $h_i$ are not unique.
As an example, consider $f = x^2 + xy + y^2 \in \gene{x,y}$.
Express $f$ as a linear combination of $x$ and $y$ in two different ways.
\item A basis $f_1,\dots,f_s$ of an ideal $I$ is said to be \emph{minimal} if no proper subset of $f_1,\dots,f_s$ is a basis of $I$. For example, $x$ $x^2$ is a basis of an ideal, but not a minimal basis since $x$ generates the same ideal.
Unfortunately, an ideal can have mminimal bases consisting of different numbers of elements.
To see this, show that $x$ and $x+x^2,x^2$ are minimal bases of the same ideal $k[x]$. Explain how this contrasts with the situation in linear algebra.
\end{enumerate}
\end{ejercicio}
\begin{solucion}
\begin{enumerate}[a.]
\item Como los escalares de $I$ como espacio vectorial son $k$, una base de $I$ sería $1,x,x^2,\dots$.
Es una base infinita, luego $I$ tiene dimensión infinita como espacio vectorial.
\item $xy-yx = 0$.
\item $f_i f_j - f_j f_i = 0$. Aquí usamos que $f_i \neq 0$ para todo $i$.
\item $x^2+xy+y^2 = (x+y)x+yy = xx+(x+y)y$.
\item Claramente $x$ es una base mínima.
Como ni $x+x^2$ ni $x^2$ genera el mismo ideal que $x+x^2,x^2$, este es mínimo.
Es fácil ver además que $\gene{x}=\gene{x+x^2,x^2}$.
Tenemos aquí un ejemplo de que las bases de ideales no tienen necesariamente la misma cantidad de elementos.
Esto lo diferencia con el caso de álgebra lineal.
\end{enumerate}
\end{solucion}

\newpage

\begin{ejercicio}{1.4.7}
Show that $\I(\V(x^n,y^m))=\langle x,y \rangle$, for any positive integers $n$ and $m$.
\end{ejercicio}
\begin{solucion}
Como $x^n$ e $y^m$ se anulan si y sólo si $x=y=0$, tenemos que $\V(x^n,y^m)= \{(0,0)\}$. Ya sabemos que $\I(\{(0,0)\}) = \gene{x,y}$, de donde se tiene el resultado.
\end{solucion}

\newpage

\begin{ejercicio}{1.4.8}
The ideal $\I(V)$ of a variety has a special property not shared by all ideals. Specifically, we define an ideal $I$ to be \emph{radical} if whenever a power $f^m$ of a polynomial $f$ is in $I$, then $f$ itself is in $I$. More succinctly, $I$ is radical when $f \in I$ if and only if $f^m \in I$ for some positive integer $m$.
\begin{enumerate}[a.]
\item Prove that $\I(V)$ is always a radical ideal.
\item Prove that $\gene{x^2,y^2}$ is not a radical ideal.
This implies that $\gene{x^2,y^2} \neq \I(V)$ for any variety $V \subseteq k^2$.
\end{enumerate}
Radicals ideals will play an important role in Chapter 4. In particular, the Nullstellensatz will imply that there is a one-to-one correspondence between varieties in $\C^n$ and radical ideals in $\C[x_1,\dots,x_n]$.
\end{ejercicio}
\begin{solucion}
\begin{enumerate}[a.]
\item Sea $f^m \in \I(V)$ con $f \in k[1,\dots,x_n]$ y $m$ entero positivo.
Supongamos que $V$ es no vacío.
Sea $p \in V$, entonces $f^m(p)=0$.
Como el ideal $\gene{0}$ es primo en $k$, tenemos que $f(p)=0$.
Como $p$ era un punto arbitrario, $f$ se anula en todo $V$, luego $f \in \I(V)$.
Esto demuestra que $\I(V)$ es radical.
\item Claramente $x \notin \gene{x^2,y^2}$, pero $x^2 \in \gene{x^2,y^2}$, luego $\gene{x^2,y^2}$ no es ideal radical.
\end{enumerate}
\end{solucion}
\newpage
\begin{ejercicio}{1.4.9}
Let $V = \V(y − x^2, z − x^3)$ be the twisted cubic. In the text, we showed that $\I(V) = \gene{y − x^2, z − x^3}$.
\begin{enumerate}
\item Use the parametrization of the twisted cubic to show that $y^2 − xz \in \I(V)$.
\item Use the argument given in the text to express $y^2 − xz$ as a combination of $y − x^2$ and $z − x^3$.
\end{enumerate}
\end{ejercicio}
\begin{solucion}
\begin{enumerate}
\item[]
\item Sustituyendo $x\to t$, $y\to t^2$ y $z\to t^3$ es claro que $y^2-xz = t^4-tt^3 = 0$, luego se anula en todos los puntos de la curva.
\item $(y+x^2)(y-x^2)-x(z-x^3) = y^2-xz$. 
\end{enumerate}
\end{solucion}

\newpage
\begin{ejercicio}{1.4.10}
Use the argument given in the discussion of the twisted cubic to show that $\I(\V(x−y)) = \gene{x-y}$. Your argument should be valid for any infinite field k.
\begin{solucion}
Sabemos que $\V(x-y) = \{(t,t)\mid t\in k\}$. Sea $f\func{k[x,y]}{k[t]}$ de manera que $x\to t$, $y\to t$. Es claro que $\ker f = \gene{x-y}$. Como $k[t]$ es dominio de integridad, $\ker f$ es un ideal primo, luego en particular es una ideal radical. 
\end{solucion}
\end{ejercicio}

\newpage
\begin{ejercicio}{1.4.11}
Let $V ⊆ \R^3$ be the curve parametrized by $(t, t^3, t^4)$.
\begin{enumerate}[a.]
\item Prove that V is an affine variety.
\item Adapt the method used in the case of the twisted cubic to determine $\I(V)$.
\end{enumerate}
\begin{solucion}
\begin{enumerate}[a.]
\item[]
\item Vamos a ver la curva es precisamente $S=\V(x^3-y,x^4-z)$. Obviamente $V\subset S$. Recíprocamente, sea $(x,y,z)\in S$. Escribiendo $x=t$ entonces $y=x^3=t^3$ y $z=t^4$. Por tanto, $V$ es una variedad afín.
\item Con un procedimiento análogo al que se realiza en la página 33 del libro, podemos ver que todo polinomio $f \in \R[x,y,z]$ puede escribirse como
$$
f = h_1(x^3-y)+h_2(x^4-z)+r
$$
donde $r\in \R[x]$. Es claro que $\gene{x^3-y,x^4-z}\subset I(S)$. Recíprocamente, veamos que si $f\in I(S)$, $r=0$. Como $f\in I(S)$ entonces $f(t,t^3,t^4)=0$, luego
$$
f(t,t^3,t^4)= h_1(t,t^3,t^4)(t^3-t^3)+h_2(t,t^3,t^4)(t^4-t^4)+r(t) = r(t) = 0$$
Como $r(t)=0$ $\forall t \in \R$ entonces $r=0$. Lo que prueba la igualdad.
\end{enumerate}
\end{solucion}
\end{ejercicio}

\newpage
\begin{ejercicio}{1.4.12}
\end{ejercicio}

\newpage
\begin{ejercicio}{1.4.13}
In Exercise 1.1.2, we showed that $x^2y+y^2x$ vanishes at all points of $\F_2^2$. More generally, let $I ⊆ \F_2[x, y]$ be the ideal of all polynomials that vanish at all points of $\F_2^2$. The goal of this exercise is to show that $I = \gene{x^2-x,y^2-y}$
\begin{enumerate}[a.]
\item Show that $\gene{x^2-x,y^2-y}\subset I$.
\item Show that every $f ∈ \F_2[x, y]$ can be written as $f = A(x^2 − x) + B(y^2 − y) + axy
+ bx + cy + d$, where $A, B ∈ \F_2[x, y]$ and $a, b, c, d ∈ \F_2$. Hint: Write f in the form $\sum_i p_i(x)y^i$ and use the division algorithm (Proposition 2 of §5) to divide each $p_i$ by $x^2 − x$. From this, you can write $f = A(x^2 − x) + q_1(y)x + q_2(y)$. Now divide $q_1$ and $q_2$ by $y^2 − y$. Again, this argument will become vastly simpler once we know the division algorithm from Chapter 2.
\item Show that $axy + bx + cy + d ∈ I$ if and only if $a = b = c = d = 0$.
\item Using parts (b) and (c), complete the proof that $I =\gene{x^2-x,y^2-y}$.
\item Express $x^2y + y^2x$ as a combination of $x^2-x$ and $y^2-y$.
\end{enumerate}
\begin{solucion}
\begin{enumerate}[a.]
\item[]
\item Es claro que si $f(x,y) = h_1(x^2-x)+h_2(y^2-y)$ entonces $f(0,0)=f(1,0)=f(0,1)=f(1,1)=0$. Por tanto $f\in I$, luego $\gene{x^2-x,y^2-y}\subset I$
\item
\item Sea $f(x,y) = axy+bx+cy+d$. Si $f\in I$ entonces se anula en cada punto. Como $f(0,0)=d$, $d=0$. Dado que $f(1,0)=b$ y $f(0,1)=c$ entonces $b=c=0$. Finalmente, $f(1,1)=a$ por lo que $a=0$. El recíproco es trivial.
\item Sea $f\in I$ cualquiera, entonces podemos escriber $f$ según la fórmula obtenida en el apartado $(b)$, pero utilizando $(c)$, tenemos que puede escribirse simplemente como $A(x^2-x)+B(y^2-y)$. Pues si alguno de las otras constantes no fueran nulas, como $g = A(x^2-x)+B(y^2-y)\in \gene{x^2-x,y^2-y} \subset I$, tendríamos que $f-g\in I$, pero por la forma de $f-g$, por el apartado $(c)$, $f-g=0$. Por tanto, $f\in \gene{x^2-x,y^2-y}$.
\item $x^2y + y^2x=y(x^2-x)+x(y^2-y)$ (recordemos que el cuerpo es $\F_2$)
\end{enumerate}
\end{solucion} 
\end{ejercicio}
\newpage

\begin{ejercicio}{1.4.15}
In the text, we defined $\I(V)$ for a variety $V ⊆ k^n$. We can generalize this as follows: if
$S ⊆ k^n$ is any subset, then we set
$$\I(S) = \{f ∈ k[x_1,\dots, x_n] | f (a_1,\dots, a_n) = 0\ \forall (a_1,\dots, a_n) ∈ S\}.$$
\begin{itemize}
\item[a.] Prove that $\I(S)$ is an ideal.
\item[b.] Let $X = \{(a, a) ∈ \R^2 | a \neq 1\}$. By Exercise 8 of §2, we know that $X$ is not an affine
variety. Determine $\I(X)$. Hint: What you proved in Exercise 8 of §2 will be useful.
See also Exercise 10 of this section.
\item[c.] Let $\Z^n$ be the points of $\C^n$ with integer coordinates. Determine $\I(Z^n)$. Hint: See Exercise
6 of §1.
\end{itemize}
\begin{solucion}
\begin{itemize}
\item[a.] Análogo a cuando es una variedad.
\item[b.] $\I(X)=\I((t,t))=\langle x-y\rangle$.
\item[c.] $\I(\Z^n)=\I(\C^n)=\{0\}$. 
\end{itemize}
\end{solucion}
\end{ejercicio}

\newpage

\begin{ejercicio}{1.4.16}
Here is more practice with ideals. Let $I$ be an ideal in $k[x_1,\dots , x_n]$.
\item[a.] Prove that $1 ∈ I$ if and only if $I = k[x_1,\dots , x_n]$.
\item[b.] More generally, prove that $I$ contains a nonzero constant if and only if $I = k[x_1,\dots, x_n]$.
\item[c.] Suppose $f , g ∈ k[x_1,\dots , x_n]$ satisfy $f^2, g^2 ∈ I$. Prove that $(f + g)^3 ∈ I$. Hint: Expand
$(f + g)^3$ using the Binomial Theorem.
\item[d.] Now suppose $f , g ∈ k[x_1,\dots , x_n]$ satisfy $f^r$ , $g^s ∈ I$. Prove that $(f + g)^{r+s−1} ∈ I$.
\end{ejercicio}
\begin{solucion}
\begin{itemize}
\item[a.] Si $1\in I$, por ser ideal, $f\cdot 1\in I$ para todo $f\in k[x_1,\dots , x_n]$. La otra inclusión es trivial.
\item[b.] Como $k$ es cuerpo, dada una constanta $a$, podemos generar $1=aa^{-1}$, por lo que podemos aplicar el apartado anterior.
\item[c.] $(f+g)^3=f^3+3f^2g+3fg^2+g^3=ff^2+3f^2g+3fg^2+gg^2\in I$. 
\item[d.] Basta hacer lo mismo que hemos hecho antes con el desarrollo general. 
\end{itemize}
\end{solucion}

\newpage

\begin{ejercicio}{1.4.17}
In the proof of Lemma 7, we showed that $x \not∈ 
\langle x^2, y^2\rangle$ in $k[x, y]$.
\begin{itemize}
\item[a.] Prove that $xy \not∈ 
\langle x^2, y^2\rangle$.
\item[b.] Prove that 1, $x$, $y$, $xy$ are the only monomials not contained in 
$\langle x^2, y^2\rangle$.
\end{itemize}
\end{ejercicio}
\begin{solucion}
\begin{itemize}
\item[a.] Cualquier polinomio de $\langle x^2, y^2\rangle$ tendrá un término en $x^2$ o en $y^2$, cosa que $xy$ no tiene. 
\item[b.] Por el mismo razonamiento, esos monomios no están contenidos. Además, cualquier otro de $k[x,y]$ tiene algún término en $x^2$ o en $y^2$, luego está en el ideal. 
\end{itemize}
\end{solucion}

\newpage

\begin{ejercicio}{1.4.18}
In the text, we showed $\I(\{(0,0)\}) = \gene{x,y}$ in $k[x,y]$. 
\begin{enumerate}[a.]
\item Generalize this by proving that the origin $0\in k^n$ has the property that $\I(\{0\}) = \gene{x_1,\dotsc,x_n}$ in $k[x_1,\dotsc,x_n]$.
\item What does part (a) say about polynomials in $k[x_1,\dotsc,x_n]$ with zero constant term?
\end{enumerate}
\end{ejercicio}
\begin{solucion}
\begin{enumerate}[a.]
\item[]
\item Una contención es trivial, pues si $f= \sum g_i x_i$ entonces  $f(0)=0$, por lo que $f\in \I(\{0\})$. Como $I(S)$ es un ideal propio, basta probar que $\gene{x_1,\dotsc,x_n}$ es maximal, pero esto es claro pues $\gene{x_1,\dotsc,x_n} = k[x_1,\dotsc,x_n]\setminus k$. Luego si existiera otro ideal que lo contiene, este ha de tener necesariamente alguna unidad y por tanto sería el total.
\item Que siempre se anulan en el 0, o equivalentemente, que conforman el ideal $\gene{x_1,\dotsc,x_n}$.

\end{enumerate}
\end{solucion}
\end{document}