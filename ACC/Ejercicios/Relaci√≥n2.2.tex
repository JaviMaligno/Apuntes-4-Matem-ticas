\documentclass[twoside]{article}
\usepackage{../../estilo-ejercicios}
\newcommand{\lex}{<_{lex}}
\newcommand{\grlex}{<_{grlex}}
\newcommand{\grevlex}{<_{grevlex}}

%--------------------------------------------------------
\begin{document}

\title{Ejercicios de Ideals, Varieties, and Algorithms (4ª Edición)}
\author{Diego Pedraza López, Javier Aguilar Martín, Rafael González López}
\maketitle

\begin{ejercicio}{2.2.1}
Rewrite each of the following polynomials, ordering the terms using the lex order, the grlex order, and the grevlex order, giving $LM(f)$, $LT(f)$, and $multideg(f)$ in each case.
\begin{enumerate}[a.]
\item $f(x,y,z) = 2x + 3y + z + x^2- z^2 + x^3$.
\item $f(x,y,z) = 2x^2y^8 - 3x^5yz^4 + xyz^3 - xy^4$.
\end{enumerate}
\end{ejercicio}
\begin{solucion}\mbox{}
\begin{enumerate}[a.]
\item Respecto al orden lex:
 \[ f = x^3 + x^2 + 2x + 3y - z^2 + z \]
 Respecto al orden grlex:
 \[ f = x^3 + x^2 - z^2 + 2x + 3y + z \]
 Respecto al orden gravlex:
 \[ f = x^3 + x^2 - z^2 + 2x + 3y + z \]
 En los tres casos: $multideg(f)=(3,0,0)$, $LM(f)=x^3$ y $LT(f)=1$.
\item Respecto al orden lex:
 \[ f = -3x^5yz^4 + 2x^2y^8 - xy^4 + xyz^3 \]
 Respecto al orden grlex:
 \[ f = -3x^5yz^4 + 2x^2y^8 - xy^4 + xyz^3 \]
 Respecto al orden gravlex:
 \[ f = 2x^2y^8 -3x^5yz^4 - xy^4 + xyz^3\]
 En orden lex y grlex: $multideg(f)=(5,1,4)$, $LM(f)=x^5yz^4$ y $LT(f)=-3$. En gravlex: $multideg(f)=(2,8,0)$, $LM(f)=x^2y^8$ y $LT(f)=2$.
\end{enumerate}
\end{solucion}

\newpage

\begin{ejercicio}{2.2.2}
Each of the following polynomials is written with its monomials ordered according to (exactly) one of lex, grlex, or grevlex order. Determine which monomial order was used in each case.
\begin{enumerate}[a.]
\item $f(x,y,z) = 7x^2y^4z - 2xy^6 + x^2y^2$.
\item $f(x,y,z) = xy^3z + xy^2z^2 + x^2z^3$.
\item $f(x,y,z) = x^4y^5z + 2x^3y^2z - 4xy^2z^4$.
\end{enumerate}
\end{ejercicio}
\begin{solucion}\mbox{}
\begin{enumerate}[a.]
\item
\begin{align*}
(2,4,1) \not\lex & (1,6,0) \lex (2,2,0)\\
(2,4,1) \grlex & (1,6,0) \grlex (2,2,0)\\
(2,4,1) \not\grevlex & (1,6,0) \grevlex (2,2,0)
\end{align*}
Luego $f$ está ordenado por grlex.
\item
\begin{align*}
(1,3,1) \lex & (1,2,2) \not\lex (2,0,3)\\
(1,3,1) \grlex & (1,2,2) \not\grlex (2,0,3)\\
(1,3,1) \grevlex & (1,2,2) \grevlex (2,0,3)
\end{align*}
Luego $f$ está ordenado por gravlex.
\item
\begin{align*}
(4,5,1) \lex & (3,2,1) \not\lex (1,2,4)\\
(4,5,1) \grlex & (3,2,1) \not\grlex (1,2,4)\\
(4,5,1) \grevlex & (3,2,1) \not\grevlex (1,2,4)
\end{align*}
Luego $f$ está ordenado por lex.
\end{enumerate}
\end{solucion}

\newpage

\begin{ejercicio}{2.2.3}
Repeat Exercise \ref{ejer:2.2.1} when the variables are ordered $z > y > x$.
\end{ejercicio}
\begin{solucion}\mbox{}
\begin{enumerate}[a.]
\item Respecto al orden lex:
 \[ f = -z^2 + z + 3y + x^3 + x^2 + 2x \]
 Respecto al orden grlex:
 \[ f = x^3 - z^2 + x^2  + z + 3y + 2x \]
 Respecto al orden gravlex:
 \[ f = x^3 - z^2 + x^2  + z + 3y + 2x \]
\item Respecto al orden lex:
 \[ f = -3x^5yz^4 + xyz^3 + 2x^2y^8 - xy^4 \]
 Respecto al orden grlex:
 \[ f = -3x^5yz^4 + 2x^2y^8 + xyz^3 - xy^4 \]
 Respecto al orden gravlex:
 \[ f = 2x^2y^8 - 3x^5yz^4 + xyz^3 - xy^4 \]
\end{enumerate}
\end{solucion}

\newpage

\begin{ejercicio}{2.2.4}
Show that grlex is a monomial order according to Definition 1.
\end{ejercicio}
\begin{solucion}\mbox{}
\begin{itemize}
\item Orden total: Cierto a partir de la definición.
\item Compatible con producto de monomios: Sea $α \grlex β$.
Si $|α|>|β|$, entonces $|α+γ|=|α|+|γ|>|β|+|γ|>|β+γ|$, luego $α+γ \grlex β+γ$.
Si $|α|=|β|$, entonces $α+γ \grlex β+γ$, pues $α+γ \lex β+γ$.
\item $(\Z_{≥0}^n,\grlex)$ está bien ordenado: Supongamos que este no fuera el caso.
Entonces existe una secuencia infinita:
\[ α(1) \grlex α(2) \grlex α(3) \grlex \dots \]
Esto implica que:
\[ |α(1)| ≥ |α(2)| ≥ |α(3)| ≥ \dots \]
Como $(\Z_{≥0},≥)$ está bien ordenado, esta cadena debe estabilizarse, digamos a partir de $n$. De manera que, como $|α(n)|=|α(m)|$ para todo $m≥n$:
\[ α(n) \lex α(n+1) \lex α(n+2) \lex \dots \]
Pero esto implicaría que $(\Z_{≥0},\lex)$ no está bien ordenado, y sabemos que no es el caso.
Hemos llegado a una contradicción.
\end{itemize}
\end{solucion}

\newpage

\begin{ejercicio}{2.2.5}
Show that grevlex is a monomial order according to Definition 1.
\end{ejercicio}
\begin{solucion}\mbox{}
\begin{itemize}
\item Orden total: Cierto a partir de la definición.
\item Compatible con producto de monomios: Sea $α \grevlex β$.
Si $|α|>|β|$, está claro.
Si $|α|=|β|$, entonces el elemento no nulo más a la derecha de $α - β$ es negativo.
Como $(α+γ)-(β+γ) = α-β$, entonces $α + γ \grevlex β+γ$.
\item $(\Z_{≥0}^n,\grevlex)$ está bien ordenado: Supongamos que este no fuera el caso.
Como en el ejercicio anterior, esto implica que existe una cadena infinita: 
\[ α(1) \grevlex α(2) \grevlex α(3) \grevlex \dots \]
tal que $n=|α(1)|=|α(2)|=\dots$.
Pero basta ver que cada componente $α(i)$ forma a su vez una cadena ascendente de enteros positivos acotados por $n$.
Es decir, cada componente debe llegar a ser constante $≤n$.
Entonces no es posible tal cadena infinita.
Hemos llegado a una contradicción.
\end{itemize}
\end{solucion}

\newpage

\begin{ejercicio}{2.2.6}
Another monomial order is the \textbf{inverse lexicographic} or \textbf{invlex} order defined by the
following: for $α, β ∈ \Z^n_{≥0}$, $α >_{invlex} β$ if and only if the rightmost nonzero entry of
$α − β$ is positive. Show that invlex is equivalent to the lex order with the variables
permuted in a certain way. (Which permutation?)
\end{ejercicio}
\begin{solucion}
Basta hacer reverse a la upla.
\end{solucion}

\newpage

\begin{ejercicio}{2.2.7}
Let $>$ be any monomial order.
\begin{enumerate}[a.]
\item Show that $α ≥ 0$ for all $α \in \Z^n_{≥0}$.
\item Show that if $x^α$ divides $x^β$, then $α ≤ β$. Is the converse true?
\item Show that if $α \in \Z_{≥0}^n$, then $α$ is the smallest element of $α + \Z_{≥0}^n = \{α+β \mid β \in \Z_{≥0}^n\}$.
\end{enumerate}
\end{ejercicio}
\begin{solucion}
\begin{enumerate}[a.]
\item Supongamos que $0 > α$.
Entonces por la propiedad de compatibilidad por producto de monomio: $0+α > α+α$, es decir $α > 2α$.
Podemos aplicar esto para obtener una cadena infinita:
\[ 0 > α > 2α > \dots \]
Esto contradice la propiedad de buen ordenamiento.
Entonces $α \geq  0$ por reducción al absurdo.
\item Si $x^α$ divide a $x^β$, entonces existe $x^γ$ tal que $x^αx^γ=x^β$.
Como $α ≤ α+γ = β$, deducimos que $α ≤ β$. Para el recíproco, basta probar que existe $\gamma$ tal que $\alpha+\gamma=\beta$. Basta tomar $\gamma=\beta-\alpha\geq 0$. Entonces $\alpha+\gamma=\beta$. Así, tenemos $x^\beta=x^\alpha x^\gamma$.

\item Consecuencia del hecho de que $α ≥ 0$.
Para todo $γ \in α + \Z_{≥0}^n$, $α ≥ 0$ implica que $α + γ ≥ α$, luego $α$ es el menor elemento del conjunto.
\end{enumerate}
\end{solucion}

\newpage

\begin{ejercicio}{2.2.8}
Write a precise definition of what it means for a system of linear equations to be in
echelon form, using the ordering given in equation (2).
\end{ejercicio}
\begin{solucion}
A system of $m$ linear equations in n variables is called an echelon system if
\begin{enumerate}
\item $m ≤ n$.
\item Every variable is the leading variable of at most one equation.
\item Every leading variable is to the left of the leading variables of all lower equations.
\item Every equation has a leading variable.
\end{enumerate}
\end{solucion}

\newpage

\begin{ejercicio}{2.2.9}
In this exercise, we will study grevlex in more detail. Let $>_{invlex}$, be the order given in
Exercise 6, and define $>_{rinvlex}$ to be the reversal of this ordering, i.e., for $α, β ∈ \Z^n_{
≥0}$
$$α >_{rinvlex} β\Leftrightarrow β >_{invlex} α.$$
Notice that rinvlex is a “double reversal” of lex, in the sense that we first reverse the
order of the variables and then we reverse the ordering itself.
\begin{itemize}
\item[a.] Show that $α >_{grevlex} β$ if and only if $|α| > |β|$, or $|α| = |β|$ and $α >_{rinvlex} β$.
\item[b.] Is rinvlex a monomial ordering according to Definition 1? If so, prove it; if not, say
which properties fail.
\end{itemize}
\end{ejercicio}
\begin{solucion}\
\begin{enumerate}
\item[a.] $α >_{grevlex} β$ if and only if $|α| > |β|$ or $|α| = |β|$ and $\beta >_{invlex}\alpha\Leftrightarrow \alpha>_{rinvlex}\beta$.
\item[b.] Falla que no es un buen orden. Para ellos, basta construir una cadena creciente que no termine en $>_{invlex}$, lo cual es equivalente a construirla en $>_{lex}$, lo cual es trivial de conseguir. Esto nos dará una cadena decreciente en $>_{rinvlex}$ que no termina.
\end{enumerate}
\end{solucion}

\newpage

\begin{ejercicio}{2.2.10}
In $\Z_{≥0}$ with the usual ordering, between any two integers, there are only a finite number
of other integers. Is this necessarily true in $\Z^n_{≥0}$ for a monomial order? Is it true for the
grlex order?
\end{ejercicio}
\begin{solucion}
Esto es equivalente a comprobar si hay una cantidad finita de elementos entre $0$ y $\gamma$ para cualquier $\gamma\in \Z^n_{≥0}$. Supongamos que no, entonces, como el orden es total, tendríamos
\[
\gamma\geq\gamma_1\geq\gamma_2\geq\dots\geq 0.
\]
En particular, $\gamma_1\geq\gamma_2\geq\cdots$ es una sucesión decreciente, por lo que se tiene que estabilizar en algún punto, así que hay una cantidad finita.
\end{solucion}

\newpage

\begin{ejercicio}{2.2.13}
Prove that $1 < x < x^2 < x^3 <\cdots$ is the unique monomial order on $k[x]$.
\end{ejercicio}
\begin{solucion}
Esto es equivalente a probar que el único orden monomial en $\Z_{\geq 0}$ es
\[
0<1<2<\cdots
\] 
Como el orden es total, de existir otro, digamos $\prec$,  tendría que cumplir que $n+1\prec n$ para algún $n\geq 0$. Entonces, reiterando obtendríamos una cadena decreciente 
\[
n\succ n+1\succ n+2\succ\cdots 
\]
que no se estabiliza, luego no es un orden monomial.
\end{solucion}

\end{document}
