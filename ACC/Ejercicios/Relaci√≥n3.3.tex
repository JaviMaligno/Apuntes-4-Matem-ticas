\documentclass[twoside]{article}
\usepackage{../../estilo-ejercicios}

%--------------------------------------------------------
\begin{document}

\title{Ejercicios de Ideals, Varieties, and Algorithms (4ª Edición)}
\author{Diego Pedraza López, Javier Aguilar Martín, Rafael González López}
\maketitle

\begin{ejercicio}{3.3.1}
In diagram (3) in the text, prove carefully that $F = π_m \circ i$ and $i(k^m) = V$.
\end{ejercicio}
\begin{solucion}
Dado $(t_1,\dots, t_m)\in k^m$
\[
\pi_m\circ i(t_1,\dots, t_m)=\pi_m(t_1,\dots, t_m, f_1(t1,\dots, t_m),\dots, f_n(t_1,\dots, t_m))=
\]
\[
(f_1(t_1,\dots, t_m),\dots, f_n(t_1,\dots, t_m))=F(t_1,\dots, t_m).
\]
En las ecuaciones anteriores tenemos $i(t_1,\dots, t_m)=(t_1,\dots, t_m, f_1(t_1,\dots, t_m),\dots, f_n(t_1,\dots, t_m))$, que son justamente los puntos de $V$.
\end{solucion}

\newpage

\begin{ejercicio}{3.3.2}
When $k = \C$, the conclusion of Theorem 1 can be strengthened. Namely, one can show
that there is a variety $W \subsetneq \V(I_m)$ such that $\V(I_m) \setminus W ⊆ F(\C^m)$. Prove this using the
Closure Theorem.
\end{ejercicio}
\begin{solucion}
Como $F(\C^m)=\pi_m(V)$ por el Ejercicio \ref{ejer:3.3.1}, usando el teorema de la clausura (ii) tenemos el resultado. 

\end{solucion}


\newpage

\begin{ejercicio}{3.3.3}
Give an example to show that Exercise \ref{ejer:3.3.2} is false over $\R$. Hint: $t^2$ is always positive.
\end{ejercicio}
\begin{solucion}
 Consideremos $F:\R\to \R$ dada por $t\mapsto t^2$. $F(\R)=\R_+$. Si consideramos el ideal $I=\gene{x-t^2}\subset \R[t,x]$, lo tenemos expresado con una base de Gröbner. Entonces, $I_1=I\cap k[x]=\{0\}$, por lo que $\V(I_1)=\R$. Los conjuntos algebraicos de $\R$ son los conjuntos finitos y el total, por lo que no es posible encontrar $W\subsetneq\R$ de modo que $\R\setminus W\subseteq \R_+$. 
\end{solucion}


\newpage

\begin{ejercicio}{3.3.4}
In the text, we proved that over $\C$, the tangent surface to the twisted cubic is defined by 	the equation
\[g_7 = x^3z − (3/4)x^2y^2 − (3/2)xyz + y^3 + (1/4)z^2 = 0.\]
We want to show that the same is true over $\R$. If $(x, y, z)$ is a real solution of the above
equation, then we proved (using the Extension Theorem) that there are $t, u ∈ \C$ such that
\begin{align*}
&x = t + u,\\
&y = t^2 + 2tu,\\
&z = t^3 + 3t^2u.
\end{align*}
Use the Gröbner basis given in the text to show that $t$ and $u$ are real. This will prove that
$(x, y, z)$ is on the tangent surface in $\R^3$. Hint: First show that $u$ is real.
\end{ejercicio}
\begin{solucion}\
Utilizamos la base de Gröbner que se da en el texto. Sea una solución $(x,y,z)\in \R^3 \subset \C^3$, sabemos que existen $t,u\in \C$ tal que $(t,u,x,y,z) \in \V(I)$. Por tanto, esta tripleta anula la base de Gröbner y de $g_3$ deducimos
$$
0 = u(x^2-y) -x^3 +(3/2)xy -(1/2)z
$$
Si $x^2-y=0$, por $g_2$ tenemos que $u^2 = 0$, luego $u=0\in \R$. En otro caso,
$$
u = \frac{x^3 -(3/2)xy + (1/2)z}{x^2-y}
$$
Por lo que $u \in \R$. Por $g_1$, tenemos que $t=x-u$, por lo que $t\in \R$.
\end{solucion}


\newpage

\begin{ejercicio}{3.3.5}
In the parametrization of the tangent surface of the twisted cubic, show that the parameters
$t$ and $u$ are uniquely determined by $x$, $y$, and $z$. Hint: The argument is similar to what
you did in Exercise \ref{ejer:3.3.4}.
\end{ejercicio}
\begin{solucion}
Efectivamente, dados $(x,y,z)$ siguiente el argumento del ejercicio anterior obtenmos los $t,u$.
\end{solucion}

\newpage

\begin{ejercicio}{3.3.6}
Let $S$ be the parametric surface defined by
\begin{align*}
&x = uv,\\
&y = u^2,\\
&z = v^2.
\end{align*}
\begin{enumerate}[a.]
\item Find the equation of the smallest variety $V$ that contains $S$.
\item Over $\C$, use the Extension Theorem to prove that $S = V$. Hint: The argument is
similar to what we did for the tangent surface of the twisted cubic.
\item Over $\R$, show that $S$ only covers “half” of $V$. What parametrization would cover the
other “half”?
\end{enumerate}
\end{ejercicio}
\begin{solucion}
\begin{enumerate}[a.]
\item[]
\item Con SAGE obtenemos la base de Gröbner
$$
G=\{v^2 - z, uz - vx, u^2 - y, x^2 - yz, uv - x, ux - vy\}
$$
Luego la ecuación es $x^2-yz$.
\item Tenemos que que $I_2 = \{v^2-z,x^2-yz\}$. Como el primer polinomio tiene coeficiente constante en la mayor potencia de $v$, tenemos que por el Teorema de Extensión cualquier solución $(x,y,z)\in \V(I_2)$ se extiende a una solución $(u,x,y,z)\in \V(I_1)$. Análogamente, tenemos que los coeficientes líderes de las mayores potencias de $u$ definen la variedad
$$
\V(0,z,1,0,v,y) = \emptyset
$$
de donde se deduce el resultado volviendo a aplicar el Teorema de Extensión.
\item Sobre $\R$ no cubre valores negativos de $y$ ni de $z$, cosa que en $\C$ sí. Para la otra mitad simplemente deberíamos parametrizar
\begin{align*}
&x = uv,\\
&y = -u^2,\\
&z = -v^2.
\end{align*}
Esto es suficiente pues, al ser $x^2=yz$, necesitamos que el producto $yz>0$. 
\end{enumerate}
\end{solucion}
\newpage

\begin{ejercicio}{3.3.7}
Let $S$ be the parametric surface
\[x = uv,\]
\[y = uv^2,\]
\[z = u^2.\]
\begin{enumerate}[a.]
\item Find the equation of the smallest variety $V$ that contains $S$.
\item Over $\C$, show that $V$ contains points which are not on $S$. Determine exactly which
points of $V$ are not on $S$. Hint: Use lexicographic order with $u > v > x > y > z$.
\end{enumerate}
\end{ejercicio}
\begin{solucion}\
\begin{enumerate}[a.]
\item Con el orden indicado en el enunciado encontramos la base de Gröbner
\[
\{u^2 - z, uv - x, ux - vz, uy - x^2, v^2z - x^2, vx - y, vyz - x^3, x^4 - y^2z\}
\]
así que la ecuación es $x^4-y^2z=0$. 
\item Usamos el teorema de Extensión reiteradamente. Una solución $(x,y,z)\in\V(x^4-y^2z)=\V(I_2)$ se extiende a $(v,x,y,z)\in \V(I_1)$ siempre que $(x,y,z)\notin\V(z,x,yz)=\V(z,x)$, es decir, siempre que $xz\neq 0$. Por último, una solución $(v,x,y,z)\in\V(I_1)$ se extiende a una solución $(u,v,x,y,z)\in\V(I)$ siempre pues $u^2$ tiene un coeficiente constante. 

En el caso de que $x=0$, en $S$ esto siempre implica que $y=0$, pero en la ecuación $x^4-y^2z=0$, $x=0\Rightarrow y^2z=0$, por lo que lo puntos de la forma $(0,y,0)$ están contenidos en $V$ para todo $y$. 

En el caso $z=0$ no hay nada más que añadir, ya que en la parametrización esto implica $y=x=0$, y en la ecuación implica $x^4=0\Rightarrow x=0$. 
\end{enumerate}
\end{solucion}


\newpage

\begin{ejercicio}{3.3.8}
The Enneper surface is defined parametrically by
\[x = 3u + 3uv^2 − u^3,\]
\[y = 3v + 3u^2v − v^3,\]
\[z = 3u^2 − 3v^2.\]
\begin{enumerate}[a.]
\item Find the equation of the smallest variety $V$ containing the Enneper surface. It will be
a very complicated equation!
\item  Over $\C$, use the Extension Theorem to prove that the above equations parametrize
the entire surface $V$. Hint: There are a lot of polynomials in the Gröbner basis. Keep
looking—you will find what you need.
\end{enumerate}

\end{ejercicio}
\begin{solucion}\
\begin{enumerate}[a.]
\item Hay 15 polinomios muy largos en la base de Gröbner así que solo vamos a escribir el que nos da la ecuación
\begin{gather*}
x^6 + 3x^4y^2 - 1/9x^4z^3 + 3x^4z - 54x^4 + 3x^2y^4 - 2/9x^2y^2z^3 + 6x^2y^2z + 324x^2y^2\\
 + 1/243x^2z^6 - 5/9x^2z^4 + 9x^2z^2 +729x^2 + y^6 - 1/9y^4z^3 + 3y^4z - 54y^4 + 1/243y^2z^6\\ 
 - 5/9y^2z^4  + 9y^2z^2  + 729y^2 - 1/19683z^9 + 4/243z^7 - 2z^5 + 108z^3 - 2187z
\end{gather*}
\item Entre todos esos polinomios podemos encontrar coeficientes constantes en las máximas potencias de $u$ y de $v$ por lo que siempre podemos extender todas las soluciones, lo cual prueba que la parametrización cubre todo $V$ en $\C$. 
\end{enumerate}
\end{solucion}


\newpage

\begin{ejercicio}{3.3.9}
The Whitney umbrella surface is given parametrically by
\[x = uv,\]
\[y = v,\]
\[z = u^2.\]
\begin{enumerate}[a.]
\item Find the equation of the smallest variety containing the Whitney umbrella.
\item Show that the parametrization fills up the variety over $\C$ but not over $\R$. Over $\R$,
exactly what points are omitted?
\item Show that the parameters $u$ and $v$ are not always uniquely determined by $x$, $y$, and $z$.
Find the points where uniqueness fails and explain how your answer relates to the
picture.
\end{enumerate}
\end{ejercicio}
\begin{solucion}\
\begin{enumerate}[a.]
\item Tenemos la base de Gröbner 
\[
\{u^2 - z, ux - yz, uy - x, v - y, x^2 - y^2z\}
\]
por lo que la ecuación es $x^2-y^2z$. 
\item En $\C$, por el teorema de extensión primero podemos encontrar el parámetro $v$ por tener $v$ coeficiente constante y a continuación podemos encontrar $u$ por tener $u^2$ coeficiente constante. NO VEO CUÁLES FALTAN SOBRE R
\item Está claro que $v$ está totalmente determinado por el valor de $y$. Por otra parte $z$ nos determina dos posibles valores de $u$ si $z\neq 0$ y un solo valor si $z=0$. En este último caso tenemos entonces ambos parámetros determinados. Sin embargo, en el otro caso, si además añadimos que $y=v=0$, tenemos que necesariamente $x=0$, por lo que cualquiera de los dos valores de $u$ sería válido. En cualquier caso en el que $x\neq 0$ podríamos despejar $u$ pues $v$ tampoco podría ser 0. Así que los puntos donde no está determinado son los del eje $Z$, que se corresponden en el dibujo de la variedad con la línea de intersección consigo misma. 
\end{enumerate}
\end{solucion}


\newpage

\begin{ejercicio}{3.3.10}
Consider the curve in $\C^n$ parametrized by $x_i = f_i(t)$, where $f_1,\dots , f_n$ are polynomials in
$\C[t]$. This gives the ideal
\[I = 
\gene{x_1 − f_1(t), \dots , x_n − f_n(t)} ⊆ \C[t, x_1, \dots , x_n].\]
\begin{enumerate}[a.]
\item Prove that the parametric equations fill up all of the variety $\V(I_1) ⊆ \C^n$.
\item Show that the conclusion of part (a) may fail if we let $f_1,\dots  , f_n$ be rational functions.
Hint: See §3 of Chapter 1.
\item Even if all of the $f_i$’s are polynomials, show that the conclusion of part (a) may fail if
we work over $\R$.
\end{enumerate}
\end{ejercicio}
\begin{solucion}
\end{solucion}


\newpage

\begin{ejercicio}{3.3.11}
This problem is concerned with the proof of Theorem 2.
\begin{enumerate}[a.]
\item Take $h ∈ k[x_1, \dots , x_n$] and let $f_i$, $g_i$ be as in the theorem with $g = g_1 \cdots g_n$. Show
that if $N$ is sufficiently large, then there is $F ∈ k[t_1, \dots , t_m, x_1, \dots , x_n]$ such that
$g^Nh = F(t_1, \dots , t_m, g_1x_1, \dots , g_nx_n)$.
\item Divide $F$ from part (a) by $x_1 − f_1, \dots , x_n − f_n$. Then, in this division, replace $x_i$ with
$g_ix_i$ to obtain (10).
\item Let $k$ be an infinite field and let $f , g ∈ k[t_1, \dots , t_m]$. Assume that $g \neq 0$ and that $f$
vanishes on $k^m \ \V(g)$. Prove that $f$ is the zero polynomial. Hint: Consider $fg$.
\item Complete the proof of Theorem 2 using ideas from the proof of Theorem 1.
\end{enumerate}
\end{ejercicio}
\begin{solucion}
\end{solucion}


\newpage

\begin{ejercicio}{3.3.12}
Consider the parametrization (6) given in the text. For simplicity, let $k = \C$. Also let
$I = 
\gene{vx − u^2, uy − v^2, z − u}$ be the ideal obtained by “clearing denominators.”
\begin{enumerate}[a.]
\item Show that $I_2 = 
\gene{z(x^2y − z^3)}$.
\item Show that the smallest variety in $\C^5$ containing $i(\C^2 \ W)$ [see diagram (8)] is the
variety $\V(vx − u^2, uy − v^2, z − u, x^2y − z^3, vz − xy)$. Hint: Show that $i(\C^2 \ W) =
π_1(\V(J))$, and then use the Closure Theorem.
\item Show that $\{(0, 0, x, y, 0) | x, y ∈ \C\} ⊆ \V(I)$ and conclude that $\V(I)$ is not the
smallest variety containing $i(\C^2 \ W)$.
\item Determine exactly which portion of $x^2y = z^3$ is parametrized by (6).
\end{enumerate}
\end{ejercicio}
\begin{solucion}
\end{solucion}


\newpage

\begin{ejercicio}{3.3.13}
Given a rational parametrization as in (7), there is one case where the naive ideal $I =
\gene{g_1x_1 − f_1, \dots , g_nx_n − f_n}$ obtained by “clearing denominators” gives the right answer.
Suppose that $x_i = f_i(t)/g_i(t)$ where there is only one parameter $t$.We can assume that for
each $i$, $f_i(t)$ and $g_i(t)$ are relatively prime in $k[t]$ (so in particular, they have no common
roots). If $I ⊆ k[t, x_1, \dots , x_n]$ is as above, then prove that $\V(I_1)$ is the smallest variety
containing $F(k \ W)$, where as usual $g = g_1 \cdots g_n ∈ k[t]$ and $W = \V(g) ⊆ k$. Hint: In
diagram (8), show that $i(k \ W) = \V(I)$, and adapt the proof of Theorem 2.
\end{ejercicio}
\begin{solucion}
\end{solucion}


\newpage

\begin{ejercicio}{3.3.14}
The folium of Descartes can be parametrized by
\[x =
\frac{3t}{1 + t^3} ,\]
\[y =
\frac{3t^2}{1 + t^3} .\]
\begin{enumerate}[a.]
\item Find the equation of the folium. Hint: Use Exercise \ref{ejer:3.3.13}.
\item Over $\C$ or $\R$, show that the above parametrization covers the entire curve.
\end{enumerate}
\end{ejercicio}
\begin{solucion}
\end{solucion}


\newpage

\begin{ejercicio}{3.3.15}
In Exercise 16 to §3 of Chapter 1, we studied the parametric equations over $\R$
\[x =
\frac{(1 − t)^2x_1 + 2t(1 − t)wx_2 + t^2x_3}{
(1 − t)^2 + 2t(1 − t)w + t^2 } ,\]
\[y =
\frac{(1 − t)^2y_1 + 2t(1 − t)wy_2 + t^2y_3}{
(1 − t)^2 + 2t(1 − t)w + t^2} ,\]
where $w$, $x_1$, $y_1$, $x_2$, $y_2$, $x_3$, $y_3$ are constants and $w > 0$. By eliminating $t$, show that these
equations describe a portion of a conic section. Recall that a conic section is described
by an equation of the form
\[ax^2 + bxy + cy^2 + dx + ey + f = 0.\]
Hint: In most computer algebra systems, the Gröbner basis command allows polynomials
to have coefficients involving symbolic constants like $w, x_1$, $y_1$, $x_2$, $y_2$, $x_3$, $y_3$.
\end{ejercicio}
\begin{solucion}
\end{solucion}

\end{document}
