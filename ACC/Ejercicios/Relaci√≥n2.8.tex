\documentclass[twoside]{article}
\usepackage{../../estilo-ejercicios}
\newcommand{\lex}{<_{lex}}
\newcommand{\grlex}{<_{grlex}}
\newcommand{\grevlex}{<_{grevlex}}

\newcommand{\PhantC}{\phantom{\colon}}%
\newcommand{\CenterInCol}[1]{\multicolumn{1}{c}{#1}}%

%--------------------------------------------------------
\begin{document}

\title{Ejercicios de Ideals, Varieties, and Algorithms (4ª Edición)}
\author{Diego Pedraza López, Javier Aguilar Martín, Rafael González López}
\maketitle

\begin{ejercicio}{2.8.1}
Determine whether $f = xy^3 − z^2 + y^5 − z^3$ is in the ideal $I = 
\gene{−x^3 + y, x^2y − z}$.
\end{ejercicio}

\begin{solucion}

\end{solucion}

\newpage

\begin{ejercicio}{2.8.2}
Repeat Exercise \ref{ejer:2.8.1} for $f = x^3z − 2y^2$ and $I = 
\gene{xz − y, xy + 2z^2, y − z}$.
\end{ejercicio}
\begin{solucion}

\end{solucion}
\newpage

\begin{ejercicio}{2.8.3}
By the method of Examples 2 and 3, find the points in $\C^3$ on the variety
$$\V(x^2 + y^2 + z^2 − 1, x^2 + y^2 + z^2 − 2x, 2x − 3y − z).$$
\end{ejercicio}
\begin{solucion}

\end{solucion}

\newpage

\begin{ejercicio}{2.8.4}
Repeat Exercise \ref{ejer:2.8.3} for $\V(x^2y − z^3, 2xy − 4z − 1, z − y^2, x3 − 4zy)$.
\end{ejercicio}
\begin{solucion}

\end{solucion}
\newpage

\begin{ejercicio}{2.8.5}
Recall from calculus that a critical point of a differentiable function $f (x, y)$ is a point
where the partial derivatives $\frac{∂f}
{∂x}$ and $\frac{∂f}
{∂y}$ vanish simultaneously. When $f ∈ \R[x, y]$, it
follows that the critical points can be found by applying our techniques to the system of
polynomial equations
$$\frac{∂f}
{∂x}
=
\frac{∂f}
{∂y}
= 0.$$
To see how this works, consider the function

$$f (x, y) = (x^2 + y^2 − 4)(x^2 + y^2 − 1) + (x − 3/2)^2 + (y − 3/2)^2.$$
\begin{enumerate}[a.]
\item Find all critical points of $f (x, y)$.
\item Classify your critical points as local maxima, local minima, or saddle points. Hint:
Use the second derivative test.
\end{enumerate}
\end{ejercicio}
\begin{solucion}

\end{solucion}

\newpage

\begin{ejercicio}{2.8.6}
Fill in the details of Example 5. In particular, compute the required Gröbner basis, and
verify that this gives us (up to a constant multiple) the polynomial appearing on the
left-hand side of equation (4).
\end{ejercicio}
\begin{solucion}

\end{solucion}

\newpage

\begin{ejercicio}{2.8.7}
Let the surface $S$ in $\R^3$ be formed by taking the union of the straight lines joining pairs
of points on the lines 
$$\left\lbrace
\begin{array}{c}
x = t\\
y = 0\\
z = 1
\end{array}
\right\rbrace,
\left\lbrace
\begin{array}{c}
x = 0\\
y = 1\\
z = t\\
\end{array}
\right\rbrace$$
with the same parameter value (i.e., the same $t$). (This is a special example of a class of
surfaces called ruled surfaces.)
\begin{enumerate}[a.]
\item Show that the surface S can be given parametrically as
\begin{align*}
&x = ut,\\
&y = 1 − u,\\
&z = t + u(1 − t).
\end{align*}
\item Using the method of Examples 4 and 5, find an (implicit) equation of a variety $V$
containing the surface $S$.
\item Show $V = S$ (that is, show that every point of the variety V can be obtained by
substituting some values for $t$, $u$ in the equations of part (a). Hint: Try to “solve” the
implicit equation of $V$ for one variable as a function of the other two.
\end{enumerate}
\end{ejercicio}
\begin{solucion}\

\end{solucion}

\newpage

\begin{ejercicio}{2.8.8}
Some parametric curves and surfaces are algebraic varieties even when the given
parametrizations involve transcendental functions such as sin and cos. In this problem,
we will see that the parametric surface $T$,

\begin{align*}
&x = (2 + cos(t)) cos(u),\\
&y = (2 + cos(t)) sin(u),\\
&z = sin(t),
\end{align*}
lies on an affine variety in $\R^3$.
\begin{enumerate}[a.]
\item Draw a picture of $T$. Hint: Use cylindrical coordinates.
\item Let $a = \cos(t)$, $b = \sin(t)$, $c = \cos(u)$, $d = \sin(u)$, and rewrite the above equations
as polynomial equations in $a, b, c, d, x, y, z$.
\item The pairs $a$, $b$ and $c$, $d$ in part (b) are not independent since there are additional polynomial
identities
$$a^2 + b^2 − 1 = 0, c^2 + d^2 − 1 = 0$$
stemming from the basic trigonometric identity. Form a system of five equations by
adjoining the above equations to those from part (b) and compute a Gröbner basis for
the corresponding ideal. Use the lex monomial ordering and the variable order
$$a > b > c > d > x > y > z.$$
There should be exactly one polynomial in your basis that depends only on $x, y, z$.
This is the equation of a variety containing $T$.
\end{enumerate}
\end{ejercicio}
\begin{solucion}
\begin{enumerate}
\item[]

\end{enumerate}

\end{solucion}

\newpage

\begin{ejercicio}{2.8.9}
Consider the parametric curve $K ⊆ \R^3$ given by
\begin{align*}
&x = (2 + cos(2s))cos(3s),\\
&y = (2 + cos(2s))sin(3s),\\
&z = sin(2s).
\end{align*}
\begin{enumerate}[a.]
\item Express the equations of $K$ as polynomial equations in $x, y, z$, $a = \cos(s)$, $b = \sin(s)$.
Hint: Trig identities.
\item By computing a Gröbner basis for the ideal generated by the equations from part (a)
and $a^2+b^2−1$ as in Exercise \ref{ejer:2.8.8}, show that $K$ is (a subset of) an affine algebraic curve.
Find implicit equations for a curve containing $K$.
\item Show that the equation of the surface from Exercise \ref{ejer:2.8.8} is contained in the ideal generated
by the equations from part (b). What does this result mean geometrically? (You
can actually reach the same conclusion by comparing the parametrizations of $T$ and
$K$, without calculations.)
\end{enumerate}
\end{ejercicio}
\begin{solucion}
\begin{enumerate}
\item[]

\end{enumerate}
\end{solucion}

\newpage

\begin{ejercicio}{2.8.10}
Use the method of Lagrange multipliers to find the point(s) on the surface defined by
$x^4 + y^2 + z^2 − 1 = 0$ that are closest to the point $(1, 1, 1)$ in $\R^3$. Hint: Proceed as in
Example 3. (You may need to “fall back” on a numerical method to solve the equations
you get.)
\end{ejercicio}
\begin{solucion}
\end{solucion}

\newpage

\begin{ejercicio}{2.8.11}
Suppose we have numbers a, b, c which satisfy the equations
\begin{align*}
a + b + c = 3,\\
a2 + b2 + c2 = 5,\\
a3 + b3 + c3 = 7.
\end{align*}
\begin{enumerate}[a.]
\item Prove that $a^4 +b^4 +c^4 = 9$. Hint: Regard $a, b, c$ as variables and show carefully that
$a^4 + b^4 + c^4 − 9 ∈ 
\gene{a + b + c − 3, a^2 + b^2 + c^2 − 5, a^3 + b^3 + c^3 − 7}$.
\item Show that $a^5 + b^5 + c^5 \neq 11$.
\item What are $a^5 + b^5 + c^5$ and $a^6 + b^6 + c^6?$ Hint: Compute remainders.
\end{enumerate}
\end{ejercicio}
\begin{solucion}

\end{solucion}


\end{document}
