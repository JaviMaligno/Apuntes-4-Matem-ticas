\documentclass[twoside]{article}
\usepackage{../../estilo-ejercicios}

%--------------------------------------------------------
\begin{document}

\title{Ejercicios de Ideals, Varieties, and Algorithms (4ª Edición)}
\author{Diego Pedraza López, Javier Aguilar Martín, Rafael González López}
\maketitle

\begin{ejercicio}{3.5.1}
As in Lemma 1, let $f = \sum_{j=1}^t A_j g_j$ be a standard representation and set $N = deg(f,x_1)$.
\begin{enumerate}[a.]
\item Prove that $N \geq deg(A_j g_j, x_1)$ when $A_j g_j \neq 0$.
Hint: Recall that $multideg(f) \geq multideg(A_j g_j)$ when $A_j g_j \neq 0$.
Then explain why the first entry in $multideg(f)$ is $deg(f,x_1)$.
\item Prove that
\[ c_f = \sum_{\deg(A_j g_j, x_1) = N} c_{A_j}c_{g_j} \]
Hint: Use part (a) and compare the coefficients of $x_1^N$ in $f = \sum_{j=1}^t A_j g_j$.
\end{enumerate}
\end{ejercicio}
\begin{solucion}\mbox{}
\begin{enumerate}[(a)]
\item Tenemos que $multideg(f) \geq multideg(A_j g_j)$ (ver ejercico 2.3.4).
Como estamos en un orden lex con $x_1 > \cdots > x_n$, esto significa que:
\[ N = deg(f,x_1) \geq deg(A_j g_j, x_1) \]
\item Como $f = \sum_{j=1}^t A_j g_j$, $c_f$ será la suma de los coeficientes de $x_1^N$ en cada $ A_j g_j$. Por definición, $\deg(A_j g_j, x_1) = N$ hace que nos quedemos con los $A_jg_j$ en los que aparece $x_1^N$. El coeficiente $c_j$ correspondiente a $x_1^N$ en $A_jg_j$ se obtiene al multiplicar dos monomios en los que la suma de los exponentes de $x_1$ es igual a $N$. Los coeficientes de estos monomios deben ser necesariamente $c_{A_j}$ y $c_{g_j}$ necesariamente, pues por el apartado anterior $N \geq deg(A_j g_j, x_1)$. Efectivamente, si $c_{A_j}$ (análogamente $c_{g_j}$) multiplicara a una pontencia mayor de $x_1$ que la que hace que $N = deg(A_j g_j, x_1)$, entonces  $N < deg(A_j g_j, x_1)$, lo cual no es posible. 
\end{enumerate}
\end{solucion}

\newpage

\begin{ejercicio}{3.5.2}
Suppose that $k$ is a field and $\varphi \colon k[x_1,\dots,x_n] \to k[x_1]$ is a ring homomorphism that is the identity on $k$ and maps $x_1$ to $x_1$.
Given an ideal $I \subseteq k[x_1,\dots,x_n]$, prove that $\varphi(I) \subseteq k[x_1]$ is an ideal.
(In the proof of Theorem 2, we use this result when $\varphi$ is the map that evaluates $x_i$ at $a_i$ for $2 \leq i \leq n$.)
\end{ejercicio}
\begin{solucion}
Las hipótesis sobre el homomorfismo nos permiten deducir que este es sobreyectivo, pues para cada monomio $ax_1^n \in k[x_1]$ podemos tomar como preimagen el monomio $ax_1^n \in k[x_1,\dots,x_n]$. Sabemos que la sobreyectividad es suficiente para probar el resultado, pero vamos a verlo explícitamente.
\begin{itemize}
\item Sean $x,y \in \varphi(I)$ entonces $\exists a,b\in I$ tales que $\varphi(a)=x$ y $\varphi(b)=y$. Por tanto, como $a+b \in I$, $\varphi(a+b)=\varphi(a)+\varphi(b)=x+y\in \varphi(I)$.
\item Sean $x \in \varphi(I)$ y $\lambda \in k[x_1]$, entonces tenemos $a\in I$ tal que $\varphi(a)=x$ y, por sobreyectividad, $b\in k[x_1,\dotsc,x_n]$ tal que $\varphi(b)=\lambda$. Sabemos que $ab\in I$, luego $\varphi(ab)=\varphi(a)\varphi(b)=\lambda x \in \varphi(I)$.
\end{itemize}
\end{solucion}

\newpage

\begin{ejercicio}{3.5.3}
In the proof of Theorem 2, show that (1) follows from the assertion that $g_j(x_1,a) \in \gene{g_0(x_1,a)}$ for all $g_j \in G$.
\end{ejercicio}
\begin{solucion}
Si todo $g_j \in G$ verifica que $g_j(x_1,a) \in \gene{g_0(x_1,a)}$, entonces todo polinomio expresable como combinación polinomial de los $g_j(x_1,a)$ también lo verifica. Como $G$ es base, dado $f\in I$, podemos expresarlo como $f=\sum A_jg_j$, así que $f(x_1,a)=\sum A_j(x_1,a)g_j(x_1,a)$. Luego los $g_j$ son una base de $\gene{g_0(x_1,a)}$, como queríamos probar.
\end{solucion}
\newpage

\begin{ejercicio}{3.5.4}
This exercise will explore the example $I = \gene{x^2y+xz+1, xy-xz^2+z-1}$ discussed in this text.
\begin{enumerate}[a.]
\item Show that the partial solution $(b,c) = (0,0)$ does not extend to a solution $(a,0,0) \in \V(I)$.
\item In the text, we showed that $g_o = g_1$ for the partial solution $(1,1)$.
Show that $g_o = g_3$ works for all partial solutions different from $(1,1)$ and $(0,0)$.
\end{enumerate}
\end{ejercicio}
\begin{solucion}
\begin{enumerate}[a.]
\item[]
\item Con la base de Gröbner del libro, al sustituir $y=z=0$ en $g_2$ obtenemos un polinomio constante igual -1, por lo que no hay ninguna solución para este polinomio. 
\item Como $g_3$ tiene grado 1 en $x$, por el Teorema 2 basta que $c_3(a)\neq 0$ para que $g_3=g_o$ dada una solución parcial $a\in\V(I_1)$. Recordemos que
\[
g_3=(z^3 − z^2)x − y + z^2 + z − 1,
\]
de modo que $c_3=z^3-z^2$, que se anula en $z=0$ y en $z=1$. Por tanto, $g_3=g_o$ para cualquier solución parcial en la que $z$ no sea ni 0 ni 1. En particular será soluciones distintas de $(1,1)$ y $(0,0)$, pero veamos  si $z=1$ implica $y=1$ y si $z=0$ implica $y=0$. Para ello basta sustituir en $g_4=y^2 − 2yz^2 − yz + y + 2z^4 − z^3$ (que era el polinomio que generaba $I_1$):
\[
g_4(y,0)=y^2+y=0\Rightarrow y=0, y=-1
\]
\[
g_4(y,1)=y^2-2y-y+y+2-1=y^2-2y+1=0\Rightarrow y=1
\]
Vemos que tenemos además otra solución parcial para la cual no vale $g_3$, que es $(-1,0)$. Para ese caso sería óptimo $g_2=(y − z^2)x + z − 1$. Así que $g_o=g_3$ para toda solución parcial diferente de $(1,1)$, $(0,0)$ y $(-1,0)$. 
\end{enumerate}
\end{solucion}

\newpage

\begin{ejercicio}{3.5.5}
Evaluation at $a$ is sometimes called \emph{specialization}.
Given $I \subseteq k[x_1,\dots,x_n]$ with lex Gröbner basis $G = \{g_1,\dots,g_t\}$, we get the specialized basis $\{g_1(x_1,a),\dots,g_t(x_1,a)\}$.
Discarding the polynomials that specialize to zero, we get $G' = \{g_j(x_1,a) \mid g_j(x_1,a) \neq 0\}$.
\begin{enumerate}[a.]
\item Show that $G'$ is a basis of the ideal $\{f(x_1,a) \mid f \in I\} \subseteq k[x_1]$.
\item If in addition $a \in \V(I_1)$ is a partial solution satisfying the hypothesis of Theorem 2, prove that $G'$ is a Gröbner basis of $\{f(x_1,a) \mid f \in I\}$.
\end{enumerate}
The result of part (b) is an example of a \emph{specialization theorem} for Gröbner bases.
We will study the specialization of Gröbner bases in more detail in Chapter 6.
\end{ejercicio}
\begin{solucion}
\begin{enumerate}[a.]
\item[]
\item Análogo al Ejercicio \ref{ejer:3.5.3}.
\item En las condiciones del Teorema 2 se tiene que existe $g_o\in G$ de modo que $\{f(x_1,a) \mid f \in I\}=\gene{g_o(x_1,a)}\in k[x_1]$ y además $g_o$ es minimal en cuanto al grado de $x_1$. Por tanto, $g_o$ forma una base de Gröbner del ideal, así que también forma una base de Gröbner cualquier base obtenida añadiendo elementos no nulos del ideal al conjunto unitario formado por $g_o$. 

ESTOY BASTANTE SEGURO DE QUE ESTO ESTÁ MAL
\end{enumerate}
\end{solucion}
\newpage

\begin{ejercicio}{3.5.6}
Show that Theorem 2 remains true if we replace lex order for $x_1 > \cdots > x_n$ with any monomial order for which $x_1$ is greater that all monomials in $x_2,\dots,x_n$.
This is an order of $1$-elimination type in the terminology of Exercise 3.1.5.
Hint: You will need to show that Lemma 1 of this section holds for such monomial orders.
\end{ejercicio}
\begin{solucion}
En ningún momento de la prueba del Lemma 1 (ejercicio \ref{ejer:3.5.1}) ni del Teorema 2 se usa el orden particular del resto de variables, así que sería análogo. Es más, en lugar de $x_1$ podríamos fijar cualquier otra variable como mayor de todas y seguiría siendo análogo.
\end{solucion}
\newpage

\begin{ejercicio}{3.5.7}
Use the strategy explained in the discussion following Theorem 2 to find all solutions of the system of equations given in Example 3 of Chapter 2, section 8.
\end{ejercicio}
\begin{solucion}
Recordemos que en ese ejemplo, tras obtener la base de Gröbner nos quedó el sistema
\begin{align*}
&g_1=λ − \frac{3}{2}x − \frac{3}{2}yz − \frac{167616}{3835}z^6 +\frac{36717}{590}z^4 − \frac{134419}{7670}z^2,\\
&g_2=x^2 + y^2 + z^2 − 1,\\
&g_3=xy − \frac{19584}{3835}z^5 +\frac{1999}{295}z^3 − \frac{6403}{3835}z,\\
&g_4=xz + yz^2 − \frac{1152}{3835}z^5 − \frac{108}{295}z^3 +\frac{2556}{3835}z,\\
&g_5=y^3 + yz^2 − y − \frac{9216}{3835}z^5 +\frac{906}{295}z^3 − \frac{2562}{3835}z,\\
&g_6=y^2z − \frac{6912}{3835}z^5 +\frac{827}{295}z^3 − \frac{3839}{3835}z,\\
&g_7=yz^3 − yz − \frac{576}{59}z^6 +\frac{1605}{118}z^4 − \frac{453}{118}z^2,\\
&g_8=z^7 − \frac{1763}{1152}z^5 +\frac{655}{1152}z^3 − \frac{11}{288}z.
\end{align*}
Además se habían encontrado soluciones para el último polinomio, es decir soluciones parciales en $\V(I_3)$
\[
z = 0, ±1, ±2/3, ±\sqrt{11}/8\sqrt{2}.
\]
Tenemos que $I_2=\gene{g_5,g_6,g_7,g_8}$, por lo que las soluciones parciales anteriores se podrán extender a $\V(I_2)$ siempre que no estén en $\V(c_5,c_6,c_7)=\V(z^3,z,1)=\emptyset$, es decir, que siempre se puede extender a una solución parcial de $\V(I_2)$. Buscamos ahora el polinomio óptimo para extender cada solución parcial de $\V(I_1)$. Para $z=0$ es claro que se trata de $g_5$.
\[
g_5(y,0)=y^3-y\Rightarrow y=0,y=1,y=-1
\]
Para $\pm 1$ es $g_6$, pues el coeficiente de $y$ en $g_7$ es $z^3-z$.
\[
g_6(y,1)=y^2 − \frac{6912}{3835} +\frac{827}{295} − \frac{3839}{3835}=y^2=0\Rightarrow y=0
\]
\[
g_6(y,-1)=-g_6(y,1)
\]
Para el resto de soluciones, el polinomio óptimo es $g_7$. 
\[
g_7(y,2/3)=10/81 - 10y/27=0\Rightarrow y=1/3
\]
\[
g_7(y,-2/3)=10/81 - 10y/27=0\Rightarrow y=-1/3
\]
\[
g_7(y,\sqrt{11}/8\sqrt{2})=y = -351 \sqrt{11/2}/320
\]
\[
g_7(y,-\sqrt{11}/8\sqrt{2})=351 \sqrt{11/2}/320
\]
Recapitulando, hemos encontrado las soluciones parciales
\[
(0,0), (1,0), (-1,0), (0,1), (0,-1), (1/3,2/3), (-1/3,-2/3), 
\]
\[
(-351 \sqrt{11/2}/320,\sqrt{11}/8\sqrt{2}), (351 \sqrt{11/2}/320,-\sqrt{11}/8\sqrt{2}).
\]
Ahora tenemos que ver cuáles se extienden a $\V(I_1)=\V(g_2,\dots, g_8)$. Serán las que no pertenezcan a $\V(c_2, c_3,c_4)=\V(1,y,z)=\emptyset$, es decir, todas. Para $(0,0)$ el polinomio óptimo es $g_2$, que nos da las soluciones parciales $(\pm 1,0,0)$. Para $(1,0)$ y $(-1,0)$ el óptimo es $g_3$, que nos da  respectivamente $(0,\pm 1, 0)$. Para $(0,1)$ y $(0,-1)$ el óptimo es $g_4$, que nos da las soluciones $(0,0,\pm 1)$. 

Para el resto de soluciones podemos elegir indistintamente $g_3$ o $g_4$ como polinomio óptimo. Lo haremos con $g_3$ por tener menos términos. Haciendo las sustituciones pertinentes obtenemos el conjunto de soluciones parciales de $\V(I_1)$ que aparece en el libro:

\begin{align*}
&z = 0; y = 0; x = ±1,\\
&z = 0; y = ±1; x = 0,\\
&z = ±1; y = 0; x = 0,\\
&z = 2/3; y = 1/3; x = −2/3,\\
&z = −2/3; y = −1/3; x = −2/3,\\
&z =\sqrt{11}/8\sqrt{2}; y = −3\sqrt{11}/8\sqrt{2}; x = −3/8,\\
&z = −\sqrt{11}/8\sqrt{2}; y = 3\sqrt{11}/8\sqrt{2}; x = −3/8.
\end{align*}

Finalmente, todas las soluciones parciales encontradas anteriormente se extienden a $\V(I)$ por tener $\lambda$ coeficiente constante, de modo que todas las soluciones se obtienen al despejar $\lambda$ de $g_1$. 
\end{solucion}
\end{document}
