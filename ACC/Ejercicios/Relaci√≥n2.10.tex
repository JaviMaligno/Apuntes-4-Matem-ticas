\documentclass[twoside]{article}
\usepackage{../../estilo-ejercicios}
\newcommand{\lex}{<_{lex}}
\newcommand{\grlex}{<_{grlex}}
\newcommand{\grevlex}{<_{grevlex}}

\newcommand{\PhantC}{\phantom{\colon}}%
\newcommand{\CenterInCol}[1]{\multicolumn{1}{c}{#1}}%

%\usepackage{fourier}
%\usepackage{arcs}
%--------------------------------------------------------
\begin{document}

\title{Ejercicios de Ideals, Varieties, and Algorithms (4ª Edición)}
\author{Diego Pedraza López, Javier Aguilar Martín, Rafael González López}
\maketitle

\begin{ejercicio}{2.10.1}
Let $S = (c_1, \dots , c_s)$ and $T = (d_1, \dots , d_s) ∈ (k[x_1, \dots , x_n])^s$ be syzygies on the leading
terms of $F = (f_1, \dots , f_s)$.
\begin{enumerate}[a.]
\item Show that $S + T = (c_1 + d_1, \dots , c_s + d_s)$ is also a syzygy.
\item Show that if $g ∈ k[x_1,\dots , x_n]$, then $g  S = (gc_1, \dots , gc_s)$ is also a syzygy.
\end{enumerate}
\end{ejercicio}

\begin{solucion}\
\begin{enumerate}[a.]
\item 
\[
\sum_{i=1}^s(c_i+d_i)LT(f_i)=\sum_{i=1}^s c_iLT(f_i)+\sum_{i=1}^s d_iLT(f_i)=0.
\]
\item
\[
\sum_{i=1}^sgc_iLT(f_i)=g\sum_{i=1}^s c_iLT(f_i)=0.
\]
\end{enumerate}
\end{solucion}

\newpage

\begin{ejercicio}{2.10.2}
Given any $G = (g_1, \dots , g_s) ∈ (k[x_1, \dots , x_n])^s$ , we can define a syzygy on $G$ to be an $s$-tuple
$S = (h_1,\dots , h_s) ∈ (k[x_1, \dots , x_n])^s$ such that
$\sum_i h_ig_i = 0$. [Note that the syzygies
we studied in the text are syzygies on $LT(G) = (LT(g_1), \dots , LT(g_s))$.]
\begin{enumerate}[a.]
\item Show that if $G = (x^2 − y, xy − z, y^2 − xz)$, then $(z,−y, x)$ defines a syzygy on $G$.
\item Find another syzygy on $G$ from part (a).
\item Show that if $S, T$ are syzygies on $G$, and $g ∈ k[x_1,\dots , x_n]$, then $S+T$ and $gS$ are also
syzygies on $G$.

\end{enumerate}
\end{ejercicio}
\begin{solucion}\
\begin{enumerate}[a.]
\item 
\[
(x^2-y)z+(xy-z)(-y)+(y^2-xz)x=x^2z-yz-xy^2+yz+xy^2-x^2z=0.
\]
\item Claramente el opuesto al del apartado anterior, $(-z,y, -x)$, también sería un syzygy.
\item Totalmente análogo al ejercicio \ref{ejer:2.10.1}, ya que en ningún momento se usó que $LT(f_i)$ fuera el término líder de un polinomio.
\end{enumerate}
\end{solucion}
\newpage

\begin{ejercicio}{2.10.3}
Let $M$ be an $m × (m + 1)$ matrix of polynomials in $k[x_1, \dots , x_n]$. Let $I$ be the ideal generated
by the determinants of all the $m × m$ submatrices of $M$ (such ideals are examples
of determinantal ideals).
\begin{enumerate}[a.]
\item Find a $2×3$ matrix $M$ such that the associated determinantal ideal of $2×2$ submatrices
is the ideal with generators $G$ as in part (a) of Exercise \ref{ejer:2.10.2}.
\item Explain the syzygy given in part (a) of Exercise \ref{ejer:2.10.2} in terms of your matrix.
\item Give a general way to produce syzygies on the generators of a determinantal ideal.
Hint: Find ways to produce $(m + 1) × (m + 1)$ matrices containing $M$, whose determinants
are automatically zero.
\end{enumerate}
\end{ejercicio}
\begin{solucion}\
\begin{enumerate}[a.]
\item 
\[
\begin{pmatrix}
 x& y & z\\
 1& x & y \\
\end{pmatrix}
\]

\item Consideramos la matriz de determinante nulo
\[
\begin{pmatrix}
 x& y & z\\
 1& x & y \\
 x &y & z
\end{pmatrix}
\]
Podemos de hecho calcular de hecho su determinante desarrollando por la tercera fila, lo cual nos daría justamente la combinación lineal del ejercicio \ref{ejer:2.10.2}.

\item En general, dada una matriz polinomial $m\times (m+1)$, podemos añadir una combinación lineal de las $m$ filas como $(m+1)$-ésima fila. El syzygy será la suma alternada de los elementos de dicha fila comenzando con signo $(-1)^{m+1}$.
\end{enumerate}
\end{solucion}

\newpage

\begin{ejercicio}{2.10.4}
Prove that the syzygy $S_{ij}$ defined in (1) is homogeneous of multidegree $γ$.
\end{ejercicio}
\begin{solucion}
Recordamos que el syzygy definido en (1) era el $S$-polinomio genérico
\[
S_{ij}=\frac{x^\gamma}{LT(f_i)}e_i-\frac{x^\gamma}{LT(f_j)}e_j,
\]
con $x^\gamma=\lcm(LM(f_i),LM(f_j))$ y $e_i=(0,\dots,0,1,0,\dots, 0)$. Entonces es claro que es homogéneo de multigrado $\gamma$, pues $\frac{x^\gamma}{LT(f_i)}$ tiene multigrado $\alpha(i)=\gamma-\mathrm{multideg}(f_i)$, por lo que $\alpha(i)+\mathrm{multideg}(f_i)=\gamma$ y análogamente para $j$.
\end{solucion}
\newpage

\begin{ejercicio}{2.10.5}
Complete the proof of Lemma 4 by showing that the decomposition into homogeneous
components is unique. Hint: First show that if $S =
\sum_{α} S_{α}'$, where $S_{α}'$ has multidegree
$α$, then, for a fixed $i$, the $i$-th components of the $S_{α}'$ are either 0 or have multidegree equal
to $α − \textrm{multideg}( f_i)$ and, hence, give distinct terms as $α$ varies.
\end{ejercicio}
\begin{solucion}
Sea $S=\sum_\alpha S_\alpha'$ con $S_\alpha'=(h_{\alpha 1},\dots,h_{\alpha s})$ una descomposición distinta. Por definición de elemento homogéneo, la componente $i$-ésima de $S_\alpha'$ o bien es 0 o bien tiene multigrado igual a $α − \textrm{multideg}( f_i)$, por lo que cada $\alpha$ da un $S_\alpha'$ distinto. Por tanto basta probar que $h_{\alpha i}=h_{\alpha i}'$ para cada $\alpha$. Como $h_{\alpha i}f_i$ tiene el mismo multigrado que $h_{\alpha i}'f_i$, tenemos que  $h_{\alpha i}$ y $h_{\alpha i}'$ son proporcionales. Si no son todos nulos (en cuyo caso ya se tendría la igualdad), debe ser la misma constante de proporcionalidad para todo $i$ para seguir siendo syzygies. Por último, esta constante debe ser 1 para que la suma dé $S$ y no un múltiplo. 
\end{solucion}

\newpage

\begin{ejercicio}{2.10.6}
Suppose that $S$ is a homogeneous syzygy of multidegree $α$ in $S(G)$.
\begin{enumerate}[a.]
\item Prove that $S  G$ has multidegree $< α$.
\item Use part (a) to show that Corollary 7 follows from Theorem 6.
\end{enumerate}
\end{ejercicio}
\begin{solucion}\
\begin{enumerate}[a.]
\item Sea $G=(g_1,\dots, g_s)$ y $S=(h_1,\dots, h_s)$. Entonces $GS=\sum_{i=1}^s g_ih_i$. Tenemos que $\mathrm{multideg}(h_i)=\alpha-\mathrm{multideg}(g_i)$, así que a lo sumo el multigrado de $GS$ podrá ser $\alpha$. Pero la componente de grado $\alpha$ es justamente $\sum_{i=1}^s LT(g_i)h_i=0$ por definición, luego el multigrado de $SG$ es estrictamente menor que $\alpha$. 

\item Si $G$ es un base de Gröbner, entonces $SG$ tiene resto 0 al dividir por $G$, lo cual implica que tiene una representación estándar por el lema 9.2.

Si $SG$ tiene una representación estándar, entonces $\alpha>\mathrm{multideg}(SG)>\mathrm{multideg}(A_ig_i)$, luego por el teorema 6 $G$ es una base de Gröbner. 
\end{enumerate}
\end{solucion}

\newpage

\begin{ejercicio}{2.10.7}
Complete the proof of Proposition 8 by proving the formula expressing $S_{ij}$ in terms of
$S_{il}$ and $S_{jl}$.
\end{ejercicio}
\begin{solucion}
Tenemos que probar la fórmula
\[
S_{ij}=\frac{x^{\gamma_{ij}}}{x^{\gamma_{il}}}S_{il}-\frac{x^{\gamma_{ij}}}{x^{\gamma_{jl}}}S_{jl}.
\]
Para ello desarrollamos el segundo miembro paso a paso.
\[
\frac{x^{\gamma_{ij}}}{x^{\gamma_{il}}}S_{il}=\frac{x^{\gamma_{ij}}}{x^{\gamma_{il}}}\left(\frac{x^{\gamma_{il}}}{LT(f_i)}e_i-\frac{x^{\gamma_{il}}}{LT(f_l)}e_l\right)=\frac{x^{\gamma_{ij}}}{LT(f_i)}e_i-\frac{x^{\gamma_{ij}}}{LT(f_l)}e_l.
\]
Análogamente, 
\[
\frac{x^{\gamma_{ij}}}{x^{\gamma_{jl}}}S_{jl}=\frac{x^{\gamma_{ij}}}{LT(f_j)}e_j-\frac{x^{\gamma_{ij}}}{LT(f_l)}e_l.
\]
Haciendo la diferencia se obtiene el resultado. 
\end{solucion}

\newpage

\begin{ejercicio}{2.10.8}
Let $G$ be a finite subset of $k[x_1, \dots , x_n]$ and let $f ∈ 
\gene{G}$. If $\overline{f}^G = r \neq 0$, then show that
$f →_{G'} 0$, where $G'
= G ∪ \{r\}$. This fact is used in the proof of Theorem 9.
\end{ejercicio}
\begin{solucion}


\end{solucion}

\newpage

\begin{ejercicio}{2.10.9}
In the proof of Theorem 9, we claimed that for every value of $B$, if $1 ≤ i < j ≤ t$ and
$(i, j) \not∈
B$, then condition (6) was true. To prove this, we needed to show that if the
claim held for $B$, then it held when $B$ changed to some $B'$. The case when $(i, j) \not∈
B'$ but
$(i, j) ∈ B$ was covered in the text. It remains to consider when $(i, j) \not∈B' ∪ B$. In this
case, prove that (6) holds for $B'$. Hint: Note that (6) holds for $B$. There are two cases
to consider, depending on whether $B'$ is bigger or smaller than $B$. In the latter situation,
$B'
= B \ \{(l,m)\}$ for some $(l,m) \neq (i, j)$.
\end{ejercicio}
\begin{solucion}

\end{solucion}

\newpage

\begin{ejercicio}{2.10.10}
In this exercise, we will study the ordering on the set $\{(i, j) | 1 ≤ i < j ≤ t\}$ described
in the proof of Theorem 9. Assume that $B = ∅$, and recall that $t$ is the length of $G$ when
the algorithm stops.
\begin{enumerate}[a.]
\item Show that any pair $(i, j)$ with $1 ≤ i < j ≤ t$ was a member of $B$ at some point during
the algorithm.
\item Use part (a) and $B = ∅$ to explain how we can order the set of all pairs according to
when a pair was removed from $B$.
\end{enumerate}
\end{ejercicio}
\begin{solucion}
\end{solucion}

\newpage

\begin{ejercicio}{2.10.11}
Consider $f:1 = x^3−2xy$ and $f_2 = x^2y−2y^2+x$ and use grlex order on $k[x, y]$. These polynomials
are taken from Example 1 of §7, where we followed Buchberger’s algorithm
to show how a Gröbner basis was produced. Redo this example using the algorithm of
Theorem 9 and, in particular, keep track of how many times you have to use the division
algorithm.
\end{ejercicio}
\begin{solucion}


\end{solucion}
\newpage

\begin{ejercicio}{2.10.12}
Consider the polynomials
$$x^{n+1} − yz^{n−1}w, xy^{n−1} − z^n, x^nz − y^nw,$$
and use grevlex order with $x > y > z > w$. Mora [see LAZARD (1983)] showed that the
reduced Gröbner basis contains the polynomial
$$z^{n^2+1} − y^{n^2}
w.$$
Prove that this is true when n is 3, 4, or 5. How big are the Gröbner bases?
\end{ejercicio}
\begin{solucion}
\end{solucion}

\newpage

\begin{ejercicio}{2.10.13}
In this exercise, we will look at some examples of how the term order can affect the
length of a Gröbner basis computation and the complexity of the answer.
\begin{enumerate}[a.]
\item Compute a Gröbner basis for $I = 
\gene{x^5 + y^4 + z^3 − 1, x^3 + y^2 + z^2 − 1}$ using lex and
grevlex orders with $x > y > z$. You will see that the Gröbner basis is much simpler
when using grevlex.
\item Compute a Gröbner basis for $I = 
\gene{x^5 + y^4 + z^3 − 1, x^3 + y^3 + z^2 − 1}$ using lex
and grevlex orders with $x > y > z$. This differs from the previous example by a
single exponent, but the Gröbner basis for lex order is significantly nastier (one of its
polynomials has 282 terms, total degree 25, and a largest coefficient of 170255391).
\item Let $I = 
\gene{x^4 − yz^2w, xy^2 − z^3, x^3z − y^3w}$ be the ideal generated by the polynomials
of Exercise \ref{ejer:2.10.12} with $n = 3$. Using lex and grevlex orders with $x > y > z > w$, show
that the resulting Gröbner bases are the same. So grevlex is not always better than
lex, but in practice, it is usually a good idea to use grevlex whenever possible.
\end{enumerate}
\end{ejercicio}
\begin{solucion}
\end{solucion}

\end{document}
