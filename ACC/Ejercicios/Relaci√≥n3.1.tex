\documentclass[twoside]{article}
\usepackage{../../estilo-ejercicios}

%--------------------------------------------------------
\begin{document}

\title{Ejercicios de Ideals, Varieties, and Algorithms (4ª Edición)}
\author{Diego Pedraza López, Javier Aguilar Martín, Rafael González López}
\maketitle

\begin{ejercicio}{3.1.1}
Let $I\subset k[x_1,\dotsc,x_n]$ be an ideal.
\begin{enumerate}[a.]
\item Prove that $I_l = I\cap k[x_{l+1},\dotsc,x_n]$ is an ideal of $k[x_{l+1},\dotsc,x_n]$.
\item Prove that ideal $I_{l+1} \subset k[x_{l+2},\dotsc,x_n]$ is the first elimination ideal $I_l \subset k[x_{l+1},\dotsc,x_n]$. This observation allows to use the Extension Theorem multiple times when eliminating more than one variable.
\end{enumerate}
\end{ejercicio}
\begin{solucion}
\begin{enumerate}[a.]
\end{enumerate}
\end{solucion}

\newpage

\begin{ejercicio}{3.1.2}

\end{ejercicio}
\begin{solucion}
\begin{enumerate}[a.]
\end{enumerate}
\end{solucion}


\newpage

\begin{ejercicio}{3.1.3}

\end{ejercicio}
\begin{solucion}
\begin{enumerate}[a.]
\end{enumerate}
\end{solucion}


\newpage

\begin{ejercicio}{3.1.4}

\end{ejercicio}
\begin{solucion}
\begin{enumerate}[a.]
\end{enumerate}
\end{solucion}


\newpage

\begin{ejercicio}{3.1.5}

\end{ejercicio}
\begin{solucion}
\begin{enumerate}[a.]
\end{enumerate}
\end{solucion}


\newpage

\begin{ejercicio}{3.1.6}

\end{ejercicio}
\begin{solucion}
\begin{enumerate}[a.]
\end{enumerate}
\end{solucion}


\newpage

\begin{ejercicio}{3.1.7}

\end{ejercicio}
\begin{solucion}
\begin{enumerate}[a.]
\end{enumerate}
\end{solucion}


\newpage

\begin{ejercicio}{3.1.8}

\end{ejercicio}
\begin{solucion}
\begin{enumerate}[a.]
\end{enumerate}
\end{solucion}


\newpage

\begin{ejercicio}{3.1.9}

\end{ejercicio}
\begin{solucion}
\begin{enumerate}[a.]
\end{enumerate}
\end{solucion}


\end{document}
