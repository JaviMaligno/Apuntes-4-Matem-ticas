\documentclass[twoside]{article}
\usepackage{../../estilo-ejercicios}

%--------------------------------------------------------
\begin{document}

\title{Ejercicios de Ideals, Varieties, and Algorithms (4ª Edición)}
\author{Diego Pedraza López, Javier Aguilar Martín, Rafael González López}
\maketitle

\begin{ejercicio}{3.1.1}
Let $I\subset k[x_1,\dotsc,x_n]$ be an ideal.
\begin{enumerate}[a.]
\item Prove that $I_l = I\cap k[x_{l+1},\dotsc,x_n]$ is an ideal of $k[x_{l+1},\dotsc,x_n]$.
\item Prove that ideal $I_{l+1} \subset k[x_{l+2},\dotsc,x_n]$ is the first elimination ideal $I_l \subset k[x_{l+1},\dotsc,x_n]$. This observation allows to use the Extension Theorem multiple times when eliminating more than one variable.
\end{enumerate}
\end{ejercicio}
\begin{solucion}
\begin{enumerate}[a.]
\item[]
\item Probemos que es un ideal
\begin{itemize}
\item Claramente $I_l \subset k[x_{l+1},\dotsc,x_n]$.
\item Si $f,g\in I_l$, entonces solo tienen términos en $x_{l+1},\dotsc,x_n$, luego la suma tendrá solo esos términos. Como $f+g\in I$ y $f+g\in k[x_{l+1},\dotsc,x_n]$ tenemos que $f+g\in I_l$.
\item Tomando $a\in k$ y $f \in I_l$, entonces $af$ solo tiene términos en $x_{l+1},\dotsc,x_n$, luego el producto tendrá solo esos términos. Se deduce, como en el apartado anterior, que $af\in I_l$.
\end{itemize}
\item Es inmediato si renombramos los índices de las variables y consideramos $I'=I_{l+1}$ que $I_1'=I_{l+1}$.
\end{enumerate}
\end{solucion}

\newpage

\begin{ejercicio}{3.1.2}

\end{ejercicio}
\begin{solucion}
\begin{enumerate}[a.]
\end{enumerate}
\end{solucion}


\newpage

\begin{ejercicio}{3.1.3}

\end{ejercicio}
\begin{solucion}
\begin{enumerate}[a.]
\end{enumerate}
\end{solucion}


\newpage

\begin{ejercicio}{3.1.4}

\end{ejercicio}
\begin{solucion}
\begin{enumerate}[a.]
\end{enumerate}
\end{solucion}


\newpage

\begin{ejercicio}{3.1.5}

\end{ejercicio}
\begin{solucion}
\begin{enumerate}[a.]
\end{enumerate}
\end{solucion}


\newpage

\begin{ejercicio}{3.1.6}

\end{ejercicio}
\begin{solucion}
\begin{enumerate}[a.]
\end{enumerate}
\end{solucion}


\newpage

\begin{ejercicio}{3.1.7}

\end{ejercicio}
\begin{solucion}
\begin{enumerate}[a.]
\end{enumerate}
\end{solucion}


\newpage

\begin{ejercicio}{3.1.8}

\end{ejercicio}
\begin{solucion}
\begin{enumerate}[a.]
\end{enumerate}
\end{solucion}


\newpage

\begin{ejercicio}{3.1.9}

\end{ejercicio}
\begin{solucion}
\begin{enumerate}[a.]
\end{enumerate}
\end{solucion}


\end{document}
