\documentclass[twoside]{article}
\usepackage{../../estilo-ejercicios}

%--------------------------------------------------------
\begin{document}

\title{Ejercicios de Ideals, Varieties, and Algorithms (4ª Edición)}
\author{Diego Pedraza López, Javier Aguilar Martín, Rafael González López}
\maketitle

\begin{ejercicio}{1.5.1}
Over the complex numbers $\C$, Corollary 3 can be stated in a stronger form. Namely, prove that if $f \in \C[x]$ is a polynomial of degree $n > 0$, then $f$ can be written in the form $f = c(x-a_1)\dots(x-a_n)$, where $c,a_1,\dots,a_n \in \C$ and $c \neq 0$.
Note that this result  holds for \emph{any} algebraically closed field.
\end{ejercicio}
\begin{solucion}
Si $n = 1$, tenemos que $f(x) = cx+d$ con $c \neq 0$, luego podemos escribir $f(x) = c(x-d/c)$.
Si $n > 1$, por el teorema fundamental del cálculo $f$ tiene una raíz $a \in \C$.
Es más, tenemos que $f(x) = h(x)(x-a)+b$ por división de polinomios.
Como $f(a) = h(a)\cdot 0 + b = b$, deducimos que $f(x) = h(x)(x-a)$, donde $h$ es un polinomio de grado $n-1$.
Aplicando este proceso $n-1$ veces y aplicando por último el caso $n=1$, obtenemos el resultado.
\end{solucion}

\newpage

\begin{ejercicio}{1.5.2}
Although Corollary 3 is simple to prove, it has some nice consequences. For example, consider the $n \times n$ Vandermonde determinant determined by $a_1,\dots,a_n$ in a field $k$:
\[ det\begin{pmatrix}
1 & a_1 & a_1^2 & \dots & a_1^{n-1}\\
1 & a_2 & a_2^2 & \dots & a_2^{n-2}\\
\vdots & \vdots & \vdots & & \vdots\\
1 & a_n & a_n^2 & \dots & a_n^{n-1}
\end{pmatrix}\]
Prove that this determinant is nonzero when the $a_i$'s are distinct.
\end{ejercicio}
\begin{solucion}
Probaremos la implicación equivalente: si el determinante es cero, entonces los $a_i$ no son todos distintos.

Supongamos que el determinante es cero.
Entonces las columnas son linealmente dependientes.
Es decir, hay $c_0,\dots,c_{n-1}$ tal que:
\[ c_0 \begin{pmatrix} 1\\\vdots\\1\end{pmatrix} + c_1\begin{pmatrix} a_1\\\vdots\\a_n\end{pmatrix}+ \dots + c_{n-1}\begin{pmatrix} a_1^{n-1}\\\vdots\\a_n^{n-1}\end{pmatrix} = \begin{pmatrix}0\\\vdots\\0\end{pmatrix}\]
Esto nos está dando $n$ soluciones del polinomio de grado $n-1$ dado por $c_0 + c_1 x + \dots + c_{n-1}x^{n-1}$.
Como este polinomio tiene como máximo $n-1$ raíces distintas, esto quiere decir que $a_i = a_j$ para algún $i \neq j$.
\end{solucion}

\newpage

\begin{ejercicio}{1.5.3}
The fact that every ideal of $k[x]$ is principal (generated by one element) is special to the case of polynomials in one variable.
In this exercise we will see why.
Namely, consider the ideal $I=\gene{x,y} \subseteq k[x,y]$.
Prove that $I$ is not a principal ideal.
It follows that the treatment of gcd’s given in this section applies only to polynomials in one variable.
One can compute gcd’s for polynomials of $≥ 2$ variables, but the theory involved is more complicated.
\end{ejercicio}
\begin{solucion}
Supongamos que $I = \gene{f}$ para algún $f \in k[x,y]$.
Entonces existe $g$ tal que $fg = x$.
Como $\deg(fg) = \deg(f) + \deg(g) = \deg(x) = 1$, entonces o bien $f$ es constante, o bien, $g$ es constante.
Si $f$ es constante, entonces $I$ es el ideal total, pero claramente $1 \notin \gene{x,y}$, pues todo elmento de $\gene{x,y}$ tiene grado mayor o igual que $1$.
Entonces $g(x) = c$, luego $f = x/c$.
Con un argumento equivalente, llegamos a que $f = y/c'$.
Como $x \neq y$, llegamos a una contradicción.
\end{solucion}

\newpage

\begin{ejercicio}{1.5.4}
If $h$ is the gcd of $f$, $g \in k[x]$, then prove that there are $A$, $B \in k[x]$, such that $A f + B g = h$.
\end{ejercicio}
\begin{solucion}
Como $\gene{f,g} = \gene{h}$, entonces $h = A f + B g$ para algún $A, B \in k[x]$.
\end{solucion}

\newpage

\begin{ejercicio}{1.5.5}
If $f,g \in k[x]$, then prove that $\gene{f-qg,g} = \gene{f,g}$ for any $q$ in $k[x]$.
This will prove equation (4) in the text.
\end{ejercicio}
\begin{solucion}
Evidente, pues $f-qg \in \gene{f,g}$ y $f = (f-qg)+qg \in \gene{f-qg,g}$.
\end{solucion}
\end{document}
