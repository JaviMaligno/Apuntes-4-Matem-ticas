\documentclass[twoside]{article}
\usepackage{../../estilo-ejercicios}

%--------------------------------------------------------
\begin{document}

\title{Ejercicios de Ideals, Varieties, and Algorithms (4ª Edición)}
\author{Diego Pedraza López, Javier Aguilar Martín, Rafael González López}
\maketitle

\begin{ejercicio}{1.5.1}
Over the complex numbers $\C$, Corollary 3 can be stated in a stronger form. Namely, prove that if $f \in \C[x]$ is a polynomial of degree $n > 0$, then $f$ can be written in the form $f = c(x-a_1)\dots(x-a_n)$, where $c,a_1,\dots,a_n \in \C$ and $c \neq 0$. Note that this result  holds for \emph{any} algebraically closed field.
\end{ejercicio}
\begin{solucion}
Si $n = 1$, tenemos que $f(x) = cx+d$ con $c \neq 0$, luego podemos escribir $f(x) = c(x-d/c)$.
Si $n > 1$, por el teorema fundamental del álgebra $f$ tiene una raíz $a \in \C$.
Es más, tenemos que $f(x) = h(x)(x-a)+b$ por división de polinomios.
Como $f(a) = h(a)\cdot 0 + b = b$, deducimos que $f(x) = h(x)(x-a)$, donde $h$ es un polinomio de grado $n-1$.
Aplicando este proceso $n-1$ veces y aplicando por último el caso $n=1$, obtenemos el resultado.
\end{solucion}

\newpage

\begin{ejercicio}{1.5.2}
Although Corollary 3 is simple to prove, it has some nice consequences. For example, consider the $n \times n$ Vandermonde determinant determined by $a_1,\dots,a_n$ in a field $k$:
\[ det\begin{pmatrix}
1 & a_1 & a_1^2 & \dots & a_1^{n-1}\\
1 & a_2 & a_2^2 & \dots & a_2^{n-2}\\
\vdots & \vdots & \vdots & & \vdots\\
1 & a_n & a_n^2 & \dots & a_n^{n-1}
\end{pmatrix}\]
Prove that this determinant is nonzero when the $a_i$'s are distinct.
\end{ejercicio}
\begin{solucion}
Probaremos la implicación equivalente: si el determinante es cero, entonces los $a_i$ no son todos distintos.

Supongamos que el determinante es cero.
Entonces las columnas son linealmente dependientes.
Es decir, $\exists c_0,\dots,c_{n-1}\in \C$ no todos nulos tal que:
\[ c_0 \begin{pmatrix} 1\\\vdots\\1\end{pmatrix} + c_1\begin{pmatrix} a_1\\\vdots\\a_n\end{pmatrix}+ \dots + c_{n-1}\begin{pmatrix} a_1^{n-1}\\\vdots\\a_n^{n-1}\end{pmatrix} = \begin{pmatrix}0\\\vdots\\0\end{pmatrix}\]
Esto nos está dando $n$ soluciones del polinomio de grado $n-1$ dado por $c_0 + c_1 x + \dots + c_{n-1}x^{n-1}$.
Como este polinomio tiene como máximo $n-1$ raíces distintas, esto quiere decir que $a_i = a_j$ para algún $i \neq j$.
\end{solucion}

\newpage

\begin{ejercicio}{1.5.3}
The fact that every ideal of $k[x]$ is principal (generated by one element) is special to the case of polynomials in one variable.
In this exercise we will see why.
Namely, consider the ideal $I=\gene{x,y} \subseteq k[x,y]$.
Prove that $I$ is not a principal ideal.
It follows that the treatment of gcd’s given in this section applies only to polynomials in one variable.
One can compute gcd’s for polynomials of $≥ 2$ variables, but the theory involved is more complicated.
\end{ejercicio}
\begin{solucion}
Supongamos que $I = \gene{f}$ para algún $f \in k[x,y]$.
Entonces existe $g$ tal que $fg = x$.
Como $\deg(fg) = \deg(f) + \deg(g) = \deg(x) = 1$, entonces o bien $f$ es constante, o bien, $g$ es constante.
Si $f$ es constante, entonces $I$ es el ideal total, pero claramente $1 \notin \gene{x,y}$, pues todo elmento de $\gene{x,y}$ tiene grado mayor o igual que $1$.
Entonces $g(x) = c$, luego $f = x/c$.
Con un argumento equivalente, llegamos a que $f = y/c'$.
Como $x \neq y$, llegamos a una contradicción.
\end{solucion}

\newpage

\begin{ejercicio}{1.5.4}
If $h$ is the gcd of $f$, $g \in k[x]$, then prove that there are $A$, $B \in k[x]$, such that $A f + B g = h$.
\end{ejercicio}
\begin{solucion}
Como $\gene{f,g} = \gene{h}$, entonces $h = A f + B g$ para algún $A, B \in k[x]$.
\end{solucion}

\newpage

\begin{ejercicio}{1.5.5}
If $f,g \in k[x]$, then prove that $\gene{f-qg,g} = \gene{f,g}$ for any $q$ in $k[x]$.
This will prove equation (4) in the text.
\end{ejercicio}
\begin{solucion}
Evidente, pues $f-qg \in \gene{f,g}$ y $f = (f-qg)+qg \in \gene{f-qg,g}$.
\end{solucion}

\newpage

\begin{ejercicio}{1.5.6}
Given $f_1,\dots , f_s ∈ k[x]$, let $h = \gcd( f_2,\dots , f_s)$. Then use the equality 
$\langle h\rangle = 
 \langle f_2,\dots , f_s\rangle$
to show that 
 $\langle f_1, h\rangle = 
 \langle f_1, f_2,\dots, f_s\rangle$.
\end{ejercicio}
\begin{solucion}
Dado que $h$ divide a $f_i$, $i=2,\dotsc,s$, entonces es claro que $f_i\in\gene{f,h}$, de donde se deduce $\gene{f_1,\dotsc,f_s}\subset \gene{f,h}$. Recíprocamente, por utilizando la igualdad, tenemos que $h=g_1f_1+\dotsc f_sg_s$, luego $f_1,h\in\gene{f_1,\dotsc,f_s}$. 
\end{solucion}

\newpage

\begin{ejercicio}{1.5.7}
If you are allowed to compute the gcd of only two polynomials at a time (which is true
for some computer algebra systems), give pseudocode for an algorithm that computes
the gcd of polynomials $f_1,\dots, f_s ∈ k[x]$, where $s > 2$. Prove that your algorithm works.
\end{ejercicio}
\begin{solucion}
Input: $f_1,\dots, f_s$.

Output: $\gcd(f_1,\dots, f_s)$.

$d=\gcd(f_1,f_2)$

FOR $i=3,\dots, s$

$d=\gcd(d,f_i)$\\

El algoritmo funciona porque siempre puede calcular el gcd de dos polinomios. Además para porque hay un número finito de pasos. Por otro lado, claramente $d |\gcd(f_1,\dots, f_s)$ pues lo divide a todos por separado. Asímismo, $\gcd(f_1,\dots, f_s)|d$ aplicando Bézout en cada paso del bucle, ya que podremos dividir siempre por el máximo común divisor. 


\end{solucion}

\newpage

\begin{ejercicio}{1.5.8}
Use a computer algebra system to compute the following gcd’s:
\begin{itemize}
\item[a.] $\gcd(x^4 + x^2 + 1, x^4 − x^2 − 2x − 1, x^3 − 1)$.
\item[b.] $\gcd(x^3 + 2x^2 − x − 2, x^3 − 2x^2 − x + 2, x^3 − x^2 − 4x + 4)$.
\end{itemize}
\end{ejercicio}

\begin{solucion}
En Wolfram
\begin{itemize}
\item[a.] 1
\item[b.]$x-1$
\end{itemize}
Además se puede comprobar que el algoritmo descrito en el ejercicio anterior da la solución correcta.
\end{solucion}
\newpage
\begin{ejercicio}{1.5.9}
Use the method described in the text to decide whether $x^2 − 4$ is an element of the ideal
$\langle x^3 + x^2 − 4x − 4, x^3 − x^2 − 4x + 4, x^3 − 2x^2 − x + 2\rangle$.
\end{ejercicio}
\begin{solucion}
En primer lugar calculamos el gcd de los generadores, que es $x-2$. Como $x^2-4=(x-2)(x+2)$, tenemos que $x^2-4\in \langle x-2\rangle$, lo cual es equivalente a que pertenezca al ideal original
\end{solucion}

\newpage
\begin{ejercicio}{1.5.10}
Give pseudocode for an algorithm that has input $f , g ∈ k[x]$ and output $h, A, B ∈ k[x]$ where $h = gcd( f , g)$ and $Af + Bg = h$. Hint: The idea is to add variables A, B, C,D to
the algorithm so that $Af + Bg = h$ and $Cf + Dg = s$ remain true at every step of the algorithm. Note that the initial values of A, B, C,D are 1, 0, 0, 1, respectively. You may
find it useful to let $quotient( f , g)$ denote the quotient of f on division by g, i.e., if the division algorithm yields f = qg + r, then $q = quotient( f , g)$.
\end{ejercicio}
\begin{solucion}
Copiar y pegar el Algoritmo de Euclides extendido, que no es más que ir despejando en las divisiones del Algoritmo de Euclides tradicional.
\end{solucion}
\newpage
\begin{ejercicio}{1.5.11}
In this exercise we will study the one-variable case of the consistency problem from §2.
Given $f_1,\dots , f_s ∈ k[x]$, this asks if there is an algorithm to decide whether $\V( f_1,\dots, f_s)$
is nonempty. We will see that the answer is yes when $k = \C$.
\begin{itemize}
\item[a.] Let $f ∈ \C[x]$ be a nonzero polynomial. Then use Theorem 7 of §1 to show that
$\V( f) = ∅$ if and only if $f$ is a non-zero constant.
\item[b.] If $f_1,\dots, f_s ∈ \C[x]$, prove $\V( f_1,\dots, f_s) = ∅$ if and only if $\gcd( f_1, . . . , f_s) = 1$.
\item[c.] Describe (in words, not pseudocode) an algorithm for determining whether or not
$\V( f_1,\dots , f_s)$ is nonempty.
\end{itemize}
\end{ejercicio}
\begin{solucion}
\begin{itemize}
\item[]
\item[a.] Se sigue de que todo polinomio no constante tiene al menos una raíz en $\C$ y de que un polinomio consante no nulo no tiene raíces.
\item[b.] La implicación a la izquierda es clara pues el ideal generado por dichos polinomios será todo $\C[x]$. Para la otra, $\V( f_1,\dots, f_s) = ∅$ es equivalente a que $\I(\V( f_1,\dots, f_s)) =\C[x]$, lo cual es equivalente a que $\langle 1\rangle= \I(\V( f_1,\dots, f_s))$, de donde se deduce el resultado. Hemos usado que $\sqrt{I}=\C[x]$ si y solo si $I=\C[x]$.
\item[c.] Basta calcular el gcd y comprobar si es constante o no.
\end{itemize}
\end{solucion}

\newpage

\begin{ejercicio}{1.5.12}
Since we are working over the complex numbers, we know by Exercise 1 that $f$
factors completely, i.e.,
$$f = c(x − a_1)^{r_1}\cdots(x − a_l)^{r_l} ,$$
where $a_1,\dots, a_l ∈ \C$ are distinct and $c ∈ \C \ \{0\}$.
Define the polynomial
$$f_{red} = c(x − a_1)\cdots (x − a_l).$$
The polynomials $f$ and $f_{red}$ have the same roots, but their multiplicities may differ. In
particular, all roots of $f_{red}$ have multiplicity one. We call fred the reduced or square-free
part of $f$ . The latter name recognizes that $f_{red}$ is the square-free factor of $f$ of largest
degree.
\begin{itemize}
\item[a.] Show that $\V(f) = \{a_1,\dots, a_l\}$.
\item[b.] Show that $\I(\V( f )) = 
 \langle f_{red}\rangle$.
 \end{itemize}
\end{ejercicio}
\begin{solucion}
\begin{itemize}
\item[]
\item[a.] Se tiene por definición.
\item[b.] Como $\V(f)=\V(f_{red})$, entonces $\I(V(f))=\I(\V(f_{red}))=\sqrt{\gene{f_{red}}}$. Veamos que $I={\gene{f_{red}}}$ es radical. Sea $g^r\in I$, entonces como $f_{red}$ solo tiene factores lineales, esto significa que $g^r=c^rf^r$, luego $g=cf\in I$. 
\end{itemize}
\end{solucion}
\newpage
\begin{ejercicio}{1.5.13}
We will study the formal derivative of $f = c_0x^n +c_1x^{n−1} +\cdots+c_{n−1}x+c_n ∈ \C[x]$. The formal derivative is defined by the usual formulas from calculus:
$$
f' = nc_0x^{n−1} + (n − 1)c_1x^{n−2} + \cdots + c^{n−1} + 0.
$$
Prove that the following rules of differentiation apply:
\begin{align*}
(af)'&=af' \qquad \text{when }a \in \C\\
(f+g)'&=f'+g'\\
(fg)'&=f'g+fg'
\end{align*}
\end{ejercicio}
\begin{solucion}
Si de verdad necesitas probarlo...
\end{solucion}
\newpage


\begin{ejercicio}{1.5.14} In this exercise we will use the differentiation properties of Exercise 13 to compute
$\gcd(f,f')$ when $f\in\C[x]$.
\begin{itemize}
\item[a.] Suppose $f = (x−a)^r h$ in $\C[x]$, where $h(a) \neq 0$. Then prove that $f'= (x−a)^{r−1} h_1$, where $h_1 ∈ \C[x]$ does not vanish at $a$. Hint: Use the product rule.
\item[b.] Let $f = (x−a_1)^{r_1}\cdots(x−a_l)^{r_l}$ be the factorization of $f$, where $a_1,\dotsc, a_l$ are distinct.Prove that $f'$ is a product $f' = (x − a_1)^{r_1−1}\cdots(x − a_l)^{r_l−1} H$, where $H ∈ \C[x]$ is a polynomial vanishing at none of $a_1,\dotsc, a_l$.
\item[c.] Prove that $g(f,f') = (x-a_1)^{r_1-1}\cdots(x-a_n)^{r_n-1}$.
\end{itemize}
\end{ejercicio}
\begin{solucion}
\begin{itemize}
\item[]
\item[a.] Sea $g=(x-a)^r$, entonces 
$$
f' = (gh)' = g'h+gh' = r(x-a)^{r-1}h+(x-a)^rh' = (x-a)^{r-1}(rh+(x-a)h')
$$
Tomamos $h_1(x)=rh(x)+(x-a)h'(x)$. Además, $h_1(a)=rh(a)\neq 0$.
\item[b.] Probémoslo por inducción sobre el número de factores distintos. El caso $n=1$ es claro, veamos el caso $n=2$. Si $f=c(x-a)^{r}(x-b)^s$
$$
 f' = (x-a)^{r-1}(r(x-b)^s+(x-a)s(x-b)^{s-1}) = (x-a)^{r-1}(x-b)^{s-1}(r(x-b)+s(x-a))
$$
Además, $H(x)=r(x-b)+s(x-a)$ es tal que $H(a),H(b)\neq 0$, por lo que tenemos probado el caso $n=2$. Supongamos que el resultado es cierto para $n-1$ entonces, sea $f = (x−a_1)^{r_1}\cdots(x−a_n)^{r_n}$. Definimos $g=(x-a_1)^{r_1}\cdots (x-a_{n-1})^{r_{n-1}}$. Entonces podemos aplicar la hipótesis de inducción sobre $g$.
\begin{align*}
f'&=(g(x-a_n)^{r_n})'= g'(x-a_n)^{r_n}+gr_n(x-a_n)^{r_n-1} \\ 
& =(x − a_1)^{r_{1}−1}\cdots(x − a_{n-1})^{r_{n-1}−1}G(x)(x-a_n)^{r_n}+gr_n(x-a_n)^{r_n-1}\\
&=(x − a_1)^{r_{1}−1}\cdots(x − a_{n})^{r_{n}−1}(G(x)(x-a_n)+r_n(x-a_1)\cdots (x-a_{n-1}))\\
&=(x − a_1)^{r_{1}−1}\cdots(x − a_{n})^{r_{n}−1}H(x)
\end{align*}
Es claro que si $1\leq i \leq n-1$ entonces $H(a_i) = G(a_i)(a_i-a_n)\neq 0$ por hipótesis. También es claro que $H(a_n)= \prod_{i=1}^{n-1}(a_n-a_i) \neq 0$ por hipótesis.
\item  Sea $h$ nuestro candidato a $\gcd(f,f')$. Es claro que $h\mid f,f'$. Veamos $f/h$ y $f'/h$ son coprimos (luego no puede existir otro factor común no trivial que divida a ambos).
$$
f/h = \prod_{i=1}^n x-a_i \qquad f'/h = H(x)
$$
Sabemos que $H(a_i)\neq 0$ $\forall i$, pero $a_i$ son precisamente los factores de $f/h$, luego son coprimos.
\end{itemize}
\end{solucion}



\newpage
\begin{ejercicio}{1.5.15}
Consider the square-free part $f_{red}$ of a polynomial $f ∈ \C[x]$ defined in Exercise 12.
\begin{itemize}
\item[a.] Use Exercise 14 to prove that fred is given by the formula
$$f_{red} =
\frac{f}{
\gcd( f , f')}
.$$
The virtue of this formula is that it allows us to find the square-free part without factoring $f $. This allows for much quicker computations.
\item[b.] Use a computer algebra system to find the square-free part of the polynomial
$x^{11} − x^{10} + 2x^8 − 4x^7 + 3x^5 − 3x^4 + x^3 + 3x^2 − x − 1$.
\end{itemize}

\end{ejercicio}
\begin{solucion}
\begin{itemize}
\item[]
\item[a.] Se tiene inmediatamente al hacer la división.
\item[b.] Llamamos $f=x^{11} − x^{10} + 2x^8 − 4x^7 + 3x^5 − 3x^4 + x^3 + 3x^2 − x − 1$. Entonces $f'=11x^{10}-10x^9+16x^7-28x^6+15x^4-12x^3+3x^2+6x-1$, $\gcd(f,f')=x^6 - x^5 + x^3 - 2 x^2 + 1$, $f_{red}=x^5+x^2-x-1=(x-1)(x+1)(x^3+x+1)$ (el último se descompone como facotres complejos un poco feos). 
\end{itemize}
\end{solucion}

\newpage

\begin{ejercicio}{1.5.16}
Use Exercises 12 and 15 to describe (in words, not pseudocode) an algorithm whose
input consists of polynomials $f_1,\dots, f_s ∈ \C[x]$ and whose output consists of a basis of
$\I(\V( f_1,\dots, f_s))$. It is more difficult to construct such an algorithm when dealing with
polynomials of more than one variable.
\end{ejercicio}
\begin{solucion}
En primer lugar calcularíamos el gcd de los polinomios y luego al resultado le calculamos la forma reducida. 
\end{solucion}

\newpage

\begin{ejercicio}{1.5.17}
Find a basis for the ideal $\I(\V(x^5 − 2x^4 + 2x^2 − x, x^5 − x^4 − 2x^3 + 2x^2 + x − 1))$.
\end{ejercicio}
\begin{solucion}
Aplicando el algoritmo anterior
$$\I(\V(x^5 − 2x^4 + 2x^2 − x, x^5 − x^4 − 2x^3 + 2x^2 + x − 1))=\I(\V(x^4 - 2 x^3 + 2 x - 1))=\langle (x+1)(x-1)\rangle.$$
\end{solucion}

\end{document}
