\documentclass[cursovd_portada.tex]{subfiles}
\begin{document}


\chapter{Espacio Tangente a una Variedad Diferenciable en un Punto}
\section{Introducci\'{o}n.}
\hs Una de las ideas fundamentales en el estudio de las variedades
diferenciables es la de {\it aproximaci\'{o}n lineal}. Esta es un
concepto familiar cuyo origen est\'{a} en el c\'{a}lculo
diferencial en espacios eucl\'{\i}deos, donde, por ejemplo, una
aplicaci\'{o}n de $\R^m$ a $\R^n$ puede ser aproximada por su
derivada total. Tambi\'{e}n es sabido que a las curvas y a las
superficies regulares se les asocia en cada punto un espacio
vectorial de su misma dimensi\'{o}n, formados por sus vectores
tangentes en el punto, que son vectores de $\R^3$, es decir, que
pueden ser pensados como ``flechas" de origen el punto y, por
tanto, tienen una clara interpretaci\'{o}n f\'{\i}sica. En el caso
de las variedades diferenciables y debido a su definici\'{o}n
abstracta (es decir, no existiendo el espacio eucl\'{\i}deo
ambiente de las curvas y superficies), la situaci\'{o}n es m\'{a}s
compleja. As\'{\i}, para que tenga sentido hablar de
aproximaciones lineales a una variedad diferenciable, ser\'{a}
necesario introducir la noci\'{o}n de espacio tangente a la
variedad en un punto, que ser\'{a} una especie de ``modelo lineal"
para la variedad ``alrededor" del punto.
\par
Para comprender un poco mejor el por qu\'{e} de la definici\'{o}n  de vector tangente a una variedad en un punto que se va
a dar en esta lecci\'{o}n, es necesario recordar dos hechos. El primero de ellos es que todo vector tangente a una
superficie regular en un punto es el vector tangente a una curva contenida en la superficie que pasa por el punto.
Aunque la idea de curva diferenciable en una variedad es bastante intuitiva, no lo es tanto la de vector tangente
a una tal curva. Puesto que la definici\'{o}n de variedad diferenciable se construye sobre la cuesti\'{o}n  de identificar
qu\'{e} funciones son diferenciables, la propiedad de un vector tangente geom\'{e}trico m\'{a}s f\'{a}cilmente generalizable es su
acci\'{o}n sobre funciones diferenciables. As\'{\i}, un vector tangente en un punto a una curva diferenciable contenida en
una variedad diferenciable puede ser considerado como la derivaci\'{o}n de las funciones diferenciables en el punto
que toma la derivada de cada funci\'{o}n a lo largo de la curva.
\par
Por otra parte, hay que observar que en $\R^m$ un vector $\underline{v}$ en un punto $p$,
$\underline{v}=(v_1,\dots ,v_m)$ puede interpretarse como un operador de funciones diferenciables. Esto quiere
decir que si $f$ es una funci\'{o}n diferenciable en un entorno $U$ de $p$, $f:U\subseteq\R^m\fl\R$, entonces
$\underline{v}$ asigna a $f$ el n\'{u}mero real $\underline{v}(f)$ que es la derivada (direccional) de $f$ en la
direcci\'{o}n de $\underline{v}$ valorada en $p$, dada por:
$$\underline{v}(f)=\sum_{i=1}^mv_i\left(\ddep{f}{u_i}\right)_p\in\R.$$
\hs Esta operaci\'{o}n satisface las siguientes dos propiedades:
$$\underline{v}(\lambda f+\mu g)=\lambda\underline{v}(f)+\mu\underline{v}(g),$$
$$\underline{v}(fg)=f(p)\underline{v}(g)+g(p)\underline{v}(f),$$
donde $f$ y $g$ son funciones diferenciables en $p$ y $\lambda$ y $\mu$ n\'{u}meros reales. La primera propiedad dice
que $\underline{v}$ es un operador lineal ($\R$--lineal) y la segunda que es una derivaci\'{o}n. Debe hacerse notar
que la operaci\'{o}n de derivar depende s\'{o}lo de las propiedades locales de las funciones en entornos suficientemente
peque\~{n}os del punto.
\par
La uni\'{o}n de estos dos hechos permite obtener una definici\'{o}n general de vector tangente a una variedad en un punto.
\section{Vectores Tangentes en un Punto de una Va\-rie\-dad Diferenciable.}
\begin{defi}
Una {\bf Curva Diferenciable} en una variedad diferenciable $M$ es una aplicaci\'{o}n diferenciable
$\a\in\mathcal{F}((a,b),M)$, donde $(a,b)$ es un intervalo abierto (degenerado o no) de la recta real.
\end{defi}
Recu\'{e}rdese que $\a\in\mathcal{F}((a,b),M)$ quiere decir que para toda carta $(U,\vp)$ en $M$, tal que
$\a^{-1}(U)\neq\emptyset$, se verifica que la aplicaci\'{o}n $\vp\circ\a:\a^{-1}(U)\fl\R^m$ es diferenciable.
\begin{defi}
Sea $\a\in\mathcal{F}((a,b),M)$ una curva diferenciable en $M$ y $t_0\in (a,b)$. Se llama {\bf Vector Tangente} a
la curva $\a$ en $\a(t_0)$ a la aplicaci\'{o}n
$$\a'(t_0):\mathcal{F}(\a(t_0))\fl\R:f\mapsto\a'(t_0)f=\left.\dderi{(f\circ\a)}{t}\right|_{t_0}.$$
\hs Una curva diferenciable $\a:(a,b)\fl M$ en una variedad diferenciable $M$ se dice {\bf Regular} si $\a'(t)\neq
0$, para cualquier $t\in (a,b)$.
\end{defi}
Debe observarse que, si $f\in\mathcal{F}(\a(t_0))$, entonces $f$ est\'{a} definida en un abierto $U\subseteq M$ tal
que $\a(t_0)\in U$. Por tanto, $f\circ\a$ est\'{a} definida en $\a^{-1}(U)$, que es abierto de $(a,b)$ (por la
continuidad de $\a$). Adem\'{a}s, como
$$\a'(t_0)f=\left.\dderi{(f\circ\a)}{t}\right|_{t_0}=\lim_{t\rightarrow t_0}\frac{f(\a(t))-f(\a(t_0))}{t-t_0},$$
el vector tangente mide la variaci\'{o}n de la velocidad sobre la curva mediante la funci\'{o}n $f$.
\par
La siguiente proposici\'{o}n, cuya demostraci\'{o}n se sigue inmediatamente de la definici\'{o}n y de las propiedades de la
derivada usual, permite interpretar, por similitud con lo que ocurre en $\R^m$, el vector tangente a una curva en
un punto como una derivaci\'{o}n (direccional) en dicho punto.
\begin{prop}
Sean $\a\in\esp{F}((a,b),M)$ una curva diferenciable en $M$,
$t_0\in (a,b)$, $f,g\in\mathcal{F}(\a(t_0))$ y $\lambda,\mu\in\R$.
Entonces, el vector tangente a $\a$ en $\a(t_0)$ verifica las
siguientes propiedades:
\begin{enumerate}
\item $\a'(t_0)(\lambda f+\mu g)=\lambda\a'(t_0)f+\mu\a'(t_0)g$.
\item $\a'(t_0)(fg)=f(\a(t_0))\a'(t_0)g+g(\a(t_0))\a'(t_0)f$.
\end{enumerate}
\end{prop}
\begin{prop}
Sean $\a\in\esp{F}((a,b),M)$ una curva diferenciable en $M$, $t_0\in (a,b)$ y $f,g\in\mathcal{F}(\a(t_0))$ tales
que coinciden en un entorno de $\a(t_0)$. Entonces $\a'(t_0)f=\a'(t_0)g$.
\end{prop}
\begin{defi}
Sea $p\in M$. Se llama {\bf Vector Tangente} a $M$ en $p$ al vector tangente en $p$ a cualquier curva
diferenciable en $M$ que pase por $p$. Al conjunto de los vectores tangentes a $M$ en $p$ se le llama {\bf Espacio
Tangente} a $M$ en $p$.
\end{defi}
Siempre existen curvas diferenciables en $M$ que pasan por $p$. Por ejemplo, la curva constante $\a:\R\fl
M:t\mapsto\a(t)=p$, cuyo vector tangente es $\a'(t)=0$, para todo $t\in\R$.
\par
Por otra parte, si $(U,\vp=(x_1,\dots ,x_m))$ es un s.l.c. entorno de $p$, entonces $\vp(p)=(p_1,\dots
,p_m)\in\vp(U)\subseteq\R^m$. Para cada $i=1,\dots ,m$, sea $\epsilon>0$ tal que
$$(p_1,\dots ,p_{i-1},p_i+t,p_{i+1},\dots ,p_m)\in\vp(U),$$
con $\vert t\vert <\epsilon$, que existe ya que $\vp(U)$ es abierto. Entonces
$$\a_i:(-\epsilon,\epsilon)\fl M:t\mapsto\vp^{-1}((p_1,\dots ,p_{i-1},p_i+t,p_{i+1},\dots ,p_m))$$
es una curva diferenciable en $M$, con $\a_i(0)=p$ y $\a_i'(0)$ es, por tanto, un vector tangente a $M$ en $p$,
que se denotar\'{a} por:
$$\left(\dep{x_i}\right)_p.$$
\hs Esta notaci\'{o}n est\'{a} justificada porque si $f\in\mathcal{F}(p)$, se tiene que
$$(f\circ\a_i)(t)=(f\circ\vp^{-1})((p_1,\dots ,p_{i-1},p_i+t,p_{i+1},\dots ,p_m))=(f\circ\vp^{-1}\circ h_i)(t),$$
donde $h_i:(-\epsilon,\epsilon)\fl\vp(U):t\mapsto h_i(t)=(p_1,\dots ,p_{i-1},p_i+t,p_{i+1},\dots ,p_m)$. Ahora
bien, $h_i(0)=\vp(p)$ y, por la Regla de la Cadena en espacios eucl\'{\i}deos:
$$\a_i'(0)f=\left.\dderi{(f\circ\a_i)}{t}\right|_{t=0}=(\nabla(f\circ\vp^{-1}))_{h_i(0)}.h'_i(0)=$$
$$=\left(\ddep{(f\circ\vp^{-1})}{u_i}\right)_{\vp(p)}= \left(\ddep{f}{x_i}\right)_p.$$
\hs A partir de ahora, se denotar\'{a}:
$$\left(\dep{x_i}\right)_pf=\left(\ddep{f}{x_i}\right)_p,\mbox{ }i=1,\dots ,m.$$
\hs Por las propiedades de los vectores tangentes a curvas, se deduce que todo vector tangente a $M$ en $p$ es una
aplicaci\'{o}n $u:\mathcal{F}(p)\fl\R$, verificando:
\begin{enumerate}
\item $u(\lambda f+\mu g)=\lambda u(f)+\mu u(g)$, $f,g\in\esp{F}(p)$, $\lambda,\mu\in\R$ (condici\'{o}n de
$\R$--linealidad);
\item $u(fg)=f(p)u(g)+g(p)u(f)$, $f,g\in\esp{F}(p)$ (Condici\'{o}n de Leibnitz);
\item $u(f)=u(g)$, si $f,g\in\mathcal{F}(p)$ coinciden en un entorno de $p$.
\end{enumerate}
\hs Si se denota por $T_p(M)$ al conjunto de las aplicaciones $u:\mathcal{F}(p)\fl\R$ que son $\R$--lineales y que
satisfacen la Condici\'{o}n de Leibnitz, se puede probar que dicho conjunto es, realmente, el espacio tangente a $M$
en $p$.
\begin{lemma}
Sean $M$ una variedad diferenciable, $p\in M$, $f\in\esp{F}(p)$ y $(U,\vp=(x_1,\dots ,x_m))$ un s.l.c. entorno de
$p$. Entonces, en un cierto entorno de $p$ se verifica que
$$f=f(p)+\sum_{i=1}^mf_i.(x_i-x_i(p)),$$
donde $f_i\in\esp{F}(p)$ y $f_i(p)=\left(\ddep{f}{x_i}\right)_p$.
\end{lemma}
\begin{lemma} Sean $M$ una variedad diferenciable, $p\in M$ y $u\in T_p(M)$.
\begin{enumerate}
\item Si ${\bf c}\in\esp{F}(M)$ es la funci\'{o}n constante dada por ${\bf c}(q)=c$ para todo $q\in M$ y $c\in\R$,
entonces, se verifica que $u({\bf c})=0$.
\item Si $f\in\mathcal{F}(p)$ se anula en un entorno de $p$, entonces $u(f)=0$.
\item Si $f,g\in\mathcal{F}(p)$ coinciden en un entorno de $p$, entonces $u(f)=u(g)$.
\end{enumerate}
\end{lemma}
Sea ahora un s.l.c. $(U,\vp=(x_1,\dots ,x_m))$ entorno de $p$. Dados $u\in T_p(M)$ y cualquier funci\'{o}n
$f\in\mathcal{F}(p)$, en virtud de los lemas anteriores se tiene que:
$$u=\sum_{i=1}^mu(x_i)\left(\dep{x_i}\right)_p.$$
\hs En particular, si $\a:(a,b)\fl M$ es una curva sobre $M$ con $\a(t_0)=p$, como $\a'(t_0)\in T_p(M)$:
$$\a'(t_0)=\sum_{i=1}^m\a'(t_0)x_i\left(\dep{x_i}\right)_p=
\left.\sum_{i=1}^m\dderi{(x_i\circ\a)}{t}\right|_{t_0}\left(\dep{x_i}\right)_p.$$ \hs Usando esta expresi\'{o}n, puede
deducirse que, dado $u\in T_p(M)$ y considerando la curva diferenciable en $M$,
$$\a:(-\epsilon,\epsilon)\fl M:t\mapsto \vp^{-1}((p_1+tu(x_1),\dots ,p_m+tu(x_m))),$$
para cierto $\epsilon>0$ que existe por la continuidad de $\vp$, donde $(p_1,\dots ,p_m)=\vp(p)$, se verifica que
$\a(0)=p$ y $\a'(0)=u$.
\begin{teorema}
El conjunto de los vectores tangentes a una variedad diferenciable $M$ en un punto $p$ es $T_p(M)$, es decir, los
vectores tangentes a $M$ en $p$ son las aplicaciones $\R$-lineales de dominio $\mathcal{F}(p)$ con valores reales
que verifican la Condici\'{o}n de Leibnitz.
\end{teorema}
Ahora, en el conjunto de las curvas diferenciables en $M$ que pasan por $p$ y que se denota por $\esp{C}(p)$, se
puede definir una relaci\'{o}n de equivalencia $\esp{R}$ por: $\a\esp{R}\b$ si y s\'{o}lo si $\a$ y $\b$ tienen el mismo
vector tangente en $p$.
\begin{prop}
Los conjuntos $\esp{C}(p)/\esp{R}$ y $T_p(M)$ est\'{a}n en correspondencia biyectiva.
\end{prop}
De forma natural se definen en $T_p(M)$ la suma y producto por
n\'{u}meros reales que lo convierten en un espacio vectorial real.
A continuaci\'{o}n se va a encontrar una base (y, por tanto, a
saber la dimensi\'{o}n) de este espacio vectorial. Para ello, la
expresi\'{o}n anterior de $\a'(t_0)$ indica el camino a seguir.
\begin{teorema}
Dado un punto $p$ de una variedad diferenciable $M$ y una carta local $(U,\vp=(x_1,\dots ,x_m))$ entorno de $p$,
los vectores tangentes
$$\left(\dep{x_i}\right)_p,\mbox{ }i=1,\dots ,m,$$
forman una base de $T_p(M)$, con lo que ${\rm dim}(T_p(M))=m$.
\end{teorema}
Como consecuencia, se obtiene que hay tantas bases de $T_p(M)$ como cartas locales entorno de $p$. Adem\'{a}s, si
$(U,\vp=(x_1,\dots ,x_m))$ y $(V,\psi=(y_1,\dots ,y_m))$ son dos de tales cartas, entonces el cambio de bases
viene dado por:
$$\left( \dep{y_j}\right)_p=\sum_{i=1}^m\left(\ddep{x_i}{y_j}\right)_p\left( \dep{x_i}\right)_p.$$
\hs Debe observarse que la matriz de cambio de bases es, realmente, la matriz jacobiana de la aplicaci\'{o}n de
transici\'{o}n $\vp\circ\psi^{-1}$ en $\psi(p)$, pues:
$$\left(\ddep{x_i}{y_j}\right)_p=\left(\ddep{(u_i\circ\vp\circ\psi^{-1})}{u_j}\right)_{\psi(p)}.$$
\section{Espacio Cotangente.}
\begin{defi}
Dado un punto $p$ de una variedad diferenciable $M$, se llama {\bf Espacio Cotangente} a $M$ en $p$ al espacio
vectorial dual del espacio tangente a $M$ en $p$, que se denota por $T_p^*(M)$ y a cuyos elementos se llaman {\bf
Covectores} en $p$.
\end{defi}
Recu\'{e}rdese que $T_p^*(M)=\{u^*:T_p(M)\fl\R /u^* {\rm es}\mbox{ }{\rm lineal}\}$.
\begin{defi}
Sea $f\in\esp{F}(p)$. Se llama {\bf Diferencial} de $f$ en $p$ a la funci\'{o}n $(\dif f)_p:T_p(M)\fl\R$ tal que
$(\dif f)_pu=u(f)$, para cualquier vector $u\in T_p(M)$.
\end{defi}
\begin{prop}
Dada $f\in\esp{F}(p)$, entonces $(\dif f)_p\in T_p^*(M)$.
\end{prop}
Dado $(U,\vp=(x_1,\dots ,x_m))$ un s.l.c. entorno de $p$, en virtud de la proposici\'{o}n anterior se tiene que
$(\de{x_i})_p\in T_p^*(M)$, para cualquier $i=1,\dots ,m$. Adem\'{a}s, se puede comprobar f\'{a}cilmente que
$\{(\de{x_i})_p\}_{i=1,\dots ,m}$ es una base de $T_p^*(M)$, dual de la base
$\{\left(\dep{x_i}\right)_p\}_{i=1,\dots ,m}$ de $T_p(M)$.
\begin{prop}
Todo covector en $p$ es la diferencial de alguna funci\'{o}n diferenciable en $p$.
\end{prop}

\newpage

\section{Ejercicios.}
\begin{enumerate}
\item Sean $M$ una variedad diferenciable, $p\in M$ y
$(U,\vp=(x_1,\dots ,x_m))$, $(V,\psi=(y_1,\dots ,y_m))$ dos
s.l.c.entorno de $p$, tales que $x_1=y_1$. ?`Es
$\left(\dep{x_1}\right)_p=\left(\dep{y_1}\right)_p$? \item Sean $M$ una variedad
diferenciable, $p \in M$, $W$ un entorno abierto de $p$ y $f,g \in
{\cal F}(W)$. Probar que:
\begin{enumerate}
\item[(a)] Si $\forall q \in W$ y $\forall u \in T_q(M)$ se tiene
que $u(f) = 0$, entonces $f$ es constante en un entorno de $p$.
\item[(b)] Si $\forall q \in W$ y $\forall u \in T_q(M)$ se
verifica que $u(f) = u(g)$, entonces $f-g$ es constante en un
entorno de $p$.
\end{enumerate}
\item Sean $M$ una variedad diferenciable, $p \in M$ y
$\alpha:I\longrightarrow M$ una curva diferenciable pasando por
$p$ (es decir, se puede suponer que $0 \in I$ y que $\alpha(0) =
p$). Si $\alpha'(0) = 0$, probar que la aplicaci\'{o}n $$L:{\cal
F}(p) \longrightarrow \R:f \longmapsto L(f) = \left.
\frac{d^2(f \circ \alpha)}{dt^2} \right |_{t=0}$$ define un vector
tangente a $M$ en $p$. \item Sean $M$ una variedad diferenciable,
$p\in M$ y $(U,\vp=(x_1,\dots ,x_m))$ un s.l.c. entorno de $p$.
Dado cualquier covector $\om\in T_p^*(M)$, calcular su
expresi\'{o}n en funci\'{o}n de la base $\{(\de x_i)_p/i=1,\dots
,m\}$.
\end{enumerate}
\end{document}