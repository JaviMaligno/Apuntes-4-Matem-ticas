\documentclass[cursovd_portada.tex]{subfiles}

\begin{document}

\chapter*{Ap\'endice A.\\Teoremas de Extensi\'on}
%\newtheorem{lemaap}{Lema}
%\newtheorem{propoap}{Proposici\'{o}n}
\addcontentsline{toc}{chapter}{Ap\'{e}ndice A. Teoremas de Extensi\'{o}n.}
\section*{A.1 Lema de Extensi\'{o}n de Funciones Diferenciables.}
\addcontentsline{toc}{section}{A.1\hspace{0.5 em} Lema de Extensi\'{o}n de Funciones Diferenciables.}
\begin{lemaap}
Sea $M$ una variedad diferenciable y sea $p\in U\subseteq M$, con $U$ entorno abierto de $p$. Entonces, existen un
entorno abierto de $p$, $V$ y una funci\'{o}n diferenciable $f\in {\cal F}(M)$, tales que:
\begin{enumerate}
\item $\overline{V}\subseteq U$;
\item $0\leq f\leq 1$; $f\equiv 1$ en $\overline{V}$; $f\equiv 0$ en $M-U$.
\end{enumerate}
\end{lemaap}
{\sc Demostraci\'{o}n:} Obs\'{e}rvese que basta probarlo en el caso de que $U$ est\'{e} contenido en el dominio de una carta
entorno de $p$. En efecto, si $(U_{\a}, \vp_{\a})$ es una tal carta, dado cualquier abierto $U$ en las hip\'{o}tesis,
se prueba el Lema para $U\cap U_{\a}$ y se tendr\'{\i}a $\overline{V} \subseteq U\cap U_{\a}\subseteq U$, $f\equiv 1$
en $\overline{V}$ y $f\equiv 0$ en $M-(U\cap U_{\a})\supseteq M-U$.
\par
Ahora, se puede suponer que la mencionada carta est\'{a} centrada en $p$. De\-n\'{o}\-te\-se por $(W,\vp)$, $\vp (p)=0$.
Sean $(x_1,\dots ,x_m)$ sus funciones coordenadas y sea $a\in \R$, $a>0$, tal que el cubo
$$Q=\{(u_1,\dots ,u_m)\in \R^m /\vert u_i \vert\leq a\}$$
est\'{e} contenido en $\vp (U)$. Sea $b\in\R$ con $a>b>0$. Consid\'{e}rense los abiertos de U:
$$U_p=\{q\in U/\vert x_i(q) \vert <a,1\leq i\leq m\},$$
$$V_p=\{q\in U/\vert x_i(q) \vert <b,1\leq i\leq m\}.$$
\hs Entonces, $U_p$ y $V_p$ son entornos abiertos de $p$, verificando $\overline{V}_p\subseteq U_p$ y
$\overline{U}_p\subseteq U$. Sea $g:\R\fl \R$ una funci\'{o}n ${\cal C}^{\infty}$  con las propiedades: $g(t)=1$ para
$\vert t \vert\leq b$, $0<g(t)<1$ para $b<\vert t\vert <a$ y $g(t)=0$ para $\vert t \vert\geq a$. Tal funci\'{o}n se
puede definir de la siguiente forma:
\par
En primer lugar, sea $k:\R\fl \R$ definida por
$$k(t)=\left\{
\begin{array}{lc}
e^{-1/t^2}, & t>0,\\
0, & t\leq 0,
\end{array}
\right.$$ que es ${\cal C}^{\infty}$. Sea ahora
$$h(t)=\frac{k(t)}{k(t)+k(1-t)},$$
que tambi\'{e}n es ${\cal C}^{\infty}$ y que verifica:
$$\left\{
\begin{array}{lll}
h(t)=0 & {\rm si} & t\leq 0,\\
0<h(t)<1 & {\rm si} & 0<t<1,\\
h(t)=1 & {\rm si} & t\geq 1.
\end{array}
\right.$$
%\vspace{10 cm}
Finalmente, sea
$$g(t)=h(\frac{t+a}{a-b})h(\frac{-t+a}{a-b}),$$
que es ${\cal C}^{\infty}$ y cumple las condiciones requeridas (dibujar las gr\'{a}ficas de $k(t)$, $h(t)$ y $g(t)$).
A partir de ella, se define $f_p:M\fl\R$ por:
$$f_p(q)=\left\{
\begin{array}{lll}
g(x_1(q))\dots g(x_m(q)) & {\rm si} & q\in U,\\
0 & {\rm si} & q\notin \overline{U}_p.
\end{array}
\right.$$ \hs La funci\'{o}n $f_p$ es diferenciable (!`ejercicio!), pues lo es en $U$, en $M-\overline{U}_p$ (que son
abiertos) y en la intersecci\'{o}n es nula, ya que si $q\in U\cap (M-\overline{U}_p)$, entonces
$q\notin\overline{U}_p$, es decir, existe $i$ tal que $\vert x_i(q)\vert\geq a$. Por tanto, $g(x_i(q))=0$ y
$f_p(q)=0$.
\par
En estas condiciones, como $\overline{U}_p\subseteq U$, tomando $V=V_p$, $f_p$ es la buscada. En efecto,
$f_p\equiv 1$ en $\overline{V}_p$. Adem\'{a}s, $0\leq f_p\leq 1$ en $M$, pues, junto con lo anterior se tiene que
$0<f_p<1$ en $U_p-\overline{V}_p$ y $f_p\equiv 0$ en $M-U_p$. Finalmente, $f_p\equiv 0$ en $M-U$, ya que
$f_p\equiv 0$ en $M-\overline{U}_p\supseteq M-U$.\hfill $\Box$
%\bigskip
\begin{teoap}
{\bf (Lema de Extensi\'{o}n de Funciones Diferenciables).} Sea $h\in{\cal F}(U)$, donde $U$ es un abierto conteniendo
a $p\in M$. Entonces, existen un entorno abierto $V$ de $p$, con $\overline{V}\subseteq U$ y una funci\'{o}n
$g\in{\cal F}(M)$ tales que $g\equiv h$ en $V$ y $g\equiv 0$ en $M-U$.
\end{teoap}
{\sc Demostraci\'{o}n:} Sea $W$ un entorno abierto de $p$ tal que $\overline{W} \subseteq U$. Aplicando el Lema 1 a
$W$, se encuentran otro entorno $V$ de $p$ y una funci\'{o}n diferenciable $f$ de $M$ cumpliendo las tesis de dicho
Lema 1.
\par
Ahora, sea $g:M\fl\R$ definida por:
$$g(q)=\left\{
\begin{array}{lll}
f(q)h(q) & {\rm si} & q\in U,\\
0 & {\rm si} & q\in M-\overline{W}.
\end{array}
\right.$$ \hs Para ver que $g\in{\cal F}(M)$ (!`ejercicio!) basta ver que est\'{a} bien definida en $U\cap
(M-\overline{W})$. Pero si $q\in U-\overline{W}$, entonces $q\notin\overline{W}$, esto es, $q\notin W$ y, por
tanto, $f(q)=0$ y $g(q)=0$.
\par
Finalmente, si $q\in V$, entonces $q\in\overline{V}$, luego $f(q)=1$ y $g(q)=h(q)$.\hfill $\Box$
\section*{A.2 Lema de Extensi\'{o}n de Campos Diferenciables.}
\addcontentsline{toc}{section}{A.2\hspace{0.5 em} Lema de Extensi\'{o}n de Campos Diferenciables.}
\begin{lemaap}
Sea $K$ un compacto contenido en un abierto $G$ de una variedad diferenciable $M$. Entonces, existe
$f\in\mathcal{F}(M)$ tal que $0\leq f\leq 1$, $f(K)\equiv 1$ y $f(M-G)\equiv 0$.
\end{lemaap}
{\sc Demostraci\'{o}n:} Sea $p\in K$. Aplicando el Lema 1 de la secci\'{o}n A.1, puesto que $p\in G$ que es abierto,
existen $V_p$, entorno abierto de $p$ y $f_p\in\mathcal{F}(M)$ tales que $\overline{V_p}\subseteq U$, $0\leq
f_p\leq 1$, $f_p\equiv 1$ en $\overline{V_p}$ y $f_p\equiv 0$ en $M-G$. Por tanto, la familia $\{V_p/p\in K\}$ es
un recubrimiento por abiertos de $K$, que es compacto, luego puede extraerse un subrecubrimiento finito
$\{V_{p_1},\dots ,V_{p_r}\}$.
\par
Sean $f_{p_1},\dots ,f_{p_r}\in\mathcal{F}(M)$ las correspondientes funciones. Entonces, considerando
$f=1-(1-f_{p_1})\cdots(1-f_{p_r})$, se puede comprobar que $f$ es la funci\'{o}n buscada.\hfill $\Box$
\begin{teoap}
{\bf (Lema de Extensi\'{o}n de Campos Diferenciables).} Sea $X\in\mathcal{X}(U)$ y $V\supseteq U$ otro abierto.
Entonces, dado cualquier $p\in U$, existen $\overline{X}\in\mathcal{X}(V)$ y $W$ abierto con $p\in W\subseteq U$
tales que $\overline{X}_q=X_q$, para todo $q\in W$ y $\overline{X}=0$ en $V-U$.
\end{teoap}
{\sc Demostraci\'{o}n:} Sea $p\in U$. Entonces, al ser $M$ localmente compacta, existen abiertos $W$ y $G$ de clausura
compacta tales que $p\in W\subseteq\overline{W}\subseteq G\subseteq\overline{G}\subseteq U\subseteq V$ y existe
una funci\'{o}n, en virtud del lema anterior, $f\in\mathcal{F}(V)$ que vale 1 en $\overline{W}$ y vale 0 en $V-G$. Sea
ahora $q\in V$. Se define el campo de vectores $\overline{X}$ por:
$$\overline{X}_q=\left\{
\begin{array}{ccl}
f(q)X_q & & q\in U;\\
0 & & q\in V-\overline{G}.
\end{array}
\right.$$ \hs $\overline{X}$ est\'{a} bien definido, pues si $q\in U\cap(V-\overline{G})$, entonces $q\notin G$ y
$f(q)=0$. Adem\'{a}s, $\overline{X}\in\mathcal{X}(V)$, pues vale $fX$ en $U$ (con $fX\in\mathcal{X}(U)$) y 0
(diferenciable) en $V-\overline{G}$, siendo $V=U\cup(V-\overline{G})$. Finalmente, si $q\in W$, se tiene que
$f(q)=1$, es decir, $\overline{X}_q=X_q$ y, como $V-U\subseteq V-\overline{G}$, entonces $\overline{X}=0$ en
$V-U$.\hfill $\Box$
\section*{A.3 Lema de Extensi\'{o}n de Campos Diferenciables de Tensores Covariantes.}
\addcontentsline{toc}{section}{A.3\hspace{0.5 em} Lema de Extensi\'{o}n de Campos Diferenciables de Tensores
Covariantes.}
\begin{teoap}
{\bf (Lema de Extensi\'{o}n de Campos Diferenciables de Tensores Covariantes).} Dados dos abiertos $V\subseteq
U\subseteq M$ en una variedad diferenciable, si $T\in T_r(V)$, entonces, para cualquier $p\in V$ existe un entorno
abierto de $p$, $W_p\subseteq V$ y existe $S\in T_r(U)$ tales que $S|_{W_p}=T|_{W_p}$. El mismo resultado se tiene
en $\Lambda_r(V)$.
\end{teoap}
{\sc Demostraci\'{o}n:} Sean $(U',\vp=(x_1,\dots ,x_m))$ un s.l.c de $U$, con $p\in U'\subseteq V$ y $T'=T|_{U'}\in
T_r(U')$. Entonces:
$$T'=\sum_{i_1,\dots ,i_r=1}^mf_{i_1\dots i_r}\de x_{i_1}\otimes\cdots\otimes\de x_{i_r}.$$
\hs Considerando funciones $g_{i_1\dots i_r},h_1,\dots ,h_m\in\mathcal{F}(U)$ que coincidan con las funciones
$f_{i_1\dots i_r},x_1,\dots ,x_m$ en un cierto entorno abierto de $p$, $W_p\subseteq U'\subseteq V$ (que existen
por el Lema de Extensi\'{o}n de Funciones Diferenciables) y definiendo $S$ como
$$S=\sum_{i_1,\dots ,i_r=1}^mg_{i_1\dots i_r}\de h_{i_1}\otimes\cdots\otimes\de h_{i_r},$$
se tiene que:
$$S|_{W_p}=T'|_{W_p}=T|_{U'}|_{W_p}=T|_{W_p}.$$
\hs Para probar el resultado en $\Lambda_r(U)$ basta cambiar los productos tensoriales por productos
exteriores.\hfill $\Box$

\end{document}