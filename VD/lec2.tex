\documentclass[cursovd_portada.tex]{subfiles}
\begin{document}


\chapter{Aplicaciones Diferenciables entre Variedades Diferenciables}
\section{Aplicaciones Diferenciables.}
\begin{defi}
Sean $M$ y $N$ dos variedades diferenciables y $G\subseteq M$ un abierto. Una aplicación continua $f:G\fl N$ , se
dice {\bf Diferenciable en un punto} $p\in G$ si existe una carta local $(U,\vp)$ en $M$ entorno de $p$ y existe
una carta local $(V,\psi)$ en $N$ entorno de $f(p)$ tales que
$$\psi\circ f\circ\vp^{-1}:\vp(G\cap U\cap f^{-1}(V))\fl\psi f(G\cap U\cap f^{-1}(V))$$
es diferenciable en un entorno de $\vp(p)$ contenido en $\vp(G\cap U\cap f^{-1}(V))$. La aplicación $f$ se dice
diferenciable en $G$ si es diferenciable en todos los puntos de $G$.
\end{defi}
\begin{prop}
La definición anterior no depende de las cartas elegidas entornos de $p$ y $f(p)$, respectivamente.
\end{prop}
\begin{dem}
Sean $(\tilde{U},\tilde{φ})$ y $(\tilde{V},\tilde{ψ})$ cartas locales en $M$ y $N$ respectivamente con $p \in \tilde{U}$ y $f(p) \in \tilde{V}$. Si tomamos $\tilde{W} = \tilde{φ}(φ^{-1}(W) \cap \tilde{U} \cap f^{-1}(\tilde{V}))$, donde $W$ es un entorno de $φ(p)$. En $\tilde{W}$:
\[ \tilde{ψ} \circ f \circ \tilde{φ}^{-1} = (\tilde{ψ}\circ ψ^{-1}) \circ (ψ \circ f \circ φ^{-1}) \circ (φ \circ \tilde{φ}^{-1}) \]
Como $\tilde{ψ} \circ ψ^{-1}$ y $φ \circ \tilde{φ}^{-1}$ son cambios de cartas y $ψ \circ f \circ φ^{-1}$ es diferenciable en $φ(p)$, entonces $\tilde{ψ} \circ f \circ \tilde{φ}^{-1}$ es diferenciable. \QED
\end{dem}
Obsérvese que, por definición, toda aplicación
diferenciable es continua. Al conjunto de las aplicaciones
continuas de $G$ en $N$ y diferenciables en $G$ se denota por
$\esp{F}(G,N)$. En particular, si $G=M$, se tiene el conjunto
$\esp{F}(M,N)$ de las aplicaciones diferenciables de todo $M$ en
$N$.
\begin{prop}
Una aplicación continua $f:M\fl N$ pertenece a $\esp{F}(M,N)$ si y sólo si para cualesquiera cartas locales
$(U,\vp)$ en $M$ y $(V,\psi)$ en $N$ tales que $U\cap f^{-1}(V)\neq\emptyset$ se tiene que:
$$\psi\circ f\circ\vp^{-1}:\vp(U\cap f^{-1}(V))\fl\psi f(U\cap f^{-1}(V))$$
es diferenciable.
\end{prop}
\begin{dem}\mbox{}
\begin{itemize}
	\item[($\Rightarrow$)] Supongamos que $f \in \esp{F}(M,N)$. Sean $(U,φ)$ y $(V,ψ)$ cartas locales de $M$ y $N$ respectivamente tal que $U \cap f^{-1}(V) \neq 0$. Sea $q \in φ(U \cap f^{-1}(V))$ cualquiera. Entones existe $p \in U \cap f^{-1}(V)$ con $φ(p)=q$. Luego $p \in U$ y $f(p) \in V$. Utilizando que la definición no depende de las cartas, $ψ \circ f \circ φ^{-1}$ es diferenciable en un entorno $W$ de $φ(p)=q$. Luego $ψ \circ f \circ φ^{-1}$ es diferenciable en $q$.
	\item[($\Leftarrow$)] Sea $(U,φ)$ una carta en $M$ tal que $p \in U$ y $(V,ψ)$ una carta en $N$ tal que $f(p) \in V$. Como $p \in f^{-1}(V)$, entonces $p \in U \cap f^{-1}(V)$. Luego $U \cap f^{-1}(V) \neq \emptyset$. Por hipótesis, $ψ \circ f \circ φ^{-1}$ es diferenciable en $φ(U \cap f^{-1}(V))$. \QED
\end{itemize}
\begin{coro} Sea $f : G \subseteq M \to N$ continua. $f \in \esp{F}(G,N)$ si y sólo si para toda carta $(U,φ)$ en $M$ y toda carta $(V,ψ)$ en $N$ tal que  $G \cap U \cap f^{-1}(V) \neq \emptyset$, $ψ \circ f \circ φ^{-1}$ es diferenciable en $φ(G \cap U \cap f^{-1}(V))$.
\end{coro}
\begin{coro} Sea $f : M \to \R$ continua. $f \in \esp{F}(M)$ si y sólo si para toda carta $(U,φ)$ en $M$, $f \circ φ^{-1}$ es diferenciable en $φ(U)$. Esto se debe a que en $\R$ tenemos como atlas una sola carta con la identidad.
\end{coro}
\begin{coro} Sea $f : G \subseteq M \to \R$ continua con $G$ abierto. $f \in \esp{F}(G)$ si y sólo si para toda carta $(U,φ)$ en $M$ tal que $G \cap U \neq \emptyset$, $f \circ φ^{-1}$ es diferenciable en $φ(G \cap U)$.
\end{coro}
\end{dem}
Hagamos notar c\'omo la aplicaci\'on de la proposici\'on anterior es la que ``va por debajo'' en el siguiente diagrama conmutativo:
\[
\begin{tikzcd}
& M  \arrow[r, "f"] & N & \\
& (U,\varphi)\arrow[u,phantom, "\rotatebox{90}{$\subseteq$}"]\arrow[d,"\varphi"']  & (V,\psi)\arrow[d,"\psi"]\arrow[u,phantom, "\rotatebox{90}{$\subseteq$}"] &  \\
\varphi(U\cap f^{-1}(V))\arrow[r,phantom, "{\subseteq}"] & \varphi(U) \arrow[r,"\psi\circ f\circ\varphi^{-1}"] & \psi(V)& 
\end{tikzcd}
\]

\

\begin{defi} Una aplicación biyectiva $f\in\esp{F}(M,N)$ se dice que es un {\bf Difeomorfismo} si
$f^{-1}\in\esp{F}(N,M)$. Dos variedades diferenciables se dicen
{\bf Difeomorfas} si existe un difeomorfismo entre ambas.
\end{defi}
Cuando la variedad $N$ es $\R$ con su estructura eucl\'{\i}dea, las aplicaciones de $M$ en $\R$ se suelen llamar
funciones. As\'{\i}, a una función continua $f:G\fl\R$, si es diferenciable (en el sentido de la Definición 2.1.1) en
un punto $p\in M$, se llama {\bf Función Diferenciable en $p$} y al conjunto de tales funciones se denota por
$\esp{F}(p)$ y si es diferenciable en todo $G$ (que podr\'{\i}a ser el propio $M$), se llama {\bf Función Diferenciable
en $G$} y el conjunto de tales funciones se denota por $\esp{F}(G)$, verificándose que:
$$\esp{F}(G)=\bigcap_{p\in G}\esp{F}(p).$$
\begin{prop}
Dado $G\subseteq M$ abierto, $\esp{F}(G)$ es un álgebra asociativa, conmutativa y con elemento unidad para las
operaciones suma y producto de funciones y producto por números reales.
\end{prop}
\begin{ejs}
{\rm Sea $(U,\vp=(x_1,\dots ,x_m))$ un s.l.c. Entonces:
\begin{enumerate}
\item $x_i\in\esp{F}(U)$, para todo $i=1,\dots ,m$.
\item Sea $f\in\esp{F}(U)$. Dado cualquier $i=1,\dots ,m$ se define:
$$\ddep{f}{x_i}:p\in U\longmapsto \left(\ddep{f}{x_i}\right)_p=\left(\ddep{(f\circ\vp^{-1})}{u_i}\right)_{\vp(p)}\in\R.$$
\hs Veamos que $\displaystyle{\ddep{f}{x_i}}\in\esp{F}(U)$, $i=1,\dots ,m$. Sea $(V,ψ)$ una carta cualquiera en $M$ tal que $U \cap V \neq \emptyset$. Basta ver que $\displaystyle{\ddep{f}{x_i}}\circ ψ^{-1}$ es diferenciable en $φ(U \cap V)$:
\[ \ddep{f}{x_i} \circ ψ^{-1} (q) = \left(\ddep{f}{x_i}\right)_{ψ^{-1}(q)} = \left.\ddep{(f \circ φ^{-1})}{u_i}\right|_{φ(ψ^{-1}(q))} \]
Luego
\[ \ddep{f}{x_i} \circ ψ^{-1} = \ddep{(f \circ φ^{-1})}{u_i} \circ (φ \circ ψ^{-1}) \in C^{\infty} \]
\end{enumerate}}
\end{ejs}
\begin{prop}
{\bf (Lema de Extensión de Funciones Diferenciables).} Sea $h\in\esp{F}(G)$, donde $G$ es un abierto conteniendo a
$p\in M$. Entonces, existen un entorno abierto $V$ de $p$, con $\overline{V}\subseteq G$ y una función
$f\in\esp{F}(M)$ tales que $f\equiv h$ en $V$ y $f\equiv 0$ en $M-G$.
\end{prop}
\begin{teorema}
Una aplicación continua $f\in\esp{F}(M,N)$ si y sólo si
para toda función $g\in\esp{F}(N)$, $g\circ f\in\esp{F}(M)$.
\end{teorema}
\begin{dem}
Supongamos que $f \in \esp{F}(M,N)$ y sea $g \in \esp{F}(N)$ cualquiera. Queremos probar que $g \circ f \in \esp{F}(M)$, es decir, que $(g \circ f) \circ φ^{-1} : φ(U) \to \R$ es diferenciable para toda carta $(U,φ)$ en $M$.

Para ello, sea $(U,φ)$ en $M$ cualquiera y veamos que $(g \circ f) \circ φ^{-1}$ es diferenciable en un punto $q \in φ(U)$ cualquiera.

Como $q \in φ(U)$, existe $p \in U$ tal que $q = φ(p)$. En particular, $p \in M$, con lo que $f(p) \in N$ y podemos tomar una carta $(V,ψ)$ en $N$ tal que $f(p) \in V$. Ahora bien, como $p \in U$ y $f(p) \in V$, se tiene que $U \cap f^{-1}(V) \neq \emptyset$. Así, como, por hipótesis, $f \in \esp{F}(M,N)$, podemos afirmar que $ψ \circ f \circ φ^{-1}$ es diferenciable en $φ(U \cap f^{-1}(V))$.

Por otra parte, como $(V,ψ)$ es carta en $N$ y $g \in \esp{F}(N)$, sabemos que $g \circ ψ^{-1} : ψ(V) \to \R$ es diferenciable.

Así, en $φ(U \cap f^{-1}(V))$, que es entorno abierto de $φ(p)=q$, podemos escribir
\[ (g \circ f) \circ φ^{-1} = (g \circ ψ^{-1} \circ (ψ \circ f \circ φ^{-1}) \]
que es diferenciable al obtenerse como composición de aplicaciones diferenciables. Por lo tanto, $(g \circ f) \circ φ^{-1}$ es diferenciable en $q$ y queda probada la implicación directa.

Recíprocamente, supongamos que $g \circ f \in \esp{F}(;)$, para toda $g \in \esp{F}(N)$ y probemos que $f \in \esp{F}(M,N)$, es decir, consideremos dos cartas cualesquiera $(U,φ)$ en $M$ y $(V,ψ)$ en $N$ tales que $U \cap f^{-1}(V) \neq \emptyset$ y veamos que la aplicación $ψ \circ f \circ φ^{-1} : φ(U \cap f^{-1}(V)) \to \R^n$ es diferenciable. Para ello, basta probar que sus componentes $u_i \circ ψ \circ f \circ f^{-1} = y_i \circ f \circ φ^{-1}$ son diferenciables para todo $i = 1,\dots,n$ donde hemos denotado por $(y_1,\dots,y_n)$ las componentes de la carta $(V,ψ)$. Por tanto, sea $q \in φ(U \cap f^{-1}(V))$ cualquiera y veamos que cada $y_i \circ f \circ φ^{-1}$ es diferenciable en $q$.

Si probamos que existen un abierto $\tilde{W}$ on $q \in \tilde{W} \subseteq φ(U \cap f^{-1}(V))$ y una función $\tilde{y}_i \in \esp{F}(N)$ taes que en $\tilde{W}$ se verifica que $y_i \circ f \circ φ^{-1} = \tilde{y}_i \circ f \circ φ^{-1}$, habremos acabado, pues, aplicando la hipótesis, se tiene que $\tilde{y}_i \circ f \in \esp{F}(M)$, con lo que $\tilde{y}_i \circ f \circ φ^{-1}$ es diferenciable. Así, $y_i \circ f \circ φ^{-1}$ es diferenciable en $q$, pues en un entorno de dicho punto coincide con la aplicación diferenciable $\tilde{y}_i \circ f \circ φ^{-1}$.

Ahora bien, para probar la existencia de tales $\tilde{W}$  $\tilde{y}_i$, razonamos de la siguiente manera. Como $q \in φ(U \cap f^{-1}(V))$, existe $p \in U \cap f^{-1}(V)$ tal que $q = φ(p)$. Así, $f(p) \in V$. Además, sabemos que $V$ es abierto de $N$ y que $y_i \in \esp{F}(V)$ (pues las componentes de una carta son diferenciables en el dominio de la misma. Aplicando ahora el Lema de Extensión de Funciones Diferenciables, obtenemos que eisten un entorno abierto $W$ de $f(p)$ con $\overline{W} \subseteq V$ y una función $\tilde{}_i \in \esp{F}(N)$ tales que $\tilde{y}_i = y_i$ en $W$ e $\tilde{y}_i = 0$ en $N \setminus V$. Finalmente, bastará tomar $\tilde{W} = φ(U \cap f^{-1}(W))$.
\qed
\end{dem}
\section{Ejercicios.}
\begin{enumerate}
\item Sean $M$ y $N$ dos variedades diferenciables y $G\subseteq M$ un abierto.
\begin{enumerate}
\item Si $f\in\esp{F}(G,N)$ y $H\subseteq G$ es un abierto, probar que $f|_H\in\esp{F}(H,N)$.
\item Si $f:G\fl N$ es una aplicación y $\{G_i\}_{i\in I}$ es un recubrimiento por abiertos de $G$ (es decir,
$G=\bigcup_{i\in I}G_i$) tal que $f|_{G_i}\in\esp{F}(G_i,N)$, para todo $i\in I$, probar que $f\in\esp{F}(G,N)$.
\end{enumerate}
\item Sean $M$ una variedad diferenciable y $G\subseteq M$ un
abierto. Probar que la aplicación inclusión
$i:G\hookrightarrow M$ es diferenciable. \item Probar que la
composición de aplicaciones diferenciables es una
aplicación diferenciable. \item Probar que la aplicación
identidad entre una variedad diferenciable y ella misma es un
difeomorfismo. \item Sean $(M,\esp{A}_1)$ y $(M,\esp{A}_2)$ dos
variedades diferenciables. ?`Bajo qué condiciones es ${\rm
Id}: (M,\esp{A}_1)\fl (M,\esp{A}_2)$ una aplicación
diferenciable? ?`Cuándo es un difeomorfismo? \item Probar que
cualquier aplicación constante es diferenciable. %\item Probar
%que si dos variedades diferenciables son difeomorfas, entonces
%tienen la misma dimensión.
\end{enumerate}
\end{document}