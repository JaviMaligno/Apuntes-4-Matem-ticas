\documentclass[Cursovd_portada.tex]{subfiles}

\begin{document}

\chapter{Campos de Vectores sobre\\ una Variedad Diferenciable}
\section{Campos de Vectores Diferenciables.}
\begin{defi}
Un {\bf Campo de Vectores} $X$ en un abierto $U\subseteq M$ de una variedad diferenciables es una ley que a cada
punto $p\in U$ le asigna un vector tangente $X_p\in T_p(M)$.
\end{defi}

\begin{ej}
{\rm Sea $(U,\vp=(x_1,\dots ,x_m))$ un s.l.c. en $M$. Entonces, para cualquier $i=1,\dots ,m$,
$$\dep{x_i}:p\in U\mapsto\left(\dep{x_i}\right)_p$$
es un campo de vectores en $U$.}
\end{ej}

Un campo de vectores $X$ en $U$ también puede interpretarse como una aplicación que envía funciones diferenciables
en $U$ en funciones de $U$, de la siguiente forma: si $f\in\mathcal{F}(U)$, se define $Xf:U\fl\R$ por
$(Xf)(p)=X_pf$.
\begin{defi}
Un campo de vectores en $U$ se dice {\bf Diferenciable} si para toda función $f\in\mathcal{F}(U)$ se tiene que
$Xf\in\mathcal{F}(U)$. Al conjunto de campos diferenciables en $U$ se denota por $\mathcal{X}(U)$.
\end{defi}
Por tanto, un campo diferenciable en $U$ puede interpretarse como una aplicación
$X:\mathcal{F}(U)\fl\mathcal{F}(U)$. Esta aplicación verifica las dos siguientes propiedades:
\begin{enumerate}
\item $X(\lambda f+\mu g)=\lambda Xf+\mu Xg$, para cualesquiera $f,g\in\mathcal{F}(U)$ y $\lambda,\mu\in\R$
($\R$--linealidad).
\item $X(fg)=gXf+fXg$, para cualesquiera $f,g\in\mathcal{F}(U)$.
\end{enumerate}
\hs Por esta razón, se dice que los campos diferenciables de vectores son derivaciones sobre el conjunto de las
funciones diferenciables.
\par
\begin{lemma}
Si $(U,\vp=(x_1,\dots ,x_m))$ es un s.l.c. en $M$, entonces, dado un campo de vectores $X$ en $U$,
para cualquier $p\in U$ se puede escribir
$$X_p=\sum_{i=1}^m(X_p(x_i))\left(\dep{x_i}\right)_p,$$
es decir,
$$X=\sum_{i=1}^mf_i\dep{x_i},$$
para cierta funciones $f_i:U\fl\R$, donde $f_i=Xx_i$, para todo
$i=1,\dots ,m$.
\end{lemma}
\begin{prop}
Sea $X$ un campo de vectores en $U$. Las condiciones siguientes son equivalentes:
\begin{enumerate}
\item $X\in\mathcal{X}(U)$. \item Si $(V,\vp=(x_1,\dots ,x_m))$ es
un s.l.c. con $V\subseteq U$, entonces
$$X|_V=\sum_{i=1}^m(X|_Vx_i)\left.\dep{x_i}\right\vert_V,$$
donde $X|_Vx_i\in\mathcal{F}(V)$, para todo $i=1,\dots ,m$. \item
Para todo $p\in U$, existe $(V,\vp=(x_1,\dots ,x_m))$, s.l.c.
entorno de $p$, con $V\subseteq U$, tal que
$$X|_V=\sum_{i=1}^m(X|_Vx_i)\left.\dep{x_i}\right\vert_V,$$
donde $X|_Vx_i\in\mathcal{F}(V)$, para todo $i=1,\dots ,m$.
\end{enumerate}
\end{prop}

\begin{dem}\mbox{}
\begin{enumerate}
\item[$(1 \Rightarrow 2)$] Basta probar que $X|_V \in \mathcal{X}(V)$ y aplicar el lema anterior. Veamos que para todo $p \in V$ existe $W$ abierto tal que $X|_V f$ es diferenciable en $W$. Por el lema de extensión de funciones, existe $W$ abierto tal que $\overline{W} \subseteq V$ y existe $h \in \mathcal{F}(M)$ tal que $h = f$ en $W$. Entonces $X h|_U \in \mathcal{F}(U)$, luego $(X h|_U)|_W \in \mathcal{F}(W)$. Finalmente, veamos que $(X|_V f)|_W = (X h |_U)_W$. En efecto, para cualquier $q \in W$:
\[ (X|_V f)|_W(q) = (X|_V f)_q = (X|_V)_q f = X_q f \]
\[ (X h|_U)|_W(q) = (X h|_U)_q = X_q h|_U \]
Como $f = h|_U$ en $W$, $(X|_V f)|_W(q) = (X h|_U)|_W(q)$ para todo punto $q \in W$, luego $(X|_V f)|_W = (X h|_U)|_W \in \mathcal{F}(W)$.
\item[$(2 \Rightarrow 3)$] Trivial.
\item[$(3 \Rightarrow 1)$] Sea $f \in \mathcal{F}(U)$. Sea $p \in U$ cuaquiera. Por hipótesis, existe $(V,φ)$ carta en $M$ tal que $p \in V \subseteq U$ y $X|_V = \sum_{i=1}^m f_i \frac{\partial}{\partial x_i} \in \mathcal{X}(V)$. Como $f \in \mathcal{F}(U)$ y $V \subseteq U$, $f|_V \in \mathcal{F}(V)$. Luego $X|_V f|_V \in \mathcal{F}(V)$. Para cualquier punto $q \in V$:
\[ (X f)|_V (q) = (X f)_q = X_q f \]
\[ (X|_V f|_V) (q) = (X|_V)_q f|_V = X_q f|_V \]
Como $f = f|_V$ en $V$, $(X f)|_V = X|_V f|_V \in \mathcal{F}(V)$. Por la localidad de la diferenciabilidad, $X f \in \mathcal{F}(U)$ y $X \in \mathcal{X}(U)$. $\QED$
\end{enumerate}
\end{dem}

\

\begin{prop}
En $\mathcal{X}(U)$ se pueden definir las siguientes operaciones:
\begin{enumerate}
\item Suma: si $X,Y\in\mathcal{X}(U)$, $(X+Y)_p=X_p+Y_p$, para todo $p\in U$. Se tiene que $X+Y\in\mathcal{X}(U)$.
\item Producto por números reales: si $X\in\mathcal{X}(U)$ y $\lambda\in\R$, $(\lambda X)_p=\lambda X_p$, para todo
$p\in U$. Se tiene que $\lambda X\in\mathcal{X}(U)$.
\item Producto por funciones diferenciables: si $X\in\mathcal{X}(U)$ y $f\in\mathcal{F}(U)$, $(fX)_p=f(p)X_p$, para todo
$p\in U$. Se tiene que $fX\in\mathcal{X}(U)$.
\end{enumerate}
\end{prop}
\begin{dem}\mbox{}
\begin{enumerate}
\item $\forall g \in \mathcal{F}(U)$, $((X+Y)g)_p = (X+Y)_pg = (X_p+Y_p)g = X_p+Y_pg = (Xg)_p+(Yg)_p = (Xg+Yg)_p$, luego $(X+Y)g = Xg+Yg \in \mathcal{F}(U)$.
\item[3.] $((fX)g)_p = (fX)_pg = f(p)X_pg = (fXg)_p$, luego $(fX)g = f(Xg) \in \mathcal{U}$.
\item[2.] Tómese $g$ la función constante $λ$. $\QED$
\end{enumerate}
\end{dem}
\begin{prop} $\mathcal{X}(U)$ es un espacio vectorial real con la suma y el producto de números reales y un
$\mathcal{F}(U)$--módulo con la suma y el producto por funciones diferenciables.
\end{prop}
\begin{prop}
Si $(U,\vp=(x_1,\dots ,x_m))$ es un s.l.c. en $M$, entonces el $\mathcal{F}(U)$--módulo $\mathcal{X}(U)$ está
finitamente generado y tiene como base:
$$\left\{\dep{x_1},\dots ,\dep{x_m}\right\}.$$
\end{prop}
En $\mathcal{X}(U)$ puede definirse otra operación interna, llamada {\it Producto Corchete}, mediante
$[X,Y]f=X(Yf)-Y(Xf)$, para todos $X,Y\in\mathcal{X}(U)$ y $f\in\mathcal{F}(U)$. Esta operación tiene las
siguientes propiedades:
\begin{prop}
Dados $X,Y,Z\in\mathcal{X}(U)$ y $f,g\in\mathcal{F}(U)$, se verifica que:
\begin{enumerate}
\item $[X+Y,Z]=[X,Z]+[Y,Z]$; $[X,Y+Z]=[X,Y]+[X,Z]$.
\item $[X,Y]=-[Y,X]$ (anticonmutatividad).
\item $[X,[Y,Z]]+[Y,[Z,X]]+[Z,[X,Y]]=0$ (Identidad de Jacobi).
\item $[fX,gY]=fg[X,Y]+f(Xg)Y-g(Yf)X$.
\item Si $(U,\varphi=(x_1,\dots,x_n)$ es ua carta en $M$.  $\left[\dfrac{\partial}{\partial x_i},\dfrac{\partial}{\partial x_j}\right]=0\ \forall i,j=1,\dot, n$. 
\end{enumerate}
\end{prop}
\begin{dem}\
\begin{enumerate}
\item Ejercicio trivial
\item Trivial. 
\item Ejercicio.
\item Con números reales se tiene la linealidad porque un campo sobre una constante es 0. 

$\forall p\in U,\forall h\in\calF(p)$
\begin{gather*}
[fX,gY]_ph=(fX)_p((gY)h)-(gY)_p((fX)h)=f(p)(X_pg\cdot(Yh)_p+g(p)\cdot X_p(Yh))-g(p)(Y_pf\cdot(Xh)=\\
f(p)g(p)(X_p(Yh)-Y_p(Xh))+f(p)(X_pg)Y_ph-g(p)(Y_pf)X_ph=\\
f(p)g(p)[X,Y]_p+f(p)(X_p g)Y_p-g(p)(Y_pf)X_p
\end{gather*}
$\QED$


\item Es una cuestión.
\end{enumerate}
\end{dem}
\begin{coro}
$\mathcal{X}(U)$ con las operaciones suma, producto por escalares (nú\-me\-ros reales, que realmente son casos
particulares de funciones diferenciables, las funciones constantes) y producto corchete es un álgebra de Lie real.
\end{coro}

\begin{prop}
El corchete es un campo diferenciable.
\end{prop}
\begin{dem}
Se deja como ejercicio comprobar que es campo de vectores comprobando que su imagen es un vector tangente viendo que cumple las propiedades de los vectores tangentes. Probemos la diferenciabilidad.

$\forall p\in U\ [X,Y]f(p)=[X,Y]_pf=X_p(Yf)-Y_p(Xf)=(X(YF)-Y(Xf))_p\Rightarrow [X,Y]f=X(Yf)-Y(Xf)$. Como ambos términos son diferenciables, el corchete es diferenciable. $\QED$
\end{dem}

\begin{teorema} {\bf (Lema de Extensión de Campos Diferenciables).} Sea $X\in\mathcal{X}(U)$ y $V\supseteq U$ otro abierto.
Entonces, dado cualquier $p\in U$, existen $\overline{X}\in\mathcal{X}(V)$ y $W$ abierto con $p\in W\subseteq U$
tales que $\overline{X}_q=X_q$, para todo $q\in W$ y $\overline{X}=0$ en $V-U$.
\end{teorema}
\begin{dem}
Apéndice. $\QED$
\end{dem}

\begin{nota}[Una visión alternativa] Dada una variedad $M$ de dimensión $m$ se puede definir $TM = \coprod_{p \in M} T_p(M)$ llamado el fibrado tangente. Se puede dotar a $TM$ de estructura de V.D. de dimensión $2m$. Para cada $(U, φ=(x_1,\dots,x_m))$ cara en $M$ se define una carta $(\tilde{U},\tilde{φ})$ en $TM$: $\forall (p,v) \in \tilde{U} \subseteq TM$, $\tilde{φ}((p,v)) = (x_1(p),\dots,x_m(p),v(x_1),\dots,v(x_m)) \in \R^{2m}$. La proyección $π: TM \to M$ que asocia $(p,v) \in TM$ a $p$ está en $\mathcal{F}(TM,M)$.

En este contexto, podemos definir $X$ en $M$ como una aplicación $X : M \to TM$ tal que $π \circ X = id_M$. Es decir, $X$ es una sección\footnote{\url{https://en.wikipedia.org/wiki/Section_(fiber_bundle)}} del fibrado $TM$. Se puede probar que $X \in \mathcal{X}(M)$ si y sólo si $X \in \mathcal{F}(M,TM)$.
\end{nota}
\section{Campos $f$-relacionados}
\hs Sean $M$ y $N$ dos variedades diferenciables, $f\in\mathcal{F}(M,N)$ y $U\subseteq M$ un abierto. Se podría
pensar en definir un campo de vectores $Y$, imagen de $X$ por $f$, escribiendo $Y_{f(p)}=f_{*p}X_p\in
T_{f(p)}(N)$. Esto no es siempre posible por diferentes razones. En primer lugar, $f(U)$ no tiene por qué ser
abierto. Además, si $f$ no es inyectiva, puede haber puntos de $N$ que tengan asignados más de un vector. En
efecto, si $f(p)=f(q)$, siendo $p\neq q$, entonces, en general, $Y_{f(p)}=f_{*p}X_p\neq
f_{*q}X_q=Y_{f(q)}=Y_{f(p)}$. En todo caso, aunque $f(U)$ sea abierto y $f$ sea inyectiva, puede suceder que el
campo de vectores $Y$ no sea diferenciable.
\par
Lo que se va a hacer es considerar campos de vectores en $N$ que sean ``imagen" de un campo diferenciable en $M$ y
a ambos se llamarán $f$--relacionados.
\begin{defi}
Sean $M$ y $N$ dos variedades diferenciables, $f\in\mathcal{F}(M,N)$ y $X\in\mathcal{X}(M)$ e $Y\in\mathcal{X}(N)$
dos campos. Se dice que $X$ e $Y$ están $f$--relacionados si $f_{*p}X_p=Y_{f(p)}$, para cualquier $p\in M$ y se
escribe $X\stackrel{f}\sim Y$.
\end{defi}

\newpage

\begin{prop}
Dados $X_1,X_2\in\mathcal{X}(M)$, $Y_1,Y_2\in\mathcal{X}(N)$, $f\in\mathcal{F}(M,N)$ y $g\in\mathcal{F}(N)$, se
verifican las siguientes propiedades:
\begin{enumerate}
\item Si $X_1\stackrel{f}\sim Y_1$ y $X_2\stackrel{f}\sim Y_2$, entonces, para cualesquiera $\lambda,\mu\in\R$,
$\lambda X_1+\mu X_2\stackrel{f}\sim \lambda Y_1+\mu Y_2$.
\item Si $X_1\stackrel{f}\sim Y_1$ y $X_2\stackrel{f}\sim Y_2$, entonces $[X_1,X_2]\stackrel{f}\sim [Y_1,Y_2]$.
\item Si $X_1\stackrel{f}\sim Y_1$, entonces $(g\circ f)X_1\stackrel{f}\sim gY_1$.
\end{enumerate}
\end{prop}
\begin{dem}\
\begin{enumerate}
\item $\forall p\in M\ f_{*p}(\lambda X_2+\mu X_2)_p=f_{*p}(\lambda X_{1p}+\mu X_{2p})=\lambda f_{*p}X_{1p}+\mu f_{*p}X_{2p}=(\lambda Y_1 +\mu Y_2)_{f(p)}$.
\item $\forall p\in M,\forall h\in\calF(p)\ f_{*p}[X_1,x_2]_ph=[X_1,X_2]_p(h\circ f)=X_{1p}(X_2(h\circ f))-X_{2p}(X(h\circ f))$. Para cualquier $q$ del dominio, $(X_2(h\circ f))_q=X_{2q}(h\circ f)=(f_{*q}X_{21})h=Y_{2f(q)}h=(Y_2h)_{f(q)}=(Y_2h\circ f)_q$. Luego la expresión anterior la podemos reemplazar por $X_{1p}(Y_2h\circ f)-X_{2p}(Y_1h\circ f)=(f_{*p}X_{1p})(Y_2h)-(f_{*p}X_{2p}(Y_1h)=Y_{1f(p)}(Y_2h)-Y_{2f(p)}(Y_1h)=[Y_1,Y_2]_{f(p)}h.$
\item $\forall p\in M\ f_{*p}((g\circ g)X_1)_p=f_{*p}(g(f(p))X_{1p})=g(f(p))f_{*p}X_{1p}=g(f(p))Y_{1f(p)}=(gY_1)_{f(p)}$. $\QED$
\end{enumerate}
\end{dem}
\begin{coro}
Dada $f\in\mathcal{F}(M,N)$, el conjunto de los campos diferenciables en $N$ que están $f$--relacionados con algún
campo diferenciable de $M$ es un subálgebra de Lie de $\mathcal{X}(N)$.
\end{coro}
En el caso de que $f\in\mathcal{F}(M,N)$ sea un difeomorfismo, se
puede definir una aplicación
$f_*:\mathcal{X}(M)\fl\mathcal{X}(N)$, mediante
$(f_*X)_q=f_{*f^{-1}(q)}X_{f^{-1}(q)}$, para todo $q\in N$. Puede
comprobarse que $f_*$ está bien definida, es decir, que $f_*X$
es un campo diferenciable en $N$ si $X$ es un campo diferenciable
en $M$. De hecho, $(f_*X)g=X(g\circ f)\circ f^{-1}$, para toda
$g\in \mathcal{F}(N)$. Además, $f_*X$ es el único campo de
$N$ que est\'a $f$--relacionado con el campo $X$ de $M$.
\begin{prop}
Si $f\in\mathcal{F}(M,N)$ es un difeomorfismo, entonces 
\begin{enumerate}
\item $f_*$ está bien definida.
\item $f_*$ es un isomorfismo de álgebras de Lie (lineal y se mete en los dos argumentos del corchete).
\item $\forall X\in\mathcal{X}(M)$, $f_*X$ es el único campo de $\mathcal{X}(M)$ tal que $X\overset{f}{\sim} f_*X$.
\end{enumerate}
\end{prop}
\begin{dem}\
\begin{enumerate}
\item $\forall X\in\mathcal{X}(M),\forall q\in N$ vamos a probar que $f_*X\in\mathcal{X}(N)$. $(f_*X)_q=f_{*f^{-1}(q)}X_{f^{-1}(q)}$. El primero está en $T_q(N)$ y el segundo en $T_{f^{-1}(q)}(M)$, por lo que es campo. Ahora veamos que es diferenciable.
$\forall g\in\calF(N),\forall q\in N$ $((f_*X)g)_q=(f_*X)_qg=(f_{*f^{-1}(q)}X_{f^{-1}(q)})g=X_{f^{-1}(q)}(g\circ f)=(X(g\circ f))_{f^{-1}(q)}=(X(g\circ f)\circ f^{-1})_q$.

La fórmula $(f_*X)g=X(g\circ f)\circ f^{-1}$ es útil.
\item Ejercicio. Utilizando el apartado 3 y la proposición anterior es trivial.
\item $(f_*X)_{f(p)}=f_{*f^{-1}(f(p))}X_{f^{-1}(f(p))}=f_{*p}X_p$. Probemos la unicidad. 

Sea $Y\in\mathcal{X}(N)$ tal que $X\overset{f}{\sim}$. Entonces $f_{*p}X_p=Y_{f(p)}\forall p\in M$. Por otra parte, $\forall q\in N, Y_q=Y_{f(f^{-1}(q))}$, que por lo anterior es igual a $f_{*f^{-1}(q)}X_{f^{-1}(q)}=(f_*X)_q\Rightarrow Y=f_*X$. $\QED$
\end{enumerate}
\end{dem}
\begin{nota}
{\rm Cabe preguntarse si todo automorfismo de $\mathcal{X}(M)$ proviene, mediante el proceso anterior, de una
transformación (difeomorfismo) de $M$. Este es un problema abierto.}
\end{nota}
\section{Ejercicios.}
\begin{enumerate}
\item Dados una variedad diferenciable $M$, un punto $p\in M$ y un
vector tangente $u\in T_p(M)$, probar que existe un campo
$X\in\mathcal{X}(M)$ tal que $X_p=u$. \item Si $(U,\vp=(x_1,\dots
,x_m))$ es un s.l.c., probar que
$\left[\dep{x_i},\dep{x_j}\right]=0$, para todos $i,j=1,\dots ,m$.
\item Dados $(U,\vp=(x_1,\dots ,x_m))$ un s.l.c. y
$X,Y\in\mathcal{X}(U)$ tales que
$$X=\sum_{i=1}^mX^i\dep{x_i},\mbox{ }Y=\sum_{j=1}^mY^j\dep{x_j},$$
calcular $[X,Y]$ en función de la base $\left\{\dep{x_1},\dots ,\dep{x_m}\right\}$.
\end{enumerate}

\end{document}