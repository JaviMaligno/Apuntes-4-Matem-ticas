\documentclass[cursovd_portada.tex]{subfiles}

\begin{document}


%\hyphenation{equi-va-len-cia}\hyphenation{pro-pie-dad}\hyphenation{res-pec-ti-va-men-te}\hyphenation{sub-es-pa-cio}
\chapter{Complementos de Topología\\ General.}
\section{Espacios Topológicos.}
\begin{defi}
Un {\bf Espacio Topológico} es un par $(X,T)$, donde $X$ es un conjunto y $T$ una familia de subconjuntos de $X$,
llamada una {\bf Topología sobre $X$} y cuyos elementos son llamados {\bf Conjuntos Abiertos}, verificándose las
siguientes propiedades:
\begin{enumerate}
\item El conjunto vacío $\emptyset$ y el propio $X$ son (conjuntos) abiertos.
\item La intersección de cualquier cantidad finita de abiertos es un abierto.
\item La unión de cualquier cantidad de abiertos es un abierto.
\end{enumerate}
\end{defi}
\begin{defi}
Un {\bf Espacio Topológico} es un conjunto $X$ tal que para cada punto $x\in X$ existe una familia $\esp{N}_x$ de
subconjuntos de $X$, llamados {\bf Entornos del punto $x$}, cumpliéndose:
\begin{enumerate}
\item Cada punto está en todos sus entornos.
\item La intersección de dos entornos de un punto es también entorno de ese punto.
\item Un conjunto que contenga a un entorno de un punto es, a su vez, entorno de dicho punto.
\item Dado un entorno $N$ de un punto $x$, existe otro entorno $M$ de $x$ tal que $N$ es entorno de todos los
puntos de $M$.
\end{enumerate}
\end{defi}
\begin{teorema}
Las dos definiciones anteriores son equivalentes.
\end{teorema}
\begin{nota}
Según se elija una u otra de las definiciones anteriores, las siguientes afirmaciones se tendrán bien como
definición, bien como proposición:
\begin{enumerate}
\item Un conjunto es entorno de un punto si y sólo si existe un abierto conteniendo al punto y contenido en el
conjunto.
\item Un conjunto es abierto si y sólo si es entorno de todos sus puntos.
\end{enumerate}
\end{nota}
\begin{prop}
Un conjunto es abierto si y sólo si para cada punto suyo existe otro abierto que contiene al punto y está
contenido en el conjunto.
\end{prop}
\begin{defi}
Sea $(X,T)$ un espacio topológico. Una familia de subconjuntos de $X$, $\esp{B}\subseteq T$, se dice que es una
{\bf Base de la Topología} si todo abierto es unión de elementos de $\esp{B}$.
\end{defi}
\begin{teorema}
Sea $\esp{B}\subseteq T$. Las condiciones siguientes son equivalentes:
\begin{enumerate}
\item $\esp{B}$ es base de $T$.
\item Para cada abierto y para cada punto suyo, existe un elemento de $\esp{B}$ que contiene al punto y está
contenido en el abierto.
\end{enumerate}
\end{teorema}
\begin{defi}
Sea $(X,T)$ un espacio topológico y $x$ un punto de $X$. Una familia $\esp{B}_x$ de entornos de $x$ se dice {\bf
Base de Entornos} de $x$ si todo entorno de $x$ contiene algún elemento de la familia.
\end{defi}
\begin{defi}
En un espacio topológico $(X,T)$, los conjuntos complementarios de los conjuntos abiertos se llaman {\bf Conjuntos
Cerrados} y la familia de los conjuntos cerrados se denota por $\esp{F}$.
\end{defi}
\begin{prop}
La familia $\esp{F}$ de los conjuntos cerrados de un espacio to\-po\-ló\-gi\-co $(X,T)$ verifica las siguientes
propiedades:
\begin{enumerate}
\item El conjunto vacío $\emptyset$ y el propio conjunto $X$ son (conjuntos) cerrados.
\item La unión de cualquier cantidad finita de cerrados es un cerrado.
\item La intersección de cualquier cantidad de cerrados es un cerrado.
\end{enumerate}
\end{prop}
\begin{teorema}
Sea $X$ un conjunto y $\esp{F}$ una familia de subconjuntos de $X$ que verifican las mismas propiedades que una
familia de conjuntos cerrados, recogidas en la proposición anterior. Entonces, existe una única topología sobre
$X$, definida por
$$T=\{G\subseteq X/X-G\in\esp{F}\},$$
para la que $\esp{F}$ es la familia de cerrados.
\end{teorema}
A continuación, se van a presentar un resultado que permite construir una topología sobre un conjunto $X$ a partir
de cualquier familia de sus subconjuntos.
\begin{teorema}\label{teorema}
Sea $X$ un conjunto y sea $\esp{A}$ una familia de subconjuntos de $X$. Entonces la familia formada por el
conjunto vacío $\emptyset$, el propio conjunto $X$ y todas las uniones posibles que se puedan realizar con todas
las intersecciones finitas de elementos de $\esp{A}$ es una topología sobre $X$, que contiene a $\esp{A}$ y que,
además, es la topología más peque\~{n}a de las que contienen a $\esp{A}$.
\end{teorema}
\begin{ejer}
\begin{enumerate}
{\rm
\item Escríbanse con terminología matemática todas las definiciones y resultados anteriores. Sería también
conveniente probar dichos resultados.
\item ?`Bajo qué condiciones una familia $\esp{A}$ de subconjuntos de un conjunto $X$ es base de la topología que
se obtiene según el proceso descrito en el Teorema \ref{teorema} (partiendo de la propia $\esp{A}$)?
\item Sea $X$ un conjunto y $d$ una métrica sobre él. Probar que existe una topología sobre $X$ para la que las
bolas abiertas de $d$ forman base. Dicha topología se llama {\bf Topología Métrica asociada a $d$}.}
\end{enumerate}
\end{ejer}
\begin{ejer}
\begin{enumerate}
{\rm
\item Sea $X$ un conjunto cualquiera y $T_{dis}=\esp{P}(X)$. $T_{dis}$ es una topología sobre $X$, llamada {\bf
Topología Discreta} y es la topología mayor, es decir, con más elementos, que puede construirse sobre $X$. Además,
la familia formada por el conjunto vacío $\emptyset$ y todos los conjuntos unitarios es base de dicha topología.
\item En el conjunto de los números reales $\R$, considérese la familia formada por todos los intervalos abiertos
y constrúyase la menor topología que contiene a tal familia, siguiendo el proceso descrito en el Teorema \ref{teorema}.
Dicha topología se llama {\bf Topología Euclídea de $\R$}, es la topología métrica asociada a la métrica euclídea
y tiene a la familia de todos los intervalos abiertos como base.
\item En $\R^m$ considérese la topología métrica asociada a la métrica euclídea. Dicha topología se llama {\bf
Topología Euclídea de $\R^m$} y las bolas abiertas (por ejemplo, en $\R^2$ son los discos abiertos) forman una
base de ella.}
\end{enumerate}
\end{ejer}
\begin{defi}
Sea $(X,T)$ un espacio topológico, $A\subseteq X$ y $x\in X$. Se dice que:
\begin{enumerate}
\item $x$ es un {\bf Punto Adherente} a $A$ si todo abierto que contenga a $x$ corta a $A$. Al conjunto de los
puntos adherentes a $A$ se le llama la {\bf Clausura} de $A$ y se denota por $\overline{A}$.
\item $x$ es un {\bf Punto de Acumulación} de $A$ si todo abierto que contenga a $x$ corta a $A$ en algún punto distinto
de $x$. Al conjunto de los puntos de acumulación de $A$ se le llama el {\bf Derivado} de $A$ y se denota por $A'$.
\item $x$ es un {\bf Punto Interior} de $A$ si hay algún abierto que contenga a $x$ contenido en $A$. Al conjunto de los
puntos interiores de $A$ se le llama el {\bf Interior} de $A$ y se denota por ${\rm int}(A)$.
\end{enumerate}
\end{defi}

\newpage

\begin{prop}
Sea $(X,T)$ un espacio topológico. Entonces:
\begin{enumerate}
\item La clausura de un conjunto es el menor cerrado que lo contiene. Por tanto, un conjunto es cerrado si y sólo
si coincide con su clausura.
\item El interior de un conjunto es el mayo abierto contenido en él. Por tanto, un conjunto es abierto si y sólo
si coincide con su interior.
\end{enumerate}
\end{prop}
\begin{defi}
Sea $(X,T)$ un espacio topológico y $A\subseteq X$. Se llama {\bf To\-po\-lo\-gía Relativa} o {\bf Topología
Inducida} de $X$ a $A$ a:
$$T_A=\{G\cap A/G\in T\}.$$
\hs Al par $(A,T_A)$ se le llama {\bf Subespacio Topológico} de $(X,T)$. Una propiedad se dice que es {\bf
Hereditaria} para una clase de subconjuntos de $X$ si la verica el espacio topológico $(X,T)$ y cualquier
subespacio $(A,T_A)$, con $A$ perteneciendo a la clase especificada.
\end{defi}
\begin{ejer}
{\rm Probar que, efectivamente, $T_A$ es una topología sobre $A$.}
\end{ejer}
\section{Aplicaciones entre Espacios Topológicos. Ho\-meo\-mor\-fis\-mos.}
\begin{defi}
Una aplicación entre dos espacios topológicos $f:(X,T_X)\fl (Y,T_Y)$ se dice {\bf Continua en un punto $x\in X$}
si para cualquier abierto $G\in T_Y$ entorno de $f(x)$, se tiene que $f^{-1}(G)$ es entorno de $x$ y se dice {\bf
Continua} si lo es en todo punto de $X$.
\end{defi}
\begin{teorema}
Sea $f:(X,T_X)\fl (Y,T_Y)$ una aplicación. Las condiciones siguientes son equivalentes:
\begin{enumerate}
\item $f$ es continua.
\item La anti--imagen por $f$ de cualquier abierto de $Y$ es abierto de $X$.
\item La anti--imagen por $f$ de cualquier cerrado de $Y$ es cerrado de $X$.
\end{enumerate}
\end{teorema}
\begin{defi}
Una aplicación $f:(X,T_X)\fl (Y,T_Y)$ se dice {\bf Abierta} (respectivamente, {\bf Cerrada}) si la imagen por $f$
de cualquier abierto de $X$ (respectivamente, de cualquier cerrado) es un abierto de $Y$ (respectivamente, un
cerrado).
\end{defi}
\begin{defi} Una aplicación $f:(X,T_X)\fl (Y,T_Y)$ se dice que es un {\bf Homeomorfismo} si es biyectiva, continua
y su inversa es también continua.
\end{defi}
\begin{teorema}
Sea $f:(X,T_X)\fl (Y,T_Y)$ una aplicación biyectiva. Las condiciones siguientes son equivalentes:
\begin{enumerate}
\item $f$ es un homeomorfismo.
\item $f$ es continua y abierta.
\item $f$ es continua y cerrada.
\end{enumerate}
\end{teorema}
\begin{defi} Una propiedad se dice {\bf Propiedad Topológica} o {\bf Invariante Topológico}, si se conserva por
homeomorfismos, es decir, si de verificarla un espacio topológico la verifican todos los espacios topológicos
homeomorfos a él.
\end{defi}
\begin{defi}
Una aplicación $f(X,T_X)\fl (Y,T_Y)$ continua se dice que es un {\bf Homeomorfismo Local} si para todo punto $x\in
X$, existe un entorno abierto $U_x\in T_X$ de $x$ tal que $f(U_x)$ es abierto en $Y$ y $f|_{U_x}:U_x\fl f(U_x)$ es
un homeomorfismo.
\end{defi}
\begin{prop}
Todo homeomorfismo local es una aplicación abierta.
\end{prop}
\section{Construcción de Topologías mediante aplicaciones. Topologías Producto y Cociente.}
\begin{prop}
Sean $X$ un conjunto, $\{(Y_i,T_i)\}_{i\in I}$ una familia de espacios topológicos y $\{f_i:X\fl Y_i/i\in I\}$ una
familia de aplicaciones. Entonces, existe la menor topología $T$ sobre $X$ que hace continuas a todas las
aplicaciones $f_i$, llamada {\bf Topología Inicial} de las $f_i$ y que es la generada, siguiendo el proceso
descrito en el Teorema \ref{teorema}, por la familia $\esp{A}$ de subconjuntos de $X$:
$$\esp{A}=\bigcup_{i\in I}\{f_i^{-1}(G)/G\in T_i\}.$$
\end{prop}
\begin{coro}
Sean $X$ un conjunto, $(Y,T_Y)$ un espacio topológico y $f:X\fl (Y,T_Y)$ una aplicación. Entonces,
$T=\{f^{-1}(G)/G\in T_Y\}$ es la menor topología sobre $X$ que hace continua a la aplicación $f$ y se llama {\bf
Topología Inicial} de $f$. Además, si $f$ es biyectiva, la topología inicial la convierte en un homeomorfismo.
\end{coro}
\begin{defi}
Dada una familia de espacio topológicos $\{(X_i,T_i)\}_{i\in I}$ y dado su producto cartesiano $X=\prod_{i\in
I}X_i$, a la topología inicial sobre $X$ de las proyecciones $\pi_i:X\fl X_i$ se le llama la {\bf Topología
Producto} de las $T_i$ y se denota por $T_{\prod}$ y al espacio topológico $(X,T_{\prod})$ se le llama {\bf
Espacio Producto} .
\end{defi}
\begin{nota}
{\rm En el caso de un número finito de factores, una base de la topología producto está formada por los productos
de abiertos de cada uno de los factores. Además, también en este caso, la topología relativa de una producto es la
topología producto de las relativas de cada uno de los factores.}
\end{nota}
\begin{prop}
Sea $\{(X_i,T_i)\}_{i=1,\dots ,m}$ una familia finita de espacios to\-po\-ló\-gi\-cos y sea $X=\prod_{i=1}^mX_i$,
dotado de la topología producto. Dado otro espacio topológico $(Y,T_Y)$, una aplicación $f:(Y,T_Y)\fl
(X,T_{\prod})$ es continua si y sólo si $\pi_i\circ f:(Y,T_Y)\fl (X_i,T_i)$ es continua, para todo $i=1,\dots ,m$.
\end{prop}
\begin{prop}
Sean $X$ un conjunto, $\{(Y_i,T_i)\}_{i\in I}$ una familia de espacios topológicos y $\{f_i:Y_i\fl X/i\in I\}$ una
familia de aplicaciones. Entonces, existe la mayor topología $T$ sobre $X$ que hace continuas a todas las
aplicaciones $f_i$, llamada {\bf Topología Final} de las $f_i$.
\end{prop}
\begin{coro}
Sean $X$ un conjunto, $(Y,T_Y)$ un espacio topológico y $f:(Y,T_Y)\fl X$ una aplicación. Entonces, $T=\{G\subseteq
X/ f^{-1}(G)\in T_Y\}$ es la mayor topología sobre $X$ que hace continua a la aplicación $f$ y se llama {\bf
Topología Final} de $f$. Además, si $f$ es biyectiva, la topología final la convierte en un homeomorfismo.
\end{coro}
\begin{defi}
Sean $(X,T)$ un espacio topológico y $\esp{R}$ una relación de equivalencia sobre $X$. Si $\pi:X\fl X/\esp{R}$ es
la proyección canónica, entonces la topología final de $\pi$ sobre el espacio cociente $X/\esp{R}$, denotada por
$T_{\esp{R}}$ se llama {\bf Topología Cociente} y al espacio topológico $(X/\esp{R},T_{\esp{R}})$ se le llama {\bf
Espacio Cociente} sobre $X$ por la relación $\esp{R}$.
\end{defi}
\begin{prop}
Sean (X,T) un espacio topológico, $(X/\esp{R},T_{\esp{R}})$ un espacio cociente sobre $X$, $(Y,T_Y)$ otro espacio
topológico y $f:(X/\esp{R},T_{\esp{R}})\fl (Y,T_Y)$ una aplicación. Entonces, $f$ es continua si y sólo si
$f\circ\pi:(X,T)\fl (Y,T_Y)$ es continua.
\end{prop}
\section{Axiomas de Separación.}
\begin{defi}
Sea $(X,T)$ un espacio topológico. Se dice que es:
\begin{enumerate}
\item $T_1$ si todo par de puntos distintos de $X$ se puede separar por conjuntos abiertos, es decir, si para cada
uno de los puntos existe un abierto que lo contiene y no contiene al otro punto.
\item $T_2$ o de {\bf Haussdorf} si todo par de puntos distintos de $X$ se puede separar por abiertos dijuntos.
\item {\bf Regular} si todo conjunto cerrado y todo punto que no pertenezca a él se pueden separar por abiertos
disjuntos.
\item $T_3$ si es regular y $T_1$.
\item {\bf Normal} si todo par de cerrados disjuntos se puede separar por abiertos disjuntos.
\item $T_4$ si es normal y $T_1$.
\end{enumerate}
\end{defi}
\begin{prop}
\begin{enumerate}
\item El axioma $T_i$ implica el axioma $T_j$, para todo $i>j$.
\item Los axiomas de separación son propiedades topológicas.
\item Los axiomas $T_1$ y $T_2$ son propiedades hereditarias para todo subespacio.
\item Un espacio topológico es $T_1$ si y sólo si todo conjunto unitario es cerrado.
\item Un espacio topológico es regular si y sólo si para todo abierto $G$ y para todo punto $x\in G$,
existe otro abierto $U$ tal que $x\in U\subseteq\overline{U}\subseteq G$.
\item Un espacio topológico es normal si y sólo si para todo abierto $G$ y para todo cerrado $F$ que lo contenga,
existe otro abierto $U$ tal que $G\subseteq U\subseteq\overline{U}\subseteq F$.
\end{enumerate}
\end{prop}
\begin{teorema}
{\bf (Lema de Uryshon).} Un espacio topológico $(X,T)$ es normal si y sólo si para cada par de cerrados disjuntos
$F_1$ y $F_2$ de $X$, existe una aplicación continua $f:(X,T)\fl [0,1]$ (con la topología euclídea) tal que
$f(F_1)=0$ y $f(F_2)=1$.
\end{teorema}
\section{Axiomas de Numerabilidad.}
\begin{defi}
Sea $(X,T)$ un espacio topológico. Un subconjunto $D$ de $X$ se dice {\bf Denso} si $\overline{D}=X$.
\end{defi}
\begin{defi}
Se dice que un espacio topológico $(X,T)$ es:
\begin{enumerate}
\item {\bf Primero Numerable} ($1^{\underline{o}}N$) si todo punto de $X$ tiene una base de entornos numerable.
\item {\bf Segundo Numerable} ($2^{\underline{o}}N$) si $T$ tiene una base numerable.
\item De {\bf Lindeloff} si todo recubrimiento de $X$ por abiertos admite un subrecubrimiento numerable.
\item {\bf Separable} si existe un subconjunto de $X$ denso y numerable.
\end{enumerate}
\end{defi}
\begin{prop}
\begin{enumerate}
\item Los axiomas de numerabilidad son propiedades to\-po\-ló\-gi\-cas.
\item El axioma $2^{\underline{o}}N$ implica a todos los demás.
\item Los axiomas $1^{\underline{o}}N$ y $2^{\underline{o}}N$ son propiedades hereditarias para cualquier subespacio.
\item Todo espacio topológico regular y de Lindeloff es normal.
\end{enumerate}
\end{prop}
\section{Compacidad.}
\begin{defi}
Un espacio topológico $(X,T)$ se dice {\bf Compacto} si de todo recubrimiento de $X$ por abiertos se puede extraer
un subrecubrimiento finito.
\end{defi}
\begin{prop}
\begin{enumerate}
\item La compacidad se coserva por aplicaciones continuas. En consecuencia, es una propiedad topológica.
\item La compacidad es una propiedad hereditaria para subespacios cerrados.
\item En un espacio topológico $T_2$, todo subespacio compacto es cerrado.
\item Toda aplicación continua y biyectiva de un espacio topológico compacto en un espacio topológico $T_2$ es un
homeomorfismo.
\item Todo espacio topológico $T_2$ y compacto es $T_4$.
\end{enumerate}
\end{prop}
\begin{defi}
Un espacio topológico $(X,T)$ se dice {\bf Localmente Compacto} si todo punto de $X$ tiene una base de entornos
formada por conjuntos compactos.
\end{defi}
\begin{teorema}
Un espacio topológico $T_2$ es localmente compacto si y sólo si todo punto tiene un entorno compacto.
\end{teorema}
\begin{prop}
\begin{enumerate}
\item La compacidad local es una propiedad topológica.
\item La compacidad local es una propiedad hereditaria para subespacios abiertos y para subespacios cerrados.
\item Todo espacio $T_2$ y localmente compacto es $T_3$.
\end{enumerate}
\end{prop}
\begin{prop}
Sea $(X,T)$ un espacio topológico $T_2$ y localmente compacto. Entonces, para todo abierto $G$ y para todo punto
$p\in G$, existe otro abierto $H$ de clausura compacta y tal que $p\in H\subseteq\overline{H}\subseteq G$.
\end{prop}

\newpage

\section{Conexión.}
\begin{defi}
Un espacio topológico $(X,T)$ se dice {\bf Conexo} si no existen dos subconjuntos propios abiertos (o cerrados) de
$X$, disjuntos y que recubran a $X$.
\end{defi}
\begin{teorema}
Sea $(X,T)$ un espacio topológico. Las condiciones siguientes son equivalentes:
\begin{enumerate}
\item $(X,T)$ es conexo.
\item Los únicos subconjuntos de $X$ que son a la vez abiertos y cerrados son el conjunto vacío $\emptyset$ y el
propio $X$.
\item Toda aplicación continua de $(X,T)$ en $\R$ que tome dos valores, toma todos los valores intermedios.
\end{enumerate}
\end{teorema}
\begin{prop}
La conexión no es, en general, una propiedad hereditaria. Además, se conserva por aplicaciones continuas, con lo
que, en consecuencia, es una propiedad topológica.
\end{prop}
\begin{defi}
En un espacio topológico $(X,T)$, se llama {\bf Componente Conexa} del punto $x\in X$ al mayor subconjunto conexo
de $X$ que contenga a $x$.
\end{defi}
\begin{prop}
Las componentes conexas de un espacio topológico $(X,T)$ son subconjuntos cerrados y forman una partición de $X$.
Además, el número de componentes conexas es una propiedad topológica.
\end{prop}
\begin{defi}
Sea $(X,T)$ un espacio topológico y $x\in X$. Se llama {\bf Orden de Conexión} de $x$ en $X$ al número de
componentes conexas de $C_x-\{x\}$, donde $C_x$ denota la componente conexa de $x$ en $X$.
\end{defi}
\begin{prop}
Sea $f:(X,T_X)\fl (Y,T_Y)$ un homeomorfismo. Entonces, para todo $x\in X$, $x$ y $f(x)$ tiene  el mismo orden de
conexión. En consecuencia, el orden de conexión de los puntos es una propieda topológica.
\end{prop}
\begin{defi}
Un espacio topológico $(X,T)$ se dice {\bf Localmente Conexo} si todo punto de $X$ tiene una base de entornos
formada por conjuntos conexos.
\end{defi}
\begin{prop}
La conexión local es una propiedad topológica y hereditaria para abiertos. Además, las componentes conexas de un
espacio topológico localmente conexo son también conjuntos abiertos.
\end{prop}
\begin{defi}
Un {\bf Arco} en un espacio topológico $(X,T)$ es la imagen por un homeomorfismo $f$ de $[0,1]$ (con la topología
euclídea) en $(X,T)$. Dados dos puntos $x,y\in X$, se llama {\bf Arco desde $x$ hasta $y$} a un arco en $X$ tal
que $f(0)=x$ y $f(1)=y$.
\end{defi}
\begin{nota}
{\rm Obsérvese que los arcos son, por definición, subconjuntos conexos de $X$. Por otra parte si la aplicación $f$
es sólo continua (y no un homeomorfismo), surge la noción de {\bf Camino} en $X$ y todas las definiciones y
resultados tienen su versión paralela.}
\end{nota}
\begin{defi}
Un espacio topológico $(X,T)$ se dice {\bf Conexo por Arcos} o {\bf Arcoconexo}, si dados dos puntos cualesquiera
de $X$, existe un arco en $X$ desde uno de los puntos hasta el otro.
\end{defi}
\begin{prop}
\begin{enumerate}
\item La conexión por arcos es una propiedad topológica.
\item Todo espacio topológico conexo por arcos es conexo.
\end{enumerate}
\end{prop}
\end{document}