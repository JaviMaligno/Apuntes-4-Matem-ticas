\documentclass[cursovd_portada.tex]{subfiles}

\begin{document}

\chapter*{Ap\'endice B.\\Teorema de la Funci\'on Impl\'{\i}cita.}
%\newtheorem{lemaap}{Lema}
%\newtheorem{propoap}{Proposici\'{o}n}
\addcontentsline{toc}{chapter}{Ap\'{e}ndice B. Teorema de la Funci\'on Impl\'{\i}cita.}
\section*{B.1 Teorema de la Funci\'{o}n Impl\'{\i}cita.}
\addcontentsline{toc}{section}{B.1\hspace{0.5 em} Teorema de la Funci\'{o}n Impl\'{\i}cita.}
\begin{teoap}
Sea $f\in\mathcal{F}(M,N)$, $q\in N$ y $P=f^{-1}(\{q\})\neq\emptyset$. Si $f_{*p}$ es sobre, para todo $p\in P$,
entonces $P$ tiene estructura \'{u}nica de variedad diferenciable de dimensi\'{o}n $m-n$, tal que $(P,i)$ es subvariedad
regular de $M$, con $i:P\hookrightarrow M$ la inclusi\'{o}n.
\end{teoap}
{\sc Demostraci\'{o}n:} Sea $(V;y_1,\dots ,y_n)$ una carta de $N$ entorno de $q$. Si $p\in P$, entonces $p\in M$ y
$f(p)=q$. Como $f_{*p}$ es sobre, por el Corolario 2.1.5 del Teorema de la Funci\'{o}n Inversa, existe
$$(U,\vp=(y_1\circ f,\dots ,y_n\circ f,x_{n+1},\dots ,x_m))$$
carta local en $M$ y se puede tomar $f(U)\subseteq V$.
\par
Consid\'{e}rese en $P$ la topolog\'{\i}a relativa de $M$,
$$T_P=\{G\cap P/G\in T_M\}$$
y sea el abierto de $P$, $U\cap P$. Se define:
$$\bar{\vp}:U\cap P\fl\R^{m-n}:z\mapsto\bar{\vp}(z)=(x_{n+1}(z),\dots ,
x_m(z)).$$ \hs Se tiene que $\bar{\vp}=\pi\circ\vp|_{U\cap P}$, donde $\pi:\R^m\fl\R^{m-n}$ es la proyecci\'{o}n dada
por $\pi(u_1,\dots ,u_m)=(u_{n+1},\dots ,u_m)$. En estas condiciones, es f\'{a}cil probar que:
$$\bar{\vp}(U\cap P)=\pi(\vp(U)\cap\{y_1(q),\dots ,y_n(q)\}\times\R^{m-n}).$$
\hs Como
$$\pi|_{\{y_1(q),\dots ,y_n(q)\}\times\R^{m-n}}:\{y_1(q),\dots ,y_n(q)\}\times
\R^{m-n}\fl\R^{m-n}$$ es un homeomorfismo, se deduce que $\bar{\vp}(U\cap P)$ es un abierto de $\R^{m-n}$. Adem\'{a}s,
se prueba sin dificultad que
$$\bar{\vp}:U\cap P\fl\bar{\vp}(U\cap P)$$
es biyectiva, continua (pues $\bar{\vp}=\pi\circ\vp$) y de inversa continua (puesto que
$\bar{\vp}^{-1}=\vp^{-1}\circ j|_{\bar{\vp}(U\cap P)}$, donde:
$$j:\R^{m-n}\fl\R^m:(u_{n+1},\dots ,u_m)\mapsto(y_1(q),\dots ,y_n(q),u_{n+1},
\dots ,u_m)).$$ \hs Por tanto, $(U\cap P,\bar{\vp})$ es una carta local en P entorno de $p$ de dimensi\'{o}n $m-n$.
Haciendo variar $p$ en $P$, se tiene la familia de cartas $\{(U_p\cap P,\bar{\vp}_p)\}_{p\in P}$, cuyos dominios
recubren a $P$. Con esta familia, $P$ tiene estructura de variedad diferenciable de dimensi\'{o}n $m-n$. Para ello,
basta probar que, dos a dos, estas cartas est\'{a}n relacionadas, pero eso es f\'{a}cil, ya que si $P\cap U_p\cap
U_{p'}\neq \emptyset$, entonces
$$\bar{\vp}_{p'}\circ{\bar{\vp}_p}^{-1}=\pi\circ\vp_{p'}\circ\vp_p\circ j|_
{\bar{\vp}_p(U_p\cap U_{p'}\cap P)}$$ es diferenciable, por ser composici\'{o}n de diferenciables.
\par
Ahora, se prueba que la inclusi\'{o}n $i:P\hookrightarrow M$ es diferenciable. Adem\'{a}s, su diferencial, $i_{*p}$, es
inyectiva, pues si $\bar{\vp}=(z_{n+1}, \dots ,z_m)$, entonces $z_k=x_k\circ i$ ($k=n+1,\dots ,m$) y
$$i_{*p}\left(\frac{\partial}{\partial z_k}\right)_p=\sum_{j=1}^m\left(\frac{\partial
(x_j\circ i)}{\partial z_k}\right)_p\left(\frac{\partial}{\partial x_j}\right)_p=\left(\frac{
\partial}{\partial x_k}\right)_p,$$
con lo que $i_{*p}$ transforma bases en vectores linealmente independientes. Por tanto, $(P,i)$ es una subvariedad
regular de $M$ de dimensi\'{o}n $m-n$.
\par
Finalmente, se tiene la unicidad de la estructura, seg\'{u}n el teorema que dice que si un subconjunto de una variedad
diferenciable tiene estructura de variedad con la topolog\'{\i}a relativa tal que con la inclusi\'{o}n es una subvariedad,
entonces la estructura es \'{u}nica. \hfill $\Box$

\

La demostraci\'on del Teorema del Rango es similar. Puede
consultarse en: J. M. Lee. {\it Manifolds and Differential
Geometry}. Graduate Studies in Mathematics Vol. 107. American
Mathematical Society, 2009.

\section*{A.2 Teorema de Frobenius.}
\addcontentsline{toc}{section}{A.2\hspace{0.5 em} Teorema de Frobenius.}
\begin{lemaap} Sean $M$ una variedad diferenciable, $p\in M$ y $X\in\mathcal{X}(M)$ verificando $X_p\neq 0$. Entonces,
existe un sistema de coordenadas $(U;x_1,\dots ,x_m)$, centrado en $p$, tal que:
$$X|_U=\frac{\partial}{\partial x_1}.$$
\end{lemaap}
{\sc Demostraci\'{o}n:} Primero, se probar\'{a} que existe un s.l.c. $(V,y_1,\dots , y_m)$, centrado en $p$ tal que:
$$X_p=\left(\frac{\partial}{\partial y_1}\right)_p.$$
\hs En efecto, dado cualquier s.l.c. $(z_1,\dots ,z_m)$ centrado en $p$, se tiene que
$$X_p=\sum_{i=1}^ma_i(\frac{\partial}{\partial z_i})_p,$$
donde se puede suponer que $a_1\neq 0$ (?`por qu\'{e}?). Sea, ahora:
$$\left .
\begin{array}{lll}
y_1 & = & \displaystyle{\frac{1}{a_1}}z_1;\\
y_i & = & z_i-\displaystyle{\frac{a_i}{a_1}}z_1\mbox{ }(2\leq i\leq m).
\end{array}
\right \}
$$
\hs Es claro que $\{y_1,\dots ,y_m\}$ son funciones independientes en $p$ y, por tanto, $\tau=(y_1,\dots ,y_m)$
constituye, en virtud del Corolario 2.1.2 del Teorema de la Funci\'{o}n Inversa, un s.l.c. centrado en $p$, con
dominio que se denotar\'{a} por $V$, cum\-plien\-do:
$$\left(\frac{\partial}{\partial y_1}\right)_p=\sum_{i=1}^m(\frac{\partial z_i}{\partial
y_1})_p(\frac{\partial}{\partial z_i})_p=X_p.$$ \hs A continuaci\'{o}n, consid\'{e}rese el grupo uniparam\'{e}trico de
transformaciones locales engendrado por $X$, $(U_p,\vp_t,\varepsilon)$, en un entorno $U_p$ de $p$, $\vert t\vert
<\varepsilon$. Denotando por $\pi$ a la proyecci\'{o}n
$$\pi:\R^m\fl\R^{m-1}:(u_1,\dots ,u_m)\mapsto (u_2,\dots ,u_m)$$
y tomando $W=(-\varepsilon,\varepsilon)\times\pi(\tau(U_p\cap V))$, que es un entorno de $0\in\R^m$, se puede definir la
aplicaci\'{o}n
$$\sigma:W\fl M:(t,b_2,\dots ,b_m)\mapsto\vp_t(\tau^{-1}(0,b_2,\dots ,b_m)),$$
que es diferenciable, pues $\sigma=\vp_t\circ\tau^{-1}\circ({\bf 0}\times\pi)$ (siendo ${\bf 0}$ la aplicaci\'{o}n
constante cero). Adem\'{a}s, $\sigma_{*0}$ es un isomorfismo. En efecto, si para cada $i=1,\dots ,m$ se toma la curva
$$t\mapsto (0,\dots ,t,\dots ,0)$$
($t$ en el lugar $i$--\'{e}simo), que pasa por el origen para $t=0$, es claro que
$$\left(\frac{\partial}{\partial u_i}\right)_0\in T_0(\R^m)$$
es un vector tangente a dicha curva en el origen y, por tanto,
$$\sigma_{*0}\left(\frac{\partial}{\partial u_i}\right)_0$$
es tangente a la curva
$$t\mapsto\sigma(0,\dots ,t,\dots ,0)$$
($t$ en el lugar $i$--\'{e}simo) en $t=0$. Ahora bien, para $i=1$,
$$\sigma(t,0,\dots ,0)=\vp_t(\tau^{-1}(0,\dots ,0))=\vp_t(p)=\vp_p(t),$$
luego:
$$\sigma_{*0}(\frac{\partial}{\partial u_1})_0=\vp'_p(0)=X_p=\left(\frac{\partial}
{\partial y_1}\right)_p.$$ \hs Para $i>1$, se tiene que
$$\sigma(0,\dots ,t,\dots ,0)=\vp_0(\tau^{-1}(0,\dots ,t,\dots ,0))=\tau^{-1}
(0,\dots ,t,\dots ,0)$$ y, en consecuencia:
$$\sigma_{*0}\left(\frac{\partial}{\partial u_i}\right)_0=\sum_{j=1}^m\left(\frac{\partial(y_j
\circ\sigma)(0,\dots ,u_i,\dots ,0)}{\partial u_i})_0(\frac{\partial}{\partial y_j}\right)_p=$$
$$=\sum_{j=1}^m\left(\frac{\partial(u_j\circ\tau\circ\tau^{-1})(0,\dots ,u_i,\dots ,
0)}{\partial u_i}\right)_0\left(\frac{\partial}{\partial y_j}\right)_p=$$
$$=\sum_{j=1}^m\delta_{ij}\left(\frac{\partial}{\partial y_j}\right)_p=(\frac{\partial}
{\partial y_i})_p.$$ \hs En resumen, $\sigma_{*0}$ transforma base en base, con lo que es un isomorfismo, lo que
quiere decir que $\sigma$ es no singular en el origen. En virtud del Corolario 2.1.2 del Teorema de la Funci\'{o}n
Inversa, se deduce que $\vp=\sigma^{-1}$ es una aplicaci\'{o}n coordenada en un cierto entorno $U$ de $p$. Sea, pues,
$(U,\vp)=(U;x_1,\dots ,x_m)$ dicho entorno coordenado. Entonces, consid\'{e}rese la curva imagen mediante $\sigma$ de
la curva
$$t\mapsto(t,b_2,\dots ,b_m)$$
de $\R^m$ (cuyo vector tangente es
$$\left(\frac{\partial}{\partial u_1}\right)_{(t,b_2,\dots ,b_m)})$$
y que tiene como vector tangente a:
$$\sigma_{*(t,b_2,\dots ,b_m)}\left(\frac{\partial}{\partial u_1}\right)_{(t,b_2,\dots ,
b_m)}.$$ \hs Pero, por otra parte, $\sigma(t,b_2,\dots ,b_m)=\vp_t(\tau^{-1}(0,b_2,\dots , b_m))$, de donde,
puesto que $\vp_t$ es el grupo uniparam\'{e}trico de transformaciones locales engendrado por $X$, se sigue que:
$$\sigma_{*(t,b_2,\dots ,b_m)}\left(\frac{\partial}{\partial u_1}\right)_{(t,b_2,\dots ,
b_m)}=X_{\sigma(t,b_2,\dots ,b_m)}.$$ \hs Ahora, como $\sigma(t,b_2,\dots ,b_m)$ recorre todos los puntos de $U$ y
como, por otra parte. $\forall u\in\vp(U)$ se tiene que
$$\sigma_{*u}\left(\frac{\partial}{\partial u_1}\right)_u=\sum_{j=1}^m\left(\frac{\partial(x_j
\circ\sigma)}{\partial u_1}\right)_u\left(\frac{\partial}{\partial x_j}\right)_{\sigma(u)}=$$
$$=\sum_{j=1}^m\left(\frac{\partial(u_j\circ\vp\circ\sigma)}{\partial u_1}\right)_u\left(\frac{
\partial}{\partial x_j}\right)_{\sigma(u)}=\sum_{j=1}^m\delta_{j1}\left(\frac{\partial}
{\partial x_j}\right)_{\sigma(u)}=\left(\frac{\partial}{\partial x_1}\right)_{\sigma(u)},$$ se deduce que
$$X_{\sigma(t,b_2,\dots ,b_m)}=\left(\frac{\partial}{\partial x_1}\right)_{\sigma(t,
b_2,\dots ,b_m)}$$ y, por consiguiente:
$$X|_U=\frac{\partial}{\partial x_1}.\eqno\Box$$
\par \bigskip
\begin{teoap}
({\bf Teorema de Frobenius).} Sea $\mathcal{D}$ una distribuci\'{o}n $n$--di\-men\-sio\-nal, diferenciable e
involutiva en una variedad diferenciable $M$ y sea $p\in M$. Entonces, existe un sistema local de coordenadas
centrado en $p$, $(U;x_1,\dots x_m)$, tal que la placa
$$x_i=0,\mbox{ }i=n+1,\dots ,m,$$
es una subvariedad integral de $\mathcal{D}$ pasando por $p$.
\end{teoap}
{\sc Demostraci\'{o}n:} Se probar\'{a}, por inducci\'{o}n en $n$, que para cada punto $p\in M$ existe un s.l.c. $(x_1,\dots,
x_m)$, centrado en $p$, tal que
$$\left\{\frac{\partial}{\partial x_1},\dots ,\frac{\partial}{\partial x_n}\right\}$$
forman una base local de campos para ${\cal D}$ y, por tanto,
$$x_i=0,\mbox{ }i=n+1,\dots ,m,$$
es una subvariedad integral de ${\cal D}$ (?`por qu\'{e} se tiene el resultado con \'{e}sto?).
\par
Para $n=1$, el Lema 1 asegura que se verifica el teorema. Si $n\geq 2$ y el teorema es cierto para $n-1$, sea
${\cal D}=<X_1,\dots ,X_n>$ en un entorno de $p$. Entonces, el Lema 1 implica que existe un s.l.c., centrado en
$p$, con coordenadas $(y_1,\dots ,y_m)$, tal que:
$$X_n=\frac{\partial}{\partial y_n}.$$
\hs Ahora, sean:
$$X'_i=X_i-(X_iy_n)X_n,\mbox{ }1\leq i\leq n-1.$$
\hs Entonces, $X'_iy_n=0$, $i=1,\dots ,n-1$. Adem\'{a}s, es claro que $X_ny_n=1$. Como ${\cal D}=<X'_1,\dots
,X'_{n-1},X_n>$ (?`por qu\'{e}?) y es involutiva, se tiene que
$$[X'_i,X'_j]=\sum_{k=1}^{n-1}a_{ij}^kX'_k+a_{ij}X_n,\mbox{ }1\leq i,j\leq
n-1,$$ donde es f\'{a}cil comprobar que $a_{ij}=0$ (?`por qu\'{e}?), $\forall i,j=1,\dots , n-1$. Por tanto, la
distribuci\'{o}n ${\cal D}'=<X'_1,\dots ,X'_{n-1}>$ tambi\'{e}n es involutiva. Por la hip\'{o}tesis de inducci\'{o}n, se sigue que
existe un s.l.c. $(z_1, \dots ,z_m)$, centrado en $p$, tal que ${\cal D}'=<Z_1,\dots ,Z_{n-1}>$, con:
$$Z_i=\frac{\partial}{\partial z_i},\mbox{ }1\leq i\leq n-1.$$
\hs Es evidente que $\{Z_1\dots ,Z_{n-1}\}$ difiere de $\{X'_1,\dots ,X'_{n-1} \}$ en una transformaci\'{o}n lineal no
singular (pues representa un cambio de base para ${\cal D}'$ en un entorno de $p$). Esto implica que $Z_iy_n=0$,
$i= 1,\dots ,n-1$ (?`por qu\'{e}?). En consecuencia, ${\cal D}=<Z_1,\dots ,Z_{n-1}, X_n>$ (?`por qu\'{e}?). En estas
condiciones, sean:
$$[Z_i,X_n]=\sum_{j=1}^{n-1}b_i^jZ_j+b_iX_n,\mbox{ }1\leq i\leq n-1.$$
\hs De nuevo, es f\'{a}cil probar que $b_i=0$ (?`por qu\'{e}), $\forall i=1,\dots ,n-1$. Por tanto:
$$[Z_i,X_n]=\sum_{j=1}^{n-1}b_i^jZ_j+b_iX_n.\eqno(1)$$
Ahora bien, como en funci\'{o}n de las coordendas $(z_1,\dots ,z_m)$ se tiene que
$$X_n=\sum_{a=1}^m\xi_a\frac{\partial}{\partial z_a},$$
entonces (!`comprobar!):
$$[Z_i,X_n]=\sum_{a=1}^m\frac{\partial\xi_a}{\partial z_i}\frac{\partial}
{\partial z_a}.\eqno(2)$$ \hs Comparando (1) y (2), se sigue que
$$\frac{\partial\xi_a}{\partial z_i}=0,\mbox{ }1\leq i\leq n-1,\mbox{ }n\leq a
\leq m,$$ (?`por qu\'{e}?) lo que indica que $\xi_a$, $a=n,\dots ,m$, son funciones de $z_n, \dots ,z_m$ s\'{o}lamente. Por
tanto, ${\cal D}=<Z_1,\dots ,Z_{n-1},X'_n>$, donde
$$X'_n=\sum_{a=n}^m\xi_a\frac{\partial}{\partial z_a}$$
(!`justificar ambas afirmaciones!). Este campo depende s\'{o}lo de las coordenadas $z_n,\dots ,z_m$, luego efectuando
un cambio de coordenadas
$$(z_n,\dots ,z_m)\mapsto (w_n,\dots ,w_m)$$
que no afecte a $z_1,\dots ,z_{n-1}$ (?`c\'{o}mo?), se consigue que
$$X'_n=\frac{\partial}{\partial w_n},$$
(de nuevo, ?`c\'{o}mo?) lo que completa la demostraci\'{o}n.\hfill $\Box$

\newpage

\section*{A.3 Lema de Poincar\'{e}.}
\addcontentsline{toc}{section}{A.3\hspace{0.5 em} Lema de Poincar\'{e}.}
\begin{teoap}
{\bf (Lema de Poincar\'{e}).} Si $U$ es una bola abierta de $\R^m$ y $\omega\in \Lambda_r(U)$, $r\geq 1$, es cerrada,
entonces existe $\a\in\Lambda_{r-1}(U)$ tal que $\omega=\de \a$.
\end{teoap}
{\sc Demostraci\'{o}n:} Sup\'{o}ngase que $U$ est\'{a} centrada en el origen. Entonces, se va a definir un operador
$$K:\Lambda_r(U)\fl\Lambda_{r-1}(U)$$
tal que $\de K+K\de =Id$. De aqu\'{\i} que, dada $\omega$, $\de K\omega+ K\de \omega=\de K\omega=\omega$, con lo que
bastar\'{\i}a tomar $\a=K\omega$.
\par
Ahora, si $\underline{x}=(x_1,\dots ,x_m)\in U$, cualquier $r$--forma $\theta$ se puede escribir como:
$$\theta=\sum_{1\leq i_1 <\cdots <i_r\leq m}a_{i_1\dots i_r}(\underline{x})
\de x_{i_1}\wedge\cdots\wedge\de x_{i_r}.$$ \hs Entonces, se define:
$$K\theta=$$
$$=\sum_{1\leq i_1 <\cdots <i_r\leq m}\left( \int_0^1 a_{i_1\dots i_r}(t\underline{x})t^{r-1}dt\right)
\sum_{j=1}^r(-1)^ {j-1}x_{i_j}\de x_{i_1}\wedge\cdots\wedge\widehat{\de x_{i_j}}\wedge \cdots\wedge\de x_{i_r}.$$
\par\medskip
Es f\'{a}cil verificar que $K$ es $\R$--lineal. Como d tambi\'{e}n lo es, basta probar que $\de K\theta+K\de
\theta=\theta$ para $\theta=a(\underline{x}) \de x_1\wedge\cdots\wedge\de x_r$. En estas condiciones:
$$K\theta=\left( \int_0^1 a(t\underline{x})t^{r-1}dt\right) \sum_{j=1}^r(-1)^
{j-1}x_j\de x_1\wedge\cdots\wedge\widehat{\de x_j}\wedge\cdots\wedge \de x_r.$$ \hs Por tanto:
$$\de K\theta=\sum_{k=1}^m\dep{x_k}\left( \int_0^1 a(t\underline{x})t^{r-1}
dt\right)\de x_k\wedge\sum_{j=1}^r(-1)^{j-1}x_j\de x_1\wedge\cdots\wedge \widehat{\de x_j}\wedge\cdots\wedge\de
x_r+$$
$$+\left( \int_0^1 a(t\underline{x})t^{r-1}dt\right) \de (\sum_{j=1}^r(-1)^
{j-1}x_j\de x_1\wedge\cdots\wedge\widehat{\de x_j}\wedge\cdots \wedge\de x_r)=$$
$$=\sum_{k=1}^m\left( \int_0^1 \ddep{a(t\underline{x})}{x_k}t^{r-1}
dt\right)\de x_k\wedge\sum_{j=1}^r(-1)^{j-1}x_j\de x_1\wedge\cdots\wedge \widehat{\de x_j}\wedge\cdots\wedge\de
x_r+$$
$$+\left( \int_0^1 a(t\underline{x})t^{r-1}dt\right) r\de x_1\wedge\cdots
\wedge\de x_r=$$
$$=\sum_{k=1}^m\left( \int_0^1 \left[ \sum_{j=1}^r\ddep{a}{x_j}(t\underline{x})
x_jt^r+ra(t\underline{x})t^{r-1}\right] dt\right) \de x_1\wedge\cdots\wedge \de x_r+$$
$$+(-1)^{r-1}\sum_{k=r+1}^m\left( \int_0^1 \ddep{a}{x_k}
(t\underline{x})t^rdt\right) \sum_{j=1}^r(-1)^{j-1}x_j\de x_1\wedge\cdots \wedge\widehat{\de
x_j}\wedge\cdots\wedge\de x_r\wedge\de x_k. \eqno(1)$$ \hs Por otra parte,
$$\de \theta=(-1)^r\sum_{k=r+1}^m\ddep{a}{x_k}(\underline{x})\de x_1
\wedge\cdots\wedge\de x_r\wedge\de x_k,$$ con lo que se obtiene:
$$K\de \theta=(-1)^r\sum_{k=r+1}^m\left( \int_0^1 \ddep{a}{x_k}(t
\underline{x})t^rdt\right) \sum_{j=1}^r(-1)^{j-1}x_j\de x_1\wedge\cdots \wedge\widehat{\de
x_j}\wedge\cdots\wedge\de x_r\wedge\de x_k+$$
$$+(-1)^r\sum_{k=r+1}^m\left( \int_0^1 \ddep{a}{x_k}(t\underline{x})t^rdt
\right) (-1)^{r+1-1}x_k\de x_1\wedge\cdots\wedge\de x_r=$$
$$=(-1)^r\sum_{k=r+1}^m\left( \int_0^1 \ddep{a}{x_k}(t\underline{x})t^rdt
\right) \sum_{j=1}^r(-1)^{j-1}x_j\de x_1\wedge\cdots\wedge\widehat{\de x_j}\wedge\cdots\wedge\de x_r\wedge\de
x_k+$$
$$+\sum_{k=r+1}^m\left( \int_0^1 \ddep{a}{x_k}(t\underline{x})t^rdt\right) x_k
\de x_1\wedge\cdots\wedge\de x_r.\eqno(2)$$ \hs En resumen, sumando (1) y (2):
$$\de K\theta+K\de \theta=\left(\int_0^1\left[ \sum_{k=1}^mx_k\ddep{a}
{x_k}(t\underline{x})t^r+ra(t\underline{x})t^{r-1}\right] dt\right) \de x_1 \wedge\cdots\wedge\de x_r=$$
$$=\left( \int_0^1\frac{d}{dt}(a(t\underline{x})t^r)dt\right) \de x_1\wedge\cdots\wedge
\de x_r=\theta.$$ \hfill$\Box$.

\end{document}