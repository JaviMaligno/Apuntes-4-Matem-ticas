	\documentclass[twoside]{article}
\usepackage{../../estilo-ejercicios}
\newcounter{ejercicio}
%--------------------------------------------------------
\begin{document}

\title{Ejercicios de Variedades Diferenciables - Tema 4}
\author{Javi, Rafa, Diego}
\maketitle

\begin{ejercicio}{1}
Probar que la diferencial de una aplicación $f : \R^m \to \R^n$, considerada
como aplicación entre variedades diferenciables, coincide con la diferencial
habitual de $\R^n$.
\end{ejercicio}
\begin{solucion}
Tenemos que $T_p(\R^n)\cong \R^n$, luego podemos pensar que la diferencial actúa sobre vectores de $\R^n$. Así, dados $v\in\R^m$, y $\alpha(t)=a+tv, a\in\R^m$. 
$$(f_*v)=(f\circ \alpha)'(p)=(f(a+tv))'(p)=\nabla f(p)\cdot v=df(p)(v)$$
\end{solucion}

\newpage

\begin{ejercicio}{2}
Probar que si $f \in \calF(M,N)$ es un difeomorfismo, entonces su diferencial en
cada punto es un isomorfismo.
\end{ejercicio}
\begin{solucion}
Sea $f^{-1}$ la inversa de $f$. Entonces $f^{-1}_*=(f_*)^{-1}$. Probarlo es inmediato usando que $(f\circ g)_*=f_*\circ g_*$ y que $(Id_M)_*=Id_{T_p(M)}$.
\end{solucion}


\newpage

\begin{ejercicio}{3}
Probar que el rango de la diferencial de una aplicación diferenciable entre
dos variedadades diferenciables no depende de las cartas tomadas.
\end{ejercicio}
\begin{solucion}
Dada una aplicación diferenciable $f:M\to N$, $p\in M$, el rango de su diferencial $f_*:T_p(M)\to T_p(N)$ es la dimensión de $f(M)$, que es el rango de la matriz jacobiana respecto de las cartas escogidas. Sean $(U,\varphi), (U',\varphi')$, $p\in U\cap U'$ cartas de $M$ y $(V,\psi),(V',\psi'),f(p)\in V\cap V'$ cartas de $N$. Sea $A$ la matriz con respecto a $(U,\varphi)$ y $(V,\psi)$, $B$ respecto de $(U',\varphi'),(V',\psi')$, y $C$ la matriz de cambio de base. Entonces $B=CA$. Como $C$ es regular por ser una matriz de cambio de base, $rg(B)=rg(A)$ como queríamos demostrar.
\end{solucion}

\newpage
\begin{ejercicio}{4}
Probar que las proyecciones de un producto de variedades diferenciables en
cada uno de los factores son sumersiones.
\end{ejercicio}
\begin{solucion}
Tenemos por los problemas propuestos que las proyecciones son aplicaciones diferenciables, así que basta ver que la diferencial es sobreyectiva. Sean $M,N$ variedades diferenciables y consideremos la proyección $\pi_1:M\times N\to M$. Sean $(U,\varphi=(x_1,\dots, x_m))$ y $(V,\psi=(y_1,\dots,y_n))$ cartas locales de $M$ y $N$ respectivamente. Fijamos $t=(p,q)\in U\times V$. Entonces consideramos $\pi_{1*}:T_t(M\times N)\to T_p(M)$. Es fácil ver que $T_t(M\times N)\cong T_p(M)\times T_q(N)$\footnote{\url{https://math.stackexchange.com/questions/413766/tangent-space-of-product-manifold}}. Así pues, sea $u\in T_q(M)$. Basta tomar un vector cualquiera de la forma $(u,v)\in T_p(M)\times T_q(N)$ para que tenga a $u$ por imagen. En efecto, sean $\alpha,\beta
$ curvas diferenciables sobre $M$ y $N$ respectivamente, que tengan cada una a $u$ y a $v$ como vector tangente, entonces
$$\pi_{1*}(u,v)=(\pi_1(\alpha,\beta))'=\alpha'=u$$
\end{solucion}

\newpage

\begin{ejercicio}{5}\label{1}
Sean $f\in\mathcal{F}(M,N)$, $p\in M$ y $g\in\mathcal{F}(f(p))$. Probar que $$f^*_p((dg)_{f(p)})=(d(g\circ f))_p.$$
\end{ejercicio}
\begin{solucion}
Sea $v\in T_p(M)$. 
$$f^*_p((dg)_{f(p)})v=(dg)_{f(p)}(f_{*p}v)=(f_{*p}v)g=v(g\circ f)=(d(g\circ f))_pv$$
\end{solucion}

\newpage

\begin{ejercicio}{6}
Probar que toda superficie regular es una subvariedad regular de $\R^3$.
\end{ejercicio}
\begin{solucion}
Diego o Alberto Daza
\end{solucion}

\newpage

\begin{ejercicio}{7}
Probar que cualquier subvariedad abierta de una variedad diferenciable es
subvariedad regular.
\end{ejercicio}
\begin{solucion}
Sea $M$ una variedad diferenciable y $G\subseteq M$ un abierto al que se dota de estructura de subvariedad abierta de $M$. Entonces, por la proposición 4.2.5 basta ver que $(G,i)$ con $i:G\hookrightarrow M$ es subvariedad de $M$. Para ello, tenemos que probar que $i$ es una inmersión inyectiva. Que es inyectiva es evidente, así que tenemos que probar que es inmersión, es decir, que dado $p\in G$, $i_*:T_p(G)\to T_p(M)$ es inyectiva. Sea $v\in T_p(G)$ tal que $i_*v=0$. Entonces $\forall f\in\calF(G,M)$, 
$$0=(i_*v)f =v(i\circ f)=v(f)$$
con lo que necesariamente $v=0$ como aplicación lineal, es decir, que debe ser el vector nulo, con lo que se tiene el resultado.
\end{solucion}

\newpage

\begin{ejercicio}{8}
Probar que $S^n$ es subvariedad regular de $\R^{n+1}$.
\end{ejercicio}
\begin{solucion}
Es un caso particular del ejercicio 6.
\end{solucion}

\newpage

\begin{ejercicio}{9}
Sea $M$ una variedad diferenciable y $p\in M$. Si $\{\omega_1,\dots,\omega_m\}$ es una base de
$T^*_p (M)$, probar que existe una carta local $(U, \varphi = (x_1, \dots, x_m))$ entorno de
$p$ tal que $(dx_i)_p = \omega_i, i = 1, \dots,m$.
\end{ejercicio}
\begin{solucion}
Dado que todo covector es la diferencial de alguna función diferenciable en $p$, tenemos que $\omega_i=(df_i)_p, i=1,\dots, m$, por lo que basta tomar $\varphi=(f_1,\dots,f_m)$ y como $U$ la intersección de los dominios donde las $f_i$ son diferenciables, que es abierto no vacío porque hay una cantidad finita y en todos ellos está $p$.
\end{solucion}

\newpage

\begin{ejercicio}{10}
Probar que no existen inmersiones de $S^1$ en $\R$.
\end{ejercicio}
\begin{solucion}
Sea $f\in\calF(S^1,\R)$. En particular, $f$ es continua, por lo que $f(S^1)$ es compacto. Por tanto, $f$ alcanza un máximo en algún punto $p\in S^1$ y un mínimo en algún otro punto $q\in S^1$. Podemos tomar un entorno de $p$ que sea homeomorfo a $\R$, de modo que aplicando el ejercicio 1, la diferencial de $f$ es la diferencial habitual de $\R$ en $\R$. Por tanto $df(p)=df(q)=0$. Si $p\neq q$, claramente $df$ no es inyectiva, y si $p=q$, por ser máximo y mínimo, esto implica que la imagen de $f$ solo contiene un punto, es decir, $f$ es constante, de modo que $df=0$ y seguiría sin ser inyectiva.
\end{solucion}

\newpage

\begin{ejercicio}{11}
Sea $f \in \calF(M,N$) un difeomorfismo local inyectivo entre dos variedades
diferenciables. Probar que $f$ es un difeomorfismo de $M$ en un cierto abierto
de $N$.
\end{ejercicio}
\begin{solucion}
\end{solucion}

\newpage

\begin{ejercicio}{12}
Sean $M$ subvariedad de $\overline{M}$ y $N$ subvariedad de $\overline{N}$ . Sea $f : \overline{M} \to \overline{N}$ una
aplicación diferenciable. ¿Bajo qué condiciones mínimas se puede asegurar
que $f|_M : M \to N$ es diferenciable?
\end{ejercicio}
\begin{solucion}
Como ser diferenciable es una propiedad local y $f|_M=f$ en $M$, dotanto a $M$ y a $N$ de las topologías relativas correspondientes basta que $f(M)\subseteq N$.
\end{solucion}

\newpage

\begin{ejercicio}{13}
Probar que la relación ``ser equivalentes'' entre subvariedades de una variedad
diferenciable es una relación de equivalencia.
\end{ejercicio}
\begin{solucion}
\end{solucion}

\newpage

\begin{ejercicio}{14}
Sean $M \subseteq N$ tales que $(M, i)$ es subvariedad de $N$. Probar que $M$ y $N$
tienen la misma dimensión si y solo si $M$ es subvariedad abierta de $N$.
\end{ejercicio}
\begin{solucion}
$\boxed{\Rightarrow}$ Sea $n$ la dimensión de $N$ y $M$. Se prueba\footnote{\url{https://math.stackexchange.com/questions/84577/open-subsets-in-a-manifold-as-submanifold-of-the-same-dimension}} que $M$ es abierto de $N$. Por el problema 2.5 sabemos que para $i:M\to i(M)$ existe una única estructura diferenciable que convierte a $i$ en difeomorfismo. En particular la convierte en homeomorfismo. Como $i(M)$ tiene la topología relativa a $M$ esto implica que $M$ también la tiene. Entonces sabemos que $i$ es homeomorfismo sobre su imagen. Sea entonces $(U\cap M,\psi)$ una carta de $M$. Vamos a probar que $\psi$ es la restricción a $U\cap M$ de alguna carta local de $N$. Podemos suponer que $U\cap M$ es conexo, pues de lo contrario nos restringiríamos a sus componentes conexas. Además también podemos suponer que es un entorno básico. Sea $(U\cap M,\varphi)$ una carta de $N$ y $f$ un homeomorfismo entre $\varphi(U\cap M)\cong\psi(U\cap M)$ que existe porque ambos son entornos básicos de $\R^n$. Consideremos el siguiente diagrama conmutativo
\[
\begin{tikzcd}
U\cap M\subseteq N\arrow[r,"\varphi"] & \varphi(U\cap M)\arrow[d,"f"]\arrow[d,"\cong"']\\
U\cap M\subseteq M\arrow[u,hookrightarrow, "i"']\arrow[r,"\psi"]& \psi(U\cap M)
\end{tikzcd}
\]
Tenemos pues que $\psi=f\circ\varphi\circ i: U\cap M\subseteq M\to\varphi(U\cap M)$. Pero esta aplicación es igual a $f\circ\varphi|_{U\cap M}:U\cap M\subset N\to\psi(U\cap M)$. Como $f\circ\varphi$ es homeomorfismo, $(U\cap M,f\circ\varphi)$ es una carta de $N$ y además $(U\cap M,\psi)=(U\cap M,f\circ\varphi|_{U\cap M})$, por lo que tenemos lo que queríamos probar.

$\boxed{\Leftarrow}$ Supongamos que $M$ es subvariedad abierta de $N$. En particular $M$ es abierto en $N$. Por lo tanto existe alguna carta en la estructura diferenciable de $N$ de la forma $(M, \varphi)$, tal que $\varphi(M)\subseteq\R^n$. Además, por el ejercicio 7, $M$, es una subvariedad regular, es decir, que $i:M\hookrightarrow i(M)=M$ es homeomorfismo. Tomemos una carta de la estructura diferenciable de $M$ como subvariedad abieta $(M\cap U,\psi|_{M\cap U})$, donde $(U,\psi)$ es una carta de $N$. Supongamos que $M$ tiene una cierta dimensión $m$ y observemos el siguiente diagrama conmutativo.
\[
\begin{tikzcd}
U\cap M\subseteq M\arrow[r,"\psi|_{M\cap U}"]\arrow[d,hookrightarrow, "i"'] & \psi|_{M\cap U}(M\cap U)\subseteq \R^m\\
U\cap M\subseteq N\arrow[r,"\varphi|_{M\cap U}"] & \varphi|_{M\cap U}(M\cap U)\subseteq\R^n\arrow[u, "\psi|_{M\cap U}\circ i\circ\varphi^{-1}|_{M\cap U}"']
\end{tikzcd}
\]
La aplicación $\psi|_{M\cap U}\circ i\circ\varphi^{-1}|_{M\cap U}$ es homeomorfismo por ser composición de homeomorfismos y obtenemos $m=n$ por el teorema de invariancia de dominio.
\end{solucion}

\newpage

\begin{ejercicio}{15}
¿Para qué valores de $a \in R$ es el conjunto
$$M = \{(x, y, z) \in \R^3\mid x^2 + y^2 - z + 2 = 0; z = a\}$$
una subvariedad regular de dimensión 1 de $\R^3$?
\end{ejercicio}
\begin{solucion}
\end{solucion}

\end{document}