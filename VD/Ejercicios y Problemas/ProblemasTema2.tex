	\documentclass[twoside]{article}
\usepackage{../../estilo-ejercicios}
%\newcommand{\A}{{\mathcal{A}}}
%--------------------------------------------------------
\begin{document}

\title{Problemas de Variedades Diferenciables - Tema 2}
\author{Javi, Rafa, Diego}
\maketitle



\begin{ejercicio}{5}
Sea $N$ una variedad diferenciable y $f : M \to N$ una biyección.
Probar que se puede dotar a $M$ de una única estructura de variedad diferenciable
que convierte a $f$ en un difeomorfismo. ¿De qué dimensión?
\end{ejercicio}
\begin{solucion}
Dotamos a $M$ de la topología inicial, con la cual se convierte en homeomorfismo y por lo tanto $M$ será T2 y 2ºN. Denotemos por $\calA_N$ al atlas de $N$, el cual supondremos que es de dimensión $n$. Vamos ahora a construir el atlas de $M$. Sea $p\in M\Rightarrow f(p)\in N\Rightarrow\exists (U,\varphi)\in\calA_N\mid f(p)\in U$. Como $f$ es homeomorfismo, $f^{-1}(U)$ es abierto de $M$ tal que $p\in f^{-1}(U)$. La aplicacíón 
$$\varphi\circ f:f^{-1}(U)\to\varphi(U)\subset\R^n$$
es homeomorfismo por ser composición de homeomorfismos. Así pues, $(f^{-1}(U),\varphi\circ f)$ es una carta local de $M$ de dimensión $n$ , con $p\in f^{-1}(U)$. El atlas $\calA_M$ será la familia construida con estas cartas, que tiene tantas cartas como tiene $\calA_N$. Veamos que están relacionadas dos a dos. Sean $(f^{-1}(U),\varphi\circ f),(f^{-1}(V),\psi\circ f)$ donde $(U,\varphi),(V,\psi)\in\calA_N$. Supongamos que $f^{-1}(U)\cap f^{-1}(V)=f^{-1}(U\cap V)\neq\emptyset$, por lo que $U\cap V\neq\emptyset$. Como $f$ es biyectiva, el cambio de cartas
$$(\varphi\circ )\circ(\psi\circ f)^{-1}:\psi\circ f(f^{-1}(U\cap V))\to\varphi\circ f(f^{-1}(U\cap V))$$
es igual a
$$\varphi\circ\psi^{-1}:\psi(U\cap V)\to\varphi(U\cap V),$$
que es el cambio de cartas en $N$, por lo que es diferenciable. El otro cambio de cartas es análogo. 

Probemos ahora que $f$ es difeomorfismo. Sea $(f^{-1}(U),\varphi\circ f)$ de $\calA_M$ y $(V,\psi)$ de $\calA_N$ tal que $f^{-1}(U)\cap f^{-1}(V)\neq\emptyset$. Tenemos que ver que es diferenciable la aplicación
$$\psi\circ f\circ(\varphi\circ f)^{-1}:(\varphi\circ f)(f^{-1}(U\cap V))\to\psi f(\varphi\circ f)^{-1}[\varphi f(f^{-1}(U\cap V))]$$
Simplificando, obtenemos un cambio de cartas en $N$, que es diferenciable, por lo que tenemos que $f$ lo es. Tenemos que probar lo mismo para $f^{-1}$, es decir, que dadas $(U,\psi)\in\calA_N$ y $(f^{-1}(U),\varphi\circ f)$ tales que $V\cap (f^{-1})^{-1}(f(U))=V\cap U\neq\emptyset$ se tiene la diferenciabilidad de la aplicación
$$(\varphi\circ f)\circ f^{-1}\circ\psi^{-1}:\psi(U\cap V)\to\varphi ff^{-1}\psi^{-1}(\psi(U\cap V)).$$
De nuevo, simplificando obtenemos un cambio de cartas en $N$. 

Nos queda probar la unicidad de la estructura. Para ello, supongamos que hay otra estructura de variedad diferenciable sobre $M$ que convierte a $f$ en difeomorfismo. Llamemos $\calA=\{(O_i,\phi_i)\}_{i\in I}$ al atlas de esta estructura. Queremos ver que este atlas está relacionado con $\calA_M$. Para ello, tomamos $(O,\phi)\in\calA$ y $(f^{-1}(U),\varphi\circ f)$ tales que $O\cap f^{-1}(U)\neq\emptyset$. Entonces
$$(\varphi\circ f)\circ\phi^{-1}:\phi(O\cap f^{-1}(U))\to\varphi f\phi^{-1}(\phi(O\cap f^{-1}(U))$$
es diferenciable porque $\calA$ hace que $f$ sea diferenciable. La otra composición es análoga. Así, los dos atlas forman parte de la misma estructura diferencible.

\end{solucion}

\newpage
\begin{ejercicio}{8}
Sea $M$ una variedad diferenciable. Probar que la aplicación
$$\Delta : M \to M \times M : x \mapsto (x, x)$$
es diferenciable.
\end{ejercicio}
\begin{solucion}
Sea $(U,\varphi)$ una carta local de $M$. Sabemos por un ejercicio del tema anterior que las cartas locales de $M\times M$ son de la forma $(V\times W, \phi\times\psi)$, donde $(V,\phi)$ y $(W,\psi)$ son cartas locales de $M$. Por tanto, para que $U\cap\Delta^{-1}(V\times W)\neq\emptyset$, debe ser $V\times W=U\times U$. Además, como $(U\times U,\phi\times\psi)$ debe ser producto de cartas locales, necesariamente $\phi$ y $\psi$ deben estar relacionadas con $\varphi$. 

Apliquemos pues, la definición de aplicación diferenciable a este contexto. Tenemos que probar que la siguiente aplicación es diferenciable
$$(\phi\times\psi)\circ\Delta\circ\varphi^{-1}: \varphi(U)\to (\phi\times\psi)\circ\Delta(U).$$
Para verlo más claramente vamos a seguir el diagrama a continuación que descompone la aplicación en cada paso
\[
\begin{tikzcd}
y\in\varphi(U)\arrow[r, "\varphi^{-1}"]\arrow[drr, "(\phi\times\psi)\circ\Delta\circ\varphi^{-1}"'] & \varphi^{-1}(y)\in U\arrow[r,"\Delta"] & (\varphi^{-1}(y), \varphi^{-1}(y)))\in \Delta(U)\arrow[d,"\phi\times\psi"]\\
& &  (\phi\circ\varphi^{-1}(y),\psi\circ\varphi^{-1}(y))\in (\phi\times\psi)\circ\Delta(U)
\end{tikzcd}
\]
Cada una de las componentes, $\phi\circ\varphi^{-1}$ y $\psi\circ\varphi^{-1}$, son diferenciables por las observaciones del primer párrafo. Valiéndonos del ejercicio 2.7, esto implica que la aplicación completa es diferenciable, como queríamos demostrar.

\end{solucion}

\end{document}