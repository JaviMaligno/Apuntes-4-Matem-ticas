	\documentclass[twoside]{article}
\usepackage{../../estilo-ejercicios}

%--------------------------------------------------------
\begin{document}

\title{Problemas de Variedades Diferenciables - Tema 5}
\author{Javi, Rafa, Diego}
\maketitle

\begin{ejercicio}{1}
En $\R^3$ se consideran los campos de vectores diferenciables 
$$X=xe^z\frac{\partial}{\partial x}+(1+xyz)\frac{\partial}{\partial y}+z\frac{\partial}{\partial z},\quad Y=2z\frac{\partial}{\partial x}+\cos{y}\frac{\partial}{\partial y}+\frac{\partial}{\partial z}$$
y los campos de tensores covariantes diferenciables
\begin{gather*}
T=x^2dx\otimes dx+dx\otimes dz+e^xdy\otimes dx +ydy\otimes dz+xdz\otimes dy+dz\otimes dz,\\
S=x^3ydx+\cos{z}dy+ydz
\end{gather*}
\begin{enumerate}
\item Calcular $T\otimes S, S(X),T(X,Y)$ Y $(T\otimes S)(X,Y,X)$.
\item Calcular $\omega=Alt(T)$ y $\theta=Alt(S)$. 
\item Calcular $\omega\land\theta,\theta(X),\omega(X,Y)$ y $(\omega\land\theta)(X,Y,X)$.
\end{enumerate}
\end{ejercicio}
\begin{solucion}
\begin{enumerate}
\item[]

\item $S(X)=x^3yxe^z+\cos{z}(1+xyz)+yz\in\calF(\R^3)$. 

$T(X,Y)=x^3e^z+e^x(1+xyz)2z+y(1+xyz)+xz\cos{y}+z$

$T\otimes S(X,Y,X)=T(X,Y)S(X)$

\begin{align*}
T\otimes S &= \sum_{i,j,k=1}^3T\left(\parcial{}{x_i},\parcial{}{x_j}\right)S\left(\parcial{}{x_k}\right) dx_i\otimes dx_j\otimes dx_k \\
&=
x^5y dx\otimes dx \otimes dx + x^2\cos{z}dx\otimes dx\otimes dy + x^3y dx\otimes dx \otimes dx \otimes dz +\\
&+ x^3dx\otimes dz \otimes dx + \cos{z}dx\otimes dz \otimes dy + ydx\otimes dz \otimes dz+\\
&+ e^xyx^3 dy\otimes dx \otimes dx + e^x\cos{z}dy\otimes dx \otimes dy + e^xy dy\otimes dx \otimes dz \\
&+ \dotsc
\end{align*}
\item $\theta=Alt(S)=S$ por ser $1$-forma. 

$w= Alt(T)=f_1dx\land dy+f_2 dx\land dz + f_3 dy\land dz$.
\begin{align*}
f_1 &= Alt(T)\left(\frac{\partial}{\partial x},\frac{\partial}{\partial y}\right)=\frac{1}{2}\left(T\left(\frac{\partial}{\partial x},\frac{\partial}{\partial y}\right)-T\left(\frac{\partial}{\partial y},\frac{\partial}{\partial x}\right)\right)=-\frac{e^x}{2}\\
f_2 &= Alt(T)\left(\frac{\partial}{\partial x},\frac{\partial}{\partial z}\right)=\frac{1}{2}\left(T\left(\frac{\partial}{\partial x},\frac{\partial}{\partial z}\right)-T\left(\frac{\partial}{\partial z},\frac{\partial}{\partial x}\right)\right)=\frac{1}{2}\\
f_3 &= Alt(T)\left(\frac{\partial}{\partial y},\frac{\partial}{\partial z}\right)=\frac{1}{2}\left(T\left(\frac{\partial}{\partial y},\frac{\partial}{\partial z}\right)-T\left(\frac{\partial}{\partial z},\frac{\partial}{\partial y}\right)\right)=\frac{y-x}{2}
\end{align*}
\item 
\begin{align*}
w\land \theta &= -\frac{e^x}{2}y  dx\land dy \land dz +\frac{\cos(z)}{2}dx\land dz \land dy + \frac{y-x}{2}x^3 y dy\land dz \land dx\\
&=\left(-\frac{e^x}{2}y  -\frac{\cos(z)}{2} +  \frac{y-x}{2}x^3 y \right)dx\land dy \land dz\\
\theta(X)&=Alt(S)(X)=S(X)\\
w(X,Y)&= -\frac{e^x}{2}dx\land dy (X,Y) + \frac{1}{2} dx\land dz (X,Y) + \frac{y-x}{2} dy\land dz(X,Y)\\
(w\land \theta)(X,Y,X) & = -(w\land \theta)(X,Y,X) = 0
\end{align*}
\end{enumerate}
Nota: Álgebra Tensorial $T(U)=\bigoplus_{r\geq 0}T_r(U)$. Álgebra Exterior $\Lambda(U)=\bigoplus_{r\geq 0} \Lambda_r(U)$.
\end{solucion}

\newpage

\begin{ejercicio}{3}
Sea $\R^2$. Se consideran los campos:
\[
X=x\frac{\partial}{\partial x}+y\frac{\partial}{\partial y},\quad Y=-y\frac{\partial}{\partial x}+x\frac{\partial}{\partial y}.
\]
Calcular $[X,Y]$.
\end{ejercicio}
\begin{solucion}
\begin{gather*}
[X,Y]=\left[x\frac{\partial}{\partial x}+y\frac{\partial}{\partial y}, -y\frac{\partial}{\partial x}+x\frac{\partial}{\partial y}\right]=\\
\left[x\frac{\partial}{\partial x},-y\frac{\partial}{\partial x}\right]+\left[x\frac{\partial}{\partial x},x\frac{\partial}{\partial x}\right]+\left[y\frac{\partial}{\partial y},-y\frac{\partial}{\partial x}\right]+\left[y\frac{\partial}{\partial y},x\frac{\partial}{\partial y}\right]
\end{gather*}
Desarrollamos cada uno de los corchetes.
\[
\left[x\frac{\partial}{\partial x},-y\frac{\partial}{\partial x}\right]=-xy\left[\frac{\partial}{\partial x},\frac{\partial}{\partial x}\right]+x\frac{
\partial (-y)}{\partial x}\frac{\partial}{\partial x}+y\frac{\partial x}{\partial x}\frac{\partial}{\partial x}=y\frac{\partial}{\partial x}
\]
Análogamente calculamos los demás y al sumar da 0.
\end{solucion}

\begin{ejercicio}{6}
Sean $\R^3$, $X \in \mathcal{X}(\R^3)$ dado por $X = x\frac{\partial}{\partial x} + e^z\frac{\partial}{\partial y}$ y sea $f :
\R^3 \to \R^3 : (x, y, z) \mapsto (x + y, y, 7)$. ¿Está $X$ $f-$relacionado con $X$?
\end{ejercicio}
\begin{solucion}
Sea $p=(x,y,z)\in\R^3$, entonces $f(p)=(x+y,y,7)$. Por tanto, $$X_{f(p)}=(x+y)\frac{\partial}{\partial x} + e^7\frac{\partial}{\partial y}.$$ %Como $f_{*p}$ es homomorfismo de espacios vectoriales, 
%$$f_{*p}X_p=xf_{*p}\left(\frac{\partial}{\partial x}\right) + e^zf_{*p}\left(\frac{\partial}{\partial y}\right).$$
Como la estructura diferencial de $\R^3$ es la usual, la matriz de $f_{*p}$ viene dada por la jacobiana de $f$, es decir, 
\[f_{*p}X_p=\begin{pmatrix}
1 & 1 & 0\\
0 & 1 & 0\\
0 & 0 & 0
\end{pmatrix}\begin{pmatrix}
x\\
e^z\\
0
\end{pmatrix}=\begin{pmatrix}
x+e^z\\
e^z\\
0
\end{pmatrix}\]
Reescribiendo este vector con la base de $T_p(\R^3)$, tenemos que 
$$f_{*p}X_p=(x+e^z)\frac{\partial}{\partial x} + e^z\frac{\partial}{\partial y}\neq Y_{f(p)},$$
por lo que $X$ no está $f-$relacionado con $X$.
\end{solucion}

\newpage

\begin{ejercicio}{}
Probar que la suma de tensores y el producto tensorial de tensores son tensores.
\end{ejercicio}
\begin{solucion}
\end{solucion}

\newpage

\begin{ejercicio}{}
$X=\frac{\partial}{\partial x} +y \frac{\partial}{\partial y}$. $Y=y\frac{\partial}{\partial x}$. $f(x,y)=(e^x,x+y)$. Entonces $f_\ast[X,Y]	=[f_\ast X,f_\ast Y]$ donde $f$ sea diferenciable $(f_\ast X)g = X(g\circ f)\circ f^{-1}$.
\end{ejercicio}
\begin{solucion}
\end{solucion}
\end{document}