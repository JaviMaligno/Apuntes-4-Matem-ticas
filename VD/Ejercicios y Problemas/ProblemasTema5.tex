	\documentclass[twoside]{article}
\usepackage{../../estilo-ejercicios}

%--------------------------------------------------------
\begin{document}

\title{Problemas de Variedades Diferenciables - Tema 5}
\author{Javi, Rafa, Diego}
\maketitle

\begin{ejercicio}{3}
Sea $\R^2$. Se consideran los campos:
\[
X=x\frac{\partial}{\partial x}+y\frac{\partial}{\partial y},\quad Y=-y\frac{\partial}{\partial x}+x\frac{\partial}{\partial y}.
\]
Calcular $[X,Y]$.
\end{ejercicio}
\begin{solucion}
\begin{gather*}
[X,Y]=\left[x\frac{\partial}{\partial x}+y\frac{\partial}{\partial y}, -y\frac{\partial}{\partial x}+x\frac{\partial}{\partial y}\right]=\\
\left[x\frac{\partial}{\partial x},-y\frac{\partial}{\partial x}\right]+\left[x\frac{\partial}{\partial x},x\frac{\partial}{\partial x}\right]+\left[y\frac{\partial}{\partial y},-y\frac{\partial}{\partial x}\right]+\left[y\frac{\partial}{\partial y},x\frac{\partial}{\partial y}\right]
\end{gather*}
Desarrollamos cada uno de los corchetes.
\[
\left[x\frac{\partial}{\partial x},-y\frac{\partial}{\partial x}\right]=-xy\left[\frac{\partial}{\partial x},\frac{\partial}{\partial x}\right]+x\frac{
\partial (-y)}{\partial x}\frac{\partial}{\partial x}+y\frac{\partial x}{\partial x}\frac{\partial}{\partial x}=y\frac{\partial}{\partial x}
\]
Análogamente calculamos los demás y al sumar da 0.
\end{solucion}

\begin{ejercicio}{6}
Sean $\R^3$, $X \in \mathcal{X}(\R^3)$ dado por $X = x\frac{\partial}{\partial x} + e^z\frac{\partial}{\partial y}$ y sea $f :
\R^3 \to \R^3 : (x, y, z) \mapsto (x + y, y, 7)$. ¿Está $X$ $f-$relacionado con $X$?
\end{ejercicio}
\begin{solucion}
Sea $p=(x,y,z)\in\R^3$, entonces $f(p)=(x+y,y,7)$. Por tanto, $$X_{f(p)}=(x+y)\frac{\partial}{\partial x} + e^7\frac{\partial}{\partial y}.$$ %Como $f_{*p}$ es homomorfismo de espacios vectoriales, 
%$$f_{*p}X_p=xf_{*p}\left(\frac{\partial}{\partial x}\right) + e^zf_{*p}\left(\frac{\partial}{\partial y}\right).$$
Como la estructura diferencial de $\R^3$ es la usual, la matriz de $f_{*p}$ viene dada por la jacobiana de $f$, es decir, 
\[f_{*p}X_p=\begin{pmatrix}
1 & 1 & 0\\
0 & 1 & 0\\
0 & 0 & 0
\end{pmatrix}\begin{pmatrix}
x\\
e^z\\
0
\end{pmatrix}=\begin{pmatrix}
x+e^z\\
e^z\\
0
\end{pmatrix}\]
Reescribiendo este vector con la base de $T_p(\R^3)$, tenemos que 
$$f_{*p}X_p=(x+e^z)\frac{\partial}{\partial x} + e^z\frac{\partial}{\partial y}\neq Y_{f(p)},$$
por lo que $X$ no está $f-$relacionado con $X$.
\end{solucion}

\end{document}