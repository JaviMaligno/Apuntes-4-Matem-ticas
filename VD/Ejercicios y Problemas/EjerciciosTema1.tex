	\documentclass[twoside]{article}
\usepackage{../../estilo-ejercicios}

%--------------------------------------------------------
\begin{document}

\title{Problemas de Variedades Diferenciables - Tema 1}
\author{Javi, Rafa, Diego}
\maketitle



\begin{ejercicio}{1}\label{1}
Sean $\calA$ un atlas sobre $M$ y $(U,\varphi) \in \calA$ una carta local. Si $V \subseteq U$ es un abierto, probar que
$(V, \varphi|_V )$ es una carta local admisible en $\calA$.
\end{ejercicio}
\begin{solucion}
Veamos que es carta. Por hipótesis $V$ es abierto de $U$, luego también es abierto de $M$. Además, sabemos que la restricción de un homeomorfismo es un homeomorfismo, lo que prueba que es carta. Para ver que es admisible en $\calA$, consideremos $(S,\psi)\in \calA$ tal que $S\cap V \neq \emptyset$ (otro caso sería trivial). En tal caso $\forall x \in \varphi|_V (S\cap V)$
$$
\psi \circ \varphi|_V^{-1}\equiv \psi \circ \varphi^{-1} \in \mathcal{C}^\infty \qquad \varphi|_V \circ \psi^{-1}\equiv \varphi \circ \psi \in \mathcal{C}^\infty
$$

\end{solucion}

\newpage

\begin{ejercicio}{2}
Sea $M$ una variedad diferenciable. Probar que, dado cualquier $p \in M$, existe una carta local
$(U, \varphi)$ de la estructura diferenciable centrada en $p$ (es decir, tal que $\varphi(p) = 0$).
\end{ejercicio}
\begin{solucion}

Sea $(U,\psi)$ una carta cualquiera de la estructura diferenciable tal que $\psi(p)=q\neq 0$, basta tomar $(U,\varphi)$ con $\varphi=\psi-q$. Se tiene que $\varphi(p)=0$ y claramente sigue siendo homeomorfismo y manteniendo las propiedades de diferenciabilidad de $\psi$, por lo que está en la estructura diferenciable.
\end{solucion}

\newpage

\begin{ejercicio}{3}
Sea $M$ una variedad diferenciable y $\calA$ el atlas maximal de la estructura diferenciable. Probar
que los dominios de las cartas locales de $\calA$ forman base de la topología de $M$.
\end{ejercicio}
\begin{solucion}
\end{solucion}

\newpage

\begin{ejercicio}{4}
Sea $(M,\mathcal{A})$ una variedad diferenciable y $G\subseteq M$ un abierto. Probar que,
dando a $G$ la topología relativa de la de $M$,
$$\mathcal{B} = \{(G \cap U, \varphi|_{G\cap U})\}_{(U,\varphi)\in\mathcal{A}}$$
es un atlas sobre $G$ que lo dota de estructura de variedad diferenciable de la
misma dimensión que $M$, llamada \textbf{Estructura de Subvariedad Abierta}
de $G$.
\end{ejercicio}
\begin{solucion}
Primero veamos que los elementos de $\mathcal{B}$ son cartas locales. Por definición son abiertos de la topología relativa, y las restricciones de homeomorfismos siguen siendo homeomorfismos. De hecho, $\varphi|_{G\cap U}(G\cap U)=\varphi(G\cap U)=\varphi(G)\cap\varphi(U)$, que es abierto del mismo $\R^m$ (conservando así la dimensión) por ser intersección finita de abiertos. Para ver que las cartas están relacionadas entre sí basta hacer un razonamiento  análogo al del ejercicio \ref{1}. Por último, para ver que recubren $G$, 
$$G=M\cap G=\left(\bigcup_{(U,\varphi)\in\mathcal{A}}U\right)\cap G= \bigcup_{(U,\varphi)\in\mathcal{A}}(U\cap G).$$
\end{solucion}
\newpage
\begin{ejercicio}{5}
Probar que todo espacio vectorial real $m$-dimensional admite una estructura
diferenciable canónica de dimensión $m$.
\end{ejercicio}
\begin{solucion}
Sea $V$ un espacio vectorial real $m$-dimensional. Entonces, por álgebra lineal existe una aplicación lineal biyectiva (y por tanto homeomorfismo con la topología inicial) $f:V\to\R^m$. Basta tomar el atlas unitario $\mathcal{A}=\{(V,f)\}$ para dotar a $V$ de estructura diferenciable.  
\end{solucion}
\newpage
\begin{ejercicio}{6}
Probar que $\C^m$ es una variedad diferenciable de dimensión $2m$.
\end{ejercicio}
\begin{solucion}
Basta notar que $\C^m$ es un $\R$-espacio vectorial de dimensión $2m$ y aplicar el ejercicio anterior.
\end{solucion}
\newpage
\begin{ejercicio}{7}
Probar que el producto cartesiano de dos variedades diferenciables $M$ y $N$
es una variedad diferenciable de dimensión la suma de las dimensiones de
las variedades factores. De hecho, si $(U,\varphi)$ es una carta de $M$ y $(V,\psi )$ una
carta de $N$, se verifica que $(U \times V,\varphi\times\psi )$ es una carta sobre $M \times N$.
\end{ejercicio}
\begin{solucion}
Sean $\mathcal{A}_1$ un atlas de dimensión $m$ para $M$ y $\mathcal{A}_2$ un atlas de dimensión $n$ para $N$. Para probar lo que se nos pide, construimos el atlas $\mathcal{A}=\{(U\times V,\varphi\times\psi): (U,\varphi)\in\mathcal{A}_1, (V,\psi)\in\mathcal{A}_2\}$. Veamos que sus elementos son cartas locales. En primer lugar, $U\times V$ es abierto de la topología producto y $\varphi\times\psi(U\times V)=\varphi(U)\times\psi(V)$ es un abierto de $\R^n\times\R^n=\R^{n+m}$. Además, el producto cartesiano de homeomorfismos es claramente homeomorfismo. Probemos ahora que las cartas están relacionadas dos a dos. 

Sean $(U_1\times V_1,\varphi_1\times\psi_1)\in\mathcal{A}_1$ y $(U_2\times V_2,\varphi_2\times\psi_2)\in\mathcal{A}_2$. Entonces 
$$(\varphi_1\times\psi_1)\circ(\varphi_2\times\psi_2)^{-1}=(\varphi_1\times\psi_1)\circ(\varphi_2^{-1}\times\psi_2^{-1})=(\varphi_1\circ\varphi_2^{-1},\psi_1\circ\psi_2^{-1}).$$
Como cada componente es infinitamente diferenciable por hipótesis, el proucto también lo es. Finalmente, veamos que las cartas recubren $M\times N$,
$$M\times N= \bigcup_{(U,\varphi)\in\mathcal{A}_1}U\times\bigcup_{(V,\psi)\in\mathcal{A}_1}V=\bigcup_{(U\times V,\varphi\times\psi)\in\mathcal{A}}U\times V.$$
\end{solucion}

\end{document}