\documentclass[twoside]{article}
\usepackage{estilo-ejercicios}
\usepackage{fancyhdr}
\usepackage{empheq}
\usepackage{xcolor}

\newsavebox\MBox
\newcommand\Cline[2][red]{{\sbox\MBox{$#2$}%
  \rlap{\usebox\MBox}\color{#1}\rule[-1.2\dp\MBox]{\wd\MBox}{0.5pt}}}
\newcommand{\V}{\wedge}
\newcommand{\parcial}[2]{\frac{\partial #1}{\partial #2}}
%--------------------------------------------------------
\begin{document}
\pagestyle{fancy}
\lhead{Ejercicio Navideño}
\rhead{Javier Aguilar Martín, Javier González Doña}


\begin{problema}
Resolver los siguientes apartados:
\begin{enumerate}
\item Sean
\begin{gather*}
ω = f_1dx ∧ dy + f_2dx ∧ dz + f_3dy ∧ dz ∈ Λ_2(\R^3),\\
θ = g_1dx + g_2dy + g_3dz ∈ Λ_1(\R^3).
\end{gather*}
Elegir valores no constantes para las funciones $f_1$, $f_2$, $f_3$, $g_1$, $g_2$, $g_3$ y comprobar
que:
$$Alt(ω ⊗ θ) = \frac{1}{3}
ω ∧ θ.$$

\item Sean
\begin{align*}
ω &=f_1dx ∧ dy + f_2dx ∧ dz + f_3dx ∧ dt\\
&+f_4dy ∧ dz + f_5dy ∧ dt + f_6dz ∧ dt ∈ Λ_2(\R^4),\\
θ &= g_1dx + g_2dy + g_3dz + g_4dt ∈ Λ_1(\R^4).
\end{align*}
Elegir valores no constantes para las funciones $f_1$, $f_2$, $f_3$, $f_4$, $f_5$, $f_6$, $g_1$, $g_2$, $g_3$, $g_4$
y comprobar que:
$$Alt(ω ⊗ θ) = \frac{1}{3}
ω ∧ θ.$$
\end{enumerate}
\end{problema}

\begin{solucion}\
\begin{enumerate}
\item Tomemos las funciones $f_1 = g_1 = e^x$, $f_2 = g_2 = e^y$ y $f_3 = g^3 = e^z$. \\
Obtenemos entonces las siguientes formas diferenciales: \begin{equation*}
\begin{split}
	 & \omega = e^x dx \V dy + e^y dx \V dz + e^z dy\V dz \\
	 & \theta = e^x dx + e^y dy + e^z dz
	\end{split}
\end{equation*}
Calculemos $\omega \V \theta.$ Se tiene: \begin{equation*}
\begin{split}
\omega \V \theta & = (e^x dx \V dy + e^y dx \V dz + e^z dy\V dz) \V ( e^x dx + e^y dy + e^z dz) \\ & = e^{x+z} dx \V dy \V dz + e^{2y} dx \V dz \V dy + e^{z+x} dy\V dz \V dx \\ & = (2e^{x+z} + e^{2y})dx \V dy \V dz.
\end{split}
\end{equation*}
A continuación, calculamos $Alt (\omega \otimes \theta).$ \\
Como $\omega \in T_2(\mathcal{\R}^2)$ y $\theta \in T_1(\mathcal{\R})$, se sigue que  $\omega \otimes \theta \in T_3(\mathcal{\R}^3)$, y así, $ Alt (\omega \otimes \theta) \in \varLambda_3(\mathcal{\R}^3)$. Por tanto, en virtud del  lema 3 para formas, sabemos que \begin{equation*}
\begin{split}
Alt(\omega \otimes \theta) & = Alt (\omega \otimes \theta)\left(\parcial{}{x},\parcial{}{y},\parcial{}{z}\right) dx \V dy \V dz.
\end{split}
\end{equation*}
Así, para probar la igualdad deseada, basta ver que \begin{equation*}
 Alt (\omega \otimes \theta) \left(\parcial{}{x},\parcial{}{y},\parcial{}{z}\right) = \frac{1}{3} (2e^{x+z} + e^{2y}).
\end{equation*}
Para ello, tenemos \begin{equation*}
\begin{split}
 & Alt (\omega \otimes \theta) \left(\parcial{}{x},\parcial{}{y},\parcial{}{z}\right)  =  \frac{1}{3!}\left[ ω\left(\parcial{}{x},\parcial{}{y}\right)θ\left(\parcial{}{z}\right)-ω\left(\parcial{}{y},\parcial{}{x}\right)θ\left(\parcial{}{z}\right)-ω\left(\parcial{}{z},\parcial{}{y}\right)θ\left(\parcial{}{x}\right)-\right.\\
&\left.ω\left(\parcial{}{x},\parcial{}{z}\right)θ\left(\parcial{}{y}\right)+ω\left(\parcial{}{z},\parcial{}{x}\right)θ\left(\parcial{}{y}\right)+ω\left(\parcial{}{y},\parcial{}{z}\right)θ\left(\parcial{}{x}\right)\right]=\\ 
& \frac{1}{3!} \left(e^xe^z + e^xe^z +e^ze^x - e^ye^y - e^ye^y +   e^ze^x    \right) = \frac{2}{3}e^{x+z} + \frac{1}{3}e^{2y},
\end{split}
\end{equation*} 
como queríamos probar.

\item Elegimos $f_1=f_2=f_3=f_4=f_5=f_6=x$ y $g_1=g_2=g_3=g_4=y$. Así, las formas diferenciales resultantes son
\begin{align*}
ω &=xdx ∧ dy + xdx ∧ dz + xdx ∧ dt\\
&+xdy ∧ dz + xdy ∧ dt + xdz ∧ dt ,\\
θ &= ydx + ydy + ydz + ydt.
\end{align*}

Empezamos calculando $ω ∧ θ$.
\begin{align*}
ω ∧ θ=&(xdx ∧ dy + xdx ∧ dz + xdx ∧ dt+ xdy ∧ dz + xdy ∧ dt + xdz ∧ dt)\wedge\\
&(ydx + ydy + ydz + ydt)=\\
&xy\Cline[blue]{dx ∧ dy\wedge dz}+ xy\Cline[red]{dx ∧ dy\wedge dt}+ xy\Cline[blue]{dx ∧ dz\wedge dy}+ xy\Cline[green]{dx\wedge dz\wedge dt}+\\
&xy\Cline[red]{dx ∧ dt\wedge dy}+ xy\Cline[green]{dx ∧ dt\wedge dz} + xy\Cline[blue]{dy ∧ dz\wedge dx}+ xy\Cline[yellow]{dy ∧ dz\wedge dt}+\\ 
& xy\Cline[red]{dy ∧ dt\wedge dx} + xy\Cline[yellow]{dy ∧ dt\wedge dz}+  xy\Cline[green]{dz ∧ dt\wedge dx}+   xy\Cline[yellow]{dz ∧ dt\wedge dy}=\\
   &\boxed{xy\Cline[blue]{dx ∧ dy\wedge dz}+ xy\Cline[red]{dx ∧ dy\wedge dt} +xy\Cline[green]{dx\wedge dz\wedge dt} + xy\Cline[yellow]{dy ∧ dz\wedge dt}}
\end{align*}
\begin{comment}
\begin{empheq}[box=\fbox]{align*}
   &xydx\wedge dy\wedge dz + xydx\wedge dy\wedge dt +\\
& xydx\wedge dz\wedge dt + xydy\wedge dz\wedge dt
\end{empheq}
\end{comment}

\begin{comment}
\bf 1ª MANERA}

Observamos que para tres $1$-formas $\alpha$, $\beta$ y $\gamma$ se tiene
\begin{align*}
&\alpha\wedge\beta\wedge\gamma =3!Alt(\alpha\otimes\beta\otimes\gamma) =\frac{3!}{3!}\sum_{\sigma\in S_3}\varepsilon(\sigma)\sigma(\alpha)\otimes\sigma(\beta)\otimes\sigma(\gamma)=\\
&\alpha\otimes\beta\otimes\gamma -\beta\otimes\alpha\otimes\gamma-\gamma\otimes\beta\otimes\alpha -\alpha\otimes\gamma\otimes\beta + \gamma\otimes\alpha\otimes\beta + \beta\otimes\gamma\otimes\alpha
\end{align*}

{\bf 2ª MANERA}
\end{comment}
Usando el lema 3 para formas,
\begin{align*}
Alt(ω ⊗ θ) =& Alt(ω ⊗ θ)\left(\parcial{}{x},\parcial{}{y},\parcial{}{z}\right)dx\wedge dy\wedge dz +Alt(ω ⊗ θ)\left(\parcial{}{x},\parcial{}{z},\parcial{}{t}\right)dx\wedge dz\wedge dt+\\
& Alt(ω ⊗ θ)\left(\parcial{}{x},\parcial{}{y},\parcial{}{t}\right)dx\wedge dy\wedge dt+ Alt(ω ⊗ θ)\left(\parcial{}{y},\parcial{}{z},\parcial{}{t}\right)dy\wedge dz\wedge dt
\end{align*}

Tenemos que ver que cada coeficiente es un $\frac{1}{3}xy$. Hacemos el primero y los demás serán análogos.

\begin{align*}
&\frac{1}{3!}\left[ ω\left(\parcial{}{x},\parcial{}{y}\right)θ\left(\parcial{}{z}\right)-ω\left(\parcial{}{y},\parcial{}{x}\right)θ\left(\parcial{}{z}\right)-ω\left(\parcial{}{z},\parcial{}{y}\right)θ\left(\parcial{}{x}\right)-\right.\\
&\left.ω\left(\parcial{}{x},\parcial{}{z}\right)θ\left(\parcial{}{y}\right)+ω\left(\parcial{}{z},\parcial{}{x}\right)θ\left(\parcial{}{y}\right)+ω\left(\parcial{}{y},\parcial{}{z}\right)θ\left(\parcial{}{x}\right)\right]=\\
&\frac{1}{3!}(xy+xy+xy-xy-xy+xy)=\frac{2}{6}xy=\boxed{\frac{1}{3}xy}
\end{align*}

Por lo que tenemos el resultado.


\end{enumerate}






\end{solucion}
\end{document}
⊗