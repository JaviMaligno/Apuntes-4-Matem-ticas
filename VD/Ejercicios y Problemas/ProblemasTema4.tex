	\documentclass[twoside]{article}
\usepackage{../../estilo-ejercicios}

%--------------------------------------------------------
\begin{document}

\title{Problemas de Variedades Diferenciables - Tema 4}
\author{Javi, Rafa, Diego}
\maketitle

\begin{ejercicio}{6}
Sean $\R^3$, $X \in \mathcal{X}(\R^3)$ dado por $X = x\frac{\partial}{\partial x} + e^z\frac{\partial}{\partial y}$ y sea $f :
\R^3 \to \R^3 : (x, y, z) \mapsto (x + y, y, 7)$. ¿Está $X$ $f-$relacionado con $X$?
\end{ejercicio}
\begin{solucion}
Sea $p=(x_0,y_0,z_0)\in\R^3$, entonces $f(p)=(x_0+y_0,y_0,7)$. Por tanto, $$X_{f(p)}=(x_0+y_0)\left(\frac{\partial}{\partial x}\right)_{f(p)} + e^7\left(\frac{\partial}{\partial y}\right)_{f(p)}.$$ %Como $f_{*p}$ es homomorfismo de espacios vectoriales, 
%$$f_{*p}X_p=xf_{*p}\left(\frac{\partial}{\partial x}\right) + e^zf_{*p}\left(\frac{\partial}{\partial y}\right).$$
Como la estructura diferencial de $\R^3$ es la usual, la matriz de $f_{*p}$ viene dada por la jacobiana de $f$, es decir, 
\[f_{*p}X_p=\begin{pmatrix}
1 & 1 & 0\\
0 & 1 & 0\\
0 & 0 & 0
\end{pmatrix}\begin{pmatrix}
x_0\\
e^{z_0}\\
0
\end{pmatrix}=\begin{pmatrix}
x_0+e^{z_0}\\
e^{z_0}\\
0
\end{pmatrix}\]
Reescribiendo este vector con la base de $T_p(\R^3)$, tenemos que 
$$f_{*p}X_p=(x_0+e^{z_0})\left(\frac{\partial}{\partial x}\right)_p + e^{z_0}\left(\frac{\partial}{\partial y}\right)_p\neq X_{f(p)}\text{ en general},$$
por lo que $X$ no está $f-$relacionado con $X$.
\end{solucion}
\newpage
\begin{ejercicio}{7}
Sea $f\func{M}{N}$ una aplicación diferenciable. Probar:
\begin{enumerate}
\item Si $f$ es sobreyectiva, entonces, $\forall X \in \mathcal{X}(M)$ fijo, existe a lo más un campo de vectores $Y\in \mathcal{X}(N)$ tal que $X$ está $f$-relacionado con $Y$. 
\item Si $f$ es una inmersión, entonces, $\forall Y \in \mathcal{X}(N)$, existe a lo más un campo de vectores $X\in \mathcal{X}(M)$ tal que $X$ está $f$-relacionado con $Y$. 
\item Si $f$ es una inmersión, entonces, dado $Y \in \mathcal{X}(N)$, existe $X\in \mathcal{X}(M)$, $f$-relacionado con $Y$ si y solo si $\forall p \in M$, $Y_{f(p)}\in f_{\ast p}(T_p(M))$.
\end{enumerate}
\end{ejercicio}
\begin{solucion}
\begin{enumerate}
\item[]
\item Tenemos que probar que, dada la existencia, debe haber unicidad. Sea $X\in\mathcal{X}(M)$ fijo y supongamos que existen dos campos de vectores $Y,Y'\in \mathcal{X}(N)$ tales que $X$ está $f$-relacionados con $Y$ e $Y'$, es decir
$$f_{\ast p}X_p = Y_{f(p)} \quad \forall p\in M \qquad
f_{\ast q}X_q = Y'_{f(q)} \quad \forall q\in M $$ 
Sea $w\in N$, como $f$ es sobreyectiva, entonces $\exists p \in M$ tal que $w=f(p)$. Se sigue trivialmente que
$$
Y_w = f_{\ast p}X_p = Y'_w
$$
Como son iguales para todo $w\in N$, entonces $Y=Y'$.
\item Análogamente al apartado anterior, sea $Y\in\mathcal{X}(N)$ fijo y supongamos que existen dos campos de vectores $X,X'\in \mathcal{X}(N)$ tales que $X,X'$ está $f$-relacionados con $Y$ e $Y'$, es decir
$$f_{\ast p}X_p = Y_{f(p)} \quad \forall p\in M \qquad
f_{\ast q}X'_q = Y_{f(q)} \quad \forall q\in M $$ 
Por la propia definición de inmersión, es decir, que $f_{\ast}$ sea inyectiva en cada punto, tenemos que, como $f_{\ast p}X_p = Y_{f(p)} = f_{\ast p}X'_p$ $\forall p\in M$ entonces $ X_p = X'_p$ $\forall p\in M$, es decir, $X=X'$.
\item Sea $f$ una inmersión y sea $Y \in \mathcal{X}(N)$.
\begin{itemize}
\item Supongamos que $\exists X \in \mathcal{X}(M)$ tal que $X$ está $f$-relacionado con $Y$. Por definición $\forall p\in M$ $Y_{f(p)}=f_{\ast p}X_p$. Como $X_p\in T_p(M)$ entonces ${Y_{f(p)}\in f_{\ast p}(T_p(M))}$.
\item Recíprocamente, supongamos que $\forall p \in M$, $Y_{f(p)}\in f_{\ast p}(T_p(M))$. Como $f_{\ast p}$ es una inmersión, es inyectiva. En particular, es un isomorfismo sobre su imagen, luego $\forall p\in M$ podemos considerar $X_p = f_{\ast p}^{-1}(Y_{f(p)})\in T_p(M)$, lo cual está correctamente definido por la hipótesis. Obviamente $X$ es un campo de vectores y verifica que
$$f_{\ast p} X_p = f_{\ast p} f_{\ast p}^{-1} Y_p = Y_p \quad \forall p \in M$$
Por lo que $X$ está $f$-relacionado con $Y$. Resta probar que $X$ es un campo diferenciable. Sea $g \in \mathcal{F}(M)$, entonces 
$$
X g \func{M}{\R} \qquad (Xf)(p) = X_p g = (f_{\ast p}^{-1} Y_{f(p)})f = Y_{f(p)}(g \circ f^{-1})
$$
Como $g$ y $f^{-1}$ son diferenciables, la composición también lo es y al ser $Y$ campo diferenciable, tenemos que $Y(g\circ f^{-1})$ también es diferenciable, por lo que $Xf$ también lo es.
\end{itemize} 
\end{enumerate}
\end{solucion}

\end{document}