	\documentclass[twoside]{article}
\usepackage{../../estilo-ejercicios}

%--------------------------------------------------------
\begin{document}

\title{Problemas de Variedades Diferenciables - Tema 4}
\author{Javi, Rafa, Diego}
\maketitle

\begin{ejercicio}{6}
Sean $\R^3$, $X \in \mathcal{X}(\R^3)$ dado por $X = x\frac{\partial}{\partial x} + e^z\frac{\partial}{\partial y}$ y sea $f :
\R^3 \to \R^3 : (x, y, z) \mapsto (x + y, y, 7)$. ¿Está $X$ $f-$relacionado con $X$?
\end{ejercicio}
\begin{solucion}
Sea $p=(x,y,z)\in\R^3$, entonces $f(p)=(x+y,y,7)$. Por tanto, $$X_{f(p)}=(x+y)\frac{\partial}{\partial x} + e^7\frac{\partial}{\partial y}.$$ 
Como la estructura diferencial de $\R^3$ es la usual, la matriz de $f_{*p}$ viene dada por la jacobiana de $f$, es decir, 
\[f_{*p}X_p=\begin{pmatrix}
1 & 1 & 0\\
0 & 1 & 0\\
0 & 0 & 0
\end{pmatrix}\begin{pmatrix}
x\\
e^z\\
0
\end{pmatrix}=\begin{pmatrix}
x+e^z\\
e^z\\
0
\end{pmatrix}\]
Reescribiendo este vector con la base de $T_p(\R^3)$, tenemos que 
$$f_{*p}X_p=(x+e^z)\frac{\partial}{\partial x} + e^z\frac{\partial}{\partial y}\neq Y_{f(p)},$$
por lo que $X$ no está $f-$relacionado con $X$.
\end{solucion}
\newpage
\begin{ejercicio}{7}
Sea $f\func{M}{N}$ una aplicación diferenciable. Probar:
\begin{enumerate}
\item Si $f$ es sobreyectiva, entonces, $\forall X \in \mathcal{X}(M)$ fijo, existe a lo más un campo de vectores $Y\in \mathcal{X}(N)$ tal que $X$ está $f$-relacionado con $Y$. 
\item Si $f$ es una inmersión, entonces, $\forall Y \in \mathcal{X}(N)$, existe a lo más un campo de vectores $X\in \mathcal{X}(M)$ tal que $X$ está $f$-relacionado con $Y$. 
\item Si $f$ es una inmersión, entonces, dado $\forall Y \in \mathcal{X}(N)$, existe $X\in \mathcal{X}(M)$, $f$-relacionado con $Y$ si y solo si $\forall p \in M$, $Y_{f(p)}\in f_{\ast p}(T_p(M))$.
\end{enumerate}
\end{ejercicio}
\begin{solucion}
\end{solucion}

\end{document}