	\documentclass[twoside]{article}
\usepackage{../../estilo-ejercicios}

%--------------------------------------------------------
\begin{document}

\title{Ejercicios de Variedades Diferenciables - Tema 2}
\author{Javi, Rafa, Diego}
\maketitle



\begin{ejercicio}{1}\label{1}
Sean $M$ y $N$ dos variedades diferenciables y $G \subseteq M$ un abierto.
\begin{itemize}
\item[(a)] Si $f \in \calF(G,N)$ y $H \subseteq G$ es un abierto, probar que $f|_H \in \calF(H,N)$.
\item[(b)] Si $f: G \to N$ es una aplicación y $\{G_i\}_{i\in I}$ es un recubrimiento por
abiertos de $G$ (es decir, $G = \cup_{i\in I} G_i$) tal que $f|_{G_i} \in \calF(G_i,N)$, para
todo $i \in I$, probar que $f \in \calF(G,N)$.
\end{itemize}
\end{ejercicio}
\begin{solucion}\
\begin{itemize}
\item[(a)]Como $f\in\calF(G,N)$, tenemos que dadas dos cartas $(U,\varphi)$ de $G$ y $(V,\psi)$ de $N$ tales que $U\cap f^{-1}(V)\neq\emptyset$ entonces
$$\psi\circ f\circ\varphi^{-1}:\varphi(U\cap f^{-1}(V))\to\psi f(U\cap f^{-1}(V))\in\mathcal{C}^\infty$$ 
Como $H\subseteq G$, posee estructura de subvariedad abierta, por lo que podemos tomar una carta de la forma $(U\cap H,\varphi|_H)$. Supongamos entonces que $(U\cap H)\cap f^{-1}(V)\neq\emptyset$. Tenemos que comprobar que 
$$\psi\circ f|_H\circ\varphi|_H^{-1}:\varphi|_H((U\cap H)\cap f|_H^{-1}(V))\to\psi f|_H((U\cap H)\cap f^{-1}(V))\in\mathcal{C}^\infty$$ 
Pero esta aplicación no es más que la restricción de la aplicación anterior al abierto $H$, por lo que efectivamente es diferenciable.

\newpage

\item[(b)] De las hipótesis obtenemos que dada $(U_i,\varphi_i)$ de $G_i$ y $(V,\psi)$ en $N$ tales que $U_i\cap f|_{G_i}^{-1}(V)\neq\emptyset$ se tiene
$$\psi\circ f|_{G_i}\circ\varphi_i^{-1}:\varphi_i(U_i\cap f|_{G_i}^{-1}(V))\to\psi f|_{G_i}(U\cap f|_{G_i}^{-1}(V))\in\mathcal{C}^\infty$$
Dada una carta $(U,\varphi)$ de $G$, sabemos que $U=\cup_{i\in \Lambda}U_i$ y $\varphi=\cup_{i\in \Lambda}\varphi$, donde $\Lambda\subseteq I$. Por tanto, basta ver que $\cup_{i\in\Lambda}\psi\circ f|_{G_i}\circ\varphi_i^{-1}\in\mathcal{C}^\infty$. Claramente es diferenciable en cada abierto. En la intersección basta tomar el cambio de cartas adecuado. Por ejemplo, tomemos $U_i\cap U_j\neq\emptyset$, entonces, en ese abierto, $\psi\circ f|_{G_i}\circ\varphi_i^{-1}=(\psi\circ f|_{G_i}\circ\varphi_j^{-1})\circ\varphi_j\circ\varphi_i^{-1}$. La primera parte es diferenciable por hipótesis y la segunda por ser un cambio de cartas. Así pues, la aplicación es diferenciable en las intersecciones y por tanto, en toda la unión.
\end{itemize}
\end{solucion}

\newpage

\begin{ejercicio}{2}
Sean $M$ una variedad diferenciable y $G \subseteq M$ un abierto. Probar que la
aplicación inclusión $i : G \to M$ es diferenciable.
\end{ejercicio}
\begin{solucion}
Sea $(U\cap G,\varphi|_G)$ una carta de $G$ como subvariedad de $M$ y sea $(V,\psi)$ tal que $U\cap G\subseteq V$ y $\psi$ está relacionada con $\varphi|_G$. Claramente $U\cap G\cap i^{-1}(U\cap G)=U\cap G$. Tenemos que la aplicación 
$$\psi\circ i\circ\varphi|_G^{-1}:\varphi|_G(U\cap G)\to \psi i(U\cap G)=\psi(U\cap G)$$
coincide con $\varphi\circ\varphi|_G^{-1}$, que por hipótesis es diferenciable.
\end{solucion}

\newpage

\begin{ejercicio}{3}
Probar que la composición de aplicaciones diferenciables es una aplicación
diferenciable.
\end{ejercicio}
\begin{solucion}
Sean $f:M\to N$ y $g:N\to Y$ dos aplicaciones diferenciables. Vamos a probar que $g\circ g:M\to Y$ es diferenciable. Para ello, sean $(U,\varphi)$ una carta de $M$ y $(X,\phi)$ una carta de $Y$, tales que $U\cap (gf)^{-1}(X)\neq\emptyset$. Tenemos que probar que
$$\phi\circ gf\circ \varphi^{-1}:\varphi(U\cap (gf)^{-1}(X)\neq\emptyset)\to \phi gf(U\cap (gf)^{-1}(X)\neq\emptyset)\in\mathcal{C}^\infty$$
Para ello basta tomar una carta $(V,\psi)$ de $N$ tal que $U\cap f^{-1}(V)\neq\emptyset\neq V\cap g^{-1}(X)$. Así pues,
$$\phi\circ gf\circ \varphi^{-1}=(\phi\circ g\circ\psi^{-1})\circ(\psi\circ f\circ \varphi^{-1}).$$
La primera parte es diferenciable por ser $g$ diferenciable y la segunda lo es por ser $f$ diferenciable, con lo que la composición es diferenciable, como queríamos probar.
\end{solucion}

\newpage

\begin{ejercicio}{4}
Probar que la aplicación identidad entre una variedad diferenciable y ella
misma es un difeomorfismo.
\end{ejercicio}
\begin{solucion}
Supongamos que tenemos una varidad diferenciable $M$ de dimensión $m$. Como $Id^{-1}=Id$ basta ver que la identidad es diferenciable. Esto esto es claro, puesto que dada una carta $(U,\varphi)$ de $M$ 
$$\varphi\circ Id\circ\varphi^{-1}=\varphi(U)\to\varphi(U)=Id_{\R^m}|_{\varphi(U)},$$
que es diferenciable.
\end{solucion}

\newpage

\begin{ejercicio}{5}
Sean $(M,\calA_1)$ y $(M,\calA_2)$ dos variedades diferenciables. ¿Bajo qué condiciones
es $Id : (M,\calA_1) \to (M,\calA_2)$ una aplicación diferenciable? ¿Cuándo es un
difeomorfismo?
\end{ejercicio}
\begin{solucion}
Consideremos $(U,\varphi)\in\calA_1, (V,\psi)\in\calA_2$ tales que $U\cap V\neq\emptyset$. Para que $Id$ sea diferenciable, 
$$\psi\circ Id\circ\varphi^{-1}=\psi\circ\varphi:\varphi(U\cap V)\to\psi(U\cap V)\in \mathcal{C}^\infty$$
Si además queremos que sea difeomorfismo, se tendrá que cumplir también
$$\varphi\circ Id\circ\psi^{-1}=\varphi\circ\psi{-1}:\psi(U\cap V)\to\varphi(U\cap V)\in\mathcal{C}^\infty$$
Lo cual significa que todas las cartas estén relacionadas, es decir, que $\calA_1$ y $\calA_2$ sean compatibles.
\end{solucion}

\newpage

\begin{ejercicio}{6}
Probar que cualquier aplicación constante es diferenciable.
\end{ejercicio}
\begin{solucion}
Sean $M$ y $N$ dos variedades diferenciables. Supongamos que $N$ es de dimensión $n$. Sea $P:M\to N$ la aplicación constante $x\mapsto p\in N\ \forall x\in M$. Sea $(U,\varphi)$ una carta de $M$ y $(V,\psi)$ una carta de $N$ tal que $p\in V$. Claramente $U\cap P^{-1}(V)=U$. Así pues, tenemos la aplicación
$$\psi\circ P\circ\varphi^{-1}:\varphi(U)\to\psi P(U)=\psi(p)\in\R^n,$$
que es diferenciable por ser la aplicación constante en $\R^n$. 
\end{solucion}



\end{document}