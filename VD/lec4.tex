\documentclass[cursovd_portada.tex]{subfiles}
\begin{document}

\chapter{Aplicaci\'{o}n Diferencial de una\\ Aplicaci\'{o}n
Diferenciable}
 \hs La definici\'{o}n de espacio tangente en cada
punto de una variedad diferenciable y, por tanto, la idea de
``aproximar" la variedad en el punto por un espacio vectorial,
permite, a su vez, dada una aplicaci\'{o}n diferenciable entre dos
variedades diferenciables, definir la noci\'{o}n de diferencial de
tal aplicaci\'{o}n en cada punto de la variedad dominio como una
aplicaci\'{o}n lineal entre los espacios tangentes del punto y del
punto imagen, que generaliza a la bien conocida entre espacios
eucl\'{\i}deos. El conocimiento de las propiedades de tales
aplicaciones es de gran utilidad, pues est\'{a}n directamente
relacionadas con la naturaleza diferenciable de la aplicaci\'{o}n
de partida.
\section{Aplicaci\'{o}n Diferencial y Aplicaci\'{o}n Co\-di\-fe\-ren\-cial de una Aplicaci\'{o}n Diferenciable.}
\begin{defi}
Sean $M$ y $N$ dos variedades diferenciables, $f\in\mathcal{F}(M,N)$ y $p\in M$. Se llama {\bf Diferencial} de $f$
en $p$ a la aplicaci\'{o}n $f_{*p}:T_p(M)\fl T_{f(p)}(N)$ definida de la siguiente forma: si $u\in T_p(M)$ es
cualquier vector tangente a $M$ en $p$ y $\a:(a,b)\fl M$ es una curva diferenciable que pasa por $p$ y cuyo vector
tangente en $p$ es $u$ (es decir, si existe $t_0\in (a,b)$ con $\a(t_0)=p$ y $\a'(t_0)=u$), entonces
$f_{*p}u=(f\circ\a)'(t_0)$.
\end{defi}
Obs\'{e}rvese que, desde luego, $f_{*p}u$ es un vector tangente a $N$ en $f(p)$, pues $f\circ\a:(a,b)\fl N$ es una
curva diferenciable en $N$, que pasa por $f(p)$, ya que $f(\a(t_0))=f(p)$ y, por tanto, $(f\circ\a)'(t_0)\in
T_{f(p)}(N)$. Adem\'{a}s, la definici\'{o}n anterior no depende de la curva $\a$ elegida pasando por $p$ y con vector
tangente en $p$ igual a $u$. En efecto, si $\b:(c,d)\fl M$ es otra curva en las mismas condiciones, entonces
existe $\tau_0\in(c,d)$ tal que $\b(\tau_0)=p$ y $\b'(\tau_0)=u$. Es claro que $f\circ\b:(c,d)\fl N$ es una curva
diferenciable que pasa por $f(p)$, ya que $(f\circ\b)(\tau_0)=f(p)$ y su vector tangente en $f(p)$ es $f_{*p}u$,
pues dada cualquier $g\in\mathcal{F}(f(p))$, se tiene que:
$$(f\circ\b)'(\tau_0)g=\left.\dderi{(g\circ(f\circ\b))}{\tau}\right|_{\tau=\tau_0}=
\left.\dderi{((g\circ f)\circ\b)}{\tau}\right|_{\tau=\tau_0}=$$
$$=\b'(\tau_0)(g\circ f)=u(g\circ f)=\a'(t_0)(g\circ f)=$$
$$=\left.\dderi{((g\circ f)\circ\a)}{t}\right|_{t=t_0}=\left.\dderi{(g\circ(f\circ\a))}{t}\right|_{t=t_0}=
(f\circ\a)'(t_0)g=(f_{*p}u)g.$$
\begin{prop}
Si $f\in\mathcal{F}(M,N)$, $p\in M$, $u\in T_p(M)$ y $g\in\mathcal{F}(f(p))$, entonces se verifica:
$$(f_{*p}u)g=u(g\circ f).$$
\end{prop}
\begin{prop}
{\bf (Propiedades de la Diferencial).}
\begin{enumerate}
\item {\bf (Regla de la Cadena).} Sean $f\in\mathcal{F}(M,N)$ y $g\in\mathcal{F}(N,P)$. Entonces, para
cualquier $p\in M$ se verifica que $(g\circ f)_{*p}=g_{*f(p)}\circ f_{*p}$.
\item Dado $p\in M$, entonces $({\rm Id}_M)_{*p}={\rm Id}_{T_p(M)}$.
\item Si $f,g\in\mathcal{F}(M,N)$ y $f\equiv g$ en un entorno de $p\in M$, entonces $f_{*p}=g_{*p}$.
\end{enumerate}
\end{prop}
\begin{prop}
La diferencial en cada punto de una aplicaci\'{o}n diferenciable es una aplicaci\'{o}n lineal.
\end{prop}
En consecuencia, cada diferencial tiene asociada una matriz. Para calcularla, sean $(U,\vp=(x_1,\dots ,x_m))$ un
s.l.c. de $M$ entorno de $p$ y $(V,\psi=(y_1,\dots ,y_n))$ un s.l.c. de $N$ entorno de $f(p)$. Para estas cartas
locales, las bases de $T_p(M)$ y $T_{f(p)}(N)$ son
$$\left\{\left(\dep{x_1}\right)_p,\dots ,\left(\dep{x_m}\right)_p\right\}\mbox{ }{\rm y}\mbox{
}\left\{\left(\dep{y_1}\right)_{f(p)},\dots ,\left(\dep{y_n}\right)_{f(p)}\right\},$$ respectivamente. Entonces,
como
$$f_{*p}\left(\dep{x_i}\right)_p=\sum_{j=1}^n\left[f_{*p}\left(\dep{x_i}\right)_p(y_j)\right]\left(\dep{y_j}
\right)_{f(p)}=$$ $$=\sum_{j=1}^n\left[\left(\ddep{(y_j\circ f)}{x_i}\right)_p\right]\left(\dep{y_j}
\right)_{f(p)},$$ la matriz de $f_{*p}$ respecto de estas bases viene dada por
$$f_{*p}=\left[\left(\ddep{y_j\circ f}{x_i}\right)_p\right],$$
que es la matriz jacobiana de la aplicaci\'{o}n $\psi\circ f\circ\vp^{-1}:\vp(U\cap f^{-1}(V))\fl\psi(f(U\cap
f^{-1}(V)))$.
\begin{defi}
La matriz anterior se denomina {\bf Matriz Jacobiana} de $f$ en $p$ respecto de las cartas dadas. En el caso en
que ${\rm dim}(M)={\rm dim}(N)$, el determinante de esta matriz se llama {\bf Jacobiano} de $f$ en $p$ y se denota
por $J_p(f)$.
\end{defi}
\begin{prop} Si $f\in\mathcal{F}(M,N)$, siendo $M$ conexa y $f_{*p}=0$, para cual\-quier $p\in M$, entonces $f$
es constante.
\end{prop}
\begin{defi}
Una aplicaci\'{o}n $f\in\mathcal{F}(M,N)$ se dice que es {\bf No Singular} en $p\in M$ si $f_{*p}$ es inyectiva, es
decir, si $\ker f_{*p}=0$. Al rango (dimensi\'{o}n de la imagen) de $f_{*p}$ se le denomina {\bf Rango} de $f$ en $p$.
El punto $p$ se dice {\bf Regular} para $f$ si $f_{*p}$ es sobreyectiva. En caso contrario, $p$ se dice {\bf
Cr\'{\i}tico}. Si $p$ es cr\'{\i}tico para $f$, el punto $q=f(p)$ se denomina {\bf Valor Cr\'{\i}tico}. De lo contrario se llama
{\bf Valor Regular}.
\end{defi}
Obs\'{e}rvese que si ${\rm dim}(M)={\rm dim}(N)$, entonces $p\in M$ es regular para $f$ si y s\'{o}lo si $f_{*p}$ es un
isomorfismo, es decir, si y s\'{o}lo si $J_p(f)\neq 0$.
\begin{defi}
Una aplicaci\'{o}n $f\in\mathcal{F}(M,N)$ se dice que es una {\bf Inmersi\'{o}n} si es no singular en todos los puntos de
$M$, es decir, si su diferencial en cada punto es una aplicaci\'{o}n inyectiva y se dice que es una {\bf Sumersi\'{o}n} si
todos los puntos de $M$ son regulares para $f$, es decir, si su diferencial en cada punto es una aplicaci\'{o}n
sobreyectiva.
\end{defi}
\begin{defi}
Sea $f\in\mathcal{F}(M,N)$ y $p\in M$. Se llama {\bf Aplicaci\'{o}n Co\-di\-fe\-ren\-cial} de $f$ en $p$ a la
aplicaci\'{o}n dual de $f_{*p}$, denotada por $f_p^*$ y dada por
$$f_p^*:T_{f(p)}^*(N)\fl T_p^*(M):\om\mapsto f_p^*\om$$
tal que, para cualquier $u\in T_p(M)$, $(f_p^*\om)u=\om(f_{*p}u)$.
\end{defi}
\begin{prop}
{\bf (Regla de la Cadena de la Codiferencial).} Sean $f\in\mathcal{F}(M,N)$, $g\in\mathcal{F}(N,P)$ y $p\in M$.
Entonces, se verifica que $(g\circ f)_p^*=f_p^*\circ g_{f(p)}^*$.
\end{prop}
\begin{prop}
Sean $f\in\mathcal{F}(M,N)$ y $p\in M$. Entonces, se verifican:
\begin{enumerate}
\item $f_{*p}$ es inyectiva si y s\'{o}lo si $f_p^*$ es sobreyectiva.
\item $f_{*p}$ es sobreyectiva si y s\'{o}lo si $f_p^*$ es inyectiva.
\end{enumerate}
\end{prop}

\newpage

\section{Noci\'{o}n de Subvariedad.}
\hs La mayor\'{\i}a de los ejemplos familiares de variedades
diferenciables son subconjuntos de otras variedades (puede
pensarse en los abiertos de una variedad diferenciable, las
superficies re\-gu\-la\-res en $\R^3$ o la figura ocho en $\R^2$).
Cabe preguntarse cu\'{a}ndo la estructura diferenciable del
subconjunto viene ``inducida" de alguna manera por la de la
variedad ambiente y cu\'{a}l es la relaci\'{o}n entre ambas
estructuras. Esta relaci\'{o}n tiene que ver con el hecho de que
el espacio tangente a la variedad subconjunto sea subespacio del
espacio tangente de la variedad ambiente en cada punto. Aparece
as\'{\i} la noci\'{o}n de subvariedad.
\begin{defi}
Una {\bf Subvariedad} de una variedad diferenciable $N$ es un par $(M,f)$, donde $M$ es otra variedad
diferenciable y $f\in\mathcal{F}(M,N)$ es una inmersi\'{o}n inyectiva.
\end{defi}
En los ejemplos antes mencionados (subvariedades abiertas,
superficies re\-gu\-la\-res o la figura ocho con cualquiera de las
dos estructuras conocidas), la inmersi\'{o}n inyectiva que las
convierte en subvariedad es la inclusi\'{o}n. Sin embargo, esto no
siempre ocurre. Por ejemplo, consid\'{e}rese el intervalo abierto
$(-1,1)$ dotado de la estructura diferenciable dada por la carta
global $\vp(t)=t^3$. Se tiene que $(-1,1)$ es un subconjunto de
$\R$ con la estructura eucl\'{\i}dea y la inclusi\'{o}n
$i:(-1,1)\hookrightarrow\R$ no es una inmersi\'{o}n, pues ni
siquiera es diferenciable. Con todo, el par $((-1,1),f)$, con
$f:(-1,1)\fl\R$ la inmersi\'{o}n $f(x)=x^3$ s\'{\i} es una
subvariedad de $\R$.
\begin{prop}\label{prop}
Sea $(M,f)$ una subvariedad de la variedad diferenciable $N$.
Entonces, puede dotarse a $f(M)$ de una estructura de variedad
diferenciable tal que $(f(M),i)$, con $i:f(M)\hookrightarrow N$ la
inclusi\'{o}n, es una subvariedad de $N$.
\end{prop}
Obs\'{e}rvese que, en el resultado anterior, la topolog\'{\i}a de $f(M)$ es, en general, m\'{a}s fina que la topolog\'{\i}a relativa
inducida por la de $N$, pues la inclusi\'{o}n, al ser diferenciable, es continua. Esto lleva a pensar que la noci\'{o}n de
subvariedad es m\'{a}s sutil que la an\'{a}loga para espacios topol\'{o}gicos. En efecto, como ya se ha dicho, el ocho con
cualquiera de las dos estructuras conocidas es subvariedad de $\R^2$ con la inclusi\'{o}n como inmersi\'{o}n inyectiva y,
sin embargo, no es subespacio topol\'{o}gico de $\R^2$ en ninguno de los dos casos.
\begin{defi}
Se dice que una aplicaci\'{o}n $f\in\mathcal{F}(M,N)$ es una {\bf In\-crus\-ta\-ci\'{o}n} si es un homeomorfismo sobre su
imagen, dotada \'{e}sta de la topolog\'{\i}a relativa de $N$, es decir, si $f:M\fl f(M)\subseteq N$ (dando a $f(M)$ la
topolog\'{\i}a relativa de $N$) es una aplicaci\'{o}n biyectiva, continua y abierta o cerrada. \end{defi}
\begin{defi}
Una subvariedad $(M,f)$ de una variedad diferenciable $N$ se dice que es una {\bf Subvariedad Regular} si $f$ es,
adem\'{a}s, una incrustaci\'{o}n.
\end{defi}
Por tanto, para subvariedades regulares, las im\'{a}genes de \'{e}stas por la inmersi\'{o}n, dotadas de la estructura dada por
la Proposici\'{o}n \ref{prop}, son tambi\'{e}n subespacios topol\'{o}gicos de la variedad ambiente. En el caso particular de
considerar subvariedades que sean subconjuntos con la inclusi\'{o}n, se obtiene:
\begin{prop}
Sean $M\subseteq N$ dos variedades diferenciables tales que, si $i:M\hookrightarrow N$ es la inclusi\'{o}n, $(M,i)$ es
subvariedad de $N$. Entonces, $(M,i)$ es subvariedad regular de $N$ si y s\'{o}lo si $M$ tiene la topolog\'{\i}a relativa
de $N$.
\end{prop}
\section{El Teorema de la Funci\'{o}n Inversa.}
\hs Dada una subvariedad $(M,f)$ de una variedad diferenciable
$N$, si ${\rm dim}(M)={\rm dim}(N)$, la diferencial de $f$ en cada
punto de $M$ es un isomorfismo de espacios vectoriales. Este hecho
ocurre, en particular, si $f$ es un difeomorfismo,
verific\'{a}ndose que dos variedades diferenciables difeomorfas
son, cada una de ellas, subvariedad de la otra con inmersi\'{o}n
el difeomorfismo o su inverso, seg\'{u}n el caso.
\par Cabe preguntarse, entonces, por el rec\'{\i}proco de este resultado: Dada una subvariedad de una variedad
diferenciable, ambas de la misma dimensi\'{o}n, ?`son difeomorfas
a trav\'{e}s de la inmersi\'{o}n? M\'{a}s en general, si una
aplicaci\'{o}n diferenciable entre variedades diferenciables
cumple que su diferencial en cada punto es un isomorfismo, ?`es un
difeomorfismo? La respuesta es negativa, pues basta considerar una
subvariedad abierta de una variedad diferenciable (aqu\'{\i} la
inmersi\'{o}n es la inclusi\'{o}n, que no es un difeomorfismo, al
no ser sobreyectiva).
\par Sin embargo, s\'{\i} puede probarse, mediante el Teorema de la Funci\'{o}n Inversa en Variedades, que una tal aplicaci\'{o}n
es un difeomorfismo local. De este teorema se deducen importantes
colorarios sobre las carta locales de ambas variedades. Por otra
parte, la condici\'{o}n adicional que debe impornerse a la
aplicaci\'{o}n para que s\'{\i} sea un difeomorfismo es la
biyectividad.
\begin{teorema}
{\bf (Teorema de la Funci\'{o}n Inversa en $\R^m$).} Sean
$G\subseteq\R^m$ un abierto, $f:G\fl\R^m$ una aplicaci\'{o}n
diferenciable y $p\in G$ tal que
$$\left|\left(\ddep{(u_i\circ f)}{u_j}\right)_{p}\right|\neq 0.$$
Entonces, existe un abierto $U$, con $p\in U\subseteq G$, tal que
$f(U)$ es abierto de $\R^m$ y $f|_U:U\fl f(U)$ es un
difeomorfismo.
\end{teorema}
\begin{teorema}
{\bf (Teorema de la Funci\'{o}n Inversa en Variedades).} Sean
$f\in\mathcal{F}(M,N)$ y $p\in M$ tal que $f_{*p}$ es un
isomorfismo. Entonces, existe un abierto $U$, con $p\in U\subseteq
M$, tal que $f(U)$ es abierto de $N$ y $f|_U:U\fl f(U)$ es un
difeomorfismo.
\end{teorema}
\begin{coro}
Si $f\in\mathcal{F}(M,N)$ verifica que $f_{*p}$ es un isomorfismo
para todo $p\in M$, entonces $f$ es un difeomorfismo local.
\end{coro}
\begin{defi}
Un conjunto de funciones $\{f_1,\dots ,f_k\}$ diferenciables en
$p\in M$ se dice que es {\bf Independiente} en $p$ si sus
diferenciales $(\de{f_1})_p,\dots, (\de{f_k})_p$ son linealmente
independientes en $T_p^*(M)$.
\end{defi}
\begin{coro}
Sea $\{f_1,\dots ,f_m\}$ un conjunto de funciones independiente en
$p\in M$, con ${\rm dim}(M)=m$. Entonces, las funciones $f_1,\dots
,f_m$ son las funciones coordenadas de un s.l.c. en un entorno de
$p$.
\end{coro}
\begin{coro}
Sea $\{f_1,\dots ,f_k\}$ un conjunto de funciones independiente en
$p\in M$, con ${\rm dim}(M)=m>k$. Entonces, las funciones
$f_1,\dots ,f_k$ forman parte de un s.l.c. en un entorno de $p$.
\end{coro}
\begin{coro}
Sea $\{f_1,\dots ,f_k\}$ un conjunto de funciones diferenciables
en $p\in M$, $k>{\rm dim}(M)$, tales que $(\de{f_1})_p,\dots
,(\de{f_k})_p$ generen $T_p^*(M)$. Entonces, un subconjunto de
tales funciones forman un s.l.c. en un entorno de $p$.
\end{coro}
\begin{coro}
Sea $f\in\mathcal{F}(M,N)$ tal que $f_{*p}$ es sobreyectiva, para
cierto $p\in M$. Sea $(y_1,\dots ,y_n)$ un s.l.c. en un entorno de
$f(p)$. Entonces, las funciones \ $y_1\circ f,\dots ,y_n\circ f$
forman parte de un s.l.c. en un entorno de $p$. Si $f$ es una
sumersi\'{o}n, las parejas de cartas as\'{\i} obtenidas para cada
punto de $M$ se denominan {\bf Cartas Adaptadas a la
Sumersi\'{o}n}.
\end{coro}
\begin{coro}
Sea $f\in\mathcal{F}(M,N)$ tal que $f_{*p}$ es inyectiva, para
cierto $p\in M$. Sea $(y_1,\dots ,y_n)$ un s.l.c. en un entorno de
$f(p)$. Entonces, un subconjunto de las funciones $y_1\circ
f,\dots ,y_n\circ f$ forman un s.l.c. en un entorno de $p$, siendo
$f$ inyectiva en este entorno. Si $f$ es una inmersi\'{o}n, las
parejas de cartas as\'{\i} obtenidas para cada punto de $M$ se
denominan {\bf Cartas Adaptadas a la Inmersi\'{o}n}.
\end{coro}
Una cuesti\'{o}n que surge de manera natural es averiguar qu\'{e}
condiciones debe cumplir una aplicaci\'{o}n $f$ tal que su
diferencial en cada punto es un isomorfismo para ser un
difeomorfismo. Dicha condici\'{o}n es la biyectividad, ya que, al
ser la diferencial de $f$ un isomorfismo en cada punto, $f$ es un
difeomorfismo local. Como es biyectiva, $f^{-1}$ es diferenciable
en cada punto de $N$, es decir, $f$ es un difeomorfismo. Sin
embargo, para variedades $2^{\underline{o}}N$, la condici\'{o}n se
puede debilitar algo. Para ello, ser\'{a}n necesarios algunos
resultados de Teor\'{\i}a de la Medida en variedades.
\par
Recu\'{e}rdese que un subconjunto $A\subseteq\R^m$ se dice que
tiene {\it Medida (de Lebesgue) Nula} si dado $\epsilon>0$, existe
$\{Q_n\}_{n\in\N}$, familia numerable de cubos, tal que:
$$A\subseteq\bigcup_{n\in\N}Q_i\mbox{ }y\mbox{ }\sum_{n\in\N}{\rm vol}(Q_n)<\epsilon.$$
\hs En estas condiciones se verifican las siguientes propiedades:
\begin{enumerate}
\item Cualquier subconjunto de un conjunto de medida nula es, a su
vez, de medida nula. \item La uni\'{o}n numerable de conjuntos de
medida nula en $\R^m$ es un conjunto de medida nula. \item La
imagen por una aplicaci\'{o}n de clase $\mathcal{C}^1$ de $\R^m$
en $\R^m$ de un conjunto de medida nula es un conjunto de medidad
nula. \item $\R^n\times\{x_0\}$, con $x_0\in\R^{m-n}$, es de
medida nula en $\R^m$.
\end{enumerate}
\begin{defi}
Sea $M$ una variedad diferenciable y $A\subseteq M$. Se dice que
$A$ tiene {\bf Medida (de Lebesgue) Nula} si para toda familia de
cartas locales $\{(U_i,\vp_i)\}_{i\in I}$ cuyos dominios recubran
a $A$, se cumple que $\vp_i(A\cap U_i)$ es de medida nula en
$\R^m$, para todo $i\in I$.
\end{defi}
\begin{prop}
Sean $M$ y $N$ dos variedades diferenciables y
$f\in\mathcal{F}(M,N)$. Entonces:
\begin{enumerate}
\item Si $A\subseteq M$ es un conjunto de medida nula en $M$, se
cumple que $f(A)$ es un conjunto de medida nula en $N$. \item Si
$m<n$, se cumple que $f(M)$ es un conjunto de medida nula en $N$.
\end{enumerate}
\end{prop}
\hs Se puede ahora probar el resultado buscado. Debe resaltarse
que en todo el desarrollo anterior se ha usado que $M$ verifica el
axioma $2^{\underline{o}}N$, por lo que la si\-guien\-te
proposici\'{o}n y su corolario s\'{o}lo son v\'{a}lidos para
variedades diferenciables $2^{\underline{o}}N$.
\begin{prop}
Sea $f\in\mathcal{F}(M,N)$ una aplicaci\'{o}n sobreyectiva y sea
$p\in M$ tal que $f_{*p}$ es inyectiva. Entonces, $f_{*p}$ es un
isomorfismo.
\end{prop}
\begin{coro}
Sea $f\in\mathcal{F}(M,N)$ una inmersi\'{o}n biyectiva. Entonces,
$f$ es un difeomorfismo
\end{coro}
\section{Factorizaci\'{o}n de Aplicaciones. Unicidad de Subvariedades.}
\hs Dada una variedad diferenciable $N$, una subvariedad de $N$ es un par $(M,f)$ donde $M$ es otra variedad
diferenciable y $f\in\esp{F}(M,N)$ es una inmersi\'{o}n inyectiva, es decir, es una aplicaci\'{o}n diferenciable,
inyectiva y tal que su diferencial es inyectiva en cada punto de $M$. Esto quiere decir que una misma variedad
diferenciable tiene tantas subvariedades como inmersiones puedan establecerse a ella desde otras variedades
diferenciables. En particular, todas las variedades difeomorfas a una dada son subvariedades de ella y debe
recordarse que cualquier conjunto biyectivo con una variedad diferenciable puede dotarse de una estructura de
variedad diferenciable que convierte a la biyecci\'{o}n en un difeomorfismo. Por tanto, se hace necesario ``controlar"
de alguna forma la famila de las subvariedades de una variedad diferenciable.
\begin{defi}
Dos subvariedades $(M_1,f_1)$ y $(M_2,f_2)$ de una variedad $N$ se
dice que son {\bf Equivalentes} si existe un difeomorfismo
$f:M_1\fl M_2$ tal que $f_1=f_2\circ f$.
\end{defi}
$$\begin{diagram}
\node{M_1} \arrow{e,t}{f_1} \arrow{se,b}{f} \node{N} \\
\node[2]{M_2} \arrow{n,r}{f_2}
\end{diagram}$$
\hs Puede comprobarse sin dificultad que \'{e}sta es una
relaci\'{o}n de equivalencia en la clase de todas las
subvariedades de $N$, que permite trabajar con m\'{a}s comodidad
con dicha clase. Por tanto, es conveniente dar condiciones para la
existencia de tal difeomorfismo. Para ello, se plantear\'{a} un
problema m\'{a}s general.

Sea $f\in\mathcal{F}(M,N)$ una aplicaci\'{o}n diferenciable entre
dos variedades diferenciables y sea $(P,g)$ una subvariedad de
$N$. Como $g$ es inyectiva, si $f$ {\it factoriza a trav\'{e}s de}
$g$, es decir si $f(M)\subseteq g(P)$, entonces se puede definir
una \'{u}nica aplicaci\'{o}n $f_0:M\fl P$ tal que $g\circ f_0=f$.
No siempre esta aplicaci\'{o}n es diferenciable.
$$\begin{diagram}
\node{M} \arrow{e,t}{f} \arrow{se,b,..}{f_0} \node{N} \\
\node[2]{P} \arrow{n,r}{g}
\end{diagram}$$
\begin{teorema}
{\bf (Lema de Factorizaci\'{o}n).} Sea $f\in\mathcal{F}(M,N)$ una
aplicaci\'{o}n diferenciable entre dos variedades diferenciables y
sea $(P,g)$ una subvariedad de $N$ tal que $f$ factoriza a
trav\'{e}s de $g$. Entonces:
\begin{enumerate}
\item Si $f_0$ es continua, entonces es diferenciable. \item Si
$(P,g)$ es una subvariedad regular de $N$, entonces $f_0$ es
continua.
\end{enumerate}
\end{teorema}
\begin{prop}
Cada clase de equivalencia de subvariedades de una variedad
diferenciable $N$ tiene un \'{u}nico representante de la forma
$(A,i)$ donde $A$ es un subconjunto de $N$ con estructura de
variedad diferenciable y la inclusi\'{o}n $i:A\hookrightarrow N$
es una inmersi\'{o}n.
\end{prop}
\hs Es posible completar a\'{u}n m\'{a}s la proposici\'{o}n
anterior, pues tambi\'{e}n se tiene que la estructura de la que se
ha dotado a $f(N)$ es la \'{u}nica posible con la propiedad de que
la subvariedad $(f(N),i)$ sea equivalente a $(N,f)$. De hecho, si
existiesen en $f(N)$ dos estructuras diferenciables
$\mathcal{A}_1$ y $\mathcal{A}_2$ tales que, con ambas, $(f(N),i)$
es una subvariedad de $M$ equivalente a $(N,f)$, esto es, si
existiesen dos difeomorfismos $g_1:N\fl(f(N),\mathcal{A}_1)$ y
$g_2:N\fl(f(N),\mathcal{A}_2)$ tales que $f=i\circ g_1=i\circ
g_2$, entonces $g_2\circ
g_1^{-1}:(f(N),\mathcal{A}_1)\fl(f(N),\mathcal{A}_2)$ es un
difeomorfismo, por composici\'{o}n de difeomorfismos y,
adem\'{a}s, es la identidad, por su construcci\'{o}n. De aqu\'{\i}
se deduce que las dos estructuras dadas en $f(N)$ son compatibles
y, por tanto, la unicidad buscada.
\par
\hs Es habitual afirmar que existe una \'{u}nica subvariedad de
una variedad diferenciable $N$ cumpliendo cierta propiedad. Esta
unicidad se entender\'{a} salvo la equivalencia definida
anteriormente. En particular, si las subvariedades de $N$ se
consideran como subconjuntos con estructura de variedad
diferenciable tal que la inclusi\'{o}n correspondiente es una
inmersi\'{o}n, la unicidad significa \'{u}nico subconjunto con
\'{u}nica topolog\'{\i}a 2$^{\underline{o}}N$ y \'{u}nica
estructura diferenciable. En general, dado un subconjunto
$A\subseteq N$, no existe una \'{u}nica estructura diferenciable
sobre $A$ tal que $(A,i)$ sea una subvariedad de $N$, si es que
existe alguna. No obstante, se tienen dos teoremas de unicidad que
imponen condiciones a la topolog\'{\i}a de $A$ para tal
situaci\'{o}n
\begin{teorema}
{\bf (Primer Teorema de Unicidad).} Sean $N$ una variedad
diferenciable y $A\subseteq N$ un subconjunto. Fijada una
topolog\'{\i}a sobre $A$, existe, a lo m\'{a}s, una estructura
diferenciable en $A$ tal que $(A,i)$ es una subvariedad de $N$.
\end{teorema}
\begin{teorema}
{\bf (Segundo Teorema de Unicidad).} Sean $N$ una variedad
diferenciable y $A\subseteq N$ un subconjunto. Si con la
topolog\'{\i}a relativa $A$ tiene una estructura diferenciable tal
que $(A,i)$ es una subvariedad de $N$, entonces $A$ tiene
estructura de variedad diferenciable \'{u}nica, es decir, que para
que $A$ sea subvariedad de $N$ tiene que ser, precisamente, con la
topolog\'{\i}a relativa y con la misma estructura diferenciable.
\end{teorema}
\section{Teorema de la Funci\'{o}n Impl\'{\i}cita. Teorema de Whitney.}
\begin{teorema}
{\bf (Teorema de la Funci\'{o}n Impl\'{\i}cita).} Sean
$f\in\mathcal{F}(M,N)$ una aplicaci\'{o}n diferenciable entre dos
variedades diferenciables, $q\in N$ y
$P=f^{-1}(\{q\})\neq\emptyset$. Si $f_{*p}$ es sobreyectiva, para
cualquier $p\in P$, entonces $P$ tiene una \'{u}nica estructura de
variedad diferenciable tal que $(P,i)$ es una subvariedad regular
de $M$ de dimensi\'{o}n ${\rm dim}(M)-{\rm dim}(N)$.
\end{teorema}
\begin{coro} Sea $f:\R^m\fl\R^n$ una aplicaci\'{o}n
diferenciable. Si el conjunto $P=\{p\in\R^m/f(p)=0\}$ es no
vac\'{\i}o y el rango de la matriz jacobiana de $f$ vale $n$ en
todos los puntos de $P$, entonces $P$ es una subvariedad regular
de $\R^m$ de dimensi\'{o}n $m-n$.
\end{coro}

\newpage

El Teorema de la Funci\'on Impl\'{\i}cita puede generalizarse.
Recordemos que, dados $f\in \mathcal{F}(M,N)$ y $p \in M$, se
llama {\it rango} de $f$ en $p$ al rango (dimensi\'on de la
imagen) de $f_{*p}$. As\'{\i}, se tiene:

\begin{teorema}
{\bf (Teorema del Rango o de la Subvariedad de Nivel).} Sean
$f\in\mathcal{F}(M,N)$ una aplicaci\'{o}n diferenciable entre dos
variedades diferenciables, $q\in N$ y
$P=f^{-1}(\{q\})\neq\emptyset$. Si $f$ tiene rango constante $k$
en un entorno abierto de cada $p\in P$, entonces $P$ tiene una
\'{u}nica estructura de variedad diferenciable tal que $(P,i)$ es
una subvariedad regular cerrada de $M$ de codimensi\'{o}n $k$.
\end{teorema}

\begin{teorema}
Toda variedad diferenciable compacta es difeomorfa a una subvariedad cerrada de un espacio eucl\'{\i}deo de dimensi\'{o}n
suficientemente grande, es decir, puede ser incrustada en un espacio eucl\'{\i}deo.
\end{teorema}
\begin{teorema}
{\bf (Teorema de Whitney).} Toda variedad diferenciable paracompacta $m$--dimensional es difeomorfa a una
subvariedad cerrada de un espacio af\'{\i}n $(2m+1)$--dimensional.
\end{teorema}



\section{Ejercicios.}
\begin{enumerate}
\item Probar que la diferencial de una aplicaci\'{o}n $f:{\bf R}^m
\longrightarrow {\bf R}^n$, considerada como aplicaci\'{o}n entre
variedades diferenciables, coincide con la diferencial habitual de
${\bf R}^n$. \item Probar que si $f\in\esp{F}(M,N)$ es un
difeomorfismo, entonces su diferencial en cada punto es un
isomorfismo. \item Probar que el rango de la diferencial de una
aplicaci\'{o}n diferenciable entre dos variedadades diferenciables
no depende de las cartas tomadas. \item Probar que las
proyecciones de un producto de variedades diferenciables en cada
uno de los factores son sumersiones. \item Sean
$f\in\esp{F}(M,N)$, $p\in M$ y $g\in\esp{F}(f(p))$. Probar que
$$f_p^*((\dif g)_{f(p)})=(\dif (g\circ f))_p.$$ \item Probar que
toda superficie regular es una subvariedad regular de $\R^3$.
\item Probar que cualquier subvariedad abierta de una variedad
diferenciable es subvariedad regular. \item Probar que $S^n$ es
subvariedad regular de $\R^{n+1}$.
\item Sea $M$ una variedad diferenciable y $p\in M$. Si
$\{\om_1,\dots ,\om_m\}$ es una base de $T_p^*(M)$, probar que
existe una carta local $(U,\vp=(x_1,\dots ,x_m))$ entorno de $p$
tal que $(\de{x_i})_p=\om_i$, $i=1,\dots ,m$. \item Probar que
no existen inmersiones de $S^1$ en $\R$. \item Sea
$f\in\mathcal{F}(M,N)$ un difeomorfismo local inyectivo entre dos
variedades diferenciables. Probar que $f$ es un difeomorfismo de
$M$ en un cierto abierto de $N$. \item Sean $M$ subvariedad de
$\bar{M}$ y $N$ subvariedad de $\bar{N}$. Sea $f: \bar{M}
\longrightarrow \bar{N}$ una aplicaci\'{o}n diferenciable. ?`Bajo
qu\'{e} condiciones m\'{\i}nimas se puede asegurar que $f|_M:M
\longrightarrow N$ es diferenciable? \item Probar que la
relaci\'{o}n ``ser equivalentes" entre subvariedades de una
variedad diferenciable es una relaci\'{o}n de equivalencia.
\item Sean $M\subseteq N$ tales que $(M,i)$ es subvariedad de $N$. Probar que $M$ y $N$ tienen la misma dimensi\'on si y s\'olo si $M$ es subvariedad abierta de $N$.
\item
?`Para qu\'{e} valores de $a\in\R$ es el conjunto
$$M=\{(x,y,z)\in\R^3/x^2+y^2-z+2=0;z=a\}$$
una subvariedad regular de dimensi\'{o}n 1 de $\R^3$?
\end{enumerate}
\end{document}