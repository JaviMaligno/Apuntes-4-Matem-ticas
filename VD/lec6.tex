\documentclass[cursovd_portada.tex]{subfiles}

\begin{document}

\chapter{Campos de Tensores\\ Covariantes Diferenciables sobre una Va\-rie\-dad Diferenciable}
\section{Campos de Tensores Covariantes Diferenciables.}
\begin{defi}
Dada una variedad diferenciable $M$ y un abierto $U\subseteq M$, un {\bf Campo de Tensores $r$--Covariantes
Diferenciable} sobre $U$ es una aplicación $r$--lineal del $\esp{F}(U)$--módulo $\esp{X}(U)$ en $\esp{F}(U)$, es
decir, una aplicación
$$T:\esp{X}(U)\times\stackrel{r)}\cdots\times\mathcal{X}(U)\fl\mathcal{F}(U)$$
$r$-lineal (con escalares las funciones diferenciables sobre $U$). Al conjunto de los campos de tensores
$r$-covariantes diferenciables sobre $U$ se denota por $T_r(U)$. Se conviene que $T_0(U)=\mathcal{F}(U)$.
\end{defi}
\begin{prop}
Sean $T\in T_r(U)$ y $X_1,\dots ,X_r\in\mathcal{X}(U)$. Se verifican:
\begin{enumerate}
\item Si $V\subseteq U$ es un abierto e $i\in\{1,\dots ,r\}$ es tal que $X_i|_V=0$, entonces $T(X_1,\dots ,X_r)=0$
en $V$.
\item Si $p\in U$ e $i\in\{1,\dots ,r\}$ es tal que $X_i(p)=0$, entonces $T(X_1,\dots ,X_r)(p)=0$.
\item Si $Y_1,\dots ,Y_r\in\mathcal{X}(U)$ verifican que $Y_i(p)=X_i(p)$, para cierto $p\in U$ y para cualquier
$i\in\{1,\dots ,r\}$, entonces $T(X_1,\dots ,X_r)(p)=T(Y_1,\dots ,Y_r)(p)$.
\end{enumerate}
\end{prop}
\begin{nota}\label{tensores}
{\rm El nombre de campo de tensores $r$--covariantes proviene del
hecho de que un tal objeto puede también interpretarse como
una aplicación $T$ que a cada punto $p\in U$ le hace
corresponder un tensor $r$ veces covariante\footnote{Para obtener
más información sobre tensores puede consultarse el
Apéndice B del temario.} sobre $T_p(M)$, es decir, una
aplicación
$$T_p:T_p(M)\times\stackrel{r)}\cdots\times T_p(M)\fl\R$$
$r$ veces lineal (con escalares los números reales), definida por
$$T_p(u_1,\dots ,u_r)=T(X_1,\dots ,X_r)(p),\mbox{ }u_1,\dots u_r\in T_p(M),$$
donde $X_1,\dots ,X_r$ son campos de vectores diferenciables sobre $U$ tales que $X_i(p)=u_i$, $i=1,\dots ,r$ (por
la proposición anterior, se sabe que el valor de $T(X_1,\dots ,X_r)(p)$ no depende de la elección de los campos
$X_1,\dots ,X_r$ en las condiciones requeridas). En este sentido, puede hablarse de restricciones de campos de
tensores covariantes diferenciables a  abiertos. En efecto, sea $T\in T_r(U)$ y $V\subseteq U$ un abierto. Se
define $T|_V$ por
$$T|_V(X_1,\dots X_r)(p)=T_p(X_1(p),\dots, X_r(p))=T(\overline{X}_1,\dots ,\overline{X}_r)(p),$$
para cualquier $p\in V$ y $X_1,\dots X_r\in\mathcal{X}(V)$, siendo
$\overline{X}_1,\dots ,\overline{X}_r$ campos de vectores
diferenciables en $U$, tales que $\overline{X}_i=X_i$ en un cierto
entorno de $p$ contenido en $V$, $i=1,\dots ,r$. Por otra parte,
es inmediato observar que, dados $T,S\in T_r(U)$, $T=S$ si y
sólo si $T_p=S_p$ para todo $p \in U$.}


\end{nota}
\begin{prop}Sea $M$ una variedad diferenciable y $U\subseteq M$ un abierto. Se verifican:
\begin{enumerate}
\item Si $T\in T_r(U)$ y $V\subseteq U$ es un abierto, entonces $T|_V\in T_r(V)$.
\item Si
$$U=\bigcup_{i\in I}U_i,$$
con $U_i$ abierto para todo $i\in I$ y $T$ es una aplicación $r$--lineal de
$\esp{X}(U)\times\stackrel{r)}\cdots\times\mathcal{X}(U)$ en el conjunto de las funciones de $U$ en $\R$, tal que
$T|_{U_i}\in T_r(U_i)$, para cualquier $i\in I$, entonces $T\in T_r(U)$.
\end{enumerate}
\end{prop}
\begin{ej}
{\rm Sea $f\in\mathcal{F}(U)$. Se define $\de{f}:\mathcal{X}(U)\fl\mathcal{F}(U)$ por $(\de{f})X=Xf$. Es fácil
comprobar que $\de{f}\in T_1(U)$ y se lee {\it diferencial de $f$}. El nombre y la notación se justifican en el
hecho de que, según la nota anterior, si $p\in U$, se tiene que $(\de{f})_p$ coincide con la diferencial de la
función $f$ en $p$, pues, dado $u\in T_p(M)$, $(\de{f})_pu=((\de{f})X)(p)=(Xf)(p)=X_pf=u(f)$, para cualquier campo
$X\in\mathcal{X}(U)$ tal que $X_p=u$, efectivamente.}
\end{ej}
En $T_r(U)$ se pueden definir las siguientes operaciones de modo natural: suma, producto por números reales y
producto por funciones diferenciables. Se prueba sin dificultad que, dados $T,S\in T_r(U)$, $\lambda\in\R$ y
$f\in\mathcal{F}(U)$, se tiene que $T+S,\lambda T,fT\in T_r(U)$. Además, dados $T\in T_r(U)$ y $S\in T_s(U)$, se
define la operación {\it Producto Tensorial} de $T$ y $S$, denotada por $T\otimes S$, mediante:
$$T\otimes S:\esp{X}(U)\times\stackrel{r+s)}\cdots\times\mathcal{X}(U)\fl\mathcal{F}(U):$$
$$(X_1,\dots ,X_{r+s})\mapsto(T\otimes S)(X_1,\dots ,X_{r+s})=T(X_1,\dots ,X_r)S(X_{r+1},\dots ,X_{r+s}).$$
\hs Se tiene que $T\otimes S\in T_{r+s}(U)$.
\par
En estas condiciones, se verifica que, si $r>0$, $T_r(U)$ con la suma y el producto por números reales es un
espacio vectorial real, con la suma y el producto por funciones diferenciables es un $\mathcal{F}(U)$--módulo y
que
$$T(U)=\bigoplus_{r=0}^{\infty}T_r(U)$$
es un álgebra asociativa, no conmutativa y con elemento unidad, llamada {\it álgebra Tensorial} de $U$.
\begin{prop}
Sea $(U,\vp=(x_1,\dots ,x_m))$ un s.l.c. en $M$. Entonces, la familia
$$\{\de{x_{i_1}}\otimes\cdots\otimes\de{x_{i_r}}/i_1,\dots ,i_r=1,\dots ,m\}$$
es una base de $T_r(U)$ como $\mathcal{F}(U)$--módulo. Además, si $T\in T_r(U)$, en función de esa base se tiene
que:
$$T=\sum_{i_1,\dots ,i_r=1}^mT\left(\dep{x_{i_1}},\dots
,\dep{x_{i_r}}\right)\de{x_{i_1}}\otimes\cdots\otimes\de{x_{i_r}}.$$
\end{prop}
\begin{dem}
Veamos que es sistema generador. Sean $X_1,\dots, X_r\in\mathcal{X}(U)$ cualesquiera. Tenemos $X_j=\sum_{i_j=1}^mX_jx_{i_j}\frac{\partial}{\partial x_{i_j}}\ \forall j=1,\dots,r$. Así
$$T(X_1,\dots,X_r)=T(\sum_{i_1=1}^mX_jx_{i_1}\frac{\partial}{\partial x_{i_1}},\dots,\sum_{i_r=1}^mX_jx_{i_r}\frac{\partial}{\partial x_{i_r}})=$$
$$\sum_{i_1,\dots i_r=1}^m(X_1x_{i_1})\dots(X_rx_{i_r})T(\frac{\partial}{\partial x_{i_1}},\dots,\frac{\partial}{\partial x_{i_r}})=\sum_{i_1,\dots i_r=1}^mdx_{i_1}(X_1)\dots dx_{i_r}(X_r)=T(\frac{\partial}{\partial x_{i_1}},\dots,\frac{\partial}{\partial x_{i_r}})=$$
$$\sum_{i_1,\dots i_r=1}^mdx_{i_1}\otimes\dots\otimes dx_{i_r}T(\frac{\partial}{\partial x_{i_1}},\dots,\frac{\partial}{\partial x_{i_r}})(X_1,\dots,X_r)$$
Ahora la independencia lineal. Sea $\sum_{i_1,\dots,i_r=0}^m f_{i_1,\dots, i_r}dx_{i_1}\otimes\dots\otimes dx_{i_r}=0$. Entonces 
$$\sum_{i_1,\dots,i_r=0}^m f_{i_1,\dots, i_r}dx_{i_1}\otimes\dots\otimes dx_{i_r}(\frac{\partial}{\partial x_{j_1}},\dots,\frac{\partial}{\partial x_{j_r}})=0$$
Pero $dx_{i_1}(\frac{\partial }{\partial x_{j_1}})=\delta_{i,j_1}$ y así. Entonces solo queda $f_{j_1,\dots,j_r}\neq 0$ para cada caso en el que todos valgan 1, por lo que todos los coeficientes son 0. $\QED$. 
\end{dem}

\begin{defi}
Dada una variedad diferenciable $M$ y un abierto $U\subseteq M$, un {\bf Campo de Tensores $r$--Covariantes y
1--Contravariantes Diferenciable} sobre $U$ es una aplicación $r$--lineal del $\esp{F}(U)$--módulo $\esp{X}(U)$ en
$\mathcal{X}(U)$, es decir, una aplicación
$$T:\esp{X}(U)\times\stackrel{r)}\cdots\times\mathcal{X}(U)\fl\mathcal{X}(U)$$
$r$--lineal (con escalares las funciones diferenciables sobre $U$). Al conjunto de los campos de tensores
$r$--covariantes y 1--contravariantes diferenciables sobre $U$ se denota por $T_r^1(U)$.
\end{defi}
\section{Formas Diferenciales.}
\begin{defi}
Dada una variedad diferenciable $M$ y un abierto $U\subseteq M$, una {\bf $r$--Forma Diferencial} sobre $U$ es un
campo de tensores $r$--covariantes diferenciable y alternado sobre $U$, es decir, una aplicación
$$\om:\esp{X}(U)\times\stackrel{r)}\cdots\times\mathcal{X}(U)\fl\mathcal{F}(U)$$
$r$--lineal (con escalares las funciones diferenciables sobre $U$) alternada, esto es, verificando que
$$\om(X_1,\dots ,X_r)=\varepsilon(\sigma)\om(X_{\sigma(1)},\dots ,X_{\sigma(r)}),$$
para cualquier $\sigma\in S_r$ y $X_1,\dots ,X_r\in\mathcal{X}(U)$. Al conjunto de las $r$-formas diferenciales
sobre $U$ se denota por $\Lambda_r(U)$. Se conviene que $\Lambda_0(U)=\mathcal{F}(U)$.
\end{defi}
\begin{nota}
{\rm Es fácil comprobar que $\Lambda_r(U)$ es un subespacio vectorial y un sub\-mó\-du\-lo de $T_r(U)$. Además,
$\Lambda_1(U)=T_1(U)$. }
\end{nota}
\begin{defi}
Dado $T\in T_r(U)$, se llama {\bf Alternado} de $T$ a la $r$--forma diferencial $Alt(T)$ definida mediante
$$Alt(T)(X_1,\dots ,X_r)=\frac{1}{r!}\sum_{\sigma\in S_r}\varepsilon(\sigma)T(X_{\sigma(1)},\dots
,X_{\sigma(r)}),$$ para cualesquiera $X_1,\dots ,X_r\in\mathcal{X}(U)$.
\end{defi}
\begin{ejer}
{\rm Probar que, efectivamente, $Alt(T)\in\Lambda_r(U)$, para cualquier $T\in T_r(U)$. Probar, además, que
la aplicación $Alt:T_r(U)\fl\Lambda_r(U)$ es lineal (con escalares las funciones diferenciables sobre $U$) y que
restringido a $\Lambda_r(U)$ es la identidad.}
\end{ejer}
Dadas $\om\in\Lambda_r(U)$ y $\theta\in\Lambda_s(U)$, se define la operación {\it Producto Exterior} de $\om$ y
$\theta$, denotada por $\om\wedge\theta$, mediante
$$\om\wedge\theta=\frac{(r+s)!}{r!s!}Alt(\om\otimes\theta),$$
es decir, si $X_1,\dots ,X_{r+s}\in\mathcal{X}(U)$:
$$(\om\wedge\theta)(X_1,\dots ,X_{r+s})=$$
$$=\frac{1}{r!s!}\sum_{\sigma\in S_{r+s}}\varepsilon(\sigma)\om(X_{\sigma(1)},\dots
,X_{\sigma(r)})\theta(X_{\sigma(r+1)},\dots ,X_{\sigma(r+s)}).$$ \hs Claramente,
$\om\wedge\theta\in\Lambda_{r+s}(U)$. Además, razonando por inducción, se tiene que si $\om_i\in\Lambda_{r_i}(U)$,
$i=1,\dots ,k$, entonces:
$$\om_1\wedge\cdots\wedge\om_k=\frac{(r_1+\cdots +r_k)|}{r_1!\dots r_k!}Alt(\om_1\otimes\cdots\otimes\om_k).$$
\begin{prop}
Sean $\om\in\Lambda_r(U)$ y $\theta\in\Lambda_s(U)$. Entonces $\om\wedge\theta=(-1)^{rs}\theta\wedge\om$. En
consecuencia, si $r$ es impar, $\om\wedge\om=0$.
\end{prop}
\begin{prop}
Sean $\om_1,\dots ,\om_r\in\Lambda_1(U)$. Entonces:
\begin{enumerate}
\item Para cualquier $\sigma\in S_r$, $\om_1\wedge\cdots\wedge\om_r=\varepsilon(\sigma)\om_{\sigma(1)}\wedge\cdots
\wedge\om_{\sigma(r)}$.
\item Dados cualesquiera $X_1,\dots ,X_r\in\mathcal{X}(U)$:
$$(\om_1\wedge\cdots\wedge\om_r)(X_1,\dots ,X_r)={\rm det}(\om_i(X_j)).$$
\end{enumerate}
\end{prop}
\begin{prop}
Sea $(U,\vp=(x_1,\dots ,x_m))$ un s.l.c. en $M$. Entonces, la familia
$$\{\de{x_{i_1}}\wedge\cdots\wedge\de{x_{i_r}}/1\leq i_1<\cdots <i_r\leq m\}$$
es una base de $\Lambda_r(U)$ como $\mathcal{F}(U)$--módulo. Además, si $\om\in\Lambda_r(U)$, en función de esa
base se tiene que:
$$\om=\sum_{1\leq i_1 <\cdots <i_r\leq m}\om\left(\dep{x_{i_1}},\dots
,\dep{x_{i_r}}\right)\de{x_{i_1}}\wedge\cdots\wedge\de{x_{i_r}}.$$
\end{prop}
\begin{coro}
Si $r>m$, entonces $\Lambda_r(U)=\{0\}$.
\end{coro}
\begin{prop}
Se verifica que
$$\Lambda(U)=\bigoplus_{r=0}^m\Lambda_r(U)$$
con el producto exterior es un álgebra asociativa, no conmutativa y con elemento unidad.
\end{prop}


\section{Pull-Back de un Campo de Tensores.}
\begin{defi}
Sean $M$ y $N$ dos variedades diferenciables y
$f\in\mathcal{F}(M,N)$. Dado $T\in T_r(N)$, se define $f^*T\in
T_r(M)$ por $f^*T=T\circ f$, si $r=0$ y $f^*T(X_1,\dots ,X_r)(p)=
T(\overline{X}_1,\dots ,\overline{X}_r)(f(p))$, si $r>0$, donde
$p$ es un punto de $M$, $X_1,\dots ,X_r\in\mathcal{X}(M)$ y
$\overline{X}_1,\dots ,\overline{X}_r\in\mathcal{X}(N)$ son tales
que $f_{*p}X_i(p)=\overline{X}_i(f(p))$, para todo $i=1,\dots ,r$.
De manera equivalente, se puede definir
$$(f^*T)_p(u_1,\dots,u_r)=T_{f(p)}(f_{*p}u_1,\dots,f_{*p}u_r),$$
para todo $p\in M$ y todos $u_1,\dots,u_r \in T_p(M)$.
\end{defi}
Esta definición permite, al ser extendida por linealidad, obtener una aplicación $f^*:T(N)\fl T(M)$, asociada a
$f$.
\begin{prop}
Dada $f\in\mathcal{F}(M,N)$, se verifican las siguientes
propiedades:
\begin{enumerate}
\item $f^*:T_r(N)\fl T_r(M)$ es $\R$--lineal, para todo $r\geq 0$.
\item $f^*(gT)=(g\circ f)f^*T$, para cualquier $T\in T(N)$ y
cualquier $g\in \mathcal{F}(N)$. \item $f^*(T\otimes
S)=f^*T\otimes f^*S$, para cualesquiera $T,S\in T(N)$. \item
$f^*(Alt(T))=Alt(f^*T)$, para cualquier $T\in T(N)$. \item
$f^*(\om\wedge\theta)=f^*\om\wedge f^*\theta$, para cualesquiera
$\om,\theta\in\Lambda(N)$. \item $f^*(\de{g})=\de{(f^*g)}$, para
cualquier $g\in\mathcal{F}(N)$.
\end{enumerate}
\end{prop}

\begin{nota}
Si $w\in\Lambda_1(N)$ y $p\in M$, entonces $(f^*w)_p=f^*_p(w_{f(p)})$.
\begin{proof}
$\forall u\in T_p(M)$, por definición de pull-back y de $f^*_p$,
$$(f^*w)_pu=w_{f(p)}(f_{*p}u)=)f^*pw_{f(p)})u.$$
\end{proof}
\end{nota}

\begin{ej}
Sea $S$ una superficie regular en $\R^3$. Consideramos $g=dx\otimes dx + dy\otimes dy + dz\otimes dz\in T_2(\R^3)$ (métrica euclídea).
Tenemos que la inclusión $i:S\hookrightarrow \R^3$ es diferenciable, luego podemos considerar $i^*g\in T_2(S)$. $\forall p\in S$, $(i^* g)_p(u_1,u_2)=g_{i(p)}(i_{*p}u_1,i_{*p}u_2)=(i_{*p}u_1)\cdot (i_{*p}u_2)$. En geometría local esto es precisamente la IFF de $u_1$ y $u_2$, luego $i^*g=I$. 
\end{ej}

\begin{nota}
En $\R^4$, además de la métrica euclídea, podemos introducir la métrica $g=dx\otimes dx + dy\otimes dy + dz\otimes dz-dt\otimes dt \in T_2(\R^4)$, que es la métrica de Lorentz-Minkowsky. Si $m_p(v,v)>0$ se llama vector espacial, si es $<0$ temporal y si es $=0$ luminoso. Esta métrica es semi-Riemanniana.

También podemos adaptarla a $\R^3$ cambiándo el signo de la $z$ a la euclídea. Si tomamos $\mathbb{H}^2=\{(x,y,z)\in\R^3\mid x^2+y^2-z^2=-1\}\subseteq (\R^3,m)$. Entonces $i:\mathbb{H}\hookrightarrow\R^3$ induce $h=i^*m$ que es una métrica Riemanniana asociada a la geometría hiperbólica. Con $S^2$ se induce $ \mathring{g}$, que se corresponde con la geometría esférica. 
\end{nota}

\newpage

\section{Ejercicios.}
\begin{enumerate}
\item Probar que la diferencial de cualquier función constante
es nula. \item Sea $(U,\vp=(x_1,\dots, x_m))$ un s.l.c. y
$f\in\mathcal{F}(U)$. Probar que:
$$\de{f}=\sum_{i=1}^m\ddep{f}{x_i}\de{x_i}.$$
\item Dadas $\om\in\Lambda_r(U)$ y $\theta\in\Lambda_s(U)$, probar
que $(\om\wedge\theta)_p=\omega_p\wedge\theta_p$ (para la
definición del segundo producto exterior, consúltese el
Apéndice C). \item Si $M$ y $N$ son dos variedades
diferenciables, $f\in\mathcal{F}(M,N)$, $X_1,\dots
,X_r\in\mathcal{X}(M)$, $Y_1,\dots ,Y_r\in\mathcal{X}(N)$, tales
que $X_i\stackrel{f}\sim Y_i$ para todo $i=1,\dots ,r$ y $T\in
T_r(N)$, probar que $(f^*T)(X_1,\dots ,X_r)=T(Y_1,\dots ,Y_r)\circ
f$. \item Si $M$ y $N$ son dos variedades diferenciables,
$f\in\mathcal{F}(M,N)$ y $T\in T_r(N)$, hallar la expresión
local de $f^*T$. \item Sean $f\in\mathcal{F}(M,N)$ y
$g\in\mathcal{F}(N,P)$ dos aplicaciones diferenciables. Probar que
$(g\circ f)^*=f^*\circ g^*$.
\end{enumerate}
\end{document}