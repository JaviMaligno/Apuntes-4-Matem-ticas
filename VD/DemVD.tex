\documentclass[twoside]{report}
\usepackage{../estilo-apuntes}
\newcommand {\At} {{\mathcal{A}}}
%--------------------------------------------------------
\begin{document}
\chapter{Demostraciones de VD}

\begin{prop}[(1.2.1) en los apuntes]
Todo atlas está contenido en un único atlas maximal.
\end{prop}

\begin{dem}
Sean $\At$ atlas y definimos:
\[ \At^+ = \At \cup \{\text{todas las cartas admisibles en }\At \} \]
A demostrar:
\begin{enumerate}
	\item $\At^+$ atlas.
	\begin{itemize}
		\item $\At^+$ es una familia de cartas.
		\item $\At^+$ recubren la variedad.
		\item Sean $(U,φ)$, $(V,ψ) \in \At^+$:
		\begin{itemize}
			\item Si $(U,φ), (V,ψ) \in \At$, entonces están relacionadas.
			\item Si $(U,φ) \in \At$ y $(V,ψ)$ es admisible (o viceversa), $(U,φ)$ y $(V,ψ)$ están relacionadas.
			\item Supongamos que $(U,φ)$ y $(V,ψ)$ son admisibles en $\At$. Veamos que:
			\[ ψ \circ φ^{-1} : φ(U\cap V) \to ψ(U \cap V) \in C^{\infty} \]
			\[ φ \circ ψ^{-1} : ψ(U\cap V) \to φ(U \cap V) \in C^{\infty} \]
			Como $U,V \subset M$, entonces $U \cap V \subset M$, luego:
			\[ U \cap V = U \cap V \cap M = U \cap V \cap \left(\bigcup_{i \in I}U_i\right) = \bigcup_{i \in I}(U \cap V \cap U_i) \]
			donde $\At = \{(U_i,φ_i)\}_{i \in I}$. Usando que $φ$ es biyectiva: $φ(U \cap V) = \bigcup_{i \in I} φ(U \cap V \cap U_i)$, que es un abierto de $\R^m$. Bata rpobar que $ψ \cap φ^{-1}$ es $C^\infty$ en cada $φ(U \cap V \cap U_i)$ $\forall i \in I$.

			En $φ(U \cap V \cap U_i)$: $ψ \circ φ^{-1} = ψ \circ φ_i^{-1} \circ φ_i \circ φ^{-1}$. Asociando los términos y observando que $ψ \circ φ_i^{-1}$ y $φ_i \circ φ^{-1}$ son cambios de cartas en un atlas $C^\infty$, deducimos que $ψ \circ φ^{-1}$ es $C^\infty$ en $U \cap V$.

			Por lo tanto todas las cartas están relacionadas dos a dos.
		\end{itemize}
	\end{itemize}
	\item $\At \subseteq \At^+$. Evidente.
	\item $\At^+$ maximal. Sea $\tilde{\At}$ un atlas tal que $\At \subseteq \At^+ \subseteq \tilde{\At}$. Como todas las cartas de $\tilde{\At}$ son admisibles en $\At$, entonces $\tilde{\At} \subseteq \At^+$ por definición, luego $\tilde{\At} = \At^+$.
	\item $\At^+$ único. Sea $\At^{++}$ un atlas maximal tal que $\At \subseteq \At^{++}$. Como todas las cartas de $\At^{++}$ son admisibles en $\At$, por definición $\At^{++} \subseteq \At^+$. Por la propiedad 2, $\At^{++} = \At^+$.
\end{enumerate}
\end{dem}

\begin{prop}[(1.2.2) en los apuntes]
La relación de compatibilidad entre atlas de la misma dimensión es una relación de equivalencia.
\end{prop}

\begin{dem}\mbox{}
\begin{itemize}
	\item[(Reflexiva)] $\At \sim \At$ pues $\At \cup \At = \At$ atlas.
	\item[(Simétrica)] $\At_1 \sim \At_2$ implica que $\At_2 \sim \At_1$, pues $\At_2 \cup \At_1 = \At_1 \cup \At_2$ es un atlas.
	\item[(Transitiva)] Veamos que $\At_1 \sim \At_2$ y $\At_2 \sim \At_3$ implica que $\At_1 \sim \At_3$. Dado que $\At_1,\At_2$ está en un atlas maximal $\At^+$ y $\At_2,\At_3$ está en un atlas maximal $\At^{++}$, se deduce que $\At^+=\At^{++}$ por unicidad de atlas maximal que contiene a $\At_2$. Entonces $\At_1 \sim \At_3$.
\end{itemize}
\end{dem}


\begin{ejer}[(Problema 1.4) en los apuntes]
Dotar al Grupo General Lineal $Gl(n,\R)$ (conjunto de las matrices reales cuadradas de dimensión $n$ y deteminante no nulo) de una estrucuta de variedad diferenciable.
\end{ejer}

\begin{dem}
Primero dotaremos estructura de variedad diferenciable a $\mathcal{M}(m \times n, \R)$.

Sea $φ : \mathcal{M}(m \times n, \R) \to \R^{m\cdot n}$ definida por:
\[ φ((a_{ij})_{ij}) \mapsto (a_{11},\dots,a_{1n},a_{21},\dots,a_{2n},\dots,a_{m1},\dots,a_{mn}) \]
Claramente $φ$ es una biyección. Existe una única topología en $\mathcal{M}(m\times n, \R)$ tal que $φ$ es un homeomorfismo. Esta topología inicial será $T_2$ y $2º$-numerable por ser propiedades topológicas presente en la topología euclídea de $\R^{m\cdot n}$. Tomamos el atlas $\At = \{(\mathcal{M}(m\times n,\R),φ)\}$ de dimensión $m\cdot n$. Veamos que $\At$ es efectivamente un atlas:
\begin{enumerate}
	\item $(\mathcal{M}(m\times n,\R),φ)$ es una carta de dimensión $m\cdot n$:
	\begin{itemize}
		\item $\mathcal{M}(m\times n,\R)$ es un abierto de $\mathcal{M}(m\times n, \R)$.
		\item $φ(\mathcal{M}(m\times n,\R) = \R^{m\cdot n}$.
		\item $φ$ es un homeomorfismo.
	\end{itemize}
	\item $\{\mathcal{M}(m\times n,\R)\}$ recubre todo el espacio $\mathcal{M}(m\times n,\R)$.
	\item Las cartas están relacionadas $2$ a $2$.
\end{enumerate}

A continuación, es evidente que $\mathcal{M}(n,\R)$ es una variedad diferenciable de dimensión $n^2$.

Por último, para ver que $Gl(n,\R)$ es una variedad diferenciable de dimensión $n^2$, basta demostrar que $Gl(n,\R)$ es un abierto de $\mathcal{M}(n,\R)$. Consideramos la función $\det : \mathcal{M}(n,\R) \to \R$. Como $\det$ es continuo y $Gl(n,\R) = \det^{-1}(\R\setminus\{0\})$ se deduce que $Gl(n,\R)$ es abierto.

Por lo tanto, $Gl(n,\R)$ se le dota la subvariedad diferencial relativa abierta de $\mathcal{M}(n,\R)$.
\end{dem}
\end{document}