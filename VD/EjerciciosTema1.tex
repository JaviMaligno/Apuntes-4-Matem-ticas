\documentclass[twoside]{article}
\usepackage{estilo-ejercicios}

%--------------------------------------------------------
\begin{document}

\title{Ejercicios de Variedades Diferenciables - Tema 1}
\author{Javi, Rafa, Diego}
\maketitle



\begin{ejercicio}{1}
Sean $\mathcal{A}$ un atlas sobre $M$ y $(U,\varphi)\in\mathcal{ A }$ una carta local. Si $V\subseteq U$ es un
abierto, probar que $(V, \varphi|_V )$ es una carta local admisible en $\mathcal{A}$.
\end{ejercicio}
\begin{solucion}
En primer lugar hay que probar que $(V, \varphi|_V )$ es una carta local. $V$ es abierto por definición. Como $\varphi$ es homeomorfismo por hipótesis, $\varphi|_V (V)=\varphi(V)$ es abierto y es homeomorfo a $V$. Para ver que es admisible en $\mathcal{A}$, sea $(W,\psi)\in\mathcal{A}$ otra carta local. Si $W\cap V=\emptyset$ entonces están relacionadas por definición, así que supongamos que la intersección es no vacía. Sea entonces $\varphi|_V\circ\psi^{-1}:\psi(W\cap V)\to\varphi_|V (W\cap V)$. Como $W\cap V\subseteq V$, $\varphi|_V (W\cap V)=\varphi(W\cap V)$, por lo que $\varphi|_V\circ\psi^{-1}=\varphi\circ\psi^{-1}\in\mathcal{C}^\infty$ por hipótesis. Análogamente se haría con $\psi\circ\varphi|_V^{-1}$.
\end{solucion}

\begin{ejercicio}{2}
Sea $M$ una variedad diferenciable. Probar que, dado cualquier $p \in M$, existe
una carta local $(U, \varphi)$ de la estructura diferenciable centrada en $p$ (es decir,
tal que $\varphi(p) = 0$).
\end{ejercicio}
\begin{solucion}
Sea $(U,\psi)$ una carta cualquiera de la estructura diferenciable tal que $\psi(p)=q\neq 0$, basta tomar $(U,\varphi)$ con $\varphi=\psi-q$. Se tiene que $\varphi(p)=0$ y claramente sigue siendo homeomorfismo y manteniendo las propiedades de diferenciabilidad de $\psi$, por lo que está en la estructura diferenciable.
\end{solucion}

\begin{ejercicio}{3}
Sea $M$ una variedad diferenciable y $\mathcal{A}$ el atlas maximal de la estructura
diferenciable. Probar que los dominios de las cartas locales de $\mathcal{A}$ forman
base de la topología de $M$.
\end{ejercicio}
\begin{solucion}
Sea $U$ un abierto de $M$. Sea $x\in U$. Alrededor de $x$ existe un entorno abierto $C\subseteq U$ que admite un homeomorfismo $\psi_C$ con algún entorno abierto del origen en $\R^m$ (de hecho podemos suponer por el ejercicio anterior que $\psi_C(x)=0$). Sea ahora $S$ una bola abierta centrada en el origen, de modo que $0\in S\subseteq \psi_C(B)$. Así pues, $x\in\psi_C^{-1}(S)\subseteq U$. Por tanto, para cada abierto $U$ de $M$ y para todo punto $x\in U$ hemos encontrado una carta $(\psi_C^{-1}(S),\psi_C)$ tal que $x\in\psi_C^{-1}(S)\subseteq U$, lo cual significa que las cartas forman una base para la topología de $M$.
\end{solucion}

\begin{ejercicio}{4}
Sea $(M,\mathcal{A})$ una variedad diferenciable y $G\subseteq M$ un abierto. Probar que,
dando a $G$ la topología relativa de la de $M$,
$$\mathcal{B} = \{(G \cap U, \varphi|_{G\cap U})\}_{(U,\varphi)\in\mathcal{A}}$$
es un atlas sobre $G$ que lo dota de estructura de variedad diferenciable de la
misma dimensión que $M$, llamada \textbf{Estructura de Subvariedad Abierta}
de $G$.
\end{ejercicio}
\begin{solucion}
\end{solucion}

\begin{ejercicio}{5}
Probar que todo espacio vectorial real $m$-dimensional admite una estructura
diferenciable canónica de dimensión $m$.
\end{ejercicio}
\begin{solucion}
\end{solucion}

\begin{ejercicio}{6}
Probar que $C^m$ es una variedad diferenciable de dimensión $2^m$.
\end{ejercicio}
\begin{solucion}
\end{solucion}

\begin{ejercicio}{7}
Probar que el producto cartesiano de dos variedades diferenciables $M$ y $N$
es una variedad diferenciable de dimensión la suma de las dimensiones de
las variedades factores. De hecho, si $(U,\varphi)$ es una carta de $M$ y $(V,\psi )$ una
carta de $N$, se verifica que $(U \times V,\varphi\times\psi )$ es una carta sobre $M \times N$.
\end{ejercicio}
\begin{solucion}
\end{solucion}

\end{document}