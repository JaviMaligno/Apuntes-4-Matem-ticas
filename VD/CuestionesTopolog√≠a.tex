\documentclass{article}
\usepackage{amsmath,accents}%
\usepackage{amsfonts}%
\usepackage{amssymb}%
\usepackage{comment}
\usepackage{graphicx}
\usepackage{mathrsfs}
\usepackage[utf8]{inputenc}
\usepackage{amsfonts}
\usepackage{amssymb}
\usepackage{graphicx}
\usepackage{mathrsfs}
\usepackage{setspace}
\usepackage{amsthm}
\usepackage{nccmath}
\usepackage[spanish]{babel}
\usepackage{multirow}
\usepackage{hyperref}
\usepackage{tikz-cd}
\usepackage{pgf,tikz}
\usetikzlibrary{arrows}
\usetikzlibrary{cd}
\usetikzlibrary{babel}
\theoremstyle{plain}
\hypersetup{colorlinks=true,citecolor=red, linkcolor=blue}

\renewcommand{\baselinestretch}{1,4}
\setlength{\oddsidemargin}{0.5in}
\setlength{\evensidemargin}{0.5in}
\setlength{\textwidth}{5.4in}
\setlength{\topmargin}{-0.25in}
\setlength{\headheight}{0.5in}
\setlength{\headsep}{0.6in}
\setlength{\textheight}{8in}
\setlength{\footskip}{0.75in}

\theoremstyle{definition}

\newtheorem{theorem}{Teorema}[section]
\newtheorem{acknowledgement}{Acknowledgement}
\newtheorem{algorithm}{Algorithm}
\newtheorem{axiom}{Axiom}
\newtheorem{case}{Case}
\newtheorem{claim}{Claim}
\newtheorem{propi}[theorem]{Propiedades}
\newtheorem{condition}{Condition}
\newtheorem{conjecture}{Conjecture}
\newtheorem{coro}[theorem]{Corolario}
\newtheorem{criterion}{Criterion}
\newtheorem{defi}[theorem]{Definición}
\newtheorem{example}[theorem]{Ejemplo}
\newtheorem{exercise}{Ejercicio}
\newtheorem{lemma}[theorem]{Lema}
\newtheorem{nota}[theorem]{Nota}
\newtheorem{sol}{Solución}
\newtheorem*{sol*}{Solución}
\newtheorem{prop}[theorem]{Proposición}
\newtheorem{remark}{Remark}

\newtheorem{dem}[theorem]{Demostración}

\newtheorem{summary}{Summary}

\providecommand{\abs}[1]{\lvert#1\rvert}
\providecommand{\norm}[1]{\lVert#1\rVert}
\providecommand{\ninf}[1]{\norm{#1}_\infty}
\providecommand{\numn}[1]{\norm{#1}_1}
\providecommand{\gabs}[1]{\left|{#1}\right|}
\newcommand{\bor}[1]{\mathcal{B}(#1)}
\providecommand{\func}[2]{\colon{#1}\longrightarrow{#2}}
\newcommand{\R}{\mathbb{R}}
\newcommand{\Q}{\mathbb{Q}}
\newcommand{\Z}{\mathbb{Z}}
\newcommand{\F}{\mathbb{F}}
\newcommand{\C}{\mathbb{C}}
\newcommand{\X}{\chi}
\providecommand{\Zn}[1]{\Z / \Z #1}
\newcommand{\resi}{\varepsilon_L}
\newcommand{\cee}{\mathbb{C}}
\providecommand{\conv}[1]{\overset{#1}{\longrightarrow}}
\providecommand{\gene}[1]{\langle{#1}\rangle}
\providecommand{\convcs}{\xrightarrow{CS}}
% xrightarrow{d}[d]
\setcounter{exercise}{0}
\newcommand{\cicl}{\mathcal{C}}

\begin{document}
\title{Cuestiones de Topología - Variedades Diferenciables}
\author{Javi, Rafa, Diego}
\maketitle

\begin{exercise}

Sean $(X,T)$ un espacio topológico y $A,B$ dos subconjuntos de $X$ tales que $A\subseteq B\subseteq X$. Probar:
\begin{itemize}
\item[(a)] $(T|_B)|_A=T|_A$.
\item[(b)] Si $A\in T$, entonces $A\in T|_B$.
\item[(c)] El recíproco de (b) no es cierto en general, pero sí cuando $B$ es abierto (es decir, si $A\in T|_B$ y $B\in T$, entonces $A\in T$).
\end{itemize}

\end{exercise}
\begin{sol*}\
\begin{itemize}
\item[(a)] Se va a probar por doble contención. Sea $U\in (T|_B)|_A$. Por definición $U= (G\cap B)\cap A$ con $G\in T$. Como $A\subset B$, $U=(G\cap B)\cap A=G\cap A$, que por definición significa que $U\in T|_A$. El recíproco se consigue realizando la misma prueba en sentido inverso. 
\item[(b)] Como $A\subset B$, se cumple $A=A\cap B$. Por tantopodemos expresar $A$ como intersección de un elemento de $T$ con $B$.
\item[(c)] Como contraejemplo de que no es cierto en general podemos usar $X=(-1,2)$ con la topología euclídea, $B=[0,1)$. Tomando $A=[0,\frac{1}{2})$ vemos que $A=(-1,\frac{1}{2})\cap B$, por lo que es abierto de la topología relativa a $B$, pero no es abierto en $X$ puesto que el $0$ está en $A$ pero no en su interior.

Si $B$ es abierto, entonces $A=G\cap B$ siendo tanto $G$ como $B$ abiertos, de modo que $A$ es un abierto de $T$ por ser intersección de dos abiertos.
\end{itemize}

\end{sol*}

\begin{exercise}\label{2}
Probar que un conjunto es abierto si y solo si es entorno de todos sus puntos. Utilizar esta caracterización para probar que $[0,1)$ no es abierto euclídeo de $\R$.
\end{exercise}

\begin{sol*}
Llamemos $G$ al conjunto y supongamos que es abierto. Un conjunto es entorno de un punto si y solo si existe un abierto conteniendo al punto y contenido en el conjunto. Para todos los puntos de $G$ se tiene que $G$ es un abierto que contiene al punto y además $G$ está contenido en sí mismo, por lo que se verifica que $G$ es entorno de todos sus puntos. Supongamos ahora ue $G$ es entorno de todos sus puntos. Esto significa que $\forall p\in G$, $\exists U_p\subseteq G$ abierto tal que $p\in U_p$. Por tanto se tiene que $\bigcup_{p\in G} U_p\subseteq G$. Por otro lado, si $p\in G$ entonces $p\in U_p$ por definición de $U_p$. Por tanto, $p\in \bigcup_{p\in G} U_p\subseteq G$ $\forall p\in G$, es decir $\bigcup_{p\in G} U_p\supseteq G$, con lo que concluimos que $G=\bigcup_{p\in G} U_p$, que es abierto por ser unión de abiertos.

Para probar que $[0,1)$ no es abierto euclídeo basta ver que este conjunto no es entorno del $0$. Esto es así porque no podemos incluir ninguna bola abierta de radio positivo centrada en el $0$ dentro del intervalo.
\end{sol*}

\begin{exercise}
Sean $(X,T)$ un espacio topológico y $\mathcal{B}\subseteq T$. Probar que $\mathcal{B}$ es base de $T$ si y solo si $\forall G\in T$ y $\forall p\in G$, $\exists B\in\mathcal{B}$ tal que $p\in B\subseteq G$. 
\end{exercise}

\begin{sol*}
Supongamos que $\mathcal{B}$ es base. Entonces, dado $G\in T$ podemos expresar $G=\bigcup_{B\in\mathcal{B}} B$. Por tanto, si $p\in G$, entonces $p\in\bigcup_{B\in\mathcal{B}}B$, por lo que existe $B_p$ tal que $p\in B_p\subseteq G$. 

Para el recíproco, tomemos un abierto cualquiera $G\in T$ y consideremos $U=\bigcup_{p\in G} B_p$. Claramente $G\subseteq U$, puesto que todos los puntos de $G$ están contenidos en $U$. Además, como por hipótesis $\forall p\ B_p\subseteq G$ se tiene la inclusión contraria, por lo que $G=U$, como queríamos demostrar.
\end{sol*}

\begin{exercise}
Sea $X$ un conjunto y $\Omega$ una familia de subconjuntos de $X$. Sea $\mathcal{B}_\Omega=\{X\}\cup\{$ Intersecciones finitas de elementos de $\Omega\}$ y $T_\Omega=\{\emptyset\}\cup\{$Uniones cualesquiera de elementos de $\mathcal{B}_\Omega\}$. Probar que $T_\Omega$ es la menor topología sobre $X$ que contiene a $\Omega$. Probar también que $\mathcal{B}_\Omega$ es base de $T_\Omega$.
\end{exercise}

\begin{sol*}
\end{sol*}

\begin{exercise}
Sean $(X,T_X)$, $(Y,T_Y)$ dos espacios topológicos y $f:X\rightarrow Y$ una aplicación biyectiva. Probar que son equivalentes:
\begin{itemize}
\item[(a)] $f^{-1}$ es continua.
\item[(b)] $f$ es abierta.
\item[(c)] $f$ es cerrada.
\end{itemize}
Utilizar este resultado para caracterizar los homeomorfismos.
\end{exercise}

\begin{sol*}\
\begin{itemize}
\item[(a)$\Rightarrow$(b)] Sea $G\subseteq X$ abierto. Por ser $f^{-1}$ continua, usando que $f$ es biyectiva se tiene que $(f^{-1})^{-1}(G)=f(G)$ es abierto, por lo que $f$ es abierta.
\item[(b)$\Rightarrow$(c)] Dado $G\subseteq X$ abierto se tiene que $f(G)$ es abierto por hipótesis. Tomando el complementario, $G^c$ (que es cerrado por ser complementario de un abierto) tenemos gracias a la biyectividad de $f$ que $f(G^c)=f(G)^c$ que es cerrado por ser complementario de un abierto, como queríamos demostrar. De forma análoga se probaría (c)$\Rightarrow$ (b).
\item[(b)$\Rightarrow$(a)] Por ser $f$ abierta, si $G\subseteq X$ es abierto, entonces por biyectividad $f(G)=(f^{-1})^{-1}(G)$ es abierto. 
\end{itemize}
Como caracterización, $f$ es homeomorfismo si y solo si es biyectiva, continua y abierta (o cerrada). 
\end{sol*}

\begin{exercise}
Sea $(X,T_X)$ un espacio topológico y $f:(X,T_X)\longrightarrow Y$ una aplicación biyectiva. Probar que $T_Y=\{f(G)\mid G\in T_X\}$ coincide con la topología final de $f$ (es decir, $\{U\subseteq Y\mid f^{-1}(U)\in T_X\}$) y es la única topología sobre $Y$ que convierte a $f$ en homeomorfismo.
\end{exercise}

\begin{sol*}
\end{sol*}

\begin{exercise}
Sea $(Y,T_Y)$ un espacio topológico y $f:X\longrightarrow (Y,T_Y)$ una aplicación biyectiva. Probar que $T_X=\{G\subseteq X\mid f(G)\in T_Y\}$ coincide con la topología inicial de $f$ (es decir, $\{f^{-1}(U)\mid U\in T_Y\}$) y es la única topología sobre $X$ que convierte a $f$ es homeomorfismo. 
\end{exercise}

\begin{sol*}
Sea $G\in T_X$, de modo que $f(G)\in T_Y$ y $G=f^{-1}f(G)$, por lo que basta tomar $U=f(G)$ para ver que $G$ pertenece a la topología inicial. Por otro lado, si tomamos $f^{-1}(U)$ en la topología inicial, por ser $f$ sobreyectiva se tiene que $U=f(G)$ para algún $G\subseteq X$, por lo que $f^{-1}(U)\in T_X$.

Para ver que convierte a $f$ en homeomorfismo, como ya es biyectiva, necesitamos que sea continua y abierta. La continuidad se deduce del apartado anterior, pues si $U\in T_Y$ entonces $f^{-1}(U)\in T_X$. Claramente es abierta puesto que $f(G)\in T_Y$ para todo abierto $G$ de esta topología por definición. 

Veamos la unicidad. Por la definición de topología inicial, esta topología es la más pequeña que hace que $f$ sea continua. Pero además no puede existir una topología mayor con la que $f$ siga siendo abierta, pues para cualquier conjunto $H\notin T_X$, $f(H)\notin T_Y$.
\end{sol*}

\begin{exercise}
Probar que todo homeomorfismo local es una aplicación abierta.
\end{exercise}

\begin{sol*}
Sea $f:X\to Y$ un homeomorfismo local. Por ser homeomorfismo local, $\forall x\in X\ \exists N_x$ tal que $N_x\cong f(N_x)$. Sea $G\subseteq X$ abierto. Entonces, por el ejercicio \ref{2} se tiene que $G=\bigcup_{x\in G} (N_x\cap G)$, por lo que $f(G)=f(\bigcup_{x\in G} (N_x\cap G))=\bigcup_{x\in G}f(N_x\cap G)\cong \bigcup_{x\in G} N_x\cap G$ que es abierto por ser una unión de abiertos.
\end{sol*}


\end{document}